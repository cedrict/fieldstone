
\index{general}{Newtonian fluid}
Simply put, a Newtonian fluid is a fluid in which the viscous stresses at 
every point are linearly proportional 
to the local strain rate.
Mathematically speaking, this means that the fourth-order tensor ${\bm C}$ relating the viscous stress 
tensor to the strain rate tensor does not depend on the stress state and velocity of the flow.
\begin{equation}
{\bm \tau}={\bm C} : \dot{\bm \varepsilon}
\end{equation}
One very often makes the assumption that the fluid is isotropic, i.e. its mechanical properties are the 
same along any direction. As a consequence the fourth order viscosity tensor 
${\bm C}$ is symmetric and will have only two independent real parameters: 
a bulk viscosity coefficient, that defines the resistance of the medium to gradual uniform compression; 
and a dynamic viscosity coefficient $\eta$ that expresses its resistance to gradual 
shearing\footnote{We here neglect the so-called rotational viscosity coefficient which results 
from a coupling between the fluid flow and the rotation of the individual particles}.

Rather logically we denote by non-Newtonian fluids which are not Newtonian, i.e. their viscosity (tensor)
depends on stress. Such fluids are part of our daily life, e.g. honey, toothpaste, paint, blood, or shampoo.
They are also sometimes denoted as Generalized Newtonian Fluid \index{general}{Generalized Newtonian Fluid}. 

\begin{center}
\includegraphics[width=8cm]{images/rheology/nnf}\\
{\captionfont no idea where this comes from ...}
\end{center}
