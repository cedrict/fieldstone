\begin{flushright} {\tiny {\color{gray} basis\_q1p\_2D.tex}} \end{flushright}
%~~~~~~~~~~~~~~~~~~~~~~~~~~~~~~~~~~~~~~~~~~~~~~~~~~~~~~~~~~~~~~~~~~~~~~~~~~~~~~~~~~~~~~~~~~~~~~~~~~

\begin{verbatim}
4===========3
|           |   (r_1,s_1)=(-1,-1)
|           |   (r_2,s_2)=(1,-1)
|     5     |   (r_3,s_3)=(1,1)
|           |   (r_4,s_4)=(-1,1)
|           |   (r_5,s_5)=(0,0)
1===========2
\end{verbatim}

\begin{itemize}
\item 
In Bai (1997) \cite{bai97}: "It is well known that the equal-order bilinear velocity-bilinear 
continuous pressure element - the $Q_1\times Q_1$, element - exhibits a certain spurious pressure mode.
In the paper we propose a new stabilized $Q_1\times Q_1$ combination for the velocity and
pressure with three internal degrees of freedom added to the velocity space, that is, one degree of
freedom for each component of the velocity and one degree of freedom shared by both components of
the velocity."

Two versions are proposed, if I understand it correctly.
The first one is given in Eq.~(7) (three extra dofs: $u_5$, $v_5$, $w$):
\begin{eqnarray}
u^h(r,s) &=& \sum_{i=1}^4 N_i (r,s) u_i + \left[ u_5 - \frac{w}{4}(1-s) \right] (1-r^2)(1-s^2) \nonumber\\
v^h(r,s) &=& \sum_{i=1}^4 N_i (r,s) v_i + \left[ v_5 - \frac{w}{4}(1-r) \right] (1-r^2)(1-s^2) 
\end{eqnarray}
The second one in Eq.~(23) (four extra dofs: $u_5$, $v_5$, $u_6$, $v_6$):
\begin{eqnarray}
u^h(r,s) &=& \sum_{i=1}^4 N_i (r,s) u_i + \left[ u_5 +u_6(r+s) \right] (1-r^2)(1-s^2) \nonumber\\
v^h(r,s) &=& \sum_{i=1}^4 N_i (r,s) v_i + \left[ v_5 +v_6(r+s) \right] (1-r^2)(1-s^2) 
\end{eqnarray}

\item In Franca \etal (2007) \cite{fros07}: 
"Stabilized finite element method for Stokes equations with piecewise continuous 
bilinear approximations for both velocity and pressure variables. The velocity
field is enriched with piecewise polynomial bubble functions with null average at element
edges."

It looks like they are proposing (see their Eq.~(2.6)):
\begin{eqnarray}
u^h(r,s) &=& \sum_{i=1}^4 N_i (r,s) u_i + (\alpha + \gamma s)\frac{1}{2}(r^2+s^2-\frac43) \nn\\ 
v^h(r,s) &=& \sum_{i=1}^4 N_i (r,s) v_i + (\beta + \gamma r) \frac{1}{2}(r^2+s^2-\frac43)  
\end{eqnarray}

\item In Kwon \& Park (2014) \cite{kwpa14}: 
"We introduce a new stable MINI-element pair for incompressible Stokes equations on
quadrilateral meshes, which uses the smallest number of bubbles for the velocity. The pressure is 
discretized with the $P_1$-midpoint-edge-continuous elements and each component of the velocity field is
done with the standard $Q_1$-conforming elements enriched by one bubble a quadrilateral."

\item  In Lamichhane (2017) \cite{lami17}: "We consider a quadrilateral MINI
finite element for approximating the solution
of Stokes equations using a quadrilateral mesh. We use the standard bilinear finite
element space enriched with element-wise defined bubble functions for the velocity
and the standard bilinear finite element space for the pressure space. With a simple
modification of the standard bubble function we show that a single bubble function is
sufficient to ensure the inf-sup condition.
This is a refinement of Bai (1997) \cite{bai97} where the author enriches the velocity space with
more than a single vector bubble function per element. In this article we show that 
with a small modification of the standard bubble function we can get the stability just 
by using a single vector bubble function per element."

\input{lamichhane2D}

\end{itemize}


\Literature: Mons \& Roge (1992) \cite{moro92}, 
Li \etal (2009) \cite{lihc09}, Knobloch \& Tobiska (2000) \cite{knto00}, 
Franca \etal (1993) \cite{frha93}, Idelsohn \etal (1995) \cite{idsn95}.
