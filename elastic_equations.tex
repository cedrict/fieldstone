\begin{flushright} {\tiny {\color{gray} elastic\_equations.tex}} \end{flushright}
%~~~~~~~~~~~~~~~~~~~~~~~~~~~~~~~~~~~~~~~~~~~~~~~~~~~~~~~~~~~~~~~~~~~~~~~~~~~~~~~~~~~~~~~~~~~~~~~~~~

{\large \color{orange} This will be moved to Section \ref{chapt:elasticity}}

What follows is mostly borrowed from Becker \& Kaus lecture notes \cite{beka}.

The strong form of the PDE that governs force balance in a medium is given by
\[
\vec{\nabla}\cdot{\bm \sigma}  + \vec{f} = \vec{0}
\]
where ${\bm \sigma}$ is the stress tensor and $\vec{f}$ is a body force.

The stress tensor is related to the strain tensor through the generalised 
Hooke's law\footnote{\url{https://en.wikipedia.org/wiki/Hooke's_law}}:
\begin{equation}
\sigma_{ij}=\sum_{kl}C_{ijkl}\varepsilon_{kl} 
\qquad
\text{or}
\qquad
{\bm \sigma} = {\bm C} : {\bm \varepsilon}
\label{eq:oone}
\end{equation}
where ${\bm C}$ is the fourth-order elastic tensor.

Due to the inherent symmetries of ${\bm \sigma}$, ${\bm \varepsilon}$, and ${\bm C}$, 
only 21 elastic coefficients of the latter are independent. 
For isotropic linear media (which have the same physical properties in any direction), ${\bm C}$ 
can be reduced to only two independent numbers (for example the bulk modulus $K$ and the shear modulus $G$ 
that quantify the material's resistance to changes in volume and to shearing deformations, respectively).
Thus
\[
C_{ijkl} = \lambda \delta_{ij}\delta_{kl} + \mu (\delta_{ik}\delta_{jl}+\delta_{il}\delta_{jk})
\]
so that Eq.~\eqref{eq:oone} becomes:
\[
\sigma_{ij}=\lambda \varepsilon_{kk} \delta_{ij} + 2\mu \varepsilon_{ij}
\]
or
\begin{mdframed}[backgroundcolor=blue!5]
\begin{equation}
{\bm \sigma}=\lambda (\vec{\nabla}\cdot\vec{u}) {\bm 1} +2\mu {\bm \varepsilon}(\vec{u}) \label{eq:twoELAST}
\end{equation}
\end{mdframed}
where $\lambda$ is the Lam\'e parameter and $\mu$ is the shear 
modulus\footnote{It is also sometimes written $G$}.
The term $\vec{\nabla}\cdot\vec{u}$ is the isotropic dilation.

\index{general}{Lam\'e Parameter} 
\index{general}{Shear Modulus}

This can be re-written in the 6-dimensional stress/strain space as
\[
\underbrace{
\left(
\begin{array}{c}
\sigma_{xx} \\
\sigma_{yy} \\
\sigma_{zz} \\
\sigma_{xy} \\
\sigma_{xz} \\
\sigma_{yz} 
\end{array}
\right)}
_{\vec{\sigma}}
=
\underbrace{
\left(
\begin{array}{cccccc}
\lambda+2\mu & \lambda & \lambda & 0 & 0 & 0 \\ 
\lambda & \lambda+2\mu & \lambda & 0 & 0 & 0 \\ 
\lambda & \lambda & \lambda+2\mu & 0 & 0 & 0 \\
0 & 0 & 0 & \mu & 0 & 0 \\ 
0 & 0 & 0 & 0 & \mu & 0 \\ 
0 & 0 & 0 & 0 & 0 & \mu  
\end{array}
\right)}
_{{\bm C}}
\cdot
\underbrace{
\left(
\begin{array}{c}
\varepsilon_{xx} \\
\varepsilon_{yy} \\
\varepsilon_{zz} \\
\varepsilon_{xy} \\
\varepsilon_{xz} \\
\varepsilon_{yz} 
\end{array}
\right)}
_{\vec{\varepsilon}}
\]
or, in terms of the compliance matrix ${\bm C}^{-1}$,
\index{general}{Compliance Matrix}
\[
\vec{\varepsilon} 
= {\bm C}^{-1} \cdot \vec{\sigma}
\]
with
\[
{\bm C}^{-1}
=
\frac{1}{\mu(3\lambda+2\mu)}
\left(
\begin{array}{cccccc}
\lambda+\mu & -\lambda/2 & -\lambda/2 & 0 & 0 & 0 \\
-\lambda/2 & \lambda+\mu & -\lambda/2 & 0 & 0 & 0 \\
-\lambda/2 & -\lambda/2 & \lambda+\mu & 0 & 0 & 0 \\
0 & 0 & 0 & 3\lambda+2\mu & 0 & 0 \\ 
0 & 0 & 0 & 0 & 3\lambda+2\mu & 0 \\ 
0 & 0 & 0 & 0 & 0 & 3\lambda+2\mu  
\end{array}
\right)
\]
If we define the Young's modulus as $E=\mu(3\lambda+2\mu)/(\lambda+\mu)$ 
and the Poisson's ratio as $\nu=\lambda(\lambda+\mu)/2$, then
\[
{\bm C}^{-1}
=
\frac{1}{E}
\left(
\begin{array}{cccccc}
1 & -\nu & -\nu & 0 & 0 & 0 \\
-\nu & 1 & -\nu & 0 & 0 & 0 \\
-\nu & -\nu & 1 & 0 & 0 & 0 \\
0 & 0 & 0 & 2(1+\nu) & 0 & 0 \\ 
0 & 0 & 0 & 0 & 2(1+\nu) & 0 \\ 
0 & 0 & 0 & 0 & 0 & 2(1+\nu) 
\end{array}
\right)
\]
Note that the determinant  of ${\bm C}^{-1}$ is $8(1+\nu)^5(1-2\nu)E^{-6}$,
so that when $\nu\rightarrow 1/2$ (incompressible material), the compliance
matrix is singular and the stress cannot be given as a function of strain \cite{lubliner}.


The strain tensor is related to the displacement as follows: \index{general}{Strain Tensor}
\[
{\bm \varepsilon}(\vec{u}) 
= \frac{1}{2}(\vec{\nabla}\vec{u} + (\vec{\nabla}\vec{u})^T)
\]
The incompressibility (or bulk modulus) $K$ is defined as $p=-K \vec{\nabla}\cdot\vec{u}$ 
where $p$ is the pressure with \index{general}{Bulk Modulus}
\begin{eqnarray}
p&=&-\frac{1}{3} \text{tr}({\bm \sigma}) \nonumber\\
 &=& -\frac{1}{3} [ \lambda (\vec{\nabla}\cdot\vec{u}) \text{tr}[{\bm 1}] + 2 \mu {\rm tr}[{\bm \varepsilon}(\vec{u})]] \nonumber\\
 &=& -\frac{1}{3} [ \lambda (\vec{\nabla}\cdot\vec{u})  3  + 2 \mu  (\vec{\nabla}\cdot\vec{u}) ] \nonumber\\
 &=& -\left[ \lambda + \frac{2}{3} \mu \right] (\vec{\nabla}\cdot\vec{u})  
\end{eqnarray}
so that 
\begin{mdframed}[backgroundcolor=blue!5]
\[
p=-K \vec{\nabla}\cdot\vec{u} 
\qquad
\text{with}
\qquad
K=\lambda+\frac{2}{3}\mu
\]
\end{mdframed}

\begin{remark}
Eq. (\ref{eq:oone}) and (\ref{eq:twoELAST}) are analogous to the ones that one has to solve
in the context of viscous flow using the penalty method. In this case $\lambda$ is the penalty coefficient, 
$\vec{u}$ is the velocity, and $\mu$ is then the dynamic viscosity.
\end{remark}

The Lam\'e parameter and the shear modulus are also linked to $\nu$ the poisson ratio, 
and $E$, Young's modulus: \index{general}{Poisson Ratio} \index{general}{Young's Modulus}
\[
\lambda=\mu\frac{2\nu}{1-2\nu}
=\frac{\nu E}{(1+\nu)(1-2\nu)}
\quad\quad
{\rm with}
\quad\quad
E=2\mu(1+\nu)
\]
The shear modulus, expressed often in GPa, describes the material's response to shear stress.
The poisson ratio describes the response in the direction orthogonal to uniaxial stress.
The Young modulus, expressed in GPa, describes the material's strain response to uniaxial stress in the 
direction of this stress.


\Literature: solvers for 3D Stokes and elasticity problems with
heterogeneous coefficients \cite{samb20}















