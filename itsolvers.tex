
In what follows, we want to solve the system of linear equations
\begin{equation}
{\bm A}\cdot \vec{x} = \vec{b} 
\end{equation}
for the vector $\vec{x}$, 
where the known $n\times n$ matrix ${\bm A}$ is 
symmetric (i.e., ${\bm A}^T = {\bm A}$), 
positive-definite (i.e. $\vec{x}^T\cdot {\bm A} \cdot \vec{x} > 0$ 
for all non-zero vectors $\vec{x}$ in $\mathbb{R}^n$), 
and real, and $\vec{b}$ is known as well. 
We denote the unique solution of this system by $\vec{x}^\star$.


%........................................
\subsubsection{Stationary iterative methods}

\url{https://en.wikipedia.org/wiki/Iterative_method}

Basic examples of stationary iterative methods use a splitting of the matrix ${\bm A}$ such as
\[
{\bm A}={\bm D}+{\bm L}+{\bm U}
\]
where D is only the diagonal part of A, L is the strict lower triangular part of A and
U is the strict upper triangular part of A.

For instance:
\[
{\bm A}=
\left(
\begin{array}{ccc}
1 & 5 & 8 \\
6 & 4 & 2 \\
-1 & 7 & 5
\end{array}
\right)
\qquad
\Rightarrow
\qquad
{\bm D}=
\left(
\begin{array}{ccc}
1 & 0 & 0 \\
0 & 4 & 0 \\
0 & 0 & 5
\end{array}
\right)
\quad
{\bm L}=
\left(
\begin{array}{ccc}
0 & 0 & 0 \\
6 & 0 & 0 \\
-1 & 7 & 0
\end{array}
\right)
\quad
{\bm U}=
\left(
\begin{array}{ccc}
0 & 5 & 8 \\
0 & 0 & 2 \\
0 & 0 & 0
\end{array}
\right)
\]


CHANGE with M and N
The iterative method is defined by:
\begin{equation}
{\bm D} \cdot \vec{T}^{{\color{Fuchsia}k+1}} = -({\bm L} + {\bm U}) \cdot \vec{T}^{\color{Fuchsia}k} 
+ \vec{b}
\qquad k=0,1,\dots
\end{equation}
where $\vec{T}^{\color{Fuchsia}0}$ is the initial guess (often taken to be zero).
Note that the superscript denotes the iteration number and has nothing to 
do with the time step.


\begin{itemize}
\item Richardson method: 
\item Jacobi method: 
\item Damped Jacobi method: 
\item Gauss–Seidel method: 
\item Successive over-relaxation method (SOR):
\item Symmetric successive over-relaxation (SSOR):
\end{itemize}

\index{general}{BiCG solver}
\index{general}{Jacobi solver}
\index{general}{Gauss-Seidel solver}
\index{general}{GMRES solver}
\index{general}{CG solver}
\index{general}{SSOR solver}


%........................................
\subsubsection{Krylov subspace methods}

\begin{itemize}
\item {\color{purple} Conjugate Gradient}
\footnote{\url{https://en.wikipedia.org/wiki/Conjugate_gradient_method}} 

It was first proposed by Hestenes and Stiefel in 1952 \cite{hest52}.
The method solves an SPD system ${\bm A}\cdot \vec{x} = \vec{b}$ of size $n$.
In theory (i.e. exact arithmetic) it does so in $n$ iterations.
Each iteration requires a few inner products in $\mathbb{R}^n$ and one matrix-vector multiplication.
With roundoff error, CG can work poorly (or not at all), but for some 
${\bm A}$ (and $\vec{b}$), can get good approximate solution in $<<n$ iterations.

As an iterative method, the conjugate gradient method monotonically (in the energy norm) 
improves approximations $\vec{x}_k$ to the exact solution and
may reach the required tolerance after a relatively small (compared to the problem size) number 
of iterations. The improvement is typically linear and its speed is determined by the 
condition number $\kappa({\bm A})$ of the system matrix ${\bm A}$: 
the larger $\kappa({\bm A})$ is, the slower the improvement.

If $\kappa({\bm A})$ is large, preconditioning is commonly used to replace the 
original system ${\bm A} \cdot \vec{x}-\vec{b}=\vec{0}$ 
with ${\bm M}^{-1}\cdot ({\bm A} \cdot \vec{x}-\vec{b})=\vec{0}$ 
such that $\kappa({\bm M}^{-1}\cdot {\bm A})$ 
is smaller than $\kappa({\bm A})$. 

The resulting method is called the Preconditioned Conjugate Gradient method (PCG).
An extreme case of preconditioner is ${\bm M}={\bm A}^{-1}$ but it is a silly case
since applying the preconditioner is as difficult as solving the system in the 
first place.
In the end the goal is to find a matrix ${\bm M}$ that is cheap to multiply, 
and is an approximate inverse of ${\bm A}$ 
(or at least has a more clustered spectrum than ${\bm A}$).

\begin{center}
\frame{\includegraphics[height=5.6cm]{images/solvers/cgwiki}}
\frame{\includegraphics[height=5.6cm]{images/solvers/pcgwiki}}
\frame{\includegraphics[height=5.6cm]{images/solvers/shew94}}\\
{\captionfont Top: algorithms as obtained from Wikipedia (Left: CG; Right: PCG);
Bottom: algorithm from Shewchuk (1994) \cite{shew94}.}
\end{center}

Also available on Wikipedia is a (naive) MATLAB implementation of the CG algorithm:
\begin{center}
\frame{\includegraphics[height=6cm]{images/solvers/cgwiki2}}
\end{center}
We see that its implementation is actually rather simple and straightforward!

\Literature: Shewchuk, An Introduction to the Conjugate 
Gradient Method Without the Agonizing Pain \cite{shew94}

The CG and PCG algorithms are used in Section~\ref{ss:schurpcg}.
It is implemented in \stone~15,16,82.








\item Biconjugate Gradient method
\footnote{\url{https://en.wikipedia.org/wiki/Biconjugate_gradient_method}}
\item Biconjugate Gradient stabilised method
\footnote{\url{https://en.wikipedia.org/wiki/Biconjugate_gradient_stabilized_method}}
\item MINRES
\item Generalized minimal residual method (GMRES)
\footnote{\url{https://en.wikipedia.org/wiki/Generalized_minimal_residual_method}}
\end{itemize}





