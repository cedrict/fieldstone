
%---------------------------
\subsection{1D segments}

%.....................................
\subsubsection{Linear basis functions}

Let us start with the mass matrix (which we encountered in 
Section~\ref{sec:diff1D} -- although we leave the $\rho C_p$ term out):
\begin{equation}
{\bm M}_e=\int_{\Omega_e} \vec{N}^T \vec{N} dV
= \int_{-1}^{+1} \vec{N}^T \vec{N} dr
\end{equation}
on the reference element, with 
\[
{\vec N}^T = 
\left(
\begin{array}{c}
N_1(r) \\ N_2(r)
\end{array}
\right)
=
\frac{1}{2}
\left(
\begin{array}{c}
1-r \\ 1+r
\end{array}
\right)
\]
We have 
\begin{eqnarray}
\int_{-1}^{+1} N_1(r) N_1(r) dr &=& 2/3 \\ 
\int_{-1}^{+1} N_1(r) N_2(r) dr &=& 1/3 \\
\int_{-1}^{+1} N_2(r) N_2(r) dr &=& 2/3
\end{eqnarray}

Following the procedure in Section~\ref{sec:diff1D} we arrive at
\[
{\bm M}^e= \frac{1}{3} 
\left(
\begin{array}{cc}
2  & 1 \\
1 & 2
\end{array}
\right)
\]
The lumped mass matrix is then
\begin{eqnarray}
\bar{\bm M}^e 
&=&
\frac{1}{3}
\left(
\begin{array}{cc}
2+1  & 0 \\
0 & 1+2
\end{array}
\right)
=
\left(
\begin{array}{cc}
1  & 0 \\
0 & 1
\end{array}
\right)
\end{eqnarray}

\begin{remark} 
The sum of all the terms in the mass matrix must be equal to 2. Indeed:
\begin{eqnarray}
\sum_{ij} M_{ij} 
&=&\sum_{ij} \int_{-1}^{+1} N_i N_j dr \nn\\
&=&\int_{-1}^{+1} (N_1N_1+N_1N_2+N_2N_1+N_2N_2) dr\nn\\
&=&\int_{-1}^{+1} [N_1(N_1+N_2)+N_2(N_1+N_2)]dr\nn\\
&=&\int_{-1}^{+1} (N_1 + N_2) dr\nn\\
&=& 2\nn
\end{eqnarray}
\end{remark}


%.....................................
\subsubsection{Quadratic basis functions}
There are now three nodes in the segment so that the mass matrix 
is now a $3\times3$ matrix. We have (see Section~\ref{sec:bf1}) 
\begin{equation}
{\vec N}^T(r) = 
\left(
\begin{array}{c}
N_1(r) \\ 
N_2(r) \\ 
N_3(r) 
\end{array}
\right)
=
\left(
\begin{array}{c}
\frac{1}{2} r (r-1) \\
1-r^2 \\
\frac{1}{2} r (r+1) 
\end{array}
\right)
\end{equation}
We then have to compute
\begin{eqnarray}
\int_{-1}^{+1} N_1(r) N_1(r) dr &=& \frac{8}{30}  = 0.26666 \nn\\
\int_{-1}^{+1} N_1(r) N_2(r) dr &=& \frac{4}{30}  =0.13333  \nn\\
\int_{-1}^{+1} N_1(r) N_3(r) dr &=& -\frac{2}{30} =-0.06666...\nn\\ 
\int_{-1}^{+1} N_2(r) N_2(r) dr &=& \frac{16}{15} = 1.06666 \nn\\
\int_{-1}^{+1} N_2(r) N_3(r) dr &=& \frac{4}{30}  =0.13333 \nn\\
\int_{-1}^{+1} N_3(r) N_3(r) dr &=& \frac{8}{30} = 0.26666  \nn
\end{eqnarray}
and finally 
\begin{equation}
{\bm M}^e 
=
\frac{1}{30}
\left(
\begin{array}{ccc}
8  & 4 & -2 \\
4  & 32 & 4 \\
-2 & 4 & 8
\end{array}
\right)
\end{equation}


The lumped mass matrix is then
\begin{eqnarray}
\bar{\bm M}^e 
&=&
\frac{1}{30}
\left(
\begin{array}{ccc}
8 + 4  -2 & 0 & 0\\
0 & 4 + 32 + 4 \\
0 & 0 & -2 + 4 + 8
\end{array}
\right) \nn\\
&=&
\frac{1}{30}
\left(
\begin{array}{ccc}
10 & 0 & 0\\
0 & 40 & 0\\
0 & 0 & 10 
\end{array}
\right) \nn\\
&=&
\frac{1}{3}
\left(
\begin{array}{ccc}
1 & 0 & 0\\
0 & 4 & 0\\
0 & 0 & 1 
\end{array}
\right) 
\end{eqnarray}

We can easily verify that
\[
\sum_{ij} M_{ij} = 2
\qquad
\qquad
\sum_{ij} \bar{M}_{ij} = 2
\]


%.....................................
\subsubsection{Cubic basis functions}
There are now four nodes in the segment so that the mass matrix 
is now a $4\times4$ matrix. We have (see Section~\ref{sec:bf3}) 
\begin{equation}
{\vec N}^T(r) = 
\left(
\begin{array}{c}
N_1(r) \\ 
N_2(r) \\ 
N_3(r) \\ 
N_4(r) 
\end{array}
\right)
=
\frac{1}{16}
\left(
\begin{array}{c}
 -1+  r+9r^2- 9r^3  \\ 
  9-27r-9r^2+27r^3  \\
  9+27r-9r^2-27r^3  \\
 -1-  r+9r^2+ 9r^3  
\end{array}
\right)
\end{equation}


\begin{eqnarray}
\int_{-1}^{+1} N_1(r) N_1(r) dr &=&  \frac{1}{256}\frac{4096}{105} \nn\\ 
\int_{-1}^{+1} N_1(r) N_2(r) dr &=&  \frac{1}{256}\frac{1056}{35} \nn\\
\int_{-1}^{+1} N_1(r) N_3(r) dr &=& -\frac{1}{256}\frac{384}{35} \nn\\
\int_{-1}^{+1} N_1(r) N_4(r) dr &=&  \frac{1}{256}\frac{608}{105} \nn\\
\int_{-1}^{+1} N_2(r) N_2(r) dr &=&  \frac{1}{256}\frac{6912}{35} \nn\\
\int_{-1}^{+1} N_2(r) N_3(r) dr &=& -\frac{1}{256}\frac{864}{35} \nn\\
\int_{-1}^{+1} N_2(r) N_4(r) dr &=& -\frac{1}{256}\frac{384}{35} \nn\\
\int_{-1}^{+1} N_3(r) N_3(r) dr &=&  \frac{1}{256}\frac{6912}{35}\nn\\
\int_{-1}^{+1} N_3(r) N_4(r) dr &=&  \frac{1}{256}\frac{1056}{35}\nn\\
\int_{-1}^{+1} N_4(r) N_4(r) dr &=&  \frac{1}{256}\frac{4096}{105}\nn
\end{eqnarray}

and finally 
\begin{equation}
{\bm M}^e 
=
\frac{1}{16}\frac{1}{105}
\left(
\begin{array}{cccc}
256 & 198 & -72  & 38  \\
198 & 1296 & -162 & -72 \\
-72 & -162 & 1296 & 198 \\
38 & -72 & 198 & 256
\end{array}
\right)
\end{equation}

The lumped mass matrix is then
\begin{eqnarray}
\bar{\bm M}^e 
&=&
\frac{1}{16}\frac{1}{105}
\left(
\begin{array}{cccc}
256 + 198 -72  +38 & 0 & 0 & 0  \\
0 & 198 + 1296  -162 -72 & 0 & 0\\
0 & 0 & -72 -162 + 1296 + 198 & 0\\
0 & 0 & 0 & 38 -72 + 198 + 256
\end{array}
\right) \nn\\
&=&
\frac{1}{16}\frac{1}{105}
\left(
\begin{array}{cccc}
420 & 0 & 0 & 0  \\
0 & 1260 & 0 & 0\\
0 & 0 & 1260 & 0\\
0 & 0 & 0 & 420
\end{array}
\right) \nn\\
&=&
\frac{1}{4}
\left(
\begin{array}{cccc}
1 & 0 & 0 & 0  \\
0 & 3 & 0 & 0\\
0 & 0 & 3 & 0\\
0 & 0 & 0 & 1
\end{array}
\right) \nn
\end{eqnarray}


We can easily verify that
\[
\sum_{ij} M_{ij} = 2
\qquad
\qquad
\sum_{ij} \bar{M}_{ij} = 2
\]


%.....................................
\subsubsection{Quartic basis functions}
There are now five nodes in the segment so that the mass matrix 
is now a $5\times5$ matrix. We have (see Section~\ref{sec:bf4}) 
\begin{equation}
{\vec N}^T(r) = 
\left(
\begin{array}{c}
N_1(r) \\ 
N_2(r) \\ 
N_3(r) \\ 
N_4(r) \\ 
N_5(r) 
\end{array}
\right)
=
\frac{1}{6}
\left(
\begin{array}{c}
  r- r^2 -4r^3 +4r^4 \\
  -8r+16 r^2 +8r^3 -16 r^4  \\
6 -30r^2+24r^4   \\
  8r+16 r^2 -8r^3 -16 r^4  \\
  -r- r^2 +4r^3 +4r^4
\end{array}
\right)
\end{equation}


\begin{eqnarray}
\int_{-1}^{+1} N_1(r) N_1(r) dr &=& \frac{1}{36}\frac{1168}{315} \nn\\ 
\int_{-1}^{+1} N_1(r) N_2(r) dr &=& \frac{1}{36}\frac{1184}{315} \nn\\ 
\int_{-1}^{+1} N_1(r) N_3(r) dr &=&-\frac{1}{36}\frac{232}{105}  \nn\\ 
\int_{-1}^{+1} N_1(r) N_4(r) dr &=& \frac{1}{36}\frac{32}{45}    \nn\\ 
\int_{-1}^{+1} N_1(r) N_5(r) dr &=&-\frac{1}{36}\frac{116}{315}  \nn\\ 
\int_{-1}^{+1} N_2(r) N_2(r) dr &=& \frac{1}{36}\frac{1024}{45}  \nn\\ 
\int_{-1}^{+1} N_2(r) N_3(r) dr &=&-\frac{1}{36}\frac{512}{105}  \nn\\ 
\int_{-1}^{+1} N_2(r) N_4(r) dr &=& \frac{1}{36}\frac{1024}{315} \nn\\ 
\int_{-1}^{+1} N_2(r) N_5(r) dr &=& \frac{1}{36}\frac{32}{45}    \nn\\ 
\int_{-1}^{+1} N_3(r) N_3(r) dr &=& \frac{1}{36}\frac{832}{35}    \nn\\ 
\int_{-1}^{+1} N_3(r) N_4(r) dr &=& -\frac{1}{36}\frac{512}{105}    \nn\\ 
\int_{-1}^{+1} N_3(r) N_5(r) dr &=& -\frac{1}{36}\frac{232}{105}    \nn\\ 
\int_{-1}^{+1} N_4(r) N_4(r) dr &=& \frac{1}{36}\frac{1024}{45}    \nn\\ 
\int_{-1}^{+1} N_4(r) N_5(r) dr &=& \frac{1}{36}\frac{1184}{315}    \nn\\ 
\int_{-1}^{+1} N_5(r) N_5(r) dr &=& \frac{1}{36}\frac{1168}{315}    
\end{eqnarray}


\begin{equation}
{\bm M}^e
=\frac{1}{36}
\frac{1}{315}
\left(
\begin{array}{ccccc}
1168 & 1184 & -696 & 224 & -116 \\
1184 & 7168   & -1536  &  1024 &  224 \\
-696 & -1536 & 7488  & -1536  & -696  \\
224   & 1024 & -1536 & 7168 & 1184 \\
-116 & 224   & -696 & 1184 & 1168
\end{array}
\right)
\end{equation}

The lumped mass matrix is then
\begin{eqnarray}
\bar{\bm M}^e 
&=&
=\frac{1}{36}
\frac{1}{315}
\left(
\begin{array}{ccccc}
1764 &0&0&0&0\\
0&8064 &0&0&0\\
0&0&3024 &0&0\\
0&0&0&8064 &0\\
0&0&0&0&1764 
\end{array}
\right)
=
\frac{1}{45}
\left(
\begin{array}{ccccc}
7 &0&0&0&0 \\
0&32 &0&0&0 \\
0&0&12  &0&0 \\
0&0&0&32   &0 \\
0&0&0&0&7      \\
\end{array}
\right)
\end{eqnarray}


We can once again easily verify that
\[
\sum_{ij} M_{ij} = 2
\qquad
\qquad
\sum_{ij} \bar{M}_{ij} = 2
\]


Note that all the integrals above were done very conveniently 
with the WolframAlpha software/website\footnote{\url{https://www.wolframalpha.com/}}.
Example:

\begin{center}
\includegraphics[width=12cm]{images/app_massmatrix/wolframalpha}
\end{center}







%---------------------------
\subsection{Quadrilaterals: rectangular elements}

We here assume that each element is a rectangle of size $h_x \times h_y$. 




%---------------------------
\subsection{Hexahedra: cuboid elements}

We here assume that each element is a cuboid\footnote{\url{https://en.wikipedia.org/wiki/Cuboid}} 
of size $h_x \times h_y \times h_z$. 









