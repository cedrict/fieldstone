
As we have seen above, one usually solves the heat transport equation (i.e. 
an advection-diffusion equation) in this form (source terms are neglected):
\begin{equation}
\rho C_p \left( \frac{\partial T}{\partial t} + {\vec \upnu}\cdot {\vec\nabla T} \right)
= {\vec \nabla} \cdot k \vec\nabla T 
\end{equation}
As we have seen in Section~\ref{ss:hte}, the diffusion term is actually the divergence of the heat flux
$\vec{q}=-k \vec \nabla T$. We could then choose to keep the heat flux as an unknown and 
solve a coupled system of equations instead:
\begin{eqnarray}
\rho C_p \left( \frac{\partial T}{\partial t} + {\vec \upnu}\cdot {\vec\nabla T} \right)
&=& - {\vec \nabla} \cdot \vec{q} \nn\\ 
\vec{q} &=& - k \vec \nabla T \nn
\end{eqnarray}
or, 
\begin{eqnarray}
\rho C_p \left( \frac{\partial T}{\partial t} + {\vec \upnu}\cdot {\vec\nabla T} \right)
+ {\vec \nabla} \cdot \vec{q} &=& \vec{0} \label{eq:htediff1}\\ 
\vec{q} + k \vec \nabla T &=& \vec{0} \label{eq:htediff2}
\end{eqnarray}
We have seen that the two left hand side terms of the first equation become 
${\bm M}^\uptheta \cdot \dot{\vec{T}}$ and ${\bm K}_a \cdot \vec{T}$.
Let $N^\uptheta$ be the temperature basis functions so that the temperature inside an element is 
given by
\begin{equation}
T^h({\vec r}) = \sum_{i=1}^{m_T} N^\uptheta_i ({\vec r}) T_i = \vec N^\uptheta \cdot \vec T
\end{equation}
where $\vec T$ is a vector of length $m_T$.
Let $N^q$ be the heat flux basis functions, and let us define (in 2D) $\vec{q}=(\qx,\qy)$ so that 
\begin{eqnarray}
\qx^h({\vec r}) &=& \sum_{i=1}^{m_q} N^q_i ({\vec r}) \qx_i = \vec{N}^q \cdot \vec{\qx} \\
\qy^h({\vec r}) &=& \sum_{i=1}^{m_q} N^q_i ({\vec r}) \qy_i = \vec{N}^q \cdot \vec{\qy} 
\end{eqnarray}
where $\vec{N}^q$, $\vec{\qx}$ and $\vec{\qy}$ are vectors of length $m_q$.
The weak form of the third term of Eq.~(\ref{eq:htediff1}) is then
\begin{eqnarray}
\int_\Omega N_i^\theta  \nabla \cdot \vec{q} \;  d\Omega
&=& -\int_\Omega N_i^\theta  
\left(  \frac{\partial }{\partial x}\qx + \frac{\partial }{\partial y}\qy \right) d\Omega \nn\\
&=& \int_\Omega N_i^\theta  \left(  \frac{\partial \vec{N}^q}{\partial x}\cdot \vec{\qx}
+  \frac{\partial \vec{N}^q}{\partial y}\cdot \vec{\qy} \right) d\Omega \nn\\
&=& \int_\Omega N_i^\theta  \frac{\partial \vec{N}^q}{\partial x}\cdot  \vec{\qx} \;  d\Omega + 
\int_\Omega  N_i^\uptheta \frac{\partial \vec{N}^q}{\partial y} \cdot \vec{\qy} \;  d\Omega \nn
\end{eqnarray}
Writing this last equation for $i=1,...m_T$ yields
\[
\underbrace{\left( \int_\Omega \vec{N}^\theta  \frac{\partial \vec{N}^q}{\partial x} \; d\Omega \right)}_{{\bm H}_x} \cdot\vec{\qx}+ 
\underbrace{\left( \int_\Omega  \vec{N}^\uptheta \frac{\partial \vec{N}^q}{\partial y}  \;  d\Omega\right)}_{{\bm H}_y} \cdot \vec{\qy}
\]
In the end, we obtain:
\begin{equation}
{\bm M}^\uptheta \cdot \dot{\vec{T}} + {\bm K}_a \cdot \vec{T} 
+ {\bm H}_x \cdot \vec{\qx}
+ {\bm H}_y \cdot \vec{\qy}
=\vec{0} \label{eq:htediff3}
\end{equation}
Turning now to Eq.~(\ref{eq:htediff2}), its weak form is
\[
\int_\Omega N_i^q \left( \vec{q} + k \vec \nabla T \right) d\Omega = \vec{0}
\]
and we can decompose it in its $x$ and $y$ components:
%\begin{eqnarray}
%\int_\Omega N_i^q \left( \qx + k  \frac{\partial T}{\partial x} \right) d\Omega &=& 0 \nn\\
%\int_\Omega N_i^q \left( \qy + k  \frac{\partial T}{\partial y} \right) d\Omega &=& 0 \nn
%\end{eqnarray}
%We process these further as follows:
\begin{eqnarray}
0&=&\int_\Omega N_i^q \left( \qx^h + k  \frac{\partial T^h}{\partial x} \right) \; d\Omega  \nn\\
&=& \int_\Omega N_i^q \left( \vec{N}^q \cdot \vec{\qx} 
+ k  \frac{\partial \vec{N}^\uptheta}{\partial x} \cdot \vec{T} \right) \; d\Omega \nn\\ 
&=& \int_\Omega N_i^q \vec{N}^q \cdot \vec{\qx} \; d\Omega  
+ \int _\Omega k N_i^q   \frac{\partial \vec{N}^\uptheta}{\partial x}\cdot \vec{T} \;  d\Omega \nn\\ 
0&=&\int_\Omega N_i^q \left( \qy^h + k  \frac{\partial T^h}{\partial y} \right) \;  d\Omega \nn\\
&=& \int_\Omega N_i^q \left( \vec{N}^q \cdot \vec{\qy} 
+ k  \frac{\partial \vec{N}^\uptheta}{\partial y} \cdot \vec{T} \right) \;  d\Omega \nn\\ 
&=& \int_\Omega N_i^q \vec{N}^q \cdot \vec{\qy} \; d\Omega
+ \int _\Omega k N_i^q   \frac{\partial \vec{N}^\uptheta}{\partial y} \cdot \vec{T} \;  d\Omega 
\end{eqnarray}
Writing these equations for $i=1,...m_q$ yields:
\begin{eqnarray}
0&=& \int_\Omega \vec{N}^q \vec{N}^q \cdot \vec{\qx} \; d\Omega
+ \int _\Omega k \vec{N}^q   \frac{\partial \vec{N}^\uptheta}{\partial x}\cdot \vec{T} \; d\Omega \nn\\ 
&=& \underbrace{\left( \int_\Omega \vec{N}^q \vec{N}^q d\Omega \right)}_{{\bm M}^q}  \cdot \vec{\qx} 
+ \underbrace{\left(\int _\Omega k \vec{N}^q  \frac{\partial \vec{N}^\uptheta}{\partial x}\; d\Omega\right)}_{{\bm G}_x} \cdot \vec{T} \label{eq:htediff4}\\ 
0&=& \int_\Omega \vec{N}^q \vec{N}^q \cdot \vec{\qy} \; d\Omega
+ \int _\Omega k \vec{N}^q   \frac{\partial \vec{N}^\uptheta}{\partial y} \cdot \vec{T} \;  d\Omega \nn\\ 
&=& \underbrace{\left( \int_\Omega \vec{N}^q \vec{N}^q d\Omega \right)}_{{\bm M}^q}  \cdot \vec{\qy} 
+ \underbrace{\left(\int _\Omega k \vec{N}^q  \frac{\partial \vec{N}^\uptheta}{\partial y} \; d\Omega\right)}_{{\bm G}_y} \cdot \vec{T} \label{eq:htediff5}
\end{eqnarray}

Finally Eqs.~(\ref{eq:htediff3},\ref{eq:htediff4},\ref{eq:htediff5}) 
can be combined and yield the following system (assuming an implicit backward 
Euler time scheme):
\[
\left(
\begin{array}{ccc}
{\bm M}^\uptheta + {\bm K}_a \delta t & {\bm H}_x \delta t  & {\bm H}_y \delta t \\
{\bm G}_x & {\bm M}^q & 0 \\
{\bm G}_y & 0 & {\bm M}^q 
\end{array}
\right)
\cdot
\left(
\begin{array}{c}
\vec{T}^{n+1}\\
\vec{\qx}^{n+1}\\
\vec{\qy}^{n+1}
\end{array}
\right)
=
\left(
\begin{array}{c}
{\bm M}^\uptheta \cdot \vec{T}^n\\
\vec{0} \\
\vec{0}
\end{array}
\right)
\]
If we choose $m_q=m_T$ and $N^q=N^\uptheta$ then
%then ${\bm H}_x = {\bm G}_x$,  ${\bm H}_y = {\bm G}_y$  and 
${\bm M}^\uptheta = {\bm M}^q = {\bm M}$ 
so that 
\[
\left(
\begin{array}{ccc}
{\bm M} + {\bm K}_a \delta t & {\bm H}_x \delta t & {\bm H}_y \delta t\\
{\bm G}_x & {\bm M} & 0 \\
{\bm G}_y & 0 & {\bm M}
\end{array}
\right)
\cdot
\left(
\begin{array}{c}
\vec{T}^{n+1}\\
\vec{\qx}^{n+1}\\
\vec{\qy}^{n+1}
\end{array}
\right)
=
\left(
\begin{array}{c}
{\bm M} \cdot \vec{T}^n\\
\vec{0} \\
\vec{0}
\end{array}
\right)
\]
Also, if $k$ is constant in space then ${\bm G}_{x,y}=k {\bm H}_{x,y}$. 
Rather interestingly, one could write Eqs.~(\ref{eq:htediff4},\ref{eq:htediff5}) 
as
\begin{eqnarray}
\vec{\qx}^{n+1} &=& - ({\bm M}^q)^{-1} \cdot {\bm G}_x \cdot \vec{T}^{n+1} \\
\vec{\qy}^{n+1} &=& - ({\bm M}^q)^{-1} \cdot {\bm G}_y \cdot \vec{T}^{n+1}
\end{eqnarray}
and inject it in Eq.~(\ref{eq:htediff3}) to yield:
\begin{equation}
{\bm M}^\uptheta \cdot \dot{\vec{T}} + [ {\bm K}_a 
- {\bm H}_x \cdot ({\bm M}^q)^{-1} \cdot {\bm G}_x 
- {\bm H}_y \cdot ({\bm M}^q)^{-1} \cdot {\bm G}_y ] \cdot \vec{T}^{n+1}
=\vec{0} 
\end{equation}
which means that we can directly solve for temperature! 
Rather interestingly, it is not equivalent to 
Eq.~(\ref{eq:hte_ibe}). Food for thought ...

We will see that this approach bears a lot of resemblance to the one taken in the context 
of Discontinuous Galerkin methods. 


























