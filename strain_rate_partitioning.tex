\index{general}{Strain rate partitioning}
\begin{flushright} {\tiny {\color{gray} strain\_rate\_partitioning.tex}} \end{flushright}
%~~~~~~~~~~~~~~~~~~~~~~~~~~~~~~~~~~~~~~~~~~~~~~~~~~~~~~~~~~~~~~~~~~~~~~~~~~~~~~~~~~~~~~~~~~~~~~~~~~

When multiple viscous deformation mechanisms are present, one needs more dashpots, and 
more complicated element diagrams than the ones above occur (also when adding plastic deformation).
Two important rules are to be remembered:
1) for parallel components, stresses are additive, strain rates are equal in each; 
2) for components in series, stresses are equal in each and strain rates are additive. 
Let us then look in the next subsections at various assemblies of dashpots and plastic elements.

%-----------------------------------------
\subsection{two viscous dampers in series} 


\begin{center}
\begin{tikzpicture}
%\draw[fill=gray!23,gray!23](0,0) rectangle (6,4.5);
%\draw[step=0.5cm,gray,very thin] (0,0) grid (6,4.5); %background grid

%left wall
\node[] at (0.1,2.7) {$\tau$};
\draw[line width=1mm] (0.5,1.5) -- (0.5,3.5) ;   
\draw [->] (0,2.5) -- (0.45,2.5);
\draw [->] (0,2) -- (0.45,2);
\draw [->] (0,3) -- (0.45,3);

%dashpots
\draw[thick] (0.5,2.5) -- (2,2.5) ;   
\draw[thick] (2.5,2.5) -- (4,2.5) ;   
\draw[thick] (4.5,2.5) -- (5.5,2.5) ;   
\draw[thick] (1.5,3) -- (2.5,3) -- (2.5,2) -- (1.5,2);  
\draw[thick] (3.5,3) -- (4.5,3) -- (4.5,2) -- (3.5,2);  

\node[] at (2,1.5) {$\eta_1$};
\node[] at (4,1.5) {$\eta_2$};

%wall
\draw[line width=2mm] (5.5,1.5) -- (5.5,3.5) ;   
\draw[thick] (5.5,1.5)  -- (5.75,1.75) ;   
\draw[thick] (5.5,1.75) -- (5.75,2) ;   
\draw[thick] (5.5,2)    -- (5.75,2.25) ;   
\draw[thick] (5.5,2.25) -- (5.75,2.5) ;   
\draw[thick] (5.5,2.5)  -- (5.75,2.75) ;   
\draw[thick] (5.5,2.75) -- (5.75,3) ;   
\draw[thick] (5.5,3)    -- (5.75,3.25) ;   
\draw[thick] (5.5,3.25) -- (5.75,3.5) ;

\draw[>=triangle 45, <->] (0.5,0.95) -- (3.,0.95);
\draw[>=triangle 45, <->] (3.,0.95) -- (5.5,0.95);
\node[] at (2,0.65)   {$\dot\varepsilon_{1}$};
\node[] at (4.25,0.65)   {$\dot\varepsilon_{2}$};

\end{tikzpicture}

\end{center}

Each is subjected to the same stress $\tau$ but deforms 
with its own strain rate $\dot{\varepsilon}_1$ and $\dot{\varepsilon}_2$ and we have 
\begin{equation}
\dot{\varepsilon}_T 
= \dot{\varepsilon}_1 + \dot{\varepsilon}_2
= \frac{\tau}{2\eta_1} + \frac{\tau}{2\eta_2}
\end{equation}
The effective viscosity of this combination is denoted $\eta_{eff}$ and is such that 
$\eta_{eff}=\tau/2\dot{\varepsilon}_T$, which means that 
\[
\frac{\tau}{2\eta_{eff}} = \frac{\tau}{2\eta_1} + \frac{\tau}{2\eta_2}
\]
or, 
\[
\eta_{eff}= \left( \frac{1}{\eta_1} + \frac{1}{\eta_2} \right)^{-1}
\]
i.e. it follows that the effective viscosity of two or more viscous dampers in series is the harmonic 
average of the individual viscosities of the dampers.

In general, for n dampers in series:
\[
\eta_{eff}= \left( \sum_{i=1}^n \frac{1}{\eta_i} \right)^{-1}
\]




%-------------------------------------------
\subsection{two viscous dampers in parallel}

\begin{center}
\begin{tikzpicture}
%\draw[fill=gray!23,gray!23](0,0) rectangle (7,4);
%\draw[step=0.5cm,gray,very thin] (0,0) grid (6.5,4.5); %background grid

%left wall
\node[] at (0.1,2.7) {$\tau$};
\draw[line width=1mm] (0.5,1.5) -- (0.5,3.5) ;   
\draw [->] (0,2.5) -- (0.45,2.5);
\draw [->] (0,2) -- (0.45,2);
\draw [->] (0,3) -- (0.45,3);

%dashpots
\draw[thick] (0.5,2.5) -- (1.5,2.5) ;   
\draw[thick] (4.5,2.5) -- (5.5,2.5) ;   
\draw[thick] (3,3.5) -- (1.5,3.5) -- (1.5,1.5) -- (3,1.5);  
\draw[thick] (3.5,3.5) -- (4.5,3.5) -- (4.5,1.5) -- (3.5,1.5);  
\draw[thick] (2.5,4) -- (3.5,4) -- (3.5,3) -- (2.5,3);  
\draw[thick] (2.5,2) -- (3.5,2) -- (3.5,1) -- (2.5,1);  

\node[] at (3.8,4) {$\eta_1$};
\node[] at (3.8,1) {$\eta_2$};

%wall
\draw[line width=2mm] (5.5,1.5) -- (5.5,3.5) ;   
\draw[thick] (5.5,1.5)  -- (5.75,1.75) ;   
\draw[thick] (5.5,1.75) -- (5.75,2) ;   
\draw[thick] (5.5,2)    -- (5.75,2.25) ;   
\draw[thick] (5.5,2.25) -- (5.75,2.5) ;   
\draw[thick] (5.5,2.5)  -- (5.75,2.75) ;   
\draw[thick] (5.5,2.75) -- (5.75,3) ;   
\draw[thick] (5.5,3)    -- (5.75,3.25) ;   
\draw[thick] (5.5,3.25) -- (5.75,3.5) ;

\draw[>=triangle 45, <->] (0.5,0.5) -- (5.5,0.5);
\node[] at (3,0.)   {$\dot\varepsilon_T$};

\end{tikzpicture}


\end{center}
 
each is deformed with the same strain rate $\dot{\varepsilon}_T$
and their stresses add up:
\[
\tau = \tau_1 + \tau_2 = 2 \eta_1 \dot{\varepsilon}_T  + 2 \eta_2 \dot{\varepsilon}_T
\]
and since we define the effective viscosity as $\tau = 2 \eta_{eff} \dot{\varepsilon}_T$ then it follows:
\[
2 \eta_{eff} \dot{\varepsilon}_T = 2 \eta_1 \dot{\varepsilon}_T  + 2 \eta_2 \dot{\varepsilon}_T
\]
or, 
\[
\eta_{eff} = \eta_1 + \eta_2 
\]
i.e., the effective viscosity of two or more viscous dampers is the sum of their viscosities ({\sl 
but not their arithmetic mean!}).



%----------------------------------------------------------------
\subsection{one viscous damper and a plastic element in parallel}

\begin{center}
\begin{tikzpicture}
%\draw[step=0.5cm,gray,very thin] (0,0) grid (6.5,5); %background grid

\node[] at (0.1,2.7) {$\tau$};
\draw[line width=1mm] (0.5,1.5) -- (0.5,3.5) ;   
\draw [->] (0,2.5) -- (0.45,2.5);
\draw [->] (0,2) -- (0.45,2);
\draw [->] (0,3) -- (0.45,3);

%horizontal lines
\draw[thick] (.5,2.5) -- (1.5,2.5) ;   
\draw[thick] (4.5,2.5) -- (5.5,2.5) ;   

%dashpot
\draw[thick] (2.5,4) -- (3.5,4) -- (3.5,3) -- (2.5,3);  
\node[] at (3,4.4) {$\eta_m$};

\draw[thick] (3,3.5) -- (1.5,3.5) -- (1.5,1.5) -- (2,1.5);  
\draw[thick] (3.5,3.5) -- (4.5,3.5) -- (4.5,1.5) -- (4,1.5);  

%plastic elt
\draw[thick] (2,1.5) -- (2.5,1.4) -- (3.5,1.4) ;   
\draw[thick] (2.5,1.6) -- (3.5,1.6) -- (4,1.5) ;   
\node[] at (3,1.92) {$Y$};

%wall
\draw[line width=2mm] (5.5,1.5) -- (5.5,3.5) ;   
\draw[thick] (5.5,1.5)  -- (5.75,1.75) ;   
\draw[thick] (5.5,1.75) -- (5.75,2) ;   
\draw[thick] (5.5,2)    -- (5.75,2.25) ;   
\draw[thick] (5.5,2.25) -- (5.75,2.5) ;   
\draw[thick] (5.5,2.5)  -- (5.75,2.75) ;   
\draw[thick] (5.5,2.75) -- (5.75,3) ;   
\draw[thick] (5.5,3)    -- (5.75,3.25) ;   
\draw[thick] (5.5,3.25) -- (5.75,3.5) ;

\draw[>=triangle 45, <->] (0.5,0.95) -- (5.5,0.95);
\node[] at (3.,0.45)   {$\dot\varepsilon_{T}$};

\end{tikzpicture}

\end{center}

The effective 'plastic' viscosity of the plastic element is $\eta_p =  \frac{Y}{2 \dot{\varepsilon}_T}$ so 
the effective viscosity of this setup is then  
\[
\eta_{eff} = \frac{Y}{2 \dot{\varepsilon}_T}+\eta_m
\]
which is the viscosity of a Bingham fluid (see Section~\ref{sec:bingham}).




%-----------------------------------------------------
\subsection{two viscous dampers and a plastic element} 


\begin{center}

\begin{center}
\begin{tikzpicture}
%\draw[fill=gray!23,gray!23](0,0) rectangle (7,5);
%\draw[step=0.5cm,gray,very thin] (0,0) grid (7,5); %background grid

\draw[thick] (0.5,2.5) -- (2,2.5) ;  
\draw[thick] (2.5,2.5) -- (4.5,2.5) ;  
\draw[thick] (7.5,2.5) -- (9,2.5) ;  

\draw[thick] (1.5,3) -- (2.5,3) -- (2.5,2) -- (1.5,2);  
\draw[thick] (5.5,4) -- (6.5,4) -- (6.5,3) -- (5.5,3);  

\node[] at (2,1.6) {$\eta_M$};
\node[] at (6,4.4) {$\eta_m$};

\draw[thick] (6,3.5) -- (4.5,3.5) -- (4.5,1.5) -- (5,1.5);  
\draw[thick] (6.5,3.5) -- (7.5,3.5) -- (7.5,1.5) -- (7,1.5);  

\draw[thick] (5,1.5) -- (5.5,1.4) -- (6.5,1.4) ;  
\draw[thick] (5.5,1.6) -- (6.5,1.6) -- (7,1.5) ;  

\node[] at (6,1.) {$c$};


%\draw[thick] (1,1) -- (3,1.2) -- (2.7,3) -- (1.1,3.1) -- cycle;  
%\node[] at (0.8,0.8) {0};
%\node[] at (3.2,1) {1};
%\node[] at (2.9,3.1) {2};
%\node[] at (0.9,3.2) {3};
%\draw[black,fill=black] (1,1)     circle (2pt); \draw (1,1) circle (4pt);
%\draw[black,fill=black] (3,1.2)   circle (2pt); \draw (3,1.2) circle (4pt);
%\draw[black,fill=black] (2.7,3)   circle (2pt); \draw (2.7,3) circle (4pt);
%\draw[black,fill=black] (1.1,3.1) circle (2pt); \draw (1.1,3.1) circle (4pt);
%\draw[black,fill=black] (3.1,0.2) circle (2pt); \node[] at (3.4,0.2) {$\vec\upnu$};
%\draw (4.1,0.2) circle (4pt); 
%\node[] at (2.5,4.5) {4 vel. nodes, 4 press. nodes};
\end{tikzpicture}\\
\end{center}

\end{center}
This rheology would be called visco-viscoplastic.
The algorithm goes then as follows:
\begin{enumerate}
\item Assume we know $\eta_v$ and $\dot\varepsilon_T$ (from previous iteration), as well as the plasticity parameters $Y$ and $\eta_m$.
\item if $2 \eta_v \dot\varepsilon_T < Y$ the stress is below the yield stress value and plasticity is not active. Use $\eta_v$ in the material model and $\dot\varepsilon_v=\dot\varepsilon_T$.

\item if $2 \eta_v \dot\varepsilon_T > Y$ the stress is above the yield value, which is not allowed. In this case the plastic element is 'switched on'. In that case the viscous damper is in series with the (visco)plastic element. The former deforms with a strain rate $\dot\epsilon_v$ while the latter with $\dot\epsilon_{vp}$ (both under the same stress $\tau$) and we have  $\dot\varepsilon_T = \dot\varepsilon_v  + \dot\varepsilon_{vp}$. 

\begin{eqnarray}
\dot\varepsilon_T 
&=& \dot\varepsilon_v + \dot\varepsilon_{vp}  \nonumber\\
&=& \dot\varepsilon_v + \frac{\tau}{2 \eta_{vp}} \nonumber\\
&=& \dot\varepsilon_v + \frac{\tau}{2 \left( \frac{Y}{2\dot\varepsilon_{vp}} + \eta_m  \right)} \nonumber\\
&=& \dot\varepsilon_v + \frac{\tau}{2 \left( \frac{Y}{2(\dot\varepsilon_T-\dot\varepsilon_v)}+\eta_m\right)} \nonumber\\
\dot\varepsilon_T - \dot\varepsilon_v 
&=& \frac{\tau}{2 \left( \frac{Y}{2(\dot\varepsilon_T-\dot\varepsilon_v)}+\eta_m\right)} \nonumber\\
2 (\dot\varepsilon_T - \dot\varepsilon_v)
\left( \frac{Y}{2(\dot\varepsilon_T-\dot\varepsilon_v)}+\eta_m\right) &=& \tau \nonumber\\
Y +  2(\dot\varepsilon_T - \dot\varepsilon_v) \eta_m &=& \tau \nonumber\\
Y +  2(\dot\varepsilon_T - \frac{\tau}{2 \eta_v}) \eta_m &=& \tau \nonumber\\
Y +  (2\eta_v \dot\varepsilon_T - \tau) \frac{\eta_m}{\eta_v} &=& \tau \nonumber\\
Y +  2\eta_m \dot\varepsilon_T  &=& \tau (1 + \frac{\eta_m}{\eta_v} ) \nonumber
\end{eqnarray}
and finally 
\begin{equation}
\tau  = \frac{Y + 2 \eta_m \dot\varepsilon_T} {1+ \frac{\eta_m}{\eta_v} }
\end{equation}
Note that this solution exists even when $\eta_m=0$, and then rather logically $\tau=Y$.

\item Once we have $\tau$, we can easily compute $\dot\epsilon_v = \frac{\tau}{2\eta_v}$

\item We then compute $\dot\varepsilon_{vp} = \dot\varepsilon_T- \dot\varepsilon_v$ which 
we use to compute $\eta_{vp}$:

\begin{eqnarray}
\eta_{vp} 
&=& \frac{Y}{2\dot\varepsilon_{vp}}+\eta_m \nn\\
&=& \frac{Y}{2(\dot\varepsilon_{T}-\dot\varepsilon_{v})}+\eta_m \nn\\
&=& \frac{Y}{2(\dot\varepsilon_{T}-\frac{\tau}{2\eta_v})}+\eta_m \nn\\
&=& \frac{Y}{2(\dot\varepsilon_{T}- \frac{Y + 2 \eta_m \dot\varepsilon_T} {1+ \frac{\eta_m}{\eta_v} }   
\frac{1}{2\eta_v})}+\eta_m \nn\\
&=& \frac{Y}{2\dot\varepsilon_{T}- \frac{Y + 2 \eta_m \dot\varepsilon_T} {\eta_v + \eta_m}     }+\eta_m \\
&=& \frac{Y(\eta_v+\eta_m)}{2(\eta_v+\eta_m) \dot\varepsilon_{T}- (Y + 2 \eta_m \dot\varepsilon_T) }+\eta_m \\
&=& \frac{Y(\eta_v+\eta_m)}{2 \eta_v \dot\varepsilon_{T}- Y  }+\eta_m \\
&=& \frac{Y(\eta_v+\eta_m)/2\eta_v}{ \dot\varepsilon_{T}- Y/2\eta_v  }+\eta_m 
\end{eqnarray}


\item Having obtained $\eta_{vp}$ we can compute the final effective viscosity
\[
\eta_{eff} = \left( \frac{1}{\eta_v}  + \frac{1}{\eta_{vp}}  \right)^{-1}
\]
\end{enumerate}

On the following plots are shown $\tau$, 
$\dot\varepsilon_{vp}$, $\dot\varepsilon_v$, $\eta_vp$, and $\eta_{eff}$ 
as a function of  $\dot\varepsilon_T$: 

\begin{center}
\includegraphics[width=5.25cm]{images/rheology/vvp/tau.pdf}
\includegraphics[width=5.25cm]{images/rheology/vvp/strainrates.pdf}
\includegraphics[width=5.25cm]{images/rheology/vvp/viscosities.pdf}\\
{\captionfont Obtained for $\eta_m=10^{21}$, $Y=20$MPa and $\eta_v=10^{25}$. Python code 
in images/rheology/vvp/}
\end{center}

In the following plots the resulting stress $\tau$ and effective viscosities $\eta_{eff}$
are compared between the above approach ('new') and the simpler (and naive) 
approach where $\dot\varepsilon_T$ 
is used in $\eta_{vp}$ instead of $\dot\varepsilon$ ('old'). In this particular case 
we see that it makes a difference at low strain rates close to the brittle-ductile transition.

\begin{center}
\includegraphics[width=8cm]{images/rheology/vvp/tau_comp.pdf}
\includegraphics[width=8cm]{images/rheology/vvp/viscosities_comp.pdf}\\
{\captionfont Obtained for $\eta_m=10^{21}$, $Y=20$MPa and $\eta_v=10^{25}$. Python code 
in images/rheology/vvp/}
\end{center}

\begin{remark}
The introduction of the damper $\eta_m$ in parallel with the plastic element has an unavoidable
effect: the stress $\tau$ becomes larger than $Y$ at high strain rate values! Since the $vp$ 
block is akin to a bingham fluid, this is no surprise.
\end{remark}

\begin{remark}
The viscous dashpot $\eta_v$ also acts as a maximum viscosity cutoff: if $\eta_{vp}$ becomes (very) large, i.e. $\eta_{vp} \gg \eta_v$, then $\eta_{eff} \rightarrow \eta_v$.
Conversely, if $\eta_p=Y/2\dot\varepsilon_{vp}$ becomes (very) small, i.e. $\eta_p \ll \eta_m$ then $\eta_m$ acts as a minimum viscosity limiter, i.e. $\eta_{vp} \rightarrow \eta_m$. 
Since $\eta_m \ll \eta_v$ then $\eta_{eff} \rightarrow \eta_m$.
\end{remark}

\underline{A simple regularisation} This idea originates in Massmeyer \etal (2013) \cite{madd13}. We postulate
\[
\tilde{\eta}_{eff} = \left(  1 - \exp (- \frac{\dot\varepsilon_T}{\dot\varepsilon_{T}^c}) \right)
\left( \frac{Y}{2 \dot\varepsilon_T} + \eta_m \right)
\]
where $\dot\varepsilon_{T}^c$ is the critical strain  rate at which the transition viscous to 
viscous-viscoplastic occurs given by $\dot\varepsilon_{T}^c=Y/2\eta_v$.
When $\dot\varepsilon_{T} \ll \dot\varepsilon_{T}^c$ then the exponential term tends to zero and 
\[
\tilde{\eta}_{eff} \rightarrow  \frac{Y}{2 \dot\varepsilon_T} + \eta_m 
\]
and if $\dot\varepsilon_{T} \rightarrow \infty$ then $\tilde{\eta}_{eff}\rightarrow \eta_m$.
Conversely if $\dot\varepsilon_T \rightarrow 0$ then we can carry out a Taylor expansion of the exponential 
term ($\exp x \sim 1 + x$ when $x$ is small).
\[
\tilde{\eta}_{eff} \sim \left(  \frac{\dot\varepsilon_T}{\dot\varepsilon_{T}^c} \right)
\left( \frac{Y}{2 \dot\varepsilon_T} + \eta_m \right)
\rightarrow 
\frac{\dot\varepsilon_T}{\dot\varepsilon_{T}^c}  \frac{Y}{2 \dot\varepsilon_T}  = \eta_v
\]
At low strain rates the viscosity does not 'explode' but actually converges to the background viscosity $\eta_v$.
The stress $\tau$ corresponding to this viscosity is simply $\tilde{\tau} = 2 \tilde{\eta}_{eff}$. 
Both $\tilde{\tau}$ and $ \tilde{\eta}_{eff}$ are plotted hereunder:


\begin{center}
\includegraphics[width=7.5cm]{images/rheology/vvp/tau_reg.pdf}
\includegraphics[width=7.5cm]{images/rheology/vvp/viscosities_reg.pdf}\\
{\captionfont Obtained for $\eta_m=10^{21}$, $Y=20$MPa and $\eta_v=10^{25}$. Python code 
in images/rheology/vvp/}
\end{center}


\includegraphics[width=7.5cm]{images/rheology/vvp/ratio_visc.pdf}



%which, if  yields the following effective viscosity:
%\[
%\eta_{eff} = \left( \frac{1}{\eta_M}  + \frac{1}{\frac{Y}{2 \dot{\varepsilon}_e} + \eta_m}  \right)^{-1}
%\]
%When the strain rate becomes very small,  $\dot{\varepsilon}_e \rightarrow 0$, $\eta_{eff}\rightarrow \eta_{M}$.
%When the strain rate becomes very large,  $\dot{\varepsilon}_e \rightarrow \infty$, $\eta_{eff}\rightarrow \eta_{m}$.
%We can then rewrite the above equation as a function of $\eta_{min}$ and $\eta_{max}$:
%\[
%\eta_{eff} = \left( \frac{1}{\eta_{max}}  + \frac{1}{\frac{c}{2 \dot{\varepsilon}_e} + \eta_{min}}  \right)^{-1}
%\]
%
%The effective viscosity is plotted here for various values of the minimum viscosity (for $c$=200MPa and $\eta_{max}=10^{25}Pa.s$:
%\includegraphics[width=8cm]{images/viscoplasticity/nu_eff}












%---------------------------------------------------
\subsection{two nonlinear viscous dampers in series} 

\begin{center}
\input{tikz/tikz_dashpotdashpot_series_nl}
\end{center}

There are two dashpots in series, one accounts for dislocation creep, the other for diffusion creep.
The algorithm goes then as follows:
\begin{enumerate}
\item Assume we know $\dot\varepsilon_T$ (from previous iteration). 
\item The dashpots are in series so 
\[
\dot\varepsilon_T = \dot\varepsilon_{ds} + \dot\varepsilon_{df} 
\]
with
\begin{eqnarray}
\dot\varepsilon_{ds}  &=& A_{ds} \tau^n \exp \left(-\frac{Q_{ds}+pV_{ds}}{RT}\right) \label{sr_ds1} \\
\dot\varepsilon_{df}  &=& A_{df} \tau   \exp \left(-\frac{Q_{df}+pV_{df}}{RT}\right) \label{sr_df1} 
\end{eqnarray}
such that we are in fact looking for the stress value $\tau$ so that 
\[
\dot\varepsilon_T = 
A_{ds} \tau^n \exp \left(-\frac{Q_{ds}+p V_{ds}}{RT}\right) 
+
A_{df} \tau   \exp \left(-\frac{Q_{df}+p V_{df}}{RT}\right) 
\]
or, we must find the zero of the function ${\cal F}(\tau)$: 
\[
{\cal F}(\tau) =  \dot\varepsilon_T 
- A_{ds} \tau^n \exp \left(-\frac{Q_{ds}+p V_{ds}}{RT}\right) 
- A_{df} \tau   \exp \left(-\frac{Q_{df}+p V_{df}}{RT}\right) 
\]
This equation can be solved with a Newton-Raphson algorithm
and the iterations will be of the form:
\[
\tau_{n+1} = \tau_n - \frac{{\cal F}(\tau_n)}{{\cal F}'(\tau_n)}
\]
where the derivative of the function ${\cal F}$ with respect to $\tau$ reads:
\[
{\cal F}'(\tau)=\frac{\partial {\cal F}}{\partial \tau}=
- A_{ds} n \tau^{n-1} \exp\left(-\frac{Q_{ds}+pV_{ds}}{RT}\right)
- A_{df} \exp\left(-\frac{Q_{df}+pV_{df}}{RT}\right) 
\]
Once the value of $\tau$ is found, 
the strain rate values of Eqs. (\ref{sr_ds1}) and (\ref{sr_df1})
can be computed and so can the respective effective viscosities:
\begin{eqnarray}
\eta_{ds} 
&=& \frac{1}{2} A_{ds}^{1/n} \dot\varepsilon_{ds}^{\frac{1}{n}-1} \exp \left(\frac{Q_{ds}+pV_{ds}}{nRT}\right) \\
\eta_{df} 
&=& \frac{1}{2} A_{df}^{1/n}  \exp \left(\frac{Q_{df}+pV_{df}}{RT}\right) 
\end{eqnarray}
Their average effective viscosity $\tilde{\eta}_{eff}$ is given by 
\[
\tilde{\eta}_{eff} = \left( \frac{1}{\eta_{ds}} + \frac{1}{\eta_{df}} \right)^{-1}
\]
\end{enumerate}


Rather importantly, as we will see hereafter, the following variant is implemented 
in some codes (e.g. \douar, \fantom, \sopale, and probably many others) 
so as to bypass these costly Newton iterations:
\begin{enumerate}
\item compute $\eta_{ds}$ and $\eta_{df}$ with the {\it same} strainrate $\dot\varepsilon_T$, 
pressure and temperature values
\item average them by means of an harmonic average
\end{enumerate}
In this case, we have
\[
\dot{\varepsilon}_{\color{red} T}= 
A_{df} \tau_{df} \exp\left(-\frac{Q_{df}+pV_{df}}{RT}\right)
\quad\quad\quad
\dot{\varepsilon}_{\color{red} T}= 
A_{ds} \tau_{ds}^n \exp\left(-\frac{Q_{ds}+pV_{ds}}{RT}\right)
\]
or, 
\begin{eqnarray}
\eta_{ds} 
&=& \frac{1}{2} A_{ds}^{1/n} \dot\varepsilon_{\color{red}T}^{\frac{1}{n}-1} \exp \left(\frac{Q_{ds}+pV_{ds}}{nRT}\right) \\
\eta_{df} 
&=& \frac{1}{2} A_{df}^{1/n}  \exp \left(\frac{Q_{df}+pV_{df}}{RT}\right) 
\end{eqnarray}
We see that this simplification has consequences on the dislocation creep viscosity only.


\paragraph{A concrete example}
Let us consider a vertical section of upper mantle, from 660\si{\km} depth to 30\si{\km} depth.
The lithosphere is assumed to be 90\si{\km} thick. The temperature at the moho (the top
of the domain) is set to 550C, 1330C at the LMB and 1380C at the bottom.
A constant strainrate $\dot{\epsilon}_T=10^{-15}\si{\per\second}$ is assumed. 
We assume that the pressure is lithostatic (for simplicity 
the density is taken to be constant at 3300kg/m$^3$).
The temperature and pressure fields are shown hereunder:
\begin{center}
\includegraphics[width=6cm]{images/rheology/effvisc/temperature.pdf}
\includegraphics[width=6cm]{images/rheology/effvisc/pressure.pdf}
\end{center}

Material properties are taken from Karato \& Wu (1993) \cite{kawu93}.
The (fortran) code is available in {\tt images/rheology/effvisc/}.

In what follows, the values obtained with Newton iterations are coined 'NR'
and those obtained without are coined 'CHEAP'.
The diffusion and dislocation creep viscosities can be
computed for both algorithms and are shown hereunder
(As mentioned earlier the diffusion creep viscosity is independent of strain rate so
is the same for both):
\begin{center}
\includegraphics[width=6cm]{images/rheology/effvisc/both_mu_ds.pdf}
\includegraphics[width=6cm]{images/rheology/effvisc/both_mu_df.pdf}
\end{center}
We can also plot the resulting effective viscosity 
$\eta_{eff}$ for both approaches and we see that the differences 
are larger than 20\%. This is shown here under on the left, 
alongside with the partitioning of the strain rate as a function of depth:
\begin{center}
\includegraphics[width=8cm]{images/rheology/effvisc/both_mueff.pdf}
\includegraphics[width=8cm]{images/rheology/effvisc/both_sr.pdf}
\end{center}





%----------------------------------------------------------
\subsection{multiple viscous dampers and a plastic element} 


\begin{center}

\begin{center}
\begin{tikzpicture}
%\draw[fill=gray!23,gray!23](0,0) rectangle (7,5);
%\draw[step=0.5cm,gray,very thin] (0,0) grid (12,5); %background grid

\node[] at (0.1,2.7) {$\tau$};
\draw[line width=1mm] (0.5,1.5) -- (0.5,3.5) ;  
\draw [->] (0,2.5) -- (0.45,2.5);
\draw [->] (0,2) -- (0.45,2);
\draw [->] (0,3) -- (0.45,3);

%linear viscous

\draw[thick] (1.,3) -- (2.,3) -- (2.,2) -- (1.,2);  
\draw[thick] (0.5,2.5) -- (1.5,2.5) ;  
\draw[>=triangle 45, <->] (0.5,0.95) -- (2.5,0.95);
\node[] at (1.7,0.75) {$\dot\varepsilon_{v}$};
\node[] at (1.5,1.6) {$\eta_{m}$};


%dislocation creep
\draw[thick] (2.,2.5) -- (4,2.5) ;  
\draw[thick] (4.5,2.5) -- (6.5,2.5) ;  
\draw[thick] (3.5,3) -- (4.5,3) -- (4.5,2) -- (3.5,2);  
\node[] at (4,1.6) {$\eta_{ds}$};
\draw[>=triangle 45, <->] (2.5,0.95) -- (5,0.95);
\node[] at (4,0.75)   {$\dot\varepsilon_{ds}$};

%diffusion creep
\draw[thick] (6.,3) -- (7.,3) -- (7.,2) -- (6.,2);  
\node[] at (6.5,1.6) {$\eta_{df}$};
\draw[thick] (7,2.5) -- (8.5,2.5) ;  
\draw[>=triangle 45, <->] (5,0.95) -- (7.5,0.95);
\node[] at (6.5,0.75)   {$\dot\varepsilon_{df}$};

%viscoplastic element
\draw[thick] (9.5,4) -- (10.5,4) -- (10.5,3) -- (9.5,3);  
\node[] at (10,4.3) {$\eta_m$};
\node[] at (10,1.9) {$Y$};
\draw[thick] (10,3.5) -- (8.5,3.5) -- (8.5,1.5) -- (9,1.5);  
\draw[thick] (10.5,3.5) -- (11.5,3.5) -- (11.5,1.5) -- (11,1.5);  
\draw[thick] (9,1.5) -- (9.5,1.4) -- (10.5,1.4) ;  
\draw[thick] (9.5,1.6) -- (10.5,1.6) -- (11,1.5) ;  
\draw[thick] (11.5,2.5) -- (13,2.5) ;  
\draw[>=triangle 45, <->] (7.5,0.95) -- (13,0.95);
\node[] at (10.2,0.75) {$\dot\varepsilon_{vp}$};

%wall
\draw[line width=2mm] (13,1.5) -- (13,3.5) ;  
\draw[thick] (13,1.5)  -- (13.25,1.75) ;  
\draw[thick] (13,1.75) -- (13.25,2) ;  
\draw[thick] (13,2)    -- (13.25,2.25) ;  
\draw[thick] (13,2.25) -- (13.25,2.5) ;  
\draw[thick] (13,2.5)  -- (13.25,2.75) ;  
\draw[thick] (13,2.75) -- (13.25,3) ;  
\draw[thick] (13,3)    -- (13.25,3.25) ;  
\draw[thick] (13,3.25) -- (13.25,3.5) ;  





\end{tikzpicture}\\
\end{center}

\end{center}

The algorithm goes then as follows:
\begin{enumerate}
\item Assume we know $\dot\varepsilon_T$ (from previous iteration), 
as well as the plasticity parameters $Y$ (a constant in the case of von Mises, or a pressure-dependent 
quantity otherwise) and $\eta_m$.
\item We start by assuming that the plasticity 'block' is not active ($\dot{\varepsilon}_{vp}=0$): we have then three dampers in series. 
We need their associated strain rates 
$\dot{\varepsilon}_{df}$ and $\dot{\varepsilon}_{ds}$ which are such that 
\[
\dot\varepsilon_T = 
\dot\varepsilon_v + \dot\varepsilon_{ds} + \dot\varepsilon_{df} 
\]
with
\begin{eqnarray}
\dot\varepsilon_v &=& \frac{\tau}{2 \eta_v} \label{sr_v} \\
\dot\varepsilon_{ds}  &=& A_{ds} \tau^n \exp \left(-\frac{Q_{ds}+pV_{ds}}{RT}\right) \label{sr_ds} \\
\dot\varepsilon_{df}  &=& A_{df} \tau   \exp \left(-\frac{Q_{df}+pV_{df}}{RT}\right) \label{sr_df} 
\end{eqnarray}
such that we are in fact looking for the stress value $\tau$ so that 
\[
\dot\varepsilon_T = 
A_{ds} \tau^n \exp \left(-\frac{Q_{ds}+p V_{ds}}{RT}\right) 
+
A_{df} \tau   \exp \left(-\frac{Q_{df}+p V_{df}}{RT}\right) 
+
\frac{\tau}{2 \eta_v}
\]
or, we must find the zero of the function ${\cal F}$: 
\[
{\cal F}(\tau) = 
\dot\varepsilon_T 
- A_{ds} \tau^n \exp \left(-\frac{Q_{ds}+p V_{ds}}{RT}\right) 
- A_{df} \tau   \exp \left(-\frac{Q_{df}+p V_{df}}{RT}\right) 
- \frac{\tau}{2 \eta_v} 
\]
This equation can be solved with a Newton-Raphson algorithm
and the iterations will be of the form:
\[
\tau_{n+1} = \tau_n - \frac{{\cal F}(\tau_n)}{{\cal F}'(\tau_n)}
\]
where the derivative of the function ${\cal F}$ with respect to $\tau$ reads:
\[
{\cal F}'(\tau)=\frac{\partial {\cal F}}{\partial \tau}=
- A_{df} \exp\left(-\frac{Q_{df}+pV_{df}}{RT}\right) 
- A_{ds} n \tau^{n-1} \exp\left(-\frac{Q_{ds}+pV_{ds}}{RT}\right)
- \frac{1}{2\eta_v}
\]
Once the value of $\tau$ is found, 
the strain rate values of Eqs. (\ref{sr_ds}), (\ref{sr_df}) and (\ref{sr_v}) 
can be computed and so can the respective effective viscosities:
\begin{eqnarray}
\eta_{ds} 
&=& \frac{1}{2} A_{ds}^{1/n} \dot\varepsilon_{ds}^{\frac{1}{n}-1} \exp \left(\frac{Q_{ds}+pV_{ds}}{nRT}\right) \\
\eta_{df} 
&=& \frac{1}{2} A_{df}^{1/n}  \exp \left(\frac{Q_{df}+pV_{df}}{RT}\right) 
\end{eqnarray}
Their average effective viscosity $\tilde{\eta}_{eff}$ is given by 
\[
\tilde{\eta}_{eff} = \left( \frac{1}{\eta_{ds}} + \frac{1}{\eta_{df}} + \frac{1}{\eta_v} \right)^{-1}
\]


\item if $\tau =2 \tilde{\eta}_{eff} \dot\varepsilon_T < Y$ the stress is below the yield stress value 
and the plasticity element is indeed not active. Use $\tilde{\eta}_{eff}$ in the material model.

\item if $\tau=2 \tilde{\eta}_{eff} \dot\varepsilon_T > Y$ the stress is above the yield value, which is not 
allowed. In this case the plastic element must be present and active and the viscous dampers are then 
in series with the (visco)plastic element. The formers deform 
with a strain rate $\dot{\varepsilon}_v$, $\dot\epsilon_{ds}$ and $\dot{\epsilon}_{df}$ 
while the latter with $\dot\epsilon_{vp}$ (all under the same tress $\tau$) 
and we have  $\dot\varepsilon_T = \dot{\varepsilon}_v + \dot\varepsilon_{ds} + \dot\varepsilon_{df} + \dot\varepsilon_{vp}$ so:

\begin{eqnarray}
\dot\varepsilon_T - \dot{\varepsilon}_v(\tau) - \dot\varepsilon_{ds}(\tau) - \dot\varepsilon_{df}(\tau)  
&=& \dot\varepsilon_{vp}  \nonumber\\
&=& \frac{\tau}{2 \left( \frac{Y}{2\dot\varepsilon_{vp}} + \eta_m  \right)} 
\nonumber\\
%&=&  \frac{\tau}{2 \left( \frac{Y}{2 (\dot\varepsilon_T -\dot{\varepsilon}_v(\tau) -\dot\varepsilon_{ds}(\tau) 
%+ \dot\varepsilon_{df}(\tau)   )} + \eta_m  \right)} 
%\nonumber\\
\dot\varepsilon_T -  \dot{\varepsilon}_v(\tau) -\dot\varepsilon_{ds}(\tau) - \dot\varepsilon_{df}(\tau) 
&=&
\frac{\tau}{2 \left( \frac{Y}{2 (\dot\varepsilon_T -\dot{\varepsilon}_v(\tau) -\dot\varepsilon_{ds}(\tau) 
+ \dot\varepsilon_{df}(\tau)   )} + \eta_m  \right)} \nonumber
\\
2 [\dot\varepsilon_T -\dot{\varepsilon}_v(\tau) -\dot\varepsilon_{ds}(\tau) - \dot\varepsilon_{df}(\tau) ]
 \left( \frac{Y}{2 (\dot\varepsilon_T -\dot{\varepsilon}_v(\tau) -\dot\varepsilon_{ds}(\tau) + \dot\varepsilon_{df}(\tau)   )} + \eta_m  \right) &=& \tau 
\nonumber\\
Y + 2 (\dot\varepsilon_T - \dot{\varepsilon}_v(\tau) -\dot\varepsilon_{ds}(\tau) - \dot\varepsilon_{df}(\tau) ) \eta_m  &=& \tau \nonumber 
\end{eqnarray}
As before, we must find the zero of the function ${\cal F}$: 
\begin{eqnarray}
{\cal F}(\tau) 
&=& Y + 2 [\dot\varepsilon_T -\dot{\varepsilon}_v(\tau)- \dot\varepsilon_{ds}(\tau) -\dot\varepsilon_{df}(\tau) ]\eta_m -\tau \nonumber\\
&=& Y + 2 \left[ 
\dot\varepsilon_T - \frac{\tau}{2\eta_v}-A_{ds} \tau^n \exp \left(-\frac{Q_{ds}+pV_{ds}}{RT}\right) 
- A_{df} \tau   \exp \left(-\frac{Q_{df}+pV_{df}}{RT}\right)  
\right]\eta_m -\tau \nn
\end{eqnarray}
Because dislocation creep involves the $n$-th power of the stress we will here also need 
to find the zero by means of a Newton-Raphson algorithm. 

We have:
\begin{eqnarray}
\frac{\partial {\cal F}}{\partial \tau} &=&
\left[
- \frac{1}{\eta_v} 
-2 \frac{\partial  \dot\varepsilon_{ds}(\tau) }{\partial \tau} 
-2 \frac{\partial  \dot\varepsilon_{df}(\tau) }{\partial \tau} 
\right] \eta_m - 1
\end{eqnarray}

\begin{eqnarray}
{\cal F}(\tau) / 2\eta_m 
&=& 
\frac{Y}{2\eta_m}  + \dot\varepsilon_T 
- \frac{\tau}{2\eta_v}
- A_{ds}(p,T) \tau^n 
- A_{df}(p,T) \tau  
- \frac{\tau}{2\eta_m} \\
&=& 
- A_{ds}(p,T) \tau^n - \left(A_{df}(p,T) +  \frac{1}{2\eta_v} -\frac{1}{2\eta_m} \right) \tau
+ 
\left( \frac{Y}{2\eta_m}  + \dot\varepsilon_T \right) =0
\end{eqnarray}



\end{enumerate}

Note that when $\eta_m=0$ we logically recover $\tau=Y$ as the stress cannot exceed the yield strength $Y$.

Although this approach is probably the most consistent in terms of physics, the presence 
of the Newton-Raphson iterations makes it very expensive since this procedure is to be repeated 
for every quadrature point or every particle.

Let us consider a concrete example: we set $Y=20\si{\mega\pascal}$, $\eta_v=10^{25}\si{pascal}$, 
$\eta_m=10^{20}\si{pascal}$. The domain is one-dimensional of depth $660\si{km}$. The density is
assumed to be constant at $3300\si{\kg\per\cubic\metre}$. Dislocation and diffusion creep parameters
are taken from Karato \& Wu (1993) \cite{kawu93}. The temperature is linear is $20\si{\celsius}$ 
at the surface, $550\si{\celsius}$ at $30\si{km}$ depth, $1330\si{celsius}$ at $90\si{\km}$ depth 
and $1380\si{\celsius}$ at the bottom. Pressure is assumed to be lithostatic. 
The python program and the gnuplot script are in {\sl images/rheology/example}.

In the code I consider two cases: 'old' and 'new'. The latter is described above. 
'old' goes as follows: loop over total strain rate values. 
Compute dislocation and diffusion creep viscosities with it. Compute harmonic 
average of these with linear viscosity. Compute deviatoric stress value. 
use it in dislocation and diffusion formulae to arrive at respective strainrates. 

\newpage
\begin{center}
\includegraphics[width=5.5cm]{images/rheology/example/map_sr_df_old-1}
\includegraphics[width=5.5cm]{images/rheology/example/map_sr_df_new-1}
\includegraphics[width=5.5cm]{images/rheology/example/map_sr_df_diff-1}\\
\includegraphics[width=5.5cm]{images/rheology/example/map_sr_ds_old-1}
\includegraphics[width=5.5cm]{images/rheology/example/map_sr_ds_new-1}
\includegraphics[width=5.5cm]{images/rheology/example/map_sr_ds_diff-1}\\
\includegraphics[width=5.5cm]{images/rheology/example/map_sr_v_old-1}
\includegraphics[width=5.5cm]{images/rheology/example/map_sr_v_new-1}
\includegraphics[width=5.5cm]{images/rheology/example/map_sr_v_diff-1}\\
\includegraphics[width=5.5cm]{images/rheology/example/map_etaeff_old-1}
\includegraphics[width=5.5cm]{images/rheology/example/map_etaeff_new-1}
\includegraphics[width=5.5cm]{images/rheology/example/map_etaeff_diff-1}\\
\includegraphics[width=5.5cm]{images/rheology/example/map_tau_old-1}
\includegraphics[width=5.5cm]{images/rheology/example/map_tau_new-1}
\includegraphics[width=5.5cm]{images/rheology/example/map_tau_diff-1}\\
\includegraphics[width=5.5cm]{images/rheology/example/profile_sr-1}
\includegraphics[width=5.5cm]{images/rheology/example/profile_tau-1}
\includegraphics[width=5.5cm]{images/rheology/example/profile_etaeff-1}\\
\includegraphics[width=5.5cm]{images/rheology/example/map_isplast_old-1}
\includegraphics[width=5.5cm]{images/rheology/example/map_isplast_new-1}\\
{\captionfont Viscous branch: (ds+df+v) 'old' stands 
for the old approach when $\dot{\varepsilon}_T$
was used for all mechanisms. 'new' stands for the new approach and the right strain rate 
decomposition.}
\end{center}



\newpage
\begin{center}
\includegraphics[width=5.5cm]{images/rheology/example/map_etaeff_old_pl-1}
\includegraphics[width=5.5cm]{images/rheology/example/map_etaeff_new_pl-1}
\includegraphics[width=5.5cm]{images/rheology/example/map_etaeff_diff_pl-1}\\
\includegraphics[width=5.5cm]{images/rheology/example/map_tau_old_pl-1}
\includegraphics[width=5.5cm]{images/rheology/example/map_tau_new_pl-1}
\includegraphics[width=5.5cm]{images/rheology/example/map_tau_diff_pl-1}\\
\includegraphics[width=5.5cm]{images/rheology/example/profile_sr_pl-1}
\includegraphics[width=5.5cm]{images/rheology/example/profile_tau_pl-1}
\includegraphics[width=5.5cm]{images/rheology/example/profile_etaeff_pl-1}\\
{\captionfont Visco-viscoplastic rheology: (ds+df+v+vp)} 
\end{center}





\begin{remark}
Chenin \etal (2019) \cite{chmd19}, 
base their rheological model on the additive decomposition of the following
deviatoric strain rate tensor ${\bm \varepsilon}^d$:
\[
{\bm \varepsilon}^d =
{\bm \varepsilon}^{el}+
{\bm \varepsilon}^{pl}+
{\bm \varepsilon}^{ds}+
{\bm \varepsilon}^{df}+
{\bm \varepsilon}^{pe}
\]
where the five strain rate terms correspond respectively to the elastic, plastic, 
and viscous creep (dislocation, diffusion, peierls) contributions. 
This implies that all these elements are in series and the associated 
viscosities are then averaged with an harmonic mean. 
Rather interestingly, it is then stated that "this strain rate equation is nonlinear
and solved locally on cell centroids and vertices in order to define the current effective viscosity 
and stress \cite{poso08}."
\end{remark}

\Literature: \textcite{hoor89,lopr90,homo90,scps01,lova01,anpa19,elga10}







