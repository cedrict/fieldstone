\begin{flushright} {\tiny {\color{gray} elements1D.tex}} \end{flushright}


%------------------------------------------
\subsection{Linear basis functions ($Q_1$) \label{sec:bf1}}
\index{general}{$Q_1$}

Let $f(r)$ be a $C^1$ function on the interval $[-1:1]$ with $f(-1)=f_1$  and $f(1)=f_2$.
\begin{center}
\includegraphics[width=8cm]{images/linshapefct.png}
\end{center}
Let us assume that the function $f(r)$ is to be approximated on $[-1,1]$ by the first order polynomial 
\begin{equation}
f^h(r)=a+br \label{eqquad1}
\end{equation}
Then it must fulfil
\begin{eqnarray}
f^h(r=-1)&=&a-b =f_1 \nonumber\\
f^h(r=+1)&=&a+b =f_2 \nonumber
\end{eqnarray}
This leads to  
\begin{eqnarray}
a&=&\frac{1}{2}(f_1+f_2)  \nn\\
b&=&\frac{1}{2}(-f_1+f_2)  
\end{eqnarray}
and then replacing $a,b$ in Eq.~\eqref{eqquad1} by the above values one gets
\[
f^h(r) = \left[  \frac{1}{2}(1-r)\right] f_1 + \left[ \frac{1}{2}(1+r) \right] f_2
\]
or
\[
f^h(r)=\sum_{i=1}^2 N_i(r) f_1
\]
with
\begin{mdframed}[backgroundcolor=blue!5]
\begin{eqnarray}
N_1(r) &=& \frac{1}{2} (1-r) \nonumber\\
N_2(r) &=& \frac{1}{2} (1+r)
\end{eqnarray}
\end{mdframed}

\begin{center}
\includegraphics[width=8cm]{images/basis1D/linear.pdf}\\
{\captionfont Plot of the two linear functions $N_1(r)$ and $N_2(r)$.}
\end{center}

\newpage
%------------------------------------------
\subsection{Quadratic basis functions ($Q_2$) \label{sec:bf2}}
\index{general}{$Q_2$}

Let $f(r)$ be a $C^1$ function on the interval $[-1:1]$ with $f(-1)=f_1$, $f(0)=f_2$ and $f(1)=f_3$.
\begin{center}
\includegraphics[width=8cm]{images/quadshapefct.png}
\end{center}
Let us assume that the function $f(r)$ is to be approximated on $[-1,1]$ by the second order polynomial 
$f^h(r)$:
\begin{equation}
f(r)=a+br+cr^2 \label{eqquad}
\end{equation}
Then it must fulfil
\begin{eqnarray}
f^h(r=-1)&=&a-b+c = f_1 \nonumber\\
f^h(r=0) &=&a\quad\quad\quad\;     = f_2 \nonumber\\
f^h(r=+1)&=&a+b+c = f_3 \nonumber
\end{eqnarray}
This leads to
\begin{eqnarray}
a&=&f_2   \nn\\
b&=&\frac{1}{2}(-f_1+f_3)  \nn\\
c&=&\frac{1}{2}(f_1+f_3-2f_2) 
\end{eqnarray}
and then replacing $a,b,c$ in Eq.~\eqref{eqquad} by the above values on gets
\[
f^h(r)=\left[\frac{1}{2}r(r-1)\right] f_1 + (1-r^2) f_2 + \left[\frac{1}{2}r(r+1)\right] f_3
\]
or,
\[
\boxed{
f^h(r) = \sum_{i=1}^3 N_i(r) f_i
}
\]
with
\begin{mdframed}[backgroundcolor=blue!5]
\begin{eqnarray}
N_1(r) &=& \frac{1}{2}r(r-1) \nonumber\\
N_2(r) &=& (1-r^2) \nonumber\\ 
N_3(r) &=& \frac{1}{2}r(r+1) 
\end{eqnarray}
\end{mdframed}

\begin{center}
\includegraphics[width=8cm]{images/basis1D/quadratic.pdf}\\
{\captionfont Plot of the three quadratic functions $N_1(r)$, $N_2(r)$ and $N_3(r)$.}
\end{center}
Note that $Q_2$ basis functions can take negative values. 

We will later need the first-order derivatives of these functions:
\begin{mdframed}[backgroundcolor=blue!5]
\begin{eqnarray}
\frac{\partial N_1}{\partial r} &=& r-\frac{1}{2} \nonumber\\
\frac{\partial N_2}{\partial r} &=& -2r \nonumber\\ 
\frac{\partial N_3}{\partial r} &=& r+\frac{1}{2}
\end{eqnarray}
\end{mdframed}


%------------------------------------------
\subsection{Cubic basis functions ($Q_3$) \label{sec:bf3}}
\index{general}{$Q_3$}

We proceed as previously by assuming that the third-order 
polynomial representation of function $f(r)$ is given by
\[
f^h(r)=a+br+cr^2+dr^3
\]
with the nodes at position -1,-1/3, +1/3 and +1.
It then must fulfil all four conditions:
\begin{eqnarray}
f(-1)   &=& a-b+c-d = f_1 \nonumber\\
f(-1/3) &=& a-\frac{b}{3}+\frac{c}{9}-\frac{d}{27} = f_2 \nonumber\\
f(+1/3) &=& a-\frac{b}{3}+\frac{c}{9}-\frac{d}{27} = f_3 \nonumber\\
f(+1)   &=& a+b+c+d = f_4 \nonumber
\end{eqnarray}
Adding the first and fourth equation and the second and third, one arrives at
\[
f_1+f_4 = 2a+2c \quad\quad\quad f_2+f_3=2a+\frac{2c}{9}
\]
and finally:
\[
a=\frac{1}{16} \left( -f_1 + 9f_2 + 9f_3 - f_4  \right)
\]
\[
c=\frac{9}{16}\left(f_1-f_2-f_3+f_4\right)
\]
Combining the original 4 equations in a different way yields
\[
2b+2d=f_4-f_1 
\quad\quad\quad
\frac{2b}{3} + \frac{2d}{27} = f_3-f_2
\]
so that
\[
b=\frac{1}{16} \left( f_1 - 27f_2 + 27f_3 -f_4   \right)
\]
\[
d=\frac{9}{16} \left( -f_1 + 3f_2 - 3f_3 + f_4 \right)
\]
Finally,
\begin{eqnarray}
f^h(r) 
&=& a+b+cr^2+dr^3 \nonumber\\
&=& \frac{1}{16} (-1+  r +9r^2 - 9r^3 )f_1 \nonumber\\ 
&+& \frac{1}{16} ( 9-27r -9r^2 +27r^3 )f_2 \nonumber\\ 
&+& \frac{1}{16} ( 9+27r -9r^2 -27r^3 )f_3 \nonumber\\ 
&+& \frac{1}{16} (-1-  r +9r^2 + 9r^3 )f_4 \nonumber\\ 
&=& \sum_{i=1}^4 N_i(r) f_i \nonumber
\end{eqnarray}
where (see also for example \cite[p49]{li06})
\begin{mdframed}[backgroundcolor=blue!5]
\begin{eqnarray}
N_1&=& \frac{1}{16} (-1+  r+9r^2- 9r^3 ) \nonumber\\ 
N_2&=& \frac{1}{16} ( 9-27r-9r^2+27r^3 ) \nonumber\\ 
N_3&=& \frac{1}{16} ( 9+27r-9r^2-27r^3 ) \nonumber\\ 
N_4&=& \frac{1}{16} (-1-  r+9r^2+ 9r^3 ) \nonumber
\end{eqnarray}
\end{mdframed}

\begin{center}
\includegraphics[width=8cm]{images/basis1D/cubic.pdf}\\
{\captionfont Plot of the four cubic functions $N_1(r)$, $N_2(r)$, $N_3(r)$ and $N_4(r)$.}
\end{center}

Let us now verify that these functions can represent any polynomial function up to third order:

\begin{itemize}
\item
Let us assume $f(r)=C$, then
\[
f^h(r) = \sum N_i(r) f_i = \sum_i N_i C = C \sum_i N_i  = C
\]
so that a constant function is exactly reproduced, as expected.
This is a very important property of the $N_i$ functions: They must fulfil $\sum\limits_i N_i =1$.

\item
Let us assume $f(r)= r$, then $f_1=-1$, $f_2=-1/3$, $f_3=1/3$ and $f_4=+1$. We then have
\begin{eqnarray}
f^h(r) 
&=& \sum N_i(r) f_i  \nonumber\\
&=& - N_1(r) -\frac{1}{3}N_2(r) + \frac{1}{3}N_3(r)  + N_4(r) \nonumber\\
&=& [-(-1+  r+9r^2- 9r^3 ) \nn\\
&&- \frac{1}{3} ( 9-27r-9r^2-27r^3 ) \nn\\
&&+ \frac{1}{3} ( 9+27r-9r^2+27r^3 ) \nn\\
&&+ (-1-  r+9r^2+ 9r^3 )]/16 \nonumber\\
&=& [-r +9r + 9r -r]/16  + ... 0 ... \nonumber\\
&=& r   
\end{eqnarray}

\item The cases $f(r)=r^2$ and $f(r)=r^3$ are left as exercise.

\end{itemize}

The basis functions first-order derivatives are given by
\begin{mdframed}[backgroundcolor=blue!5]
\begin{eqnarray}
\frac{\partial N_1}{\partial r}&=& \frac{1}{16}  (  1 +18r - 27r^2 ) \nonumber\\ 
\frac{\partial N_2}{\partial r}&=& \frac{1}{16}  (-27 -18r + 81r^2 ) \nonumber\\ 
\frac{\partial N_3}{\partial r}&=& \frac{1}{16}  (+27 -18r - 81r^2 ) \nonumber\\ 
\frac{\partial N_4}{\partial r}&=& \frac{1}{16}  ( -1 +18r + 27r^2 ) \nonumber
\end{eqnarray}
\end{mdframed}

We can also verify that the derivatives are also properly approximated:

\begin{itemize}
\item
Let us assume $f(r)=C$, then
\begin{eqnarray}
\frac{\partial f^h}{\partial r} 
&=& \sum_i \frac{\partial N_i}{\partial r} f_i  \nonumber\\
&=&  C \sum_i \frac{\partial N_i}{\partial r}  \nonumber\\
&=& \frac{C}{16} [  (  1 +18r - 27r^2 ) 
+ (-27 -18r + 81r^2 )  
+  (+27 -18r - 81r^2 ) 
+ ( -1 +18r + 27r^2 ) ]  \nonumber\\
&=& 0 \nonumber
\end{eqnarray}

\item
Let us assume $f(r)= r$, then $f_1=-1$, $f_2=-1/3$, $f_3=1/3$ and $f_4=+1$. We then have
\begin{eqnarray}
\frac{\partial f^h}{\partial r} 
&=& \sum_i \frac{\partial N_i}{\partial r} f_i  \nonumber\\
&=& \frac{1}{16} [  -(  1 +18r - 27r^2 ) 
 -\frac{1}{3} (-27 -18r + 81r^2 )  
 +\frac{1}{3} (27 -18r - 81r^2 )
 + ( -1 +18r + 27r^2 ) ]  \nonumber\\
&=& \frac{1}{16} [-2 + 18 + 54r^2 - 54r^2] \nonumber\\
&=& 1 \nonumber
\end{eqnarray}

\item
Let us assume $f(r)= r^2$, then $f_1=1$, $f_2=1/9$, $f_3=1/9$ and $f_4=1$. We then have
\begin{eqnarray}
\frac{\partial f^h}{\partial r} 
&=& \sum_i \frac{\partial N_i}{\partial r} f_i  \nonumber\\
&=& \frac{1}{16} \left[  
(  1 +18r - 27r^2 ) 
+\frac19 (-27 -18r + 81r^2 )  
+\frac19  (27 -18r - 81r^2 )
+ ( -1 +18r + 27r^2 ) \right]  \nonumber\\
&=& \frac{1}{16}(32r) \nn\\
&=& 2r
\end{eqnarray}
as expected.




\end{itemize}

%---------------------------------------------------------------
\subsection{Quartic basis functions ($Q_4$) \label{sec:bf4}}
\index{general}{$Q_4$}

The 1D basis polynomial is given by
\[
f_h(r)=a+br+cr^2+dr^3+er^4
\]
with the nodes at position -1,-1/2, 0, +1/2 and +1.
The function $f^h(r)$ must then fulfil 
\begin{eqnarray}
f_h(-1)   &=& a-b+c-d+e = f_1 \nonumber\\
f_h(-1/2) &=& a-\frac{b}{2}+\frac{c}{4}-\frac{d}{8}+\frac{e}{16} = f_2 \nonumber\\
f_h(0)    &=& a = f_3 \nonumber\\
f_h(+1/2) &=& a-\frac{b}{2}+\frac{c}{4}-\frac{d}{8}+\frac{e}{16} = f_4 \nonumber\\
f_h(+1)   &=& a+b+c+d+e = f_5 \nonumber
\end{eqnarray}
or, 
\begin{equation}
\left(
\begin{array}{ccccc}
 1  &  -1  &  1 &  -1 &  1 \\ 
 1  &  -1/2  &  1/4 &  -1/8 &  1/16 \\ 
 1  &   0    &  0   &   0   & 0 \\
 1  &  1/2  &  1/4 &  1/8 &  1/16 \\ 
 1  &  1  &  1 &  1 &  1 
\end{array}
\right)
\left(
\begin{array}{c}
a \\ b \\ c \\ d \\ e
\end{array}
\right)
=
\left(
\begin{array}{c}
f_1 \\ f_2 \\ f_3 \\ f_4 \\ f_5
\end{array}
\right)
\end{equation}
The third line gives $a=f_3$ so that
\begin{equation}
\underbrace{
\left(
\begin{array}{ccccc}
-1  &  1   &  -1 &  1 \\ 
-1/2 &  1/4 & -1/8 &  1/16 \\ 
 1/2 &  1/4 &  1/8 &  1/16 \\ 
 1  &  1   &  1 &  1 
\end{array}
\right)}_{A}
\left(
\begin{array}{c}
b \\ c  \\ d \\ e
\end{array}
\right)
=
\left(
\begin{array}{c}
f_1 -f_3 \\ f_2 -f_3\\ f_4-f_3 \\ f_5 -f_3
\end{array}
\right)
\end{equation}
The inverse of the matrix $A$ is:
\[
A^{-1}=
\frac{1}{6}
\left(
\begin{array}{ccccc}
1 & -8 & 8 & -1 \\
-1 & 16 & 16 & -1 \\
-4 & 8 & -8 & 4 \\
4 & -16 & -16 & 4
\end{array}
\right)
\]
so that 
\[
\left(
\begin{array}{c}
b \\ c \\ d \\ e
\end{array}
\right)
=
\frac{1}{6}
\left(
\begin{array}{ccccc}
1 & -8 & 8 & -1 \\
-1 & 16 & 16 & -1 \\
-4 & 8 & -8 & 4 \\
4 & -16 & -16 & 4
\end{array}
\right)
\cdot
\left(
\begin{array}{c}
f_1 -f_3 \\ f_2 -f_3\\ f_4-f_3 \\ f_5 -f_3
\end{array}
\right)
\]
and then 
\begin{eqnarray}
b &=& \frac{1}{6} \left( f_1 -8f_2 +8 f_4 -f_5     \right) \\
c &=& \frac{1}{6} \left( -f_1 +16f_2 -30f_3    + 16f_4- f_5   \right) \\
d &=& \frac{1}{6} \left( -4f_1 +8f_2     -8f_4+ 4 f_5   \right) \\
e &=& \frac{1}{6} \left( 4f_1 -16f_2 +24f_3 -16f_4+ 4 f_5   \right) 
\end{eqnarray}
Finally
\begin{eqnarray}
f_h(r) 
&=& a+br+cr^2+dr^3+er^4 \nn\\
&=& f_3 + 
\frac{1}{6} \left( f_1 -8f_2 +8 f_4 -f_5     \right)  r  +
\frac{1}{6} \left( -f_1 +16f_2 -30f_3    + 16f_4- f_5   \right) r^2 + \nn\\ &&
\frac{1}{6} \left( -4f_1 +8f_2     -8f_4+ 4 f_5   \right) r^3 +
\frac{1}{6} \left( 4f_1 -16f_2 +24f_3 -16f_4+ 4 f_5   \right) r^4 \nn\\
&=& \frac{1}{6} \left(  r- r^2 -4r^3 +4r^4\right) f_1 \nn\\
&+& \frac{1}{6} \left(  -8r+16 r^2 +8r^3 -16 r^4\right) f_2 \nn\\
&+& \left( 1 -5r^2+4r^4  \right) f_3 \nn\\
&+& \frac{1}{6} \left(  8r+16 r^2 -8r^3 -16 r^4\right) f_4 \nn\\
&+& \frac{1}{6} \left(  -r- r^2 +4r^3 +4r^4\right) f_5 \nn
\end{eqnarray}
with 
\begin{mdframed}[backgroundcolor=blue!5]
\begin{eqnarray}
N_1(r)&=& \frac{1}{6} \left(  r- r^2 -4r^3 +4r^4\right) \nn\\
N_2(r)&=& \frac{1}{6} \left(  -8r+16 r^2 +8r^3 -16 r^4\right)  \nn\\
N_3(r)&=& \left( 1 -5r^2+4r^4  \right) \nn \\
N_4(r)&=& \frac{1}{6} \left(  8r+16 r^2 -8r^3 -16 r^4\right)  \nn\\
N_5(r)&=& \frac{1}{6} \left(  -r- r^2 +4r^3 +4r^4\right) 
\end{eqnarray}
\end{mdframed}

\begin{center}
\includegraphics[width=8cm]{images/basis1D/quartic.pdf}\\
{\captionfont Plot of the 5 quartic basis functions.}
\end{center}

The basis functions derivative are given by
\begin{mdframed}[backgroundcolor=blue!5]
\begin{eqnarray}
\frac{\partial N_1}{\partial r}&=& \frac{1}{6}(1-2r-12r^2+16r^3) \nn\\
\frac{\partial N_2}{\partial r}&=& \frac{1}{6}(-8+32r+24r^2-64r^3) \nn\\
\frac{\partial N_3}{\partial r}&=& -10r+16r^3 \nn\\
\frac{\partial N_4}{\partial r}&=& \frac{1}{6} (8+32r-24r^2-64r^3) \nn\\
\frac{\partial N_5}{\partial r}&=& \frac{1}{6} (-1-2r+12r^2+16r^3) 
\end{eqnarray}
\end{mdframed}


%---------------------------------------------------------------
\subsection{Fifth-order basis functions ($Q_5$) \label{sec:bf5}}
\index{general}{$Q_5$}

Following the methodology presented hereafter for $Q_6$, we arrive at 

\begin{eqnarray}
\bN_1(r) &=& -\frac{625}{768}(r+\frac35)(r+\frac15)(r-\frac15)(r-\frac35)(r-1) \nn\\
\bN_2(r) &=&  \frac{3125}{768}(r+1)(r+\frac15)(r-\frac15)(r-\frac35)(r-1) \nn\\
\bN_3(r) &=& -\frac{3125}{384}(r+1)(r+\frac35)(r-\frac15)(r-\frac35)(r-1) \nn\\
\bN_4(r) &=&  \frac{3125}{384}(r+1)(r+\frac35)(r+\frac15)(r-\frac35)(r-1) \nn\\
\bN_5(r) &=& -\frac{3125}{768}(r+1)(r+\frac35)(r+\frac15)(r-\frac15)(r-1) \nn\\
\bN_6(r) &=&  \frac{625}{768}(r+1)(r+\frac35)(r+\frac15)(r-\frac15)(r-\frac35) 
\end{eqnarray}

or, 

\begin{mdframed}[backgroundcolor=blue!5]
\begin{eqnarray}
\bN_1(r) &=& -\frac{1}{768} (625r^5-625r^4-250r^3+250r^2+9r-9) \nn\\
\bN_2(r) &=& \frac{25}{768} (125r^5-75r^4-130r^3+78r^2+5r-3) \nn\\
\bN_3(r) &=& -\frac{25}{384} (125r^5-25r^4-170r^3+34r^2+45r-9) \nn\\
\bN_4(r) &=& \frac{25}{384} (125r^5+25r^4-170r^3-34r^2+45r+9) \nn\\
\bN_5(r) &=& -\frac{25}{768} (125r^5+75r^4-130r^3-78r^2+5r+3) \nn\\
\bN_6(r) &=& \frac{1}{768} (625r^5+625r^4-250r^3-250r^2+9r+9) 
\end{eqnarray}
\end{mdframed}

with the derivatives given by

\begin{mdframed}[backgroundcolor=blue!5]
\begin{eqnarray}
\frac{\partial N_1}{\partial r}&=& -\frac{1}{768}(3125r^4-2500r^3-750r^2+500r+9 ) \nn\\ 
\frac{\partial N_2}{\partial r}&=& \frac{25}{768}(625r^4-300r^3-390r^2+156r+5  ) \nn\\ 
\frac{\partial N_3}{\partial r}&=& -\frac{25}{384}(625r^4-100r^3-510r^2+68r+45  ) \nn\\ 
\frac{\partial N_4}{\partial r}&=& \frac{25}{384}(625r^4+100r^3-510r^2-68r+45  ) \nn\\ 
\frac{\partial N_5}{\partial r}&=& -\frac{25}{768}(625r^4+300r^3-390r^2-156r+5  ) \nn\\ 
\frac{\partial N_6}{\partial r}&=& \frac{1}{768}(3125r^4+2500r^3-750r^2-500r+9 ) 
\end{eqnarray}
\end{mdframed}




\begin{center}
\includegraphics[width=11cm]{images/basis1D/Q5.pdf}\\
{\captionfont Plot of the 6 fifth-order basis functions.}
\end{center}


These functions are used in \stone~\ref{f152}.

%---------------------------------------------------------------
\subsection{Sixth-order basis functions ($Q_6$) \label{sec:bf6}}
\index{general}{$Q_6$}

The 1D basis polynomial is given by
\[
f_h(r)=a+br+cr^2+dr^3+er^4+fr^5+gr^6
\]
with the nodes at position -1,-2/3, -1/3, 0, +1/3, +2/3 and +1.
The function $f^h(r)$ must then fulfil 
%\begin{eqnarray}
%f_h(-1)   &=&  a -        b +         c -            d +e -f +g =f_1 \nn\\
%f_h(-2/3) &=&  a -\frac23 b + \frac49 c -\frac{8}{27}d +e -f +g =f_2\nn\\
%f_h(-1/3) &=&  a -\frac13 b + \frac19 c -\frac{1}{27}d +e -f +g =f_3\nn\\
%f_h(0)    &=&  a                                                = f_4 \nn\\
%f_h(+1/3) &=&  a +\frac13 b + \frac19 c +\frac{1}{27}d +e +f +g =f_5\nn\\
%f_h(+2/3) &=&  a +\frac23 b + \frac49 c +\frac{8}{27}d +e +f =g =f_6\nn\\
%f_h(+1)   &=&  a +        b +         c +            d +e +f +g =f_7 \nn
%\end{eqnarray}


\[
\left(
\begin{array}{ccccccc}
1& -1       & 1       & -1           & 1 & -1 & 1  \\
1& -\frac23 & \frac49 & -\frac{8}{27}& \frac{16}{81} & -\frac{32}{243} & \frac{64}{729}  \\
1& -\frac13 & \frac19 & -\frac{1}{27}& \frac{1}{81}  & -\frac{1}{243}  & \frac{1}{729}  \\
1& 0        & 0       & 0            &0 & 0 & 0 \\
1& \frac13  & \frac19 & \frac{1}{27} & \frac{1}{81}  & \frac{1}{243}  & \frac{1}{729}  \\
1& \frac23  & \frac49 & \frac{8}{27} & \frac{16}{81} & \frac{32}{243} & \frac{64}{729}  \\
1& 1        & 1       & 1            & 1 & 1 & 1  
\end{array}
\right)
\cdot
\left(
\begin{array}{c}
a \\ b \\ c \\ d \\ e \\ f \\ g
\end{array}
\right)
=
\left(
\begin{array}{c}
f_1 \\ f_2 \\ f_3 \\ f_4 \\ f_5 \\ f_6 \\ f_7
\end{array}
\right)
\]


The middle line yields $a=f_4$, so that we have:

\[
\left(
\begin{array}{cccccc}
 -1       & 1       & -1           & 1 & -1 & 1  \\
 -\frac23 & \frac49 & -\frac{8}{27}& \frac{16}{81} & -\frac{32}{243} & \frac{64}{729}  \\
 -\frac13 & \frac19 & -\frac{1}{27}& \frac{1}{81}  & -\frac{1}{243}  & \frac{1}{729}  \\
 \frac13  & \frac19 & \frac{1}{27} & \frac{1}{81}  & \frac{1}{243}  & \frac{1}{729}  \\
 \frac23  & \frac49 & \frac{8}{27} & \frac{16}{81} & \frac{32}{243} & \frac{64}{729}  \\
 1        & 1       & 1            & 1 & 1 & 1  
\end{array}
\right)
\cdot
\left(
\begin{array}{c}
b \\ c \\ d \\ e \\ f \\ g
\end{array}
\right)
=
\left(
\begin{array}{c}
f_1 -f_4 \\ f_2 -f_4\\ f_3 -f_4 \\ f_5 -f_4\\ f_6 -f_4\\ f_7-f_4
\end{array}
\right)
\]
Multiplying all lines by 729, we obtain:

\[
\frac{1}{729}
\left(
\begin{array}{cccccc}
 -729 & 729 & -729 & 729 & -729 & 729  \\
 -486 & 324 & -216 & 144 & -96  & 64  \\
 -243 & 81  & -27  & 9   & -3   & 1  \\
 243  & 81  & 27   & 9   & 3    & 1  \\
 486  & 324 & 216  & 144 & 96   & 64  \\
729   & 729 & 729  & 729 & 729  & 729  
\end{array}
\right)
\cdot
\left(
\begin{array}{c}
b \\ c \\ d \\ e \\ f \\ g
\end{array}
\right)
=
\left(
\begin{array}{c}
f_1 -f_4 \\ f_2 -f_4\\ f_3 -f_4 \\ f_5 -f_4\\ f_6 -f_4\\ f_7-f_4
\end{array}
\right)
\]

The inverse\footnote{\url{https://physandmathsolutions.com/Matrices/matrix_inverse/matrix_inverse_6x6.php}} of this matrix is:

\begin{verbatim}
-0.00006859 	0.00061728 	-0.00308642 	0.00308642 	-0.00061728 	0.00006859
0.00006859 	-0.00092593 	0.00925926 	0.00925926 	-0.00092593 	0.00006859
0.00077160 	-0.00617284 	0.01003086 	-0.01003086 	0.00617284 	-0.00077160
-0.00077160 	0.00925926 	-0.03009259 	-0.03009259 	0.00925926 	-0.00077160
-0.00138889 	0.00555556 	-0.00694444 	0.00694444 	-0.00555556 	0.00138889
0.00138889 	-0.00833333 	0.02083333 	0.02083333 	-0.00833333 	0.00138889
\end{verbatim}

Obviously, this is not a very practical approach anymore. One could 
solve the system by hand, making sure to keep fractions but it will be 
cumbersome. Let us turn to another approach.

The nodes inside the reference element are as follows:

\begin{verbatim}
(1) (2)  (3) (4) (5)  (6) (7)
-|---|----|---+---|----|---|-
-1 -2/3 -1/3  0  1/3  2/3  1
\end{verbatim}

Basis function $\bN_1(r)$ is a 6th order polynomial expression 
that should be 1 at node 1, and 0 at others,
i.e. at $r=-2/3,-1/3,0,1/3,2/3,1$. It must then be of the form:
\[
\bN_1(r) = \alpha(r+\frac23)(r+\frac13)(r)(r-\frac13)(r-\frac23)(r-1)
\]
When evaluated at $r=-1$, we get
\[
\bN_1(r=-1) 
= \alpha(-\frac13)(-\frac23)(-1)(-\frac43)(-\frac53)(-2)
= \alpha\frac{80}{81}
\]
Since this quantity must be 1, we have
\[
1 = \alpha \frac{80}{81}
\quad
\rightarrow
\quad
\alpha=\frac{81}{80}
\]
so that 
\begin{eqnarray}
\bN_1(r)
&=& \frac{81}{80}(r+\frac23)(r+\frac13)(r)(r-\frac13)(r-\frac23)(r-1) \nn\\
&=& \frac{81}{80} \frac{1}{81} (3r+2)(3r+1)(r)(3r-1)(3r-2)(r-1) \nn\\
&=& \frac{1}{80} (9r^2-4)(9r^2-1)(r^2-r) \nn\\
&=& \frac{1}{80} (81r^4 -45r^2 +4)(r^2-r) 
\end{eqnarray}

Moving to $\bN_2(r)$, we have
\[
\bN_2(r)= \alpha(r+1)(r+\frac13)(r)(r-\frac13)(r-\frac23)(r-1)
\]
which must be equal to 1 for $r=-2/3$:
\begin{eqnarray}
\bN_2(r=-2/3)
&=& \alpha(-\frac23+1)(-\frac23+\frac13)(-\frac23)(-\frac23-\frac13)(-\frac23-\frac23)(-\frac23-1)\nn\\
&=& \alpha (\frac13)(-\frac13)(-\frac23)(-1)(-\frac43)(-\frac53)\nn\\
&=& -\alpha \frac{40}{243}
\end{eqnarray}
so that 
\[
\bN_2(r)= -\frac{243}{40}(r+1)(r+\frac13)(r)(r-\frac13)(r-\frac23)(r-1)
\]

Moving to $\bN_3(r)$, we have
\[
\bN_3(r)= \alpha(r+1)(r+\frac23)(r)(r-\frac13)(r-\frac23)(r-1)
\]
which must be equal to 1 for $r=-1/3$:
\begin{eqnarray}
\bN_3(r=-1/3)
&=& \alpha(-\frac13+1)(-\frac13+\frac23)(-\frac13)(-\frac13-\frac13)(-\frac13-\frac23)(-\frac13-1) \nn\\
&=& \alpha(\frac23)(\frac13)(-\frac13)(-\frac23)(-1)(-\frac43) \nn\\
&=& \alpha \frac{16}{243}
\end{eqnarray}
so that 
\[
\bN_3(r)= \frac{243}{16}(r+1)(r+\frac23)(r)(r-\frac13)(r-\frac23)(r-1)
\]
Likewise, we arrive at the rest of the basis functions. In the end:

\begin{eqnarray}
\bN_1(r) &=& \frac{81}{80}(r+\frac23)(r+\frac13)(r)(r-\frac13)(r-\frac23)(r-1) \nn\\
\bN_2(r) &=& -\frac{243}{40}(r+1)(r+\frac13)(r)(r-\frac13)(r-\frac23)(r-1) \nn\\
\bN_3(r) &=& \frac{243}{16}(r+1)(r+\frac23)(r)(r-\frac13)(r-\frac23)(r-1) \nn\\
\bN_4(r) &=& -\frac{81}{4}(r+1)(r+\frac23)(r+\frac13)(r-\frac13)(r-\frac23)(r-1) \nn\\
\bN_5(r) &=& \frac{243}{16}(r+1)(r+\frac23)(r+\frac13)(r)(r-\frac23)(r-1) \nn\\
\bN_6(r) &=& -\frac{243}{40}(r+1)(r+\frac23)(r+\frac13)(r)(r-\frac13)(r-1) \nn\\
\bN_7(r) &=& \frac{81}{80}(r+1)(r+\frac23)(r+\frac13)(r)(r-\frac13)(r-\frac23)
\end{eqnarray}
or 

\begin{mdframed}[backgroundcolor=blue!5]
\begin{eqnarray}
\bN_1(r) &=& \frac{1}{80}(81r^6-81r^5-45r^4+45r^3+4r^2-4r) \nn\\
\bN_2(r) &=& -\frac{9}{40}(27r^6 -18r^5 -30r^4 +20r^3 +3r^2 -2r) \nn\\ 
\bN_3(r) &=& \frac{9}{16} (27r^6 -9r^5 -39r^4 +13r^3 +12r^2 -4r) \nn\\ 
\bN_4(r) &=& -\frac{1}{4} (81r^6 - 126r^4+49r^2 -4) \nn\\ 
\bN_5(r) &=& \frac{9}{16} (27r^6 +9r^5 -39r^4 -13r^3 +12r^2 +4r) \nn\\ 
\bN_6(r) &=& -\frac{9}{40}(27r^6 +18r^5 -30r^4 -20r^3 +3r^2 +2r) \nn\\ 
\bN_7(r) &=&  \frac{1}{80}(81r^6+81r^5-45r^4-45r^3+4r^2+4r) 
\end{eqnarray}
\end{mdframed}

\begin{center}
\includegraphics[width=11cm]{images/basis1D/Q6.pdf}\\
{\captionfont Plot of the 7 six-order basis functions.}
\end{center}

Using WolframAlpha\footnote{\url{https://www.wolframalpha.com/}}, we arrive at 

\begin{eqnarray}
\frac{d \bN_1}{dr} &=& \frac{1}{80} (486r^5 - 405r^4 - 180r^3 + 135 r^2 +8r-4) \nn\\
\frac{d \bN_2}{dr} &=& -\frac{9}{20} (81r^5 - 45r^4 -60r^3 + 30r^2 +3r -1) \nn\\ 
\frac{d \bN_3}{dr} &=& \frac{9}{16} (162r^5-45r^4 -156r^3 +39r^2 +24r -4) \nn\\ 
\frac{d \bN_4}{dr} &=& \frac{1}{2} (-243r^5+252r^3-49r) \nn\\
\frac{d \bN_5}{dr} &=& \frac{9}{16} (162r^5 +45r^4 -156r^3 -39r^2 +24r +4) \nn\\ 
\frac{d \bN_6}{dr} &=& -\frac{9}{20} (81r^5 +45r^4 -60r^3 - 30r^2 +3r +1) \nn\\ 
\frac{d \bN_7}{dr} &=& \frac{1}{80} (486r^5 + 405r^4 - 180r^3 -135r^2 +8r+4) 
\end{eqnarray}

These functions are used in \stone~\ref{f152}.





%---------------------------------------------------------------
\subsection{A generic approach to 1D basis functions \label{sec:bfgeneric}}

In order to define basis functions of order $n$ each element 
must have $n+1$ nodes.
The $i-th$ basis function for an $n-th$ order approximation
is given by:
\[
\bN_i(r) = 
\frac{
\prod_{j=0,j \ne i}^n (r-r_j)
}
{
\prod_{j=0,j \ne i}^n (r_i-r_j)
}
\]

Let us see in practice how this works and start with $n=2$ (i.e. $Q_2$ basis functions).
In the reference element we have $r_0=-1$, $r_1=0$ and $r_2=+1$, so that 
\begin{eqnarray}
\bN_0 
&=& \frac{(r-r_1)(r-r_2)}{(r_0-r_1)(r_0-r_2)} \nn\\
&=& \frac{(r-0)(r-1)}{(-1-0)(-1-1)} \nn\\
&=& \frac12 r(r-1) \nn\\
\bN_1
&=& \frac{(r-r_0)(r-r_2)}{(r_1-r_0)(r_1-r_2)} \nn\\
&=& \frac{(r+1)(r-1)}{(0+1)(0-1)} \nn\\
&=& 1-r^2 \nn\\
\bN_2 
&=& \frac{(r-r_0)(r-r_1)}{(r_2-r_0)(r_2-r_1)} \nn\\
&=& \frac{(r+1)(r-0)}{(1+1)(1-0)} \nn\\
&=& \frac12 r(r+1) 
\end{eqnarray}
These are the basis functions obtained in Section~\ref{sec:bf2}.

\todo[inline]{What about derivatives?}












\newpage

Remark:

According to Guermond\footnote{\url{https://people.tamu.edu/~guermond/M661_FALL_2015/chap2.pdf}}:
``While the choice of equidistant nodes appears somewhat natural, it is appro-
priate only when working with low-degree polynomials. The main difficulty
comes from the oscillatory nature of the Lagrange polynomials as the num-
ber of interpolation nodes grows. This phenomenon is often referred to as the
Runge phenomenon [359] (see also Meray [313]''
