\begin{flushright} {\tiny {\color{gray} mms\_sofo87.tex}} \end{flushright}
%~~~~~~~~~~~~~~~~~~~~~~~~~~~~~~~~~~~~~~~~~~~~~~~~~~~~~~~~~~~~~~~~~~~~~~~~~~~~~~~~~~~~~~~~~~~~~~~~~~

This is presented in \textcite{sofo87} (1987). The velocity field is given by
\[
\vec\upnu(x,y) = (x^3,-3x^2y) 
\]
and the pressure is 
\[
p(x,y)=x^3+y^3-1/2
\]
so that, assuming that the viscosity is 1, the body force is:
\[
\vec{b} = (-6x+3x^2,6y+3y^2)
\] 
Note that I have added the $-1/2$ term to the pressure so that $\int\int p dxdy=0$.
The root mean square velocity over a unit square is 
\[
\upnu_{rms} 
= \sqrt{ \int_0^1\int_0^1 (u^2+v^2) dx dy }
= \sqrt{ \int_0^1\int_0^1 (x^6 + 9 x^4 y^2) dx dy }
= \sqrt{ \frac{1}{7} + 9 \frac{1}{5} \frac{1}{3}  } 
= \sqrt{ \frac{26}{35} }
\simeq 0.861892 
\]
The strain rate tensor terms are
\begin{eqnarray}
\dot{\varepsilon}_{xx} &=& 3x^2  \nonumber\\
\dot{\varepsilon}_{yy} &=& -3x^2 \nonumber\\ 
\dot{\varepsilon}_{xy} &=& -3xy  \nn
\end{eqnarray}

Another one mentioned in the paper:
\[
\vec\upnu(x,y) =(x^2,-2xy)
\qquad
p(x,y)=0
\qquad
\vec{b}=(-2,0) 
\]


