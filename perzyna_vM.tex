\begin{flushright} {\tiny {\color{gray} perzyna\_vM.tex}} \end{flushright}
%~~~~~~~~~~~~~~~~~~~~~~~~~~~~~~~~~~~~~~~~~~~~~~~~~~~~~~~~~~~~~~~~~~~~~~~~~~~~~~~~~~~~~~~~~~~~~~~~~~

\subsection{von Mises plasticity following Zienkiewicz (1975)}

What follows is borrowed from Zienkiewicz (1975) \cite{zien75}.

\begin{center}
\fbox{\includegraphics[width=6cm]{images/perzyna/zien75}}\\
{\captionfont Taken from \textcite{zien75}.}
\end{center}

We start from section 13.4.2 of the paper with the Perzyna formulation of the plastic 
strain\footnote{from which I have removed the unnecessary/uncommon $\sqrt 3$ terms}. 
\[
\dot{\bm \varepsilon}^{vp} = \gamma \langle \phi({\FFF}) \rangle \frac{\partial \QQQ}{\partial \bm\sigma}
\]
Associative plasticity is used, i.e. ${\FFF}^{\text{\tiny vM}}={\cal \QQQ}^{\text{\tiny vM}}$, and the von Mises yield
criterion is ${\FFF}^{\text{\tiny vM}}=\sqrt{{\III}_2(\bm \tau)}- Y$ so
\begin{eqnarray}
\dot{\bm \varepsilon}^{vp} 
&=& \gamma \Big\langle \phi\left(\sqrt{{\III}_2(\bm \tau)}- Y\right)  \Big\rangle 
\frac{\partial (\sqrt{{\III}_2(\bm \tau)}- Y)}{\partial \bm\sigma} \nn\\
&=& \gamma \big\langle \phi\left(\sqrt{{\III}_2(\bm \tau)}- Y\right)  \Big\rangle 
\frac{\partial \sqrt{{\III}_2(\bm \tau)}}{\partial \bm\sigma} \nn\\
&=& \gamma \Big\langle \phi\left(\sqrt{{\III}_2(\bm \tau)}- Y\right)  \Big\rangle 
\frac{1}{2 \sqrt{{\III}_2(\bm \tau)}} \frac{\partial {\III}_2(\bm \tau)}{\partial \bm\sigma}
\end{eqnarray}
Using results of Section~\ref{ss:recapInv} for the partial derivative of the second invariant
we find\footnote{This is the same equation as Eq.~(14) of 
\textcite{zijo78}}:
\begin{eqnarray}
\dot{\bm \varepsilon}^{vp} 
&=& \gamma \Big\langle \phi\left(\sqrt{{\III}_2(\bm \tau)}- Y \right)  \Big\rangle 
\frac{1}{2 \sqrt{{\III}_2(\bm \tau)}} {\bm \tau}
\end{eqnarray}
which we can also write\footnote{There most likely is a confusion in 
the paper between $\sigma$ and $\tau$ there.}
\[
\dot{\bm \varepsilon}^{vp} =
\frac{1}{2\eta} \bm\tau
\qquad
{\rm with}
\qquad
\frac{1}{2\eta} = \gamma \Big\langle \phi(\sqrt{{\III}_2(\bm \tau)}- Y) \Big\rangle 
\frac{1}{2 \sqrt{{\III}_2(\bm \tau)}}
\]
Note that it here follows that the flow is incompressible since 
the visco-plastic strain rate tensor is proportional to the deviatoric stress tensor
so it is deviatoric itself!

%Interestingly the author then states:
%``As mentioned before this form is convenient for solving problems with
%prescribed tractions. We now seek a form of $\bm\Gamma$ which will 
%be applicable for prescribed velocity problems.'' Not sure what to make of it, though.

From the definition of the second moment invariant:
\[
{\III}_2(\bm\tau)
=\frac12 \bm\tau:\bm\tau 
=\frac12 (2\eta \dot{\bm \varepsilon}^{vp}):(2\eta \dot{\bm \varepsilon}^{vp})
=4 \eta^2   \left(\frac12 \dot{\bm \varepsilon}^{vp}: \dot{\bm \varepsilon}^{vp}\right)
=4\eta^2 {\III}_2  \dot{\bm \varepsilon}^{vp}_e
\]
from which $\eta$ can be found as a function of strain rates and 
hence $\bm\Gamma(\dot{\bm\varepsilon})$ becomes available (WHAT IS GAMMA?!).
Note that annoyingly the author defines the second invariant as 
$2 \dot{\bm\varepsilon}:\dot{\bm\varepsilon}$ in Eq.~(13.50) of the paper.

It follows that 
\[
\tau_e =\sqrt{{\III}_2(\bm\tau)}=2\eta \dot{\varepsilon}_e^{vp}
\]

Then we drop the $\langle \cdot \rangle$ as we assume to be above yield and 
we also assume a power-law form $\phi(\FFF)=\FFF^n$ so that 
we can solve explicitly for $\eta$:

\begin{eqnarray}
\frac{1}{2\eta} 
&=& \gamma \Big\langle \phi(\sqrt{{\III}_2(\bm \tau)}- Y)  \big\rangle 
\frac{1}{2 \sqrt{{\III}_2(\bm \tau)}} \nn\\
\frac{1}{2\eta} 
&=& \gamma \left(2\eta \sqrt{{\III}_2(\dot{\bm\varepsilon})}- Y\right)^n  \frac{1}{2 \; 2 \eta \sqrt{{\III}_2(\dot{\bm\varepsilon})}} \nn\\
\frac{1}{2\eta} 
&=& \gamma \left(2\eta \dot\varepsilon_e - Y\right)^n  \frac{1}{2 \; 2 \eta \dot\varepsilon_e } \nn\\
2 \dot\varepsilon_e
&=& \gamma (2\eta \dot\varepsilon_e - Y)^n  \nn\\
2 \dot\varepsilon_e / \gamma
&=&  (2\eta \dot\varepsilon_e - Y)^n  \nn\\
(2 \dot\varepsilon_e / \gamma)^{1/n}
&=&  2\eta \dot\varepsilon_e - Y \nn\\
\eta &=& \frac{ Y + (2 \dot\varepsilon_e / \gamma)^{1/n}}{2  \dot\varepsilon_e}
\end{eqnarray}
This form is convenient for plastic, visco-plastic and creep phenomena\footnote{Down 
to various $\sqrt{2}$ or $\sqrt{3}$ coefficients here or there, it 
is also to be found in Vilotte etal \cite{vidm82,vidm84,vimd86}}.
This can be re-written
\begin{eqnarray}
\eta 
&=& \frac{ Y + (2 \dot\varepsilon_e / \gamma)^{1/n}}{2  \dot\varepsilon_e} \nn\\
&=& \frac{ Y }{2  \dot\varepsilon_e}
+ \frac{ (2 \dot\varepsilon_e / \gamma)^{1/n}}{2  \dot\varepsilon_e} \nn\\
&=& \frac{ Y }{2  \dot\varepsilon_e}
+ \frac{ (2  / \gamma)^{1/n}}{2 }   \dot\varepsilon_e^{\frac1n-1} \nn\\
&=& \frac{ Y }{2  \dot\varepsilon_e}
+ \frac{1 }{2 } (\gamma/2)^{-1/n}  \dot\varepsilon_e^{\frac1n-1} 
\label{eq:zien75:a}
\end{eqnarray}
We often use dislocation creep/power-law rheologies and these yield an effective 
viscosity $\frac12 A^{-1/n} \dot\varepsilon_e^{\frac1n-1}$ (the temperature and/or 
pressure-dependent exponential has been omitted for simplicity - it is a power law rheology).
The equation above is then the sum of the 'plastic viscosity' and the 'viscous creep viscosity' 
-- which corresponds to a dashpot and a plastic element in parallel.
Note that the expression above is very similar to the one for Bingham or Herschel-Bulkley visco-plastic
models.

If $n=1$ then we find (as in \textcite{vidm82} (1982))
\begin{eqnarray}
\eta 
&=& \frac{ Y }{2  \dot\varepsilon_e}
+ \frac{1 }{\gamma } 
\end{eqnarray}
and $\gamma$ is then the inverse of the (linear) viscosity of the dashpot.
Also if $n=1$ then $\bar\eta=\gamma^{-1}$.

For pure plasticity then $\gamma \rightarrow \infty$ and we have here simply
\[
\eta = \frac{Y}{2  \dot\varepsilon_e}
\]
As stated in Vilotte etal (1982) \cite{vidm82}: 
\begin{displayquote}
{\color{darkgray}
The plastic flow law of Eq.~\eqref{eq:zien75:a} permits us to
represent in a single expression both the rigid-perfectly plastic flow
($\gamma\rightarrow \infty$)
and the common power creep law without plastic limit ($Y=0$)
(usually referred to as the Norton-Hoff law)\footnote{\url{https://en.wikipedia.org/wiki/Viscoplasticity}}.}
\end{displayquote}
\textcite{vidm82} then explain that the fluidity $\gamma$ can depend on temperature $T$ in the form 
\[
\gamma = \gamma_0 \exp(-Q/RT)
\]

In this case Eq.~\eqref{eq:zien75:a} becomes
\begin{eqnarray}
\eta 
&=& \frac{ Y }{2  \dot\varepsilon_e} + \frac{1 }{2 } ( \gamma_0 \exp(-Q/RT) /2)^{-1/n}  \dot\varepsilon_e^{\frac1n-1}  \nn\\
&=& \frac{ Y }{2  \dot\varepsilon_e} + \frac{1 }{2 } ( \gamma_0 /2)^{-1/n}  \dot\varepsilon_e^{\frac1n-1}  \exp(Q/nRT) \nn\\
&=& \frac{ Y }{2  \dot\varepsilon_e} + \frac{1 }{2 } A^{-1/n}  \dot\varepsilon_e^{\frac1n-1}  \exp(Q/nRT) 
\end{eqnarray}
where we have defined $A=\gamma_0/2$,
which is the sum of the 'plastic viscosity' and a 'dislocation creep viscosity'.

In conclusion we find that this formulation allows us to represent linear, power law,
perfectly plastic and visco-plastic materials.
Also, we know that the additional term $\bar\eta=\gamma^{-1}$ introduces a length scale
in the shear bands by limiting the viscosity value in said shear bands.

%%--------------------------------------------------------------------
%\subsection{Looking at other yield criteria \& various remarks}

%In the previous section the quantity $Y$ does not need to be a constant.
%For instance, the Drucker-Prager yield criterion can also be cast as
%$F= \sqrt{{\cal I}_2(\bm\tau)} - Y$ with $\alpha {\cal I}_1(\bm\sigma)+k$.
%Likewise the Mohr-Coulomb yield criterion of Eq.~\eqref{eq:mcF} can be written:
%\begin{equation}
%F^{\text{\tiny MC}}=
%\sqrt{  {\cal I}_2({\bm \tau})  } 
%- \frac{ c \cos \phi - \frac{1}{3} {\cal I}_1({\bm \sigma}) \sin \phi }{
%\cos \theta - \frac{1}{\sqrt{3}} \sin \theta  \sin \phi 
%}
%\end{equation}

\begin{remark}
If we use a non-associative plasticity then often $\QQQ=\sqrt{{\III}_2(\bm\tau)}$.
and the formulation of the previous section remains valid. In that case 
we have $C_1=0$ and $C_3=0$ which allows to easily arrive at a relationship 
of the type $\dot{\bm\varepsilon}^{vp} = \frac{1}{2\eta} \bm\tau$ where 
$\eta$ is a scalar viscosity. 

However, if $\QQQ$ is such that $C_3\neq 0$, then we have a problem because 
even by setting $n=1$ I do not know how to arrive at a scalar viscosity, 
and even thinking of $\eta$ as a tensor then I am stuck, see Section about 
Choi \& Petersen (2015) hereafter.
\end{remark}







