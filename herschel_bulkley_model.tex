
\index{general}{Herschel-Bulkley model}

The Herschel-Bulkley model is effectively a combination of the power-law model and 
a simple plastic model:
\begin{eqnarray}
{\bm \tau} &=& 2 \left(  K \dot{\varepsilon}_e^{n-1} 
+ \frac{\tau_0}{\dot{\varepsilon}}\right)\dot{\bm \varepsilon} \qquad \text{ if } {\tau}_{e}>\tau_0 \\
\dot{\bm \varepsilon} &=& {\bm 0} \qquad\qquad \text{if }{\tau}_{e} \leq \tau_0 
\end{eqnarray}
in which $\tau_0$ is the yield stress, $K$ the consistency, and $n$ is the flow index \index{general}{Flow Index} \cite{demj04}.
The flow index measures the degree to which the fluid is shear-thinning ($n<1$) or shear-thickening ($n>1$).
If $n=1$ and $\tau_0=0$ the model reduces to the Newtonian model. 

The term between parenthesis above is the nonlinear effective viscosity. 
Concretely, the implementation goes as 
follows\footnote{\url{https://en.wikipedia.org/wiki/Herschel-Bulkley_fluid}}:
\begin{equation}
\eta(\dot{\bm \varepsilon}) = 
\left\{
\begin{array}{cc}
\eta_0 & \dot{\varepsilon}_e\leq \dot{\varepsilon}_0 \\ 
K \dot{\varepsilon}_e^{n-1} + \frac{\tau_0}{\dot{\varepsilon}_e} & \dot{\varepsilon}_e \geq \dot{\varepsilon}_0
\end{array}
\right.
\end{equation}
The limiting viscosity $\eta_0$ is chosen such that 
$\eta_0 =  K \dot{\varepsilon}_0^{n-1} + \frac{\tau_0}{\dot{\varepsilon}_0}$

A large limiting viscosity means that the fluid will only flow in response to a large applied force. 
This feature captures the Bingham-type behaviour of the fluid. 
Note that when strain rates are large, the power-law behavior dominates. 

As we have seen for Bingham fluids, the equations above are not easily amenable to implementation so that 
one usually resorts to regularisation, which is a modification of the 
equations by introducing a new material parameter which controls the exponential 
growth of stress. This way the equation is valid for both yielded 
and unyielded areas (Blackery \& Mitsoulis (1997) \cite{blmi97},
Papanastasiou (1987) \cite{papa87}, Zinani \& Frey (2007) \cite{zifr07}, 
Sverdrup \etal (2018) \cite{svna18}):
\begin{equation}
\eta(\dot{\varepsilon}_e) 
= K \dot{\varepsilon}_e^{n-1} + \frac{\tau_0}{\dot{\varepsilon}_e} [1 - \exp(-m \dot{\varepsilon}_e)] 
\end{equation}
When the strain rate becomes (very) small a Taylor expansion of the regularisation 
term yields $1- \exp(-m \dot{\varepsilon}) \sim m \dot{\varepsilon} $ so that 
$\eta_{eff} \rightarrow m \tau_0$.
However, it seems more physically meaningful to replace $m$ by a reference strain 
rate value $\dot{\varepsilon}_0$ so that 
\begin{mdframed}[backgroundcolor=blue!5]
\begin{equation}
\eta_{eff}(\dot{\bm \varepsilon}) 
= K \dot{\varepsilon}_e^{n-1} + \frac{\tau_0}{\dot{\varepsilon}_e} 
\left[1 - \exp\left(-\frac{\dot{\varepsilon}_e}{\dot{\varepsilon}_0} \right) \right]
\end{equation}
\end{mdframed}
In this case, when strain rate becomes (very) small a Taylor expansion of the regularisation
term yields
\begin{equation}
\frac{\tau_0}{\dot{\varepsilon}_e} \left[1 - 
\exp\left(-\frac{\dot{\varepsilon}_e}{\dot{\varepsilon}_0} \right) \right]
\simeq 
\frac{\tau_0}{\dot{\varepsilon}_e} \frac{\dot{\varepsilon}_e}{\dot{\varepsilon}_0}
=\frac{\tau_0}{\dot{\varepsilon}_0} 
\end{equation}
This has the dimensions of a viscosity and this is effectively the definition 
of a maximum viscosity $\eta_{max}$.

\noindent\Literature: 
\begin{itemize}
\item Viscous flow with large free surface motion (Huerta \& Liu (1988) \cite{huli88});
\item Numerical simulation of thermal plumes (Massmeyer \etal \cite{madd13}); 
\item Flows Through a Sudden Axisymmetric Expansion (Machado \etal \cite{mazf}, 
      Jay \etal (2001) \cite{jamp01}); 
\item Dam break problem (Ancey \& Cochard (2009) \cite{anco09}, 
      Cochard \& Ancey (2009) \cite{coan09}, Balmforth \etal \cite{bafp09};
\item Weakly compressible Poiseuille flow (Taliadorou (2009) \cite{tagm09});
\item Flow past cylinders in tubes (Mitsoulis \& Galazoulas (2009) \cite{miga09});
\item Determination of yield surfaces (Burgos \& Alexandrou (1999) \cite{buae99});
\item Carbopol hydrogel rheology for experimental
      tectonics and geodynamics \textcite{dicf15} (2015);
\item Flow past a sphere(disc) \textcite{demj04} (2004); 
      Gavrilov \etal (2017) \cite{gafp17}). 
\item Progress in numerical simulation of yield stress fluid flows, \textcite{sawa17} (2017); 
\item elastoviscoplastic model based on the Herschel–Bulkley viscoplastic model \textcite{sara09} (2019).
\end{itemize}





