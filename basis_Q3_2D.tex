\begin{flushright} {\tiny {\color{gray} \tt basis\_Q3\_2D.tex}} \end{flushright}
%~~~~~~~~~~~~~~~~~~~~~~~~~~~~~~~~~~~~~~~~~~~~~~~~~~~~~~~~~~~~~~~~~~~~~~~~~~~~~~~~~~~~~~~~~~~~~~~~~~

Inside an element a possible local numbering of the nodes is as follows:

\input{tikz/tikz_q32d}

As shown in Section~\ref{sec:bf3} the 1D cubic basis functions are given by:
\begin{align}
\bN_1(r)&=(-1   +r +9r^2 - 9r^3)/16 & \bN_1(s)&=(-1   +s +9s^2 - 9s^3)/16 \nonumber\\
\bN_2(r)&=(+9 -27r -9r^2 +27r^3)/16 & \bN_2(s)&=(+9 -27s -9s^2 +27s^3)/16 \nonumber\\
\bN_3(r)&=(+9 +27r -9r^2 -27r^3)/16 & \bN_3(s)&=(+9 +27s -9s^2 -27s^3)/16 \nonumber\\
\bN_4(r)&=(-1   -r +9r^2 + 9r^3)/16 & \bN_4(s)&=(-1   -s +9s^2 + 9s^3)/16 \nonumber
\end{align}
and the resulting 2D basis functions are simply the tensor product of the above 1D ones:

\begin{mdframed}[backgroundcolor=blue!5]
\begin{eqnarray}
\bN_{01}(r,s)&=&\bN_1(r)\bN_1(s) = (-1   +r +9r^2 - 9r^3)/16 \cdot (-1  +s +9s^2 - 9s^3)/16 \nonumber\\
\bN_{02}(r,s)&=&\bN_2(r)\bN_1(s) = (+9 -27r -9r^2 +27r^3)/16 \cdot (-1  +s +9s^2 - 9s^3)/16 \nonumber\\
\bN_{03}(r,s)&=&\bN_3(r)\bN_1(s) = (+9 +27r -9r^2 -27r^3)/16 \cdot (-1  +s +9s^2 - 9s^3)/16 \nonumber\\
\bN_{04}(r,s)&=&\bN_4(r)\bN_1(s) = (-1   -r +9r^2 + 9r^3)/16 \cdot (-1  +s +9s^2 - 9s^3)/16 \nonumber\\
\bN_{05}(r,s)&=&\bN_1(r)\bN_2(s) = (-1   +r +9r^2 - 9r^3)/16 \cdot (9 -27s -9s^2 +27s^3)/16 \nonumber\\
\bN_{06}(r,s)&=&\bN_2(r)\bN_2(s) = (+9 -27r -9r^2 +27r^3)/16 \cdot (9 -27s -9s^2 +27s^3)/16 \nonumber\\
\bN_{07}(r,s)&=&\bN_3(r)\bN_2(s) = (+9 +27r -9r^2 -27r^3)/16 \cdot (9 -27s -9s^2 +27s^3)/16 \nonumber\\
\bN_{08}(r,s)&=&\bN_4(r)\bN_2(s) = (-1   -r +9r^2 + 9r^3)/16 \cdot (9 -27s -9s^2 +27s^3)/16 \nonumber\\
\bN_{09}(r,s)&=&\bN_1(r)\bN_3(s) = (-1   +r +9r^2 - 9r^3)/16 \cdot (9 +27s -9s^2 -27s^3)/16 \nn\\
\bN_{10}(r,s)&=&\bN_2(r)\bN_3(s) = (+9 -27r -9r^2 +27r^3)/16 \cdot (9 +27s -9s^2 -27s^3)/16 \nn\\
\bN_{11}(r,s)&=&\bN_3(r)\bN_3(s) = (+9 +27r -9r^2 -27r^3)/16 \cdot (9 +27s -9s^2 -27s^3)/16 \nn\\
\bN_{12}(r,s)&=&\bN_4(r)\bN_3(s) = (-1   -r +9r^2 + 9r^3)/16 \cdot (9 +27s -9s^2 -27s^3)/16 \nn\\
\bN_{13}(r,s)&=&\bN_1(r)\bN_4(s) = (-1   +r +9r^2 - 9r^3)/16 \cdot (-1   -s +9s^2 + 9s^3)/16\nn\\
\bN_{14}(r,s)&=&\bN_2(r)\bN_4(s) = (+9 -27r -9r^2 +27r^3)/16 \cdot (-1   -s +9s^2 + 9s^3)/16\nn\\
\bN_{15}(r,s)&=&\bN_3(r)\bN_4(s) = (+9 +27r -9r^2 -27r^3)/16 \cdot (-1   -s +9s^2 + 9s^3)/16\nn\\
\bN_{16}(r,s)&=&\bN_4(r)\bN_4(s) = (-1   -r +9r^2 + 9r^3)/16 \cdot (-1   -s +9s^2 + 9s^3)/16
\end{eqnarray}
\end{mdframed}

\begin{center}
\includegraphics[width=4cm]{images/basis_Q3_2D/N1}
\includegraphics[width=4cm]{images/basis_Q3_2D/N2}
\includegraphics[width=4cm]{images/basis_Q3_2D/N3}
\includegraphics[width=4cm]{images/basis_Q3_2D/N4}\\
\includegraphics[width=4cm]{images/basis_Q3_2D/N5}
\includegraphics[width=4cm]{images/basis_Q3_2D/N6}
\includegraphics[width=4cm]{images/basis_Q3_2D/N7}
\includegraphics[width=4cm]{images/basis_Q3_2D/N8}\\
\includegraphics[width=4cm]{images/basis_Q3_2D/N9}
\includegraphics[width=4cm]{images/basis_Q3_2D/N10}
\includegraphics[width=4cm]{images/basis_Q3_2D/N11}
\includegraphics[width=4cm]{images/basis_Q3_2D/N12}\\
\includegraphics[width=4cm]{images/basis_Q3_2D/N13}
\includegraphics[width=4cm]{images/basis_Q3_2D/N14}
\includegraphics[width=4cm]{images/basis_Q3_2D/N15}
\includegraphics[width=4cm]{images/basis_Q3_2D/N16}\\
{\captionfont Surface representation of the basis functions on the reference element.
{\color{gray} in images/basis\_Q3\_2D/ }}
\end{center}
The derivatives are trivial to obtain from the derivatives of the 1D basis functions, 
e.g.
\[
\frac{\partial \bN_{13}}{\partial r} = 
\frac{\partial \bN_{1}}{\partial r} \bN_3(s) 
\]
These basis functions are used in \stone 19.
