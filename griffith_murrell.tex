\index{general}{Griffith-Murrell}
\begin{flushright} {\tiny {\color{gray} griffith\_murrell.tex}} \end{flushright}
%~~~~~~~~~~~~~~~~~~~~~~~~~~~~~~~~~~~~~~~~~~~~~~~~~~~~~~~~~~~~~~~~~~~~~~~~~~~~~~~~~~~~~~~~~~~~~~~~~~

The Griffith-Murrell yield criterion \cite{brau94,brbe95,babr97} is not often used. 
Extending the work of Griffith (1921) to three dimensional stress distributions, 
Murrell (1963) suggested the following criterion for rock failure expressed 
in terms of the principal stresses:
\[
(\sigma_1-\sigma_2)^2 + (\sigma_2-\sigma_3)^2 + (\sigma_3-\sigma_1)^2
+
24T_0 (\sigma_1+\sigma_2+\sigma_3)=0
\]
where $T_0$ is a material property called the tensile strength. In principal stress space, 
this criterion is represented by a paraboloid of revolution around the pressure (or hydrostatic) axis.

Using the definition of ${\cal I}_2({\bm \tau})$ and ${\cal I}_1({\bm \sigma})$, it also writes:
\[
{\cal I}_2({\bm \tau}) - 12 T_0 p =0
\]
which is the formulation used in Hansen \etal (2000) \cite{hanl00}, although the authors
use the lithostatic pressure instead of the full pressure. They also use a tensile 
strength parameter $T_0^e$ and a compressive strength parameter $T_0^c$, both around a few tens 
of MPas.
