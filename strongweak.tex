\begin{flushright} {\tiny {\color{gray} \tt strongweak.tex}} \end{flushright}
%------------------------------------------------------------------------------

\index{general}{strong form} 

As we have seen in Section~\ref{sec:diff1D}
the strong form consists of the governing equation and the boundary conditions, i.e. 
the mass, momentum and energy conservation equations supplemented with Dirichlet and/or Neumann
boundary conditions on (parts of) the boundary. Ultimately we have two main unknowns that 
we wish to solve for: velocity (a vector) and pressure (a scalar).

\index{general}{weak form}
To develop the finite element formulation, the partial differential equations 
must be restated in an integral form called the weak form. In essence the PDEs are 
first multiplied by an arbitrary function and integrated over the domain.

