\begin{flushright} {\tiny {\color{gray} recap\_invariants.tex}} \end{flushright}
%~~~~~~~~~~~~~~~~~~~~~~~~~~~~~~~~~~~~~~~~~~~~~~~~~~~~~~~~~~~~~~~~~~~~~~~~~~~~~~~~~~~~~~~~~~~~~~~~~~

When it comes to stress invariants, I urge the reader to be extremely careful when considering 
their source(s). As we have seen these come in two flavours (e.g. principal and moment invariants)
and they are often written for the full stress or deviatoric tensor. On top of it all, 
typos are common like in any source and the occasional minus sign or factor 2 or 3 can be missing.
This is the reason why I have spent substantial time re-deriving those in the past pages 
with a consistent set of notations:
\begin{center}
\begin{tabular}{ll}
\hline
${\bm \sigma}$ & (full) stress tensor \\
$\sigma_1$, $\sigma_2$, $\sigma_3$ & principal stresses \\ 
${\bm \tau}$   & deviatoric stress tensor \\
$\tau_1$, $\tau_2$, $\tau_3$ & principal deviatoric stresses \\ 
${\III}_1({\bm T})$ & first moment invariant of tensor ${\bm T}$ \\
${\III}_2({\bm T})$ & second moment invariant of tensor ${\bm T}$ \\
${\III}_3({\bm T})$ & third moment invariant of tensor ${\bm T}$ \\
${\tau}_{e}=\sqrt{{\III}_2({\bm \tau})}$ & effective deviatoric stress \\
$\dot{{\varepsilon}}_{e}=\sqrt{{\III}_2(\dot{\bm \varepsilon}^d)}$ & effective deviatoric strain rate \\
\hline
\end{tabular}
\end{center}
Proofs of all the following relationships are given in Appendix~\ref{app:invariants}.

\begin{itemize}
\item First invariant %-----------------------------------

\begin{eqnarray}
{\III}_1(\bm\sigma) &=& \sigma_{xx}+\sigma_{yy}+\sigma_{zz} \nn\\
{\III}_1(\bm\tau) &=& 0 \nn\\ 
\frac{\partial {\III}_1(\bm\sigma)}{\partial \bm\sigma} &=& {\bm 1}  \nn
\end{eqnarray}

\item Second invariant %-----------------------------------

\begin{eqnarray}
{\III}_2(\bm\tau) 
&=& \frac12 {\bm \tau}:{\bm \tau} \nn\\
&=& \frac12 {\rm tr} [{\bm \tau}\cdot {\bm \tau}] \nn\\
&=& \frac12 \sum_{ij} \tau_{ij}\tau_{ji}  \nn\\
&=& \frac12 ( \tau_{xx}^2 + \tau_{yy}^2 +\tau_{zz}^2 + 2\tau_{xy}^2+ 2\tau_{xz}^2+ 2\tau_{yz}^2) \nn\\
&=& \frac{1}{6}\left[(\sigma_{xx}-\sigma_{yy})^2 + (\sigma_{yy}-\sigma_{zz})^2 
+ (\sigma_{xx}-\sigma_{zz})^2 \right]  + \sigma_{xy}^2 + \sigma_{xz}^2 + \sigma_{yz}^2 \nn\\
&=&  -\frac16 {\III}_1(\bm\sigma)^2 + {\III}_2(\bm\sigma) \nn\\
\frac{\partial {\III}_2(\bm\tau)}{\partial \bm\sigma} &=& \bm\tau   \nn
\end{eqnarray}

\item Third invariant %-----------------------------------

\begin{eqnarray}
{\III}_3(\bm\tau) 
&=& \frac13 \sum_{ijk} \tau_{ij}\tau_{jk}\tau_{ki} \nn\\
&=& {\rm det}(\bm\tau) \nn\\
&=& \frac13 {\rm tr} [{\bm \tau}\cdot {\bm \tau} \cdot {\bm \tau}] \nn\\
&=& \frac{2}{27}  {\III}_1(\bm\sigma)^3 - \frac23{\III}_1(\bm\sigma) {\III}_2(\bm\sigma)
+{\III}_3(\bm\sigma) \nn\\
\frac{\partial {\III}_3(\bm\tau)}{\partial \bm\sigma} &=&
\end{eqnarray}

\begin{equation}
\theta_{\rm L}=\frac{1}{3} \sin^{-1} 
\left( -\frac{3\sqrt{3}}{2} \frac{{\III}_3({\bm \tau})}{{\III}_2({\bm \tau})^{3/2}} \right)
\end{equation}


\end{itemize}


\begin{eqnarray}
\frac{\partial {\III}_1(\bm\sigma)}{\partial \bm\sigma} &=& {\bm 1} \\
\frac{\partial {\III}_2(\bm\sigma)}{\partial \bm\sigma} &=& {\bm \sigma} \\
\frac{\partial {\III}_3(\bm\sigma)}{\partial \bm\sigma} &=& \bm\sigma \cdot \bm\sigma
\end{eqnarray}



























