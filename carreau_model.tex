\index{general}{Carreau model} 
\begin{flushright} {\tiny {\color{gray} \tt carreau\_model.tex}} \end{flushright}
%~~~~~~~~~~~~~~~~~~~~~~~~~~~~~~~~~~~~~~~~~~~~~~~~~~~~~~~~~~~~~~~~~~~~~~~~~~~~~~~~~~~~~~~~~~~~~~~~~~

Note that this model is sometimes called Bird-Carreau in the literature. \index{general}{Bird-Carreau model}
As explained in Reddy \cite{reddybook2}, the power-law model poses no restriction on 
how small or large the viscosity may become, which may prove problematic once 
implemented as it can lead to runaway effects (strain rate becomes large $\rightarrow$
viscosity becomes smaller $\rightarrow$ strain rate becomes larger, etc ...).
This problem is alleviated in the so-called Carreau
\footnote{\url{https://en.wikipedia.org/wiki/Carreau_fluid}} model \cite{carr72} 
(see for example Zinani \& Frey (2007) \cite{zifr07}). 
The viscosity is then given by\footnote{In \cite{saramito}
we find a slightly different expression where the square exponent is
omitted.}
\begin{equation}
\eta(\dot{\varepsilon}_{e}) = \eta_\infty + (\eta_0-\eta_\infty) 
\left(1 + (\lambda \dot{\varepsilon}_{e})^2 \right)^{(n-1)/2}
\end{equation}
where $\eta_0$, $\eta_\infty$, $\lambda$ and $n\in[0,1]$ are material parameters. 
$\lambda$ is called the relaxation time: it is the inverse of the shear rate at which 
the fluid changes from Newtonian to power-law behavior.
This expression is to be found in the Comsol 
manual\footnote{\url{https://doc.comsol.com/5.3/doc/com.comsol.help.cfd/cfd_ug_fluidflow_single.06.062.html}}. 

At low strain rate a Carreau fluid behaves as a Newtonian fluid with 
viscosity $\eta_0$. At intermediate strain rates 
$\dot{\varepsilon}_{e} \lambda \sim 1$ a Carreau fluid behaves 
as a Power-law fluid. At high strain rate, a Carreau fluid behaves 
as a Newtonian fluid again with viscosity $\eta_\infty$.
 
\begin{center}
\includegraphics[width=7cm]{images/rheology/carreau/carreau.pdf}
\includegraphics[width=6cm]{images/rheology/carreau/carreau1}\\
{\captionfont Left: Carreau model effective viscosity as a function of 
the product $\lambda \dot{\varepsilon}_{e}$. Right: taken from 
video at \url{https://youtu.be/qErs5zZV4BQ}.}
\end{center}

Note that the (Bird)-Carreau-Yasuda model \cite{yaac81,osru14} is 
very similar to the standard (Bird)-Carreau:
\begin{equation}
\eta = \eta_\infty + (\eta_0-\eta_\infty) 
\left(1 + (\lambda \dot{\varepsilon}_{e})^a \right)^{(n-1)/a}
\end{equation}
It is for instance used in van de Vosse \etal (2003) \cite{vadv03} to model blood.
\index{general}{Bird-Carreau-Yasuda model}

Flows in a Lid-Driven Cavity with this rheology are presented in \cite{zifr07,shal09}.

\Literature: Bercovici (1993) \cite{berc93}, Bercovici (1995) \cite{berc95},
Marcotte (2000) \cite{marc00}, Huerta \& Liu (1988) \cite{huli88}.
