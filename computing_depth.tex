
\begin{flushright} {\tiny {\color{gray} computing\_depth.tex}} \end{flushright}

In the case of a perfectly rectangular, cylindrical or spherical domain, 
computing the depth of any given point inside the domain is trivial. 
However, when the free surface becomes somewhat distorted, the concept of 
depth needs to be refined. What follows is an attempt at bringing clarity
as to how to compute depth in all cases.

The depth $d({\bm r})$ satisfies the equation:
\[
\frac{{\bm g}}{|{\bm g}|} \cdot {\bm \nabla} d = 1
\]
with $d=0$ at the surface.

This is a form of steady-state advection equation (the time derivative is zero, 
there is no diffusion, nor any source term).

Given the boundary conditions, one could solve this equation 
over the whole domain. 

Note that in the case of a cartesian box, ${\bm g}=-g {\bm u}_z$,
we need to solve 
\[
- \frac{\partial}{\partial z} d = 1
\]
For a flat top surface at $d(z=L_z)=0$ so that in the end
\[
d(z)=L_z-z
\]

