\index{general}{$P_1\times P_0$}
\begin{flushright} {\tiny {\color{gray} \tt  pair\_p1p0.tex}} \end{flushright}
%~~~~~~~~~~~~~~~~~~~~~~~~~~~~~~~~~~~~~~~~~~~~~~~~~~~~~~~~~~~~~~~~~~~~~~~~~~~~~~~~~~~~~~~~~~~~~~~~~~



example 3.70 in \textcite{john16} (book),


Elman Silvester Wathen say (5.3.3) that 
\begin{displayquote}
{\color{darkgray}
It can be readily stabilized using the pressure jump
stabilization together with an appropriate macroelement subdivision.}
\end{displayquote}
See \textcite{nosi98} (1998) for globally and locally stabilised versions. 

\textcite{qizh07b} (2007) state: 
\begin{displayquote}
{\color{darkgray}
The element is unstable for any mesh since
the dimension of the discrete velocity space is always less than that of the pressure space (with
Dirichlet boundary condition).
However, this element provides optimal approximations for both
the velocity and the pressure on many mesh families.} 
\end{displayquote}

\textcite{arno93} (1993) states: 
\begin{displayquote}
{\color{darkgray}
Unfortunately, this simplest possible Stokes element is notoriously unstable. On any tri-
angulation with at least three vertices on the boundary the dimension of the pressure
space exceeds that of the velocity space [...] and the finite
dimensional problem is singular. 
Moreover, while the discrete velocity field $u_h$ is uniquely
determined (as it is for any conforming method for the Stokes problem), for this choice of
elements $u_h$ belongs to the space of divergence-free fields piecewise linear fields, and on
many meshes, for example on a uniform diagonal mesh of the square [...],
this space is known to reduce to zero. So even after accounting for the indeterminancy of
the pressure we have no convergence.}
\end{displayquote}

Example 3.2 of \textcite{bobf08} (2008) explains neatly the locking phenomenon and how 
to circumvent it via a so-called cross-grid macroelement. See also 
\textcite{hokl03} (2003).

In his lecture notes\footnote{\url{https://www.math.tamu.edu/~guermond/}}, 
Guermond states 
\begin{displayquote}
{\color{darkgray}
A simple alternative to the 
$Q_1\times P_0$ element consists of using the $P_1\times P_0$ element.
Let ${\cal T}_h$  be a mesh of $D$ composed of affine simplices, and approximate the 
velocity with continuous piecewise linear polynomials and the pressure with
(discontinuous) piecewise constants. Since the velocity is piecewise linear, its
divergence is constant on each simplex. As a result, testing the divergence
of the velocity with piecewise constants enforces the divergence to be zero
everywhere. That is to say, the $P_1\times P_0$ finite element yields a velocity 
approximation that is exactly divergence-free [...]. Unfortunately,
this pair does not satisfy the inf-sup condition.}
\end{displayquote}

I have not found a paper yet which showcases its accuracy on a manufactured solution
and compares it to other element pairs.


