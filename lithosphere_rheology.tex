\begin{flushright} {\tiny {\color{gray} lithosphere\_rheology.tex}} \end{flushright}
%~~~~~~~~~~~~~~~~~~~~~~~~~~~~~~~~~~~~~~~~~~~~~~~~~~~~~~~~~~~~~~~~~~~~~~~~~~~~~~~~~~~~~~~~~~~~~~~~~~




\begin{center}
\includegraphics[height=5cm]{images/rheology/budr08}\\
{\captionfont Schematic view of the three most common first order rheological models of the continental 
lithosphere under a strain rate of 10$^{-14}$s$^{-1}$ . 
In all three models the upper crust has its frictional strength increased with pressure and depth. 
(a) The jelly sandwich model has a weak mid-lower crust and a strong mantle composed of dry olivine. 
(b) The cr\`eme br\^ul\'ee model assumes that the mantle is weak, due to the presence of water and high 
temperature deformation, and the dry and brittle crust determines the strength of the lithosphere. 
(c) The banana split model assumes that the lithosphere as a whole has its strength greatly reduced
due to various strain weakening and feedback processes \cite{budr08}}
\end{center}

\begin{center}
\includegraphics[width=8cm]{images/rheology/bird99}\\
{\captionfont Taken from \cite{bird99}.
Typical vertical distribution of maximum shear stress in continental lithosphere 
undergoing compressional (right) or extensional (left) strain at $10^{-15}$s. 
Friction controls level of shear stress in upper part of crust and sometimes in mantle lithosphere;
then, below brittle/ductile transition, shear stress is controlled by thermally-activated dislocation creep.}
\end{center}

Molnar \cite{moln92} discusses the validity of the Brace-Goetze strength profiles. 
In particular, he has this to say about the power law parameters:
{\it
The uncertainty alone in $Q$ alone renders calculated strengths 
uncertain by 10 times at temperatures of about 700C.
Correspondingly, that uncertainty in Q is approximately equivalent 
to an uncertainty of about 100C in temperature.
}


\begin{center}
\begin{tabular}{l|l}
\hline

Wet Quartzite & upper crust  \cite{jahu12,wabj08} \\
              & upper continental crust \cite{kecw09,cube11} \\
              & lower crust  \cite{jahu12,wabj08} \\
              & ocean sediment \cite{kecw09} \\
Dry Olivine   & lithosphere  \cite{hube07}\\
              & sublithospheric mantle \cite{hube07}\\
Dry Maryland Diabase & lower crust \cite{wabj08,wabj08b} \\
                     & lower continental crust \cite{kecw09,cube11} \\
                     & oceanic crust \cite{wabj08,kecw09,wabj08b,cube11} \\
Wet Olivine   & continental mantle lithosphere \cite{wabj08,wabj08b} \\
              & oceanic mantle lithosphere \cite{wabj08,wabj08b} \\
              & sublithospheric mantle \cite{wabj08,kecw09,wabj08b} \\
              & mantle lithosphere \cite{kecw09} \\
\hline
\end{tabular}
\end{center}



\Literature \cite{buwa06,budr08,rana97a,rana97b}

\todo[inline]{I need to talk about Byerlee's law. \cite{byer78}}


