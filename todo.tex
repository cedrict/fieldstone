\documentclass[a4paper]{article}
\usepackage[cm]{fullpage}
\usepackage{graphicx}

%%%%%%%%%%%%%%%%%%%%%%%%%%%%%%%%%%%%%%%%%%%%%%%%%%%%%%%%%%%
%Bibliography stuff
%%%%%%%%%%%%%%%%%%%%%%%%%%%%%%%%%%%%%%%%%%%%%%%%%%%%%%%%%%%
\usepackage[maxnames=6]{biblatex}
\addbibresource{biblio_geosciences.bib}
 

%%%%%%%%%%%%%%%%%%%%%%%%%%%%%%%%%%%%%%%%%%%%%%%%%%%%%%%%%%%%%%%%%%%%%%%%%%%%%%%
%%%%%%%%%%%%%%%%%%%%%%%%%%%%%%%%%%%%%%%%%%%%%%%%%%%%%%%%%%%%%%%%%%%%%%%%%%%%%%%
\begin{document}

\begin{center}
\includegraphics[width=5cm]{images/garfield-computer}
\end{center}

\tableofcontents

%%%%%%%%%%%%%%%%%%%%%%%%%%%%%%%%%%%%%%%%%%%%%%%%%%%%%%%%%%%%%%%%%%%%%%%%%%%%%%%
\newpage
\section{Improvements to FieldStone itself}

\begin{itemize}
\item ask Bob about log Newton rheology stuff
\item write axisymm energy eq 
\item sort our s40rts stones/theory
\item change cell type in all stones containing Q2
\item find Dave Whipp vids \url{https://www.youtube.com/@helsinkiuniversitygeodynam6511/videos}
\begin{itemize}
\item set 2: Kinematics of plate tectonics
\item set 3: forces and stresses
\item set 4: measuring stress and strain 
\item set 5: basics of elasticity
\item set 6: flexure of the lithosphere
\item set 7: heat conduction and production
\item set 8: thermal processes in the lithosphere
\item set 9: basic of fluid mechanics
\item set 10: plate-driving forces
\item set 11: brittle deformation and faulting
\item set 12: viscous deformation, strength of the lithosphere
\item set 13: climatic, geomorphic and geodynamic processes
\end{itemize}
\item scan biblio of Yuen, Olson, Turcotte, Travis
\item produce appendix about making presentations dos/donts 
\item finish 133 about double jacobian
\item write about Scott-Vogelius element, document literature. 
\item incorporate Scott-Vogelius in f120. P2P-1 on bary meshes
\item do P1P0 on special bary mesh
\item do QkQk-1Q-1 stone
\item replace cal I,K, by III, KKK, FFF, QQQ in invariants 
\item check Q2Q1+graddiv vs Q2Q1 or Q2P-1. report. paper ?
  \item write about impose bc on el matrix
  \item write about stream functions 
  \item write section in features about thermo mechanical simulations and how/why we solve vp, then T.
  \item write Scott about matching compressible2 setup with his paper
  \item write about vorticity-velocity method: \cite{gats91,gust93,dehu95,ergq99,amct04,spez87}
  \item write about flexural isostasy \cite{maie12}, bottom Sopale
  \item write about splines as basis functions \cite{chri92}. second or third order basis functions , using extra nodes instead of using more nodes per element. 
  Smaller matrix than Q2 or Q3 but: spline coeffs on nodes are no more unknowns. Plus bc are complicated. Does it work well with visc contrasts ?
  \item write about jacobian tensor and norm for Q1 rectangles. and more ?
  \item write about Courant nb
  \item write about number representation, binary, min/max ... 
  \item appendix E finish/update
  \item generalised averaging of ELEFANT paper
  \item write about Aitken extrapolation, see for example \cite{jolm17}

\end{itemize}

%%%%%%%%%%%%%%%%%%%%%%%%%%%%%%%%%%%%%%%%%%%%%%%%%%%%%%%%%%%%%%%%%%%%%%%%%%%%%%%
\newpage
\section{Ideas for new stones}

\begin{itemize}
\item Write Stone to redo Salt tectonics experiments of \cite{dacl94}
\item revisit CBF and produce stone
\item DF, FEM of 1D, 2D wave equation stone, look at becker and kaus syllabus matlab file
\item write FVM code
\item write Navier-Stokes solver FEM or FDM (streamfunction and not streamfunction based)\\
\begin{itemize}
\item Navier-Stokes Solver using Finite Differences python code 
\url{https://github.com/saadtony/uCFD/blob/master/Navier Stokes Finite Difference Method.ipynb}
\item A Simple Staggered FV Code for the Navier-Stokes Equations
\url{https://github.com/saadtony/uCFD/blob/master/Navier-Stokes FVM Staggered - Driven Cavity (CHEN6355).ipynb}
\item implement Navier-Stokes eqs a la http://ww2.lacan.upc.edu/huerta/exercises/Incompressible/Incompressible\_Ex2.htm
\end{itemize}

\item amg solver
\url{https://github.com/saadtony/uCFD/blob/master/Using Sparse Solvers in Python.ipynb}

\item translate ibuprofem (surface tension code) surface tension see \cite{reddybook2}p28-29 - 
see \cite{dett04}. benchmarks in \cite{chcc12} 
\item redo  cookbook 'convection in 2d box w phase transitions' based on \textcite{chyu85}. 
\item write stone with phase change a la van den Berg \textcite{vava08}
\item Hollow sphere under internal pressure with analytical solution
. elastic and elasto-plastic solution  - see pdf document
\item Flow past a sphere(disc) \textcite{demj04} (2004); 
      \textcite{gafp17}). 
\item revisit stone 114 (inversion stokes+grav), use msc thesis 
\item ELEFANT:  parallel iterative solver w/ Arie. slice it with nthreads, parallel PCG
\item revive f90 code FDM stokes staggered grid, translate it to python 
(in 1427 folder)
\item Scott-Vogelius stone
\end{itemize}

%%%%%%%%%%%%%%%%%%%%%%%%%%%%%%%%%%%%%%%%%%%%%%%%%%%%%%%%%%%%%%%%%%%%%%%%%%%%%%%
\newpage
\section{BSc topics}

\begin{itemize}
\item run stone 88 (convection in 2D box) and look at configurational entropy 
(stone 137). Look at todo list at the end of stone 88.
\item compute elemental matrices of most common spaces with sympy
\item build library of mms, tester
\item biblio: for example, what are the sub-themes of subduction ? 
or mantle convection. re-organise biblio around these themes
\item structural geology: 2 and 3 layer (non)linear viscous folding
\item \fullcite{wiwh91}
\item Gravity of cuboid: Sort out the mess by \textcite{duti16} and \textcite{zhhu17}.
\item write stone for Kovasznay mms
\item symbolic calculations for FE matrices
\item chunk grid. benchmark of busa13?
\item Slab detachment + diff elements + Aitken
\item go after meaningful benchmarks of past 20-30 years, contact authors, digitise figures, etc ...build database
\item Compute gravity for each layer of CRUST1.0 , see stone 96
\item geoid, who what where how, redo topo and geoid calculations a la \cite{king09}
\item Mars DEM \url{https://astrogeology.usgs.gov/search/details/Mars/GlobalSurveyor/MOLA/Mars_MGS_MOLA_DEM_mosaic_global_463m/cub} gravity
\item redo Crameri benchmark rising bubble with stone 95-ish
\item compute gravity above Lex's model \cite{furc15}
\item use GEMMA moho data \url{http://gocedata.como.polimi.it/} \cite{resa15} 
\item deformation around rigid particles \cite{ilma93}
\end{itemize}


%==============================================================================
\section{GR thesis topics}
\begin{itemize} 
\item FSSA implementation, derivation, everywhere in domain \cite{sctc20}, implementation, derivation. Build elefant paper
      setup with boundary fitted mesh.
\item numerical viscosimeter \cite{batt84}, taylor couette flow, stokes flow dye cylinders
\item advection a la van Hunen, Lagrangian characteristics - see \textcite{bepo10} (2010)
\item folding experiments a la Frehner MSc/PhD thesis and papers 
\item redo/combine Morgan \& Forsyth (1988) \cite{mofo88} 3D T field around transform faults in oceanic crust
\item shear band fractals of elefant paper
\item Busse benchmark (stone 20) upgrade, run, document
\item thin layer entrainment with aspect and build stone for it 
\item axisymmetric diapir, Daly \& Raefsky \cite{dara85}; deformation around a rising diapir modeled by creeping 
      flow past a sphere, analogue \cite{crud88}
\item replicate \textcite{khfh15} \citetitle{khfh15}. Also look at \textcite{khmo21}.
\item axisymmetric stokes benchmark \textcite{lezh10} (2010)
\end{itemize}





%%%%%%%%%%%%%%%%%%%%%%%%%%%%%%%%%%%%%%%%%%%%%%%%%%%%%%%%%%%%%%%%%%%%%%%%%%%%%%%
\newpage
\section{MSc topics, guided research}

\begin{itemize}
\item Computational structural geology. Redo \fullcite{trla00} \\
Find collaborator (Alissa? Martin Drury? ...)
\item Write 3D T solver for geothermal problems. Look at/incorporate
mammoth code.
\item \textcite{vaks97} (1997) Rayleigh-Taylor problem with triangles, and remeshing.
\item produce python code for my GHOST code \textcite{thie18} 3D mesher, for Q1, Q2, P1,P2.
\item Hernlund \& Tackley pseudo annulus \fullcite{heta08}. See \textcite{josv21}
\item nonlinear viscosimeter. annulus geometry, different rheologies.
\item paper 2023 two phase flow fantom with huismans may
\item dyn topo vs true free surface... who wins?
\item build ASPECT stokes solver in python (pyamg?)
\item Biharmonic equation for stream function approach with FEM. check Schmalzl phd thesis.
check vorticity stream function formulation, see glte87.
Also check \url{https://fenicsproject.org/olddocs/dolfin/1.6.0/python/demo/documented/biharmonic/python/documentation.html} for biharmonic + disc galerkin. Can I solve Poisson eqs super fast?
\item anisotropic viscosity , see \textcite{leha08}.
redo the Rayleigh-Taylor instabilities with 
anisotropic lithospheric viscosity.
Look at the three methods listed in the other article by \textcite{leha08b}. 
Check Appendix B.2.2 or \textcite{perr19} for additional results.

\item redo \fullcite{clau82}
\item anisotropic heat conduction, see  Reddy \cite[p121]{reddybook2}, Reddy \cite[p143]{reddybook2} 
\item pressure smoothing for Q1. Get relevant literature, digest it, implement all variants in stone 12.
look at remark in \textcite{lumh24}.
\index{general}{MSc Thesis} 

\item implement a simple Newton solver and apply it to a few nonlinear  benchmarks. 

\item gravity of tesseroid , stone 98
\item look/redo/expand \fullcite{shpp13}. Pb: they have 12 orders of magnitude viscosity variations.
Use axisymmetic ? annulus?

\item in \textcite{chri84} (1984) he talks about the effective Rayleigh number 
that is evaluated using a squared strain-rate averaged viscosity:
\[
\langle \eta \rangle_2 = \int \eta \dot{\varepsilon}^2 dV/\int \dot{\varepsilon}^2 dV
\]
implement in stones and aspect ?

\item CVI 2D and 3D

\item mixing . entropy mixed with convection stone 88. compute lyapunov exponent, look at all other methods.

\item STREAM + CVI + mixing ?

\item rerun allken cookbook at high resolution with damper and divv=R
\item salt tectonics wirh Ernst \cite{veja92,vasv93,maar96,istt04,maqs06,maqs07,bakp14,feka14a}

\item look at quinquis and buiter paper about H20 release and transport, water migration in subduction
\item redo magma chamber studies \cite{cuwi14,gehn18}


\item write surface processes code
\item elasticity with markers
\item pure shear deformation of inclusions \cite{trla00}
\cite{haoc78,haoc81,riha84,zhon96,como97,mohc98,zhzu00,lezh08,leha08,mofm07,mibb09,fope91,lizh13,bugo94} 
\item redo improved method of Nusselt number calculation \cite{hohr87}
\item implement multigrid: Taras' book: translate matlab code to fortran? Use the one in Trim et al?

\item ask Taka to finish/write grain contact benchmark
\item produce stone for Taka's notch
\item priduce stone for cicrle/ellipse hole in place - taka
\item build fast/reliable schur complement solver
\item revisit all tikz of elements and add consistent colours
\item revisit diff/disl creep partition stuff
\item free-slip bc on annulus and sphere . See for example p540 Gresho and Sani book. find book \cite{deab72}.
also check \cite{ensg82} !!
\item constraints \cite{absh79}
\item look at strain-rate softening in \cite{belz02}
\item Material point method \cite{sucs94,susc96,susp07}
\item redo/explore dyn topo bench of \cite{bore19} MSc thesis!
\item check \cite{bufm19} for RT0 element use
\item GEO1442 indenter setup in plane ?
\item SIMPLE a la p667 \cite{john16}. Also look at \cite{vusb00} 
\item try Anderson acceleration for Uzawa \cite{hoow17} with m=1. Aitken method for 
fixed point iterations, see Ramiere \& Helfer \cite{rahe15}.
Check Walker \& Ni (2011) \cite{wani11}, Anderson (1965) \cite{ande65}
Boyle \& Jennings (1973) \cite{boje73}, Jennings (1971) \cite{jenn71}.
Aitken method for accelerate nonlinear problems, Chow \& Kay (1984) \cite{chka84}.
Wang \& Wang (1994) \cite{wawa94}. Richardson extrapolation. 
\item implementation of fault in FEM codes \cite{zhgu94,zhgu95}
\item remove nnp from all stone
\item use 3D benchmark of s75 for s82 !
\item write generic ker program for LBB stab check
\item look at Sobol sequences in Numerical recipes for tracer layout\\
https://people.sc.fsu.edu/\~{}jburkardt/f\_src/sobol/sobol.html
\item redo thie17 in axisymmetric geometry



\end{itemize}



with ASPECT:

\begin{itemize}
\item redo early compressional orogen study by Beaumont \cite{bequ94}
\item redo extension 3D Allken papers + \cite{poay84,katl95} 
\item redo Travis study \cite{trab90} which is close to Blankenbach \cite{blbc89}. 
Note that \cite{maie12} looks at kinetic energy for \cite{trab90} 
\end{itemize}


%%%%%%%%%%%%%%%%%%%%%%%%%%%%%%%%%%%%%%%%%%%%%%%%%%%%%%%%%%%%%%%%%%%%%%%%%%%%%%%
\newpage
\section{Questions I have, Things I need to (better) understand}

These are not particularly well posed. 

\begin{itemize}
\item how does FLAC work?
\item in FEM, time derivative handled by FDM but not FEM?
\item why not ok to prescribe T on outflow boundary. check aspect manual
\item the need to underintegrate penalty term
\item pseudo-transient methods
\item what does it mean to have a negative pressure ? should we threshold it when computing yield strength ? 
\item Why pressure bc works for Q2Q1 but not Q1P0 ??
\item is there a formal definition of shape fct, trial fct, basis fct vs test fct 
\item is there an efficient manner to start from (say) quadrilateral mesh defined by corners only, choose an 
order $n$ corresponding to $Q_n \times Q_{n-1}$ and generate nodes without doubles?
\end{itemize}


\end{document}
%%%%%%%%%%%%%%%%%%%%%%%%%%%%%%%%%%%%%%%%%%%%%%%%%%%%%%%%%%%%%%%%%%%%%%%%%%%%%%%
%%%%%%%%%%%%%%%%%%%%%%%%%%%%%%%%%%%%%%%%%%%%%%%%%%%%%%%%%%%%%%%%%%%%%%%%%%%%%%%
%%%%%%%%%%%%%%%%%%%%%%%%%%%%%%%%%%%%%%%%%%%%%%%%%%%%%%%%%%%%%%%%%%%%%%%%%%%%%%%
%%%%%%%%%%%%%%%%%%%%%%%%%%%%%%%%%%%%%%%%%%%%%%%%%%%%%%%%%%%%%%%%%%%%%%%%%%%%%%%
%%%%%%%%%%%%%%%%%%%%%%%%%%%%%%%%%%%%%%%%%%%%%%%%%%%%%%%%%%%%%%%%%%%%%%%%%%%%%%%
%%%%%%%%%%%%%%%%%%%%%%%%%%%%%%%%%%%%%%%%%%%%%%%%%%%%%%%%%%%%%%%%%%%%%%%%%%%%%%%
%%%%%%%%%%%%%%%%%%%%%%%%%%%%%%%%%%%%%%%%%%%%%%%%%%%%%%%%%%%%%%%%%%%%%%%%%%%%%%%
%%%%%%%%%%%%%%%%%%%%%%%%%%%%%%%%%%%%%%%%%%%%%%%%%%%%%%%%%%%%%%%%%%%%%%%%%%%%%%%
%%%%%%%%%%%%%%%%%%%%%%%%%%%%%%%%%%%%%%%%%%%%%%%%%%%%%%%%%%%%%%%%%%%%%%%%%%%%%%%
%%%%%%%%%%%%%%%%%%%%%%%%%%%%%%%%%%%%%%%%%%%%%%%%%%%%%%%%%%%%%%%%%%%%%%%%%%%%%%%
%%%%%%%%%%%%%%%%%%%%%%%%%%%%%%%%%%%%%%%%%%%%%%%%%%%%%%%%%%%%%%%%%%%%%%%%%%%%%%%








