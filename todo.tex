\documentclass[a4paper]{article}

%%%%%%%%%%%%%%%%%%%%%%%%%%%%%%%%%%%%%%%%%%%%%%%%%%%%%%%%%%%
%Bibliography stuff
%%%%%%%%%%%%%%%%%%%%%%%%%%%%%%%%%%%%%%%%%%%%%%%%%%%%%%%%%%%
\usepackage[maxnames=6]{biblatex}
\addbibresource{biblio_geosciences.bib}
 

%%%%%%%%%%%%%%%%%%%%%%%%%%%%%%%%%%%%%%%%%%%%%%%%%%%%%%%%%%%%%%%%%%%%%%%%%%%%%%%
%%%%%%%%%%%%%%%%%%%%%%%%%%%%%%%%%%%%%%%%%%%%%%%%%%%%%%%%%%%%%%%%%%%%%%%%%%%%%%%
\begin{document}

\tableofcontents

%%%%%%%%%%%%%%%%%%%%%%%%%%%%%%%%%%%%%%%%%%%%%%%%%%%%%%%%%%%%%%%%%%%%%%%%%%%%%%%
\newpage
\section{Improvements to FieldStone itself}

\begin{itemize}
\item write axisymm energy eq 
\item sort our s40rts stones/theory
\item change cell type in all stones containing Q2
\item find Dave Whipp vids \url{https://www.youtube.com/@helsinkiuniversitygeodynam6511/videos}
\begin{itemize}
\item set 2: Kinematics of plate tectonics
\item set 3: forces and stresses
\item set 4: measuring stress and strain 
\item set 5: basics of elasticity
\item set 6: flexure of the lithosphere
\item set 7: heat conduction and production
\item set 8: thermal processes in the lithosphere
\item set 9: basic of fluid mechanics
\item set 10: plate-driving forces
\item set 11: brittle deformation and faulting
\item set 12: viscous deformation, strength of the lithosphere
\item set 13: climatic, geomorphic and geodynamic processes
\end{itemize}
\item scan biblio of Yuen, Olson, Turcotte, Travis
\item produce appendix about making presentations dos/donts 
\item finish 133 about double jacobian
\item write about Scott-Vogelius element, document literature. 
\item incorporate Scott-Vogelius in f120. P2P-1 on bary meshes
\item replace cal I,K, by III, KKK, FFF, QQQ in invariants 
\item check Q2Q1+graddiv vs Q2Q1 or Q2P-1. report. paper ?
\end{itemize}

%%%%%%%%%%%%%%%%%%%%%%%%%%%%%%%%%%%%%%%%%%%%%%%%%%%%%%%%%%%%%%%%%%%%%%%%%%%%%%%
\newpage
\section{Ideas for new stones}

\begin{itemize}
\item DF, FEM of 1D, 2D wave equation stone, look at becker and kaus syllabus matlab file
\item write FVM code
\item write Navier-Stokes solver FEM or FDM (streamfunction and not streamfunction based)\\
\begin{itemize}
\item Navier-Stokes Solver using Finite Differences python code 
\url{https://github.com/saadtony/uCFD/blob/master/Navier Stokes Finite Difference Method.ipynb}
\item A Simple Staggered FV Code for the Navier-Stokes Equations
\url{https://github.com/saadtony/uCFD/blob/master/Navier-Stokes FVM Staggered - Driven Cavity (CHEN6355).ipynb}
\end{itemize}
\item amg solver
\url{https://github.com/saadtony/uCFD/blob/master/Using Sparse Solvers in Python.ipynb}

\item translate ibuprofem (surface tension code)
\item redo  cookbook 'convection in 2d box w phase transitions' based on \textcite{chyu85}. 
\item write stone with phase change a la van den Berg \textcite{vava08}
\item Hollow sphere under internal pressure with analytical solution
. elastic and elasto-plastic solution  - see pdf document
\item Flow past a sphere(disc) \textcite{demj04} (2004); 
      \textcite{gafp17}). 
\item revisit stone 114 (inversion stokes+grav), use msc thesis 
\item ELEFANT:  parallel iterative solver w/ Arie. slice it with nthreads, parallel PCG
\item revive f90 code FDM stokes staggered grid, translate it to python 
(in 1427 folder)
\item Scott-Vogelius stone
\end{itemize}

%%%%%%%%%%%%%%%%%%%%%%%%%%%%%%%%%%%%%%%%%%%%%%%%%%%%%%%%%%%%%%%%%%%%%%%%%%%%%%%
\newpage
\section{BSc topics}

\begin{itemize}
\item run stone 88 (convection in 2D box) and look at configurational entropy 
(stone 137). Look at todo list at the end of stone 88.
\item compute elemental matrices of most common spaces with sympy
\item build library of mms, tester
\item biblio: for example, what are the sub-themes of subduction ? 
or mantle convection. re-organise biblio around these themes
\item structural geology: 2 and 3 layer (non)linear viscous folding
\item \fullcite{wiwh91}
\item Gravity of cuboid: Sort out the mess by \textcite{duti16} and \textcite{zhhu17}.
\item write stone for Kovasznay mms
\item symbolic calculations for FE matrices
\end{itemize}

%%%%%%%%%%%%%%%%%%%%%%%%%%%%%%%%%%%%%%%%%%%%%%%%%%%%%%%%%%%%%%%%%%%%%%%%%%%%%%%
\newpage
\section{MSc topics, guided research}

\begin{itemize}
\item Computational structural geology. Redo \fullcite{trla00} \\
Find collaborator (Alissa? Martin Drury? ...)
\item Write 3D T solver for geothermal problems. Look at/incorporate
mammoth code.
\item \textcite{vaks97} (1997) Rayleigh-Taylor problem with triangles, and remeshing.
\item produce python code for my GHOST code \textcite{thie18} 3D mesher, for Q1, Q2, P1,P2.
\item Hernlund \& Tackley pseudo annulus \fullcite{heta08}. See \textcite{josv21}
\item nonlinear viscosimeter. annulus geometry, different rheologies.
\item paper 2023 two phase flow fantom with huismans may
\item dyn topo vs true free surface... who wins?
\item build ASPECT stokes solver in python (pyamg?)
\item Biharmonic equation for stream function approach with FEM. check Schmalzl phd thesis.
check vorticity stream function formulation, see glte87.
Also check \url{https://fenicsproject.org/olddocs/dolfin/1.6.0/python/demo/documented/biharmonic/python/documentation.html} for biharmonic + disc galerkin. Can I solve Poisson eqs super fast?
\item anisotropic viscosity , see \textcite{leha08}.
redo the Rayleigh-Taylor instabilities with 
anisotropic lithospheric viscosity.
Look at the three methods listed in the other article by \textcite{leha08b}. 
Check Appendix B.2.2 or \textcite{perr19} for additional results.

\item redo \fullcite{clau82}
\item anisotropic heat conduction, see  Reddy \cite[p121]{reddybook2}, Reddy \cite[p143]{reddybook2} 
\item pressure smoothing for Q1. Get relevant literature, digest it, implement all variants in stone 12.
look at remark in \textcite{lumh24}.
\index{general}{MSc Thesis} 

\item implement a simple Newton solver and apply it to a few nonlinear  benchmarks. 

\item gravity of tesseroid , stone 98
\item look/redo/expand \fullcite{shpp13}. Pb: they have 12 orders of magnitude viscosity variations.
Use axisymmetic ? annulus?

\item in \textcite{chri84} (1984) he talks about the effective Rayleigh number 
that is evaluated using a squared strain-rate averaged viscosity:
\[
\langle \eta \rangle_2 = \int \eta \dot{\varepsilon}^2 dV/\int \dot{\varepsilon}^2 dV
\]
implement in stones and aspect ?

\item CVI 2D and 3D

\item mixing . entropy mixed with convection stone 88. compute lyapunov exponent, look at all other methods.

\item STREAM + CVI + mixing ?

\item rerun allken cookbook at high resolution with damper and divv=R

\end{itemize}

%%%%%%%%%%%%%%%%%%%%%%%%%%%%%%%%%%%%%%%%%%%%%%%%%%%%%%%%%%%%%%%%%%%%%%%%%%%%%%%
\newpage
\section{Questions I have, Things I need to (better) understand}

These are not particularly well posed. 

\begin{itemize}
\item how does FLAC work?
\item how does propagator matrix work?
\item in FEM, time derivative handled by FDM but not FEM?
\item why not ok to prescribe T on outflow boundary. check aspect manual
\item the need to underintegrate penalty term
\item pseudo-transient methods
\end{itemize}







\end{document}
%%%%%%%%%%%%%%%%%%%%%%%%%%%%%%%%%%%%%%%%%%%%%%%%%%%%%%%%%%%%%%%%%%%%%%%%%%%%%%%
