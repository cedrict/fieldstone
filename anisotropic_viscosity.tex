\begin{flushright} {\tiny {\color{gray} anisotropic\_viscosity.tex}} \end{flushright}
%~~~~~~~~~~~~~~~~~~~~~~~~~~~~~~~~~~~~~~~~~~~~~~~~~~~~~~~~~~~~~~~~~~~~~~~~~~~~~~~~~~~~~~~~~~~~~~~~~~



Following the paper by Lev and Hager (2008) \cite{leha08}, 
the anisotropic viscosity enters the equation of momentum through a 'correction'
term added to the isotropic part of the constitutive equation relating
stress and strain rate \cite{mumh02}:
\[
\sigma_{ij} = -p \delta_{ij} + 2 \eta_N \dot{\varepsilon}_{ij}  - 2(\eta_N-\eta_S)\Lambda_{ijkl}\dot{\varepsilon}_{kl} 
\]
where $\eta_N$ is the normal viscosity and $\eta_S$ is the shear viscosity. 
The fourth order tensor $\Lambda$ reflects the orientation of the directors in space, 
denoted by $\vec{n}$:
\[
\Lambda_{ijkl}=\frac{1}{2} (n_i n_k \delta_{lj} + n_j n_k \delta_{il} 
+ n_i n_l \delta_{kj} n_j n_l \delta_{ik} )
- 2 n_i n_j n_k n_l 
\]
Following \cite{modm03,mumh02}, the 'directors' are advected through the model and are 
analogous to particles. The directors are
vector-particles pointing normal to the easy-glide plane or layer,
thus defining the directions associated with $\eta_N$ and $\eta_S$. 
In each time step of the calculation, the directors are advected and rotated by the
flow, and in return determine the viscosity structure for the next time
step \cite{mumc04}.

\begin{center}
\includegraphics[width=13cm]{images/rheology/leha08}\\
{\captionfont Taken from Lev \& Hager (2008) \cite{leha08}.}
\end{center}


{\scriptsize
\Literature:
\begin{itemize}
\item 
\fullcite{rida78}
\item 
\fullcite{saab84}
\item 
\fullcite{vatb98}
\item 
\fullcite{mumh02}
\item 
\fullcite{mumc03}
\item 
\fullcite{mumc04}
\item 
\fullcite{mima04}
\item 
\fullcite{momu06}
\item 
\fullcite{mumg10}
\item 
\fullcite{muso11}
\item 
\fullcite{shmv16}
\item 
\fullcite{peka18}
\item 
\fullcite{kich20}
\end{itemize}
}







