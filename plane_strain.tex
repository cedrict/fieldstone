\begin{flushright} {\tiny {\color{gray} \tt plane\_strain.tex}} \end{flushright}
%~~~~~~~~~~~~~~~~~~~~~~~~~~~~~~~~~~~~~~~~~~~~~~~~~~~~~~~~~~~~~~~~~~~~~~~~~~~~~~~~~~~~~~~~~~~~~~~~~~

We start from the 3D strain rate tensor 
\[
\dot{\bm \varepsilon}(\vec\upnu) = 
\left(
\begin{array}{ccc}
\dot{\varepsilon}_{xx} & \dot{\varepsilon}_{xy} & \dot{\varepsilon}_{xz} \\
\dot{\varepsilon}_{yx} & \dot{\varepsilon}_{yy} & \dot{\varepsilon}_{yz} \\
\dot{\varepsilon}_{zx} & \dot{\varepsilon}_{zy} & \dot{\varepsilon}_{zz} 
\end{array}
\right)
\]

The plane strain assumption is such that the problem at hand is assumed to be 
infinite in a given direction. In the case of computational geodynamics, most 2D 
modelling is a vertical section of the crust-lithosphere-mantle
and the underlying implicit assumption is then that the orogen/rift/subduction/etc ... 
is infinite in the direction perpendicular to the screen/paper.  

Let us assume that the deformation takes place in the $x,y$-plane,
so that $w=0$ (velocity in the $z$ direction is zero) and $\partial_z \rightarrow 0$ 
(no change in the $z$ direction).
We then have $\dot{\varepsilon}_{zz}=0$ as well as $\dot{\varepsilon}_{xz}=0$ 
and $\dot{\varepsilon}_{yz}=0$, so that the strain rate tensor is 
\[
\dot{\bm \varepsilon}(\vec\upnu)=
\left( \begin{array}{ccc}
\dot{\varepsilon}_{xx} & \dot{\varepsilon}_{xy} & 0 \\
\dot{\varepsilon}_{yx} & \dot{\varepsilon}_{yy} & 0 \\
0 & 0 & 0
\end{array}\right)
\]

%------------------------------------
\subsubsection{Incompressible flow}

If the flow is incompressible then the deviatoric stress tensor is given by
\[
\bm\tau 
= 2 \eta \dot{\bm \varepsilon}^d(\vec\upnu)
= 2 \eta \left(\dot{\bm \varepsilon}(\vec\upnu) 
-\frac13 \underbrace{{\rm tr}[\dot{\bm \varepsilon}]}_{=0} 
{\bm 1}\right)
= 2 \eta \dot{\bm \varepsilon}(\vec\upnu) 
=
\left(\begin{array}{ccc}
\tau_{xx} & \tau_{xy} & 0 \\
\tau_{yx} & \tau_{yy} & 0 \\
0 & 0 & 0
\end{array}\right)
\]
One then discards the unnecessary line and column in the tensor, leaving a $2\times 2$ matrix.
Finding the principal stress components is then trivial since we have done it in 2D already.

It is important to keep in mind that the invariants we need to implement 
the rheologies are ${\III}_1({\bm \sigma})$,  ${\III}_2({\bm \tau})$ and ${\III}_3({\bm \tau})$.
By formulating our yield surfaces with pressure $p=-{\III}_1({\bm \sigma})/3$ we can then 
avoid confusion, and since the other two invariants are functions of ${\bm \tau}$ the pressure 
term does not pose any problem: simply set $\tau_{xz}$, $\tau_{yz}$ and $\tau_{zz}$ to zero in the 
equations of Section~\ref{sec:stress_invariants} and we obtain:
\begin{eqnarray}
{\III}_2({\bm \tau}) &=&\frac{1}{2}(\tau_{xx}^2 + \tau_{yy}^2 ) + \tau_{xy}^2 \\ 
{\III}_3({\bm \tau}) 
&=& \frac{1}{3} \tau_{xx} (  \tau_{xx}^2 + 3 \tau_{xy}^2 ) 
+ \frac{1}{3} \tau_{yy} (3 \tau_{xy}^2 +   \tau_{yy}^2 )   \nn\\
&=& \frac{1}{3}(  \tau_{xx}^3 + 3 \tau_{xx}\tau_{xy}^2  
+ 3 \tau_{yy} \tau_{xy}^2 +   \tau_{yy}^3 )   \nn\\
&=& \frac{1}{3}(  \tau_{xx}^3 + 3 (\tau_{xx}+\tau_{yy}) \tau_{xy}^2  +  \tau_{yy}^3 )   \nn\\
&=& \frac{1}{3}(  \tau_{xx}^3 +  \tau_{yy}^3 )  \qquad \text{since } \tau_{ii}=0 
\end{eqnarray}



The principal stresses of the deviatoric stress tensor $\bm\tau$ are given by
\begin{eqnarray}
\tau_1 &=& \frac{ \tau_{xx}+\tau_{yy}}{2} 
+ \sqrt{ \left(\frac{\tau_{xx}-\tau_{yy}}{2}\right)^2 +\tau_{xy}^2 } \nn\\
\tau_2 &=& \frac{ \tau_{xx}+\tau_{yy}}{2} 
- \sqrt{ \left(\frac{\tau_{xx}-\tau_{yy}}{2}\right)^2 +\tau_{xy}^2 } 
\end{eqnarray}
The full stress tensor is then
\[
\bm\sigma = -p \bm 1 + \bm\tau
= \left(\begin{array}{ccc}
-p+\tau_{xx} & \tau_{xy} & 0 \\
\tau_{yx} & -p+\tau_{yy} & 0 \\
0 & 0 & -p
\end{array}\right)
\]
so it remains a $3\times 3$ tensor!

However, looking at the conservation of momentum, 
\[
\vec\nabla \cdot \bm\sigma + \rho \vec g = \vec 0
\]
Given the conditions for plane-strain then $\vec g$ is likely to be in 
the $xy$-plane so that the $z$ component of the equation becomes:
\[
-\partial_z p = 0
\]
and since we have $\partial_z \rightarrow 0$ anyways this equation 
is automatically fulfilled. Then, we might as well proceed 
by considering that the stress tensor is in fact 2D as the third row/column
has no incidence. In that case the pressure is given by $p=-{\III}_1(\bm\sigma)/2$.
In the plasticity yield criterion or plastic potential we will 
need the full stress ${\bm \sigma}$ only via its first invariant (i.e. the pressure). 
The other two invariants are those of the deviatoric stress. 

Let us start from the deviatoric stress tensor:
\[
\bm\tau
=
\bm\sigma - \frac12 {\III}_1(\bm\sigma)
=
\left(\begin{array}{cc}
\sigma_{xx} & \sigma_{xy} \\ 
\sigma_{xy} & \sigma_{yy} 
\end{array}\right)
-\frac{\sigma_{xx}+\sigma_{yy}}{2} 
\left(\begin{array}{cc}
1 & 0 \\ 0 & 1
\end{array}\right)
=
\left(
\begin{array}{cc}
(\sigma_{xx}-\sigma_{yy})/2 & \sigma_{xy} \\
\sigma_{xy} & -(\sigma_{xx}-\sigma_{yy})/2
\end{array}
\right)
\]
The second invariant of the deviatoric stress tensor is then 
\[
{\III}_2(\bm\tau) = \frac12 \bm\tau:\bm\tau
= \frac12 \left( 2 (\sigma_{xx}-\sigma_{yy})^2/4 + 2 \sigma_{xy}^2 \right)
%= \left(\frac{\sigma_{xx}-\sigma_{yy}}{2} \right)^2 + \sigma_{xy}^2
\]
or
\begin{mdframed}[backgroundcolor=blue!5]
\[
{\III}_2(\bm\tau) 
= \left(\frac{\sigma_{xx}-\sigma_{yy}}{2} \right)^2 + \sigma_{xy}^2
\]
\end{mdframed}


and the effective deviatoric stress
\[
\tau_e = 
\sqrt{\left(\frac{\sigma_{xx}-\sigma_{yy}}{2} \right)^2 + \sigma_{xy}^2}
\]


Remark: Using the form of ${\III}_2(\bm\tau)$ above one arrives at  
\begin{eqnarray}
\frac{\partial {\III}_2(\bm\tau)}{\partial \sigma_{xx}} 
&=&  2 \frac{1}{2}  \frac{\sigma_{xx}-\sigma_{yy}}{2} = \tau_{xx}\nn \\
\frac{\partial {\III}_2(\bm\tau)}{\partial \sigma_{yy}} 
&=& -2 \frac{1}{2}  \frac{\sigma_{xx}-\sigma_{yy}}{2} = \tau_{yy} \nn\\
\frac{\partial {\III}_2(\bm\tau)}{\partial \sigma_{xy}} 
&=& 2 \sigma_{xy} =  2 \tau_{xy}\nn
\end{eqnarray}
which is ...wrong! One should first write the second invariant 
for the generic case of the deviatoric stress tensor (without 
assuming it is symmetric):
\[
{\III}_2(\bm\tau) = \frac12 \bm\tau:\bm\tau
= \frac12 \left( 2 (\sigma_{xx}-\sigma_{yy})^2/4 + \sigma_{xy}^2 + \sigma_{yx}^2 \right)
= \left(\frac{\sigma_{xx}-\sigma_{yy}}{2} \right)^2 + \frac12\sigma_{xy}^2 + \frac12\sigma_{yx}^2
\]
Then
\begin{eqnarray}
\frac{\partial {\III}_2(\bm\tau)}{\partial \sigma_{xx}} 
&=&  2 \frac{1}{2}  \frac{\sigma_{xx}-\sigma_{yy}}{2} = \tau_{xx} \nn\\
\frac{\partial {\III}_2(\bm\tau)}{\partial \sigma_{yy}} 
&=& -2 \frac{1}{2}  \frac{\sigma_{xx}-\sigma_{yy}}{2} =  \tau_{yy} \nn\\
\frac{\partial {\III}_2(\bm\tau)}{\partial \sigma_{xy}} 
&=&  \sigma_{xy} =  \tau_{xy} \nn\\
\frac{\partial {\III}_2(\bm\tau)}{\partial \sigma_{yx}} 
&=&  \sigma_{yx} =  \tau_{yx} \nn
\end{eqnarray}
which can be simply written as
\begin{mdframed}[backgroundcolor=blue!5]
\[
\frac{\partial {\III}_2(\bm\tau)}{\partial \bm\sigma} 
=\bm\tau
\]
\end{mdframed}



















%------------------------------------
\subsubsection{Compressible flow}
If the flow is not incompressible, then the deviatoric strain rate tensor is
\[
\dot{\bm \varepsilon}^d(\vec\upnu) 
= \dot{\bm \varepsilon}(\vec\upnu) -\frac{1}{3} {\rm tr}[\dot{\bm \varepsilon}]   {\bm 1} 
= \dot{\bm \varepsilon}(\vec\upnu) -\frac{1}{3} (\dot{\varepsilon}_{xx} +\dot{\varepsilon}_{yy}   )  {\bm 1} 
=
\left(
\begin{array}{ccc}
\frac{2}{3}\dot{\varepsilon}_{xx} -\frac{1}{3}\dot{\varepsilon}_{yy} & \dot{\varepsilon}_{xy} & 0 \\
\dot{\varepsilon}_{yx} & -\frac{1}{3}\dot{\varepsilon}_{xx} +\frac{2}{3} \dot{\varepsilon}_{yy} & 0 \\
0 & 0 & -\frac{1}{3} \dot{\varepsilon}_{xx} -\frac{1}{3}\dot{\varepsilon}_{yy}
\end{array}
\right)
\]
The deviatoric stress tensor now has the form
\[
\bm\tau=
\left(\begin{array}{ccc}
\tau_{xx} & \tau_{xy} & 0 \\
\tau_{yx} & \tau_{yy} & 0 \\
0 & 0 & \tau_{zz}
\end{array}\right)
\]

We are interested in the principal components
of the deviatoric stress tensor $\bm \tau$ so that we now have the following determinant to compute:
\[
\left|  
\begin{array}{ccc}
\tau_{xx}-\lambda & \tau_{xy} & 0 \\
\tau_{xy} & \tau_{yy}-\lambda & 0 \\
0 &0 & \tau_{zz} -\lambda
\end{array}
\right|
=0
\]
which yields the following characteristic equation:
\[
(\tau_{zz} -\lambda)(\lambda-\tau_1)(\lambda-\tau_2) =0
\]
where $\tau_{1,2}$ have previously been obtained in the 2D case:
\begin{eqnarray}
\tau_1 
&=& \frac{ \tau_{xx}+\tau_{yy}}{2} 
+ \sqrt{ \left(\frac{\tau_{xx}-\tau_{yy}}{2}\right)^2 +\tau_{xy}^2 } 
\nn\\
\tau_2 &=& \frac{ \tau_{xx}+\tau_{yy}}{2} 
- \sqrt{ \left(\frac{\tau_{xx}-\tau_{yy}}{2}\right)^2 +\tau_{xy}^2 } 
\end{eqnarray}
We have 
\begin{eqnarray}
\tau_{xx}+\tau_{yy} &=& 2\eta \frac13 (\dot{\varepsilon}_{xx} + \dot{\varepsilon}_{yy}) \nn\\
\tau_{xx}-\tau_{yy} &=& 2\eta (\dot{\varepsilon}_{xx} - \dot{\varepsilon}_{yy})
\end{eqnarray}
Then 
\begin{eqnarray}
\tau_1 
&=& \frac{ \tau_{xx}+\tau_{yy}}{2} 
+ \sqrt{ \left(\frac{\tau_{xx}-\tau_{yy}}{2}\right)^2 +\tau_{xy}^2 } 
\nn\\
&=& \eta \frac13 (\dot{\varepsilon}_{xx} + \dot{\varepsilon}_{yy})
+ \eta \sqrt{ (\dot{\varepsilon}_{xx} - \dot{\varepsilon}_{yy})^2 +4  \dot{\varepsilon}_{xy}^2  } 
\nn\\
\tau_2 
&=& \frac{ \tau_{xx}+\tau_{yy}}{2} 
- \sqrt{ \left(\frac{\tau_{xx}-\tau_{yy}}{2}\right)^2 +\tau_{xy}^2 } \nn\\
&=& \eta \frac13 (\dot{\varepsilon}_{xx} + \dot{\varepsilon}_{yy})
- \eta \sqrt{ (\dot{\varepsilon}_{xx} - \dot{\varepsilon}_{yy})^2 +4  \dot{\varepsilon}_{xy}^2  } 
\end{eqnarray}
It does not look like it is going to simplify down the road ... Also, 
the third eigenvalue/principal stress remains and it is not clear whether it is 
larger or smaller than the other two.
The 3D framework is then probably the most appropriate.











Let us now turn to the second invariant of the deviatoric strain rate 
(see Eq.~\eqref{eq:I2epsd}):
\begin{eqnarray}
{\III}_2(\dot{\bm{\varepsilon}}^d)
&=& \frac{1}{2} \dot{\bm{\varepsilon}}^d:\dot{\bm{\varepsilon}}^d \\
&=& \frac{1}{2} \left[ (\dot{\varepsilon}_{xx}^d)^2 + (\dot{\varepsilon}_{yy}^d)^2 + (\dot{\varepsilon}_{zz}^d)^2   \right] 
+ (\dot{\varepsilon}_{xy}^d)^2  
+ (\dot{\varepsilon}_{xz}^d)^2  
+ (\dot{\varepsilon}_{yz}^d)^2  
\end{eqnarray}
But there is also an expression for ${\III}_2(\dot{\bm{\varepsilon}}^d)$ directly as a function of the $\dot{\bm\varepsilon}_{ij}$ components 
(see Eq.~\eqref{eq:I2epsd}):
\begin{eqnarray}
{\III}_2(\dot{\bm{\varepsilon}}^d)
&=& \frac{1}{6} \left[ (\dot{\varepsilon}_{xx}-\dot{\varepsilon}_{yy})^2 
+ (\dot{\varepsilon}_{yy}-\dot{\varepsilon}_{zz})^2 
+ (\dot{\varepsilon}_{xx}-\dot{\varepsilon}_{zz})^2 \right] 
+ \dot{\varepsilon}_{xy}^2 + \dot{\varepsilon}_{xz}^2 + \dot{\varepsilon}_{yz}^2 \\
&=& 
\frac{1}{6} \left[ (\dot{\varepsilon}_{xx}-\dot{\varepsilon}_{yy})^2 
+ (\dot{\varepsilon}_{yy})^2 
+ (\dot{\varepsilon}_{xx})^2 \right] 
+ \dot{\varepsilon}_{xy}^2 \\
&=& \frac{1}{6} \left[ \dot{\varepsilon}_{xx}^2 
-2\dot{\varepsilon}_{xx}\dot{\varepsilon}_{yy}
+\dot{\varepsilon}_{yy}^2 
+ \dot{\varepsilon}_{yy}^2 
+ \dot{\varepsilon}_{xx}^2 \right] 
+ \dot{\varepsilon}_{xy}^2 \\
&=& \frac{1}{6} \left[ 
2\dot{\varepsilon}_{xx}^2 
-2\dot{\varepsilon}_{xx}\dot{\varepsilon}_{yy}
+2\dot{\varepsilon}_{yy}^2 
\right] 
+ \dot{\varepsilon}_{xy}^2 \\
&=& 
\frac{1}{3} \left[ 
\dot{\varepsilon}_{xx}^2 
-\dot{\varepsilon}_{xx}\dot{\varepsilon}_{yy}
+\dot{\varepsilon}_{yy}^2 
\right] 
+ \dot{\varepsilon}_{xy}^2 
\end{eqnarray}

{\color{darkgray}
If we now do things the old/wrong(?) way, one would start directly from the 2D strain rate tensor 
\[
\dot{\bm \varepsilon} = 
\left(
\begin{array}{cc}
\dot{\varepsilon}_{xx} & \dot{\varepsilon}_{xy}  \\
\dot{\varepsilon}_{yx} & \dot{\varepsilon}_{yy}  
\end{array}
\right)
\]
The deviatoric strain rate tensor is then logically defined as 
\[
\dot{\bm \varepsilon}^d 
= \dot{\bm \varepsilon} -\frac{1}{2} Tr[\dot{\bm \varepsilon}]   {\bm 1} 
= \dot{\bm \varepsilon} -\frac{1}{2} (\dot{\varepsilon}_{xx} +\dot{\varepsilon}_{yy}   )  {\bm 1} 
\]
or,
\[
\dot{\bm \varepsilon}^d = 
\left(
\begin{array}{cc}
\frac{1}{2}\dot{\varepsilon}_{xx} -\frac{1}{2}\dot{\varepsilon}_{yy} & \dot{\varepsilon}_{xy}  \\
\dot{\varepsilon}_{yx} & -\frac{1}{2}\dot{\varepsilon}_{xx} +\frac{1}{2} \dot{\varepsilon}_{yy}  \\
\end{array}
\right)
\]
Let us now turn to the second invariant of the deviatoric strain rate 
(see Section 3.21 in fieldstone)
\begin{eqnarray}
{\III}_2(\dot{\bm{\varepsilon}}^d)
&=& \frac{1}{2} \dot{\bm{\varepsilon}}^d:\dot{\bm{\varepsilon}}^d \nn\\
&=& \frac{1}{2}[ (\frac{1}{2}\dot{\varepsilon}_{xx} -\frac{1}{2}\dot{\varepsilon}_{yy})^2 + (-\frac{1}{2}\dot{\varepsilon}_{xx} +\frac{1}{2} \dot{\varepsilon}_{yy})^2  ] + \dot{\varepsilon}_{xy}^2 \nn\\
&=& \frac12 [ \frac14 (2\dot{\varepsilon}_{xx}^2  -4 \dot{\varepsilon}_{xx}\dot{\varepsilon}_{yy} +2\dot{\varepsilon}_{yy}^2 )  ] + \dot{\varepsilon}_{xy}^2 \nn\\
&=& \frac14 [ \dot{\varepsilon}_{xx}^2  -2 \dot{\varepsilon}_{xx}\dot{\varepsilon}_{yy} +\dot{\varepsilon}_{yy}^2   ] + \dot{\varepsilon}_{xy}^2 
\end{eqnarray}
which is not the same as the previous expression! 
}



