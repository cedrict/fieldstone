\chapter{[WIP] The Propagator matrix method (PMM) in geodynamics} 

\begin{flushright} {\tiny {\color{gray} \tt chapter\_propagator\_matrix.tex}} \end{flushright}
%~~~~~~~~~~~~~~~~~~~~~~~~~~~~~~~~~~~~~~~~~~~~~~~~~~~~~~~~~~~~~~~~~~~~~~~~~~~~~~~~~~~~~~~~~~~~~~~~~~

\begin{verbatim}
giba66 <--- haoc81(appendix)  <--- riha84  <-- hacr85 
            haoc78?                riff84
            haoc79?                hage84

\end{verbatim}

%==============================================================================
\section{Literature}

\begin{itemize}


%-----------------------------
\item \fullcite{giba66}

An early article about the PMM, often cited. Not 
so useful since it is not concerned with the Stokes equations
of mantle dynamics.

%-----------------------------
\item \fullcite{haoc78}

3d formulation based on spherical harmonics. 

\begin{displayquote}
{\color{darkgray}
Kinematic models of the large scale flow in the mantle accompanying the observed
plate motions are calculated by neglecting thermal buoyancy forces. 
The large scale flow
is therefore determined by the mass flux imposed by the moving plates. The energy and
momentum equations decouple, and with the assumption of a radially symmetric 
Newtonian viscosity, the flow accompanying the plate motions can be obtained using 
harmonic analysis and propagator matrices.
[...]
We also assume that viscosity
is a function of radius only. This assumption may well neglect some 
important effects since the mantle temperature distribution is not 
radially symmetric, and viscosity is highly dependent upon temperature. 
The flow is assumed to be incompressible.
}
\end{displayquote}

%-----------------------------
\item \fullcite{haoc79}

Very similar in scope and technique to their 1978 paper. 

%-----------------------------
\item \fullcite{haoc81}

This article is cited a lot when people refer to the PMM. 
It presents in its appendix the PMM for both 2d and 3d cases.

\begin{displayquote}
{\color{darkgray}
The procedure for solving these equations [mass \& momentum
conservation equations] (givenin the appendix)is to separate 
the vertical and horizontal dependences
of the flow parameters. The total variation of these parameters, 
assumed periodic with horizontal period L, is expressed
in terms of a Fourier series, with each term containing a 
coefficient varying with depth multiplied by a harmonic function
of the horizontal coordinate.The flow equations can be reduced 
to a coupled set of first-order differential equations in
depth. These are solved analytically by using the propagator
matrix technique \cite{giba66}.
}
\end{displayquote}

%-----------------------------
\item \fullcite{riha84}

Refers to \textcite{haoc81} (1981). 

\begin{displayquote}
{\color{darkgray}
A familiar and useful property of the propagator matrix
formulation is that solution vectors can be propagated
through a seriesof differentmaterial layers by simply forming
the product of the individual layer matrices [...].
Therefore changes in viscosity (and density) with depth are
easily incorporated into this formalism.
}
\end{displayquote}


%-----------------------------
\item \fullcite{riff84}

3d, refers to \textcite{haoc78} (1978).

%-----------------------------
\item \fullcite{hage84}

%-----------------------------
\item \fullcite{hacr85}

%-----------------------------
\item \fullcite{ribe18}

Ribe cites Gantmacher\footnote{\url{https://www.maths.ed.ac.uk/~v1ranick/papers/gantmacher1.pdf}} 
as all others, but actually provides 
more information: vol I, p 120, Eq.~(53).
The two pages in the book rely heavily on \cite{haoc81}.
We find the same content in \cite{ribe07}. 

%-----------------------------
\item \fullcite{zhon96}

\begin{displayquote}
{\color{darkgray}
Analytic solutions for 2-D incompressible Stokes' flow with lateral variations in
viscosity have been developed with a Green’s function method and matrix propagator
techniques. The analytic solutions are developed based on the observation that lateral
variations in viscosity only result in mode coupling between viscosity and buoyancy in
the horizontal dimension and not in the vertical dimension.
}
\end{displayquote}

%-----------------------------
\item \fullcite{mawo98}

%-----------------------------
\item \fullcite{bugo94}

%-----------------------------
\item \fullcite{tosi_thesis}

\begin{displayquote}
{\color{darkgray}
After introducing the partial differential equations that govern the flow in the
mantle and the gravitational potential, we briefly review the technique of the
matrix propagator which is generally employed to solve such equations in a
spherical geometry under the approximation of laterally homogeneous viscosity.
The method allows us to obtain an analytical solution in terms of the spherical
harmonic expansion of the field quantities. In the presence of lateral viscosity
variations, this method is no longer applicable.
}
\end{displayquote}


%-----------------------------
\item \fullcite{lezh08}

%-----------------------------
\item \fullcite{lizh13}

\begin{displayquote}
{\color{darkgray}
For compressible convection with depth-dependent density and
possibly other depth-dependent thermodynamic properties, propagator 
matrix method is more effective. The propagator matrix
method has been used to obtain analytic solution of the Stokes flow
problem for incompressible (\textcite{haoc81}, 1981) and com-
pressible (\textcite{lezh08}, 2008) models. With a stream-function
and vorticity formulation, \textcite{jamc80} (1980) employed the
propagator matrix method for marginal stability analysis with heat
flux boundary conditions. \textcite{bugo94} used the propagator
matrix method for marginal stability analysis for incompressible
flows with depth-dependent viscosity. In this study, we develop a
new implementation of propagator matrix method for marginal stability 
analysis for both incompressible and compressible flows with
free-slip and isothermal boundary conditions. Our implementation
is based on a stress-velocity formulation which is similar to that
in \textcite{lezh08} (2008), but we also incorporate the linearized
energy equation. The setup of propagator matrix and the solution
procedure are discussed in Appendix A.
}
\end{displayquote}



%-----------------------------
\item \fullcite{moko18}

\begin{displayquote}
{\color{darkgray}
In this 2-D Cartesian geometry, we expand field variables in the Fourier series assuming the horizontal invariance of material properties.
}
\end{displayquote}

{\color{red} redo in a stone?}



%-----------------------------
\item \fullcite{qizp18}

\end{itemize}

%==============================================================================
\section{Theory in 2d}


%==============================================================================
\section{Theory in 3d}
