\begin{flushright} {\tiny {\color{gray} perzyna\_chpe15.tex}} \end{flushright}
%~~~~~~~~~~~~~~~~~~~~~~~~~~~~~~~~~~~~~~~~~~~~~~~~~~~~~~~~~~~~~~~~~~~~~~~~~~~~~~~~~~~~~~~~~~~~~~~~~~

\subsection{Dissecting Choi \& Petersen (2015)}

In the paper the author state:
\begin{displayquote}
{\color{darkgray}
In this paper, we confirm that using an associated flow rule is a
straightforward way of acquiring shear band orientations tightly
bound around the Coulomb angle in conventional numerical tectonic
models. We further provide a simple analysis of why an associated
flow rule constrains shear band orientations more tightly than a non-
associated counterpart.}
\end{displayquote}


The original paper \cite{chpe15} is in 2D and focuses on the Mohr-Coulomb criterion. 
The authors state that the mass conservation equation should be 
\[
\frac{\partial v_x}{\partial x}
+
\frac{\partial v_y}{\partial y}
=
R=2 \sin \psi \; \dot{\varepsilon}^p
\]
where where $R$ is the dilation rate, $\Psi$ is the dilation angle and
$\dot{\varepsilon}^p$ is the square root of the second invariant of the deviatoric plastic strain rate tensor.

After multiple reads of their paper, I originally had many questions:
\begin{itemize}
\item where does this dilation rate $R$ come from ? 
\item after reading \textit{many} papers or textbooks on plasticity 
I cannot see a factor 2 in an equation anymore without re-deriving 
it from scratch with a coherent set of notations (preferably mine in fieldstone). 
\item is this relationship still valid in 3D?
\item is it the same term for Drucker-Prager ?
\end{itemize}
Time to investigate ...
\vspace{1cm}

Let us first look at their Eq.~(3)  in which  
the MC yield function is given by the function $f$:
\[
f = \sigma_1 - N_\phi \sigma_3 - 2 \sqrt{N_\phi} c 
\]
where $\sigma_1$ and $\sigma_3$ are the greatest and the least principal stress, $N_\phi=(1+\sin \phi)/(1-\sin \phi)$.
This is a somewhat unusual formulation in the geodynamics community.

Let us then start with the MC yield criterion\footnote{\url{https://en.wikipedia.org/wiki/Mohr-Coulomb_theory}}
\begin{equation}
\tau_m = \sigma_m \sin \phi + c \cos \phi  \label{eq:mccrittt}
\end{equation}
which means that compression is assumed to be positive (the opposite as in fieldstone) and where $\tau_m$ is the magnitude of the shear stress, 
$\sigma_m$ is the normal stress, $c$ is the intercept of the failure envelope with the $\tau$ axis, 
and $\phi$ is the slope of the failure envelope. 
The quantity $c$ is called the cohesion and the angle $\phi$ is called the angle of internal friction.
We have
\[
\sigma_m=\frac{1}{2}(\sigma_1+\sigma_3) 
\]
and 
\[
\tau_m = \frac{1}{2}(\sigma_1-\sigma_3)
\]
Inserting these into Eq.~\eqref{eq:mccrittt}
\begin{equation}
 \frac{1}{2}(\sigma_1-\sigma_3) = \frac{1}{2}(\sigma_1+\sigma_3)  \sin \phi + c \cos \phi 
\end{equation}
which can be reworked as follows:
\[
\sigma_1 - \frac{1 + \sin\phi}{1-\sin\phi} \sigma_3 - 2 \frac{\cos \phi}{1-\sin\phi} c = 0
\]
The third term can further be modified as follows:
\[
\frac{\cos \phi}{1-\sin\phi}
=\frac{\sqrt{1-\sin^2 \phi}}{\sqrt{(1-\sin\phi)^2}}
=\frac{\sqrt{(1-\sin \phi)(1+\sin\phi)}}{\sqrt{(1-\sin\phi)^2}}
=\sqrt{
\frac{1+\sin\phi}{1-\sin\phi}
}
\]
Finally, we define $N_\phi$ as follows 
\[
N_\phi=\frac{1+\sin \phi}{1-\sin\phi}
\]
so that the yield condition becomes:
\[
\sigma_1 - N_\phi \sigma_3 - 2 \sqrt{N_\phi} \; c = 0
\]
which is Eq.~3 of the article by Choi \& Petersen \cite{chpe15}.

They also define the plastic potential as 
\[
g=\sigma_1 -N_\psi \sigma_3  =\tau_m - \sigma_m \sin\psi
\]

We start again from the M-C criterion (in this case $\sigma_2$ replaces $\sigma_3$):
\begin{eqnarray}
\frac{1}{2}(\sigma_1-\sigma_2) &=& - \frac{1}{2}(\sigma_1+\sigma_2)  \sin \phi + c \cos \phi  
\end{eqnarray}
In the case of incompressible flow I have established in Section~\ref{ss:plane_strain} that 
\begin{eqnarray}
\frac{\sigma_1+\sigma_2}{2} &=& \frac{\sigma_{xx}+\sigma_{yy}}{2} = \frac12 {\III}_1(\bm\sigma)\\
\frac{\sigma_1-\sigma_2}{2} &=& \sqrt{ \left(\frac{\sigma_{xx}-\sigma_{yy}}{2}\right)^2 + \sigma_{xy}^2  }
=\sqrt{ {\III}_2(\bm\tau)  }
\end{eqnarray}
so that we now have
\[
{\cal \FFF}^{\text{\tiny MC}} 
= \frac{1}{2}{\III}_1(\bm\sigma) \sin\phi  + 
\sqrt{ {\III}_2(\bm\tau)  }
-c \cos \phi 
\]
and then the plastic potential $\QQQ$ is given by
\[
{\QQQ}^{\text{\tiny MC}}=\frac{1}{2}{\III}_1(\bm\sigma) \sin\psi  + \sqrt{ {\III}_2(\bm\tau)  }
\]
We will need $\partial {\QQQ}/\partial \bm\sigma$.
By applying the chain rule we can write
\begin{eqnarray}
\frac{\partial {\QQQ}}{\partial \bm\sigma} 
&=&
\frac{\partial {\QQQ}}{\partial {\III}_1(\bm\sigma)} 
\frac{\partial {\III}_1(\bm\sigma)}{\partial \bm\sigma} 
+
\frac{\partial {\QQQ}}{\partial {\III}_2(\bm\tau)} 
\frac{\partial {\III}_2(\bm\tau)}{\partial \bm\sigma} \nn\\
&=&
\frac{\partial {\QQQ}}{\partial {\III}_1(\bm\sigma)} 
\frac{\partial {\III}_1(\bm\sigma)}{\partial \bm\sigma} 
+
\frac{\partial {\QQQ}}{\partial \sqrt{{\III}_2(\bm\tau)}} 
\frac{\partial \sqrt{ {\III}_2(\bm\tau)}   }{\partial {\III}_2(\bm\tau)} 
\frac{\partial {\III}_2(\bm\tau)}{\partial \bm\sigma} \nn\\
&=&
\frac12 \sin\psi \; {\bm 1} + \frac{1}{2 \sqrt{ {\III}_2(\bm\tau)}} 
\bm\tau
\end{eqnarray}

Ultimately we would like to be able to write $\dot{\bm \varepsilon}^{vp} = \bm\tau /(2\eta)$
where $\eta$ is the 'viscoplastic' viscosity. However, as opposed to Zienkiewicz (1975) in the 
previous section, the term $\partial {\QQQ}/\partial \bm\sigma$
is not directly/only proportional to the deviatoric stress $\bm\tau$ and we have instead:

\begin{eqnarray}
\dot{\bm \varepsilon}^{vp} 
&=& \gamma \big\langle \phi( \FFF^{\text{\tiny MC}} )  \big\rangle 
\left(\frac12 \sin\psi \;  {\bm 1} + \frac{1}{2 \sqrt{{\III}_2(\bm \tau)}} {\bm \tau} \right)
\end{eqnarray}
Right away we note that the strain rate tensor above is not deviatoric, i.e. the flow
is not incompressible.
Rather conveniently, the M-C criterion in plane strain can also 
be cast $\FFF^{\text{\tiny MC}} = \sqrt{{\III}_2(\bm \tau)} -Y = \tau_e - Y$ as in 
the von Mises case, albeit with 
$Y= -\frac12 {\III}_1(\bm\sigma) \sin\phi + c\cos\phi$.
Assuming $\phi(x)=x$ for convenience here and the argument of the brackets is positive, 
\begin{eqnarray}
\dot{\bm \varepsilon}^{vp} 
&=& \gamma \left(\sqrt{{\III}_2(\bm\tau)} - Y \right)
\left(\frac12 \sin\psi \;  {\bm 1} + \frac{1}{2 \sqrt{{\III}_2(\bm \tau)}} {\bm \tau} \right) \nn\\
&=& \gamma \left(\sqrt{{\III}_2(\bm\tau)} - Y \right)
\frac12 \sin\psi \;  {\bm 1} 
+  
\gamma \left(\sqrt{{\III}_2(\bm\tau)} - Y \right)
\frac{1}{2 \sqrt{{\III}_2(\bm \tau)}} {\bm \tau}  \nn\\
&=& \gamma \left(\tau_e - Y \right)
\frac12 \sin\psi \;  {\bm 1} 
+  
\gamma \left(\tau_e - Y \right)
\frac{1}{2 \tau_e} {\bm \tau}  \label{eq:evpcp15}
\end{eqnarray}
If we follow the procedure of Zienkiewicz (1975) of the previous section, then the deviatoric part
of the equation above would yield a viscosity
\begin{eqnarray}
\eta 
&=& \frac{Y}{2  \dot\varepsilon_e^{vp}}
+ \frac{1 }{\gamma } 
\qquad \Rightarrow \qquad 
\tau_e = 2\eta \dot\varepsilon_e^{vp} = Y + \gamma^{-1} 2  \dot\varepsilon_e^{vp}
\qquad 
\Rightarrow
\qquad
\tau_e - Y = \gamma^{-1} 2  \dot\varepsilon_e^{vp}
\end{eqnarray}
If we insert this in Eq.~\eqref{eq:evpcp15}: 
\begin{eqnarray}
\dot{\bm \varepsilon}^{vp} 
&=& \gamma 
\gamma^{-1} 2  \dot\varepsilon_e^{vp}
\frac12 \sin\psi \;  {\bm 1} 
+  
\frac{1}{2\eta}
{\bm \tau}  \nn\\
&=& 
\dot\varepsilon_e^{vp}
\sin\psi \;  {\bm 1} 
+  
\frac{1}{2\eta}
{\bm \tau}  
\end{eqnarray}
Assuming that the total strain rate is the sum of the strain rates 
associated to the various deformation mechanisms, and that all
other deformation mechanisms are deviatoric, then 
\[
div (\vec\upnu)=
\dot{\varepsilon}_{xx}
+
\dot{\varepsilon}_{yy}
=
2  \dot\varepsilon_e^{vp}
\sin\psi 
\]
This is identical to the dilation rate of \textcite{chpe15}!

For implementation details in a FEM code, please check \stone~39. 

%--------------------------------------------------------
\subsection{my take on this in 3D for Drucker-Prager}

I have established in Section~\ref{sec:dpcriterion} that in the general 3D case
\begin{equation}
\FFF^{\text{\tiny DP}}= \alpha(\phi,c) {\III}_1({\bm \sigma}) + \sqrt{ {\III}_2({\bm \tau})  } + k(\phi,c) 
=\sqrt{  {\III}_2({\bm \tau})  } -Y 
\end{equation}
with $\alpha$ and $k$ being functions of the cohesion $c$ and angle of friction $\phi$ 
(but not from the stress). Then the plastic potential is
\begin{equation}
\QQQ^{\text{\tiny DP}}= \alpha(\psi,c) {\III}_1({\bm \sigma})   +  \sqrt{  {\III}_2({\bm \tau})  } 
\end{equation}
where $\psi$ is the dilation angle.
We then have
\begin{eqnarray}
\frac{\partial \QQQ}{\partial \bm\sigma} 
&=&
\frac{\partial \QQQ}{\partial {\III}_1(\bm\sigma)} 
\frac{\partial {\III}_1(\bm\sigma)}{\partial \bm\sigma} 
+
\frac{\partial \QQQ}{\partial \sqrt{{\III}_2(\bm\tau)}} 
\frac{\partial \sqrt{ {\III}_2(\bm\tau)}   }{\partial {\III}_2(\bm\tau)} 
\frac{\partial {\III}_2(\bm\tau)}{\partial \bm\sigma} \nn\\
&=&
\alpha(\psi,c) \; {\bm 1} + \frac{1}{2 \sqrt{ {\III}_2(\bm\tau)}} 
\bm\tau
\end{eqnarray}
Then 
\begin{eqnarray}
\dot{\bm \varepsilon}^{vp} 
&=& \gamma \left(\sqrt{{\III}_2(\bm\tau)} - Y \right)
\left(\alpha(\psi,c) \;  {\bm 1} + \frac{1}{2 \sqrt{{\III}_2(\bm \tau)}} {\bm \tau} \right) \nn\\
&=& \gamma \left(\tau_e - Y \right)
\alpha(\psi,c) \;  {\bm 1} 
+  
\gamma \left(\tau_e - Y \right)
\frac{1}{2 \tau_e} {\bm \tau}  
\end{eqnarray}
Using again $\tau_e - Y = \gamma^{-1} 2  \dot\varepsilon_e^{vp}$ as in the 2D case 
we arrive finally
\[
div (\vec\upnu)=
{\rm tr}[\dot{\bm \varepsilon}^{vp}]=
\dot{\varepsilon}_{xx}
+
\dot{\varepsilon}_{yy}
+
\dot{\varepsilon}_{zz}
=
6 \alpha(\psi,c) \dot\varepsilon_e^{vp}
\]
Since $k$ does not depend on stress, the only difference between the associative
and the non-associative case is whether $\phi=\psi$ or not.

%--------------------------------------
\subsection{my take on this in 3D for MC}

I have established in fieldstone that in the general 3D case
\begin{equation}
\FFF^{\text{\tiny MC}}=\frac{1}{3} {\III}_1({\bm \sigma}) \sin \phi  + 
\sqrt{  {\III}_2({\bm \tau})  } \left( \cos \theta_{\rm L}(\bm\tau) - 
\frac{1}{\sqrt{3}} \sin \theta_{\rm L}(\bm\tau)  \sin \phi \right) - c \cos \phi
\end{equation}
Note that since $p=-{\III}_1(\bm\sigma)/3$ then we recover the usual '$p\sin\phi+c \cos\phi$'.

Following Eq.~(4) of the paper the plastic potential would be given by 
\[
\QQQ^{\text{\tiny MC}} 
=\frac{1}{3} {\III}_1({\bm \sigma}) \sin \psi  + 
\sqrt{  {\III}_2({\bm \tau})  } 
\left( \cos \theta_{\rm L}(\bm\tau) -\frac{1}{\sqrt{3}} \sin \theta_{\rm L}(\bm\tau) \sin \psi \right) 
\]
The visco-plastic strain rate would then write
\[
\dot{\bm\varepsilon}^{vp} = \gamma \langle \phi(\FFF^{\text{\tiny MC}}) \rangle 
\frac{\partial \QQQ^{\text{\tiny MC}}}{\partial \bm\sigma}
\]
We have established that 
\begin{eqnarray}
\frac{\partial \QQQ}{\partial \bm\sigma}
&=& 
C_1  \frac{\partial {\III}_1(\bm\sigma)}{\partial \bm\sigma} 
+
C_2  \frac{\partial {\III}_2(\bm\tau)}{\partial \bm\sigma} 
+
C_3  \frac{\partial {\III}_3(\bm\tau)}{\partial \bm\sigma}\nn\\
&=& C_1 {\bm 1} + C_2  {\bm \tau} + C_3 \left( \bm\tau\cdot\bm\tau -\frac23 {\III}_2(\bm\tau){\bm 1} \right)
\end{eqnarray}
and in the case of the Mohr-Coulomb criterion:
\begin{eqnarray}
C_1^{\text{\tiny MC}} &=& \frac13 \sin\phi  \\ 
C_2^{\text{\tiny MC}} 
&=& 
\frac{1}{2 \sqrt{ {\III}_2(\bm\tau)}   }   
\cos \theta_{\rm L}
\left[
(1 +  2\tan \theta_{\rm L}   \tan 3\theta_{\rm L})
+\frac{1}{\sqrt{3}} \sin\phi
( 2\tan 3\theta_{\rm L} - \tan\theta_{\rm L})
\right]
\nn\\
C_3^{\text{\tiny MC}} 
&=& 
\frac{
\sqrt{3}\sin\theta_{\rm L}
+
 \sin \phi \cos \theta_{\rm L}
}{2 {\III}_2({\bm \tau}) \cos 3\theta_{\rm L}}
\end{eqnarray}



\[
\dot{\bm\varepsilon}^{vp} = \gamma \langle \phi(\FFF) \rangle 
\left(
C_1 {\bm 1} + C_2  {\bm \tau} + C_3 \left( \bm\tau\cdot\bm\tau -\frac23 {\III}_2(\bm\tau){\bm 1} \right)
\right)
\]
Assuming brackets ok, and $\phi(x)=x^n$:
\[
\dot{\bm\varepsilon}^{vp} = \gamma (\sqrt{{\III}_2(\bm\tau)} - Y)^n 
\left(
C_1 {\bm 1} + C_2  {\bm \tau} + C_3 \left( \bm\tau\cdot\bm\tau -\frac23 {\III}_2(\bm\tau){\bm 1} \right)
\right)
\]
Taking the deviatoric part of this:
\[
\dot{\bm\varepsilon}^{vp,d} 
= \gamma (\sqrt{{\III}_2(\bm\tau)} - Y)^n 
\left(
 C_2  {\bm \tau} + C_3 \left( \bm\tau\cdot\bm\tau -\frac23 {\III}_2(\bm\tau){\bm 1} \right)
\right)
\]
so I cannot find a \underline{scalar} $\eta$ such that 
\[
\dot{\bm\varepsilon}^{vp,d} 
=\frac{1}{2\eta} {\bm \tau}
\]
I am STUCK here?!




































