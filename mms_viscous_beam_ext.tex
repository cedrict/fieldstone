\begin{flushright} {\tiny {\color{gray} \tt mms\_viscous\_beam\_ext.tex}} \end{flushright}
%~~~~~~~~~~~~~~~~~~~~~~~~~~~~~~~~~~~~~~~~~~~~~~~~~~~~~~~~~~~~~~~~~~~~~~~~~~~~~~~~~~~~~~~~~~~~~~~~~~

The domain is a Cartesian box of size $L_x \times L_y$. 
Velocity $-u_0$ is applied on the left boundary and 
velocity $+u_0$ is applied on the right boundary. 
Bottom and top boundaries are left free. 
If no vertical velocity is prescribed anywhere there is an obvious nullspace 
in the solution which is problematic (numerically of course, but also 
because the solution is then not unique). One might want to set $v=0$ at $y=L_y/2$
on each side for example. 
The solution to this problem (incompressible Stokes equations) is given by
\begin{eqnarray}
u(x,y)&=&2u_0(x/L_x-1/2)\\
v(x,y)&=&-2 u_0 L_y/L_x (y/L_y-1/2)
\end{eqnarray}
in the absence of gravity. The strain rate tensor is then:
\[
\dot{\bm \varepsilon} =
\left(
\begin{array}{cc}
\dot{\varepsilon}_{xx} & \dot{\varepsilon}_{xy} \\
\dot{\varepsilon}_{xx} & \dot{\varepsilon}_{yy} 
\end{array}
\right)
=
\left(
\begin{array}{cc}
2 u_0 /Lx & 0 \\
0 & -2 u_0 /L_x 
\end{array}
\right)
\]
and we see that the flow is indeed incompressible as the trace 
of the strain rate tensor is zero. 

The momentum equation is 
\[
-\vec\nabla p + \vec\nabla \cdot (2 \eta \dot{\bm\varepsilon}) = \rho \vec g
\]
where the viscosity $\eta$ is constant in space. 
If gravity is set to zero, we obtain:
\begin{eqnarray}
-\frac{\partial p}{\partial x} &=& 0 \\
-\frac{\partial p}{\partial y} &=& 0 
\end{eqnarray}
since the strain rate is constant in space and the divergence operator applied to it returns 
the zero tensor. We there fore can conclude that pressure should be constant. 

Since the top and bottom boundaries are free, we have ${\bm \sigma}\cdot \vec{n} = \vec{0}$ on these.
The stress tensor is given by ${\bm \sigma} = - {\bm 1} + 2 \eta \dot{\bm \varepsilon}$ and the normal on the 
top is $\vec{n}=(0,+1)$ so that on the top boundary we have
\[
- p + 2 \eta \dot{\epsilon}_{yy} = 0
\]
or, 
\[
p= 2 \eta \dot{\epsilon}_{yy} 
\]
Note that using the bottom boundary with $\vec{n}=(0,-1)$ yields the same result.



