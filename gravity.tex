WORK in PROGRESS. DUH.

We start from the Poisson equation for the gravity potential:
\begin{equation}
\Delta U = 4\pi \rho {\cal G}
\end{equation}
As a consequence, inside a domain where $\rho=0$, the equation becomes $\Delta U=0$.

Let us assume that the spherical coordinates are appropriate for the problem at hand, and that 
the potential can be decomposed as follows:
\[
U(r,\theta,\phi) = U_r(r) U\bot(\theta,\phi)
\]
The full Laplacian operator in spherical coordinates is given 
by\footnote{\url{https://en.wikipedia.org/wiki/Laplace_operator}}:
\[
\Delta U 
= 
\underbrace{\frac{1}{r^2} \frac{\partial }{\partial r}\left(r^2 \frac{\partial U}{\partial r}\right)}_{\Delta_r}
+
\underbrace{
\frac{1}{r^2 \sin\theta} \frac{\partial }{\partial \theta} \left(\sin\theta \frac{\partial U}{\partial \theta} \right) 
+
\frac{1}{r^2 \sin^2\theta} \frac{\partial^2 U }{\partial \phi^2}
}_{\Delta_\bot}
\]
we then have:
\[
(\Delta_r + \Delta_\bot)(U_r U_\bot)=0
\]
i.e., 
\[
U_\bot \Delta_r U_r + U_r \Delta_\bot U_\bot=0
\]
Assuming $U_\bot=\sum_l\sum_m U_{lm}Y_{lm}$, knowing that spherical 
harmonics functions verify
\[
r^2 \Delta_\bot Y_l^m(\theta,\phi) = -l(l+1) Y_l^m (\theta,\phi)
\]
and assuming for now that the problem at hand is 1st degree (l=1), then 
\[
\Delta_\bot Y_l^m(\theta,\phi) = -\frac{2}{r^2} Y_l^m (\theta,\phi)
\]
and then
\[
\Delta_r U_r - U_r \frac{2}{r^2}=0
\]
make a link with my 2018 paper. 

