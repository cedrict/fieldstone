
\begin{center}
\input{tikz/tikz_dashpotdashpot_series_nl}
\end{center}

There are two dashpots in series, one accounts for dislocation creep, the other for diffusion creep.
The algorithm goes then as follows:
\begin{enumerate}
\item Assume we know $\dot\varepsilon_T$ (from previous iteration). 
\item The dashpots are in series so 
\[
\dot\varepsilon_T = \dot\varepsilon_{ds} + \dot\varepsilon_{df} 
\]
with
\begin{eqnarray}
\dot\varepsilon_{ds}  &=& A_{ds} \tau^n \exp \left(-\frac{Q_{ds}+pV_{ds}}{RT}\right) \label{sr_ds1} \\
\dot\varepsilon_{df}  &=& A_{df} \tau   \exp \left(-\frac{Q_{df}+pV_{df}}{RT}\right) \label{sr_df1} 
\end{eqnarray}
such that we are in fact looking for the stress value $\tau$ so that 
\[
\dot\varepsilon_T = 
A_{ds} \tau^n \exp \left(-\frac{Q_{ds}+p V_{ds}}{RT}\right) 
+
A_{df} \tau   \exp \left(-\frac{Q_{df}+p V_{df}}{RT}\right) 
\]
or, we must find the zero of the function ${\cal F}(\tau)$: 
\[
{\cal F}(\tau) =  \dot\varepsilon_T 
- A_{ds} \tau^n \exp \left(-\frac{Q_{ds}+p V_{ds}}{RT}\right) 
- A_{df} \tau   \exp \left(-\frac{Q_{df}+p V_{df}}{RT}\right) 
\]
This equation can be solved with a Newton-Raphson algorithm
and the iterations will be of the form:
\[
\tau_{n+1} = \tau_n - \frac{{\cal F}(\tau_n)}{{\cal F}'(\tau_n)}
\]
where the derivative of the function ${\cal F}$ with respect to $\tau$ reads:
\[
{\cal F}'(\tau)=\frac{\partial {\cal F}}{\partial \tau}=
- A_{ds} n \tau^{n-1} \exp\left(-\frac{Q_{ds}+pV_{ds}}{RT}\right)
- A_{df} \exp\left(-\frac{Q_{df}+pV_{df}}{RT}\right) 
\]
Once the value of $\tau$ is found, 
the strain rate values of Eqs. (\ref{sr_ds1}) and (\ref{sr_df1})
can be computed and so can the respective effective viscosities:
\begin{eqnarray}
\eta_{ds} 
&=& \frac{1}{2} A_{ds}^{1/n} \dot\varepsilon_{ds}^{\frac{1}{n}-1} \exp \left(\frac{Q_{ds}+pV_{ds}}{nRT}\right) \\
\eta_{df} 
&=& \frac{1}{2} A_{df}^{1/n}  \exp \left(\frac{Q_{df}+pV_{df}}{RT}\right) 
\end{eqnarray}
Their average effective viscosity $\tilde{\eta}_{eff}$ is given by 
\[
\tilde{\eta}_{eff} = \left( \frac{1}{\eta_{ds}} + \frac{1}{\eta_{df}} \right)^{-1}
\]
\end{enumerate}


Rather importantly, as we will see hereafter, the following variant is implemented 
in some codes (e.g. \douar, \fantom, \sopale, and probably many others) 
so as to bypass these costly Newton iterations:
\begin{enumerate}
\item compute $\eta_{ds}$ and $\eta_{df}$ with the {\it same} strainrate $\dot\varepsilon_T$, 
pressure and temperature values
\item average them by means of an harmonic average
\end{enumerate}
In this case, we have
\[
\dot{\varepsilon}_{\color{red} T}= 
A_{df} \tau_{df} \exp\left(-\frac{Q_{df}+pV_{df}}{RT}\right)
\quad\quad\quad
\dot{\varepsilon}_{\color{red} T}= 
A_{ds} \tau_{ds}^n \exp\left(-\frac{Q_{ds}+pV_{ds}}{RT}\right)
\]
or, 
\begin{eqnarray}
\eta_{ds} 
&=& \frac{1}{2} A_{ds}^{1/n} \dot\varepsilon_{\color{red}T}^{\frac{1}{n}-1} \exp \left(\frac{Q_{ds}+pV_{ds}}{nRT}\right) \\
\eta_{df} 
&=& \frac{1}{2} A_{df}^{1/n}  \exp \left(\frac{Q_{df}+pV_{df}}{RT}\right) 
\end{eqnarray}
We see that this simplification has consequences on the dislocation creep viscosity only.


\paragraph{A concrete example}
Let us consider a vertical section of upper mantle, from 660\si{\km} depth to 30\si{\km} depth.
The lithosphere is assumed to be 90\si{\km} thick. The temperature at the moho (the top
of the domain) is set to 550C, 1330C at the LMB and 1380C at the bottom.
A constant strainrate $\dot{\epsilon}_T=10^{-15}\si{\per\second}$ is assumed. 
We assume that the pressure is lithostatic (for simplicity 
the density is taken to be constant at 3300kg/m$^3$).
The temperature and pressure fields are shown hereunder:
\begin{center}
\includegraphics[width=6cm]{images/rheology/effvisc/temperature.pdf}
\includegraphics[width=6cm]{images/rheology/effvisc/pressure.pdf}
\end{center}

Material properties are taken from Karato \& Wu (1993) \cite{kawu93}.
The (fortran) code is available in {\tt images/rheology/effvisc/}.

In what follows, the values obtained with Newton iterations are coined 'NR'
and those obtained without are coined 'CHEAP'.
The diffusion and dislocation creep viscosities can be
computed for both algorithms and are shown hereunder
(As mentioned earlier the diffusion creep viscosity is independent of strain rate so
is the same for both):
\begin{center}
\includegraphics[width=6cm]{images/rheology/effvisc/both_mu_ds.pdf}
\includegraphics[width=6cm]{images/rheology/effvisc/both_mu_df.pdf}
\end{center}
We can also plot the resulting effective viscosity 
$\eta_{eff}$ for both approaches and we see that the differences 
are larger than 20\%. This is shown here under on the left, 
alongside with the partitioning of the strain rate as a function of depth:
\begin{center}
\includegraphics[width=8cm]{images/rheology/effvisc/both_mueff.pdf}
\includegraphics[width=8cm]{images/rheology/effvisc/both_sr.pdf}
\end{center}



