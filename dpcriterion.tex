\index{general}{Drucker-Prager}
\begin{flushright} {\tiny {\color{gray} dpcriterion.tex}} \end{flushright}
%~~~~~~~~~~~~~~~~~~~~~~~~~~~~~~~~~~~~~~~~~~~~~~~~~~~~~~~~~~~~~~~~~~~~~~~~~~~~~~~~~~~~~~~~~~~~~~~~~~

The von Mises yield criterion is not suitable for modelling the yielding of frictional material 
as it does not include the effect of mean stress as observed in experiments. To overcome this 
limitation, Drucker and Prager (1952) \cite{drpr52} proposed a revised function for frictional materials.

The Drucker-Prager yield criterion has the function form
\begin{equation}
\FFF^{\text{\tiny DP}}({\bm \sigma})=\FFF \left( {\cal I}_1({\bm \sigma}), {\cal I}_2({\bm \tau}) \right) = 0 
\end{equation}
This criterion is most often used for concrete where both normal and shear stresses 
can determine failure. The Drucker-Prager yield criterion may be expressed as
\begin{mdframed}[backgroundcolor=blue!5]
\begin{equation}
\FFF^{\text{\tiny DP}}= \sqrt{{\cal I}_2({\bm \tau})} + \alpha {\cal I}_1({\bm \sigma}) + k =0  
\label{dpcriterion} 
\end{equation}
\end{mdframed}

{\color{orange} it should probably be -k .. this needs to be propagated below!}

The problem is then to choose $\alpha$ and $k$ in a meaningful way, typically in relation with the 
Mohr-Coulomb criterion.

\begin{center}
\includegraphics[width=9cm]{images/rheology/owenhinton5}\\
{\captionfont Taken from \textcite{owhi}. The Drucker-Prager yield envelope 
circumscribes the Mohr-Coulomb one.}
\end{center}

Using the parameters $\sigma_m$, $\tau_m$, $a=-\sqrt{3}\tan\uptheta$, ${\cal I}_1({\bm \sigma})$ 
and ${\cal I}_2({\bm \tau})$ of Section~\ref{sec:altinv} we have
\begin{eqnarray}
\FFF^{\text{\tiny DP}}
&=&  \sqrt{{\cal I}_2({\bm \tau})} + \alpha {\cal I}_1({\bm \sigma}) + k \nn\\
&=& \sqrt{\frac{\tau_m^2}{3}(a^2+3)} + \alpha (3\sigma_m-a\tau_m) + k \nn\\ 
&=& \tau_m \sqrt{(a^2/3+1)} + \alpha (3\sigma_m+\tau_m\sqrt{3}\tan\uptheta ) + k    \qquad ({\rm since }\; \tau_m>0)\nn\\ 
&=& \tau_m \sqrt{\tan^2\uptheta+1} + \alpha (3\sigma_m+\tau_m\sqrt{3}\tan\uptheta ) + k  \nn\\
&=& \tau_m \sqrt{ \frac{1}{\cos^2\uptheta} } + \alpha (3\sigma_m+\tau_m\sqrt{3}\tan\uptheta ) + k  \nn\\
&=& \tau_m \frac{1}{\cos\uptheta} +\alpha (3\sigma_m+\tau_m\sqrt{3}\tan\uptheta ) + k  \qquad ({\rm since }\; \cos\uptheta >0)\nn
\end{eqnarray}
$\FFF^{\text{\tiny DP}}=0$ then leads to write
\begin{eqnarray}
\tau_m  + (3 \alpha \sigma_m+k)\cos\uptheta  + \tau_m \alpha \sqrt{3}\sin\uptheta  &=&0 \nn\\
\Rightarrow \qquad \tau_m(1 + \alpha \sqrt{3}\sin\uptheta)  + (3 \alpha \sigma_m+k)\cos\uptheta &=&0 \nn
\end{eqnarray}
and finally
\[
\tau_m = -\frac{(3 \alpha \sigma_m+k)\cos\uptheta}{1 + \alpha \sqrt{3}\sin\uptheta}
= -\frac{3 \alpha \cos\uptheta}{1 + \alpha \sqrt{3}\sin\uptheta} \sigma_m 
-\frac{k\cos\uptheta}{1 + \alpha \sqrt{3}\sin\uptheta}
\]
\begin{remark}
This is the same equation as Eq.~19 of Wojciechowski \cite{wojc18} but with $\uptheta \rightarrow -\uptheta$. 
\end{remark}

\vspace{.5cm}

The Mohr-Coulomb yield criterion is  (see Eq.~\eqref{eq:mccrit})
\[
\tau_m = -\sigma_m \sin\phi + c \cos\phi
\]
so that equating both expressions of $\tau_m$ for the Drucker-Prager 
and Mohr-Coulomb criteria leads to:
\begin{eqnarray}
-\frac{3 \alpha \cos\uptheta}{1 + \alpha \sqrt{3}\sin\uptheta} &=& -\sin\phi \label{eq:qq1}\\
-\frac{k\cos\uptheta}{1 + \alpha \sqrt{3}\sin\uptheta} &=& c \cos\phi \label{eq:qq2}
\end{eqnarray}
Eq.~\eqref{eq:qq1} yields
\[
3 \alpha \cos\uptheta = \sin\phi (1 + \alpha \sqrt{3}\sin\uptheta) 
\]
\[
\Rightarrow \qquad 3 \alpha \cos\uptheta - \alpha \sqrt{3}\sin\uptheta \sin\phi = \sin\phi 
\]
and finally 
\begin{equation}
\boxed{
\alpha(\phi) =  \frac{\sin\phi}{ 3 \cos\uptheta - \sqrt{3}\sin\uptheta \sin\phi}
}
\label{eq:dpalpha}
\end{equation}
Inserting this into Eq.~\eqref{eq:qq2}:
\begin{eqnarray}
- k \cos\uptheta 
&=& c \cos \phi \left(1 +\alpha \sqrt{3} \sin\uptheta \right)  \nn\\
&=& c \cos \phi \left(1 + \frac{\sin\phi}{ 3 \cos\uptheta - \sqrt{3}\sin\uptheta \sin\phi}  \sqrt{3} \sin\uptheta\right) \nn\\
&=& c \cos \phi \left(1 + 
\frac{ \sqrt{3}\sin\phi \sin\uptheta }{ 3 \cos\uptheta - \sqrt{3}\sin\uptheta \sin\phi} \right) \nn\\
&=& c \cos \phi \left(
\frac{ 3 \cos\uptheta - \sqrt{3}\sin\uptheta \sin\phi}{ 3 \cos\uptheta - \sqrt{3}\sin\uptheta \sin\phi} 
+ 
\frac{ \sqrt{3}\sin\phi \sin\uptheta }{ 3 \cos\uptheta - \sqrt{3}\sin\uptheta \sin\phi} \right) \nn\\
&=& c \cos \phi \left(
\frac{ 3 \cos\uptheta}{ 3 \cos\uptheta - \sqrt{3}\sin\uptheta \sin\phi} \right) \nn
\end{eqnarray}
so that 
\begin{equation}
\boxed{
k(c,\phi) =- \frac{ 3\; c \cos \phi }{ 3 \cos\uptheta - \sqrt{3}\sin\uptheta \sin\phi} 
}
\label{eq:dpk}
\end{equation}
%Unsurprisingly we recover the Eqs. 20 and 21 of Wojciechowski \cite{wojc18} by replacing $\theta$ by $-\theta$.

The Drucker-Prager yield criterion which for a given $\theta$ is equal to the Mohr-Coulomb yield is then:
\begin{eqnarray}
\FFF^{\text{\tiny DP}}
&=& \sqrt{{\cal I}_2({\bm \tau})} + \alpha(\phi) {\cal I}_1({\bm \sigma}) + k(c,\phi)  \nn\\
&=& \sqrt{{\cal I}_2({\bm \tau})} 
+ \frac{\sin\phi}{ 3 \cos\uptheta - \sqrt{3}\sin\uptheta \sin\phi}  {\cal I}_1({\bm \sigma})  
- \frac{ 3\; c \cos \phi }{ 3 \cos\uptheta - \sqrt{3}\sin\uptheta \sin\phi} \nn\\
&=& \sqrt{{\cal I}_2({\bm \tau})} 
- \left[ -\frac{3 \sin\phi}{ 3 \cos\uptheta - \sqrt{3}\sin\uptheta \sin\phi}  \frac{{\cal I}_1({\bm \sigma})}{3}
+ \frac{ 3\; c \cos \phi }{ 3 \cos\uptheta - \sqrt{3}\sin\uptheta \sin\phi} \right] \label{eq:Fdp}\\
&=& \sqrt{{\cal I}_2({\bm \tau})} 
- \left[ \frac{3\; p \sin\phi}{ 3 \cos\uptheta - \sqrt{3}\sin\uptheta \sin\phi} 
+ \frac{ 3\; c \cos \phi }{ 3 \cos\uptheta - \sqrt{3}\sin\uptheta \sin\phi} \right] \nn\\
&=& \sqrt{{\cal I}_2({\bm \tau})}  
- \frac{3\; p \sin\phi  + 3\; c \cos \phi }{ 3 \cos\uptheta - \sqrt{3}\sin\uptheta \sin\phi} \nn\\ 
&=& \sqrt{{\cal I}_2({\bm \tau})}  
- \frac{p \sin\phi  + c \cos \phi }{  \cos\uptheta - \frac{1}{\sqrt{3}}\sin\uptheta \sin\phi} 
\end{eqnarray}
which, when multiplied by $\cos\uptheta - \frac{1}{\sqrt{3}}\sin\uptheta \sin\phi$, gives
the Mohr-Coulomb criterion of Eq.~\eqref{eq:mcF}. 

%------------------------------------
\subsection{The three standard types}

For $\uptheta=\pi/6$, the DP yield surface {\bf circumscribes} the MC yield 
surface and Eq.~\eqref{eq:Fdp} writes:
\begin{eqnarray}
\FFF^{\text{\tiny DP}}
&=& \sqrt{{\cal I}_2({\bm \tau})} 
- \left[ -\frac{3 \sin\phi}{ 3 \sqrt{3}/2 - \sqrt{3}/2 \; \sin\phi}  \frac{{\cal I}_1({\bm \sigma})}{3}
+ \frac{ 3\; c \cos \phi }{ 3 \sqrt{3}/2 - \sqrt{3}/2 \; \sin\phi} \right] \nn\\
&=& \sqrt{{\cal I}_2({\bm \tau})} 
- \left[ -\frac{6 \sin\phi}{\sqrt{3} (3 - \sin\phi) }  \frac{{\cal I}_1({\bm \sigma})}{3}
+ \frac{ 6\; c \cos \phi }{ \sqrt{3}(3 - \sin\phi)} \right] \nn\\
&=& \sqrt{{\cal I}_2({\bm \tau})} 
- \frac{ 6 p \sin \phi + 6\; c \cos \phi }{ \sqrt{3}(3 - \sin\phi)} \label{eq:dpc}
\end{eqnarray}
i.e.

\begin{center}
\includegraphics[width=8cm]{images/rheology/owenhinton6}\\
{\captionfont Taken from \textcite{owhi}.}
\end{center}

\begin{mdframed}[backgroundcolor=blue!5]
\begin{equation}
\FFF^{\text{\tiny DP}}
= \sqrt{{\cal I}_2({\bm \tau})} 
+ \frac{ 6 \sin \phi }{ \sqrt{3}(3 - \sin\phi)} \frac{{\cal I}_1({\bm\sigma})}{3}
- \frac{ 6\; c \cos \phi }{ \sqrt{3}(3 - \sin\phi)} 
\end{equation}
\end{mdframed}

which is the formula used in Glerum \etal (2018) \cite{gltf18}.
This is also Eq.~(14a) in Zienkiewicz \& Cormeau (1974) \cite{zico74}, 
Eq.~(7.18) in \textcite{owhi}, and Eq.~(13.10a) in Zienkiewicz (1975) \cite{zien75} 
provided it is divided altogether by $\sqrt 3$. 


For $\uptheta=-\pi/6$, the DP yield surface {\bf middle circumscribes} the MC yield surface 
and Eq.~\eqref{eq:Fdp} writes:
\begin{eqnarray}
\FFF^{\text{\tiny DP}}
&=& \sqrt{{\cal I}_2({\bm \tau})} 
- \left[ -\frac{3 \sin\phi}{ 3 \sqrt{3}/2 + \sqrt{3}/2 \sin\phi}  \frac{{\cal I}_1({\bm \sigma})}{3}
+ \frac{ 3\; c \cos \phi }{ 3 \sqrt{3}/2 + \sqrt{3}/2 \sin\phi} \right] \nn\\
&=& \sqrt{{\cal I}_2({\bm \tau})} 
- \left[ -\frac{6 \sin\phi}{\sqrt{3} (3 + \sin\phi) }  \frac{{\cal I}_1({\bm \sigma})}{3}
+ \frac{ 6\; c \cos \phi }{ \sqrt{3}(3 + \sin\phi}) \right] \nn\\
&=& \sqrt{{\cal I}_2({\bm \tau})} 
- \frac{6 p \sin\phi + 6 c \cos \phi}{\sqrt{3} (3 + \sin\phi) } 
\end{eqnarray}
This is Eq.~(7.19) of \textcite{owhi}.

Another DP formulation which {\bf inscribes} the MC yield surface is found on the 
wikipedia page of the Drucker-Prager yield criterion
\footnote{\url{https://en.wikipedia.org/wiki/Drucker-Prager_yield_criterion}. 
(but I have no idea how it is arrived at)}\todo{Look into it!}:

\begin{center}
\includegraphics[width=9cm]{images/rheology/dp_wiki1}\\
\includegraphics[width=9cm]{images/rheology/dp_wiki2}\\
\end{center}

With our notations:

\begin{eqnarray}
\FFF^{\text{\tiny DP}}
&=& \sqrt{{\cal I}_2({\bm \tau})} 
- \left[ -\frac{3 \sin\phi}{\sqrt{9+3\sin^2\phi} }  \frac{{\cal I}_1({\bm \sigma})}{3}
+ \frac{ 3\; c \cos \phi }{ \sqrt{9+3\sin^2\phi} } \right] \label{eq:dp3}
\end{eqnarray}

The yield surfaces of these three Drucker-Prager formulations are plotted against the Mohr-Coulomb
yield surface in Section~\ref{ss:envelope}. 


%%%%%%%%%%%%%%%%%%%%%%%%%%%%%%%%%%
%\paragraph{three-dimensional space}

%Let us first assume that the Drucker-Prager yield surface circumscribes the Mohr-Coulomb yield surface such that the two surfaces coincide at $\theta=\tfrac{\pi}{3}$. 
%The expression for the Mohr-Coulomb yield criterion is

%\[
%F^{MC,3D} = -\frac{1}{3}J_1 \sin \phi  + \sqrt{J_2} ( \cos \theta - \frac{1}{\sqrt{3}} \sin \theta  \sin \phi ) - c \cos \phi 
%\]
%Taking $\theta=\pi/6$ yields:
%\begin{eqnarray}
%F^{MC,3D} 
%&=& -\frac{1}{3}J_1 \sin \phi  + \sqrt{J_2} ( \frac{\sqrt{3}}{2} - \frac{1}{\sqrt{3}} \frac{1}{2}  \sin \phi ) - c \cos \phi  \nn\\
%&=& -\frac{1}{3}J_1 \sin \phi  + \frac{1}{2\sqrt{3}}\sqrt{J_2} ( 3 -  \sin \phi ) - c \cos \phi  \nn\\
%&=& -\frac{1}{3}J_1 \sin \phi  + \frac{1}{6}\sqrt{J_2} \sqrt{3}( 3 -  \sin \phi ) - c \cos \phi  \nn\\
%&=& \frac{\sqrt{3}(3-\sin\phi)}{6} \left( -  \frac{2 \sin \phi}{\sqrt{3}(3-\sin\phi)}   J_1  + \sqrt{J_2}  - \frac{6c \cos \phi}{\sqrt{3}(3-\sin\phi)} \right) \nn
%\end{eqnarray}
%The constant in front of the brackets (which always strictly positive) does not matter since we look at the sign of $F$.
%Comparing with 
%\[
%F^{DP}=\sqrt{J_2} - (\alpha J_1 + k) \label{dpcriterion} 
%\]
%we naturally set 
%\[
%\alpha =\frac{2 \sin \phi}{\sqrt{3}(3-\sin\phi)}
%\quad\quad\quad  
%k =  \frac{6c \cos \phi}{\sqrt{3}(3-\sin\phi)} 
%\]
%and since $p=\frac{1}{3}{\cal I}_1({\bm \sigma})$:
%\begin{mdframed}[backgroundcolor=blue!5]
%\begin{equation}
%F^{DP,3D} = \tau_e  - \left[ \frac{6 \sin \phi}{\sqrt{3}(3-\sin\phi)}   p  + 
%\frac{6c \cos \phi}{\sqrt{3}(3-\sin\phi)}  \right]
%\label{eqdp3D}
%\end{equation}
%\end{mdframed}
%\begin{center}
%\includegraphics[width=0.6\textwidth]{RHEOLOGY/viscoplasticity/dpcriterion.pdf}
%\end{center}


\vspace{1.3cm}
%..............
\begin{remark}
\textcite{leor89} (1989) use the Drucker-Prager plasticity model also and match it to the Mohr-Coulomb model in the 
triaxial test and formulate it as follows 
(Their definition of the second invariant of stress contains a 3/2 term):
\begin{eqnarray}
\FFF 
&=& \tau_e \sqrt{3} + \frac{6 \sin\phi}{3-\sin\phi} \left( -p  - \frac{c}{\tan \phi} \right) \nn\\
&=& \tau_e \sqrt{3} - \left( \frac{6 \sin\phi}{3-\sin\phi}  p  + c \frac{6 \cos\phi}{3-\sin\phi} \right) \nn\\
&=& \sqrt{3} \left[ \tau_e  - \left( \frac{6 \sin\phi}{\sqrt{3}(3-\sin\phi)}  p  + c \frac{6 \cos\phi}{\sqrt{3}(3-\sin\phi)} \right)  \right]
\end{eqnarray}
Except for the $\sqrt{3}$ this is identical to Eq.~\eqref{eq:dpc}.
\end{remark}

%..............
\begin{remark}
Bui \etal (2008) \cite{bufs08} use yet again another formulation:
\begin{eqnarray}
\FFF 
&=& \sqrt{{\cal I}_2({\bm \tau})} + \frac{\tan \phi}{\sqrt{9 + 12 \tan^2 \phi}} {\cal I}_1({\bm \sigma})
- \frac{3 c}{\sqrt{9 + 12 \tan^2 \phi}} \nn\\
&=& \sqrt{{\cal I}_2({\bm \tau})} + \frac{\sin \phi}{\sqrt{9\cos^2 \phi + 12 \sin^2 \phi}} {\cal I}_1({\bm \sigma}) - \frac{3 c \; \cos\phi}{\sqrt{9\cos^2\phi + 12 \sin^2 \phi}} \nn\\
&=& \sqrt{{\cal I}_2({\bm \tau})} + \frac{\sin \phi}{\sqrt{9 + 3 \sin^2 \phi}} {\cal I}_1({\bm \sigma}) - \frac{3 c \; \cos\phi}{\sqrt{9 + 3\sin^2 \phi}} \nn
\end{eqnarray}
which is identical to \eqref{eq:dp3}.
\end{remark}

%..............
\begin{remark}
Cacace \& Jacquey (2017) \cite{caja17} replace $\sqrt{{\cal I}_2({\bm \tau})}$ by 
$\sqrt{{\cal I}_2({\bm \tau})+\epsilon_0^2}$ where $\epsilon_0$ is a small non-hardening parameters 
here introduced to relax the singularity at the cone's tip of the Drucker-Prager yield envelope.
\end{remark}

\Literature:
\fullcite{cuwi14}

%-----------------------------------------------------------------------------
\subsection{What happens when $\phi=0$?}

From \eqref{eq:dpalpha} we obtain $\alpha=0$, and 
from Eq.~\eqref{eq:dpk} we obtain 
\[
k= -\frac{3c }{3 \cos\uptheta_{\rm L}}
\]
Then 
\begin{equation}
\FFF^{\text{\tiny DP}}= \sqrt{{\cal I}_2({\bm \tau})} -\frac{c }{\cos\uptheta_{\rm L}}  =0  
\end{equation}

The MC-circumscribing DP formulation ($\uptheta_L=\pi/6$) then becomes
\begin{equation}
\FFF^{\text{\tiny DP}} 
= \sqrt{{\cal I}_2({\bm \tau})} - \frac{ 6\; c }{3 \sqrt{3}} 
= \sqrt{{\cal I}_2({\bm \tau})} - \frac{ 2 c }{\sqrt{3}} 
\end{equation}
We obviously obtain an identical expression 
for the middle-circumscribing formulation ($\uptheta_L=-\pi/6$).

The MC-inscribing case is then 
\[
\FFF^{\text{\tiny DP}}
= \sqrt{{\cal I}_2({\bm \tau})} - \frac{ 3\; c \cos \phi }{ \sqrt{9} }
= \sqrt{{\cal I}_2({\bm \tau})}  - c 
\]

{\color{orange} reconcile with von Mises! }

%.............................................................................
\subsection{Dissecting the original paper by Drucker and Prager (1952)}

The authors state that a yield function which is a proper generalisation of the M-C hypothesis is:
\[
\FFF = \alpha {\cal I}_1(\bm\sigma) + \sqrt{{\cal I}_2(\bm\tau)} - k
\]
where $\alpha$ and $k$ are positive constants at each point of the material.

According to the concept of plastic potential, the stress-train relation
corresponding to this yield function is 
\[
\dot\varepsilon_{ij}^{p} = \lambda \frac{\partial \FFF}{\partial \sigma_{ij}}
\]
where $\dot\varepsilon_{ij}^{p}$ is the plastic strain rate and $\lambda$
is a positive factor of proportionality which may assume different values in space. Using the above expression for $\FFF$:
\begin{equation}
\dot\varepsilon_{ij}^{p} = \lambda \left( \alpha \delta_{ij} + \frac{\tau_{ij}}{2\sqrt{{\cal I}_2(\bm\tau)}} \right)
\label{eq:dp52:ab}
\end{equation}
A very important feature of this equation is that the plastic rate of cubical
dilation is 
\begin{equation}
{\rm tr}[\dot{\bm \varepsilon}^{p}]
=\dot\varepsilon_{ii}^{p} = 3 \alpha \lambda \ne 0
\label{eq:dp52:bc}
\end{equation}
This equation shows that plastic deformation must be accompanied by an increase
in volume if $\alpha\ne 0$. This property is known as dilatancy.

%-----------------------------------------------------------------------
\paragraph{Plane strain} 
We need to establish three expressions.  
First, from Eq.~\eqref{eq:dp52:ab} we can write 
\[
\dot\varepsilon_{zz}^{p} = \lambda \left( \alpha  + \frac{\tau_{zz}}{2\sqrt{{\cal I}_2(\bm\tau)}} \right)
\]
but since $\dot\varepsilon_{zz}=0$ in plane strain then we find
\begin{equation}
\tau_{zz} = - 2 \alpha \sqrt{{\cal I}_2(\bm\tau)}
\end{equation}
which is Eq.~(6) of the paper. 

Second, we start from the definition of the first invariant and use the equation above:
\begin{eqnarray}
{\cal I}_1(\bm \sigma)&=& \sigma_{xx}+\sigma_{yy}+\sigma_{zz} \nn\\
{\cal I}_1(\bm \sigma)&=& \sigma_{xx}+\sigma_{yy}+\tau_{zz}+\frac13 {\cal I}_1(\bm \sigma) \nn\\
{\cal I}_1(\bm \sigma)&=& \sigma_{xx}+\sigma_{yy}- 2 \alpha \sqrt{{\cal I}_2(\bm\tau)} 
+\frac13 {\cal I}_1(\bm \sigma)\nn\\
\frac23 {\cal I}_1(\bm \sigma)&=& \sigma_{xx}+\sigma_{yy}- 2 \alpha \sqrt{{\cal I}_2(\bm\tau)} \nn\\
{\cal I}_1(\bm \sigma)&=& \frac32(\sigma_{xx}+\sigma_{yy})- 3 \alpha \sqrt{{\cal I}_2(\bm\tau)} 
\label{eq:I1dpps}
\end{eqnarray}
which is Eq.~(7) of the paper.

Finally, we start from (and we use the fact that $\sigma_{xx}-\sigma_{yy}=\tau_{xx}-\tau_{yy}$)
\begin{eqnarray}
\left(\frac{\sigma_{xx}-\sigma_{yy}}{2} \right)^2 + \tau_{xy}^2
&=& \frac14 \left(\sigma_{xx}-\sigma_{yy} \right)^2 + \tau_{xy}^2 \nn\\
&=& \frac14 \left(\sigma_{xx}-\sigma_{yy} \right)^2 
+ \underbrace{\tau_{xy}^2 
+ \frac12 (\tau_{xx}^2+\tau_{yy}^2+\tau_{zz}^2)}_{{\cal I}_2(\bm\tau)}
- \frac12 (\tau_{xx}^2+\tau_{yy}^2+\tau_{zz}^2) \nn\\
&=& \frac14 \left(\tau_{xx}-\tau_{yy} \right)^2 
+ {\cal I}_2(\bm\tau)
- \frac12 \tau_{xx}^2 -\frac12 \tau_{yy}^2 -\frac12 \tau_{zz}^2 \nn\\
&=& {\cal I}_2(\bm\tau) +\frac14\tau_{xx}^2 -\frac12\tau_{xx}\tau_{yy} +\frac14\tau_{yy}^2
- \frac12 \tau_{xx}^2 -\frac12\tau_{yy}^2 -\frac12 4\alpha^2 {\cal I}_2(\bm\tau) \nn\\
&=& {\cal I}_2(\bm\tau) -\frac14\tau_{xx}^2 -\frac12\tau_{xx}\tau_{yy} -\frac14\tau_{yy}^2
-2\alpha^2 {\cal I}_2(\bm\tau) \nn\\
&=& {\cal I}_2(\bm\tau) -\frac14(\tau_{xx}^2 +2\tau_{xx}\tau_{yy} +\tau_{yy}^2)
-2\alpha^2 {\cal I}_2(\bm\tau) \nn\\
&=& {\cal I}_2(\bm\tau) -\frac14(\underbrace{\tau_{xx}+\tau_{yy}}_{-\tau_{zz}})^2
-2\alpha^2 {\cal I}_2(\bm\tau) \nn\\
&=& {\cal I}_2(\bm\tau) -\frac14 \tau_{zz}^2 -2\alpha^2 {\cal I}_2(\bm\tau) \nn\\
&=& {\cal I}_2(\bm\tau) -\frac14 4\alpha^2 {\cal I}_2(\bm\tau) -2\alpha^2 {\cal I}_2(\bm\tau) \nn\\
&=& {\cal I}_2(\bm\tau) -3\alpha^2 {\cal I}_2(\bm\tau) \nn\\
&=& {\cal I}_2(\bm\tau)(1 -3\alpha^2)
\end{eqnarray}
so that 
\begin{equation}
{\cal I}_2(\bm\tau) = \frac{1}{1 -3\alpha^2} \left[\left(\frac{\sigma_{xx}-\sigma_{yy}}{2} \right)^2 + \tau_{xy}^2 \right]
\label{eq:dp52:cd}
\end{equation}
which is Eq.~(8) of the paper.

In the paper the authors propose the yield function 
\[
\FFF = \alpha {\cal I}_1(\bm\sigma) + \sqrt{{\cal I}_2(\bm\tau)} - k
\]
We first replace the first (plane strain) invariant (see Eq.~\eqref{eq:I1dpps}):
\begin{eqnarray}
\FFF 
&=& \alpha \left[ \frac32(\sigma_{xx}+\sigma_{yy})- 3 \alpha \sqrt{{\cal I}_2(\bm\tau)} \right]
+ \sqrt{{\cal I}_2(\bm\tau)} - k \nn\\
&=&\alpha \frac32(\sigma_{xx}+\sigma_{yy})- 3 \alpha^2 \sqrt{{\cal I}_2(\bm\tau)}
+ \sqrt{{\cal I}_2(\bm\tau)} - k  \nn\\
&=&\alpha \frac32(\sigma_{xx}+\sigma_{yy}) +(1- 3 \alpha^2) \sqrt{{\cal I}_2(\bm\tau)} - k \nn
\end{eqnarray}
and we now introduce the second invariant of Eq.~\eqref{eq:dp52:cd}:
\begin{eqnarray}
\FFF &=& 
\alpha \frac32(\sigma_{xx}+\sigma_{yy}) +(1- 3 \alpha^2)
\frac{1}{(1 -3\alpha^2)^{1/2}} \left[\left(\frac{\sigma_{xx}-\sigma_{yy}}{2} \right)^2 
+ \tau_{xy}^2 \right]^{1/2} -  k  \nn\\
&=& \alpha \frac32(\sigma_{xx}+\sigma_{yy}) +(1- 3 \alpha^2)^{1/2}
\left[\left(\frac{\sigma_{xx}-\sigma_{yy}}{2} \right)^2 + \tau_{xy}^2 \right]^{1/2} 
- k  \nn\\
&=&\frac{3\alpha}{(1- 3 \alpha^2)^{1/2}}
\frac12(\sigma_{xx}+\sigma_{yy}) +
\left[\left(\frac{\sigma_{xx}-\sigma_{yy}}{2} \right)^2 + \tau_{xy}^2 \right]^{1/2} 
- \frac{k}{(1- 3 \alpha^2)^{1/2}} 
\nn\\
&=&\underbrace{\frac{3\alpha}{(1- 3 \alpha^2)^{1/2}}}_{\sin\phi}
\frac12(\sigma_{xx}+\sigma_{yy}) +
\left[\left(\frac{\sigma_{xx}-\sigma_{yy}}{2} \right)^2 + \tau_{xy}^2 \right]^{1/2} 
- \underbrace{\frac{k}{(1- 12 \alpha^2)^{1/2}}}_{c}
\underbrace{\frac{(1- 12 \alpha^2)^{1/2}}{(1- 3 \alpha^2)^{1/2}}}_{\cos\phi}
\end{eqnarray}


Note that if we define a triangle with sides $(1- 12 \alpha^2)^{1/2}$ and $3\alpha$
with hypotenuse $(1- 3 \alpha^2)^{1/2}$ then the angle $\phi$ makes sense and
we recover $\cos^2\phi+\sin^2\phi=1$.

In the end:
\[
\FFF = 
\left[\left(\frac{\sigma_{xx}-\sigma_{yy}}{2} \right)^2 + \tau_{xy}^2 \right]^{1/2}
-\left(-\frac12(\sigma_{xx}+\sigma_{yy}) \sin \phi + c \cos \phi \right)
\]
which is the Mohr-Coulomb yield criterion of Eq.~(1) in the paper.


Note that when $\alpha=0$ (yield criterion independent of the mean stress - 
incompressible flow see Eq.~\eqref{eq:dp52:bc}) then $c=k$, $\cos\phi=1$ and $\sin\phi=0$ and we find 
the Tresca yield criterion
\[
F^{TR}=  \left[\left(\frac{\sigma_{xx}-\sigma_{yy}}{2} \right)^2 + \tau_{xy}^2 \right]^{1/2} - k
\]
Also, setting $\alpha=0$ in Eq.~\eqref{eq:dp52:cd} yields a criterion that writes
\[
F^{vM}=\sqrt{{\cal I}_2(\bm\tau)} - k
\]
which is the von Mises criterion! 

\vspace{.5cm}

We now look into the derivatives of the Drucker-Prager plastic potential $\QQQ^{\text{\tiny DP}}(\bm\sigma)$.
We have
\[
\QQQ^{\text{\tiny DP}} (\bm\sigma)= \sqrt{{\cal I}_2({\bm \tau})} + \alpha {\cal I}_1({\bm \sigma}) + k 
\]
Then
\begin{eqnarray}
\frac{\partial \QQQ^{\text{\tiny DP}} }{\partial {\cal I}_1(\bm\sigma)} &=& \alpha \\
\frac{\partial \QQQ^{\text{\tiny DP}} }{\partial \sqrt{{\cal I}_2(\bm\tau)}}&=& 1  \\
\frac{\partial \QQQ^{\text{\tiny DP}} }{\partial \uptheta_{\rm L}(\bm\tau)} &=& 0 
\end{eqnarray}
The parameters $\alpha$ and $k$ can be expressed as a function of the angle of friction 
and cohesion so as to match the Mohr-Coulomb criterion in some sense (see above).
Then
\begin{eqnarray}
C_1^{\text{\tiny DP}} &=& \alpha  \\ 
C_2^{\text{\tiny DP}} &=& \frac{1}{2  \sqrt{{\cal I}_2(\bm\tau)}   }  \\ 
C_3^{\text{\tiny DP}} &=& 0  
\end{eqnarray}

{\color{orange}ToDo: check \textcite{albo12} (2012) and compare with my notes above.}


------------------------






