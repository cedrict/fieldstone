\index{general}{Arrhenius law}
\begin{flushright} {\tiny {\color{gray} arrhenius\_law.tex}} \end{flushright}
%~~~~~~~~~~~~~~~~~~~~~~~~~~~~~~~~~~~~~~~~~~~~~~~~~~~~~~~~~~~~~~~~~~~~~~~~~~~~~~~~~~~~~~~~~~~~~~~~~~

A purely temperature-dependent dimensional Arrhenius law that emulates the temperature
dependence of viscosity in silicate rock is often employed for mantle rocks 
\cite{albe00,zhzm09,vata11,bogs13b,namu13,stha13,boba19,gult19}:
\begin{equation}
\eta(T)=\eta_0 \exp \left( \frac{Q}{R}(\frac{1}{T}-\frac{1}{T_0}) \right)
\qquad 
{\rm or}
\qquad 
\eta(T)=\eta_0 \exp \left( \frac{Q}{RT} \right)
\end{equation}
where $\eta_0$ is a reference viscosity and $T_0$ its corresponding reference 
temperature.

It can also account for pressure effects as in \cite{lorg18} where the
diffusion creep viscosity (under the assumption of homogeneous grain size)
is temperature- and pressure-dependent:
\[
\eta(T)=\eta_0 \exp \left( \frac{1}{R}(\frac{Q-pV}{T}-\frac{Q}{T_0}) \right)
\]
(I find the minus sign rather suspicious)

