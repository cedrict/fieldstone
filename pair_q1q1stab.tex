
\label{ss:pairq1q1stab}
\begin{flushright} {\tiny {\color{gray} \tt pair\_q1q1stab.tex}} \end{flushright}
%~~~~~~~~~~~~~~~~~~~~~~~~~~~~~~~~~~~~~~~~~~~~~~~~~~~~~~~~~~~~~~~~~~~~~~~~~~~~~~~~~~~~~~~~~~~~~~~~~~

\begin{minipage}[t]{0.5\textwidth}
\input{tikz/tikz_q1q1}
\end{minipage}
\begin{minipage}[t]{0.5\textwidth}
\input{tikz/tikz_q1q1_3D}
\end{minipage}

The ${\bm Q}_1\times Q_1$ element is not LBB-stable but it can be stabilised. Despite
some applications in geodynamics (it is used in \textcite{bugs09} (2009) 
and \textcite{busa13} (2013)), it is not appropriate for buoyancy-driven flows, 
as shown in \textcite{thba22}.

See \textcite{nosi01} (2001) for a fourier analysis of the normal 
and stablised (a la \textcite{hufb86} (1986)) ${\bm Q}_1\times Q_1$ element.
Stabilisation is worked out out in \textcite{dobo04} (2004), \textcite{bodg06} (2006), 
and \textcite{bodo06} (2006).

\begin{itemize}
\item 
${\bm Q}_1\times P_0$-stab. Pro: stabilisation can be switched off; Con: stabilisation for deformed elements? 
problem near boundaries: incomplete stencil? choice of parameter $\beta$.
\item 
${\bm Q}_1\times Q_1$-stab. Pro: easier to implement than ${\bm Q}_1\times P_0$-stab, stabilisation local to element, 
easier when elements are not rectangular, no free parameter; Con: stabilisation cannot be switched off.
\end{itemize}

\Literature: \textcite{shry78,temr92,tezd92,grcc95,idsn95,knto00,fros07,lihc09}. 
See \textcite{brlu09} for a review of local projection stabilisation for incompressible flow problems. 

This unstable pair is also used in ice sheet modelling \textcite{heah18} , \textcite{zhjg11}, 
\textcite{zwgg07}. A ${\bm P}_1\times P_1$ version of it is used in \textcite{kahp20} (2020).



