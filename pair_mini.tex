\begin{flushright} {\tiny {\color{gray} \tt pair\_mini.tex}} \end{flushright}
%~~~~~~~~~~~~~~~~~~~~~~~~~~~~~~~~~~~~~~~~~~~~~~~~~~~~~~~~~~~~~~~~~~~~~~~~~~~~~~~~~~~~~~~~~~~~~~~~~~

\noindent
\begin{minipage}{0.48\textwidth}
The \index{general}{MINI element} MINI element was first introduced in 
\textcite{arbf84} (1984). It is also denoted by ${\bm P}_1^+\times P_1$.
It is also discussed in Section~3.6.1 of \textcite{john16} (2016) and in Section~6.1 
of \textcite{bobf08} (2008). It is thoroughly studied in \textcite{cibo19} (2019).

As explained in Braess \cite{braess}, since the support of the cubic bubble is restricted to the element, 
the associated variable (dofs living on the bubble) can be eliminated from the resulting 
system of linear equations by static condensation \cite{koko19}. \index{general}{Static Condensation}
Also, the MINI element is cheaper than the Taylor-Hood element but it is commonly accepted
that it yields a poorer approximation of the pressure.
\end{minipage}\hfill
\begin{minipage}{0.48\textwidth}
\input{tikz/tikz_mini}
\end{minipage}

\begin{remark}
Note that \textcite{frol03} (2003) propose an equal-order-linear-continuous 
velocity-pressure variables which is enriched 
with velocity {\it and} pressure bubble functions to model the Stokes problem. 
They show by static condensation that
these bubble functions give rise to a stabilized method involving least-squares forms of 
the momentum and of the
continuity equations. In some cases their approach recovers 
the MINI element. Also check \textcite{gamt08} (2008).
\end{remark}

The 3D MINI element is not very common but it is used for instance in \textcite{pico98} (1998),
\textcite{tokv09} (2009), \textcite{koko19} (2019) or \textcite{kuak24} (2024). 
It is also said to be LBB stable in Reddy \cite[p180]{reddybook2}.
It is used in \cite{furstoss} phd thesis in the context of microstructures deformation modeling, 
which itself cites \textcite{camb13} (2013).

\begin{center}
\includegraphics[width=8cm]{images/mini/mini3D}\\
{\captionfont Velocity and pressure nodes for the 3D MINI element, taken from \cite{pico98}}
\end{center}

Note that this element is used in Braess \& Wriggers (2000) \cite{brwr00} 
in the context of Arbitrary Lagrangian Eulerian 
finite element analysis of free surface flows, and also 
in \textcite{zldf07} (2007) for subduction with X-FEM technique. 
\index{general}{X-FEM}. It is also mentioned in \textcite{nath93} (1993).
The 2D element is implemented in \stone~120.

\begin{center}
\url{https://defelement.com/elements/mini.html}
\end{center}

\url{https://www.math.uci.edu/~chenlong/ifemdoc/fem/StokesP1bP1femrate.html}

MATLAB source: \url{https://nl.mathworks.com/matlabcentral/fileexchange/70996-kstok}
