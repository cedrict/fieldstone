

%........................................
\subsubsection{Cartesian coordinates}

%........................................
\subsubsection{Cylindrical coordinates}


We have $r>0$ and $\theta=[0,2\pi[$, defined in the $(x,y)-$plane.

The relation between the unit vector in Cartesian and Polar/Cylindrical coordinates
is given by:
\[
\left(
\begin{array}{c}
{\vec e}_{r} \\
{\vec e}_{\theta} \\
\end{array}
\right)
=
\left(
\begin{array}{cc}
\cos \theta & \sin \theta \\
-\sin \theta & \cos \theta
\end{array}
\right)
\cdot
\left(
\begin{array}{c}
{\vec e}_{x} \\
{\vec e}_{y} \\
\end{array}
\right)
\]
which yields
\[
\left(
\begin{array}{c}
{\vec e}_{x} \\
{\vec e}_{y} \\
\end{array}
\right)
=
\left(
\begin{array}{cc}
\cos \theta & -\sin \theta \\
\sin \theta & \cos \theta
\end{array}
\right)
\cdot
\left(
\begin{array}{c}
{\vec e}_{r} \\
{\vec e}_{\theta} \\
\end{array}
\right)
\]

so that for any vector ${\vec V}$
\begin{eqnarray}
{\vec V} 
&=& V_x {\vec e}_x  + V_y {\vec e}_y \nonumber\\
&=& V_x [(\cos \theta) {\vec e}_r - (\sin \theta) {\vec e}_\theta]  + 
    V_y [(\sin \theta) {\vec e}_r + (\cos \theta){\vec e}_\theta] \nonumber\\
&=& [V_x (\cos \theta) + V_y (\sin \theta)] {\vec e}_r +
[- V_x(\sin \theta) + V_y (\cos \theta)]{\vec e}_\theta  
\end{eqnarray}




%........................................
\subsubsection{Spherical coordinates}

On the following figure are represented the three cartesian axis, 
a point and its spherical coordinates $r,\theta,\phi$:
\begin{center}
\includegraphics[width=5cm]{images/sphcoord}\\
{\small Spherical coordinates as commonly used in physics:\\ polar angle $\theta$, and azimuthal angle $\phi$.} 
\end{center}
In this case $\theta\in[0:\pi]$ and $\phi\in]-\pi:\pi]$ and we have the following relationships:
\begin{eqnarray}
r &=& \sqrt{x^2+y^2+z^2} \\
\theta &=& acos (z/r) \\
\phi &=& atan (y/x) 
\end{eqnarray}
The inverse tangent used to compute $\phi$ must be suitably defined, taking into account the correct quadrant of $(x,y)$,
which is why the atan2 intrinsic function is used in fortran for example.    

In geography one uses latitude and longitude, represented hereunder:
\begin{center}
\includegraphics[width=10cm]{images/map.jpg}
\end{center}
\begin{itemize}
\item Latitude  $\in[-90:90]$,   or $\in[-\pi/2:\pi/2]$ 
\item Longitude $\in]-180:180]$, or $\in]-\pi:\pi]$ 
\end{itemize}

Since the colatitude is the complementary angle of the latitude, 
i.e. the difference between 90 and the latitude, 
where southern latitudes are denoted with a minus sign,
$\theta$ as shown above is actually is the colatitude.
The co-latitude is shown in red on the following figure:
\begin{center}
\includegraphics[width=3cm]{images/colatitude}
\end{center}




