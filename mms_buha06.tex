\begin{flushright} {\tiny {\color{gray} mms\_buha06.tex}} \end{flushright}
%~~~~~~~~~~~~~~~~~~~~~~~~~~~~~~~~~~~~~~~~~~~~~~~~~~~~~~~~~~~~~~~~~~~~~~~~~~~~~~~~~~~~~~~~~~~~~~~~~~

This is presented in \textcite{buha06} (2006) and \textcite{buha07} (2007) 
and apparently originates in \textcite{nosi98} (1998). 

The velocity and pressure fields are given in the unit square by
\begin{eqnarray}
u(x,y) &=& 20xy^3 \nn\\
v(x,y) &=& 5x^4-5y^4 \nn\\
p(x,y) &=& 60x^2y -20y^3 -5
\end{eqnarray}
with
\begin{eqnarray}
\frac{\partial u }{\partial x} &=& 20y^3 \nn\\
\frac{\partial u }{\partial y} &=& 60xy^2 \nn\\
\frac{\partial v }{\partial x} &=& 20x^3 \nn\\
\frac{\partial v }{\partial y} &=& -20y^3 
\end{eqnarray}

\begin{center}
\includegraphics[width=8cm]{images/mms/buha06}\\
{\captionfont Taken from \textcite{buha06}. Left is velocity, right is pressure.}
\end{center}

The flow is incompressible:
\[
div (\vec\upnu) = 
\frac{\partial u }{\partial x}
+
\frac{\partial v }{\partial y}
= 0
\]
Then the strain rate tensor is given by 
\[
\dot{\bm\varepsilon} (\vec\upnu)
=
\left(
\begin{array}{cc}
20y^3 & 30xy^2+10x^3 \\
30xy^2+10x^3 & -20y^3
\end{array}
\right)
\]
%and the pressure gradient is
%\begin{eqnarray}
%\frac{\partial p}{\partial x} &=& 120xy  \nn\\
%\frac{\partial p}{\partial y} &=& 60x^2-60y^2
%\end{eqnarray}
Assuming the viscosity $\eta=1$, then the full stress tensor is 
given by
\begin{eqnarray}
{\bm\sigma} 
&=& -p {\bm 1} + 2 \eta \dot{\bm\varepsilon} (\vec\upnu) \nn\\
&=&
\left(
\begin{array}{cc}
-60x^2y +20y^3 +5 + 40y^3 & 60xy^2+20x^3 \\
60xy^2+20x^3 & -60x^2y +20y^3 +5 -40y^3
\end{array}
\right) \nn\\
&=&
\left(
\begin{array}{cc}
-60x^2y +60y^3 +5  & 60xy^2+20x^3 \\
60xy^2+20x^3 & -60x^2y -20y^3 +5 
\end{array}
\right)
\end{eqnarray}
finally 
\[
\vec{b} 
= -\vec\nabla\cdot \bm\sigma
= 
\left(
\begin{array}{c}
-120xy + 120xy \\
60y^2 + 60x^2 -60x^2 -60y^2
\end{array}
\right)
=
\left(
\begin{array}{c}
0 \\
0
\end{array}
\right)
\]
This is particularly convenient...

It is implemented in \stone 14, 18 and 115.




