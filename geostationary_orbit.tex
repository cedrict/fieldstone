
\begin{center}
\includegraphics[width=3cm]{images/sphcoord.png}
\end{center}

In spherical coordinates\footnote{\url{https://en.wikipedia.org/wiki/Spherical_coordinate_system}}
the position, velocity and acceleration of a point are given by
\begin{eqnarray}
\vec{r} &=& r \vec{e}_r \nonumber\\
\vec{\upnu} &=& \dot{r} \vec{e}_r + r \dot\theta   \vec{e}_\theta + r \dot{\phi}  \sin\theta \vec{e}_\phi \nonumber\\
\vec{a} &=& 
(\ddot{r} - r \dot{\theta}^2  - r \dot{\phi}^2 \sin^2 \theta )\vec{e}_r \nonumber\\
&&+(r \ddot\theta + 2 \dot{r} \dot{\theta} - r \dot{\phi}^2 \sin\theta \cos \theta) \vec{e}_\theta  \nonumber\\
&&+(r \ddot{\phi} \sin\theta + 2 \dot{r}\dot{\phi} \sin\theta + 2r \dot{\theta} \dot{\phi} \cos\theta)
\vec{e}_\phi \nonumber
\end{eqnarray}

For an orbit at constant height ($r=R$, $\dot{r}=0$), and constant angular velocities (i.e. $\ddot{\theta}=0$ 
and $\ddot{\phi}=0$) we arrive at
\begin{eqnarray}
\vec{\upnu} &=&  R \dot\theta  \; \vec{e}_\theta + R \dot{\phi}  \sin\theta \; \vec{e}_\phi \\
\vec{a} &=& 
( - R \dot{\theta}^2  - R \dot{\phi}^2 \sin^2 \theta ) \;\vec{e}_r 
+(   - R \dot{\phi}^2 \sin\theta \cos \theta) \; \vec{e}_\theta 
+( 2R \dot{\theta} \dot{\phi} \cos\theta)  \;\vec{e}_\phi
\end{eqnarray}
If the orbit is the equatorial plane, we have $\theta=\pi/2$, $\sin\theta=1$, $\cos\theta=0$ (and of course $\dot\theta=0$) so now
\begin{eqnarray}
\vec{\upnu} &=&   R \dot{\phi} \; \vec{e}_\phi \\
\vec{a} &=&    - R \dot{\phi}^2  \; \vec{e}_r 
\end{eqnarray}
The acceleration is the so-called centripetal\footnote{Moving or tending to move towards a centre, as opposed to centrifugal:     moving or tending to move away from a centre.} acceleration. 
We coin $\dot{\phi}=\omega$ is the constant angular velocity.

 
`` A centripetal force (from Latin centrum, "center" and petere, "to seek") is a force that makes a body follow a curved path. The direction of the centripetal force is always orthogonal to the motion of the body and towards the fixed point of the instantaneous center of curvature of the path. 

One common example involving centripetal force is the case in which a body moves with uniform speed along a circular path. The centripetal force is directed at right angles to the motion and also along the radius towards the centre of the circular path. The mathematical description was derived in 1659 by the Dutch physicist Christiaan Huygens.

In the case of an object that is swinging around on the end of a rope in a horizontal plane, the centripetal force on the object is supplied by the tension of the rope. The rope example is an example involving a 'pull' force. 

Newton's idea of a centripetal force corresponds to what is nowadays referred to as a central force. When a satellite is in orbit around a planet, gravity is considered to be a centripetal force even though in the case of eccentric orbits, the gravitational force is directed towards the focus, and not towards the instantaneous center of curvature.''\footnote{\url{https://en.wikipedia.org/wiki/Centripetal_force}}

``
A geostationary orbit, also referred to as a geosynchronous equatorial orbit, is a circular geosynchronous orbit 35,786 km in altitude above Earth's equator, 42,164 km in radius from Earth's center, and following the direction of Earth's rotation. An object in such an orbit has an orbital period equal to Earth's rotational period, one sidereal day, and so to ground observers it appears motionless, in a fixed position in the sky. ''\footnote{\url{https://en.wikipedia.org/wiki/Geostationary_orbit}}

The centripetal force of an orbiting body of mass $m$ is then 
\[
\vec{F}_c = m \vec{a} =- m R \dot{\phi}^2 \; \vec{e}_r 
\]
The gravitational force is 
\[
\vec{F}_g = - \frac{{\cal G } M m }{R^2} \vec{e}_r
\]
From Newton's second law of motion (sum of forces = mass * acceleration) we can write
\[
m \vec{a} =- m R \dot{\phi}^2 \; \vec{e}_r  = - \frac{{\cal G } M m }{R^2} \vec{e}_r
\]
or, 
\[
R \dot{\phi}^2 
= \frac{{\cal G } M  }{R^2} 
\]
We have $\dot{\phi}= \omega = \frac{v}{R}$
so 
\[
R \left(\frac{v}{R} \right)^2
= \frac{{\cal G } M  }{R^2} 
\]
\[
v^2
= \frac{{\cal G } M  }{R} 
\]
The velocity is given by $2 \pi R / T$ where $T$ is the desired period so now
\[
\left( \frac{2 \pi R}{ T} \right)^2
= \frac{{\cal G } M  }{R} 
\]
where $T$ is the orbital period (i.e. one sidereal day), and is equal to 86164.09054 s.
In the end
\[
R = \left( \frac{ {\cal G} M T^2}{4 \pi^2} \right)^{1/3}
\]

The resulting orbital radius is 42,164 kilometres. Subtracting the Earth's equatorial radius, 6,378 kilometres, gives the altitude of 35,786 kilometres.

The orbital speed is calculated by multiplying the angular speed by the orbital radius:
\[
v = \omega R \simeq 3074.6 m/s
\]


