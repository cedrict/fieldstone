\begin{flushright} {\tiny {\color{gray} \tt pair\_q2qm1.tex}} \end{flushright}
%~~~~~~~~~~~~~~~~~~~~~~~~~~~~~~~~~~~~~~~~~~~~~~~~~~~~~~~~~~~~~~~~~~~~~~~~~~~~~~~~~~~~~~~~~~~~~~~~~~


The ${\bm Q}_2\times Q_{-1}$ element is shown in Table~3.13-2 of Gresho \& Sani's book \cite{grsa}, 
and discussed in Section~3.13.6b of the book too. It is {\it not} LBB stable
and has one chequerboard pressure mode.

It is used (alongside many other element pairs) in \textcite{chgs02} (2002) in the context of 
a flow benchmark in a 2D box. The authors conclude that
\begin{displayquote}
{\color{darkgray}
[...] the Q2-Q-1 element fared
slightly better than the Q2-P-1 . Most surprising, though, were the good results obtained with
the 'old' Taylor–Hood element, Q2-Q1 .}
\end{displayquote}

It is also used in \textcite{grsu02} (2002) on a similar benchmark setup (8:1 thermal 
cavity problem) along with ${\bm Q}_1\times P_0$, ${\bm Q}_2\times P_{-1}$ and 
${\bm Q}_2\times Q_1$. The authors state that Q2Q-1 has div- stability problems
but 
\begin{displayquote}
{\color{darkgray}
produces excellent results and is still useful in general.
[...]
If the pesky-mode instability could be effiently dealt with, then the Q2xQ-1 element
should be employed over the Q2xP-1 -especially in 3D (we believe).}
\end{displayquote}

Authors mention that it was also used in \textcite{dejo83} (1983) and that it ``performed
extremely well.''

In \textcite{zhmu16} a special version of the ${\bm Q}_k \times Q_{-(k-1)}$ element is presented.
