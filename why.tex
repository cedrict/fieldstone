\begin{flushright} {\tiny {\color{gray} \tt why.tex}} \end{flushright}

The Finite Element Method (FEM) is by no means the only method 
to solve PDEs in geodynamics, nor it is necessarily always the best one.
Other methods are employed very successfully, such as the Finite Difference 
Method (FDM), the Finite Volume Method (FVM), and to a lesser extent
the Discrete Element Method (DEM) \cite{tasy05,egho07,egsc07,funi14,jitd23}, 
the Lattice-Boltzmann method \cite{hupc08}, the Rigid Element Method \cite{lacj15},  
or the Element Free Galerkin Method (EFGM) \cite{hans03}.
I have been using FEM since 2008 and I do not have real 
experience to speak of in FVM or FDM (except for chapter 11)
so I concentrate in this book 
on what I know best. 


