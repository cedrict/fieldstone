\begin{flushright} {\tiny {\color{gray} dimensionless\_equations2.tex.tex}} \end{flushright}
%~~~~~~~~~~~~~~~~~~~~~~~~~~~~~~~~~~~~~~~~~~~~~~~~~~~~~~~~~~~~~~~~~~~~~~~~~~~~~~~~~~~~~~~~~~~~~~~~~~

Let us now consider a box heated from below and cooled from above. 
We define 4 fundamental reference quantities:
\begin{itemize}
\item a length $L_{ref}$ (\si{\metre}), ({\color{violet} $L$})
\item a temperature $T_{ref}$ (\si{\kelvin}), ({\color{violet} $\uptheta$})
\item a viscosity $\eta_{ref}$ (\si{\pascal\second}), ({\color{violet} $ML^{-1}T^{-1}$})
\item a thermal diffusion coefficient $\kappa_{ref}$ (\si{\square\metre\per\second}), ({\color{violet} $L^2T^{-1}$})
\end{itemize}
From these reference quantities one can form secondary ones, such as
\begin{itemize}
\item a time $t_{ref} = L_{ref}^2 / \kappa_{ref}$ (aka the diffusion time)
\item a velocity $\upnu_{ref} = L_{ref} / t_{ref} = \kappa_{ref}/L_{ref}$
\item an acceleration $g_{ref} = \upnu_{ref} / t_{ref} = \kappa_{ref}^2/L_{ref}^3$
\item a strain rate $\dot{\varepsilon}_{ref} = t_{ref}^{-1} = \kappa_{ref} / L_{ref}^2$
\item a pressure $p_{ref} = \eta_{ref} \dot{\varepsilon}_{ref} = \eta_{ref} t_{ref}^{-1}$
\item a reference density $\rho_{ref} = \eta_{ref} L_{ref} t_{ref}/L_{ref}^3 = \eta_{ref} L_{ref}^{-2} t_{ref}$
\item a reference mass $M_{ref} = \eta_{ref} L_{ref} t_{ref}$
\item a reference energy $E_{ref} = \eta_{ref} L_{ref} t_{ref} \frac{L_{ref}^2}{t_{ref}^2} 
= \eta_{ref} \frac{L_{ref}^3}{t_{ref}}$
\item a reference heat conductivity\footnote{Units: W/m/K} $k_{ref}= E_{ref}/t_{ref}/L_{ref}/T_{ref}
= \eta_{ref} L_{ref}^2/t_{ref}^2/T_{ref}$
\item a reference heat capacity\footnote{Units: J/kg/K} 
$C_{ref}=E_{ref}/M_{ref}/T_{ref} = L_{ref}^2 /t_{ref}^2/T_{ref} $
\item a reference heat production coefficient\footnote{Units: W/kg}
$H_{ref}= E_{ref}/t_{ref}/M_{ref} = \frac{L_{ref}^2}{t_{ref}^3}$
\item a reference heat flux\footnote{Units: \si{\watt\per\square\meter}, or \si{\kg\per\cubic\second}} 
$q_{ref}= \eta_{ref} L_{ref} t_{ref}^{-2}$
\end{itemize}
We define {\color{teal}dimensionless} quantities as follows:
\begin{equation}
{\color{teal}x} = \frac{x}{L_{ref}}
\quad
{\color{teal}\vec\upnu} = \frac{\vec\upnu}{\upnu_{ref}}
\quad
{\color{teal}t} = \frac{t}{t_{ref}}
\quad
{\color{teal} \eta} = \frac{\eta}{\eta_{ref}}
\quad
{\color{teal} g} =\frac{g}{g_{ref}}
\quad
{\color{teal} k} = \frac{k}{k_{ref}}
\quad
{\color{teal} C_p} = \frac{C_p}{C_{ref}}
\quad
{\color{teal} \rho} =\frac{\rho}{\rho_{\text{ref}}}
\end{equation}

\begin{equation}
{\color{teal} H} = \frac{H}{H_{ref}}
\quad
{\color{teal} \vec\nabla} = L_{ref}\; \vec\nabla
\quad
{\color{teal} \partial_t} = t_{ref}\; \partial_t 
\quad
{\color{teal} T} = \frac{T}{T_{ref}}
\quad
{\color{teal} \dot{\varepsilon}} = \dot{\varepsilon} \; t_{ref}
\label{eq:physics_adimrels}
\end{equation}
We start from the standard Navier-Stokes equation\footnote{\url{https://en.wikipedia.org/wiki/Navier-Stokes_equations}}
\[
\rho \frac{D \vec\upnu}{D t}
=
-\vec\nabla p + \vec\nabla \cdot (2 \eta \dot{\bm\varepsilon})
+ \rho \vec{g} 
\]
and assume that the density is temperature-dependent (Boussinesq approximation) so that
\[
\rho \frac{D \vec\upnu}{D t}
=
-\vec\nabla p + \vec\nabla \cdot (2 \eta \dot{\bm\varepsilon})
+ \rho_0(1-\alpha T) \vec{g} 
\]
and remove the hydrostatic pressure (although we keep using $p$ for simplicity, $p$ is now the dynamic pressure):
\[
\rho \frac{D \vec\upnu}{D t}
=
-\vec\nabla p + \vec\nabla \cdot (2 \eta \dot{\bm\varepsilon})
- \rho_0\alpha T \vec{g} 
\]
We divide this equation by $p_{ref}=\eta_{ref}\dot{\varepsilon}_{ref}$:
\[
\frac{1}{\eta_{ref}\dot{\varepsilon}_{ref}} \rho
\frac{D \vec\upnu}{D t} 
=
-\vec\nabla {\color{teal}p} + \vec\nabla \cdot 2 \frac{\eta}{\eta_{ref}} \frac{\dot{\bm\varepsilon}}{\dot{\varepsilon}_{ref}}
- \frac{\rho_0\alpha T \vec{g}}{\eta_{ref} \dot{\varepsilon}_{ref}} 
\]
Let us call $\vec{e}$ the positive vertical vector ($\vec{e}_z$ in Cartesian coordinates, $\vec{e}_r$ in spherical coordinates), then 
$\vec{g} = -g_0 \vec{e}$ and we can write  (using $\dot{\varepsilon}_{ref}=\kappa_{ref}/L^2_{ref}$)
\[
\frac{1 }{\eta_{ref}\dot{\varepsilon}_{ref}}
(\rho_{ref}{\color{teal}\rho})\frac{D(\upnu_{ref} {\color{teal}\vec\upnu})}{D t}
=
-\vec\nabla {\color{teal}p} + \vec\nabla \cdot 2 {\color{teal} \eta} {\color{teal} \dot{\bm\varepsilon}}
+ \frac{\rho_0\alpha T g_0 }{\eta_{ref} (\kappa_{ref}/ L_{ref}^2)} \vec{e}
\]
Finally, dividing by $L_{ref}^{-1}$ (i.e. multiplying by $L_{ref}$) yields
\[
\frac{\upnu_{ref} \rho_{ref} L_{ref}}{\eta_{ref}\dot{\varepsilon}_{ref}  }
{\color{teal}\rho} \frac{{\color{teal} D \vec\upnu}}{ t_{ref} {\color{teal} D t}}
=
-{\color{teal} \vec\nabla} {\color{teal}p} + {\color{teal} \vec\nabla} \cdot 2 {\color{teal} \eta} {\color{teal} \dot{\bm\varepsilon}}
+ \frac{\rho_0\alpha ({\color{teal}T}T_{ref}) g_0 L_{ref}^3}{\eta_{ref} \kappa_{ref}} \vec{e}
\]
and finally (using $\upnu_{ref}=L_{ref}/t_{ref}$) 
\[
\frac{\rho_{ref} \kappa_{ref}}{\eta_{ref}}
{\color{teal}\rho} \frac{{\color{teal} D \vec\upnu}}{\color{teal} D t}
=
-{\color{teal} \vec\nabla} {\color{teal}p} + {\color{teal} \vec\nabla} \cdot 2 {\color{teal} \eta} {\color{teal} \dot{\bm\varepsilon}}
+ \frac{\rho_0\alpha T_{ref} g_0 L_{ref}^3}{\eta_{ref} \kappa_{ref}} {\color{teal} T} \vec{e}
\]
In the context of a system with a temperature difference $\Delta T$ 
between the bottom and top boundaries separated by a distance $H$, one would then take $T_{ref} = \Delta T$ 
and $L_{ref}=H$ so that the equation becomes:
\[
\underbrace{\frac{\rho_{ref} \kappa_{ref}}{\eta_{ref}}}_{\Prnb^{-1}}
{\color{teal}\rho} \frac{{\color{teal} D \vec\upnu}}{\color{teal} D t}
=
-{\color{teal} \vec\nabla} {\color{teal}p} + {\color{teal} \vec\nabla} \cdot 2 {\color{teal} \eta} {\color{teal} \dot{\bm\varepsilon}}
+ \underbrace{\frac{\rho_0\alpha \Delta T g_0 H^3}{\eta_{ref} \kappa_{ref}} }_{\Ranb} {\color{teal} T}\vec{e}
\]
and we obviously recover the classical definition of the Rayleigh number.

\begin{mdframed}[backgroundcolor=blue!5]
\[
\frac{1}{\Prnb}
{\color{teal}\rho} \frac{{\color{teal} D \vec\upnu}}{\color{teal} D t}
=
-{\color{teal} \vec\nabla} {\color{teal}p} + {\color{teal} \vec\nabla} \cdot 2 {\color{teal} \eta} {\color{teal} \dot{\bm\varepsilon}}
+ \Ranb {\color{teal} T}\vec{e}
\]
\end{mdframed}


On the left side of the equation we recognize the (inverse of the) Prandlt number $\Prnb=\frac{\eta}{\rho \kappa}$. 
We can estimate the dimensionless number before the inertial term for Earth
geodynamics:
\[
\Prnb \simeq \frac{10^{20-23}}{3000 \cdot 10^{-6}} >> 10^{23}
\]
Its inverse is then extremely small and this is why we neglect the inertial terms
in mantle modelling.


Note that if the fluid is isoviscous, one can then set $\eta_{ref}=\eta=\eta_0$ and then ${\color{teal}\eta}=1$ 
%and then 
%\[
%-{\color{teal} \vec\nabla} {\color{teal}p} + {\color{teal} \vec\nabla} \cdot 2  {\color{teal} \dot{\bm\varepsilon}}
%+ \Ranb {\color{teal} T} \vec{e}= \vec{0}
%\]

Turning now to the continuity equation $\vec\nabla \cdot\vec\upnu = 0$,
it is trivial to show that  ${\color{teal} \vec\nabla } \cdot  {\color{teal}\vec\upnu} = 0$.
Finally, starting from the simple heat transport equation:
\[
\frac{\partial T}{\partial t} + \vec\upnu\cdot\vec\nabla T = \kappa \Delta T
\]
We divide each side by $T_{ref}$ so that 
\[
\frac{\partial {\color{teal}T}}{\partial t} + \vec\upnu\cdot\vec\nabla {\color{teal}T} = \kappa \Delta {\color{teal}T}
\]
We now divide each side by the reference velocity $\upnu_{ref}$ 
and we obtain
\[
\frac{L_{ref}}{\kappa_{ref}} \frac{\partial {\color{teal}T}}{\partial t} 
+ {\color{teal} \vec\upnu} \cdot\vec\nabla {\color{teal}T} 
=  \frac{L_{ref}}{\kappa_{ref}}  \kappa \Delta {\color{teal}T}
\]
We multiply each side by $L_{ref}$ and we finally get
\[
\frac{L_{ref}^2}{\kappa_{ref}} 
\frac{\partial {\color{teal}T}}{\partial t}
+ {\color{teal} \vec\upnu} \cdot  {\color{teal}\vec\nabla} {\color{teal}T} =  {\color{teal} \kappa} {\color{teal}\Delta} {\color{teal}T}
\]
and finally
\[
\frac{ {\color{teal} \partial T}}{ {\color{teal} \partial t}} 
+ {\color{teal} \vec\upnu} \cdot  {\color{teal}\vec\nabla} {\color{teal}T} =  {\color{teal} \kappa} {\color{teal}\Delta} {\color{teal}T}
\]
The set of dimensionless equations is then:

\begin{mdframed}[backgroundcolor=blue!5]
\begin{eqnarray}
-{\color{teal} \vec\nabla} {\color{teal}p} + {\color{teal} \vec\nabla} \cdot 2  {\color{teal}\eta \dot{\bm\varepsilon}}
+ \Ranb  {\color{teal} T} \vec{e} &=& \vec{0} \label{eq:adimmm1}\\
{\color{teal} \vec\nabla } \cdot  {\color{teal}\vec\upnu} &=& 0 \label{eq:adimmm2}\\
\frac{ {\color{teal} \partial T}}{ {\color{teal} \partial t}} 
+ {\color{teal} \vec\upnu} \cdot  {\color{teal}\vec\nabla} {\color{teal}T} 
&=&  {\color{teal} \kappa} {\color{teal}\Delta} {\color{teal}T} \label{eq:adimmm3}
\end{eqnarray}
\end{mdframed}

\index{general}{Extended Boussinesq Approximation}
\index{general}{EBA}
Looking now at the Extended Boussinesq Approximation (EBA), we have to conside two additional terms in the 
energy equation:
\begin{itemize}
\item the shear heating $\Phi$ (See Eq.\eqref{eq:physicsshearheating}) 
$\Phi = 2 \eta  \dot{\bm \varepsilon}^d : \dot{\bm \varepsilon}^d$ 
\item the adiabatic heating $\alpha T \vec\upnu\cdot\vec\nabla p$
\end{itemize}
We start this time from 
\[
\rho C_p \left(\frac{\partial T}{\partial t} + \vec \upnu\cdot\vec\nabla T\right)
- \vec\nabla\cdot k\vec\nabla T 
= \rho H + 2\eta\dot{\bm \varepsilon}^d : \dot{\bm \varepsilon}^d 
+\alpha T  \vec\upnu \cdot \vec\nabla p 
\]

\[
\rho C_p \left(\frac{\partial T}{\partial t} + \vec \upnu\cdot\vec\nabla T\right)
- \vec\nabla\cdot k\vec\nabla T 
= \rho H + 2\eta\dot{\bm \varepsilon}^d : \dot{\bm \varepsilon}^d 
+\alpha T  \vec\upnu \cdot \vec\nabla p 
\]

\[
\rho C_p \frac{T_{ref}}{t_{ref}} \left(\frac{\partial {\color{teal}T}}{\partial {\color{teal}t}} 
+ {\color{teal} \vec \upnu}\cdot {\color{teal}\vec\nabla} {\color{teal}T}\right)
- \frac{T_{ref}}{L_{ref}^2} {\color{teal}\vec\nabla}\cdot k
{\color{teal}\vec\nabla} {\color{teal}T} 
= \rho_{ref} {\color{teal}\rho} H + \frac{\eta_{ref}}{t_{ref}^2} 2 {\color{teal}\eta} 
{\color{teal}\dot{\bm \varepsilon}^d} : 
{\color{teal}\dot{\bm \varepsilon}^d} 
+ \frac{p_{ref}}{t_{ref}} {\color{teal} \alpha T}  
{\color{teal}\vec\upnu} \cdot {\color{teal} \vec\nabla}
{\color{teal} p} 
\]
we then use $p_{ref} = \eta_{ref} t_{ref}^{-1}$ and
$\rho_{ref} = \eta_{ref} L_{ref}^{-2} t_{ref}$
\[
\rho_{ref} C_{ref} \frac{T_{ref}}{t_{ref}} {\color{teal} \rho} {\color{teal}C_p} 
\left(\frac{ {\color{teal} \partial T}}{ {\color{teal} \partial t}} 
+ {\color{teal} \vec \upnu}\cdot {\color{teal}\vec\nabla} {\color{teal}T}\right)
- \frac{T_{ref} k_{ref}}{L_{ref}^2} 
{\color{teal}\vec\nabla}\cdot {\color{teal} k}
{\color{teal}\vec\nabla} {\color{teal}T} 
= \frac{\eta_{ref}}{L_{ref}^2} t_{ref}  \frac{L_{ref}^2}{t_{ref}^3} 
{\color{teal}\rho} {\color{teal} H} 
+ \frac{\eta_{ref}}{t_{ref}^2} 2 {\color{teal}\eta} 
{\color{teal}\dot{\bm \varepsilon}^d} : 
{\color{teal}\dot{\bm \varepsilon}^d} 
+ \frac{\eta_{ref}}{t_{ref}^2} {\color{teal} \alpha T}  
{\color{teal}\vec\upnu} \cdot {\color{teal} \vec\nabla}
{\color{teal} p} 
\]
or, multiplying all by $t^2_{ref}/\eta_{ref}$:
\[
\frac{t_{ref}^2}{\eta_{ref}}
\rho_{ref} C_{ref} \frac{T_{ref}}{t_{ref}} {\color{teal} \rho} {\color{teal}C_p} 
\left(\frac{ {\color{teal} \partial T}}{ {\color{teal} \partial t}} 
+ {\color{teal} \vec \upnu}\cdot {\color{teal}\vec\nabla} {\color{teal}T}\right)
- \frac{t_{ref}^2}{\eta_{ref}} \frac{T_{ref} k_{ref}}{L_{ref}^2} 
{\color{teal}\vec\nabla}\cdot {\color{teal} k}
{\color{teal}\vec\nabla} {\color{teal}T} 
=  
{\color{teal}\rho} {\color{teal} H} 
+  2 {\color{teal}\eta} 
{\color{teal}\dot{\bm \varepsilon}^d} : 
{\color{teal}\dot{\bm \varepsilon}^d} 
+  {\color{teal} \alpha T}  
{\color{teal}\vec\upnu} \cdot {\color{teal} \vec\nabla}
{\color{teal} p} 
\]
We then make use of $C_{ref}=L_{ref}^2 /t_{ref}^2/T_{ref}$
and $k_{ref}= \eta_{ref} L_{ref}^2/t_{ref}^2/T_{ref}$
to arrive at
\[
{\color{teal} \rho} {\color{teal}C_p} 
\left(\frac{ {\color{teal} \partial T}}{ {\color{teal} \partial t}} 
+ {\color{teal} \vec \upnu}\cdot {\color{teal}\vec\nabla} {\color{teal}T}\right)
- 
{\color{teal}\vec\nabla}\cdot {\color{teal} k}
{\color{teal}\vec\nabla} {\color{teal}T} 
=  
{\color{teal}\rho} {\color{teal} H} 
+  2 {\color{teal}\eta} 
{\color{teal}\dot{\bm \varepsilon}^d} : 
{\color{teal}\dot{\bm \varepsilon}^d} 
+  {\color{teal} \alpha T}  
{\color{teal}\vec\upnu} \cdot {\color{teal} \vec\nabla}
{\color{teal} p} 
\]



