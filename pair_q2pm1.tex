\begin{flushright} {\tiny {\color{gray} \tt pair\_q2pm1.tex}} \end{flushright}
%~~~~~~~~~~~~~~~~~~~~~~~~~~~~~~~~~~~~~~~~~~~~~~~~~~~~~~~~~~~~~~~~~~~~~~~~~~~~~~~~~~~~~~~~~~~~~~~~~~

According to \textcite{bobf08} 
{\color{darkgray} ``This element was apparently discovered 
around a blackboard at the Banff Conference on Finite Elements in 
Flow Problems (1979)''}.

\begin{center}
\begin{flushright} {\tiny {\color{gray} \tt (tikz\_p2pm1.tex)}} \end{flushright}
%~~~~~~~~~~~~~~~~~~~~~~~~~~~~~~~~~~~~~~~~~~~~~~~~~~~~~~~~~~~~~~~~~~~~~~~~~~~~~~~~~~~~~~~~~~~~~~~~~~


%\begin{center}
\begin{tikzpicture}
%\draw[fill=gray!23,gray!23](0,0) rectangle (5,5);
%\draw[step=0.5cm,gray,very thin] (0,0) grid (5,5); %background grid
\draw[thick] (1,1) -- (4,1) -- (4,3) -- (1,3) -- cycle;  
\node[] at (0.7,0.8) {0};
\node[] at (4.3,0.8) {1};
\node[] at (4.25,3.1) {2};
\node[] at (0.8,3.2) {3};
\draw[black,fill=teal] (1,1)     circle (2pt); 
\draw[black,fill=teal] (4,1)   circle (2pt); 
\draw[black,fill=teal] (4,3)   circle (2pt); 
\draw[black,fill=teal] (1,3) circle (2pt); 
\draw[black,fill=teal] (2.5,1) circle (2pt) ; 
\draw[black,fill=teal] (2.5,3) circle (2pt) ; 
\draw[black,fill=teal] (1,2) circle (2pt) ; 
\draw[black,fill=teal] (4,2) circle (2pt) ; 

\draw[violet] (2.5,2) circle (4pt);
\draw[violet] (3,2) circle (4pt);
\draw[violet] (2.5,2.5) circle (4pt);

\draw[black,fill=teal] (3.1,0.2) circle (2pt); 
\node[] at (3.4,0.2) {$\vec\upnu$};
\draw[violet] (4.1,0.2) circle (4pt); 
\node[] at (4.4,0.2) {$p$};
\node[] at (2.5,3.85) {9 vel. nodes, 3 press. nodes};
\end{tikzpicture}
%\end{center}

\end{center}

This element is crowned ``probably the most accurate 2D element''
in \textcite{grsa}.

It is characterised by piecewise bi/triquadratic velocities, 
and piecewise linear discontinuous polynomial pressure. 
The element satisfies the inf-sup condition, see p.~211 of \textcite{hugh}, or 
p.~138 of \textcite{elsw}.
It is used in \textcite{vavs89} (1989) for steady laminar flow in a curved tube. 

When using this element one must be aware of the fact that there are 
two possible choices for the definitions of the pressure space 
(mapped and un-mapped) as explained
in \textcite{boga02} (2002) 
See \stone~76,120 for their implementation.
\textcite{bobf08} state: 
\begin{displayquote}
{\color{darkgray}
On a general quadrilateral mesh, the [pressure] space 
can be defined in two different ways: either [it] 
consists of (discontinuous) piecewise linear functions, or it is built
by considering three linear shape functions on the reference unit square and mapping
them to the general elements like it is usually done for continuous 
finite elements. [...] We shall refer to the first possibility as 
unmapped pressure approach and to the second one as mapped pressure approach.

[...] 

So far, we have shown that either the 
unmapped and the mapped pressure 
approach gives rise to a stable ${\bm Q}_2\times P_{-1}$ scheme. 
However, as a consequence of the
results proved in \textcite{arbf02} (2002), we have that the mapped 
pressure approach cannot achieve 
optimal approximation order. Namely, the unmapped pressure space 
provides a second-order convergence 
in $L_2$, while the mapped one achieves only ${\cal O}(h)$ in the same norm.}
\end{displayquote}

See also discussion about mapped/unmapped in \textcite{bobf13} and 
in Section 3.6.4 of \textcite{john16}.

This element is mentioned in \textcite{kaus10} (2010) and \textcite{pefc89} (1989) 
and it is used in \textcite{freh14} (2014) to study 3D fold growth rates 
(see online supplementary material) and in \textcite{schm08} (2008).

Note that the serendipity version of this pair, 
i.e. ${\bm Q}_2^{(20)}\times P_{-1}$ is also LBB stable
as shown in p180 of Reddy \cite{reddybook2}.


