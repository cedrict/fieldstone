
In what follows $\vec\upnu$ now stands for the displacement vector, i.e. 
with units of length, not velocity. 
As before, the displacement inside an element is given by 
\begin{equation}
{\vec \upnu}^h({\vec r})=\sum_{i=1}^{m_v} N_i({\vec r})\;  {\vec \upnu}_i
\label{mixed01}
\end{equation}
where $N_i$ are the polynomial basis functions for the displacement.
Pressure does not appear in the equations so this is not a case of 
mixed FE as for the viscous Stokes flow. 

Other notations are sometimes used for Eqs.(\ref{mixed01}) and (\ref{mixed02}):
\begin{equation}
u^h({\vec r}) = \vec{N} \cdot \vec{u}
\quad\quad\quad\quad
v^h({\vec r}) = \vec{N} \cdot \vec{v}
\quad\quad\quad\quad
w^h({\vec r}) = \vec{N} \cdot \vec{w}
\end{equation} 
where ${\vec \upnu}=(u,v,w)$ and $\vec{N}$ 
is the vector containing all basis functions evaluated at location ${\vec r}$:
\begin{eqnarray}
\vec{N}^v &=& \left( N_1({\vec r}),  N_2({\vec r}),  N_3({\vec r}), \dots  N_{m_v}({\vec r}) \right) \\
\vec{N}^p &=& \left( N_1^p({\vec r}),  N_2^p({\vec r}),  N_3^p({\vec r}), \dots  N_{m_p}^p({\vec r}) \right)
\end{eqnarray}
and with 
\begin{eqnarray}
\vec{u} &=& \left( u_1,  u_2,  u_3, \dots  u_{m_v} \right) \\
\vec{v} &=& \left( v_1,  v_2,  v_3, \dots  v_{m_v} \right) \\
\vec{w} &=& \left( w_1,  w_2,  w_3, \dots  w_{m_v} \right) \\
\end{eqnarray}

%............................................
\paragraph{In three dimensions} We start from
\[
{\bm \sigma} = \lambda (\vec\nabla\cdot \vec\upnu) {\bm 1}+ 2\mu {\bm \varepsilon}
\]
where $\mu$ is the shear modulus and $\lambda$ the Lam{\'e} parameter.

\begin{eqnarray}
\sigma_{xx} &=& (\lambda+2\mu)  \varepsilon_{xx} + \lambda \varepsilon_{yy} + \lambda \varepsilon_{zz} \nn\\
\sigma_{yy} &=& \lambda \varepsilon_{xx} + (\lambda+2\mu)  {\varepsilon}_{yy} + \lambda \varepsilon_{zz}\nn\\
\sigma_{zz} &=& \lambda \varepsilon_{xx} + \lambda \varepsilon_{yy} + (\lambda+2\mu)  {\varepsilon}_{zz} \nn\\
\sigma_{xy} &=& 2\mu  {\varepsilon}_{xy} \nn\\
\sigma_{xz} &=& 2\mu  {\varepsilon}_{xz} \nn\\
\sigma_{yz} &=& 2\mu  {\varepsilon}_{yz} 
\end{eqnarray}
or, 
\[
\vec\sigma =
\left(
\begin{array}{c}
\sigma_{xx}\\ 
\sigma_{yy} \\
\sigma_{zz} \\
\sigma_{xy} \\
\sigma_{xz} \\
\sigma_{yz} 
\end{array}
\right)
=
\left(
\begin{array}{cccccc}
\lambda+2\mu & \lambda & \lambda & 0 & 0 & 0 \\
\lambda & \lambda+2\mu & \lambda & 0 & 0 & 0 \\
\lambda & \lambda & \lambda+2\mu & 0 & 0 & 0 \\
0 & 0 & 0 & \mu & 0 & 0\\
0 & 0 & 0 & 0 & \mu & 0\\
0 & 0 & 0 & 0 & 0 & \mu
\end{array}
\right)
\cdot
\left(
\begin{array}{c}
\varepsilon_{xx} \\
\varepsilon_{yy} \\
\varepsilon_{zz} \\
2\varepsilon_{xy} \\
2\varepsilon_{xz} \\
2\varepsilon_{yz} 
\end{array}
\right)
=\vec\varepsilon
\]
The rest of the procedure is pretty straightforward since it follows the same 
ideas as for the mixed viscous case, except that we here build the $\K$ matrix 
only as follows:
\[
\K=\int_{\Omega_e} {\bm B}^T \cdot {\bm D} \cdot {\bm B} \; d\Omega 
\]




%............................................
\paragraph{In two dimensions} The above relationships simplify to 
\begin{eqnarray}
\sigma_{xx} &=& (\lambda+2\mu)  \varepsilon_{xx} + \lambda \varepsilon_{yy} \\
\sigma_{yy} &=& \lambda \varepsilon_{xx} + (\lambda+2\mu)  \dot{\varepsilon}_{yy} \\
\sigma_{xy} &=& 2\mu  \dot{\varepsilon}_{xy} 
\end{eqnarray}
so 

\[
\vec\sigma =
\left(
\begin{array}{c}
\sigma_{xx}\\ 
\sigma_{yy} \\
\sigma_{xy} 
\end{array}
\right)
=
\left(
\begin{array}{ccc}
\lambda+2\mu & \lambda & 0 \\ 
\lambda & \lambda+2\mu & 0 \\
0 & 0 & \mu 
\end{array}
\right)
\cdot
\left(
\begin{array}{c}
\varepsilon_{xx} \\
\varepsilon_{yy} \\
2\varepsilon_{xy} 
\end{array}
\right)
=\vec\varepsilon
\]







