\index{Lyapunov Time}

%from wiki

Simply put, the Lyapunov time is the characteristic timescale on which a dynamical system is chaotic.
 It is defined as the inverse of a system's largest Lyapunov exponent.

The Lyapunov time mirrors the limits of the predictability of the system. By convention, it is defined as the time for the distance between nearby trajectories of the system to increase by a factor of e. However, measures in terms of 2-foldings and 10-foldings are sometimes found, since they correspond to the loss of one bit of information or one digit of precision respectively.

The Lyapunov exponent or Lyapunov characteristic exponent of a dynamical system is a quantity 
that characterizes the rate of separation of infinitesimally close trajectories. 
Quantitatively, two trajectories in phase space with initial separation $\delta \mathbf{Z}_0$ 
diverge (provided that the divergence can be treated within the linearized approximation) at a rate given by
\[
|\delta \mathbf{Z} (t)|\approx e^{\lambda t}|\delta \mathbf {Z} _{0}| 
\]
where $\lambda$ is the Lyapunov exponent. 

Measuring the Lyapunov exponent or time (or related quantities) is relevant in the context of mantle stirring. 
On the one hand it is argued that the mantle is convecting and very efficient at mixing resulting in a 
somewhat homogenous composition. On the other hand, there is are modeling studies that suggest that
whole-mantle convection can preserve heterogeneity in the presence of well-mixed mantle. 

Two approaches are taken in the literature:

\begin{itemize}
\item using marker advection
\item twin experiments \cite{becr14}
\end{itemize}



\Literature \cite{vazh99,falt02,fasa03,saad11,sato12,becr14}

