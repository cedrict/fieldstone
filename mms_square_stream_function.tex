\begin{flushright} {\tiny {\color{gray} \tt mms\_square\_stream\_function.tex}} \end{flushright}
%~~~~~~~~~~~~~~~~~~~~~~~~~~~~~~~~~~~~~~~~~~~~~~~~~~~~~~~~~~~~~~~~~~~~~~~~~~~~~~~~~~~~~~~~~~~~~~~~~~

I wish to arrive at an analytical formulation for a 2D incompressible flow in 
the square domain $[-1:1]\times[-1:1]$
The fluid has constant viscosity $\eta=1$ and is subject to free slip boundary conditions on all sides.
For reasons that will become clear in what follows I postulate the following stream function:
\begin{equation}
\Psi(x,y)=\sin( m \pi x)\sin( n\pi y)
\end{equation}
We have the velocity being defined as:
\begin{mdframed}[backgroundcolor=blue!5]
\begin{equation}
{\vec \upnu} = (u,v) = \left( \frac{\partial \Psi}{\partial y},-\frac{\partial \Psi}{\partial x} \right) 
= (n \pi \sin (m\pi x)\cos(n\pi y),-m\pi \cos(m\pi x)\sin (n\pi y))
\end{equation}
\end{mdframed}

\begin{center}
\includegraphics[width=5cm]{images/mms/square_streamfunction/vel_2x1}\\
{\captionfont Velocity field for $(m,n)=(2,1)$}
\end{center}

The strain rate components are then:
\begin{eqnarray}
\dot\varepsilon_{xx} &=&  \frac{\partial u}{\partial x} = mn \pi^ 2  \cos (m\pi x)\cos(n\pi y)   \\
\dot\varepsilon_{yy} &=&  \frac{\partial v}{\partial y} = -mn \pi^ 2  \cos (m\pi x)\cos(n\pi y)  \\
2\dot\varepsilon_{xy} &=&  \frac{\partial u}{\partial y} +  \frac{\partial v}{\partial x}    \\
&=&  \frac{\partial^2 \Psi}{\partial y^2} -  \frac{\partial^2 \Psi}{\partial x^2}    \\
&=& -n^2\pi^2 \Psi + m^2 \pi^2 \Psi \\
&=& (m^2-n^2) \pi^2   \sin( m \pi x)\sin( n\pi y)
\end{eqnarray}
Note that if $m=n$ the last term is identically zero, which is not desirable 
(flow is too 'simple')
so in what follows we will prefer $m\neq n$.

It is also easy to verify that $u=0$ on the sides and $v=0$ at the top and bottom and that the 
term $\dot\varepsilon_{xy}$ is nul on all four sides, thereby garanteeing free slip. 

Our choice of stream function yields:
\[
{\nabla}^4 \Psi= 
\frac{\partial^4 \Psi}{\partial x^4}+
\frac{\partial^4 \Psi}{\partial y^4}+
2\frac{\partial^2 \Psi}{\partial x^2 y^2}
=\pi^4 ( m^4 \Psi + n^4 \Psi + 2m^2n^2 \Psi) = (m^4 + n^4 + 2m^2n^2)\pi^4 \Psi
\]

Let us recall Eq.~(\ref{eq:sf1}):
\begin{equation}
{\vec \nabla}^4 \Psi 
=
-\frac{\partial \rho g_y}{\partial x} + \frac{\partial \rho g_x}{\partial y}   
\end{equation}
We assume $g_x=0$ and $g_y=-1$ so that we simply have 
\begin{equation}
\frac{\partial \rho}{\partial x}
=
(m^4 + n^4 + 2m^2n^2)\pi^4 \Psi 
=
(m^4 + n^4 + 2m^2n^2)\pi^4 \sin( m \pi x)\sin( n\pi y)
\end{equation}
so that (assuming the integration constant to be zero):
\[
\rho(x,y) = -\frac{m^4 + n^4 + 2m^2n^2}{m} \pi^3  \cos(m \pi x)\sin(n \pi y)
\]
The $x$-component of the momentum equation is (since $g_x=0$): 
\[
-\frac{\partial p}{\partial x} + 
\frac{\partial^2 u}{\partial x^2}+
\frac{\partial^2 u}{\partial y^2} =
-\frac{\partial p}{\partial x} 
-m^2 n \pi^3 \sin (m\pi x)\cos(n\pi y)
- n^3 \pi^3 \sin (m\pi x)\cos(n\pi y)
=0
\]
so 
\[
\frac{\partial p}{\partial x} =-(m^2n+n^3)\pi^3 \sin (m\pi x)\cos(n \pi y)
\]
and the pressure field is then (once again neglecting the integration constant):
\begin{mdframed}[backgroundcolor=blue!5]
\[
p(x,y)= \frac{m^2n+n^3}{m} \pi^2 \cos (\pi x)\cos(\pi y)
\]
\end{mdframed}
Note that we then have the interesting property that the pressure average 
over the domain is zero, i.e. $\int p dV =0$.





