\begin{flushright} {\tiny {\color{gray} mms\_ABCflow.tex}} \end{flushright}
%~~~~~~~~~~~~~~~~~~~~~~~~~~~~~~~~~~~~~~~~~~~~~~~~~~~~~~~~~~~~~~~~~~~~~~~~~~~~~~~~~~~~~~~~~~~~~~~~~~

The Arnold–Beltrami–Childress (ABC) flow or Gromeka–Arnold–Beltrami–Childress (GABC) flow is a three-dimensional incompressible velocity field. 

\url{https://en.wikipedia.org/wiki/Arnold-Beltrami-Childress_flow}

Its representation in Cartesian coordinates is the following:
\begin{eqnarray}
u &=& A\sin z +C \cos y \nn\\
v &=& B\sin x +A \cos z \nn\\
w &=& C\sin y +B \cos x \nn
\end{eqnarray}
Indeed we have
\[
\vec\nabla \cdot \vec\upnu = 
\frac{\partial u}{\partial x} +
\frac{\partial v}{\partial y} +
\frac{\partial w}{\partial z} 
\]

If we now consider a unit cube as computational domain, 
the following velocity field can then be used instead:
\begin{eqnarray}
u(x,y,z) &=& \sin (\pi z) + \cos (\pi y) \nn\\
v(x,y,z) &=& \sin (\pi x) + \cos (\pi z) \nn\\
w(x,y,z) &=& \sin (\pi y) + \cos (\pi x) \nn
\end{eqnarray}
and the pressure can be for example chosen to be 
\[
p(x,y,z)=\sin (\pi x) \cos(\pi y) \cos (\pi z)
\]
Assuming $\eta=1$, we have
\[
-\vec\nabla p + \vec\nabla \cdot 2\dot{\bm \varepsilon}(\vec\upnu) + \vec{f} = \vec{0}
\]

The velocity gradient ${\bm L}(\vec\upnu)$ is then given by
\[
{\bm L}(\vec\upnu) = \pi
\left(
\begin{array}{ccc}
0 & \cos (\pi x) & -\sin (\pi x) \\
-\sin (\pi y) & 0 & \cos (\pi y)  \\
\cos (\pi z) & -\sin (\pi z) & 0 
\end{array}
\right)
\]
and the strain rate tensor by:
\[
\dot{\bm \varepsilon}(\vec\upnu)
=\frac{1}{2}({\bm L}(\vec\upnu)+{\bm L}(\vec\upnu)^T)
=
\frac{\pi}{2}
\left(
\begin{array}{ccc}
0 &  \cos (\pi x) -\sin (\pi y)  & \cos (\pi z) -\sin (\pi x)\\
\cos (\pi x) -\sin (\pi y)  & 0 & \cos (\pi y)-\sin (\pi z)  \\
\cos (\pi z) -\sin (\pi x)  &  \cos (\pi y)-\sin (\pi z)  & 0
\end{array}
\right)
\]

We assume for simplicity that $\eta=1$ so 
\begin{eqnarray}
\vec\nabla\cdot 2\eta \dot{\bm \varepsilon}(\vec\upnu)
&=&
\pi
\vec\nabla\cdot 
\left(
\begin{array}{ccc}
0 &  \cos (\pi x) -\sin (\pi y)  & \cos (\pi z) -\sin (\pi x)\\
\cos (\pi x) -\sin (\pi y)  & 0 & \cos (\pi y)-\sin (\pi z)  \\
\cos (\pi z) -\sin (\pi x)  &  \cos (\pi y)-\sin (\pi z)  & 0
\end{array}
\right) \nn\\
&=& \pi^2
\left(
\begin{array}{ccc}
-\cos (\pi y) -\sin(\pi z) \\
-\sin (\pi x) -\cos(\pi z) \\
-\cos (\pi x) -\sin(\pi y)
\end{array}
\right) 
\end{eqnarray}
Also,
\[
\vec\nabla p = \pi
\left(
\begin{array}{ccc}
\cos (\pi x) \cos(\pi y) \cos (\pi z)\\
-\sin (\pi x) \sin(\pi y) \cos (\pi z)\\
-\sin (\pi x) \cos(\pi y) \sin (\pi z)
\end{array}
\right)
\]
In the end
\begin{eqnarray}
\vec{f} 
&=& \vec\nabla p - \vec\nabla \cdot 2 \varepsilon(\vec\upnu) \nn\\
&=& 
\pi
\left(
\begin{array}{ccc}
\cos (\pi x) \cos(\pi y) \cos (\pi z)\\
-\sin (\pi x) \sin(\pi y) \cos (\pi z)\\
-\sin (\pi x) \cos(\pi y) \sin (\pi z)
\end{array}
\right)
-
\pi^2
\left(
\begin{array}{ccc}
-\cos (\pi y) -\sin(\pi z) \\
-\sin (\pi x) -\cos(\pi z) \\
-\cos (\pi x) -\sin(\pi y)
\end{array}
\right)  \nn\\
&=& 
\left(
\begin{array}{ccc}
\pi \cos (\pi x) \cos(\pi y) \cos (\pi z) +\pi^2 \cos (\pi y) +\pi^2\sin(\pi z)\\
-\pi\sin (\pi x) \sin(\pi y) \cos (\pi z) +\pi^2 \sin (\pi x) +\pi^2\cos(\pi z)\\
-\pi\sin (\pi x) \cos(\pi y) \sin (\pi z) +\pi^2 \cos (\pi x) +\pi^2\sin(\pi y)
\end{array}
\right) 
\end{eqnarray}

\Literature: \fullcite{zhkl93}
