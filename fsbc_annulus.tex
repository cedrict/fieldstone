\begin{flushright} {\tiny {\color{gray} fsbc\_annulus.tex}} \end{flushright}
%~~~~~~~~~~~~~~~~~~~~~~~~~~~~~~~~~~~~~~~~~~~~~~~~~~~~~~~~~~~~~~~~~~~~~~~~~~~~~~~~~~~~~~~~~~~~~~~~~~

In the context of geodynamical modelling we often wish to prescribed free-slip 
boundary conditions on a given boundary of the domain. If the domain is a rectangle
which sides align with the Cartesian axis, then fixing $\upnu_x=0$ or $\upnu_y=0$
is simple and does indeed insure free-slip boundary conditions. 

However the situation is much more complicated in the case of a curved boundary, 
such as for instance the inner and outer boundaries of an annulus or spherical shell.

If the curved boundary is a circular, the procedure is as follows:
\begin{enumerate}
\item identify the node on the boundary which is to be fixed. 
\item compute its coordinate angle $\theta$ (and $\phi$ in 3D) 
\item do a rotation so as to bring it back onto the x-axis (2D) or z-axis (3D)
\item apply free slip boundary condition (now easy since parallel or perpendicular to axis)
\item rotate back
\end{enumerate}

This technique is implemented in \stone~\ref{f33}, \stone~\ref{f96} and \stone~\ref{f151}.

\paragraph{A few remarks about rotation matrices} 
In a given plane, the counter-clockwise rotation matrix by and angle $\theta$ is defined by 
\[
{\cal R}=
\left(
\begin{array}{cc}
\cos\theta & \sin\theta \\
-\sin\theta & \cos\theta
\end{array}
\right)
\]
The image of vector $\vec{V}$ by a rotation of angle $\theta$ is given by
\[
\vec{V}'={\cal R}\cdot \vec{V}
\]

Coordinate transformations of second-rank tensors involve the very same   
matrix as vector transforms. A transformation of the stress tensor ${\bm \sigma}$ ,
from the reference $xy$-coordinate system to ${\bm \sigma}'$ in a new $x'y'-$system is done as follows:
\[
{\bm \sigma}'={\cal R}\cdot {\bm \sigma}\cdot{\cal R}^T
\]



[from Wikipedia] A basic rotation (also called elemental rotation) is a rotation about one of the axes of a Coordinate system. 
The following three basic rotation matrices rotate vectors by an angle $\alpha$ 
about the x-, y-, or z-axis, in three dimensions, using the right-hand rule which codifies their 
alternating signs. 

\[
{\cal R}_x(\alpha)=
\left(
\begin{array}{ccc}
1 & 0 & 0 \\
0 & \cos\alpha & -\sin\alpha \\
0 & \sin\alpha & \cos\alpha
\end{array}
\right)
\]

\[
{\cal R}_y(\alpha)=
\left(
\begin{array}{ccc}
\cos\alpha & 0 & \sin\alpha \\
0 & 1 & 0 \\
-\sin\alpha & 0 &\cos\alpha
\end{array}
\right)
\]

\[
{\cal R}_z(\alpha)=
\left(
\begin{array}{ccc}
\cos\alpha & -\sin\alpha & 0\\
\sin\alpha & \cos\alpha & 0 \\
0 & 0 & 1 
\end{array}
\right)
\]

In my \elefant code
I first rotate around the $z$ axis by and angle $-\phi$ and then 
around axis $y$ by an angle $-\theta$ in the case of a spherical shell.

\[
{\cal R}_y(-\theta)=
\left(
\begin{array}{ccc}
\cos(-\theta) & 0 & \sin(-\theta) \\
0 & 1 & 0 \\
-\sin(-\theta) & 0 &\cos(-\theta)
\end{array}
\right)
=
\left(
\begin{array}{ccc}
\cos\theta & 0 & -\sin\theta \\
0 & 1 & 0 \\
\sin\theta & 0 &\cos\theta
\end{array}
\right)
\]

\[
{\cal R}_z(-\phi)
=
\left(
\begin{array}{ccc}
\cos(-\phi)& -\sin(-\phi) & 0\\
\sin(-\phi)& \cos(-\phi) & 0 \\
0 & 0 & 1 
\end{array}
\right)
=
\left(
\begin{array}{ccc}
\cos\phi& \sin\phi & 0\\
-\sin\phi& \cos\phi & 0 \\
0 & 0 & 1 
\end{array}
\right)
\]

These are the {\tt Rott} and {\tt Rotp} matrices in the routines.


\Literature
\begin{itemize}
\item 
Note that in some cases applying free slip boundary conditions on a curved boundary with a triangular mesh 
can be problematic as explained in \textcite{ditu13} (2013).
\item \fullcite{ensg82}
\item \fullcite{behr04} in which it is stated:\\
{\it 1. If the slip boundary coincides with a Cartesian coordinate plane, the implementation is trivial,
with the equations corresponding to the normal component of velocity simply being dropped
from the equation system.
2. If the slip boundary does not coincide with a Cartesian coordinate plane, the equations
corresponding to the velocity components at the boundary are locally aligned with the normal-
tangent-bi-tangent coordinate system, and the normal component of velocity is set to zero.
This procedure is described by \textcite{ensg82} (1982), who also advocate the use of consistent
normals for proper mass conservation.}
\end{itemize}
