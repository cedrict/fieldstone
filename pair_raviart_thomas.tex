\begin{flushright} {\tiny {\color{gray} \tt pair\_raviart\_thomas.tex}} \end{flushright}
%~~~~~~~~~~~~~~~~~~~~~~~~~~~~~~~~~~~~~~~~~~~~~~~~~~~~~~~~~~~~~~~~~~~~~~~~~~~~~~~~~~~~~~~~~~~~~~~~~~

- Raviart Thomas 0 RT0 \cite{rath77} ? mentioned/defined/drawn in 4.2.2 of 
Kanschat book. Also exist for quads see 4.2.37 
\textcite{hald03}: ``$P_1^\perp \times P_0$ symbol denotes an element with 
normal velocity nodes in the middle of each edge of the
triangulation [...]. This element, also called low order Raviart–Thomas element 
(Raviart and Thomas, 1977), is based on flux conservation on elements edges and 
the resulting scheme is very close to a finite volume scheme.''

Mentioned in \textcite{john16}, appendix B.3, example B.45: ``the normal component of v 
on each face is a constant. The normal component of functions from RT0 is
continuous across faces of the mesh cells.''

Check \textcite{brfo}

Mentioned in \textcite{chen93a} (1993).

\url{https://defelement.com/elements/raviart-thomas.html}


\url{
https://en.wikipedia.org/wiki/Raviart-Thomas_basis_functions
}

\url{
https://people.tamu.edu/~guermond//M661_FALL_2015/chap7.pdf
}

\url{
https://scicomp.stackexchange.com/questions/20245/raviart-thomas-elements-on-reference-square
}



