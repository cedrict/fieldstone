
When free slip boundary conditions are prescribed in an annulus or
hollow sphere geometry there exists a rotational nullspace, or in other words there exists
a tangential velocity field ('pure rotation') which if added or subtracted to the solution 
does not alter the solution. 

As in the pressure normalisation case (see section \ref{ss_pnorm}), the solution is simple:
\begin{enumerate}
\item fix the tangential velocity at {\it one} node on a boundary, and solve the sytem (the nullspace 
has been removed)
\item post-process the solution to have the velocity field fulfill the required conditions, i.e.
either a zero net angular momentum or a zero net angular velocity of the domain. 
\end{enumerate}

\begin{remark}
In \aspect{} this is available under the option 
"Remove nullspace = angular momentum" and "Remove nullspace = net rotation".
The "angular momentum" option removes a rotation such that the net angular momentum is zero.
The "net rotation" option removes the net rotation of the domain.
\end{remark}

In order to remove the angular momentum, we search for a rotation
vector ${\vec \omega}$ such that
\begin{equation}
\int_\Omega \rho[{\vec r} \times ({\vec v}-{\vec \omega} \times {\vec r})] \; d\vec r= \vec 0
\end{equation}
Recognizing that the angular momentum vector ${\vec H}$ is given by
\begin{equation}
{\vec H} = \int_\Omega \rho{\vec r} \times {\vec v}\; d\vec r
\end{equation}
and that the moment of inertia
(also called inertia tensor)\footnote{\url{https://en.wikipedia.org/wiki/Moment\_of\_inertia}}
 for a continuous body 
 $3\times3$ matrix ${\bm I}$ is given by
\begin{equation}
{\bm I}= 
\int_\Omega \rho(\vec r) [\vec r\cdot\vec r \; \bm 1 - \vec r \times \vec r  ] d\vec r 
\end{equation}
so that the above equation writes:
$
{\vec H}={\bm I}\cdot {\vec \omega}
$
and then ${\vec \omega}={\bm I}^{-1} \cdot {\vec H}$.
A rotation about the rotation vector ${\vec \omega}$ is then subtracted from the velocity 
solution \cite[eq. 26]{zhmt08}:
\begin{equation}
\vec v_{new} = \vec v_{old} - \vec \omega \times \vec r 
\end{equation}

\index{angular velocity} \index{angular momentum}

%...............................
\subsubsection{Three dimensions}

The angular momentum vector is given by:
\begin{equation}
\vec H = \int_\Omega \rho(\vec r) \left( 
\begin{array}{c} 
yw-zv \\ zu-xw \\ xv-yu 
\end{array} \right) d\vec r
\end{equation}
while the inertia tensor for a continuous body is given 
by
\begin{eqnarray}
\bm I
&=&\int_\Omega \rho(\vec r) [\vec r\cdot\vec r \; \bm 1 - \vec r \times \vec r  ] d\vec r \\
&=&\int_\Omega \rho(\vec r) 
\left[
\left(
\begin{array}{ccc}
r^2 & 0 & 0 \\
0 & r^2 & 0 \\
0 & 0 & r^2
\end{array}
\right)
- 
\left(
\begin{array}{ccc}
xx & xy & xz \\
yx & yy & yz \\
zx & zy & zz 
\end{array}
\right)
\right] 
d\vec r \\
&=&\int_\Omega \rho(\vec r) 
\left(
\begin{array}{ccc}
y^2+z^2 & -xy & -xz \\
-yx & x^2+z^2 & -yz \\
-zx & -zy & x^2+y^2 
\end{array}
\right)
d\vec r \\
\end{eqnarray}

The angular velocity\footnote{\url{https://en.wikipedia.org/wiki/Angular_velocity }}
 vector is given by $\vec\omega = \frac{\vec r\times \vec v}{r^2}$
so that the volume-averaged angular velocity of the cylindrical shell is:
\[
<\vec {\omega}> = \frac{1}{|\Omega|} \int_\Omega \frac{{\vec r}\times {\vec v}}{r^2} d\vec r
\]
 Note that it can be straightforwardly computed the same way as the angular momentum
(it is in fact the same equation but without the density):

%-----------------------------
\subsubsection{Two dimensions}

In two dimensions, flow is taking place in the $(x,y)$ plane. This means that $\vec r$ and $\vec v$ are coplanar, 
and therefore that $\vec H$ is perpendicular to the plane, i.e. $\vec H \propto \vec e_z$.
We have then
\begin{equation}
\vec H = \int_\Omega \rho(\vec r) \left( 
\begin{array}{c} 
0 \\ 0 \\ xv-yu 
\end{array} \right) d\vec r
\end{equation}
and 
\begin{equation}
\bm I=
\int_\Omega \rho(\vec r) 
\left(
\begin{array}{ccc}
y^2 & -xy & 0 \\
-yx & x^2 & 0 \\
0 & 0 & x^2+y^2 
\end{array}
\right)
d\vec r 
\end{equation}
The volume-averaged angular velocity is then simply:
\begin{equation}
<\omega_z> = \frac{1}{|\Omega|}\int_\Omega \frac{xv-yu}{r^2}d\vec r
\end{equation}

















