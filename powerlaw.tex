\index{general}{Power Law Rheology}


One of the simplest non-Newtonian viscosity model is the power law model, 
for which the viscosity depends on the (effective) deviatoric strain rate as follows:
\begin{equation}
\eta(\dot{\varepsilon}_e) = K \dot{\varepsilon}_{e}^{n-1}
\qquad \text{or } \qquad
\sigma = 2 K \dot{\varepsilon}_e ^n 
\end{equation}
where $n$ and $K$ are parameters. $n$ is called the power law index. $\dot{\varepsilon}_e$ 
is defined in  \eqref{eq:tauepse} and in the table here above. 
Note that a Newtonian viscosity is recovered when $n=1$. Also $n$ and $K$ may depend on temperature
(see Reddy  \cite[p339]{reddybook2}).

A so-called 'generalised' power law rheology is proposed in Iaffaldano \& bunge (2009) \cite{iabu09}:
\begin{equation}
\eta = K (\dot{\varepsilon}_{e}+\dot{\varepsilon}_0)^{n-1}
\end{equation}
so that in the rigid areas where $\dot{\varepsilon}_e \rightarrow 0$ the rheology 
uses instead a minimum strain rate value $\dot{\varepsilon}_0$.

\Literature: England \& Molnar (1997) \cite{enmo97}
