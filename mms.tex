\index{general}{MMS} 
\index{general}{Method of Manufactured Solutions}

The method of manufactured solutions is a relatively simple way of carrying out
code verification. In essence, one postulates a solution for the PDE at hand (as
well as the proper boundary conditions), inserts it in the PDE and computes the 
corresponding source term. 
The same source term and boundary conditions will then be used in a numerical 
simulation so that the computed solution can be compared with the (postulated)
true analytical solution. 

Examples of this approach are to be found in \cite{dohu03,busa13,bodg06,polp14,polp14b,lopp14,blmp16}.

%-----------------------------------------------------------------------------
\subsubsection{Analytical benchmark I \label{mms1} - "DH"}

Taken from \cite{dohu03}. We consider a two-dimensional problem 
in the square domain $\Omega=[0,1]\times[0,1]$, which possesses a closed-form analytical 
solution. The problem consists of determining the velocity field ${\vec \upnu} = (u,v)$ 
and the pressure $p$ such that 
\begin{eqnarray}
\eta \Delta {\vec \upnu} - {\vec \nabla} p + {\vec b} &=& \vec 0 \quad\quad {\rm in} \; \Omega\\
\vec{\nabla} \cdot \vec{v} &=& 0 \quad\quad {\rm in} \; \Omega\\
\vec{v}&=&\vec{0} \quad\quad {\rm on} \; \Gamma_D
\end{eqnarray}
where the fluid viscosity is taken as $\eta=1$.
The components of the body force $\vec{b}$ are prescribed as 
\begin{eqnarray}
b_x &=& (12 - 24y) x^4 + (-24 + 48y) x^3 + (-48y + 72y^2 - 48 y^3 + 12) x^2 \nonumber\\
    && + (-2 + 24y -72y^2+48y^3)x + 1-4y + 12y^2-8y^3 \nonumber\\ 
b_y &=& (8 - 48y + 48 y^2) x^3 + (-12 + 72y - 72y^2) x^2  \nonumber\\
    && + (4 - 24y + 48y^2 - 48y^3 + 24y^4) x - 12y^2 + 24y^3 - 12y^4  \nonumber
\end{eqnarray}
With this prescribed body force, the exact solution is 
\begin{eqnarray}
u(x,y) &=& x^2(1- x)^2 (2y - 6y^2 + 4y^3)  \nonumber\\
v(x,y) &=& -y^2 (1 - y)^2 (2x - 6x^2 + 4x^3) \nonumber\\
p(x,y) &=& x(1 -x)- 1/6 \nonumber 
\end{eqnarray}
Note that the pressure obeys $\int_{\Omega} p \; d\Omega = 0$.
One can turn to the spatial derivatives of the fields:
\begin{eqnarray}
\dot{\varepsilon}_{xx}=\frac{\partial u}{\partial x} &=&  (2x -6x^2 +4 x^3 ) (2y - 6y^2 + 4y^3)  \\
\dot{\varepsilon}_{yy}=\frac{\partial v}{\partial y} &=&  - (2x -6x^2 +4 x^3 ) (2y - 6y^2 + 4y^3)  \\
\dot{\varepsilon}_{xy}=\frac{1}{2}\left(\frac{\partial u}{\partial y}+\frac{\partial v}{\partial x}\right) 
&=&=\frac{1}{2}\left( x^2(1- x)^2 ( 2-12y+12y^2  ) -y^2 (1-y)^2 (2-12x+12x^2) \right)
\end{eqnarray}
with of course  ${\vec \nabla} \cdot {\vec \upnu} = 0$ and 
\begin{eqnarray}
\frac{\partial p}{\partial x} &=& 1-2x  \\
\frac{\partial p}{\partial y} &=& 0
\end{eqnarray}

The velocity and pressure fields look like:

\begin{center}
\includegraphics[height=4cm]{images/mms/Ex1_Q2Q1_velo.png}
\includegraphics[height=4cm]{images/mms/Ex1_Q2Q1_streamlines.png}
\includegraphics[height=4cm]{images/mms/Ex1_Q2Q1_pres.png}\\
{\small http://ww2.lacan.upc.edu/huerta/exercises/Incompressible/Incompressible\_Ex1.htm}
\end{center}

As shown in \cite{dohu03}, If the LBB condition is not satisfied, spurious oscillations spoil the pressure approximation. 
Figures below show results obtained with a mesh of 20x20 Q1P0 (left) and P1P1 (right) elements:
\begin{center}
\includegraphics[height=5cm]{images/mms/Ex1_Q1P0_pres.png}
\includegraphics[height=5cm]{images/mms/Ex1_P1P1_pres.png}]]
{\small http://ww2.lacan.upc.edu/huerta/exercises/Incompressible/Incompressible\_Ex1.htm}
\end{center}

Taking into account that the proposed problem has got analytical solution, it is easy to analyze convergence of the different pairs of elements:
\begin{center}
\includegraphics[height=7cm]{images/mms/Ex1_conv_qua.png}\\
{\small http://ww2.lacan.upc.edu/huerta/exercises/Incompressible/Incompressible\_Ex1.htm}
\end{center}

One can also compute the stress components:
\begin{eqnarray}
\sigma_{xx} &=&  2x^2(2x - 2)(4y^3 - 6y^2 + 2y) + 4x(-x + 1)^2*(4y^3 - 6y^2 + 2y) - x(-x + 1) + 1/6 \\
\sigma_{xy} &=&  x^2(-x + 1)^2*(12y^2 - 12y + 2) - y^2(-y + 1)^2*(12x^2 - 12x + 2) \\
\sigma_{yy} &=&  -x(-x + 1) - 2y^2(2y - 2)(4x^3 - 6x^2 + 2x) - 4y(-y + 1)^2(4x^3 - 6x^2 + 2x) + 1/6
\end{eqnarray}

All the necessary functions to do this benchmark are in {\tt mms/dh.py}:
\lstinputlisting[language=python]{mms/dh.py}

This benchmark is implemented in ASPECT \cite{aspectmanual} and in {\tt Stones 01}.

%-----------------------------------------------------------------------------
\subsubsection{Analytical benchmark II \label{mms2} - "DB2D"}

Taken from \cite{dobo04,bodg06}. It is for a unit square with $\nu=\mu/\rho=1$ and the smooth exact solution is
\begin{eqnarray}
u(x,y) &=& x+x^2 - 2xy+x^3 - 3xy^2 + x^2y \\
v(x,y) &=& -y-2xy+y^2 -3x^2y + y^3 - xy^2 \\
p(x,y) &=& xy+x+y+x^3y^2 - 4/3
\end{eqnarray}
Note that the pressure obeys $\int_{\Omega} p \; d\Omega = 0$

\begin{eqnarray}
b_x &=& - (1+y-3x^2y^2) \\
b_y &=& - (1-3x-2x^3y) 
\end{eqnarray}

This benchmark is also used in \cite{wosp14}.

%-----------------------------------------------------------------------------
\subsubsection{Analytical benchmark III \label{mms3} - "DB3D"}

This benchmark begins by postulating a polynomial solution 
to the 3D Stokes equation \cite{dobo04}:
\begin{equation}
\vec{\upnu}
=
\left(
\begin{array}{c}
x+x^2+xy+x^3y \\
y + xy + y^2 + x^2 y^2\\
-2z - 3xz - 3yz - 5x^2 yz
\end{array}
\right)
\label{eqbur}
\end{equation}
and
\begin{equation}
p = xyz + x^3 y^3z - 5/32
\end{equation}
While it is then trivial to verify that this velocity field is divergence-free (see here under),  
the corresponding body force of the Stokes equation can be computed by  
inserting this solution into the momentum equation with a given viscosity $\eta(x,y,z)$
(constant or position/velocity/strain rate dependent). 
The domain is a unit cube and velocity boundary conditions 
simply use Eq. (\ref{eqbur}). 
Note that the pressure fulfills 
\[
\int_\Omega p(x,y,z) dV = 0.  
\]
Following \cite{busa13}, the viscosity
is given by the smoothly varying function
\begin{equation}
\eta(x,y,z) = exp(1 - \beta(x(1 - x) + y(1 - y) + z(1 - z)))
\end{equation}
Choosing $\beta=0$ yields a constant velocity $\eta=e^1$ (and greatly simplifies the right-hand side).
One can easily show that the ratio of viscosities $\eta^\star$
in the system follows $\eta^\star=\exp(-3\beta/4)$ so that choosing $\beta=10$ yields
$\eta^\star\simeq 1808$ and $\beta=20$ yields $\eta^\star\simeq 3.269\times10^6$.

The exact form of the rhs is carried out in Stone \ref{f17}.

%-----------------------------------------------------------------------------
\subsubsection{Analytical benchmark IV \label{mms4} - "Bercovier \& Engelman"}

From \cite{been79}. The two-dimensional domain is a unit square. The body forces are:
\begin{eqnarray}
f_x &=& 128[ x^2(x-1)^2 12 (2y-1) + 2 (y-1)(2y-1)y(12x^2-12x+2)  ] \nn\\
f_y &=& 128[ y^2(y-1)^2 12 (2x-1) + 2 (x-1)(2x-1)y(12y^2-12y+2)  ] \nn\\
\end{eqnarray}
The solution is
\begin{eqnarray}
u &=& -256x^2(x-1)^2y(y-1)(2y-1) \nn\\
v &=&  256y^2(y-1)^2x(x-1)(2x-1) \nn\\
p &=& 0 
%p &=& (x-1/2)(y-1/2) 
\end{eqnarray}

\begin{eqnarray}
du/dx &=& 512 (1 - 2x) (-1+x) x(-1+y) y(-1+2y) \\ 
du/dy &=& -256 (-1 + x)^2 x^2 (1 - 6 y + 6 y^2) \\ 
dv/dx &=&  256y^2(y-1)^2x(x-1)(2x-1) \\ 
dv/dy &=& -512 (-1 + x) x (1 - 2 x) (-1 + y) y (-1 + 2 y) \\
\end{eqnarray}

and we can easily verify that $\vec\nabla\cdot\vec\upnu=du/dx+dv/dy=0$.

CHECK RHS !

Another choice with a non-zero pressure:
\begin{eqnarray}
f_x &=& 128[ x^2(x-1)^2 12 (2y-1) + 2 (y-1)(2y-1)y(12x^2-12x+2)  ] + y - 1/2 \nn\\
f_y &=& 128[ y^2(y-1)^2 12 (2x-1) + 2 (x-1)(2x-1)y(12y^2-12y+2)  ] + x - 1/2 \nn\\
\end{eqnarray}
The solution is
\begin{eqnarray}
u &=& -256x^2(x-1)^2y(y-1)(2y-1) \nn\\
v &=&  256y^2(y-1)^2x(x-1)(2x-1) \nn\\
p &=& (x-1/2)(y-1/2) 
\end{eqnarray}


%-----------------------------------------------------------------------------
\subsubsection{Analytical benchmark V \label{mms5} - "VJ1"}

This is taken from Appendix D1 of \cite{john16}.

The domain $\Omega$ is a unit square. We consider the stream function
\[
\phi(x,y)=1000x^2(1-x)^4y^3(1-y)^2
\]
The velocity field is defined by
\begin{eqnarray}
u(x,y) &=&  \partial_y \phi = 1000(x^2(1-x)^4 y^2 (1-y)(3-5y)  ) \\
v(x,y) &=& -\partial_x \phi = 1000(-2x(1-x)^3(1-3x)y^3(1-y)^2)
\end{eqnarray}
and it is easy to verify that $\vec\nabla\cdot\vec v=0$.

The pressure is given by:
\[
p(x,y)=\pi^2( xy^3\cos(2\pi x^2y) - x^2y \sin(2\pi xy)) + \frac{1}{8}
\]

\begin{center}
\includegraphics[width=8cm]{images/mms/mms5}\\
Taken from \cite{john16}.
\end{center}

\bscthesis \index{general}{BSc Thesis}

%-----------------------------------------------------------------------------
\subsubsection{Analytical benchmark VI \label{mms6} - "Ilinca \& Pelletier"}
\index{general}{Poiseuille flow} \index{general}{Shear Heating}

This is taken from \cite{ilpe07}.

Let us consider the Poiseuille flow of a Newtonian fluid. The channel has 
isothermal flat walls located at $y=\pm h$. The velocity distribution is parabolic:
\[
u = u_0 \left(1-\frac{y^2}{h^2} \right) 
\quad\quad\quad
v=0
\]
where $u_0$ is the maximum velocity. The (steady state) temperature field is the solution of
the advection-diffusion equation:
\[
\rho C_p \vec v \cdot \vec\nabla T
= k \Delta T + \Phi
\]
where $\Phi$ is the dissipation function given by
\[
\Phi
=\eta \left[  
2\left(\frac{\partial u}{\partial x} \right)^2 + 
2\left(\frac{\partial v}{\partial y} \right)^2 +
\left( \frac{\partial v}{\partial x} + \frac{\partial u}{\partial y} \right)^2
\right]
=
\eta \left( \frac{\partial u}{\partial y} \right)^2 = 4 \eta \frac{u_0^2 y^2}{h^4}
\]
We logically assume that $T=T(y)$ so that $\partial T/\partial x=0$ and $\vec v \cdot \vec\nabla T=0$.
We then have to solve:
\[
k \frac{\partial^2 T}{\partial y^2} + 4 \eta \frac{u_0^2 y^2}{h^4} = 0
\]
We can integrate twice and use the boundary conditions $T(y=\pm h)=T_0$ to arrive at:
\[
T(y) = T_0 + \frac{1}{3} \frac{\eta u_0^2}{k} \left[ 1-\left(\frac{y}{h}\right)^4  \right]
\]
with a maximum temperature
\[
T_M = T(y=0) = T_0 + \frac{1}{3} \frac{\eta u_0^2}{k} 
\]

%-----------------------------------------------------------------------------
\subsubsection{Analytical benchmark VII \label{mms7} - "grooves"}

This benchmark was designed by Dave May. 
The velocity and pressure fields are given by
\begin{eqnarray}
u(x,y) &=& x^3 y + x^2 + xy + x \nn\\
v(x,y) &=& -\frac{3}{2}x^2y^2 - 2xy - \frac{1}{2}y^2 - y \nn\\
p(x,y) &=& x^2y^2 + xy + 5 + p_0
\end{eqnarray}
where $p_0$ is a constant to be determined based on the type of pressure normalisation.
The viscosity is chosen to be
\begin{equation}
\eta(x,y)=-\sin(p)+1+\epsilon = -\sin (x^2y^2 + xy + 5) + 1 + \epsilon 
\end{equation}
where $\epsilon$ actually controls the viscosity contrast. Note that inserting the polynomial 
expression of the pressure inside the viscosity expression makes the problem linear. 
We have
\begin{eqnarray}
\dot{\varepsilon}_{xx} = \frac{\partial u}{\partial x} &=& 3x^2y+2x+y+1 \nn\\
\dot{\varepsilon}_{yy} = \frac{\partial v}{\partial y} &=& -3x^2y-2x-y-1 \nn\\
\dot{\varepsilon}_{xy} = \frac{1}{2}\left(\frac{\partial u}{\partial y} + \frac{\partial v}{\partial x} \right)
&=& \frac{1}{2}\left(x^3+x-3xy^2-2y \right)
\end{eqnarray}
and we can verify that the velocity field is incompressible since ${\vec \nabla}\cdot{\vec \upnu} = 
\dot{\varepsilon}_{xx} + \dot{\varepsilon}_{yy} =0$.
The pressure gradient is given by
\begin{eqnarray}
\frac{\partial p}{\partial x} &=& 2xy^2+y \nn\\
\frac{\partial p}{\partial y} &=& 2x^2y+x \nn
\end{eqnarray}
The right hand side term of the Stokes equation is such that
\begin{eqnarray}
 - \frac{\partial p}{\partial x} + \frac{\partial s_{xx}}{\partial x} + \frac{\partial s_{yx}}{\partial y} +f_x&=&0\nn\\
 - \frac{\partial p}{\partial y} + \frac{\partial s_{xy}}{\partial x} + \frac{\partial s_{yy}}{\partial y} +f_y&=&0
\end{eqnarray}
with 
\begin{eqnarray}
\frac{\partial s_{xx}}{\partial x} 
&=& \frac{\partial (2 \eta \dot{\varepsilon}_{xx}) }{\partial x} = 2 \eta \frac{\partial  \dot{\varepsilon}_{xx} }{\partial x} +  2\frac{\partial \eta }{\partial x} \dot{\varepsilon}_{xx} \nn\\
\frac{\partial s_{zx}}{\partial z} 
&=& \frac{\partial (2 \eta \dot{\varepsilon}_{zx}) }{\partial z} = 2 \eta \frac{\partial  \dot{\varepsilon}_{zx} }{\partial z} +  2\frac{\partial \eta }{\partial z} \dot{\varepsilon}_{zx} \nn\\
\frac{\partial s_{xz}}{\partial x} 
&=& \frac{\partial (2 \eta \dot{\varepsilon}_{xz}) }{\partial x} = 2 \eta \frac{\partial  \dot{\varepsilon}_{xz} }{\partial x} +  2\frac{\partial \eta }{\partial x} \dot{\varepsilon}_{xz} \nn\\
\frac{\partial s_{zz}}{\partial z} 
&=& \frac{\partial (2 \eta \dot{\varepsilon}_{zz}) }{\partial z} = 2 \eta \frac{\partial  \dot{\varepsilon}_{zz} }{\partial z} +  2\frac{\partial \eta }{\partial z} \dot{\varepsilon}_{zz} \nn\\
\frac{\partial \eta }{\partial x} &=& -z (2 x z + 1) \cos(x^2 z^2 + x z + 5) \nn\\
\frac{\partial \eta }{\partial z} &=& -x (2 x z + 1) \cos(x^2 z^2 + x z + 5) \nn\\
\frac{\partial  \dot{\varepsilon}_{xx} }{\partial x} &=& 6xz+2 \nn\\
\frac{\partial  \dot{\varepsilon}_{zx} }{\partial z} &=& -3xz-1  \nn\\
\frac{\partial  \dot{\varepsilon}_{xz} }{\partial x} &=& \frac{1}{2}(3x^2+1-3z^2)  \nn\\
\frac{\partial  \dot{\varepsilon}_{zz} }{\partial z} &=& -3x^2-1  \nn
\end{eqnarray}

\index{general}{pressure nullspace}
Velocity boundary conditions are prescribed on all four boundaries so that the pressure is known up to a constant
(the pressure solution has a nullspace), 
and the $p_0$ constant can be determined by requiring that
\[
\int_0^L\int_0^L p(x,y) \; dx dy = 
\int_0^L\int_0^L (x^2y^2+xy+5) dx dy + \int_0^L \int_0^L p_0 \; dxdy = 
\int_0^L\int_0^L (x^2y^2+xy+5) dx dy + p_0 L^2 =0 
\]
where $L$ is the size of the square domain.
Then
\[
p_0 =-  \frac{1}{L^2}  \int_0^L\int_0^L (x^2y^2+xy+5) dx dy
= -\frac{L^4}{9}-\frac{L^2}{4} - 5 
\]
\[
\]
%\begin{itemize}
%\item
%When the domain is $1\times 1$, $p_0=-\frac{1}{9}-\frac{1}{4} - 5 = -193/36$.
%\item
%When the domain is $2\times 2$, $p_0=-\frac{16}{9}-\frac{4}{4} - 5*4 = -70/9$.
%\item
%When the domain is $3\times 3$, $p_0=-\frac{81}{9}-\frac{9}{4} - 5*9 = -585/16$.
%\item
%When the domain is $4\times 4$, $p_0=-\frac{256}{9}-\frac{16}{4} - 5*16 = -1348/9$.
%\end{itemize}

As seen in the following figure, the value of $\epsilon$ controls the viscosity field amplitude.
This is simply explained by the fact that when the $\sin$ term of the viscosity takes value 1, the viscosity
is then equal to $\epsilon$.
\begin{center}
\includegraphics[width=14cm]{images/mms/mms7_mueffs}\\
Domain size 2x2 with $\epsilon=0.1, 0.01, 0.001$
\end{center}

Another interesting aspect of this benchmark is the fact that increasing the domain size
adds complexity to it as it increases the number of low viscosity zones and the spacing 
between them also decreases:

\begin{center}
\includegraphics[width=7.28cm]{images/mms/mms7_visc}
\includegraphics[width=7.28cm]{images/mms/mms7_vel}\\
\includegraphics[width=7.28cm]{images/mms/mms7_press}
\includegraphics[width=7.28cm]{images/mms/mms7_rhs}\\
Three different domain sizes (1x1, 2x2, 3x3) with $\epsilon=0.001$.
\end{center}


Finally, because the analytical expression for both components of the velocity is a polynomial, we can also
compute the root mean square velocity exactly. For instance, for a 2x2 domain:
\begin{center}
\includegraphics[width=8cm]{images/mms/mms7_vrmstheo}
\end{center}
and we end up with (for $L=2$)
\[
v_{rms} = \sqrt{\frac{1}{L^2}\frac{861752}{1575}} = \sqrt{\frac{215438}{1575}}
\simeq 11.6955560683
\]

\bscthesis \index{general}{BSc Thesis}

%-----------------------------------------------------------------------------
\subsubsection{Analytical benchmark VIII \label{mms8} - "Kovasznay"}

This flow was published by L.I.G. Kovasznay in 1948 \cite{kova48}. 
This paper presents an exact two-dimensional solution of the Navier-Stokes equations 
with a periodicity in the vertical direction, 
gives an analytical solution to the steady-state Navier-Stokes equations that is similar
which is a flow-field behind a periodic array of cylinders.

\[
u(x,y)=1-\exp(\lambda x) \cos (2\pi y)
\qquad
\qquad
v(x,y)=\frac{\lambda}{2\pi} \exp(\lambda x) \sin (2 \pi y)
\qquad
\qquad
\lambda=\frac{Re}{2}-\sqrt{\frac{Re^2}{4}+4\pi^2}
\]

Following step-55 of deal.II \footnote{\url{https://www.dealii.org/current/doxygen/deal.II/step_55.html}}
we have to 'cheat' here since we are not solving the non-linear Navier-Stokes equations, but the linear Stokes system without convective term. Therefore, to recreate the exact same solution
we move the convective term into the right-hand side.

The analytical solution is prescribed left and right, while free/no (??) slip is prescribed at top and bottom.

Solution as implemented in step-55:
\begin{verbatim}
const double pi2 = pi*pi;
  u = -exp(x*(-sqrt(25.0 + 4*pi2) + 5.0))*cos(2*y*pi) + 1;
  v = (1.0L/2.0L)*(-sqrt(25.0 + 4*pi2) + 5.0)*exp(x*(-sqrt(25.0 + 4*pi2) + 5.0))*sin(2*y*pi)/pi;
  p = -1.0L/2.0L*exp(x*(-2*sqrt(25.0 + 4*pi2) + 10.0)) 
- 2.0*(-6538034.74494422 + 0.0134758939981709*exp(4*sqrt(25.0 + 4*pi2)))/(-80.0*exp(3*sqrt(25.0 + 4*pi2)) 
+ 16.0*sqrt(25.0 + 4*pi2)*exp(3*sqrt(25.0 + 4*pi2))) 
- 1634508.68623606*exp(-3.0*sqrt(25.0 + 4*pi2))/(-10.0 + 2.0*sqrt(25.0 + 4*pi2)) 
+ (-0.00673794699908547*exp(sqrt(25.0 + 4*pi2)) 
+ 3269017.37247211*exp(-3*sqrt(25.0 + 4*pi2)))/(-8*sqrt(25.0 + 4*pi2) + 40.0) 
+ 0.00336897349954273*exp(1.0*sqrt(25.0 + 4*pi2))/(-10.0 + 2.0*sqrt(25.0 + 4*pi2));
\end{verbatim}


%-----------------------------------------------------------------------------
\subsubsection{Analytical benchmark IX \label{mms9} - "VJ2"}

It is presented in \cite{jolm17} and meant to be a peculiar case where the velocity solution 
is exactly zero. The viscosity is 1, the domain is a unit square, no-slip boundary conditions 
are prescribed everywhere. The buoyancy force is given by $\vec{b}=(0,Ra(1-y+3y^2))$ where 
$Ra>0$ is a parameter. The flow is incompressible and the analytical pressure solution 
is given by $p=Ra(y^3-y^2/2+y-7/12)$.

%-----------------------------------------------------------------------------
\subsubsection{Analytical benchmark X \label{mms10} - "VJ3"}

This benchmark comes from John et al. \cite{jolm17}.
The domain is once again the unit square. The velocity field has the form of a large vortex.

\begin{eqnarray}
u(x,y) &=& 200x^2(1-x)^2y(1-y)(1-2y) \\
v(x,y) &=& -200x(1-x)(1-2x)y^2(1-y)^2 \\
p(x,y) &=& 10\left[(x-1/2)^3y^2+(1-x)^3(y-1/2)^3 \right]
\end{eqnarray}

\begin{center}
\includegraphics[width=4.5cm]{images/benchmark_VJ3/u.pdf}
\includegraphics[width=4.5cm]{images/benchmark_VJ3/v.pdf}
\includegraphics[width=4.5cm]{images/benchmark_VJ3/p.pdf}
\end{center}

\begin{eqnarray}
\dot{\varepsilon}_{xx}=\frac{\partial u}{\partial x} &=& -400(1-x)x(2x-1)(y-1)y(2y-1)  \\
\frac{\partial u}{\partial y} &=& 200(1-x)^2x^2 (6y^2-6y+1)  \\
\frac{\partial v}{\partial x} &=& -200(6x^2-6x+1)(1-y)^2y^2  \\
\dot{\varepsilon}_{yy}=\frac{\partial v}{\partial y} &=& 400(x-1)x(2x-1)(1-y)y(2y-1) 
\end{eqnarray}
so that 
\begin{eqnarray}
\dot{\varepsilon}_{xy}
&=&\frac{1}{2} \left[ 200(1-x)^2x^2 (6y^2-6y+1)   -200(6x^2-6x+1)(1-y)^2y^2  \right] \nn\\
&=&100(1-x)^2x^2 (6y^2-6y+1)   -100(6x^2-6x+1)(1-y)^2y^2 
\end{eqnarray}
Also
\begin{eqnarray}
\frac{\partial \dot{\varepsilon}_{xx}}{\partial x} &=& 400(6x^2-6x+1)y(2y^2-3y+1) \nn\\
\frac{\partial \dot{\varepsilon}_{xy}}{\partial x} 
&=& 200 (-2 x^2 (1 - x) (6 y^2 - 6 y + 1) + 2 x (1 - x)^2 (6 y^2 - 6 y + 1) - 6 (2 x - 1) (1 - y)^2 y^2)\nn\\
&=&  100 (-2 x^2 (1 - x) (6 y^2 - 6 y + 1) + 2 x (1 - x)^2 (6 y^2 - 6 y + 1) - 6 (2 x - 1) (1 - y)^2 y^2) \nn\\
\frac{\partial \dot{\varepsilon}_{xy}}{\partial y} &=& 400 (6 x^2 - 6 x + 1) (1 - y) y^2 + 200 (1 - x)^2 x^2 (12 y - 6) - 400 (6 x^2 - 6 x + 1) (1 - y)^2 y   \nn \\
\frac{\partial \dot{\varepsilon}_{yy}}{\partial y} &=& -400x(2x^2-3x+1)(6y^2-6y+1) 
\end{eqnarray}


\begin{eqnarray}
\frac{\partial p}{\partial x} &=& 30(x-1/2)^2y^2-30(1-x)^2(y-1/2)^3 \\
\frac{\partial p}{\partial y} &=& 20(x-1/2)^3y + 30(1-x)^3(y-1/2)^2  
\end{eqnarray}

From $\vec\nabla\cdot{\bm \sigma}+\vec{b}=\vec{0}$ we can obtain the rhs as follows:
\begin{eqnarray}
\vec{b} 
&=& - \vec\nabla\cdot{\bm \sigma} \nn\\ 
&=& \vec\nabla p -  \vec\nabla\cdot{\bm s} \nn\\ 
&=& \vec\nabla p -  \vec\nabla\cdot(2 \eta \dot{\bm \varepsilon})  
\end{eqnarray}
Assuming $\eta=1$ we arrive at:
\begin{eqnarray}
b_x &=&  \frac{\partial p}{\partial x} 
-2\frac{\partial \dot{\varepsilon}_{xx}}{\partial x}  
-2\frac{\partial \dot{\varepsilon}_{xy}}{\partial y}  \\
b_y &=&  \frac{\partial p}{\partial y}  
-2\frac{\partial \dot{\varepsilon}_{xy}}{\partial x} 
-2\frac{\partial \dot{\varepsilon}_{yy}}{\partial y}  
\end{eqnarray}

All the necessary functions to do this benchmark are in {\tt mms/vj3.py}:
\lstinputlisting[language=python]{mms/vj3.py}

\begin{center}
\includegraphics[width=4.5cm]{images/mms/vj3/u}
\includegraphics[width=4.5cm]{images/mms/vj3/v}
\includegraphics[width=4.5cm]{images/mms/vj3/vel}\\
\includegraphics[width=4.5cm]{images/mms/vj3/p}
\includegraphics[width=4.5cm]{images/mms/vj3/exx}
\includegraphics[width=4.5cm]{images/mms/vj3/exy}
\end{center}

%-----------------------------------------------------------------------------
\subsubsection{Analytical benchmark XI \label{mms11} - "PPC1"}



%-----------------------------------------------------------------------------
\subsubsection{Analytical benchmark XII \label{mms11} - "PPC2"}



%-----------------------------------------------------------------------------
\subsubsection{Annulus with kinematical b.c.}

The domain is a hollow cylinder or inner radius $R_{i}=$ and outside radius $R_{o}=1$.
Boundary conditions are prescribed both on the inside and the outside with 
${\vec \upnu}=(u,v)=(-y,x)$, or in 
polar coordinates ${\vec \upnu}=r {\vec e}_\theta$.

The gravity is radial and is set to
\[
g_x=-x/r  \quad\quad g_z=-y/r
\]
where $r=\sqrt{x^2+z^2}$, which in polar coordinates is ${\vec g}=-{\vec e}_r$.
The viscosity is also set to 1, and the density is given by
\[
\rho(r)=r^n
\]
where $n$ is a positive or nul integer. The pressure is set to zero at the outer boundary.

The gradient operator in polar coordinates writes:
\[
{\vec \nabla} = \frac{\partial }{\partial r} {\vec e}_r 
+ \frac{1}{r} \frac{\partial }{\partial \theta} {\vec e}_\theta 
\]
and the Laplacian operator:
\[
\Delta = \frac{\partial^2}{\partial r^2} + \frac{1}{r} \frac{\partial }{\partial r} + \frac{1}{r^2} \frac{\partial^2 }{\partial \theta^2}
\]
Note that in our case we need to take the Laplacian of a vector, and unfortunately the Laplacian of a vector is not the Laplacian 
of the vector's coordinates in polar coordinates (unlike cartesian coordinates). 
The Laplacian of a vector is given by\footnote{https://en.wikipedia.org/wiki/Vector\_Laplacian} 
\[
\nabla^2 \vec{A} = \nabla(\nabla\cdot\vec{A}) - \nabla\times(\nabla \times\vec{A})
=
\left(
\begin{array}{l}
\frac{\partial^2 A_r}{\partial r^2} + \frac{1}{r} \frac{\partial A_r}{\partial r} - \frac{1}{r^2} A_r  + \frac{1}{r^2} \frac{\partial^2 A_r}{\partial \theta^2}  - \frac{2}{r^2} \frac{\partial A_\theta}{\partial \theta} \\ \\
\frac{\partial^2 A_\theta}{\partial r^2} + \frac{1}{r} \frac{\partial A_\theta}{\partial r} - \frac{1}{r^2} A_\theta  + \frac{1}{r^2} \frac{\partial^2 A_\theta}{\partial \theta^2}  + \frac{2}{r^2} \frac{\partial A_r}{\partial \theta} 
\end{array}
\right)
=
\left(
\begin{array}{l}
\Delta A_r \\ \\ \Delta A_\theta
\end{array}
\right)
\]
The Stokes equation writes:
\[
\mu \Delta {\vec \upnu} + \rho {\vec g} = {\vec 0 }
\]


The velocity solution is expected to be ${\vec \upnu}= r {\vec e}_\theta $.
The Stokes equation in polar coordinates then writes:
\[
-\frac{\partial p}{\partial r} + \Delta v_r + \rho(r) (- 1)  = 0 
\]
\[
-\frac{1}{r}\frac{\partial p}{\partial \theta} + \Delta v_\theta = 0
\]
Since $\Delta v_\theta = 0$, then $\frac{\partial p}{\partial \theta}=0$ and then the pressure is independent of $\theta$, 
which is what we expect since the density distribution is radial. 
We then focus on the first equation, and since $v_r=0$, we then obtain:
\[
\frac{\partial p}{\partial r}  = - \rho(r)
\]

\begin{itemize}
\item If $\rho(r)=1$, then 
\[
\frac{\partial p}{\partial r}  = - 1
\]
yields $p(r)=-r+C$ where C is a constant determined by means of b.c. ($p(r=1)=0$) so finally
\[
\boxed{
p(r)=1-r
}
\]

\item If $\rho(r)=r$, then 
\[
\frac{\partial p}{\partial r}  = - r
\]
so that $p(r)=-\frac{1}{2}r^2 + C$ and likewise
\[
\boxed{
p(r)=\frac{1}{2} (1- r^2)
}
\]
\end{itemize}
  
In general, by taking $\rho(r)=r^n$ with $n=0,1,...$ one arrives to a pressure field given by 
\[
\boxed{
p(r)=\frac{1}{n+1} (1- r^{n+1})
}
\]

\begin{center}
\includegraphics[width=6cm]{images/benchmark_annulus_mms/vel}
\includegraphics[width=6cm]{images/benchmark_annulus_mms/press}
\end{center}

This benchmark is of course very simple and the fact that the solution is independent of $\theta$
renders it not so useful. It has succesfully been implemented in ELEFANT.

%-----------------------------------------------------------------------------
\subsubsection{Viscous beam under extension}

The domain is a Cartesian box of size $L_x \times L_y$. 
Velocity $-u_0$ is applied on the left boundary and 
velocity $+u_0$ is applied on the right boundary. 
Bottom and top boundaries are left free. 
If no vertical velocity is prescribed anywhere there is an obvious nullspace 
in the solution which is problematic (numerically of course, but also 
because the solution is then not unique). One might want to set $v=0$ at $y=L_y/2$
on each side for example. 
The solution to this problem (incompressible Stokes equations) is given by
\begin{eqnarray}
u(x,y)&=&2u_0(x/L_x-1/2)\\
v(x,y)&=&-2 u_0 L_y/L_x (y/L_y-1/2)
\end{eqnarray}
in the absence of gravity. The strain rate tensor is then:
\[
\dot{\bm \varepsilon} =
\left(
\begin{array}{cc}
\dot{\varepsilon}_{xx} & \dot{\varepsilon}_{xy} \\
\dot{\varepsilon}_{xx} & \dot{\varepsilon}_{yy} 
\end{array}
\right)
=
\left(
\begin{array}{cc}
2 u_0 /Lx & 0 \\
0 & -2 u_0 /L_x 
\end{array}
\right)
\]
and we see that the flow is indeed incompressible as the trace 
of the strain rate tensor is zero. 

The momentum equation is 
\[
-\vec\nabla p + \vec\nabla \cdot (2 \eta \dot{\bm\varepsilon}) = \rho \vec g
\]
where the viscosity $\eta$ is constant in space. 
If gravity is set to zero, we obtain:
\begin{eqnarray}
-\frac{\partial p}{\partial x} &=& 0 \\
-\frac{\partial p}{\partial y} &=& 0 
\end{eqnarray}
since the strain rate is constant in space and the divergence operator applied to it returns 
the zero tensor. We there fore can conclude that pressure should be constant. 

Since the top and bottom boundaries are free, we have ${\bm \sigma}\cdot \vec{n} = \vec{0}$ on these.
The stress tensor is given by ${\bm \sigma} = - {\bm 1} + 2 \eta \dot{\bm \varepsilon}$ and the normal on the 
top is $\vec{n}=(0,+1)$ so that on the top boundary we have
\[
- p + 2 \eta \dot{\epsilon}_{yy} = 0
\]
or, 
\[
p= 2 \eta \dot{\epsilon}_{yy} 
\]
Note that using the bottom boundary with $\vec{n}=(0,-1)$ yields the same result.


\newpage
%-----------------------------------------------------------------------------
\subsubsection{Channel flow with Herschel-Bulkley rheology}

We start from the following formulation for the Herschel-Bulkley rheology:
\[
\eta_{hb}
=
\left\{
\begin{array}{lc}
\eta_0 & \dot{\varepsilon}_e\leq \dot{\varepsilon}_0 \\
K  \dot{\varepsilon}_e^{n-1} + \frac{\tau_0}{\dot{\varepsilon}_e}  
& \dot{\varepsilon}_e\geq \dot{\varepsilon}_0 
\end{array}
\right.
\]
and the limiting viscosity $\eta_0$ is such that 
\[
\eta_0 = K  \dot{\varepsilon}_0^{n-1} + \frac{\tau_0}{\dot{\varepsilon}_0}  
\]

We consider a two-dimensional channel in the $x,y$ plane. The walls 
are at $y=0$ and $y=H$ with no-slip boundary conditions. 
In the absence of gravity, the Stokes equation simplify to 
\begin{equation}
-\frac{\partial p}{\partial x}  +\frac{\partial }{\partial y} (2\eta_{hb} \dot{\varepsilon}_{xy}) =0
\qquad
\text{and}
\qquad
\dot{\varepsilon}_{xy} = \frac{1}{2} \frac{\partial u}{\partial y} 
\label{eq:hb1}
\end{equation}
where we assume the velocity $\vec\upnu=(u(y),0)$.
It then follows that 
\[
\dot{\varepsilon}_{e} 
= \sqrt{{\cal I}_2(\dot{\bm \varepsilon})} 
=\sqrt{ \frac{1}{2} \dot{\bm \varepsilon} : \dot{\bm \varepsilon} }
=\sqrt{ 
\frac{1}{2}[(\dot{\varepsilon}_{xx})^2 + (\dot{\varepsilon}_{yy})^2 + (\dot{\varepsilon}_{zz})^2]  
+ (\dot{\varepsilon}_{xy})^2  
+ (\dot{\varepsilon}_{xz})^2  
+ (\dot{\varepsilon}_{yz})^2 
}
=\sqrt{\dot{\varepsilon}_{xy}^2 }
= \left|\frac{1}{2} \frac{\partial u}{\partial y}  \right|
\]
In the case of a Newtonian fluid, the analytical solution is 
known and the velocity profile is a parabola with zero velocity on the
walls and maximum velocity in the middle. 
Although the rheology of the fluid is non-linear we assume that a 
similar velocity profile is expected (although not described by a parabola).
We then expect three zones (and we assume that the fluid flows from left to right):
\begin{itemize}

%..............................
\item In the middle, where it is expected that $\frac{\partial u}{\partial y}=0$ (at least in one point)
because of symmetry. We also therefore expect $\dot{\varepsilon}_e\leq \dot{\varepsilon}_0$ in this region
so that $\eta_{hb}=\eta_0$. How thick this region is will be determined later. 

Eq.~\eqref{eq:hb1} must then be solved 
\begin{eqnarray}
\frac{\partial p}{\partial x}  
&=&\frac{\partial }{\partial y} \left(2\eta_{hb}  \frac{1}{2}\frac{\partial u}{\partial y} \right) 
= \eta_0 \frac{\partial^2 u}{\partial y^2}  
\end{eqnarray}

Let us call $\Pi=\frac{\partial p}{\partial x} <0$, then we must solve:
\[
\frac{\partial^2 u}{\partial y^2} = \frac{\Pi}{\eta_0} 
\]
The solution is then of the form
\[
\boxed{
u(y)|_{mid} = \frac{1}{2}\frac{\Pi}{\eta_0} y^2 + 2a y + b
}
\]
and 
\[
\boxed{
\dot{\varepsilon}_{xy}|_{mid}= \frac{1}{2} \frac{\Pi}{\eta_0}y  + a
}
\]
We will determine $a$ and $b$ later. 

%..............................
\item Near the bottom wall, with $\frac{\partial u}{\partial y}>0$ so that  
$\dot{\varepsilon}_{e} = +\frac{1}{2}\left( \frac{\partial u}{\partial y}  \right)$  
and $\dot{\varepsilon}_e\geq \dot{\varepsilon}_0 $. 
We solve Eq.~\eqref{eq:hb1} again, this time with the non-linear formulation of the viscosity: 
\begin{eqnarray}
\frac{\partial p}{\partial x}  
&=&\frac{\partial }{\partial y} \left( 2 \eta_{hb}  \frac{1}{2}\frac{\partial u}{\partial y} \right)  \nn\\
&=&2 \frac{\partial }{\partial y} \left[  
\left( K  \dot{\varepsilon}_e^{n-1} + \frac{\tau_0}{\dot{\varepsilon}_e}  
\right) \frac{1}{2}\frac{\partial u}{\partial y} \right] \nn\\
&=&\frac{\partial }{\partial y} \left[  
\left( K \left|\frac{1}{2}\frac{\partial u}{\partial y}\right|^{n-1} 
+ \tau_0 \left|\frac{1}{2}\frac{\partial u}{\partial y}\right|^{-1} 
\right) \frac{1}{2}\frac{\partial u}{\partial y} \right]  \nn\\
&=&  \frac{\partial }{\partial y} \left[  
K \left(\frac{1}{2}\frac{\partial u}{\partial y}\right)^{n} + \tau_0 \right] 
\end{eqnarray}
We then must solve:
\[
\frac{\partial }{\partial y} \left[  
K \left(\frac{1}{2}\frac{\partial u}{\partial y}\right)^{n} + \tau_0 \right] 
= \Pi
\]
\[
K \left(\frac{1}{2}\frac{\partial u}{\partial y}\right)^{n} + \tau_0  = \Pi y +c
\]
\[
\left(\frac{1}{2}\frac{\partial u}{\partial y}\right)^{n}   = \frac{1}{K} ( \Pi y +c -\tau_0 )
\]
or, 
\[
\boxed{
\dot{\varepsilon}_{xy}|_{bot}=
\frac{1}{2}\frac{\partial u}{\partial y}  = \left( \frac{1}{K} ( \Pi y +c -\tau_0 ) \right)^{1/n}
}
\]
so 
\[
\boxed{
u(y)|_{bot} = 2 \frac{n}{n+1} \frac{K}{\Pi} \left( \frac{1}{K} ( \Pi y +c -\tau_0 ) \right)^{1+1/n} + d
}
\]


%..............................
\item Near the top wall, with $\frac{\partial u}{\partial y}<0$ so that 
$\dot{\varepsilon}_{e} = -\frac{1}{2}\left( \frac{\partial u}{\partial y}  \right)$ 
and $\dot{\varepsilon}_e\geq \dot{\varepsilon}_0$. We solve yet again Eq.~\eqref{eq:hb1}:
\begin{eqnarray}
\frac{\partial p}{\partial x}  
&=&\frac{\partial }{\partial y} \left( 2 \eta_{hb}  \frac{1}{2}\frac{\partial u}{\partial y} \right)  \nn\\
&=&2 \frac{\partial }{\partial y} \left[  
\left( K  \dot{\varepsilon}_e^{n-1} + \frac{\tau_0}{\dot{\varepsilon}_e}  
\right) \frac{1}{2}\frac{\partial u}{\partial y} \right] \nn\\
&=&\frac{\partial }{\partial y} \left[  
\left( K \left|\frac{1}{2}\frac{\partial u}{\partial y}\right|^{n-1} 
+ \tau_0 \left|\frac{1}{2}\frac{\partial u}{\partial y}\right|^{-1} 
\right) \frac{1}{2}\frac{\partial u}{\partial y} \right]  \nn\\
&=&-\frac{\partial }{\partial y} \left[  
\left( K \left( - \frac{1}{2}\frac{\partial u}{\partial y}\right)^{n-1} 
+ \tau_0 \left( -\frac{1}{2}\frac{\partial u}{\partial y}\right)^{-1} 
\right) \left(-\frac{1}{2}\frac{\partial u}{\partial y}\right) \right]  \nn\\
&=& -\frac{\partial }{\partial y} \left[   
K \left(-\frac{1}{2}\frac{\partial u}{\partial y}\right)^{n} + \tau_0 \right] 
\end{eqnarray}
We then must solve:
\[
-\frac{\partial }{\partial y} \left[ 
K \left(-\frac{1}{2}\frac{\partial u}{\partial y}\right)^{n} + \tau_0 \right] 
= \Pi
\]
\[
K \left(-\frac{1}{2}\frac{\partial u}{\partial y}\right)^{n} + \tau_0 = - \Pi y + e
\]
\[
 \left(-\frac{1}{2}\frac{\partial u}{\partial y}\right)^{n}  = \frac{1}{K} (-\Pi y + e - \tau_0)
\]
which yields
\[
\boxed{
\dot{\varepsilon}_{xy}|_{top}
=  - \left( \frac{1}{K} (- \Pi y +e - \tau_0 ) \right)^{1/n}
}
\]
\[
\boxed{
u(y)|_{top} = 2 \frac{n}{n+1} \frac{K}{\Pi} \left( \frac{1}{K} ( -\Pi y +e -\tau_0 ) \right)^{1+1/n} + f
}
\]


\end{itemize}

\newpage

We have 6 integration constants $a,b,c,d,e,f$ and 6 additional constraints from 
continuity or boundary conditions:
\begin{eqnarray}
(1)&&u(0) = 0 \text{ boundary condition}\\
(2)&&u(H) = 0 \text{ boundary condition}\\
(3)&&u(y_1) \text{ must be continuous}\\
(4)&&u(y_2) \text{ must be continuous}\\
(5)&&\dot{\varepsilon}_{xy}(y_1) \text{ must be continuous}\\
(6)&&\dot{\varepsilon}_{xy}(y_2) \text{ must be continuous}
\end{eqnarray}

%......................................................
\paragraph{Using symmetry to compute $a$}
Because of symmetry, we expect $y_1=H/2-\delta$ and $y_2=H/2+\delta$ with $\delta \neq 0$ 
(i.e. $y_1\neq y_2$) and we expect $u(y_1)=u(y_2)$ so that 
\[
u(y_1)|_{mid}=
\frac{1}{2}\frac{\Pi}{\eta_0} y_1^2 + 2a y_1 + b
=
\frac{1}{2}\frac{\Pi}{\eta_0} y_2^2 + 2a y_2 + b=
u(y_2)|_{mid}
\]
or, 
\[
\frac{1}{2}\frac{\Pi}{\eta_0} (y_1^2-y_2^2) + 2a (y_1-y_2) =0
\]
\[
\frac{1}{2}\frac{\Pi}{\eta_0} (y_1-y_2)(y_1+y_2) + 2a (y_1-y_2) =0
\]
\[
\frac{1}{2}\frac{\Pi}{\eta_0} (y_1+y_2) + 2a =0
\]
\[
\frac{1}{2}\frac{\Pi}{\eta_0} H + 2a =0
\]
and finally we obtain $a$:
\[
\boxed{
a = -\frac{1}{4} \frac{\Pi}{\eta_0} H
}
\]
Note that we could have obtained the same thing by enforcing that the strain rate 
at $y_1$ and $y_2$ are the opposite of one another. It then follows: 
\[
\boxed{
u(y)|_{mid} 
= \frac{1}{2}\frac{\Pi}{\eta_0} y^2  -2\frac{1}{4} \frac{\Pi}{\eta_0} H y + b
= \frac{1}{2}\frac{\Pi}{\eta_0} (y^2  -  y H) + b
}
\]
and 
\[
\boxed{
\dot{\varepsilon}_{xy}|_{mid}
= \frac{1}{2} \frac{\Pi}{\eta_0}y  -\frac{1}{4} \frac{\Pi}{\eta_0} H
= \frac{1}{2} \frac{\Pi}{\eta_0} (y  -\frac{H}{2} )
}
\]
Because of the parabola-like flow profile, we expect the strain rate 
to be zero in the middle $y=H/2$, and positive for $z_1<y<H/2$ and 
negative for $H/2<y<z_2$, which is indeed what we recover ($\Pi<0$).

%......................................................
\paragraph{Using bottom boundary condition to obtain $d$}
\[
u(y=0)|_{bot} = 2 \frac{n}{n+1} \frac{K}{\Pi} \left( \frac{1}{K} ( c -\tau_0 ) \right)^{1+1/n} + d = 0
\]
so 
\[
d = -2 \frac{n}{n+1} \frac{K}{\Pi} \left( \frac{1}{K} ( c -\tau_0 ) \right)^{1+1/n} 
\]
and then
\[
\boxed{
u(y)|_{bot} = 
2 \frac{n}{n+1} \frac{K}{\Pi} \left[ \left( \frac{1}{K} ( \Pi y +c -\tau_0 ) \right)^{1+1/n} 
- \left( \frac{1}{K} ( c -\tau_0 ) \right)^{1+1/n} \right]
}
\]


%......................................................
\paragraph{Using top boundary condition to obtain $f$}
\[
u(y=H)|_{top} = 2 \frac{n}{n+1} \frac{K}{\Pi} \left( \frac{1}{K} ( -\Pi H +e -\tau_0 ) \right)^{1+1/n} + f =0
\]
so 
\[
f= -2 \frac{n}{n+1} \frac{K}{\Pi} \left( \frac{1}{K} ( -\Pi H +e -\tau_0 ) \right)^{1+1/n} 
\]
and then
\[
\boxed{
u(y)|_{top} = 2 \frac{n}{n+1} \frac{K}{\Pi}
\left[
\left( \frac{1}{K} ( -\Pi y +e -\tau_0 ) \right)^{1+1/n} - 
\left( \frac{1}{K} ( -\Pi H +e -\tau_0 ) \right)^{1+1/n} \right]
}
\]


%......................................................
\paragraph{computing $\delta$}

The coordinates of the transitions $y_1$ and $y_2$ are the location where the strain rate 
reaches $\dot{\varepsilon}_0$. In other words:
\[
\dot{\varepsilon}_{xy}|_{mid}(y=y_1)
= \frac{1}{2} \frac{\Pi}{\eta_0} \left(y_1  -\frac{H}{2} \right)
= \frac{1}{2} \frac{\Pi}{\eta_0} \left(\frac{H}{2}-\delta  -\frac{H}{2} \right)
= -\frac{1}{2} \frac{\Pi}{\eta_0} \delta
= \dot{\varepsilon}_0
\]
or, 
\[
\delta = -\frac{2 \dot{\varepsilon}_0 \eta_0}{\Pi}
\]
Since $\Pi<0$ it adds up and $\delta>0$. We can also write
\[
\boxed{
\delta = \frac{2 \dot{\varepsilon}_0 \eta_0}{|\Pi|}
}
\]
and we will use throughout what follows:
\[
\dot{\varepsilon}_0 = - \frac{1}{2}\frac{\Pi}{\eta_0} \delta
\]


%......................................................
\paragraph{Using strain rate continuity at $y_1$ to compute $c$}
\[
\left( \frac{1}{K} ( \Pi y_1 +c -\tau_0 ) \right)^{1/n}
= \frac{1}{2} \frac{\Pi}{\eta_0} \left(y_1  -\frac{H}{2} \right)
=- \frac{1}{2} \frac{\Pi}{\eta_0} \delta
\]
\[
 \Pi y_1 +c -\tau_0 
= K\left( - \frac{1}{2} \frac{\Pi}{\eta_0} \delta\right)^{n}
\]
\[
c = K\left( - \frac{1}{2} \frac{\Pi}{\eta_0} \delta\right)^{n} + \tau_0 -\Pi y_1
\]

\[
\boxed{
c = K \dot{\varepsilon}_0^{n} + \tau_0 -\Pi y_1
}
\]




\begin{eqnarray}
u(y)|_{bot} 
&=& 
2 \frac{n}{n+1} \frac{K}{\Pi} \left[ \left( \frac{1}{K} ( \Pi y +c -\tau_0 ) \right)^{1/n+ 1 } 
- \left( \frac{1}{K} ( c -\tau_0 ) \right)^{1/n+ 1 } \right]
\nn\\
&=& 
2 \frac{n}{n+1} \frac{K}{\Pi} 
\left[ \left( \frac{1}{K} ( \Pi (y-y_1) +K \dot{\varepsilon}_0^{n}   ) \right)^{1/n+ 1 } 
- \left( \frac{1}{K} (  K  
\dot{\varepsilon}_0^{n}  -\Pi y_1    ) \right)^{1/n+ 1 } \right]
\nn\\
&=& 
2 \frac{n}{n+1} \frac{K}{\Pi} \left[ \left( \frac{\Pi}{K} (y-y_1) 
+ \dot{\varepsilon}_0^{n}    \right)^{1/n+ 1 } 
- \left(  \dot{\varepsilon}_0^{n}  - \frac{\Pi}{K}y_1     \right)^{1/n+ 1 } \right]
\nn\\
\dot{\varepsilon}_{xy}|_{bot}
&=& \left( \frac{1}{K} ( \Pi y +c -\tau_0 ) \right)^{1/n} \nn\\
&=& \left( \frac{\Pi}{K} (y-y_1) + \dot{\varepsilon}_0^{n}   \right)^{1/n} \nn
\end{eqnarray}



%......................................................
\paragraph{Using strain rate continuity at $y_2$ to compute $e$}
\[
-\left( \frac{1}{K} ( -\Pi y_2 +e -\tau_0 ) \right)^{1/n}
= \frac{1}{2} \frac{\Pi}{\eta_0} \left(y_2  -\frac{H}{2} \right)
= \frac{1}{2} \frac{\Pi}{\eta_0} \delta  
\]
\[
-\Pi y_2 +e -\tau_0 
= K\left(-\frac{1}{2} \frac{\Pi}{\eta_0} \delta  \right)^{n} 
\]
\[
e = K\left(-\frac{1}{2} \frac{\Pi}{\eta_0} \delta  \right)^{n} + \tau_0 + \Pi y_2 
\]
\[
\boxed{
e = K \dot{\varepsilon}_0^n + \tau_0 + \Pi y_2 
}
\]



\begin{eqnarray}
u(y)|_{top} &=& 2 \frac{n}{n+1} \frac{K}{\Pi}
\left[
\left( \frac{1}{K} ( -\Pi y +e -\tau_0 ) \right)^{1/n+ 1 } - 
\left( \frac{1}{K} ( -\Pi H +e -\tau_0 ) \right)^{1/n+ 1 } \right] \nn\\
&=& 2 \frac{n}{n+1} \frac{K}{\Pi}
\left[
\left( -\frac{\Pi}{K}(y+y_2)+ \dot{\varepsilon}_0^{n}  \right)^{1/n+ 1 } - 
\left( -\frac{\Pi}{K}(H+y_2)+ \dot{\varepsilon}_0^{n}  \right)^{1/n+ 1 } 
\right]
\nn\\
\dot{\varepsilon}_{xy}|_{top}
&=&  - \left( \frac{1}{K} (- \Pi y +e - \tau_0 ) \right)^{1/n} \nn\\
&=&  - \left( - \frac{\Pi}{K} (y+y_2) +  
\dot{\varepsilon}_0^{n} \right)^{1/n} \nn
\end{eqnarray}




%......................................................
\paragraph{Using velocity continuity to compute $b$} 
We use $u(y_1)|_{bot}=u(y_1)|_{mid}$: 

\[
2 \frac{n}{n+1} \frac{K}{\Pi} \left[ \left( \frac{\Pi}{K} (y_1-y_1) 
+ \dot{\varepsilon}_0^{n}    \right)^{1/n+ 1 } 
- \left( \dot{\varepsilon}_0^{n}  - \frac{\Pi}{K}y_1     \right)^{1/n+ 1 } \right]
=
\frac{1}{2}\frac{\Pi}{\eta_0} (y_1^2  -  y_1 H) + b
\]

\[
2 \frac{n}{n+1} \frac{K}{\Pi} \left[ 
\dot{\varepsilon}_0^{n+ 1} 
- \left( \dot{\varepsilon}_0^{n}  - \frac{\Pi}{K}y_1  \right)^{1/n+ 1 } \right]
=
\frac{1}{2}\frac{\Pi}{\eta_0} y_1 (y_1  - H) + b
\]
so 
\[
b= 
2 \frac{n}{n+1} \frac{K}{\Pi} \left[ 
\dot{\varepsilon}_0^{n+ 1} 
- \left( \dot{\varepsilon}_0^{n}  - \frac{\Pi}{K}y_1  \right)^{1/n+ 1 } \right]
- \frac{1}{2}\frac{\Pi}{\eta_0} y_1 (y_1  - H) 
\]

%............................................................
\paragraph{Using velocity continuity to compute $b$ - AGAIN} 
This time we use $u(y_2)|_{top}=u(y_2)|_{mid}$: 

\[
2 \frac{n}{n+1} \frac{K}{\Pi}
\left[
\left( -\frac{\Pi}{K}(y_2-y_2)+ \dot{\varepsilon}_0^{n}  \right)^{1/n+ 1 } - 
\left( -\frac{\Pi}{K}(H-y_2)+ \dot{\varepsilon}_0^{n}  \right)^{1/n+ 1 } 
\right]
= \frac{1}{2}\frac{\Pi}{\eta_0} (y_2^2  -  y_2 H) + b
\]

\[
2 \frac{n}{n+1} \frac{K}{\Pi}
\left[
\dot{\varepsilon}_0^{n+1}  - 
\left( -\frac{\Pi}{K}(H-y_2)+   \dot{\varepsilon}_0^{n}  \right)^{1/n+ 1 } 
\right]
=
\frac{1}{2}\frac{\Pi}{\eta_0} (y_2^2  -  y_2 H) + b
\]
and since $H-y_2 = H- H/2 - \delta = H/2 -\delta = y_1$ and
\[
y_2^2-y_2H = y_2(y_2 - H) = (H/2+\delta)(-y_1) 
= (-H/2-\delta)y_1
= (-H+H/2-\delta)y_1
= (-H+y_1)y_1
\] 
so that we indeed recover the same $b$ value as above. 

\newpage
To summarize:

\begin{mdframed}[backgroundcolor=blue!5]
\begin{eqnarray}
u(y)|_{bot} 
&=& 
2 \frac{n}{n+1} \frac{K}{\Pi} \left[ \left( \frac{\Pi}{K} (y-y_1) +  \dot{\varepsilon}_0^{n}    \right)^{1/n+ 1 } 
- \left(   \dot{\varepsilon}_0^{n}  - \frac{\Pi}{K}y_1     \right)^{1/n+ 1 } \right]
\nn\\
u(y)|_{mid} 
&=& \frac{1}{2}\frac{\Pi}{\eta_0} (y^2  -  y) + 
2 \frac{n}{n+1} \frac{K}{\Pi} \left[ 
+ \dot{\varepsilon}_0^{n+ 1} 
- \left( \dot{\varepsilon}_0^{n}  - \frac{\Pi}{K}y_1  \right)^{1/n+ 1 } \right]
- \frac{1}{2}\frac{\Pi}{\eta_0} y_1 (y_1  - H) 
\nn\\
u(y)|_{top} 
&=& 2 \frac{n}{n+1} \frac{K}{\Pi}
\left[
\left(-\frac{\Pi}{K}(y+y_2)+ \dot{\varepsilon}_0^{n}\right)^{\frac{1}{n}+ 1 } - 
\left(-\frac{\Pi}{K}(H+y_2)+ \dot{\varepsilon}_0^{n}\right)^{\frac{1}{n}+ 1 } 
\right]
\nn\\
\dot{\varepsilon}_{xy}|_{bot}
&=& \left( \frac{\Pi}{K} (y-y_1) + \dot{\varepsilon}_0^{n}   \right)^{1/n} \nn\\
\dot{\varepsilon}_{xy}|_{mid}
&=& \frac{1}{2} \frac{\Pi}{\eta_0} (y  -\frac{H}{2} ) \nn\\
\dot{\varepsilon}_{xy}|_{top}
&=&  - \left( - \frac{\Pi}{K} (y+y_2) +  \dot{\varepsilon}_0^{n} \right)^{1/n} \nn
\end{eqnarray}
\end{mdframed}

%............................................................
\paragraph{Let's start simple: $n=1$}

\begin{center}
\includegraphics[width=7cm]{images/mms/channel_hb/velocity}
\includegraphics[width=7cm]{images/mms/channel_hb/exy}
\end{center}



PROBLEM : it does not work for $n>1$ !!!

%\newpage
%To solve this equation it is necessary to non-dimensionalize the quantities involved. 
%The channel depth $H$ is chosen as a length scale, 
%the mean velocity $V$ is taken as a velocity scale, 
%and the pressure scale is taken to be 
%$P_0 =k (V/L_y)^n$. This analysis introduces the non-dimensional pressure gradient 
%\[
%\pi_0 = \frac{L_y}{P_0}\frac{\partial p}{\partial x}
%\]
%which is negative for flow from left to right, and the Bingham number: 
%\[
%B_n = \frac{\tau_0}{k}\left(\frac{L_y}{V}\right)^n
%\]







