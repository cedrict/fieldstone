\begin{flushright} {\tiny {\color{gray} quadrature\_triangles.tex}} \end{flushright}
%~~~~~~~~~~~~~~~~~~~~~~~~~~~~~~~~~~~~~~~~~~~~~~~~~~~~~~~~~~~~~~~~~~~~~~~~~~~~~~~~~~~~~~~~~~~~~~~~~~


Our goal is to develop a quadrature rule of the form
\[
\iint_\triangle f(r,s) \; drds 
\simeq \frac12 \sum_{i_q=1}^{n_q} \omega_{i_q} f(r_{i_q},s_{i_q})
\]
We will here add two requirements:
a) we would like to find quadrature rules which achieve the highest possible accuracy for the lowest possible number of quadrature points; b) we would like the quadrature points to possess some kind of symmetry.
Note that the factor $1/2$ in the equation above is a convention. If $f=1$ then 
the left hand term is the area of the triangle which is 1/2. Since people 
usually require that $\sum_i\omega_i=1$ then the factor 1/2 is necessary.

Before we go any further, we need to establish that
\[
\iint_\triangle \{ 1,r,s,r^2,rs,t^2,r^3,r^2s,rs^2,s^3\}\; dr ds =
\left\{
\frac12,\frac16,\frac16,\frac{1}{12},\frac{1}{24},\frac{1}{12},
\frac{1}{20},\frac{1}{60},\frac{1}{60},\frac{1}{20}
\right\}
\]
where $\triangle$ stands for the reference triangle.

%-----------------------------------------------
\subsubsection*{Gaussian quadrature of degree 1}

This means that the quadrature should be accurate for $f(r,s)=\{1,r,s\}$. 
We then obtain:
\begin{eqnarray}
\iint_\triangle 1 \; drds = \frac12 &=& \frac12 \sum_{i_q=1}^{1} \omega_{i_q} f(r_{i_q},s_{i_q}) = \frac12 \omega_1 f(r_1,s_1) = \frac12 \omega_1 \nn\\
\iint_\triangle r \; drds = \frac16 &=& \frac12 \sum_{i_q=1}^{1} \omega_{i_q} r_{i_q} 
= \frac12 \omega_1 r_1 \nn\\
\iint_\triangle s \; drds = \frac16 &=& \frac12 \sum_{i_q=1}^{1} \omega_{i_q} s_{i_q} 
= \frac12 \omega_1 s_1 \nn
\end{eqnarray}
We then obtain $\omega_1=1$ and $r_1=s_1=\frac13$.


%-----------------------------------------------
\subsubsection*{Gaussian quadrature of degree 2}

The quadrature should be accurate for $f(r,s)=\{1,r,s,r^2,rs,t^2\}$ so that
\begin{eqnarray}
\iint_\triangle 1 drds = \frac12 &=& \frac12 \sum_{i_q=1}^{n_q}\omega_{i_q} \nn\\
\iint_\triangle r drds = \frac16 &=& \frac12 \sum_{i_q=1}^{n_q}\omega_{i_q} r_{i_q}\nn\\
\iint_\triangle s drds = \frac16 &=& \frac12 \sum_{i_q=1}^{n_q}\omega_{i_q} s_{i_q}\nn\\
\iint_\triangle r^2 drds = \frac1{12} &=& \frac12 \sum_{i_q=1}^{n_q}\omega_{i_q} r_{i_q}^2\nn\\
\iint_\triangle rsdrds = \frac1{24} &=& \frac12 \sum_{i_q=1}^{n_q}\omega_{i_q} r_{i_q}s_{i_q}\nn\\
\iint_\triangle s^2 drds = \frac1{12} &=& \frac12 \sum_{i_q=1}^{n_q}\omega_{i_q} s_{i_q}^2\nn\\
\end{eqnarray}
If we set $n_q=1$ then we have 6 equations and only three unknowns $\omega_1,r_1,s_1$
so this will not work. 
If we set $n_q=2$ then we have 6 equations and six unknowns $\omega_1,r_1,s_1,\omega_2,r_2,s_2$ but we find that the quadrature is not symmetric\footnote{proof to do}.
If we set $n_q=3$ then we have 6 equations and 9 unknowns and the solution is not 
unique. Because of symmetry we for instance have to impose $r_2=s_3$ and $s_2=r_3$ 
(points 2 and 3 are symmetric with respect to the $r=s$ line).
Two common quadratures are found:
\[
(r_1,s_1)=(\frac16,\frac16) \qquad
(r_2,s_2)=(\frac23,\frac16) \qquad
(r_3,s_3)=(\frac16,\frac23) \qquad \omega_1=\omega_2=\omega_3=\frac13
\]
and 
\[
(r_1,s_1)=(0,\frac12) \qquad
(r_2,s_2)=(\frac12,0) \qquad
(r_3,s_3)=(\frac12,\frac12) \qquad \omega_1=\omega_2=\omega_3=\frac13
\]

%-----------------------------------------------
\subsubsection*{Gaussian quadrature of degree 3}

The quadrature should be accurate for $f(r,s)=\{1,r,s,r^2,rs,t^2,r^3,r^2s,rs^2,t^2\}$ 
so that
\begin{eqnarray}
\iint_\triangle 1 drds = \frac12 &=& \frac12 \sum_{i_q=1}^{n_q}\omega_{i_q} \nn\\
\iint_\triangle r drds = \frac16 &=& \frac12 \sum_{i_q=1}^{n_q}\omega_{i_q} r_{i_q}\nn\\
\iint_\triangle s drds = \frac16 &=& \frac12 \sum_{i_q=1}^{n_q}\omega_{i_q} s_{i_q}\nn\\
\iint_\triangle r^2 drds = \frac1{12} &=& \frac12 \sum_{i_q=1}^{n_q}\omega_{i_q} r_{i_q}^2\nn\\
\iint_\triangle rsdrds = \frac1{24} &=& \frac12 \sum_{i_q=1}^{n_q}\omega_{i_q} r_{i_q}s_{i_q}\nn\\
\iint_\triangle s^2 drds = \frac1{12} &=& \frac12 \sum_{i_q=1}^{n_q}\omega_{i_q} s_{i_q}^2\nn\\
\iint_\triangle r^3 drds = \frac1{20} &=& \frac12 \sum_{i_q=1}^{n_q}\omega_{i_q} r_{i_q}^3\nn\\
\iint_\triangle r^2s drds = \frac1{60} &=& \frac12 \sum_{i_q=1}^{n_q}\omega_{i_q} r_{i_q}^2 s_{i_q}\nn\\
\iint_\triangle rs^2 drds = \frac1{60} &=& \frac12 \sum_{i_q=1}^{n_q}\omega_{i_q} r_{i_q} s_{i_q}^2\nn\\
\iint_\triangle s^3 drds = \frac1{20} &=& \frac12 \sum_{i_q=1}^{n_q}\omega_{i_q} s_{i_q}^3\nn\\
\end{eqnarray}
This time $n_q=3$ will not work since we have 10 equations. Switching to $n_q=4$
we now have 12 unknowns and 10 equations so there are many possibilities. 
One of them is given in the table below. As a side note there could be cases where 
having a negative weight $\omega$ could be problematic. 


%-----------------------------------------------------------------
\subsubsection*{Tabulated quadrature rules on the reference triangle}

In what follows we use the following quadrature rule:
\[
\iint_\triangle f(r,s) \; drds 
\simeq \sum_{i_q=1}^{n_q} \omega_{i_q} f(r_{i_q},s_{i_q})
\]
with $\sum_i\omega_i=1/2$.

Quadrature rules for triangles can be found in \textcite{duna85} (1985).
The following ones are identical to those in the {\sl ip\_triangle.m} 
file of the MILAMIN code \cite{daks08}. See also \textcite{leth76} (1976)
on the topic of computation of double integrals over a triangle.

{\small
\[
\begin{array}{c|ccc|ccc}
\hline
&r_q & s_q & w_q &r_q & s_q & w_q \\ 
&\text{exact} & \text{exact} & \text{exact} & \text{approx.}& \text{approx.}& \text{approx.}\\
\hline\hline
i_q=1& 1/3 & 1/3 & 1/2\\
\hline
i_q=1 & 1/6 & 1/6 & 1/6 \\
i_q=2 & 2/3 & 1/6 & 1/6 \\
i_q=3 & 1/6 & 2/3 & 1/6 \\
\hline
i_q=1&1/3 & 1/3 & -27/96\\
i_q=2&0.6 & 0.2 &  25/96\\
i_q=3&0.2 & 0.6 &  25/96\\
i_q=4&0.2 & 0.2 &  25/96\\
\hline
i_q=1& 1-2g_1 & g_1 & w_1/2  &  0.108103018168070 & 0.44594849091596  &   \\
i_q=2& g_1 & 1-2g_1 & w_1/2  &  0.445948490915965 & 0.108103018168070 &   \\
i_q=3& g_1 & g_1    & w_1/2  &  0.445948490915965 & 0.445948490915965 &   \\
i_q=4& 1-2g_2 & g_2 & w_2/2  &  0.816847572980459 & 0.091576213509771 &   \\
i_q=5& g_2 & 1-2g_2 & w_2/2  &  0.091576213509771 & 0.816847572980459 &   \\
i_q=6& g_2 & g_2    & w_2/2  &  0.091576213509771 & 0.091576213509771 &   \\
\hline
i_q=1&&&&0.091576213509771 &  0.091576213509771    &    0.109951743655322/2.0 \\ 
i_q=2&&&&0.816847572980459 &  0.091576213509771    &    0.109951743655322/2.0 \\
i_q=3&&&&0.091576213509771 &  0.816847572980459    &    0.109951743655322/2.0 \\
i_q=4&&&&0.445948490915965 &  0.445948490915965    &    0.223381589678011/2.0 \\
i_q=5&&&&0.108103018168070 &  0.445948490915965    &    0.223381589678011/2.0 \\
i_q=6&&&&0.445948490915965 &  0.108103018168070    &    0.223381589678011/2.0 \\
\hline
i_q=1 &&&&0.1012865073235 &  0.1012865073235  &     0.0629695902724 \\
i_q=2 &&&&0.7974269853531 &  0.1012865073235  &     0.0629695902724 \\
i_q=3 &&&&0.1012865073235 &  0.7974269853531  &     0.0629695902724 \\
i_q=4 &&&&0.4701420641051 &  0.0597158717898  &     0.0661970763942 \\
i_q=5 &&&&0.4701420641051 &  0.4701420641051  &     0.0661970763942 \\
i_q=6 &&&&0.0597158717898 &  0.4701420641051  &     0.0661970763942 \\
i_q=7 &&&&0.3333333333333 &  0.3333333333333  &     0.1125000000000 \\
\hline
i_q=1&&&& 5.01426509658179e-01&  2.49286745170910e-01 &   5.83931378631895e-02 \\ 
i_q=2&&&& 2.49286745170910e-01&  5.01426509658179e-01 &   5.83931378631895e-02 \\ 
i_q=3&&&& 2.49286745170910e-01&  2.49286745170910e-01 &   5.83931378631895e-02 \\ 
i_q=4&&&& 8.73821971016996e-01&  6.30890144915020e-02 &   2.54224531851035e-02 \\ 
i_q=5&&&& 6.30890144915020e-02&  8.73821971016996e-01 &   2.54224531851035e-02 \\ 
i_q=6&&&& 6.30890144915020e-02&  6.30890144915020e-02 &   2.54224531851035e-02 \\ 
i_q=7&&&& 5.31450498448170e-02&  3.10352451033784e-01 &   4.14255378091870e-02 \\ 
i_q=8&&&& 6.36502499121399e-01&  5.31450498448170e-02 &   4.14255378091870e-02 \\ 
i_q=9&&&& 3.10352451033784e-01&  6.36502499121399e-01 &   4.14255378091870e-02 \\ 
i_q=10&&&& 5.31450498448170e-02&  6.36502499121399e-01 &   4.14255378091870e-02 \\ 
i_q=11&&&& 6.36502499121399e-01&  3.10352451033784e-01 &   4.14255378091870e-02 \\ 
i_q=12&&&& 3.10352451033784e-01&  5.31450498448170e-02 &   4.14255378091870e-02 \\ 
\hline
\end{array}
\]
}
where
\[ 
g_1 = \left(8-\sqrt{10} + \sqrt{38-44\sqrt{2/5}}\right)/18
\qquad
g_2 = \left(8-\sqrt{10} - \sqrt{38-44\sqrt{2/5}}\right)/18
\]
\[
w_1 = \left(620+\sqrt{213125-53320\sqrt{10}}\right)/3720
\qquad
w_2 = \left(620-\sqrt{213125-53320\sqrt{10}}\right)/3720
\]
All these are implemented in \stone~120 (see also \stone~112).


%-----------------------------------------------------------------
\subsubsection*{The case of a generic triangle}

Let $T$ be a triangular element with straight edges defined by the three vertices $(x_i,y_i)$ ($i=1,2,3$)
arranged in the counter-clockwise order:

INSERT FIGURE

We now need to evaluate the following integral over $T$:
\[
I = \iint_T f(x,y) \; dxdy
\]
We will proceed similarly to the quadrilateral case: first transform ('map') this triangle into 
the reference triangle $\triangle$ and then use the adequate quadrature rule. 

We can base this transformation ('mapping') on the linear ('$P_1$') basis functions\footnote{See
Section~\ref{ss:mappings}}: 
\begin{eqnarray} 
\bN_1(r,s) &=&1-r-s \nn\\
\bN_1(r,s) &=& r\nn\\
\bN_1(r,s) &=& s\nn
\end{eqnarray} 
with 
\begin{eqnarray} 
x=P(r,s)=\sum_{i=1}^3 x_i \bN_i(r,s)= x_1 \bN_1(r,s) + x_2 \bN_2(r,s) + x_3 \bN_3(r,s)  \nn\\
y=Q(r,s)=\sum_{i=1}^3 y_i \bN_i(r,s)= y_1 \bN_1(r,s) + y_2 \bN_2(r,s) + y_3 \bN_3(r,s)  \nn
\end{eqnarray} 
For example when $(r,s)\rightarrow(0,0)$ we find $(x,y)\rightarrow (x_1,y_1)$,
when $(r,s)\rightarrow(1,0)$ we find $(x,y)\rightarrow (x_2,y_2)$,
and when $(r,s)\rightarrow(0,1)$ we find $(x,y)\rightarrow (x_3,y_3)$.

Then we have 
\[
I = \iint_T f(x,y) \; dxdy = \iint_\triangle f(P(r,s),Q(r,s))\; |J(r,s)| \; drds
\]
where $J(r,s)$ is the Jacobian of the transformation:
\[
J(r,s) = 
\left|
\begin{array}{cc}
\frac{\partial x}{\partial r} & \frac{\partial y}{\partial r} \\ \\
\frac{\partial x}{\partial s} & \frac{\partial y}{\partial s} 
\end{array}
\right|
=
\left|
\begin{array}{cc}
\frac{\partial }{\partial r} \sum\limits_{i=1}^3 x_i \bN(r_i,s_i) & 
\frac{\partial }{\partial r} \sum\limits_{i=1}^3 y_i \bN(r_i,s_i) \\ \\
\frac{\partial }{\partial s} \sum\limits_{i=1}^3 x_i \bN(r_i,s_i) & 
\frac{\partial }{\partial s} \sum\limits_{i=1}^3 y_i \bN(r_i,s_i)
\end{array}
\right|
=
\left|
\begin{array}{cc}
(-x_1+x_2) & (-y_1+y_2) \\
(-x_1+x_3) & (-y_1+y_3)
\end{array}
\right|
=2 {\cal A}_T
\]
where ${\cal A}_T$ is the area of triangle $T$. In the end we have 
\[
I = \iint_T f(x,y) \; dxdy = 2 {\cal A}_T \iint_\triangle f(P(r,s),Q(r,s))\;  drds
\]
and the rhs integral can then be computed by means of quadrature.









