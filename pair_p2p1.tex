\index{general}{${\bm P}_2\times P_1$}
\begin{flushright} {\tiny {\color{gray} \tt pair\_p2p1.tex}} \end{flushright}
%~~~~~~~~~~~~~~~~~~~~~~~~~~~~~~~~~~~~~~~~~~~~~~~~~~~~~~~~~~~~~~~~~~~~~~~~~~~~~~~~~~~~~~~~~~~~~~~~~~

\noindent
\begin{minipage}{0.54\textwidth}
From Segal \cite{segal}: 
\begin{displayquote}
{\color{darkgray}
Taylor-Hood elements \cite{taho73} 
are characterized by the fact that the pressure is continuous in the region $\Omega$. 
A typical example is the quadratic triangle (${\bm P}_2\times P_1$ element).
In this element the velocity is approximated by a quadratic polynomial and the pressure by a
linear polynomial. One can easily verify that both approximations are continuous over 
the element boundaries.}
\end{displayquote}

It can be shown, Segal (1979), that this element is admissible if at least 3 elements 
are used. The quadrilateral counterpart of this triangle is the ${\bm Q}_2\times Q_1$ element.
Reddy and Gartling \cite[p179]{reddybook2} also report this element to be LBB stable.
It is also mentioned in \textcite{nath93}.

\Literature: \textcite{scan85} (1985), \textcite{lejx14} (2014), \textcite{cump20} (2020)
\end{minipage}
\hfill
\begin{minipage}{0.42\textwidth}
\begin{center}
\input{tikz/tikz_p2p1}
\end{center}
\end{minipage}

Note that the element is not stable if the triangulation contains less than three triangles
(Lemma 3.125 of \cite{john16} on page 100).

Optimal error estimates are derived for this element pair in \textcite{sten87} (1987). 

