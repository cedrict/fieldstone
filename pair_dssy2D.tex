This element is often referred to as the 'DSSY' element because of the 
four authors of the original paper: Douglas, Santos, sheen and Ye (1999) \cite{doss99}.

The non-conforming finite element space $Q_l$ is defined based on the 
reference square element on $[-1,1]^2$ :
\[
Q_l = \text{Span} \left\{ 1, r, s, \theta_l(r)-\theta_l(s)  \right\}
\qquad l=1,\; \text{or} \; 2
\]
with
\begin{eqnarray}
\theta_1(r)  &=& r^2-\frac53r^4  \nn\\
\theta_1'(r) &=& 2r-\frac{20}{3}r^3  \nn\\
\theta_2(r)  &=& r^2-\frac{25}{6} r^4 + \frac72 r^6 \\ 
\theta_2'(r) &=& 2r-\frac{50}{3} r^3 + 21 r^5
\end{eqnarray}
The dimension of $Q_l$ is four and the $\theta_l$ functions look like:
\begin{center}
\includegraphics[width=7cm]{images/dssy/theta1}
\includegraphics[width=7cm]{images/dssy/theta2}
\end{center}
We have:
\begin{itemize}
\item $\theta_1(r=-1)=\theta_1(r=+1)=-\frac23$, $\theta_1(r=0)=0$ 
\item $\theta_2(r=-1)=\theta_2(r=+1)=\frac13$, $\theta_2(r=0)=0$ 
\end{itemize}
The nodes are situated at the mid-edges of the quadrilateral:

\input{tikz/tikz_dssy2D}

The basis function corresponding to the node (1, 0) is given by
\begin{mdframed}[backgroundcolor=blue!5]
\begin{eqnarray}
\bN_1(r,s)^{(l)} &=& \frac{1}{4} - \frac{1}{2} r + \frac{\theta_l(r)-\theta_l(s)}{4 \theta_l(1)}  \nn\\
\bN_2(r,s)^{(l)} &=& \frac{1}{4} + \frac{1}{2} r + \frac{\theta_l(r)-\theta_l(s)}{4 \theta_l(1)}  \nn\\
\bN_3(r,s)^{(l)} &=& \frac{1}{4} - \frac{1}{2} s - \frac{\theta_l(r)-\theta_l(s)}{4 \theta_l(1)}  \nn\\
\bN_4(r,s)^{(l)} &=& \frac{1}{4} + \frac{1}{2} s - \frac{\theta_l(r)-\theta_l(s)}{4 \theta_l(1)}  
\end{eqnarray}
\end{mdframed}
We can easily verify that $\sum\limits_i \bN_i(r,s,t)=1$ and that $\bN_i(\vec{r}_j)=\delta_{ij}$:
\begin{eqnarray}
\bN_1^{(l)}(r_1,s_1) 
&=& \frac{1}{4} -\frac{1}{2} (-1) + \frac{\theta_l(-1)-\theta_l(0)}{4 \theta_l(1)}  
= \frac{1}{4} +\frac{1}{2}  + \frac{\theta_l(-1)}{4 \theta_l(1)}  
= \frac{1}{4} +\frac{1}{2}  + \frac{1}{4}   = 1 \nn\\
\bN_1^{(l)}(r_2,s_2)
&=& \frac{1}{4} -\frac{1}{2} (+1) + \frac{\theta_l(+1)-\theta_l(0)}{4 \theta_l(1)}  
= \frac{1}{4} -\frac{1}{2} + \frac{\theta_l(+1)}{4 \theta_l(1)}  
= \frac{1}{4} -\frac{1}{2} + \frac{1}{4}   = 0 \nn\\
\bN_1^{(l)}(r_3,s_3)
&=& \frac{1}{4} -\frac{1}{2} (0) + \frac{\theta_l(0)-\theta_l(-1)}{4 \theta_l(1)}  
= \frac14 -\frac14  = 0 \nn\\
\bN_1^{(l)}(r_4,s_4)
&=& \frac{1}{4} -\frac{1}{2} (0) + \frac{\theta_l(0)-\theta_l(+1)}{4 \theta_l(1)}  
= \frac14 -\frac14  = 0 \nn\\
\bN_2^{(l)}(r_1,s_1) 
&=& \frac{1}{4} + \frac{1}{2} (-1) + \frac{\theta_l(-1)-\theta_l(0)}{4 \theta_l(1)}  
= \frac14 -\frac12 + \frac14 = 0 \nn\\
\bN_2^{(l)}(r_2,s_2)
&=& \frac{1}{4} + \frac{1}{2} (+1) + \frac{\theta_l(+1)-\theta_l(0)}{4 \theta_l(1)}  
= \frac14 + \frac12 + \frac14 =1 \nn\\
\bN_2^{(l)}(r_3,s_3)
&=& \frac{1}{4} + \frac{1}{2} (0) + \frac{\theta_l(0)-\theta_l(-1)}{4 \theta_l(1)}  
= \frac14 - \frac14 = 0 \nn\\
\bN_2^{(l)}(r_4,s_4)
&=& \frac{1}{4} + \frac{1}{2} (0) + \frac{\theta_l(0)-\theta_l(+1)}{4 \theta_l(1)}  
= \frac14 - \frac14 = 0 \nn\\
\bN_3^{(l)}(r_1,s_1)
&=& \frac{1}{4} - \frac{1}{2} (0) - \frac{\theta_l(-1)-\theta_l(0)}{4 \theta_l(1)} 
= \frac14 -\frac14 = 0\nn\\
\bN_3^{(l)}(r_2,s_2)
&=& \frac{1}{4} - \frac{1}{2} (0) - \frac{\theta_l(+1)-\theta_l(0)}{4 \theta_l(1)} 
= \frac14 -\frac14 = 0\nn\\
\bN_3^{(l)}(r_3,s_3)
&=& \frac{1}{4} - \frac{1}{2} (-1) - \frac{\theta_l(0)-\theta_l(-1)}{4 \theta_l(1)} 
= \frac14 +\frac12 + \frac14 = 1\nn\\
\bN_3^{(l)}(r_4,s_4)
&=& \frac{1}{4} - \frac{1}{2} (+1) - \frac{\theta_l(0)-\theta_l(+1)}{4 \theta_l(1)} 
= \frac14 -\frac12 + \frac14 = 0\nn\\
\bN_4^{(l)}(r_1,s_1)
&=& \frac{1}{4} + \frac{1}{2} (0) - \frac{\theta_l(-1)-\theta_l(0)}{4 \theta_l(1)}  
= \frac14 -\frac14 =0\nn\\
\bN_4^{(l)}(r_2,s_2)
&=& \frac{1}{4} + \frac{1}{2} (0) - \frac{\theta_l(+1)-\theta_l(0)}{4 \theta_l(1)}  
= \frac14 -\frac14 =0\nn\\
\bN_4^{(l)}(r_3,s_3)
&=& \frac{1}{4} + \frac{1}{2} (-1) - \frac{\theta_l(0)-\theta_l(-1)}{4 \theta_l(1)}  
= \frac14 -\frac12 +\frac14 = 0 \nn\\
\bN_4^{(l)}(r_4,s_4)
&=& \frac{1}{4} + \frac{1}{2} (1) - \frac{\theta_l(0)-\theta_l(1)}{4 \theta_l(1)}  
= \frac14 +\frac12 +\frac14 = 1 \nn
\end{eqnarray}

The basis functions can also be explicitly written for $\theta_1$ as in Cai \etal \cite{cady99}:
\begin{eqnarray}
\bN_1(r,s)^{(l)} 
&=& \frac{1}{4} - \frac{1}{2} r - \frac38 \left[\left( r^2-\frac53r^4 \right) - \left(s^2-\frac53s^4 \right) \right] \nn\\
\bN_2(r,s)^{(l)} 
&=& \frac{1}{4} + \frac{1}{2} r - \frac38 \left[\left( r^2-\frac53r^4 \right) - \left(s^2-\frac53s^4 \right) \right] \nn\\
\bN_3(r,s)^{(l)} 
&=& \frac{1}{4} - \frac{1}{2} s + \frac38 \left[\left( r^2-\frac53r^4 \right) - \left(s^2-\frac53s^4 \right) \right] \nn\\
\bN_4(r,s)^{(l)} 
&=& \frac{1}{4} + \frac{1}{2} s + \frac38 \left[\left( r^2-\frac53r^4 \right) - \left(s^2-\frac53s^4 \right) \right] 
\end{eqnarray}

The derivatives of the basis functions are as follows:
\begin{eqnarray}
\partial_r \bN_1(r,s)^{(l)} &=&  - \frac{1}{2}  + \frac{\theta_l'(r)}{4 \theta_l(1)}  \nn\\
\partial_r \bN_2(r,s)^{(l)} &=&  + \frac{1}{2}  + \frac{\theta_l'(r)}{4 \theta_l(1)}  \nn\\
\partial_r \bN_3(r,s)^{(l)} &=&  - \frac{\theta_l'(r)}{4 \theta_l(1)}  \nn\\
\partial_r \bN_4(r,s)^{(l)} &=&  - \frac{\theta_l'(r)}{4 \theta_l(1)}  
\end{eqnarray}

\begin{eqnarray}
\partial_s \bN_1(r,s)^{(l)} &=&   -\frac{\theta_l'(s)}{4 \theta_l(1)}  \nn\\
\partial_s \bN_2(r,s)^{(l)} &=&   -\frac{\theta_l'(s)}{4 \theta_l(1)}  \nn\\
\partial_s \bN_3(r,s)^{(l)} &=&   - \frac{1}{2} + \frac{\theta_l'(s)}{4 \theta_l(1)}  \nn\\
\partial_s \bN_4(r,s)^{(l)} &=&   + \frac{1}{2} + \frac{\theta_l'(s)}{4 \theta_l(1)}  
\end{eqnarray}



Note that a correction was issued in \textcite{cads00} (2000) if a 
true quadrilateral (i.e., one having two opposite, nonparallel edges) is included in
the partition. The authors state that in the case of rectangles the original method is fine.

\Literature: 
Park \& Sheen (2003) \cite{pash03},
Jeon \etal (2013) \cite{jens13},
Park, Sheen \& Shin (2013) \cite{pass13},
Bangerth \etal (2017) \cite{baks17},
Sheen (2020) \cite{shee20}
