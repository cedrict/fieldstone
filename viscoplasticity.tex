
\Literature: 
Zienkiewicz \& Cormeau (1974) \cite{zico74,zico74b},
Zienkiewicz \& Godbole (1974) \cite{zigo74},
Zienkiewicz (1975) \cite{zien75},
Cormeau (1975) \cite{corm75},
Zienkiewicz \& Godbole (1975) \cite{zigo75},
Zienkiewicz \etal (1975) \cite{zihl75},
Zienkiewicz \etal (1978) \cite{zijo78},
Vilotte \etal (1982) \cite{vidm82},
Hughes (1983) \cite{hugh83},
Vilotte \etal (1984) \cite{vidm84},
Vermeer \& de Borst (1984) \cite{vede84},
Zienkiewicz \etal (1985) \cite{zivt85},
Vilotte \etal (1986) \cite{vimd86},
de Borst (1988) \cite{debo88},
Leroy \& Ortiz (1989) \cite{leor89},
de Borst (1991) \cite{debo91},
de Borst \& M{\"u}lhaus (1992) \cite{demu92},
Sluys \& de Borst (1992) \cite{slde92},
de Borst \etal (1993) \cite{desm93},
Wang \etal (1996) \cite{wasd96},
Wang \etal (1997) \cite{wasd97},
van der Veen \etal (1999) \cite{vavd99},
Huh \etal (1999) \cite{huhy99},
Ju \& Im (2000) \cite{juim00},
de Borst (2001) \cite{debo01},
Moresi \etal (2001) \cite{modm01},
Moresi \etal (2002) \cite{modm02},
Heeres \etal (2002) \cite{hesd02},
Kaus \etal (2004) \cite{kaps04},
Dias da Silva \etal (2004) \cite{dias04},
Bazant \& Jirasek (2002) \cite{baji02},
Sch{\"o}pfer \etal (2013) \cite{sccm13},
Le Pourhiet (2013) \cite{lepo13},
Miehe \etal (2013) \cite{miam13},
Choi \& Pedersen (2015) \cite{chpe15},
Frohne \etal (2016) \cite{frhb16},
Duretz \etal (2018) \cite{dusd18},
Duretz \etal (2019) \cite{dudl19},
Duretz \etal (2020) \cite{dudy20}.


\Literature: 
Fractal distribution of shear bands in \cite{pohp94,pohe94}.





\subsubsection{A quick and dirty implementation}

This formulation is quite easy to implement. It is widely used, e.g. \cite{will92,thfb08,spmw16}, and relies on the assumption that 
a scalar quantity $\eta_p$ (the 'effective plastic viscosity') exists such that the deviatoric stress tensor 
\begin{equation}
{\bm \tau}=2\eta_p \dot{\bm\varepsilon} \label{eqscpl1}
\end{equation}
is bounded by some yield stress value $Y$. No other spring or dashpot is present, the deformation is purely plastic no 
matter what.
From Eq. (\ref{eqscpl1}) it follows that ${\tau}_{e}= 2\eta_p \dot{\varepsilon}_{e}=Y$ which yields
\begin{mdframed}[backgroundcolor=blue!5]
\[
\eta_p = \frac{Y}{2 \dot{\varepsilon}_{e}}
\]
\end{mdframed}
This approach has also been coined the Viscosity Rescaling Method (VRM) \cite{kacha04}. 
\index{general}{VRM} \index{general}{Viscosity Rescaling Method}

It is at this stage important to realize that (i) in areas where the strain rate is low, the resulting effective viscosity will be large, and 
(ii) in areas where the strain rate is high, the resulting effective viscosity will be low. This is not without consequences since 
(effective) viscosity contrasts up to 8-10 orders of magnitude have been observed/obtained with this formulation and it makes the FE 
matrix very stiff, leading to (iterative) solver convergence issues.
In order to contain these viscosity contrasts one usually resorts to viscosity limiters $\eta_{min}$ and $\eta_{max}$ such that 
\[
\eta_{min} \leq \eta_p \leq \eta_{max}
\]
Caution must be taken when choosing both values as they may influence the final results.

However, this approach is problematic for many reasons and represents an overly simplified end-member. 




