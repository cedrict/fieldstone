\index{general}{Stream Function}

\Literature \cite{giju98}\cite{scja81}\cite{chyu84}\cite{chri84}\cite{hayu94}\cite{olwh97}




The Stream function (commonly denoted by $\Phi$ or $\Psi$) approach is a useful approach in 
fluid dynamics as it 
can provide relatively quick solutions to 2D incompressible flow problems.
Using a stream function
formulation is numerically convenient because velocity information is contained in a single scalar equation
and pressure vanishes from the solution process.
The stream function is a function of coordinates and time of an inviscid liquid.
It allows to determine the components of velocity by differentiating the stream function 
with respect to the space coordinates. 
A family of curves $\Psi = const$ represent {\it streamlines}, i.e. 
the stream function remains constant along a streamline. 
Although also valid in 3D, this approach is mostly used in 2D because of its 
relative simplicity {\color{red} REFERENCES}.

%........................................
\subsubsection{In Cartesian coordinates}

In two dimensions the velocity is obtained as follows:
\begin{equation}
{\vec \upnu} = (u,v) = \left( \frac{\partial \Psi}{\partial y},-\frac{\partial \Psi}{\partial x} \right) 
\end{equation}
Provided the function $\Psi$ is a smooth enough function, 
this automatically insures that the flow is incompressible:
\begin{equation}
{\vec \nabla}\cdot {\vec \upnu} = 
\frac{\partial u}{\partial x} + \frac{\partial v}{\partial y}
=
\frac{\partial^2 \Psi}{\partial xy} - \frac{\partial^2 \Psi}{\partial xy} =0 
\end{equation}
Assuming constant viscosity, the Stokes equation writes:
\begin{equation}
-{\vec \nabla}p + \eta \Delta {\vec \upnu} + \rho {\vec g} = \vec{0}
\end{equation}
Let us introduce the vector ${\vec{W}}$ for convenience such that in each dimension:
\begin{eqnarray}
W_x&=&-\frac{\partial p}{\partial x} 
+ \eta\left( \frac{\partial^2 u}{\partial x^2} + \frac{\partial^2 u}{\partial x^y} \right) \\
W_y&=&-\frac{\partial p}{\partial y} 
+ \eta \left(\frac{\partial^2 v}{\partial x^2} + \frac{\partial^2 v}{\partial x^y} \right) 
\end{eqnarray}
Taking the curl of the vector ${\vec{W}}$ and only considering the component perpendicular to the $xy$-plane:
\begin{equation}
\frac{\partial W_y}{\partial x} - \frac{\partial W_x}{\partial y}  = 
-\frac{\partial \rho g_y}{\partial x} + \frac{\partial \rho g_x}{\partial y}   
\end{equation}
The advantage of this approach is that the pressure terms cancel out (the curl of a gradient is always zero), 
so that:
\begin{equation}
\frac{\partial}{\partial x}\eta\left( \frac{\partial^2 v}{\partial x^2} + \frac{\partial^2 v}{\partial x^y}  \right) 
- \frac{\partial }{\partial y} \eta \left( \frac{\partial^2 u}{\partial x^2} + \frac{\partial^2 u}{\partial x^y} \right) = 
-\frac{\partial \rho g_y}{\partial x} + \frac{\partial \rho g_x}{\partial y}   
\end{equation}
and then replacing $u,v$ by the their stream function derivatives yields (for a constant viscosity):
\begin{equation}
\eta \left(\frac{\partial^4 \Psi}{\partial x^4} + 
\frac{\partial^4 \Psi}{\partial y^4} + 
2\frac{\partial^4 \Psi}{\partial x^2y^2} \right)
=
-\frac{\partial \rho g_y}{\partial x} + \frac{\partial \rho g_x}{\partial y}   
\end{equation}
or, 
\begin{equation}
\eta {\vec \nabla}^4 \Psi 
=
\left(\frac{\partial^2 }{\partial x^2} + \frac{\partial^2 }{\partial y^2} \right) 
\left(\frac{\partial^2 }{\partial x^2} + \frac{\partial^2 }{\partial y^2} \right) \Psi
=
-\frac{\partial \rho g_y}{\partial x} + \frac{\partial \rho g_x}{\partial y}   
\label{eq:sf1}
\end{equation}
Note that $\vec\nabla^2 \vec\nabla^2 = \vec\nabla^4 $ is known as the Biharmonic operator.
\index{general}{Biharmonic Operator} 

These equations are also to be found in the geodynamics literature, 
see Eq. 1.43 \cite[eq. 1.43]{tack10} or \cite[p 70-71]{gery10}.

In the presence of temperature variations and multiple compositions, 
Trim et al (2020) \cite{trlb20}  use the  following nondimensional 
equation:
\[
\left(
\frac{\partial^2 }{\partial x^2} - 
\frac{\partial^2 }{\partial y^2}  
\right)
\left[ \eta
\left(
\frac{\partial^2 \Psi}{\partial x^2} - 
\frac{\partial^2 \Psi}{\partial y^2}  
\right)
\right]
+4
\frac{\partial^2 }{\partial xy} 
\left[
\eta 
\frac{\partial^2 \Psi}{\partial xy} 
\right]
=
Ra_T \frac{\partial T}{\partial x}-
Ra_C \frac{\partial C}{\partial x}
\]
\todo[inline]{check/rederive this formula!}


%........................................
\subsubsection{In Cylindrical coordinates}

TODO

VERIFY THOSE! minus signs ?
\[
\upnu_r=\frac{1}{r}\frac{\partial \Phi}{\partial \theta} 
\]
\[
\upnu_\theta=-\frac{\partial \Phi}{\partial r} 
\]
