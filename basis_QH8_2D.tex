\begin{flushright} {\tiny {\color{gray} \tt basis\_QH8\_2D.tex}} \end{flushright}
%~~~~~~~~~~~~~~~~~~~~~~~~~~~~~~~~~~~~~~~~~~~~~~~~~~~~~~~~~~~~~~~~~~~~~~~~~~~~~~~~~~~~~~~~~~~~~~~~~~

This element is proposed in \textcite{zhxi20} (2020). Two remarks
must be made: 1) Eq.~(29) of their publication which is the definition
of the basis functions contains an error\footnote{
Answer from the author: ``N5 to N8 is missing an A in the denominator and 
the calculation program does not have this problem''}. 
2) The authors use a rather 
uncommon and annoying rotated numbering:
\begin{verbatim}
      y
      |
2=====5=====1             3=====6=====2
|           |             |           |   (r_0,s_0)=(-1,-1)   (r_4,s_4)=( 0,-1)
|           |             |           |   (r_1,s_1)=(+1,-1)   (r_5,s_5)=(+1, 0)
6           8--x          7     +     5   (r_2,s_2)=(+1,+1)   (r_6,s_6)=( 0,+1)
|           |             |           |   (r_3,s_3)=(-1,+1)   (r_7,s_7)=(-1, 0)
|           |             |           |    
3=====7=====4             0=====4=====1
Zhang & Xiang             our numbering
\end{verbatim}

For each element they define (their numbering):
\begin{eqnarray}
A   &=& \frac{1}{2} [ (x_1-x_3)(y_2-y_4)-(x_2-x_4)(y_1-y_3) ] \nn\\
m_x &=& (x_1-x_4)(y_2-y_3)-(x_2-x_3)(y_1-y_4) \nn\\
m_y &=& (x_3-x_4)(y_1-y_2)-(x_1-x_2)(y_3-y_4) \nn
\end{eqnarray}
Note that $A$ is the area of the element, and that in the case when 
the element is a rectangle then $m_x=m_y=0$.
\begin{eqnarray}
\bN_1(r,s)&=& n_1(r,s) +(m_x^2 - m_xm_y + m_y^2)\frac{E(r,s)}{D} \nn\\
\bN_2(r,s)&=& n_2(r,s) +(m_x^2 + m_xm_y + m_y^2)\frac{E(r,s)}{D} \nn\\
\bN_3(r,s)&=& n_3(r,s) +(m_x^2 - m_xm_y + m_y^2)\frac{E(r,s)}{D} \nn\\
\bN_4(r,s)&=& n_4(r,s) +(m_x^2 + m_xm_y + m_y^2)\frac{E(r,s)}{D} \nn\\
\bN_5(r,s)&=& n_5(r,s) -m_x(2Am_x+m_y^2)\frac{E(r,s)}{AD} \nn\\
\bN_6(r,s)&=& n_6(r,s) -m_y(2Am_y+m_x^2)\frac{E(r,s)}{AD} \nn\\
\bN_7(r,s)&=& n_7(r,s) +m_x(-2Am_x+m_y^2)\frac{E(r,s)}{AD} \nn\\
\bN_8(r,s)&=& n_8(r,s) +m_y(-2Am_y+m_x^2)\frac{E(r,s)}{AD} \nn
\end{eqnarray}
with 
\[
E(r,s)=(1-r^2)(1-s^2)
\qquad
D=4(4A^2+m_x^2+m_y^2)
\]
and where the $n_i$ functions are the basis functions of the 'regular' 
8-node element (see Section~\ref{sec:serendipity2D}).

This is implemented in \stone 52.

\todo[inline]{not finished. SHOW CONSISTENCY !! like in paper
email sent to author about mistake.  }

Let us verify consistency:
\begin{eqnarray}
\sum_{i=1}^8 \bN_i(r,s) 
&=& \underbrace{\sum_{i=1}^8 n_i(r,s)}_{=0} + \frac{E(r,s)}{D} 
\left[
(m_x^2 - m_xm_y + m_y^2)
+(m_x^2 + m_xm_y + m_y^2)
+(m_x^2 - m_xm_y + m_y^2)
+(m_x^2 + m_xm_y + m_y^2) \right. \nn\\
&& \left.
-m_x(2Am_x+m_y^2)\frac{1}{A}
-m_y(2Am_y+m_x^2)\frac{1}{A}
+m_x(-2Am_x+m_y^2)\frac{1}{A}
+m_y(-2Am_y+m_x^2)\frac{1}{A}
\right] \nn\\
&=& \frac{E(r,s)}{D}
\left[(4m_x^2 + 4m_y^2) + \frac{1}{A} (-4Am_x^2 - 4A m_y^2) \right]  \\
&=& 0
\end{eqnarray}


