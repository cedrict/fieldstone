
\begin{flushright} {\tiny {\color{gray} \tt fdm\_wave\_eq.tex}} \end{flushright}
%~~~~~~~~~~~~~~~~~~~~~~~~~~~~~~~~~~~~~~~~~~~~~~~~~~~~~~~~~~~~~~~~~~~~~~~~~~~~~~~~~~~~~~~~~~~~~~~~~~

We start with the one-dimensional wave equation:
\[
u_{tt}=c^2 u_{xx}  \qquad x\in[0,L], \quad t\in[0,T]
\]
which is supplemented by the initial conditions
\begin{eqnarray}
u(x,0) &=& f(x) \qquad  \forall x\in[0,L] \nn\\
u_t(x,0) &=& g(x) \qquad  \forall x\in[0,L] 
\end{eqnarray}
and the boundary conditions
\begin{eqnarray}
u(0,t) &=& 0 \qquad  \forall t \in [0,T] \nn\\
u_t(L,t) &=& 0 \qquad  \forall t\in[0,T] \label{eq:fdm_wave_bc}
\end{eqnarray}


We then proceed to discretise the space domain with $nnx$ equidistant nodes (forming $ncellx=nnx-1$ cells)
at locations $x_i$, $i=0,...nnx-1$ and the time domain with $N_t$ equidistant nodes at times
$t_i$, $i=1,...N_t$.
We define $h=L/ncellx$ and $\delta\! t=T/(N_t-1)$

We are looking for the solution $u(x,t)$ at all mesh points in space in time and denote by 
$u_i^k$ the discrete solution at location $x_i$ and time $t_k$.

Following Section~\ref{XYZ} the second-order derivatives will be replaced by central differences 
stencils, i.e. for a function $f$ we would write
\[
\frac{\partial^2  f}{\partial t^2} \simeq \frac{ f^{k+1} -2f^{k} +f^{k-1}  }{\delta\! t^2}
\]

%-----------------------------------
\subsection{Simple explicit method}
\label{ss:fdmwavess}

The first idea is just to use central differences for both time and space derivatives,
while assuming that the rhs (space) term is taken at time $t_k$:
i.e.
\[
\frac{ u_i^{k+1} -2u_i^{k} +u_i^{k-1}  }{\delta\! t^2} = 
c^2
\frac{ u_{i+1}^k -2u_{i}^k +u_{i-1}^k  }{h^2} 
\]
We then define $\alpha = c\; \delta\! t / h$, so that we can write the equation above as
\begin{equation}
u_i^{k+1} -2u_i^{k} +u_i^{k-1}  = \alpha^2 (u_{i+1}^k -2u_{i}^k +u_{i-1}^k )
\label{eq:fdmwave1}
\end{equation}
or,
\begin{mdframed}[backgroundcolor=blue!5]
\begin{equation}
u_i^{k+1} = - u_i^{k-1} +2(1-\alpha^2) u_{i}^k + \alpha^2 (u_{i+1}^k +u_{i-1}^k ) 
\label{eq:wavee3}
\end{equation}
\end{mdframed}
The parameter $\alpha$ is important and is often denoted by $C$ or $C_{CFL}$ as it is the 
Courant–Friedrichs–Lewy number\footnote{\url{https://en.wikipedia.org/wiki/Courant-Friedrichs-Lewy_condition}}.

Of course, we need to be careful about the initial conditions. Indeed, looking 
at the left hand side of Eq.~\eqref{eq:fdmwave1}, we see that we have three values of $u_i$
at three consecutive different times.
In other words, in order to compute $u_i^{k+1}$ we need two previous values 
of $u_i$, i.e. $u_i^{k-1}$ and $u_{i}^k$. 

The relationship above can be written for any node inside the domain but not for those on the 
boundaries ($x=0,L$). However this is not a problem since we need not to write such an equation
as the values of $u$ at the extremities of the domain are given by the boundary conditions of 
Eq.~\eqref{eq:fdm_wave_bc}.

The initial condition on $u$ is trivial to implement. But quid of $u_t$? In this case we would 
write a centered difference as 
\[
g(x_i)=\left. \frac{\partial u}{\partial t} \right|_{i}^0 \simeq \frac{u_i^1-u_i^{-1}}{2 \delta \! t} 
\]
so that 
\[
u_i^{-1} = u_i^1 - 2 \delta \! t \; g(x_i)
\]
and then, when inserted in Eq.~\eqref{eq:wavee3}, we get for $k=0$
\begin{eqnarray}
u_i^{1} 
&=& - u_i^{-1} +2(1-\alpha^2) u_{i}^0 + \alpha^2 (u_{i+1}^0 +u_{i-1}^0)  \nn\\
&=& - (u_i^1 - 2 \delta \! t g(x_i))  +2(1-\alpha^2) u_{i}^0 + \alpha^2 (u_{i+1}^0 +u_{i-1}^0)  \nn\\
2 u_i^1 &=& 2 \delta \! t g(x_i)  +2(1-\alpha^2) u_{i}^0 + \alpha^2 (u_{i+1}^0 +u_{i-1}^0)  \nn\\
u_i^1 &=&  \delta \! t g(x_i)  + (1-\alpha^2) f(x_{i}) + \frac{\alpha^2}{2} (f(x_{i+1}) + f(x_{i-1}) 
\end{eqnarray}
We now have all discrete $u$ values for the first and second time step so we can  
apply the algorithm delineated by Eq.~\eqref{eq:wavee3} to arrive at $u$ at subsequent time steps.

FIG 1 of Langtangen

{\color{red} explain that this is conditionally stable.}

%-----------------------------------
\subsection{Implicit method(s)}

We could overcome the problem of conditional stability by 
introducing an implicit scheme as follows (this time 
the space derivative terms are taken at time step $k+1$):
\[
\frac{ u_i^{k+1} -2u_i^{k} +u_i^{k-1}  }{\delta\! t^2} = 
c^2
\frac{ u_{i+1}^{k+1} -2u_{i}^{k+1} +u_{i-1}^{k+1}  }{h^2} 
\]
which leads to 
\[
u_i^{k+1} -2u_i^{k} +u_i^{k-1}  = 
\alpha ^2 ( u_{i+1}^{k+1} -2u_{i}^{k+1} +u_{i-1}^{k+1} )
\]
or,
\begin{mdframed}[backgroundcolor=blue!5]
\begin{equation}
-\alpha^2 u_{i+1}^{k+1}+
(1+2\alpha^2) u_i^{k+1} 
-\alpha^2 u_{i-1}^{k+1}
=2 u_i^k -u_i^{k-1}
\end{equation}
\end{mdframed}

Another approach, the Crank-Nicolson method\footnote{\url{https://en.wikipedia.org/wiki/Crank-Nicolson_method}}:
\[
\frac{ u_i^{k+1} -2u_i^{k} +u_i^{k-1}  }{\delta\! t^2} = 
c^2 \frac12 \left(
\frac{ u_{i+1}^{k+1} -2u_{i}^{k+1} +u_{i-1}^{k+1}  }{h^2} 
+
\frac{ u_{i+1}^k -2u_{i}^k +u_{i-1}^k  }{h^2} 
\right)
\]


\[
u_i^{k+1} -2u_i^{k} +u_i^{k-1} 
=
\frac{\alpha^2}{2} \left(
u_{i+1}^{k+1} -2u_{i}^{k+1} +u_{i-1}^{k+1} 
+
u_{i+1}^k -2u_{i}^k +u_{i-1}^k 
\right)
\]
or,
\begin{mdframed}[backgroundcolor=blue!5]
\begin{equation}
(1+\alpha^2) u_i^{k+1} 
-\frac{\alpha^2}{2} (u_{i+1}^{k+1} +u_{i-1}^{k+1} )
=
(2-\alpha^2) u_i^{k} 
+\frac{\alpha^2}{2} (u_{i+1}^{k} +u_{i-1}^{k} )
-u_i^{k-1} 
\end{equation}
\end{mdframed}

Note that we have already encountered this method in the context of the 
diffusion equation solved with FDM in Section~\ref{ss:fdm_advdiff1D}.

%-------------------------------------------------------------
\subsection{Reflecting boundaries}

When a wave hits a boundary, either it encounters a (Dirichlet)
boundary condition as seen so far, but it can also encounter
a different type of boundary condition: it is reflected back 
via a Neumann boundary condition, which translates into:
\[
\frac{\partial u}{\partial n} = 0, \qquad \text{or} \quad
\vec{n} \cdot \vec\nabla u = 0
\]
In the case of a 1D domain the normal is $-\vec{e}_x$ at $x=0$
and $+\vec{e}_x$ at $x=L$, which then translates into
\[
\left. -\frac{\partial u}{\partial x} \right|_{x=0} = 0
\]
\[
\left. \frac{\partial u}{\partial x} \right|_{x=L} = 0
\]
Having established the nature of these boundaries, 
we must now look at their implementation.
For accuracy reasons we would like to resort to a central 
difference scheme (see Section~XXX).
However, this poses a problem, as visible when focusing on the 
left boundary (the time here is not important):
\begin{equation}
\left. -\frac{\partial u}{\partial x} \right|_{x=0} \simeq - \frac{u_1-u_{-1}}{2 h} =0
\label{eq:waveeq6}
\end{equation}
Obviously $u_{-1}$ does not exist! 
What we can do is write Eq.~\eqref{eq:wavee3} at node $i=0$:
\[
u_0^{k+1} = - u_0^{k-1} +2(1-\alpha^2) u_{0}^k + \alpha^2 (u_{1}^k +u_{-1}^k ) 
\]
and insert in it $u_{-1}$ obtained from Eq.~\eqref{eq:waveeq6}:
\begin{eqnarray}
u_0^{k+1} 
&=& - u_0^{k-1} +2(1-\alpha^2) u_{0}^k + \alpha^2 (u_{1}^k +u_{1}^k )  \nn\\
&=& - u_0^{k-1} +2 u_{0}^k + 2\alpha^2 (u_{1}^k -2u_0^k )  
\label{eq:waveeq7}
\end{eqnarray}
Likewise, on the right boundary (i.e. at node $nnx-1$), we have
\[
\left. \frac{\partial u}{\partial x} \right|_{x=L} \simeq \frac{u_{nnx}-u_{nnx-2}}{2h}  = 0
\]
which we now insert in Eq.~\eqref{eq:wavee3} expressed at $i=nnx-1$:
\begin{eqnarray}
u_{nnx-1}^{k+1} 
&=& - u_{nnx-1}^{k-1} +2(1-\alpha^2) u_{nnx-1}^k + \alpha^2 (u_{nnx-2}^k +u_{nnx-2}^k ) \nn\\
&=& - u_{nnx-1}^{k-1} +2 u_{nnx-1}^k + 2\alpha^2 (u_{nnx-2}^k-u_{nnx-1}^k  )
\label{eq:waveeq8}
\end{eqnarray}

In his lecture notes Langtangen proposes the following rather elegant 
impementation\footnote{I have adapted it to suit my notations}:
\begin{lstlisting}
for i in range(0, Nx+1):
    ip1 = i+1 if i < Nx else i-1
    im1 = i-1 if i > 0 else i+1
    u[i] = u_1[i] + C2*(u_1[im1] - 2*u_1[i] + u_1[ip1])
\end{lstlisting}
{\color{red} adapt to my notations!}


 


%-------------------------------------------------------------
\subsection{Generalisation: variable wave velocity}


In the case when the wave velocity is not a constant in space,
e.g. there are two media next to each other,
the Laplace term is actually of the form
\[
\frac{\partial }{\partial x} (c(x)^2 \frac{\partial u}{\partial x})
\]
which then needs to be discretised. 
The generic wave equation is then 
\[
\frac{\partial^2  u}{\partial t^2} = 
\frac{\partial }{\partial x} (c(x)^2 \frac{\partial u}{\partial x})
+h(x,t)
\]
At a given node in time-space it writes
\[
\frac{\partial^2  }{\partial t^2} u(x_i,t_n)= 
\frac{\partial }{\partial x} \left(c(x_i)^2 \frac{\partial }{\partial x} u(x_i,t_n) \right)
+h(x_i,t_n)
\]
The typical way to approach this is to first define 
\[
\phi(x,t) = c(x)^2 \frac{\partial u}{\partial x}
\]
so that a centered spatial derivative of $\phi$ writes
\[
\frac{\partial \phi}{\partial x} (x_i,t_n) \simeq \frac{\phi(x_{i+1/2},t_n) - \phi(x_{i-1/2},t_n)}{h_x}  
\]
Of course we then turn our attention to (let us call $\underline{c}=c^2$ to simplify notations):
\begin{eqnarray}
\phi(x_{i+1/2},t_n) 
&=& \underline{c}(x_{i+1/2}) \frac{\partial }{\partial x} u(x_{i+1/2},t_n)
\simeq \underline{c}(x_{i+1/2})^2 \frac{u_{i+1} - u_i}{h_x} \nn\\
\phi(x_{i-1/2},t_n) 
&=& \underline{c}(x_{i-1/2}) \frac{\partial }{\partial x} u(x_{i-1/2},t_n)
\simeq \underline{c}(x_{i-1/2})^2 \frac{u_i - u_{i-1}}{h_x} 
\end{eqnarray}
Putting it all together:
\begin{eqnarray}
\frac{\partial \phi}{\partial x} (x_i,t_n) 
&\simeq &
\frac{1}{h_x} \left( \underline{c}(x_{i+1/2}) \frac{u_{i+1} - u_i}{h_x}
-\underline{c}(x_{i-1/2}) \frac{u_i - u_{i-1}}{h_x} 
\right) \nn\\
&\simeq &
\frac{1}{h_x^2} \left( \underline{c}(x_{i+1/2}) (u_{i+1} - u_i)
-\underline{c}(x_{i-1/2}) (u_i - u_{i-1})
\right) 
\end{eqnarray}
But how can we compute $\underline{c}(x_{i+1/2})$? Three methods are commonly used. 
\begin{itemize}
\item arithmetic averaging: $\underline{c}(x_{i+1/2}) \simeq \frac12 (\underline{c}(x_i)+\underline{c}(x_{i+1}))$
\item geometric averaging: $\underline{c}(x_{i+1/2}) \simeq (\underline{c}(x_i)\underline{c}(x_{i+1}))^{1/2}$
\item harmonic averaging: $\underline{c}(x_{i+1/2}) \simeq  2(1/\underline{c}(x_i)+1/\underline{c}(x_{i+1}))^{-1}$
\end{itemize}
Of course similar expressions can then be used to compute $\underline{c}(x_{i-1/2})$.



%-------------------------------------------------------------
\subsection{Generalisation: damping}

%-------------------------------------------------------------
\subsection{Periodic boundary conditions}

%---------------------------------------------------------------
\subsection{(Von Neumann) Stability analysis}

%---------------------------------------------------------------
\subsection{Numerical dispersion relation}

%---------------------------------------------------------------
\subsection{Generalisation to 2D}


We start with the two-dimensional wave equation:
\[
u_{tt}=c^2 (u_{xx} +u_{yy}) \qquad x\in\Omega, \quad t\in[0,T]
\]
on the domain $\Omega=[0,L_x]\times[0,L_y]$ with boundary $\Gamma$,
which is supplemented by the initial conditions
\begin{eqnarray}
u(x,y,0) &=& f(x,y)   \qquad  \forall (x,y)\in \Omega\nn\\
u_t(x,y,0) &=& g(x,y) \qquad  \forall (x,y)\in \Omega
\end{eqnarray}
and the boundary conditions
\begin{eqnarray}
u(x,y,t) &=& 0 \qquad \forall (x,y)\in \Gamma \qquad \forall t \in [0,T] \nn\\
u_t(x,y,t) &=& 0 \qquad \forall (x,y)\in \Gamma \qquad \forall t \in [0,T] 
\label{eq:fdm_wave_bc}
\end{eqnarray}


We then proceed to discretise the space domain with $nnx$ equidistant nodes (forming $ncellx=nnx-1$ cells) in the $x$-direction at locations $x_i$, $i=0,...nnx-1$,  
and with $nny$ equidistant nodes (forming $ncelly=nny-1$ cells) in the $y$-direction 
at locations $y_j$, $j=0,...nny-1$,
and the time domain with $N_t$ equidistant nodes at times
$t_i$, $i=1,...N_t$.
We define $h_x=L_x/ncellx$, $h_y=L_y/ncelly$ and $\delta\! t=T/(N_t-1)$.



%===================================================
\paragraph{Explicit formulation}

The first idea is just to use central differences for both time and space derivatives,
while assuming that the rhs (space) term is taken at time $t_k$:
i.e.
\[
\frac{ u_{i,j}^{k+1} -2u_{i,j}^{k} +u_{i,j}^{k-1}  }{\delta\! t^2} = 
c^2
\left(
\frac{ u_{i+1,j}^k -2u_{i,j}^k +u_{i-1,j}^k  }{h_x^2} 
+
\frac{ u_{i,j+1}^k -2u_{i,j}^k +u_{i,j-1}^k  }{h_y^2} 
\right)
\]
We then define $\alpha_x = c\; \delta\! t / h_x$,
$\alpha_y = c\; \delta\! t / h_y$, so that we can write the equation above as
\begin{equation}
u_{i,j}^{k+1} -2u_{i,j}^{k} +u_{i,j}^{k-1}  
= 
\alpha_x^2 (u_{i+1,j}^k -2u_{i,j}^k +u_{i-1,j}^k )
+
\alpha_y^2 (u_{i,j+1}^k -2u_{i,j}^k +u_{i,j-1}^k )
\label{eq:fdmwave1}
\end{equation}

\begin{mdframed}[backgroundcolor=blue!5]
\[
u_{i,j}^{k+1}
= \alpha_x^2 (u_{i+1,j}^k -2u_{i,j}^k +u_{i-1,j}^k )
+
\alpha_y^2 (u_{i,j+1}^k -2u_{i,j}^k +u_{i,j-1}^k )
+2u_{i,j}^{k} -u_{i,j}^{k-1}
\]
\end{mdframed}



%===================================================
\paragraph{Implicit formulations}

\[
\frac{ u_{i,j}^{k+1} -2u_{i,j}^{k} +u_{i,j}^{k-1}  }{\delta\! t^2} = 
c^2
\left(
\frac{ u_{i+1,j}^{k+1} -2u_{i,j}^{k+1} +u_{i-1,j}^{k+1}  }{h_x^2} 
+
\frac{ u_{i,j+1}^{k+1} -2u_{i,j}^{k+1} +u_{i,j-1}^{k+1}  }{h_y^2} 
\right)
\]

\[
u_{i,j}^{k+1} -2u_{i,j}^{k} +u_{i,j}^{k-1}  
=
\alpha_x^2 (u_{i+1,j}^{k+1} -2u_{i,j}^{k+1} +u_{i-1,j}^{k+1} )
+
\alpha_y^2 (u_{i,j+1}^{k+1} -2u_{i,j}^{k+1} +u_{i,j-1}^{k+1} )
\]

\[
u_{i,j}^{k+1}
-
\alpha_x^2 (u_{i+1,j}^{k+1} -2u_{i,j}^{k+1} +u_{i-1,j}^{k+1} )
-
\alpha_y^2 (u_{i,j+1}^{k+1} -2u_{i,j}^{k+1} +u_{i,j-1}^{k+1} )
=
2u_{i,j}^{k} -u_{i,j}^{k-1} 
\]

\begin{mdframed}[backgroundcolor=blue!5]
\[
(1+2\alpha_x^2+2\alpha_y^2)u_{i,j}^{k+1}
-
\alpha_x^2 (u_{i+1,j}^{k+1}  +u_{i-1,j}^{k+1} )
-
\alpha_y^2 (u_{i,j+1}^{k+1}  +u_{i,j-1}^{k+1} )
=
2u_{i,j}^{k} -u_{i,j}^{k-1} 
\]
\end{mdframed}






\paragraph{Crank-Nicolson}.

\begin{eqnarray}
\frac{ u_{i,j}^{k+1} -2u_{i,j}^{k} +u_{i,j}^{k-1}  }{\delta\! t^2} 
&=& 
\frac{c^2}{2}
\left(
\frac{ u_{i+1,j}^{k} -2u_{i,j}^{k} +u_{i-1,j}^{k}  }{h_x^2} 
+
\frac{ u_{i,j+1}^{k} -2u_{i,j}^{k} +u_{i,j-1}^{k}  }{h_y^2} 
\right)
\nn\\
&+&
\frac{c^2}{2}
\left(
\frac{ u_{i+1,j}^{k+1} -2u_{i,j}^{k+1} +u_{i-1,j}^{k+1}  }{h_x^2} 
+
\frac{ u_{i,j+1}^{k+1} -2u_{i,j}^{k+1} +u_{i,j-1}^{k+1}  }{h_y^2} 
\right) 
\nn\\
u_{i,j}^{k+1} -2u_{i,j}^{k} +u_{i,j}^{k-1}  
&=&
\frac{\alpha_x^2}{2}
\left(
 u_{i+1,j}^{k} -2u_{i,j}^{k} +u_{i-1,j}^{k}  
+
 u_{i+1,j}^{k+1} -2u_{i,j}^{k+1} +u_{i-1,j}^{k+1}
\right) \nn\\
&+&
\frac{\alpha_y^2}{2}
\left(
u_{i,j+1}^{k} -2u_{i,j}^{k} +u_{i,j-1}^{k}  
+
u_{i,j+1}^{k+1} -2u_{i,j}^{k+1} +u_{i,j-1}^{k+1}
\right)
\nn
\end{eqnarray}

\begin{eqnarray}
&&u_{i,j}^{k+1} 
-\frac{\alpha_x^2}{2} \left( u_{i+1,j}^{k+1} -2u_{i,j}^{k+1} +u_{i-1,j}^{k+1} \right) 
-\frac{\alpha_y^2}{2} \left( u_{i,j+1}^{k+1} -2u_{i,j}^{k+1} +u_{i,j-1}^{k+1} \right) \nn\\
&=&
 \frac{\alpha_x^2}{2} \left( u_{i+1,j}^{k} -2u_{i,j}^{k} +u_{i-1,j}^{k}  \right)
+\frac{\alpha_y^2}{2} \left( u_{i,j+1}^{k} -2u_{i,j}^{k} +u_{i,j-1}^{k}  \right)
+2u_{i,j}^{k} -u_{i,j}^{k-1}  
\end{eqnarray}



\begin{mdframed}[backgroundcolor=blue!5]
\begin{eqnarray}
&&(1+\alpha_x^2+\alpha_y^2) u_{i,j}^{k+1} 
-\frac{\alpha_x^2}{2} \left( u_{i+1,j}^{k+1}  +u_{i-1,j}^{k+1} \right) 
-\frac{\alpha_y^2}{2} \left( u_{i,j+1}^{k+1}  +u_{i,j-1}^{k+1} \right) \nn\\
&=&
 \frac{\alpha_x^2}{2} \left( u_{i+1,j}^{k}  +u_{i-1,j}^{k}  \right)
+\frac{\alpha_y^2}{2} \left( u_{i,j+1}^{k}  +u_{i,j-1}^{k}  \right)
+(2-\alpha_x^2-\alpha_y^2)u_{i,j}^{k} -u_{i,j}^{k-1}  \label{eq:fdmwave2dCR}
\end{eqnarray}
\end{mdframed}

In the case $h_x=h_y=h$ then $\alpha_x=\alpha_y=\alpha$ and we obtain
\begin{eqnarray}
&&(1+2\alpha^2) u_{i,j}^{k+1} 
-\frac{\alpha^2}{2} \left( u_{i+1,j}^{k+1}  +u_{i-1,j}^{k+1}  
+ u_{i,j+1}^{k+1}  +u_{i,j-1}^{k+1} \right) \nn\\
&=&
\frac{\alpha^2}{2} \left( u_{i+1,j}^{k}  +u_{i-1,j}^{k}  
+ u_{i,j+1}^{k}  +u_{i,j-1}^{k}  \right)
+(2-2\alpha^2)u_{i,j}^{k} -u_{i,j}^{k-1}  \label{eq:fdmwave2dCR2}
\end{eqnarray}







%===============================================
\subsection{(von Neumann) Stability of the Crank-Nicolson method}

This section is inspired by \textcite{degefa2018} (2018).

\vspace{.5cm}

A numerical method is said to be stable if the cumulative effect of all the errors is
bounded independently from the number of mesh points. 

In order to establish whether a method is stable or not one often employs the so-called
Von Neumann\footnote{\url{https://en.wikipedia.org/wiki/John_von_Neumann}} stability analysis.

In general, the Von Neumann’s procedure
introduces an error represented by a finite Fourier series and examines how this error
propagates during the solution. 

To get stability of the 2D Crank-Nicolson method via this method we substitute
$u_{i,j}$ in Eq.~\eqref{eq:fdmwave2dCR2} by $u_{i,j}=\rho^k \exp (i\beta mh) \exp (i \gamma n h)$
in the homogeneous equation with
$h$ the mesh size in both directions, $m,n$ spatial indices and $\beta,\gamma$ time indices:

\begin{eqnarray}
&&(1+2\alpha^2) u_{i,j}^{k+1} 
-\frac{\alpha^2}{2} \left( u_{i+1,j}^{k+1}  +u_{i-1,j}^{k+1}  
+ u_{i,j+1}^{k+1}  +u_{i,j-1}^{k+1} \right) \nn\\
&=&
\frac{\alpha^2}{2} \left( u_{i+1,j}^{k}  +u_{i-1,j}^{k}  
+ u_{i,j+1}^{k}  +u_{i,j-1}^{k}  \right)
+(2-2\alpha^2)u_{i,j}^{k} -u_{i,j}^{k-1} 
\end{eqnarray}



\begin{eqnarray}
&&(1+2\alpha^2) \rho^{k+1} e^{i\beta mh}e^{i \gamma n h} 
-\frac{\alpha^2}{2} \left( 
 \rho^{k+1} e^{i\beta (m+1)h}e^{i \gamma n h}
+\rho^{k+1} e^{i\beta (m-1)h}e^{i \gamma n h}
+\rho^{k+1} e^{i\beta mh}e^{i \gamma (n+1) h}
+\rho^{k+1} e^{i\beta mh}e^{i \gamma (n-1) h}
\right) \nn\\
&&=
\frac{\alpha^2}{2} \left( 
 \rho^{k} e^{i\beta (m+1)h}e^{i \gamma n h}
+\rho^{k} e^{i\beta (m-1)h}e^{i \gamma n h}
+\rho^{k} e^{i\beta mh}e^{i \gamma (n+1) h}
+\rho^{k} e^{i\beta mh}e^{i \gamma (n-1) h}
\right) \nn\\
&&+(2-2\alpha^2) \rho^k e^{i\beta mh}e^{i \gamma n h} -\rho^{k-1} e^{i\beta mh}e^{i \gamma n h}  \nn
\end{eqnarray}
Dividing by $\rho^k e^{i\beta mh}e^{i \gamma n h}$ we obtain

\[
(1+2\alpha^2) \rho  
-\frac{\alpha^2}{2} \left( 
 \rho e^{i\beta h}
+\rho e^{-i\beta h}
+\rho e^{i \gamma  h}
+\rho e^{-i \gamma  h}
\right) =
\frac{\alpha^2}{2} \left( 
  e^{i\beta h}
+ e^{-i\beta h}
+ e^{i \gamma  h}
+ e^{-i \gamma  h}
\right) 
+(2-2\alpha^2)  -\rho^{-1}   \nn
\]

We then recall that 
\[
\cos \theta
=
1 -2 \sin^2 \frac{\theta}{2}
=
\frac{e^{i\theta}+e^{-i\theta}}{2}
\]
so that the equation above becomes
\[
(1+2\alpha^2) \rho  
-\frac{\alpha^2}{2} \rho\left( 
\underbrace{ e^{i\beta h} + e^{-i\beta h}}_{2 \cos (\beta h)}
+ \underbrace{e^{i \gamma  h}+ e^{-i \gamma  h} }_{2 \cos ( \gamma h)}
\right) =
\frac{\alpha^2}{2} \left( 
\underbrace{  e^{i\beta h} + e^{-i\beta h} }_{2 \cos (\beta h)}
+ \underbrace{ e^{i \gamma  h} + e^{-i \gamma  h} }_{2 \cos( \gamma h)}
\right) 
+(2-2\alpha^2)  -\rho^{-1} 
\]

\[
(1+2\alpha^2) \rho  
-\alpha^2 \rho\left[ \cos (\beta h)  +\cos (\gamma h) \right] =
\alpha^2 \left[ \cos (\beta h) + \cos(\gamma h) \right] 
+(2-2\alpha^2)  -\rho^{-1} 
\]


\[
(1+2\alpha^2) \rho^2  
-\alpha^2 \rho^2 \left[ \cos (\beta h)  +\cos (\gamma h) \right] 
-\alpha^2 \rho \left[ \cos (\beta h) + \cos(\gamma h) \right] 
-(2-2\alpha^2)\rho  +1 =0 
\]

\[
\rho^2 \left[
1 + \alpha^2 ( 2 - \cos (\beta h)  +\cos ( \gamma h) )
\right]
+\rho
\left[
-2 + \alpha^2 ( 2 - \cos (\beta h) - \cos( \gamma h) )
\right]
+1
=0
\]
Using now the formula 
\[
1-\cos \theta
=
2 \sin^2 \frac{\theta}{2}
\]


\[
\rho^2 \left[
1 + \alpha^2 \left( 2 \sin^2 \frac{\beta h}{2})  + 2 \sin^2  \frac{ \gamma h}{2} \right)
\right]
+\rho
\left[
-2 + \alpha^2 \left( 2\sin^2 \frac{\beta h}{2}) +  2\sin ^2 \frac{\gamma h}{2} \right)
\right]
+1
=0
\]
\[
\rho^2 \left[
1 + 2\alpha^2 \left(  \sin^2 \frac{\beta h}{2})  +  \sin^2  \frac{ \gamma h}{2} \right)
\right]
+\rho
\left[
-2 + 2\alpha^2 \left( \sin^2 \frac{\beta h}{2}) +  \sin ^2 \frac{\gamma h}{2} \right)
\right]
+1
=0
\]


I do not understand the rest of the notes!






