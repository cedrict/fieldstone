\begin{flushright} {\tiny {\color{gray} pair\_bernaudi\_raugel.tex}} \end{flushright}
%~~~~~~~~~~~~~~~~~~~~~~~~~~~~~~~~~~~~~~~~~~~~~~~~~~~~~~~~~~~~~~~~~~~~~~~~~~~~~~~~~~~~~~~~~~~~~~~~~~

In \textcite{cakp15} (2015) we find: ``The BR-FEM after Bernardi and Raugel \cite{bera85} 
is a modification of the $P_2\times P_0$ FEM. It is sometimes also called reduced $P_2\times P_0$ FEM''.
They also state that this element also exists in 3D.

\begin{center}
\includegraphics[width=5cm]{images/pair_bernardi_raugel/cakp15}
\end{center}

It is also mentioned in \textcite{bobf13} although it seems it is there called the SMALL element (p474).

In Lederer: "Consider the case d = 2. [...] we only need to control 
the normal velocity at the edge, i.e. adding the
edge bubble for both components of the velocity seems to be sub optimal (with respect to
computational costs and the expected approximation properties). The idea now is to only
add the normal edge bubble."

According to \textcite{jolm17} (2017) (example 6.3), `` the velocity space in the Bernardi-Raugel
element consists of $P_1$ functions which are enriched with edge bubble functions''.
The authors also speak of 'reconstructing the test functions' and state: 
``the results of the method with reconstruction are generally more accurate.
In summary, the use of an appropriately reconstructed test function in the Bernardi–
Raugel pair of spaces led to a clear improvement of the accuracy of the computed
results compared with the standard method.''


