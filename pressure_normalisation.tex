
%..................................................
\subsubsection{Basic idea and naive implementation}

When Dirichlet boundary conditions are imposed everywhere on the boundary, 
pressure is only present by its gradient in 
the equations. It is thus determined up to an arbitrary constant (one speaks then 
of a nullspace of size 1).  \index{general}{Nullspace}
In such a case, one commonly impose the average of the pressure over the whole domain or on 
a subset of the boundary 
to have a zero average, i.e.
\begin{equation}
\int_\Omega p dV = 0
\end{equation}
Another possibility is to impose the pressure value at a single node. 

Let us assume for example that we are using $Q_1 \times P_0$ elements. Then the pressure is constant 
inside each element. 
The integral above becomes:
\begin{equation}
\int_\Omega p dV = 
\sum_e  \int_{\Omega_e} p dV = 
\sum_e  p_e \int_{\Omega_e} dV = 
\sum_e  p_e A_e = 0
\end{equation}
where the sum runs over all elements $e$ of area $A_e$.
This can be rewritten 
\[
\vec{L} \cdot \vec{\cal P}=0
\] 
and it is a constraint on the pressure solution which couples {\it all} pressure dofs. 
We can associate to it a Lagrange multiplier $\lambda$ so that we must solve the modified Stokes system:
\[
\left(
\begin{array}{ccc}
\K & \G & 0\\ 
\G^T & 0 & \vec{L}^T \\
0 & \vec{L} & 0
\end{array}
\right)
\cdot
\left(
\begin{array}{c}
\vec{\cal V} \\ \vec{\cal P} \\ \lambda
\end{array}
\right)
=
\left(
\begin{array}{c}
\vec{f} \\ \vec{h} \\ 0
\end{array}
\right)
\]
When higher order spaces are used for pressure (continuous or discontinuous)
one must then carry out the above integration numerically by means of (usually)
a Gauss-Legendre quadrature.

Although valid, this approach has one main disadvantage: it makes the Stokes matrix larger (although
marginally so -- only one row and column are added), but more importantly it prevents the use of some
of the solving strategies of Section \ref{sec:solvers}.


%..................................................
\subsubsection{Implementation -- the real deal}

Here is what Bochev and Gunzburger \cite[Section 7.6.4]{bogu09} have to say about this:
"[...] practical implementations cheat by substituting enforcement of the true zero mean constraint by using
procedures collectively known as setting the pressure datum. These procedures essentially 
amount to removing one degree of freedom from the pressure space.
Setting the pressure datum can be accomplished in many different ways, ranging
from specifying a pressure value at a grid node to more complicated approaches in
which a boundary traction is specified at a single node in lieu of the velocity condition; 
see [16, 24, 191] and the references cited therein for more details. Not surprisingly, 
in practice, the simplest approach of fixing the pressure value at a node also
happens to be the most widely used in practice."




The idea is actually quite simple and requires two steps:
\begin{enumerate}
\item remove the null space by prescribing the pressure at one location and solve the system;
\item post-process the pressure so as to arrive at a pressure field which fulfils the required normalisation (surface, volume, ...)
\end{enumerate}

The reason why it works is as follows: a constant pressure value lies in the null space, so that one can 
add or delete any value to the pressure field without consequence. As such I can choose said constant such that 
the pressure at a given node/element is zero. All other computed pressures are then relative to that one. 
The post-processing step will redistribute a constant value to all pressures (it will shift them up or down)
so that the normalising condition is respected. 


\Literature

\url{https://scicomp.stackexchange.com/questions/27645/pressure-boundary-condition-in-lid-driven-cavity-using-finite-element-method}

\url{
https://scicomp.stackexchange.com/questions/25134/mixed-finite-element-method-for-the-stokes-system-some-implementation-details}





