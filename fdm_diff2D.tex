We now revisit the transient heat equation, this time with sources/sinks for 2D problems.
In the absence of advective heat transport, the heat equation is 
\begin{equation}
\rho C_p \frac{\partial T}{\partial t} =
\vec\nabla \cdot k \vec\nabla T + Q 
\end{equation}
where $Q$ is the radiogenic heat production.
It simply writes as follows when Cartesian coordinates are used:
\begin{equation}
\rho C_p \frac{\partial T}{\partial t} = 
\frac{\partial }{\partial x} \left(  k  \frac{\partial T}{\partial x} \right)+
\frac{\partial }{\partial y} \left(  k  \frac{\partial T}{\partial y} \right)+ Q
\end{equation}
If the heat conductivity is constant in space (and so are the other coefficients), 
it writes:
\begin{equation}
\frac{\partial T}{\partial t} =
\kappa \left(  \frac{\partial^2 T}{\partial x^2} + \frac{\partial^2 T}{\partial y^2} \right)+
\tilde{Q}
\end{equation}
with $\tilde{Q}=Q/\rho C_p$.
In order to solve this equation over the Cartesian domain of size $L_x \times L_y$
we need to generate a mesh as shown hereunder:

%-/-/-/-/-/-/-/-/-/-/-/-/-/-/-/
\begin{minipage}[t]{\textwidth}
\begin{center}
\input{tikz/tikz_fdm5x4mesh1}
\end{center}
\end{minipage}
%-/-/-/-/-/-/-/-/-/-/-/-/-/-/-/

The spacing between the nodes in the $x$-direction is $h_x$ and $h_y$ is the spacing
between the nodes in the $y$ direction. There are now $nnp=nnx\times nny$ nodes in total.
The above grid is characterised by $i=0,1,2,3,4$ and $j=0,1,2,3$ and counts in total 
20 nodes.

In one dimension, the subscript indicated the node $i$. In two dimensions we therefore 
need two indices ${\color{brown}i}$ and ${\color{brown}j}$ 
to identify a node, so that the temperature at node ${\color{brown}i},{\color{brown}j}$ 
at time $n$ is denoted $T_{{\color{brown}i,j}}^n$.

%The vector $\vec{T}$ contains all the temperature unknowns, so it is a vector that is $np$-long. 
%But how should this vector be organised ? In other words, 
One question remains: should we number nodes 
row by row ? column by column ? randomly ? 
These three approaches are shown hereunder: 

\vspace{.5cm}

%-/-/-/-/-/-/-/-/-/-/-/-/-/-/-/
\begin{minipage}[t]{\textwidth}
\input{tikz/tikz_fdm5x4meshes}\\
\end{minipage}
%-/-/-/-/-/-/-/-/-/-/-/-/-/-/-/

\vspace{.5cm}

This is a critical point because the discretised PDE is formulated as a 
function of $T_{{\color{brown} i,j}}$ 
with ${\color{brown}i}=0,\dots nnx-1$ and ${\color{brown}j}=0,\dots nny-1$ 
but the vector $\vec{T}$ containing all these values (encountered in 
implicit methods)
is indexed by a single index ${\color{teal}k}=0,\dots nnp-1$. The numbering strategy determines how easy
it is to go from $({\color{brown}i},{\color{brown}j})$ to ${\color{teal}k}$ and vice versa. 
Very concretely again, where should $T_{\color{brown}3,4}$ be placed in the global 
vector of unknowns $\vec{T}$?

At the same time we cannot do away with ${\color{brown}i,j}$ indices because these are 
needed to locate the direct neighbours of any node and allow to 
form discrete derivatives. 

We then need a (preferably simple/straightforward) 'function' 
which associates to every $({\color{brown} i,j})$ a global index $k$. 
For the first grid with row-wise numbering, we have 
$0\leq {\color{brown}i} \leq 4$ , $0 \leq {\color{brown}j} \leq 3$ 
and $0 \leq {\color{teal}k} \leq 19$. It follows that 
\begin{equation}
{\color{teal} k}({\color{brown}i,j})={\color{brown}j} \cdot nnx+{\color{brown}i}
\end{equation}
This is easy to verify: ${\color{brown}i}=3$ and ${\color{brown}j}=2$ 
indeed corresponds to node \# 13, 
${\color{brown}i}=4$ and ${\color{brown}j}=1$ corresponds to node \# 9, etc ...

%-/-/-/-/-/-/-/-/-/-/-/-/-/-/-/
\begin{minipage}[t]{\textwidth}
\begin{center}
\input{tikz/tikz_fdm5x4mesh2}
\end{center}
\end{minipage}
%-/-/-/-/-/-/-/-/-/-/-/-/-/-/-/


%-/-/-/-/-/-/-/-/-/-/-/-/-/-/-/-/-/-/
\begin{center}
\begin{minipage}[t]{0.77\textwidth}
\par\noindent\rule{\textwidth}{0.4pt}

\begin{center}
\includegraphics[width=0.8cm]{images/garftr} \\
{\color{orange}Exercise FDM-9}
\end{center}

In a new code declare and assign values to 
$nnx$ and $nny$. Compute $nnp$.
Set $L_x=7$ and $L_y=6$. Compute $h_x$ and $h_y$.

Declare two arrays $xcoords$ and $ycoords$ which will 
contain the $x$ and $y$ coordinates of all $nnp$ nodes.

By means of two nested for loops
compute these coordinates \& fill both arrays. 

Visualise the nodes with matplotlib.

Tip: Make sure your code works for various 
combinations of $nnx$ and $nny$.

\par\noindent\rule{\textwidth}{0.4pt}
\end{minipage}
\end{center}






%.............................
\subsection{Explicit scheme} The simplest approach is an {\color{olive} FTCS} 
(forward time, centered space) explicit method like in 1D:
\begin{equation}
\frac{T_{{\color{brown}i,j}}^{n+1}-T_{{\color{brown}i,j}}^n}{\delta t}
= \kappa
\left(
\frac{ T_{{\color{brown}i-1,j}}^{n}-2T_{{\color{brown}i,j}}^{n}+T_{{\color{brown}i+1,j}}^{n}  }{h_x^2} + 
\frac{ T_{{\color{brown}i,j-1}}^{n}-2T_{{\color{brown}i,j}}^{n}+T_{{\color{brown}i,j+1}}^{n}  }{h_y^2}
\right)
+\tilde{Q}_{{\color{brown}i,j}}^n
\end{equation}
where we have assumed that the source term $\tilde{Q}$ 
can depend of space coordinates and therefore 
appears as $\tilde{Q}_{{\color{brown}i,j}}$ in the equation.
We define $s_x$ and $s_y$ as follows:
\begin{equation}
s_x = \frac{\kappa \delta t}{h_x^2}
\quad\quad
s_y = \frac{\kappa \delta t}{h_y^2}
\end{equation}
so that
\begin{equation}
T_{{\color{brown}i,j}}^{n+1} = T_{{\color{brown}i,j}}^n 
+ s_x ( T_{{\color{brown}i-1,j}}^{n}
-2T_{{\color{brown}i,j}}^{n}
+T_{{\color{brown}i+1,j}}^{n} ) 
+s_y ( T_{{\color{brown}i,j-1}}^{n}
-2T_{{\color{brown}i,j}}^{n}
+T_{{\color{brown}i,j+1}}^{n} ) + 
\tilde{Q}_{{\color{brown}i,j}}^n \delta t
\end{equation}
or, 
\begin{equation}
T_{{\color{teal}k}({\color{brown}i,j})}^{n+1} = 
T_{{\color{teal}k}({\color{brown}i,j})}^n 
+ s_x ( T_{{\color{teal}k}({\color{brown}i-1,j})}^{n}
-2T_{{\color{teal}k}({\color{brown}i,j})}^{n}
+T_{{\color{teal}k}({\color{brown}i+1,j})}^{n} ) 
+s_y ( T_{{\color{teal}k}({\color{brown}i,j-1})}^{n}
-2T_{{\color{teal}k}({\color{brown}i,j})}^{n}
+T_{{\color{teal}k}({\color{brown}i,j+1})}^{n} ) + 
\tilde{Q}_{{\color{teal}k}({\color{brown}i,j})}^n \delta t
\end{equation}



The scheme is stable for  
\begin{equation}
\delta t \leq \frac{\min(h_x^2,h_y^2)}{2 \kappa}
\end{equation}
Boundary conditions can be set the usual way: for example a constant (Dirichlet) temperature 
at node $({\color{brown}i},{\color{brown}j})$ is given by
\begin{equation}
T_{{\color{brown}i},{\color{brown}j}}=T_{bc} 
\end{equation}
where $T_{bc}$ is the prescribed temperature. 

%-/-/-/-/-/-/-/-/-/-/-/-/-/-/-/-/-/-/-/
\begin{center}
\begin{minipage}[t]{0.77\textwidth}
\par\noindent\rule{\textwidth}{0.4pt}
\begin{center}
\includegraphics[width=0.8cm]{images/garftr} \\
{\color{orange}Exercise FDM-10}
\end{center}

A simple (time-dependent) analytical solution for the temperature equation exists for 
the case that the initial temperature field is
\begin{equation}
T(x,y,t=0) = T_0+ T_{max} \exp \left[ -\frac{x^2+y^2}{\sigma^2}   \right]
\end{equation}
where $T_{max}$ is the maximum amplitude of the temperature perturbation 
at $(x,y) = (0, 0)$ and $\sigma$ its half-width. 

\begin{center}
\includegraphics[width=5cm]{images/fdm/gaussian}\\
{\captionfont initial temperature field}
\end{center}

The solution of the time-dependent PDE is
\begin{equation}
T(x,y,t)=T_0 + \frac{T_{max}}{1+4t\kappa/\sigma^2 } \exp \left[ -\frac{x^2+y^2}{\sigma^2 + 4t\kappa}   \right]
\end{equation}

Set $L_x$=100km and $L_y=80$km, $\kappa=10^{-6}$, $\tilde{Q}=0$, $T_{max}=100\degree$, $T_0=200\degree$, 
and $\sigma=10^4$m. 

Use the previous exercise to generate a $nnx\times nny$ grid 
in the $[-L_x/2,L_x/2]\times[-L_y/2,L_y/2]$ domain.

Write a function which takes $x$, $y$, $t$, $T_0$, $T_{max}$, $\kappa$ and $\sigma$ as argument 
and returns the analytical temperature value.

Write a an explicit FDM code which solves the 2D diffusion equation. At each time step 
prescribe on the boundary the analytical solution.  

Plot the error field
(i.e. the obtained solution minus the analytical one)
in the domain at a few times.

\par\noindent\rule{\textwidth}{0.4pt}
\end{minipage}
\end{center}
%-/-/-/-/-/-/-/-/-/-/-/-/-/-/-/-/-/-/-/



%------------------------------------------------------------------------------
\subsection{Implicit scheme} 
If we now employ a fully implicit, unconditionally stable discretization 
scheme, the discretised 
PDE becomes:
\begin{equation}
\frac{T_{{\color{brown}i},{\color{brown}j}}^{n+1}-T_{{\color{brown}i,j}}^n}{\delta t}
= \kappa
\left(
\frac{ T_{{\color{brown}i-1,j}}^{n+1}-2T_{{\color{brown}i,j}}^{n+1}+T_{{\color{brown}i+1,j}}^{n+1} }{h_x^2} + 
\frac{ T_{{\color{brown}i,j-1}}^{n+1}-2T_{{\color{brown}i,j}}^{n+1}+T_{{\color{brown}i,j+1}}^{n+1} }{h_y^2}
\right)
+\frac{Q_{{\color{brown}i,j}}^n}{\rho C_p}
\end{equation}
Rearranging terms with $n+1$ on the left and terms with $n$ on the right 
hand side gives
\begin{equation}
-s_x\; T_{{\color{brown}i+1,j}}^{n+1}
-s_y\; T_{{\color{brown}i,j+1}}^{n+1} 
+(1+2s_x+2s_y)\; T_{{\color{brown}i,j}}^{n+1} 
-s_x\;  T_{{\color{brown}i-1,j}}^{n+1} 
-s_y\;  T_{{\color{brown}i,j-1}}^{n+1} 
=
T_{{\color{brown}i,j}}^n
+\tilde{Q}_{{\color{brown}i,j}}^n \delta t
\end{equation}
or
\begin{equation}
-s_x\;           T_{{\color{teal}k}({\color{brown}i+1,j})}^{n+1}
-s_y\;           T_{{\color{teal}k}({\color{brown}i,j+1})}^{n+1} 
+(1+2s_x+2s_y)\; T_{{\color{teal}k}({\color{brown}i,j}  )}^{n+1} 
-s_x\;           T_{{\color{teal}k}({\color{brown}i-1,j})}^{n+1} 
-s_y\;           T_{{\color{teal}k}({\color{brown}i,j-1})}^{n+1} 
=
T_{{\color{teal}k}({\color{brown}i,j})}^n
+\tilde{Q}_{{\color{teal}k}({\color{brown}i,j})}^n \delta t
\end{equation}
which here again yields a linear system of equations written 
${\bm A}\cdot {\vec T} = {\vec b}$
where ${\bm A}$ is a $(nnp \times nnp)$ matrix.

%-/-/-/-/-/-/-/-/-/-/-/-/-/-/-/
\begin{minipage}[t]{\textwidth}
\begin{center}
\input{tikz/tikz_fdm5x4mesh2}
\end{center}
\end{minipage}
%-/-/-/-/-/-/-/-/-/-/-/-/-/-/-/

For simplicity boundary conditions are $T(x,y)=T_{bc}$ on all sides, so all nodes 
on the boundary have a prescribed $T_{bc}$ temperature\footnote{We assume
here again that these boundary conditions do not change with time.}:
\begin{eqnarray}
T_{\color{brown}0,0} = T_{\color{teal} 0}  &=& T_{bc} \nn\\
T_{\color{brown}1,0} = T_{\color{teal} 1}  &=& T_{bc} \nn\\
T_{\color{brown}2,0} = T_{\color{teal} 2}  &=& T_{bc} \nn\\
T_{\color{brown}3,0} = T_{\color{teal} 3}  &=& T_{bc} \nn\\
T_{\color{brown}4,0} = T_{\color{teal} 4}  &=& T_{bc} \nn\\
T_{\color{brown}0,1} = T_{\color{teal} 5}  &=& T_{bc} \nn\\
T_{\color{brown}4,1} = T_{\color{teal} 9}  &=& T_{bc} \nn\\
T_{\color{brown}0,2} = T_{\color{teal} 10} &=& T_{bc} \nn\\
T_{\color{brown}4,2} = T_{\color{teal} 14} &=& T_{bc} \nn\\
T_{\color{brown}0,3} = T_{\color{teal} 15} &=& T_{bc} \nn\\
T_{\color{brown}1,3} = T_{\color{teal} 16} &=& T_{bc} \nn\\
T_{\color{brown}2,3} = T_{\color{teal} 17} &=& T_{bc} \nn\\
T_{\color{brown}3,3} = T_{\color{teal} 18} &=& T_{bc} \nn\\
T_{\color{brown}4,3} = T_{\color{teal} 19} &=& T_{bc} \nn
\end{eqnarray}
In what follows we assume for simplicity and conciseness of notation that 
$h_x=h_y=h$ so that $s_x=s_y=s$.
The discretised PDE equation will now be applied to the interior nodes:
%\begin{eqnarray}
%-s T_{{\color{brown} i+1,j}}^{n+1}
%-s T_{{\color{brown}i,j+1}}^{n+1} 
%+(1+4s)T_{{\color{brown}i,j}}^{n+1} 
%-s T_{{\color{brown}i-1,j}}^{n+1} 
%-s T_{{\color{brown}i,j-1}}^{n+1} 
%= T_{{\color{brown}i,j}}^n 
%+\tilde{Q}_{{\color{brown}i,j}}^n
%\end{eqnarray}

\begin{itemize}
\item For node ${\color{teal}k}=6$ (${\color{brown}i}=1,{\color{brown}j}=1$):
\begin{eqnarray}
-s T_{{\color{brown}2,1}}^{n+1}
-s T_{{\color{brown}1,2}}^{n+1} 
+(1+4s)T_{{\color{brown}1,1}}^{n+1} 
-s T_{{\color{brown}0,1}}^{n+1} 
-s T_{{\color{brown}1,0}}^{n+1} 
&=& T_{{\color{brown}1,1}}^n +\tilde{Q}_{{\color{brown}1,1}}^n \delta t\nn\\
\Rightarrow \qquad
-s T_{{\color{teal} 7}}^{n+1}-s T_{{\color{teal} 11}}^{n+1} +(1+4s)T_{{\color{teal} 6}}^{n+1} 
-s T_{{\color{teal} 5}}^{n+1} -s T_{{\color{teal} 1}}^{n+1} 
&=& T_{{\color{teal} 6}}^n +\tilde{Q}_{{\color{teal} 6}}^n\delta t
\end{eqnarray}

\item For node ${\color{teal}k}=7$ (${\color{brown}i}=2,{\color{brown}j}=1$):
\begin{eqnarray}
-s T_{{\color{brown} 3,1}}^{n+1}
-s T_{{\color{brown}2,2}}^{n+1} 
+(1+4s)T_{{\color{brown}2,1}}^{n+1} 
-s T_{{\color{brown}1,1}}^{n+1} 
-s T_{{\color{brown}2,0}}^{n+1} 
&=& T_{{\color{brown}2,1}}^n 
+\tilde{Q}_{{\color{brown}2,1}}^n \delta t\nn\\
\Rightarrow \qquad
-s T_{{\color{teal} 8}}^{n+1}
-s T_{{\color{teal} 12}}^{n+1} 
+(1+4s)T_{{\color{teal}7}}^{n+1} 
-s T_{{\color{teal}6}}^{n+1} 
-s T_{{\color{teal}2}}^{n+1} 
&=& T_{{\color{teal}7}}^n 
+\tilde{Q}_{{\color{teal}7}}^n \delta t
\end{eqnarray}

\item For node ${\color{teal}k}=8$ (${\color{brown}i}=3,{\color{brown}j}=1$):
\begin{eqnarray}
-s T_{{\color{brown} 4,1}}^{n+1}
-s T_{{\color{brown}3,2}}^{n+1} 
+(1+4s)T_{{\color{brown}3,1}}^{n+1} 
-s T_{{\color{brown}2,1}}^{n+1} 
-s T_{{\color{brown}3,0}}^{n+1} 
&=& T_{{\color{brown}3,1}}^n 
+\tilde{Q}_{{\color{brown}3,1}}^n \delta t\nn\\
\Rightarrow \qquad
-s T_{{\color{teal} 9}}^{n+1}
-s T_{{\color{teal} 13}}^{n+1} 
+(1+4s)T_{{\color{teal}8}}^{n+1} 
-s T_{{\color{teal}7}}^{n+1} 
-s T_{{\color{teal}3}}^{n+1} 
&=& T_{{\color{teal}8}}^n 
+\tilde{Q}_{{\color{teal}8}}^n \delta t
\end{eqnarray}

\item For node ${\color{teal}k}=11$ (${\color{brown}i}=1,{\color{brown}j}=2$):
\begin{eqnarray}
-s T_{{\color{brown} 2,2}}^{n+1}
-s T_{{\color{brown}1,3}}^{n+1} 
+(1+4s)T_{{\color{brown}1,2}}^{n+1} 
-s T_{{\color{brown}0,2}}^{n+1} 
-s T_{{\color{brown}2,1}}^{n+1} 
&=& T_{{\color{brown}1,2}}^n 
+\tilde{Q}_{{\color{brown}1,2}}^n \delta t\nn\\
\Rightarrow \qquad
-s T_{{\color{teal} 12}}^{n+1}
-s T_{{\color{teal} 16}}^{n+1} 
+(1+4s)T_{{\color{teal}11}}^{n+1} 
-s T_{{\color{teal}10}}^{n+1} 
-s T_{{\color{teal}6}}^{n+1} 
&=& T_{{\color{teal}11}}^n 
+\tilde{Q}_{{\color{teal}11}}^n\delta t
\end{eqnarray}

\item For node ${\color{teal}k}=12$ (${\color{brown}i}=2,{\color{brown}j}=2$):
\begin{eqnarray}
-s T_{{\color{brown} 3,2}}^{n+1}
-s T_{{\color{brown}2,3}}^{n+1} 
+(1+4s)T_{{\color{brown}2,2}}^{n+1} 
-s T_{{\color{brown}1,2}}^{n+1} 
-s T_{{\color{brown}2,1}}^{n+1} 
&=& T_{{\color{brown}2,2}}^n 
+\tilde{Q}_{{\color{brown}2,2}}^n \delta t \nn\\
\Rightarrow \qquad
-s T_{{\color{teal} 13}}^{n+1}
-s T_{{\color{teal} 17}}^{n+1} 
+(1+4s)T_{{\color{teal}12}}^{n+1} 
-s T_{{\color{teal}11}}^{n+1} 
-s T_{{\color{teal}7}}^{n+1} 
&=& T_{{\color{teal}12}}^n 
+\tilde{Q}_{{\color{teal}12}}^n\delta t
\end{eqnarray}


\item For node ${\color{teal}k}=13$ (${\color{brown}i}=3,{\color{brown}j}=2$):
\begin{eqnarray}
-s T_{{\color{brown} 4,2}}^{n+1}
-s T_{{\color{brown}3,3}}^{n+1} 
+(1+4s)T_{{\color{brown}3,2}}^{n+1} 
-s T_{{\color{brown}2,2}}^{n+1} 
-s T_{{\color{brown}3,1}}^{n+1} 
&=& T_{{\color{brown}3,2}}^n 
+\tilde{Q}_{{\color{brown}3,2}}^n \delta t\nn\\
\Rightarrow \qquad
-s T_{{\color{teal} 14}}^{n+1}
-s T_{{\color{teal} 18}}^{n+1} 
+(1+4s)T_{{\color{teal}13}}^{n+1} 
-s T_{{\color{teal}12}}^{n+1} 
-s T_{{\color{teal}8}}^{n+1} 
&=& T_{{\color{teal}13}}^n 
+\tilde{Q}_{{\color{teal}13}}^n \delta t
\end{eqnarray}


\end{itemize}

Putting it all together yields the following linear system:

\begin{landscape}
\[
\underbrace{
\left(
\begin{array}{cccccccccccccccccccc}
1 & . & . & . & . & . & . & . & . & . & . & . & . & . & . & . & . & . & . & . \\ %#0
. & 1 & . & . & . & . & . & . & . & . & . & . & . & . & . & . & . & . & . & . \\ %#1
. & . & 1 & . & . & . & . & . & . & . & . & . & . & . & . & . & . & . & . & . \\ %#2
. & . & . & 1 & . & . & . & . & . & . & . & . & . & . & . & . & . & . & . & . \\ %#3
. & . & . & . & 1 & . & . & . & . & . & . & . & . & . & . & . & . & . & . & . \\ %#4
. & . & . & . & . & 1 & . & . & . & . & . & . & . & . & . & . & . & . & . & . \\ %#5
. & -s& . & . & . & -s& {1+4s} & -s& . & . & . & -s & . & . & . & . & . & . & . & . \\ %#6
. & . & -s& . & . & . & -s& {1+4s} & -s& . & . & . & -s & . & . & . & . & . & . & .\\ %#7
. & . & . & -s& . & . & . & -s & {1+4s} & -s & . & . & . & -s & . & . & . & . & . & . \\ %#8
. & . & . & . & . & . & . & . & . & 1 & . & . & . & . & . & . & . & . & . & . \\ %#9
. & . & . & . & . & . & . & . & . & . & 1 & . & . & . & . & . & . & . & . & . \\ %#10
. & . & . & . & . & . &-s & . & . & . & -s& {1+4s} & -s& . & . & .  & -s& . & . & .\\ %#11
. & . & . & . & . & . & . &-s & . & . & . & -s& {1+4s} & -s& . & . & .  & -s& . & .\\ %#12
. & . & . & . & . & . & . & . &-s & . & . & . & -s& {1+4s} & -s& . & . & .  & -s & .\\ %#13
. & . & . & . & . & . & . & . & . & . & . & . & . & . & 1 & . & . & . & . & . \\ %#14
. & . & . & . & . & . & . & . & . & . & . & . & . & . & . & 1 & . & . & . & . \\ %#15
. & . & . & . & . & . & . & . & . & . & . & . & . & . & . & . & 1 & . & . & . \\ %#16
. & . & . & . & . & . & . & . & . & . & . & . & . & . & . & . & . & 1 & . & . \\ %#17
. & . & . & . & . & . & . & . & . & . & . & . & . & . & . & . & . & . & 1 & . \\ %#18
. & . & . & . & . & . & . & . & . & . & . & . & . & . & . & . & . & . & . & 1    %#19
\end{array}
\right)
}_{\bm A}
\cdot
\underbrace{
\left(
\begin{array}{c}
T_{{\color{teal}0}}^{n+1} \\ 
T_{{\color{teal}1}}^{n+1} \\ 
T_{{\color{teal}2}}^{n+1} \\ 
T_{{\color{teal}3}}^{n+1} \\ 
T_{{\color{teal}4}}^{n+1} \\ 
T_{{\color{teal}5}}^{n+1} \\ 
T_{{\color{teal}6}}^{n+1} \\ 
T_{{\color{teal}7}}^{n+1} \\ 
T_{{\color{teal}8}}^{n+1} \\ 
T_{{\color{teal}9}}^{n+1} \\ 
T_{{\color{teal}10}}^{n+1} \\ 
T_{{\color{teal}11}}^{n+1} \\ 
T_{{\color{teal}12}}^{n+1} \\ 
T_{{\color{teal}13}}^{n+1} \\ 
T_{{\color{teal}14}}^{n+1} \\ 
T_{{\color{teal}15}}^{n+1} \\ 
T_{{\color{teal}16}}^{n+1} \\ 
T_{{\color{teal}17}}^{n+1} \\ 
T_{{\color{teal}18}}^{n+1} \\ 
T_{{\color{teal}19}}^{n+1} 
\end{array}
\right)
}_{\vec T}
=
\underbrace{
\left(
\begin{array}{c}
T_{bc} \\
T_{bc} \\
T_{bc} \\
T_{bc} \\
T_{bc} \\
T_{bc} \\
T_{{\color{teal}6}}^n + \tilde{Q}_{\color{teal}6} \delta t\\ 
T_{{\color{teal}7}}^n + \tilde{Q}_{\color{teal}7} \delta t\\ 
T_{{\color{teal}8}}^n + \tilde{Q}_{\color{teal}8} \delta t\\ 
T_{bc} \\
T_{bc} \\
T_{{\color{teal}11}}^n + \tilde{Q}_{\color{teal}11} \delta t\\ 
T_{{\color{teal}12}}^n + \tilde{Q}_{\color{teal}12} \delta t\\ 
T_{{\color{teal}13}}^n + \tilde{Q}_{\color{teal}13} \delta t\\ 
T_{bc} \\
T_{bc} \\
T_{bc} \\
T_{bc} \\
T_{bc} \\
T_{bc} 
\end{array}
\right)
}_{\vec b}
\]
Note that we now have five 'diagonals' filled with non-zero entries as opposed to three
diagonals in the 1D case.

Note again that this is a simplified matrix since we assumed that $s_x=s_y$ (see expressions
at the beginning of this section).
\end{landscape}

Let us now focus on a special case: we wish to solve the steady state diffusion equation. 
The approach is then very similar, but we must discard the time derivative term so that 
the discretised PDE becomes:
\begin{equation}
0
= \kappa
\left(
\frac{ T_{{\color{brown}i-1,j}}-2T_{{\color{brown}i,j}}+T_{{\color{brown}i+1,j}} }{h_x^2} + 
\frac{ T_{{\color{brown}i,j-1}}-2T_{{\color{brown}i,j}}+T_{{\color{brown}i,j+1}} }{h_y^2}
\right)
+\frac{Q_{{\color{brown}i,j}}}{\rho C_p}
\end{equation}
Note that we have removed the superscript $n$ and $n+1$ since there is no time discretisation.

Rearranging terms, and assuming again that $h_x=h_y$ for simplicity\footnote{We no longer use the $s_x$
and $s_y$ parameters since those contain the term $\delta t$ which is here meaningless}:
\begin{equation}
- T_{{\color{brown}i+1,j}}
- T_{{\color{brown}i,j+1}}
+4 \; T_{{\color{brown}i,j}}
-  T_{{\color{brown}i-1,j}}
-  T_{{\color{brown}i,j-1}}
=
\frac{ \tilde{Q}_{{\color{brown}i,j}} h^2} {\kappa }
\end{equation}

We would then follow the regular procedure of writing this relationship at nodes that are not on the boundary.
In the end we would obtain a matrix with an identical structure as before, but with slightly different terms
(esp. on the diagonal). 

A final remark: all this is valid for a a constant heat conductivity (which allowed us to take it out of 
the gradient terms in the original PDE and form the heat diffusivity term $\kappa$ - also constant in space).
If not the expression presented in Eq.~\eqref{eq:fdm_discterms} must be used. 




















%-/-/-/-/-/-/-/-/-/-/-/-/-/-/-/-/-/-/-/-/-/-/
\begin{center}
\begin{minipage}[t]{0.77\textwidth}
\par\noindent\rule{\textwidth}{0.4pt}
\begin{center}
\includegraphics[width=0.8cm]{images/garftr} \\
{\color{orange}Exercise FDM-11}
\end{center}

Same exercise as exercise FDM-10, but now with implicit method.

\par\noindent\rule{\textwidth}{0.4pt}
\end{minipage}
\end{center}
%-/-/-/-/-/-/-/-/-/-/-/-/-/-/-/-/-/-/-/-/-/-/


Looking at this matrix, it is clear that this approach is sub-optimal: for such a small grid counting
20 nodes, the boundary conditions enforce the temperature on 14 of them, so that these
temperatures should/could be removed from the list of unknowns, leaving a vector 
of unknowns $\vec{T}$ of size 6 (the number of nodes which are not on the boundary).
As a consequence, we would have to solve a $6\times 6$ linear system, as opposed to a $20\times 20$ one!

In this case, we focus again on nodes 6,7,8,11,12,13.
we start from 
\begin{equation}
-s T_{{\color{teal} 7}}^{n+1}
-s T_{{\color{teal} 11}}^{n+1} 
+(1+4s)T_{{\color{teal} 6}}^{n+1} 
-s T_{{\color{teal} 5}}^{n+1} 
-s T_{{\color{teal} 1}}^{n+1} 
= T_{{\color{teal} 6}}^n 
+\tilde{Q}_{{\color{teal} 6}}^n \delta t
\end{equation}
but we know that the boundary conditions impose that $T_{\color{teal}1}=0$ 
and $T_{\color{teal}5}=0$ so that the equation above simplifies to:
\begin{equation}
-s T_{{\color{teal} 7}}^{n+1}
-s T_{{\color{teal} 11}}^{n+1} 
+(1+4s)T_{{\color{teal} 6}}^{n+1} 
= T_{{\color{teal} 6}}^n 
+\tilde{Q}_{{\color{teal} 6}}^n\delta t
\end{equation}

These 6 equations can finally be combined in the expected smaller linear system:
\begin{equation}
\underbrace{
\left(
\begin{array}{cccccc}
1+4s & -s & . & -s & . & . \\
-s & 1+4s & -s & . & -s & . \\
. & -s & 1+4s & . & . & -s \\ 
-s & . & -s & 1+4s & -s & . \\
. & -s & . & -s & 1+4s & -s \\
. & . & -s & . & -s & 1+4s 
\end{array}
\right)
}_{\bm A}
\cdot
\underbrace{
\left(
\begin{array}{c}
T_{{\color{teal}6}}^{n+1} \\ 
T_{{\color{teal}7}}^{n+1} \\ 
T_{{\color{teal}8}}^{n+1} \\ 
T_{{\color{teal}11}}^{n+1} \\ 
T_{{\color{teal}12}}^{n+1} \\ 
T_{{\color{teal}13}}^{n+1} 
\end{array}
\right)
}_{\vec T}
=
\underbrace{
\left(
\begin{array}{c}
T_{{\color{teal}6}}^n + \tilde{Q}_{\color{teal}6}\delta t \\ 
T_{{\color{teal}7}}^n + \tilde{Q}_{\color{teal}7} \delta t\\ 
T_{{\color{teal}8}}^n + \tilde{Q}_{\color{teal}8} \delta t\\ 
T_{{\color{teal}11}}^n + \tilde{Q}_{\color{teal}11} \delta t\\ 
T_{{\color{teal}12}}^n + \tilde{Q}_{\color{teal}12} \delta t\\ 
T_{{\color{teal}13}}^n + \tilde{Q}_{\color{teal}13} \delta t
\end{array}
\right)
}_{\vec b}
\end{equation}
Note that is the boundary values had not been zero they would have found their way to the right hand side 
vector.

The Crank-Nicolson version of the implicit scheme is then as follows:
\begin{eqnarray}
\frac{T_{{\color{brown}i},{\color{brown}j}}^{n+1}-T_{\color{brown}i,j}^n}{\delta t}
&=& \frac12 \kappa
\left(
\frac{ T_{{\color{brown}i-1,j}}^{n+1}-2T_{{\color{brown}i,j}}^{n+1}+T_{{\color{brown}i+1,j}}^{n+1} }{h_x^2} + 
\frac{ T_{{\color{brown}i,j-1}}^{n+1}-2T_{{\color{brown}i,j}}^{n+1}+T_{{\color{brown}i,j+1}}^{n+1} }{h_y^2}
\right) \nn\\
&+& \frac12 \kappa
\left(
\frac{ T_{{\color{brown}i-1,j}}^{n}-2T_{{\color{brown}i,j}}^{n}+T_{{\color{brown}i+1,j}}^{n} }{h_x^2} + 
\frac{ T_{{\color{brown}i,j-1}}^{n}-2T_{{\color{brown}i,j}}^{n}+T_{{\color{brown}i,j+1}}^{n} }{h_y^2}
\right) 
\end{eqnarray}
The implementation of this method will require from you to bring 
all the terms in $T^{n+1}$ to the left of the equal sign 
while all the terms in $T^n$ are assumed to be known and therefore find their way into
the right hand side. 

Likewise, the Lax-Friedrichs method is as follows:
\begin{equation}
\frac{T_{{\color{brown}i},{\color{brown}j}}^{n+1}-
\frac{1}{4} \left(  
T_{{\color{brown}i-1,j}}^{n}  +
T_{{\color{brown}i+1,j}}^{n}  +
T_{{\color{brown}i,j-1}}^{n}  +
T_{{\color{brown}i,j+1}}^{n}  
\right)
}{\delta t}
= \kappa
\left(
\frac{ T_{{\color{brown}i-1,j}}^{n+1}-2T_{{\color{brown}i,j}}^{n+1}+T_{{\color{brown}i+1,j}}^{n+1} }{h_x^2} + 
\frac{ T_{{\color{brown}i,j-1}}^{n+1}-2T_{{\color{brown}i,j}}^{n+1}+T_{{\color{brown}i,j+1}}^{n+1} }{h_y^2}
\right)
+\frac{Q_{{\color{brown}i,j}}^n}{\rho C_p}
\end{equation}
Rearranging terms with $n+1$ on the left and terms with $n$ on the right hand side gives

\index{general}{Lax-Friedrichs Method}
\index{general}{Crank-Nicolson Method}


%STENCILS in 2D?

%-/-/-/-/-/-/-/-/-/-/-/-/-/-/-/-/-/-/
%\begin{center}
%\begin{minipage}[t]{0.77\textwidth}
%\par\noindent\rule{\textwidth}{0.4pt}
%\begin{center}
%\includegraphics[width=0.8cm]{images/garftr} \\
%{\color{orange}Exercise 7}
%\end{center}

%\par\noindent\rule{\textwidth}{0.4pt}
%\end{minipage}
%\end{center}
%-/-/-/-/-/-/-/-/-/-/-/-/-/-/-/-/-/-/



%.........................................................
\subsection{The 9-point stencil for the Laplace operator} \label{ss:ninepointstencil}

What follows is mostly borrowed from Wikipedia\footnote{\url{https://en.wikipedia.org/wiki/Nine-point_stencil}}.

If we discretize the 2D Laplacian by using central-difference methods, we obtain the commonly used five-point stencil, represented by the following convolution kernel: 
\[
D=
\left[
\begin{array}{ccc}
0 & 1 & 0 \\
1 & -4 & 1 \\
0 & 1 & 0 
\end{array}
\right]
\]
or,
\[
D T_{i,j} = \frac{1}{h^2} (T_{i-1,j} + T_{i+1,j,} +  T_{i,j-1} + T_{i,j+1,} -4T_{i,j} )
\]
Even though it is simple to obtain and computationally lighter, the central difference kernel possess an undesired intrinsic anisotropic property, since it doesn't take into account the diagonal neighbours.

The two most commonly used isotropic nine-point stencils are displayed below, in their convolution kernel forms. They can be obtained by the following formula
\[
D = (1-\gamma) 
\left[
\begin{array}{ccc}
0 &1 &0 \\
1 &-4 &1 \\
0 &1 &0
\end{array}
\right]
+\gamma
\left[
\begin{array}{ccc}
1/2 &0 & 1/2 \\
0 & -2 & 0 \\
1/2 &0 & 1/2 
\end{array}
\right]
\]
The first one is known by Oono-Puri,and it is obtained when $\gamma=1/2$:
\[
D=
\left[
\begin{array}{ccc}
1/4 & 2/4 & 1/4 \\
2/4 &-12/4 & 2/4 \\
1/4 & 2/4 & 1/4 
\end{array}
\right]
=
\frac14
\left[
\begin{array}{ccc}
1 & 2 & 1 \\
2 &-12 & 2 \\
1 & 2 & 1 
\end{array}
\right]
\]
The second one is known by Patra-Karttunen or Mehrstellen, and it is obtained when $\gamma=1/3$:
\[
D=
\left[
\begin{array}{ccc}
1/6 & 4/6 & 1/6 \\
4/6 & -20/6 & 4/6 \\
1/6 & 4/6 & 1/6 
\end{array}
\right]
=
\frac16
\left[
\begin{array}{ccc}
1 & 4 & 1 \\
4 & -20 & 4 \\
1 & 4 & 1 
\end{array}
\right]
\]
or,
\[
\vec\nabla^2 T_{i,j} = \frac{1}{6h^2} (T_{i+1,j+1} + T_{i-1,j+1} + T_{i+1,j-1,} + T_{i-1,j-1}
+4 (T_{i+1,j}+T_{i-1,j})
+4 (T_{i,j+1}+T_{i,j-1})
-20 T_{i,j} )
\]
Both are isotropic forms of discrete Laplacian, and in the limit of small $h$, they all become equivalent. This form is the one we find in \textcite[p64]{leveque}.

%..............................................................
\subsection{Stencils for the Laplace operator close to boundaries} \label{ss:stencilLaplaceBnd}

We here assume that the domain is a rectangle discretized with $nnx\times nny$ nodes.

{\color{red} what follows may be not correct ... check section 2.4 of misk25}


\begin{itemize}

\item Point on the left boundary (i.e. $i=1$). We use a 2nd-order central finite difference 
approach in the $y$ direction, but a forward one for the $x$ direction (only 1st-order):
\[
\vec\nabla^2 u (x_i,y_i)= \frac{\partial^2 u}{\partial x^2}(x_i,y_i)+
\frac{\partial^2 u}{\partial y^2}(x_i,y_i)
= 
\frac{u_{i,j} -2u_{i+1,j} +u_{i+2,j}}{h_x^2} + \frac{u_{i,j-1} -2u_{i,j} +u_{i,j+1}}{h_y^2}
\]
If $h_x=h_y$ then
\[
\vec\nabla^2 u (x_i,y_i)= \frac{1}{h^2}
\left(
-u_{i,j} -2u_{i+1,j} +u_{i+2,j} + u_{i,j-1}  +u_{i,j+1}
\right)
\]
corresponding to the following stencil:
\[
\begin{pmatrix}
. & . &  . & . & . \\
. & . &  . & . & . \\
1 & -2 & 1 & . & . \\
. & . &  . & . & . \\
. & . &  . & . & . 
\end{pmatrix}
+
\begin{pmatrix}
. & . &  . & . & . \\
1 & . &  . & . & . \\
-2 & . &  . & . & . \\
1 & . &  . & . & . \\
. & . &  . & . & . 
\end{pmatrix}
=
\begin{pmatrix}
. & . &  . & . & . \\
1 & . &  . & . & . \\
-1 & -2 & 1 & . & . \\
1 & . &  . & . & . \\
. & . &  . & . & . 
\end{pmatrix}
\]


\item Point on the left boundary (i.e. $i=1$). We use a 2nd-order central finite difference 
approach in the $y$ direction, but a forward one for the $x$ direction (also 2nd-order):
\[
\vec\nabla^2 u (x_i,y_i)= \frac{\partial^2 u}{\partial x^2}(x_i,y_i)+
\frac{\partial^2 u}{\partial y^2}(x_i,y_i)
= 
\frac{2u_{i,j} -5u_{i+1,j} +4u_{i+2,j}-u_{i+3,j}}{h_x^2} + \frac{u_{i,j-1} -2u_{i,j} +u_{i,j+1}}{h_y^2}
\]
If $h_x=h_y$ then
\[
\vec\nabla^2 u (x_i,y_i)= \frac{1}{h^2}
\left(
-5u_{i+1,j} +4u_{i+2,j}-u_{i+3,j} +u_{i,j-1}  +u_{i,j+1}
\right)
\]
corresponding to the following stencil:
\[
\begin{pmatrix}
. & . &  . & . & . \\
. & . &  . & . & . \\
2 &-5 & 4 & -1 &. \\
. & . &  . & . & . \\
. & . &  . & . & . 
\end{pmatrix}
+
\begin{pmatrix}
. & . &  . & . & . \\
1 & . &  . & . & . \\
-2 & . &  . & . & . \\
1 & . &  . & . & . \\
. & . &  . & . & . 
\end{pmatrix}
=
\begin{pmatrix}
. & . &  . & . & . \\
1 & . &  . & . & . \\
0 & -5 & 4 & -1 & . \\
1 & . &  . & . & . \\
. & . &  . & . & . 
\end{pmatrix}
\]


\item Point on the right boundary (i.e. $i=nnx$). We use a 2nd-order central finite difference 
approach in the $y$ direction, but a backward one for the $x$ direction (also 2nd-order):
\[
\vec\nabla^2 u (x_i,y_i)= \frac{\partial^2 u}{\partial x^2}(x_i,y_i)+
\frac{\partial^2 u}{\partial y^2}(x_i,y_i)
= 
\frac{2u_{i,j} -5u_{i-1,j} +4u_{i-2,j}-u_{i-3,j}}{h_x^2} + \frac{u_{i,j-1} -2u_{i,j} +u_{i,j+1}}{h_y^2}
\]
If $h_x=h_y$ then
\[
\vec\nabla^2 u (x_i,y_i)= \frac{1}{h^2}
\left(
-5u_{i-1,j} +4u_{i-2,j}-u_{i-3,j} +u_{i,j-1}  +u_{i,j+1}
\right)
\]
corresponding to the following stencil:
\[
\begin{pmatrix}
. & . &  . & . & . \\
. & . &  . & . & 1 \\
. & -1 & 4 & -5 & 0 \\
. & . &  . & . & 1 \\
. & . &  . & . & . 
\end{pmatrix}
\]

\item Point on the bottom boundary (i.e. $j=1$).
We use a 2nd-order central finite difference 
approach in the $x$ direction, but a backward one for the $y$ direction 
(only 1st-order):

\[
\begin{pmatrix}
. & . &  . & . & . \\
. & . &  . & . & . \\
. & . &  . & . & . \\
. & . &  . & . & . \\
. & 1 & -2 & 1 & . 
\end{pmatrix}
+
\begin{pmatrix}
. & . &  . & . & . \\
. & . &  . & . & . \\
. & . & 1 & . &  \\
. & .& -2 & . &  \\
. & . & 1 & . &   
\end{pmatrix}
=
\begin{pmatrix}
. & . &  . & . & . \\
. & . &  . & . & . \\
. & . &  1 & . & . \\
. & . &  -2 & . & . \\
. & 1 &  -1 & 1 & . 
\end{pmatrix}
\]


\item Point on the top boundary (i.e. $j=nny$). 


\[
\begin{pmatrix}
. & 1 &  -2 & 1 & . \\
. & . &  . & . & . \\
. & . &  . & . & . \\
. & . &  . & . & . \\
. & . &  . & . & . 
\end{pmatrix}
+
\begin{pmatrix}
. & . &  1 & . & . \\
. & . &  -2 & . & . \\
. & . &  1 & . & . \\
. & . &  . & . & . \\
. & . &  . & . & . 
\end{pmatrix}
=
\begin{pmatrix}
. & 1 &  -1 & 1 & . \\
. & . &  -2 & . & . \\
. & . &  1 & . & . \\
. & . &  . & . & . \\
. & . &  . & . & . 
\end{pmatrix}
\]


\item Point at top left corner (1st order)
\[
\begin{pmatrix}
1 & -2 &  1 & . & . \\
. & . &  . & . & . \\
. & . &  . & . & . \\
. & . &  . & . & . \\
. & . &  . & . & . 
\end{pmatrix}
+
\begin{pmatrix}
1 & . &  . & . & . \\
-2 & . &  . & . & . \\
1 & . &  . & . & . \\
. & . &  . & . & . \\
. & . &  . & . & . 
\end{pmatrix}
=
\begin{pmatrix}
2 & -2 &  1 & . & . \\
-2 & . &  . & . & . \\
1 & . &  . & . & . \\
. & . &  . & . & . \\
. & . &  . & . & . 
\end{pmatrix}
\]


\item Point at top left corner (2nd order)
\[
\begin{pmatrix}
2 & -5 &  4 & -1 & . \\
. & . &  . & . & . \\
. & . &  . & . & . \\
. & . &  . & . & . \\
. & . &  . & . & . 
\end{pmatrix}
+
\begin{pmatrix}
2 & . &  . & . & . \\
-5 & . &  . & . & . \\
4 & . &  . & . & . \\
-1 & . &  . & . & . \\
. & . &  . & . & . 
\end{pmatrix}
=
\begin{pmatrix}
4 & -5 &  4 & -1 & . \\
-5 & . &  . & . & . \\
4 & . &  . & . & . \\
-1 & . &  . & . & . \\
. & . &  . & . & . 
\end{pmatrix}
\]




\item Point at top right corner (1st order)

\[
\begin{pmatrix}
. & . &  1 & -2 & 1 \\
. & . &  . & . & . \\
. & . &  . & . & . \\
. & . &  . & . & . \\
. & . &  . & . & . 
\end{pmatrix}
+
\begin{pmatrix}
. & . &  . & . & 1 \\
. & . &  . & . & -2 \\
. & . &  . & . & 1 \\
. & . &  . & . & . \\
. & . &  . & . & . 
\end{pmatrix}
=
\begin{pmatrix}
. & . &  1 & -2 & 2 \\
. & . &  . & . & -2 \\
. & . &  . & . & 1 \\
. & . &  . & . & . \\
. & . &  . & . & . 
\end{pmatrix}
\]

\item Point at top right corner (2nd order)

more of the same ....

\item Point at bottom left corner ($i=j=1$)

more of the same ....

\item Point at bottom right corner

more of the same ....

\end{itemize}












%..............................................................
\subsection{The alternating-direction-implicit (ADI) technique}

This is borroed from Anderson's book, section 6.7.

We start again from the Crank-Nicolson version of the implicit scheme:
\begin{eqnarray}
\frac{T_{{\color{brown}i},{\color{brown}j}}^{n+1}-T_{\color{brown}i,j}^n}{\delta t}
&=& \frac12 \kappa
\left(
\frac{ T_{{\color{brown}i-1,j}}^{n+1}-2T_{{\color{brown}i,j}}^{n+1}+T_{{\color{brown}i+1,j}}^{n+1} }{h_x^2} + 
\frac{ T_{{\color{brown}i,j-1}}^{n+1}-2T_{{\color{brown}i,j}}^{n+1}+T_{{\color{brown}i,j+1}}^{n+1} }{h_y^2}
\right) \nn\\
&+& \frac12 \kappa
\left(
\frac{ T_{{\color{brown}i-1,j}}^{n}-2T_{{\color{brown}i,j}}^{n}+T_{{\color{brown}i+1,j}}^{n} }{h_x^2} + 
\frac{ T_{{\color{brown}i,j-1}}^{n}-2T_{{\color{brown}i,j}}^{n}+T_{{\color{brown}i,j+1}}^{n} }{h_y^2}
\right)
\end{eqnarray}
Although it is usable `as is', there was an incentive a few decennia to keep the matrix 
tridiagonal (as opposed to pentadiagonal in this case) so that the Thomas algorithm\footnote{
\url{https://en.wikipedia.org/wiki/Tridiagonal_matrix_algorithm}.
See also appendix A of Anderson's book.} could be used.

The ADI method arrives at the solution in a two-step process, where intermediate values
of $T$ are found at an intermediate time $t+\delta t/2$, as follows. For each step a tridiagonal 
matrix is generated.

In the first step replace the spatial derivative with central differences where only the $x$ derivative is treated 
implicitely:

\begin{eqnarray}
\frac{T_{{\color{brown}i},{\color{brown}j}}^{n+1/2}-T_{\color{brown}i,j}^n}{\delta t/2}
&=& \kappa \frac{ T_{{\color{brown}i-1,j}}^{n+1/2}-2T_{{\color{brown}i,j}}^{n+1/2}+T_{{\color{brown}i+1,j}}^{n+1/2} }{h_x^2} 
+\kappa \frac{ T_{{\color{brown}i,j-1}}^{n}-2T_{{\color{brown}i,j}}^{n}+T_{{\color{brown}i,j+1}}^{n} }{h_y^2}
\end{eqnarray}

The second step of the ADI scheme takes the solution to $t+\delta t$, using the known values 
at time $t+\delta t/2$:

\begin{eqnarray}
\frac{T_{{\color{brown}i},{\color{brown}j}}^{n+1}-T_{\color{brown}i,j}^{n+1/2}}{\delta t/2}
&=& \kappa \frac{ T_{{\color{brown}i-1,j}}^{n+1/2}-2T_{{\color{brown}i,j}}^{n+1/2}+T_{{\color{brown}i+1,j}}^{n+1/2} }{h_x^2} 
+\kappa 
\frac{ T_{{\color{brown}i,j-1}}^{n+1}-2T_{{\color{brown}i,j}}^{n+1}+T_{{\color{brown}i,j+1}}^{n+1} }{h_y^2}
\end{eqnarray}

The ADI method is second order in time and space.
See also \textcite{krdp08} (2008) for a 3d example.




 
