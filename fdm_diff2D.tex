We now revisit the transient heat equation, this time with sources/sinks for 2D problems.
In the absence of advective heat transport, the heat equation is 
\begin{equation}
\rho C_p \frac{\partial T}{\partial t} =
\vec\nabla \cdot k \vec\nabla T + Q 
\end{equation}
where $Q$ is the radiogenic heat production.
It simply writes as follows when Cartesian coordinates are used:
\begin{equation}
\rho C_p \frac{\partial T}{\partial t} = 
\frac{\partial }{\partial x} \left(  k  \frac{\partial T}{\partial x} \right)+
\frac{\partial }{\partial y} \left(  k  \frac{\partial T}{\partial y} \right)+ Q
\end{equation}
If the heat conductivity is constant in space (and so are the other coefficients), 
it writes:
\begin{equation}
\frac{\partial T}{\partial t} =
\kappa \left(  \frac{\partial^2 T}{\partial x^2} + \frac{\partial^2 T}{\partial y^2} \right)+
\tilde{Q}
\end{equation}
with $\tilde{Q}=Q/\rho C_p$.
In order to solve this equation over the Cartesian domain of size $L_x \times L_y$
we need to generate a mesh as shown hereunder:

%-/-/-/-/-/-/-/-/-/-/-/-/-/-/-/
\begin{minipage}[t]{\textwidth}
\begin{center}
\input{tikz/tikz_fdm5x4mesh1}
\end{center}
\end{minipage}
%-/-/-/-/-/-/-/-/-/-/-/-/-/-/-/

The spacing between the nodes in the $x$-direction is $h_x$ and $h_y$ is the spacing
between the nodes in the $y$ direction. There are now $nnp=nnx\times nny$ nodes in total.
The above grid is characterised by $i=0,1,2,3,4$ and $j=0,1,2,3$ and counts in total 
20 nodes.

In one dimension, the subscript indicated the node $i$. In two dimensions we therefore 
need two indices ${\color{brown}i}$ and ${\color{brown}j}$ 
to identify a node, so that the temperature at node ${\color{brown}i},{\color{brown}j}$ 
at time $n$ is denoted $T_{{\color{brown}i,j}}^n$.

%The vector $\vec{T}$ contains all the temperature unknowns, so it is a vector that is $np$-long. 
%But how should this vector be organised ? In other words, 
One question remains: should we number nodes 
row by row ? column by column ? randomly ? 
These three approaches are shown hereunder: 

\vspace{.5cm}

%-/-/-/-/-/-/-/-/-/-/-/-/-/-/-/
\begin{minipage}[t]{\textwidth}
\input{tikz/tikz_fdm5x4meshes}\\
\end{minipage}
%-/-/-/-/-/-/-/-/-/-/-/-/-/-/-/

\vspace{.5cm}

This is a critical point because the discretised PDE is formulated as a 
function of $T_{{\color{brown} i,j}}$ 
with ${\color{brown}i}=0,\dots nnx-1$ and ${\color{brown}j}=0,\dots nny-1$ 
but the vector $\vec{T}$ containing all these values (encountered in 
implicit methods)
is indexed by a single index ${\color{teal}k}=0,\dots nnp-1$. The numbering strategy determines how easy
it is to go from $({\color{brown}i},{\color{brown}j})$ to ${\color{teal}k}$ and vice versa. 
Very concretely again, where should $T_{\color{brown}3,4}$ be placed in the global 
vector of unknowns $\vec{T}$?

At the same time we cannot do away with ${\color{brown}i,j}$ indices because these are 
needed to locate the direct neighbours of any node and allow to 
form discrete derivatives. 

We then need a (preferably simple/straightforward) 'function' 
which associates to every $({\color{brown} i,j})$ a global index $k$. 
For the first grid with row-wise numbering, we have 
$0\leq {\color{brown}i} \leq 4$ , $0 \leq {\color{brown}j} \leq 3$ 
and $0 \leq {\color{teal}k} \leq 19$. It follows that 
\begin{equation}
{\color{teal} k}({\color{brown}i,j})={\color{brown}j} \cdot nnx+{\color{brown}i}
\end{equation}
This is easy to verify: ${\color{brown}i}=3$ and ${\color{brown}j}=2$ 
indeed corresponds to node \# 13, 
${\color{brown}i}=4$ and ${\color{brown}j}=1$ corresponds to node \# 9, etc ...

%-/-/-/-/-/-/-/-/-/-/-/-/-/-/-/
\begin{minipage}[t]{\textwidth}
\begin{center}
\input{tikz/tikz_fdm5x4mesh2}
\end{center}
\end{minipage}
%-/-/-/-/-/-/-/-/-/-/-/-/-/-/-/


%-/-/-/-/-/-/-/-/-/-/-/-/-/-/-/-/-/-/
\begin{center}
\begin{minipage}[t]{0.77\textwidth}
\par\noindent\rule{\textwidth}{0.4pt}

\begin{center}
\includegraphics[width=0.8cm]{images/garftr} \\
{\color{orange}Exercise FDM-7}
\end{center}

In a new code declare and assign values to 
$nnx$ and $nny$. Compute $nnp$.
Set $L_x=7$ and $L_y=6$. Compute $h_x$ and $h_y$.

Declare two arrays $xcoords$ and $ycoords$ which will 
contain the $x$ and $y$ coordinates of all $nnp$ nodes.

By means of two imbricated for loops
compute these coordinates \& fill both arrays. 

Visualise the nodes with matplotlib.

Tip: Make sure your code works for various 
combinations of $nnx$ and $nny$.

\par\noindent\rule{\textwidth}{0.4pt}
\end{minipage}
\end{center}






%.............................
\paragraph{Explicit scheme} The simplest approach is an {\color{olive} FTCS} 
(forward time, centered space) explicit method like in 1D:
\begin{equation}
\frac{T_{{\color{brown}i,j}}^{n+1}-T_{{\color{brown}i,j}}^n}{\delta t}
= \kappa
\left(
\frac{ T_{{\color{brown}i-1,j}}^{n}-2T_{{\color{brown}i,j}}^{n}+T_{{\color{brown}i+1,j}}^{n}  }{h_x^2} + 
\frac{ T_{{\color{brown}i,j-1}}^{n}-2T_{{\color{brown}i,j}}^{n}+T_{{\color{brown}i,j+1}}^{n}  }{h_y^2}
\right)
+\tilde{Q}_{{\color{brown}i,j}}^n
\end{equation}
where we have assumed that the source term $\tilde{Q}$ 
can depend of space coordinates and therefore 
appears as $\tilde{Q}_{{\color{brown}i,j}}$ in the equation.
We define $s_x$ and $s_y$ as follows:
\begin{equation}
s_x = \frac{\kappa \delta t}{h_x^2}
\quad\quad
s_y = \frac{\kappa \delta t}{h_y^2}
\end{equation}
so that
\begin{equation}
T_{{\color{brown}i,j}}^{n+1} = T_{{\color{brown}i,j}}^n 
+ s_x ( T_{{\color{brown}i-1,j}}^{n}
-2T_{{\color{brown}i,j}}^{n}
+T_{{\color{brown}i+1,j}}^{n} ) 
+s_y ( T_{{\color{brown}i,j-1}}^{n}
-2T_{{\color{brown}i,j}}^{n}
+T_{{\color{brown}i,j+1}}^{n} ) + 
\tilde{Q}_{{\color{brown}i,j}}^n \delta t
\end{equation}
or, 
\begin{equation}
T_{{\color{teal}k}({\color{brown}i,j})}^{n+1} = 
T_{{\color{teal}k}({\color{brown}i,j})}^n 
+ s_x ( T_{{\color{teal}k}({\color{brown}i-1,j})}^{n}
-2T_{{\color{teal}k}({\color{brown}i,j})}^{n}
+T_{{\color{teal}k}({\color{brown}i+1,j})}^{n} ) 
+s_y ( T_{{\color{teal}k}({\color{brown}i,j-1})}^{n}
-2T_{{\color{teal}k}({\color{brown}i,j})}^{n}
+T_{{\color{teal}k}({\color{brown}i,j+1})}^{n} ) + 
\tilde{Q}_{{\color{teal}k}({\color{brown}i,j})}^n \delta t
\end{equation}



The scheme is stable for  
\begin{equation}
\delta t \leq \frac{\min(h_x^2,h_y^2)}{2 \kappa}
\end{equation}
Boundary conditions can be set the usual way: for example a constant (Dirichlet) temperature 
at node $({\color{brown}i},{\color{brown}j})$ is given by
\begin{equation}
T_{{\color{brown}i},{\color{brown}j}}=T_{bc} 
\end{equation}
where $T_{bc}$ is the prescribed temperature. 

%-/-/-/-/-/-/-/-/-/-/-/-/-/-/-/-/-/-/-/
\begin{center}
\begin{minipage}[t]{0.77\textwidth}
\par\noindent\rule{\textwidth}{0.4pt}
\begin{center}
\includegraphics[width=0.8cm]{images/garftr} \\
{\color{orange}Exercise FDM-8}
\end{center}

A simple (time-dependent) analytical solution for the temperature equation exists for 
the case that the initial temperature field is
\begin{equation}
T(x,y,t=0) = T_0+ T_{max} \exp \left[ -\frac{x^2+y^2}{\sigma^2}   \right]
\end{equation}
where $T_{max}$ is the maximum amplitude of the temperature perturbation 
at $(x,y) = (0, 0)$ and $\sigma$ its half-width. 

\begin{center}
\includegraphics[width=5cm]{images/fdm/gaussian}\\
{\captionfont initial temperature field}
\end{center}

The solution of the time-dependent PDE is
\begin{equation}
T(x,y,t)=T_0 + \frac{T_{max}}{1+4t\kappa/\sigma^2 } \exp \left[ -\frac{x^2+y^2}{\sigma^2 + 4t\kappa}   \right]
\end{equation}

Set $L_x$=100km and $L_y=80$km, $\kappa=10^{-6}$, $\tilde{Q}=0$, $T_{max}=100\degree$, $T_0=200\degree$, 
and $\sigma=10^4$m. 

Use the previous exercise to generate a $nnx\times nny$ grid 
in the $[-L_x/2,L_x/2]\times[-L_y/2,L_y/2]$ domain.

Write a function which takes $x$, $y$, $t$, $T_0$, $T_{max}$, $\kappa$ and $\sigma$ as argument 
and returns the analytical temperature value.

Write a an explicit FDM code which solves the 2D diffusion equation. At each time step 
prescribe on the boundary the analytical solution.  

\par\noindent\rule{\textwidth}{0.4pt}
\end{minipage}
\end{center}
%-/-/-/-/-/-/-/-/-/-/-/-/-/-/-/-/-/-/-/



%...........................
\paragraph{Implicit scheme} 
If we now employ a fully implicit, unconditionally stable discretization scheme, the discretised 
PDE becomes:
\begin{equation}
\frac{T_{{\color{brown}i},{\color{brown}j}}^{n+1}-T_{\color{brown}i,j}^n}{\delta t}
= \kappa
\left(
\frac{ T_{{\color{brown}i-1,j}}^{n+1}-2T_{{\color{brown}i,j}}^{n+1}+T_{{\color{brown}i+1,j}}^{n+1} }{h_x^2} + 
\frac{ T_{{\color{brown}i,j-1}}^{n+1}-2T_{{\color{brown}i,j}}^{n+1}+T_{{\color{brown}i,j+1}}^{n+1} }{h_y^2}
\right)
+\frac{Q_{{\color{brown}i,j}}^n}{\rho C_p}
\end{equation}
Rearranging terms with $n+1$ on the left and terms with $n$ on the right hand side gives
\begin{equation}
-s_x\; T_{{\color{brown}i+1,j}}^{n+1}
-s_y\; T_{{\color{brown}i,j+1}}^{n+1} 
+(1+2s_x+2s_y)\; T_{{\color{brown}i,j}}^{n+1} 
-s_x\;  T_{{\color{brown}i-1,j}}^{n+1} 
-s_y\;  T_{{\color{brown}i,j-1}}^{n+1} 
=
T_{{\color{brown}i,j}}^n
+\tilde{Q}_{{\color{brown}i,j}}^n \delta t
\end{equation}
or
\begin{equation}
-s_x\;           T_{{\color{teal}k}({\color{brown}i+1,j})}^{n+1}
-s_y\;           T_{{\color{teal}k}({\color{brown}i,j+1})}^{n+1} 
+(1+2s_x+2s_y)\; T_{{\color{teal}k}({\color{brown}i,j}  )}^{n+1} 
-s_x\;           T_{{\color{teal}k}({\color{brown}i-1,j})}^{n+1} 
-s_y\;           T_{{\color{teal}k}({\color{brown}i,j-1})}^{n+1} 
=
T_{{\color{teal}k}({\color{brown}i,j})}^n
+\tilde{Q}_{{\color{teal}k}({\color{brown}i,j})}^n \delta t
\end{equation}
which here again yields a linear system of equations written ${\bm A}\cdot {\vec T} = {\vec b}$
where ${\bm A}$ is a $(nnp \times nnp)$ matrix.

Boundary conditions are $T(x,y)=0$ on all sides, so all nodes 
on the boundary have a prescribed zero temperature\footnote{We assume
here again that these boundary conditions do not change with time.}:
\begin{eqnarray}
T_{\color{brown}0,0} = T_{\color{teal} 0} &=& 0 \nn\\
T_{\color{brown}1,0} = T_{\color{teal} 1} &=& 0 \nn\\
T_{\color{brown}2,0} = T_{\color{teal} 2} &=& 0 \nn\\
T_{\color{brown}3,0} = T_{\color{teal} 3} &=& 0 \nn\\
T_{\color{brown}4,0} = T_{\color{teal} 4} &=& 0 \nn\\
T_{\color{brown}0,1} = T_{\color{teal} 5} &=& 0 \nn\\
T_{\color{brown}4,1} = T_{\color{teal} 9} &=& 0 \nn\\
T_{\color{brown}0,2} = T_{\color{teal} 10} &=& 0 \nn\\
T_{\color{brown}4,2} = T_{\color{teal} 14} &=& 0 \nn\\
T_{\color{brown}0,3} = T_{\color{teal} 15} &=& 0 \nn\\
T_{\color{brown}1,3} = T_{\color{teal} 16} &=& 0 \nn\\
T_{\color{brown}2,3} = T_{\color{teal} 17} &=& 0 \nn\\
T_{\color{brown}3,3} = T_{\color{teal} 18} &=& 0 \nn\\
T_{\color{brown}4,3} = T_{\color{teal} 19} &=& 0 \nn
\end{eqnarray}
In what follows we assume for simplicity and conciseness of notation that 
$h_x=h_y=h$ so that $s_x=s_y=s$.
The discretised PDE equation will now be applied to the interior nodes:
%\begin{eqnarray}
%-s T_{{\color{brown} i+1,j}}^{n+1}
%-s T_{{\color{brown}i,j+1}}^{n+1} 
%+(1+4s)T_{{\color{brown}i,j}}^{n+1} 
%-s T_{{\color{brown}i-1,j}}^{n+1} 
%-s T_{{\color{brown}i,j-1}}^{n+1} 
%= T_{{\color{brown}i,j}}^n 
%+\tilde{Q}_{{\color{brown}i,j}}^n
%\end{eqnarray}

\begin{itemize}
\item For node ${\color{teal}k}=6$ (${\color{brown}i}=1,{\color{brown}j}=1$):
\begin{eqnarray}
-s T_{{\color{brown}2,1}}^{n+1}
-s T_{{\color{brown}1,2}}^{n+1} 
+(1+4s)T_{{\color{brown}1,1}}^{n+1} 
-s T_{{\color{brown}0,1}}^{n+1} 
-s T_{{\color{brown}1,0}}^{n+1} 
&=& T_{{\color{brown}1,1}}^n +\tilde{Q}_{{\color{brown}1,1}}^n \delta t\nn\\
\Rightarrow \qquad
-s T_{{\color{teal} 7}}^{n+1}-s T_{{\color{teal} 11}}^{n+1} +(1+4s)T_{{\color{teal} 6}}^{n+1} 
-s T_{{\color{teal} 5}}^{n+1} -s T_{{\color{teal} 1}}^{n+1} 
&=& T_{{\color{teal} 6}}^n +\tilde{Q}_{{\color{teal} 6}}^n\delta t
\end{eqnarray}

\item For node ${\color{teal}k}=7$ (${\color{brown}i}=2,{\color{brown}j}=1$):
\begin{eqnarray}
-s T_{{\color{brown} 3,1}}^{n+1}
-s T_{{\color{brown}2,2}}^{n+1} 
+(1+4s)T_{{\color{brown}2,1}}^{n+1} 
-s T_{{\color{brown}1,1}}^{n+1} 
-s T_{{\color{brown}2,0}}^{n+1} 
&=& T_{{\color{brown}2,1}}^n 
+\tilde{Q}_{{\color{brown}2,1}}^n \delta t\nn\\
\Rightarrow \qquad
-s T_{{\color{teal} 8}}^{n+1}
-s T_{{\color{teal} 12}}^{n+1} 
+(1+4s)T_{{\color{teal}7}}^{n+1} 
-s T_{{\color{teal}6}}^{n+1} 
-s T_{{\color{teal}2}}^{n+1} 
&=& T_{{\color{teal}7}}^n 
+\tilde{Q}_{{\color{teal}7}}^n \delta t
\end{eqnarray}

\item For node ${\color{teal}k}=8$ (${\color{brown}i}=3,{\color{brown}j}=1$):
\begin{eqnarray}
-s T_{{\color{brown} 4,1}}^{n+1}
-s T_{{\color{brown}3,2}}^{n+1} 
+(1+4s)T_{{\color{brown}3,1}}^{n+1} 
-s T_{{\color{brown}2,1}}^{n+1} 
-s T_{{\color{brown}3,0}}^{n+1} 
&=& T_{{\color{brown}3,1}}^n 
+\tilde{Q}_{{\color{brown}3,1}}^n \delta t\nn\\
\Rightarrow \qquad
-s T_{{\color{teal} 9}}^{n+1}
-s T_{{\color{teal} 13}}^{n+1} 
+(1+4s)T_{{\color{teal}8}}^{n+1} 
-s T_{{\color{teal}7}}^{n+1} 
-s T_{{\color{teal}3}}^{n+1} 
&=& T_{{\color{teal}8}}^n 
+\tilde{Q}_{{\color{teal}8}}^n \delta t
\end{eqnarray}

\item For node ${\color{teal}k}=11$ (${\color{brown}i}=1,{\color{brown}j}=2$):
\begin{eqnarray}
-s T_{{\color{brown} 2,2}}^{n+1}
-s T_{{\color{brown}1,3}}^{n+1} 
+(1+4s)T_{{\color{brown}1,2}}^{n+1} 
-s T_{{\color{brown}0,2}}^{n+1} 
-s T_{{\color{brown}2,1}}^{n+1} 
&=& T_{{\color{brown}1,2}}^n 
+\tilde{Q}_{{\color{brown}1,2}}^n \delta t\nn\\
\Rightarrow \qquad
-s T_{{\color{teal} 12}}^{n+1}
-s T_{{\color{teal} 16}}^{n+1} 
+(1+4s)T_{{\color{teal}11}}^{n+1} 
-s T_{{\color{teal}10}}^{n+1} 
-s T_{{\color{teal}6}}^{n+1} 
&=& T_{{\color{teal}11}}^n 
+\tilde{Q}_{{\color{teal}11}}^n\delta t
\end{eqnarray}

\item For node ${\color{teal}k}=12$ (${\color{brown}i}=2,{\color{brown}j}=2$):
\begin{eqnarray}
-s T_{{\color{brown} 3,2}}^{n+1}
-s T_{{\color{brown}2,3}}^{n+1} 
+(1+4s)T_{{\color{brown}2,2}}^{n+1} 
-s T_{{\color{brown}1,2}}^{n+1} 
-s T_{{\color{brown}2,1}}^{n+1} 
&=& T_{{\color{brown}2,2}}^n 
+\tilde{Q}_{{\color{brown}2,2}}^n \delta t \nn\\
\Rightarrow \qquad
-s T_{{\color{teal} 13}}^{n+1}
-s T_{{\color{teal} 17}}^{n+1} 
+(1+4s)T_{{\color{teal}12}}^{n+1} 
-s T_{{\color{teal}11}}^{n+1} 
-s T_{{\color{teal}7}}^{n+1} 
&=& T_{{\color{teal}12}}^n 
+\tilde{Q}_{{\color{teal}12}}^n\delta t
\end{eqnarray}


\item For node ${\color{teal}k}=13$ (${\color{brown}i}=3,{\color{brown}j}=2$):
\begin{eqnarray}
-s T_{{\color{brown} 4,2}}^{n+1}
-s T_{{\color{brown}3,3}}^{n+1} 
+(1+4s)T_{{\color{brown}3,2}}^{n+1} 
-s T_{{\color{brown}2,2}}^{n+1} 
-s T_{{\color{brown}3,1}}^{n+1} 
&=& T_{{\color{brown}3,2}}^n 
+\tilde{Q}_{{\color{brown}3,2}}^n \delta t\nn\\
\Rightarrow \qquad
-s T_{{\color{teal} 14}}^{n+1}
-s T_{{\color{teal} 18}}^{n+1} 
+(1+4s)T_{{\color{teal}13}}^{n+1} 
-s T_{{\color{teal}12}}^{n+1} 
-s T_{{\color{teal}8}}^{n+1} 
&=& T_{{\color{teal}13}}^n 
+\tilde{Q}_{{\color{teal}13}}^n \delta t
\end{eqnarray}


\end{itemize}

Putting it all together yields the following linear system:

\begin{landscape}
\[
\underbrace{
\left(
\begin{array}{cccccccccccccccccccc}
1 & . & . & . & . & . & . & . & . & . & . & . & . & . & . & . & . & . & . & . \\ %#0
. & 1 & . & . & . & . & . & . & . & . & . & . & . & . & . & . & . & . & . & . \\ %#1
. & . & 1 & . & . & . & . & . & . & . & . & . & . & . & . & . & . & . & . & . \\ %#2
. & . & . & 1 & . & . & . & . & . & . & . & . & . & . & . & . & . & . & . & . \\ %#3
. & . & . & . & 1 & . & . & . & . & . & . & . & . & . & . & . & . & . & . & . \\ %#4
. & . & . & . & . & 1 & . & . & . & . & . & . & . & . & . & . & . & . & . & . \\ %#5
. & -s& . & . & . & -s& {1+4s} & -s& . & . & . & -s & . & . & . & . & . & . & . & . \\ %#6
. & . & -s& . & . & . & -s& {1+4s} & -s& . & . & . & -s & . & . & . & . & . & . & .\\ %#7
. & . & . & -s& . & . & . & -s & {1+4s} & -s & . & . & . & -s & . & . & . & . & . & . \\ %#8
. & . & . & . & . & . & . & . & . & 1 & . & . & . & . & . & . & . & . & . & . \\ %#9
. & . & . & . & . & . & . & . & . & . & 1 & . & . & . & . & . & . & . & . & . \\ %#10
. & . & . & . & . & . &-s & . & . & . & -s& {1+4s} & -s& . & . & .  & -s& . & . & .\\ %#11
. & . & . & . & . & . & . &-s & . & . & . & -s& {1+4s} & -s& . & . & .  & -s& . & .\\ %#12
. & . & . & . & . & . & . & . &-s & . & . & . & -s& {1+4s} & -s& . & . & .  & -s & .\\ %#13
. & . & . & . & . & . & . & . & . & . & . & . & . & . & 1 & . & . & . & . & . \\ %#14
. & . & . & . & . & . & . & . & . & . & . & . & . & . & . & 1 & . & . & . & . \\ %#15
. & . & . & . & . & . & . & . & . & . & . & . & . & . & . & . & 1 & . & . & . \\ %#16
. & . & . & . & . & . & . & . & . & . & . & . & . & . & . & . & . & 1 & . & . \\ %#17
. & . & . & . & . & . & . & . & . & . & . & . & . & . & . & . & . & . & 1 & . \\ %#18
. & . & . & . & . & . & . & . & . & . & . & . & . & . & . & . & . & . & . & 1    %#19
\end{array}
\right)
}_{\bm A}
\cdot
\underbrace{
\left(
\begin{array}{c}
T_{{\color{teal}0}}^{n+1} \\ 
T_{{\color{teal}1}}^{n+1} \\ 
T_{{\color{teal}2}}^{n+1} \\ 
T_{{\color{teal}3}}^{n+1} \\ 
T_{{\color{teal}4}}^{n+1} \\ 
T_{{\color{teal}5}}^{n+1} \\ 
T_{{\color{teal}6}}^{n+1} \\ 
T_{{\color{teal}7}}^{n+1} \\ 
T_{{\color{teal}8}}^{n+1} \\ 
T_{{\color{teal}9}}^{n+1} \\ 
T_{{\color{teal}10}}^{n+1} \\ 
T_{{\color{teal}11}}^{n+1} \\ 
T_{{\color{teal}12}}^{n+1} \\ 
T_{{\color{teal}13}}^{n+1} \\ 
T_{{\color{teal}14}}^{n+1} \\ 
T_{{\color{teal}15}}^{n+1} \\ 
T_{{\color{teal}16}}^{n+1} \\ 
T_{{\color{teal}17}}^{n+1} \\ 
T_{{\color{teal}18}}^{n+1} \\ 
T_{{\color{teal}19}}^{n+1} 
\end{array}
\right)
}_{\vec T}
=
\underbrace{
\left(
\begin{array}{c}
0\\ 
0\\ 
0\\ 
0\\ 
0\\ 
0\\ 
T_{{\color{teal}6}}^n + \tilde{Q}_{\color{teal}6} \delta t\\ 
T_{{\color{teal}7}}^n + \tilde{Q}_{\color{teal}7} \delta t\\ 
T_{{\color{teal}8}}^n + \tilde{Q}_{\color{teal}8} \delta t\\ 
0\\ 
0\\ 
T_{{\color{teal}11}}^n + \tilde{Q}_{\color{teal}11} \delta t\\ 
T_{{\color{teal}12}}^n + \tilde{Q}_{\color{teal}12} \delta t\\ 
T_{{\color{teal}13}}^n + \tilde{Q}_{\color{teal}13} \delta t\\ 
0\\ 
0\\ 
0\\ 
0\\ 
0\\ 
0 
\end{array}
\right)
}_{\vec b}
\]
Note that we now have five 'diagonals' filled with non-zero entries as opposed to three
diagonals in the 1D case.

\todo[inline]{replace zeros by some more random value corresponding to Tboundary}

Note also that this is a simplified matrix since we assumed that $s_x=s_y$.

\end{landscape}

%-/-/-/-/-/-/-/-/-/-/-/-/-/-/-/-/-/-/-/-/-/-/
\begin{center}
\begin{minipage}[t]{0.77\textwidth}
\par\noindent\rule{\textwidth}{0.4pt}
\begin{center}
\includegraphics[width=0.8cm]{images/garftr} \\
{\color{orange}Exercise FDM-9}
\end{center}

Same exercise as exercise FDM-8, but now with implicit method.

\par\noindent\rule{\textwidth}{0.4pt}
\end{minipage}
\end{center}
%-/-/-/-/-/-/-/-/-/-/-/-/-/-/-/-/-/-/-/-/-/-/


Looking at this matrix, it is clear that this approach is sub-optimal: for such a small grid counting
20 nodes, the boundary conditions enforce the temperature on 14 of them, so that these
temperatures should/could be removed from the list of unknowns, leaving a vector 
of unknowns $\vec{T}$ of size 6 (the number of nodes which are not on the boundary).
As a consequence, we would have to solve a $6\times 6$ linear system, as opposed to a $20\times 20$ one!

In this case, we focus again on nodes 6,7,8,11,12,13.
we start from 
\begin{equation}
-s T_{{\color{teal} 7}}^{n+1}
-s T_{{\color{teal} 11}}^{n+1} 
+(1+4s)T_{{\color{teal} 6}}^{n+1} 
-s T_{{\color{teal} 5}}^{n+1} 
-s T_{{\color{teal} 1}}^{n+1} 
= T_{{\color{teal} 6}}^n 
+\tilde{Q}_{{\color{teal} 6}}^n \delta t
\end{equation}
but we know that the boundary conditions impose that $T_{\color{teal}1}=0$ 
and $T_{\color{teal}5}=0$ so that the equation above simplifies to:
\begin{equation}
-s T_{{\color{teal} 7}}^{n+1}
-s T_{{\color{teal} 11}}^{n+1} 
+(1+4s)T_{{\color{teal} 6}}^{n+1} 
= T_{{\color{teal} 6}}^n 
+\tilde{Q}_{{\color{teal} 6}}^n\delta t
\end{equation}

These 6 equations can finally be combined in the expected smaller linear system:
\begin{equation}
\underbrace{
\left(
\begin{array}{cccccc}
1+4s & -s & . & -s & . & . \\
-s & 1+4s & -s & . & -s & . \\
. & -s & 1+4s & . & . & -s \\ 
-s & . & -s & 1+4s & -s & . \\
. & -s & . & -s & 1+4s & -s \\
. & . & -s & . & -s & 1+4s 
\end{array}
\right)
}_{\bm A}
\cdot
\underbrace{
\left(
\begin{array}{c}
T_{{\color{teal}6}}^{n+1} \\ 
T_{{\color{teal}7}}^{n+1} \\ 
T_{{\color{teal}8}}^{n+1} \\ 
T_{{\color{teal}11}}^{n+1} \\ 
T_{{\color{teal}12}}^{n+1} \\ 
T_{{\color{teal}13}}^{n+1} 
\end{array}
\right)
}_{\vec T}
=
\underbrace{
\left(
\begin{array}{c}
T_{{\color{teal}6}}^n + \tilde{Q}_{\color{teal}6}\delta t \\ 
T_{{\color{teal}7}}^n + \tilde{Q}_{\color{teal}7} \delta t\\ 
T_{{\color{teal}8}}^n + \tilde{Q}_{\color{teal}8} \delta t\\ 
T_{{\color{teal}11}}^n + \tilde{Q}_{\color{teal}11} \delta t\\ 
T_{{\color{teal}12}}^n + \tilde{Q}_{\color{teal}12} \delta t\\ 
T_{{\color{teal}13}}^n + \tilde{Q}_{\color{teal}13} \delta t
\end{array}
\right)
}_{\vec b}
\end{equation}
Note that is the boundary values had not been zero they would have found their way to the right hand side 
vector.

The Crank-Nicolson version of the implicit scheme is then as follows:
\begin{eqnarray}
\frac{T_{{\color{brown}i},{\color{brown}j}}^{n+1}-T_{\color{brown}i,j}^n}{\delta t}
&=& \frac12 \kappa
\left(
\frac{ T_{{\color{brown}i-1,j}}^{n+1}-2T_{{\color{brown}i,j}}^{n+1}+T_{{\color{brown}i+1,j}}^{n+1} }{h_x^2} + 
\frac{ T_{{\color{brown}i,j-1}}^{n+1}-2T_{{\color{brown}i,j}}^{n+1}+T_{{\color{brown}i,j+1}}^{n+1} }{h_y^2}
\right) \nn\\
&+& \frac12 \kappa
\left(
\frac{ T_{{\color{brown}i-1,j}}^{n}-2T_{{\color{brown}i,j}}^{n}+T_{{\color{brown}i+1,j}}^{n} }{h_x^2} + 
\frac{ T_{{\color{brown}i,j-1}}^{n}-2T_{{\color{brown}i,j}}^{n}+T_{{\color{brown}i,j+1}}^{n} }{h_y^2}
\right) 
\end{eqnarray}
The implementation of this method will require from you to bring 
all the terms in $T^{n+1}$ to the left of the equal sign 
while all the terms in $T^n$ are assumed to be known and therefore find their way into
the right hand side. 

Likewise, the Lax-Friedrichs method is as follows:
\begin{equation}
\frac{T_{{\color{brown}i},{\color{brown}j}}^{n+1}-
\frac{1}{4} \left(  
T_{{\color{brown}i-1,j}}^{n}  +
T_{{\color{brown}i+1,j}}^{n}  +
T_{{\color{brown}i,j-1}}^{n}  +
T_{{\color{brown}i,j+1}}^{n}  
\right)
}{\delta t}
= \kappa
\left(
\frac{ T_{{\color{brown}i-1,j}}^{n+1}-2T_{{\color{brown}i,j}}^{n+1}+T_{{\color{brown}i+1,j}}^{n+1} }{h_x^2} + 
\frac{ T_{{\color{brown}i,j-1}}^{n+1}-2T_{{\color{brown}i,j}}^{n+1}+T_{{\color{brown}i,j+1}}^{n+1} }{h_y^2}
\right)
+\frac{Q_{{\color{brown}i,j}}^n}{\rho C_p}
\end{equation}
Rearranging terms with $n+1$ on the left and terms with $n$ on the right hand side gives

\index{general}{Lax-Friedrichs Method}
\index{general}{Crank-Nicolson Method}


%STENCILS in 2D?

%-/-/-/-/-/-/-/-/-/-/-/-/-/-/-/-/-/-/
%\begin{center}
%\begin{minipage}[t]{0.77\textwidth}
%\par\noindent\rule{\textwidth}{0.4pt}
%\begin{center}
%\includegraphics[width=0.8cm]{images/garftr} \\
%{\color{orange}Exercise 7}
%\end{center}

%\par\noindent\rule{\textwidth}{0.4pt}
%\end{minipage}
%\end{center}
%-/-/-/-/-/-/-/-/-/-/-/-/-/-/-/-/-/-/



