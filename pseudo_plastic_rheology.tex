\begin{flushright} {\tiny {\color{gray} pseudo\_plastic\_rheology.tex}} \end{flushright}
%~~~~~~~~~~~~~~~~~~~~~~~~~~~~~~~~~~~~~~~~~~~~~~~~~~~~~~~~~~~~~~~~~~~~~~~~~~~~~~~~~~~~~~~~~~~~~~~~~~

Taken from van Hunen \etal (2002) \cite{vavv02}:
\[
\eta_y = \tau_y \dot{\varepsilon}_y^{-1/n_y} \dot{\varepsilon}_e^{(1/n_y) -1 } 
\]
where the yield stress $\tau_y$, the yield strain rate $\dot{\varepsilon}_y$ and the yield exponent $n_y$ are
prescribed parameters. In this article, $n_y=10$, $\dot{\varepsilon}_y=10^{-15}\si{\per\second}$
When $n_y=1$ the viscosity is constant and given by $\eta_{eff} = \tau_y / \dot{\epsilon}_y$.

This rheology has also been coined pseudo-plastic in Zhong \etal (1998) \cite{zhgm98}. 
Their equation is simply  
\[
\eta_{eff} = A^{1/n} \dot{\varepsilon}_e^{-1+1/n}
\]
where $A$ is the preexponent which depends on temperature, pressure, and composition.
\begin{center}
\includegraphics[width=5.5cm]{images/rheology/zhgm98}
\includegraphics[width=5.8cm]{images/rheology/pseudoplastic/stress}
\includegraphics[width=5.8cm]{images/rheology/pseudoplastic/eta_eff}\\
{\captionfont Left figure is taken from \cite{zhgm98}. Authors report $A=7.9\cdot 10^{-8} \si{\pascal^3\second}$ 
for the $n=3$ case, which makes no sense. See gnuplot script for actual values of $A$.}
\end{center}


