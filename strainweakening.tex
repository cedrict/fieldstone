Several mechanisms may contribute to strain or strain
rate dependent weakening but their relative and absolute
importance is poorly constrained. Furthermore, 
weakening mechanisms are often crudely parameterised in 
geodynamical codes with simple mathematical functions 
and a limited number of parameters. 

For example, in \cite{alht11} the authors use a von Mises plasticity formulation so that the 
rheology is parameterised by the cohesion $c$, or $c=\sigma_y$ in their notations. The
yield strength $\sigma_y$ starts is constant until the strain
$\varepsilon$ reaches the threshold value $\varepsilon_1$. It then decreases linearly
from $\sigma_y$ to $\sigma_{y}^{sw}$ between $\varepsilon_1$ and $\varepsilon_2$. 
For strain values $\varepsilon>\varepsilon_2$ , the yield strength remains constant 
at $\sigma_y^{sw}$ .

\begin{center}
\includegraphics[width=6cm]{images/strainweakening/alht11}\\
{\tiny Taken from \cite{alht11}}
\end{center}

The same authors in a subsequent study use a Drucker-Prager rheology parameterised by 
cohesion $c$ and friction angle $\phi$. They use the same approach as before but now 
both parameters are subjected to strain weakening: 

\begin{center}
\includegraphics[width=6cm]{images/strainweakening/alht12}\\
{\tiny Taken from \cite{alht12}, see also \cite{thie11}}
\end{center}

They further define the factor $R=C^0/C^{sw}=\phi^0/\phi^{sw}\geq 1$ which is a proxy
for the ratio $\sigma_y/\sigma_y^{sw}$ where $\sigma_y=p \sin\phi + c \; \cos \phi$, 
and carry out 3D crustal extensional models for $R$ between 2 and 5. 



In \cite{lemh17} the authors also define 
\[
\tau_y = p \sin (\phi(\varepsilon^p))  + c_0 \cos(\phi(\varepsilon^p))
\]
but the cohesion is regarded to be constant. 
The angle of friction $\phi$ is assumed to decrease as a function of the accumulated plastic
strain $\varepsilon^p$ to
\[
\phi(\varepsilon^p) 
=
\max \left(
\phi_\infty , \phi_0 - \frac{\varepsilon^p (\phi_0-\phi_\infty)}{\varepsilon^p_\infty}
\right)
\]
This equation defines an empirical softening relation which reduces the
friction angle linearly with accumulated plastic strain.
$\phi_0$ defines the initial friction angle, $\varepsilon^p_\infty$
represents the measure of plastic strain after which complete softening is achieved and internal
friction angle reaches $\phi_\infty$ . Plastic strain represents an integrated,
tensorial invariant measure of the deformation which has occurred
due to plastic yielding. Thus, the quantity $\varepsilon^p$ can be regarded as
a simplified measure of material damage. 

\Literature: \cite{nigo15}
