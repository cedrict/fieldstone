\chapter{[WIP] Numeriek modelleren van geodynamische processen} %%%%%%%%%%%%%%%%%%%%%%%%%%%%%%%%%%%%%%%%

\begin{flushright} {\tiny {\color{gray} \tt chapter\_syllabus\_arie.tex}} \end{flushright}
%~~~~~~~~~~~~~~~~~~~~~~~~~~~~~~~~~~~~~~~~~~~~~~~~~~~~~~~~~~~~~~~~~~~~~~~~~~~~~~~~~~~~~~~~~~~~~~~~~~

%==============================================================================
\section{Inleiding}

De zin en noodzaak van modelexperimenten 
met betrekking tot geodynamische processen is vooral gelegen in het 
volgende:
\begin{itemize}
\item Beperkte mogelijkheid van direkte (in situ) waarnemingen 
als aanvulling op indirecte bijv. seismologische- of gravimetrische
waarnemingen.
Denk hierbij bijvoorbeeld aan het beperkte bereik van diepboringen.
\item De extreem lange tijdschaal waarop de processen plaatsvinden
 - postglaciale opheffing $\sim 10^5 \ yr$, mantelconvectie overturntijd
 $\sim 10^8 \ yr$.
Het is dan praktisch onmogelijk om de afloop van de processen 
voldoende ver in de tijd te volgen.
\end{itemize}

\noindent
Verschillende manieren van modelleren die worden toegepast zijn:

\begin{itemize}

\item Fysisch modelleren (laboratorium experimenten).
voorbeelden van geodynamische toepassingen zijn
\begin{itemize}
\item Simulatie van mantelconvectie door bestudering van Rayleigh-Benard
convectie in tankmodellen.
\item Bestudering van plastische deformatie van gelaagde structuren in
centrifuge experimenten.
\end{itemize}

\item Theoretisch (wiskundig) modelleren.
Theoretische modellen kunnen worden opgesteld voor geofysische
problemen door formulering van wiskundige modelvergelijkingen.
De modelvergelijkingen zijn i.h.a. parti\"{e}le 
differentiaalvergelijkingen, die kunnen worden afgeleid uit fysische
behoudswetten.
Een in het vervolg veelvuldig besproken voorbeeld hiervan is de 
warmte (diffusie) vergelijking voor een statisch medium 
- zie bijv. hoofdstukken 2 en 3 -
die wordt afgeleid uit een thermische energiebalans,
$$
\int_V \rho c_p \frac{\partial T}{\partial t} dV =
\int_{\partial V} - {\bf q} \cdot {\bf n} dA +
\int_V H dV
$$
waarin ${\bf q} \cdot {\bf n}$ de warmtestroomdichtheid door het
oppervlak $\partial V$ voorstelt en $H$ de interne warmteproduktiviteit.

Dergelijke theoretische modellen hebben voor geofysische
problemen een lange traditie.
De gevolgde werkwijze is daarbij voornamelijk analytisch
geweest.
De modelvergelijkingen werden analytisch opgelost en de
resulterende oplossingen - bijvoorbeeld een temperatuurveld
als oplossing van de warmtevergelijking - kunnen eventueel numeriek
worden gee\*:valueerd.
Een voorbeeld hiervan vinden we bij het evalueren van de
Fourier-reeks oplossing voor een afkoelende laag waarvan 
onder- en bovenzijde op constante temperatuur gehouden worden,
$$
T(z,t) = \sum_{n=1}^{\infty} 
A_n sin ( n \pi z/d )
e^{- (n \pi /d)^2 \kappa t}
$$
met
$$
A_n = \frac{2}{d}
\int_0^d T(z,0) sin ( n \pi z/d ) dz
$$

Analytische oplosmethoden zijn vaak alleen geschikt voor gei\*:dealiseerde
modellen bijvoorbeeld met vereenvoudigde geometrie van het domein
of (stukwijs) constant veronderstelde fysische parameters die als
coefficienten in de modelvergelijkingen optreden.
Een voorbeeld van het laatste is de uniforme warmtediffusiviteit
$\kappa$ in bovenstaande Fourier oplossing.

In die gevallen waarin de wiskundige vergelijkingen niet analytisch
kunnen worden opgelost (tot evalueerbare functies worden herleid)
kan numeriek modelleren uitkomst bieden.
\end{itemize}

\noindent
Kenmerkende karakteristieken van numerieke methoden van modelleren zijn:
\begin{itemize}
\item
De oplossing wordt gezocht voor de onbekende veldgrootheden
(bijv. de temperatuur in de warmtevergelijking) in een discrete
verzameling punten in het oplossingsdomein (ruimte,tijd) - het rooster.
Dit wordt aangeduid als de domeindiscretisatie.
\item
De continue modelvergelijkingen worden m.b.v. discretisatie methoden
omgezet in stelsels algebraische vergelijkingen in de
onbekende roosterpunt waarden.
\item
De algebraische vergelijkingen worden numeriek opgelost voor gegeven
model parameters zoals begin/randvoorwaarden en coefficienten van
de vergelijkingen (bijvoorbeeld de diffusiviteit in de 
warmtevergelijking).
Het resultaat van de berekeningen is een lijst getallen corresponderend
met de roosterpuntwaarden van de onbekende grootheden.
Teneinde de resultaten beter te kunnen interpreteren worden zij
meestal nog in een nabewerkings (post processing) fase grafies 
weergegeven.
De numerieke berekeningen worden vanwege de grote omvang m.b.v. een
computer uitgevoerd.
\end{itemize}
\vspace{0.5cm}
\noindent
In dit college worden numerieke methoden behandeld die toepasbaar
zijn voor het modelleren van uiteenlopende geodynamische  processen.
Onder numeriek modelleren verstaan we hier:
het onderzoeken van modeleigenschappen, door de opgestelde 
modelvergelijkingen numeriek op te lossen voor verschillende waarden
van de modelparameters.
Technisch gezien komt dit neer op het herhaaldelijk oplossen van de
betreffende parti\"{e}le differentiaalvergelijkingen m.b.v.
numeriek wiskundige discretisatie methoden.

\noindent
Belangrijke geofysiche modellerings problemen waarvoor numerieke 
methoden worden toegepast zijn:
\begin{itemize}
\item (Energie/massa) transportmodellen
deze worden toegepast bij de bestudering van de warmtehuishouding in 
lithosfeermodellen of in stromingsmodellen toegepast op mantelconvectie
of in modellen van poreuze media.
Deze laatste modellen vinden op grote schaal toepassing in de
hydrologie en aanverwante takken van millieu engineering en
in de reservoir engineering i.v.m. de productie van olie en gas.
\item Stromingsproblemen: - visceuze vloeistof modellen toegepast
in modellen voor mantelconvectie en postglaciale opheffing.
\item Deformatie problemen: elastische en plastische deformatie 
modellen worden toegepast in modellen op uiteenlopende schaal van
gedetailleerde geologische modellen voor plooiende laagpakketten
tot grootschalige lithosfeermodellen met meerdere lithosferische
platen.
\end{itemize}

\noindent
De hier te behandelen (geo)fysische toepassingen hebben betrekking op
het numeriek oplossen van tijd- en plaatsafhankelijke vergelijkingen
in de vorm van gekoppelde parti\"{e}le differentiaalvergelijkingen.
In het algemeen wordt het probleem eerst in de ruimte gediscretiseerd.
Hierbij worden de ruimtelijke coordinaten als onafhankelijke variabelen
vervangen door over te gaan op een eindig aantal discrete 
vrijheidsgraden.
Hiermee vervangt men een parti\"{e}le differentiaal vergelijking door
een stelsel gewone differentiaalvergelijkingen met de tijd als enige
onfhankelijke variabele.
Deze stap wordt wel aangeduid als semi-discretisatie van het probleem.
Het stelsel gewone differentiaalvergelijkingen wordt dan vervolgens
in de tijd geintegreerd m.b.v. een numerieke integratie methode.
Bekende discretisatiemethoden zijn:
\begin{itemize}
\item eindige differentie- en eindige volumen methoden
\item eindige elementen methoden
\item spectraal methoden
\end{itemize}

\noindent
De beide eerste categori\"{e}n zullen hier worden behandeld.
Enkele integratiemethoden voor gewone differentiaalvergelijkingen
zijn:
\begin{itemize}
\item Euler expliciet/impliciet ook wel aangeduid als 
Euler voorwaarts/achterwaarts
\item Crank-Nicolson
\item Runge-Kutta
\end{itemize}

\noindent
De beide eerste methoden zullen hier worden toegepast.


%========================================================================================
%========================================================================================
%========================================================================================
%========================================================================================
%========================================================================================



%2.0-1
%==============================================================================
\section{Modelleren van 1-d warmte problemen}

Als eerste voorbeeld van een probleem dat met numerieke methoden
wordt gemodelleerd behandelen we het probleem van warmtediffusie
in een statisch (niet stromend) medium.
Dit probleem treedt o.a. op bij de bestudering van de
warmtehuishouding van de lithosfeer van de aarde.
In latere hoofdstukken zal aandacht worden besteed aan de behandeling
van het complexer probleem van warmtetransport in een stromend medium
dat optreedt bij de bestudering van thermische convectie in de 
aardmantel.

We beperken ons hier voor de overzichtelijkheid tot 1-D problemen.
1-D formulering van het warmteprobleem voor de lithosfeer is relevant 
omdat - op voldoende grote schaal - een horizontaal gelaagd model
toepasbaar is waarin het warmtetransport voornamelijk in
verticale richting plaatsvindt.
De te behandelen oplossingsmethoden zijn echter zonder meer uit te 
breiden naar twee en drie dimensies.
In hoofdstuk 3 worden vergelijkbare methoden behandeld voor 2-D 
problemen.
We gaan uit van de 1-D tijdafhankelijke warmtevergelijking.
\begin{equation}
     {\partial  T \over \partial t} \ =\ 
     \nabla \cdot \kappa \nabla T \ +\  Q \ =\ 
     {\partial  \over \partial z} \kappa
     {\partial  T \over \partial z} \ +\ Q
\end{equation}
met $T$ de temperatuur, $\kappa$ de warmtediffusiviteit,
$\kappa = k / \rho c_p$, $k$ de thermische conductiviteit,
$\rho$ de massadichtheid en $c_p$ de 
soortelijke warmte.

Een belangrijke toepassing van de 1-D vergelijking voor een statisch
medium is het halfruimtemodel voor afkoelende en spreidende oceaanlithosfeer
(Turcotte \& Schubert, 1982).
De advectieve term in de afgeleide naar de tijd in de warmtevergelijking
voor het (horizontaal) stromend medium met snelheid $u$,
$dT/dt = \partial T / \partial t + u \partial T / \partial x$
wordt hierin m.b.v. een ouderdomstranformatie $\tau = u/x$ omgezet
in $\partial T/ \partial \tau$.
Gaan we daarnaast uit van een stationaire toestand met 
$\partial T / \partial t = 0$ en verwaarlozen we horizontale
diffusie van warmte $\partial ^2 T / \partial x ^ 2 = 0$ dan
resulteert de tijdafhankelijke vergelijking met de leeftijd
$\tau$ in de rol van de tijdparameter.

\vspace{0.5cm}
\noindent
\bf Opgave:\rm
\newline
\small
%--------------- hier tekst tussenvoegen
Leid de bovenstaande warmte vergelijking af uit een behoudswet voor
thermische energie voor een 1-D statisch medium.
\normalsize
\vspace{0.5cm}

Het probleem wordt opgesteld voor een 1-D oplossingsdomein, 
het interval
$[z_0 , z_{N+1} ]$.
Voor het tijdafhankelijke  probleem veronderstellen we de temperatuur gegeven
voor $t = 0$, d.m.v. de beginvoorwaarde
\begin{equation}
T(z ,0) = T_b (z)
\end{equation}
Waarin $T_b (z)$ een bekende functie.
We zullen twee typen randvoorwaarden behandelen (Dirichlet/Neumann)
ook wel aangeduid als essenti\"{e}le/natuurlijke randvoorwaarden,
\begin{equation}
T(z_r ,t) = T_r (t)  \ ,\  type \ 1\ (Dirichlet)
\end{equation}
\begin{equation}
k \frac{\partial T(z_r ,t )}{\partial z} = 
q_r (t) \  ,\  type \ 2\ (Neumann)
\end{equation}
met $q_r$ de voorgeschreven warmtestroom dichtheid.
We behandelen twee verschillende discretisatie methoden.
De eerste, gebaseerd op een centrale differentie benadering
van de ruimtelijke afgeleiden in de warmtevergelijking, is beperkt
toepasbaar.
De tweede methode (eindige volume of box methode) is geschikt voor
meer algemene problemen, bijvoorbeeld bij een variabele 
geleidingscoefficient $k$.

Voor het stationaire geval $( \partial T / \partial t = 0)$
leiden beide discretisatie methoden tot een stelsel lineare 
algebraische vergelijkingen
dat m.b.v. een computer numeriek kan worden opgelost.
Voor het tijdafhankelijk geval $( \partial T / \partial t \neq  0)$
resulteert een stelsel gewone differentiaalvergelijkingen
met de tijd $t$ als onafhankelijke variabele.
In een later hoofdstuk zullen een aantal
integratiemethoden worden behandeld voor dit stelsel 
gewone differentiaalvergelijkingen.
%----------------------------------------------------------------
\subsection{Een differentiemethode met equidistant rooster}
We bespreken eerst een methode voor een equidistant rooster en met
uniforme diffusie coefficient $\kappa$.
Deze methode heeft als voordeel dat hij conceptueel eenvoudig is.
In volgende secties wordt een algemener toepasbare methode behandeld.

Definieer een 1-D rooster met vaste afstand tussen de roosterpunten
(equidistant):
\begin{equation}
z_i \ =\  z_0 + i \times \Delta z \ ,\  i\ =\ 0,1,2,3, ... , N+1
\end{equation}
met randpunten $z_0$ en $z_{N+1}$, gedeeltelijk weergegeven in
Fig.1.

%\begin{figure}[hb]
%\epsffile{fig1.eps}
%\caption{\it Onderdeel van een equidistant 1-D rooster met twee
%             rooster segmenten van lengte $\Delta z$}.
%\end{figure}

Definieer vectoren ${\bf T} (t)$, ${\bf Q} (t)$ 
met als elementen de roosterpuntwaarden
van de temperatuur 
$T_i (t) \ =\  T(z_i ,t)$
en de warmteproductiviteit
$Q_i (t) \ =\  Q(z_i ,t)$.
Voor het gemak 
laten we de aanduiding van de tijdafhankelijkheid weg in de volgende
formules.
Benader de afgeleide $\partial^2 T / \partial z^2$ 
in de warmtevergelijking m.b.v. een centrale differentie formule die
kan worden afgeleid d.m.v. een Taylor reeks ontwikkeling van de 
temperatuur,
\begin{equation}
T(z\ +\  \Delta z) \ =\  T(z) \ +\  
\Delta z {\partial T  \over \partial z} \ +\ 
{\Delta z^2  \over 2 }{\partial^2 T  \over \partial z^2} \ +\ 
{\Delta z^3  \over 6 }{\partial^3 T  \over \partial z^3} \ +\ 
{\Delta z^4  \over 24 }{\partial^4 T  \over \partial z^4} \ +\ 
 ...
\end{equation}
\vspace{0.5cm}
\begin{equation}
T(z\ -\  \Delta z) \ =\  T(z) \ -\  
\Delta z {\partial T  \over \partial z} \ +\ 
{\Delta z^2  \over 2 }{\partial^2 T  \over \partial z^2} \ -\ 
{\Delta z^3  \over 6 }{\partial^3 T  \over \partial z^3} \ +\ 
{\Delta z^4  \over 24 }{\partial^4 T  \over \partial z^4} \ -\ 
 ...
\end{equation}

\vspace{0.5cm}
\noindent
elimineer oneven machten van $\Delta z$:
\begin{equation}
T(z\ +\  \Delta z) \ -\  2T(z) \ +\  T(z- \Delta z ) \ =\ 
{\Delta z^2}         {\partial^2 T  \over \partial z^2} \ +\ 
{\Delta z^4  \over 12 }{\partial^4 T  \over \partial z^4} \ +\ 
 ...
\end{equation}

\vspace{0.5cm}
\noindent
\begin{equation}
{\partial^2 T  \over \partial z^2}
\ =\ 
{T(z\ -\  \Delta z) \ -\  2 T(z) \ +\  T(z\ +\  \Delta z)  \over \Delta z^2} 
\ -\ 
{\Delta z^2  \over 12 }{\partial^4 T  \over \partial z^4}
\ +\ 
 ...
\end{equation}

\vspace{0.5cm}
\noindent
Verwaarloozing in (9) van de termen in $\Delta z^2$, resulteert in een
discrete warmtevergelijking voor een roosterpunt $z_i$:
\begin{equation}
{\partial T_i  \over \partial t} \ =\ 
{\kappa \over \Delta z^2}
\left[
       T_{i-1} \ -\  2 T_i \ +\  T_{i+1}  \ 
\right]
\ +\ 
Q_i
\end{equation}
Bij gegeven temperatuur op de randen geldt:
$T (z_0 ) = T_0$,
$T (z_{N+1} ) = T_{N+1}$. 
De vergelijkingen voor het eerste $(i = 1)$ en laatste $(i = N)$
inwendige roosterpunt krijgen dan een rechterlid bijdrage in de 
voorgeschreven randtemperatuur $T_0$ en $T_{N+1}$.

\vspace{0.5cm}
\noindent
\begin{equation}
i \ =\  1 \ \ \to\ \ 
{\partial T_1  \over \partial t} 
\ =\ 
{\kappa \over \Delta z^2} 
\left[
       -2 T_1 \ +\  T_2 \ 
\right]
\ +\ 
{\kappa \over \Delta z^2} T_0 \ +\  Q_1
\end{equation}
\begin{equation}
i \ =\  N \ \ \to\ \ 
{\partial T_N  \over \partial t} 
\ =\ 
{\kappa \over \Delta z^2} 
\left[
       T_{N-1} \ -\  2 T_N \ 
\right]
\ +\ 
{\kappa \over \Delta z^2} T_{N+1} \ +\  Q_N
\end{equation}

\vspace{0.5cm}
\noindent
E.e.a. resulteert in
een stelsel van precies $N$ gewone differentiaal vergelijkingen
\begin{equation}
{\partial {\bf T}   \over \partial t}
\ =\ 
{\bf A} {\bf T} \ +\  {\bf R}
\end{equation}
met rechterlidvector:
\begin{equation}
{\bf R} \ =\  ( 
Q_1 \ +\  {\kappa \over \Delta z^2} T_0 ,
Q_2 ,
 ... ,
Q_{N-1} ,
Q_N \ +\  {\kappa \over \Delta z^2} T_{N+1}
)
\end{equation}
en $( N \times N )$ coefficienten matrix:
\begin{equation}
{\bf A} \ =\  - {\kappa \over  \Delta z^2}
\left[
\begin{array}{cccccccc}
 2  &-1 &  0 &  0  & . &  . &  . & . \\
-1  & 2 & -1 &  0  & . &  . &  . & . \\
 0  &-1 &  2 & -1  & . &  . &  . & . \\
 0  & 0 & -1 &  2  & . &  . &  . & . \\
 .  & . &  . &  .  & . &  . &  . & . \\
 .  & . &  . &  .  & . &  2 & -1 &  0\\
 .  & . &  . &  .  & . & -1 &  2 & -1\\
 .  & . &  . &  .  & . &  0 & -1 &  2 
\end{array}
\right]
\end{equation}
en onbekende vector 
\begin{equation}
{\bf T} (t) \ =\  ( T_1 (t), T_2 (t), ... , T_N (t))
\end{equation}
%----------------------------------------------------------------
\subsubsection{Implementatie van natuurlijke randvoorwaarden}
De behandeling van natuurlijke randvoorwaarden verschilt van die
voor essentiele randvoorwaarden in de vorige sectie.
We beschouwen een geval met voorgeschreven warmtestroomdichtheid
$q_{N+1}$ in het punt $p_{N+1}$.
De temperatuur in $p_{N+1}$ is nu ook een onbekende (vrijheidsgraad)
van het gestelde probleem en we moeten de discrete vergelijking (10)
ook in $p_{N+1}$ evalueren om een volledig stel 
$(N+1) \times (N+1)$  vergelijkingen te krijgen.

Een eenvoudige methode om de voorgeschreven warmtestroom te
implementeren is deze uit te drukken in de knooppuntwaarden d.m.v.
een differentie benadering.
We benaderen de temperatuurgradient in de uitdrukking voor de
warmtestroomdichtheid m.b.v. een centrale differentie
\begin{equation}
q_{N+1} = k {\partial T \over \partial z} |_{z_{N+1}}
=
k \frac {T_{N+2} - T_{N}} { 2 \Delta z}
+
O( \Delta z  ^2 )
\end{equation}
waarin we gebruik maken van een virtueel roosterpunt $p_{N+2}$ op
afstand $\Delta z$ buiten het oplossingsdomein.
Voor de temperatuur in het virtuele punt vinden we uit (17)
\begin{equation}
T_{N+2} = T_{N} + 2 \frac {q_{N+1}} {k} \Delta z
\end{equation}
M.b.v. (18) kan de discrete vergelijking in het randpunt $p_{N+1}$
worden geevalueerd resulterend in,
\begin{equation}
\frac {\partial T_{N+1}} {\partial t} = 
\frac {2 \kappa} {\Delta z^2}
     \left [
             T_N - T_{N+1}
     \right ]
+
\frac {2 \kappa} {\Delta z}
\frac {q_{N+1}} {k} 
+
Q_{N+1}
\end{equation}
We zien dat de inhomogene randvoorwaarde een extra bijdrage tot de
rechterlidvector levert.
Merk op dat 
dat de coefficienten matrix niet meer symmetrisch is.
De symmetrie is echter eenvoudig te herstellen door de vergelijking
voor het randpunt door 2 te delen.

\vspace{0.5cm}
\noindent
\bf Opgave:\rm
\newline
\small
Laat zien - door Taylor ontwikkeling - dat de gebruikte
randvoorwaarde inderdaad van de tweede orde is in $\Delta z$.
Ga na dat een meer voor de hand liggende voorwaartse differentie
in een eerste orde nauwkeurigheid resulteert.
\normalsize


%========================================================================================
%========================================================================================
%========================================================================================
%========================================================================================
%========================================================================================

\subsection{Een differentiemethode met variabele roosterafstand}

Bij de afleiding van de eindige differentie methode voor de 
warmtevergelijking m.b.v. de centrale differentie benadering voor
de tweede afgeleide term werd er vanuit gegaan dat het gebruikte rooster
een vaste afstand tussen de roosterpunten heeft (equidistant rooster). 
We namen bovendien aan dat de warmtegeleidingscoefficient uniform was.
Om voldoende nauwkeurigheid in de numerieke oplossing te bereiken kan
het soms noodzakelijk zijn om veel roosterpunten te gebruiken wat direct
resulteert in grotere rekentijd voor het computerprogramma en een 
groter beslag op het computergeheugen.
Deze roosterverfijning wordt bij de equidistante aanpak over het gehele
domein doorgevoerd terwijl deze in voorkomende gevallen uitsluitend op
deelgebieden wenselijk is - bijvoorbeeld daar waar de oplossing 
sterk varieert.

We behandelen hier een methode met variabele roosterafstand waarmee 
het rooster locaal kan worden verfijnd.
We zullen tegelijkertijd het algemene geval met variabele coefficient 
$k(z)$ bespreken.
We beperken ons weer tot het 1-D geval. Generalisatie tot 2-D en 3-D
toepassingen zijn direct mogelijk.
De hier te behandelen methode is bekend als de 
`eindige volume methode' of ook wel `box methode'.
We bespreken eerst het stationair probleem en zullen achteraf zien
hoe het resultaat kan worden uitgebreid voor tijdafhankelijke
problemen.

%--------------------------------------------------------------------
\subsubsection{Discretisatie van de vergelijking}
Bij de box methode wordt de parti\"{e}le differentiaalvergelijking
geintegreerd over het domein $I = [0,L]$,
De integratie wordt uitgevoerd over deelintervallen $I_i$
- de eindige volumen waaraan de methode zijn naam ontleent -
gecentreerd rond roosterpunt $p_i$,
$I_i = [z_{m_{i-1}} , z_{m_i} ]$,
zie Fig.2.

%\begin{figure}[hb]
%\epsffile{fig2.eps}
%Figuur 2:\ \it 
%Onderdeel van een 1-D rooster met twee rooster segmenten en een 
%integratie interval $I_i$.
%\end{figure}

Hierin zijn de $m_i$ de midpunten van de roosterintervallen.
Integratie van de stationaire warmtevergelijking over $I_i$ geeft
\begin{eqnarray}
\int_{z_{m_{i-1}}} ^{z_{m_i}}
    \left [
          \frac{d}{dz} k(z) 
          \frac{d}{dz} T
          \ +\ 
          H(z)
    \right ]
dz
\ =\ \nonumber \\
k(z) \frac {dT}{ dz} |_{z_{m_{i-1}}} ^{z_{m_i}}
\ +\ 
\int_{z_{m_{i-1}}} ^{z_{m_i}}
          H(z)
dz
= 0
\end{eqnarray}

Merk op dat de overgebleven afgeleide in (1) wordt geevalueerd in de
midpunten $m_{i-1}$ en $m_i$.
We benaderen vervolgens deze afgeleide in de midpunten door een 
centrale differentie in termen van de aangrenzende knooppunt waarden
en we definieren nog $k(z_{m_i}) \ =\ k_i$ en 
$z_{p_{i+1}}\ -\ z_{p_{i}}\ =\ h_i$.
\begin{equation}
k \frac {dT} {dz} |_{z_{m_i}}
\approx
k_i \frac { T_{i+1}     \ -\ T_i     }
          { z_{p_{i+1}} \ -\ z_{p_i} }
\ = \
k_i \frac { T_{i+1} \ -\ T_i} {h_i}
\end{equation}
\begin{equation}
k \frac {dT} {dz} |_{z_{m_{i-1}} }
\approx
k_{i-1} \frac { T_i     \ -\ T_{i-1}     } 
              { z_{p_i} \ -\ z_{p_{i-1}} }
\ = \
k_{i-1} \frac { T_i \ -\ T_{i-1}} {h_{i-1}}
\end{equation}
De warmteproduktie wordt bekend verondersteld en we definieren
\begin{equation}
\int_{z_{m_{i-1}}} ^{z_{m_i}} H(z) dz \ =\ F_i
\end{equation}
Invullen van (2),(3) en (4) in (1) levert,
\begin{equation}
- \frac {k_{i-1}} {h_{i-1}} T_{i-1}
\ +\ 
\left [
      \frac{k_{i-1}} {h_{i-1}}
      \ +\ 
      \frac{k_{i}} {h_{i}}
\right ]
T_i
\ -\ 
\frac{k_{i}}{h_{i}} T_{i+1}
= F_i
\end{equation}
Vergelijking (5) is een lineaire algebraische vergelijking in de 
onbekende roosterpuntenwaarden van de temperatuur $T_i$.
Door het beschreven integratie procede te herhalen voor alle 
- $N$ -
eindige volumen $I_i$ krijgen we een stelsel lineaire vergelijkingen.

\vspace{0.5cm}
\noindent
\bf Opgave:\rm
\newline
\small
Ga na dat het stelsel volledig is als we de temperatuur op de randen
van het interval $[0,L]$ voorschrijven m.b.v. twee essentiele 
randvoorwaarden.
Volledig betekent hier dat er evenveel vergelijkingen als onbekenden
zijn.
\normalsize
\vspace{0.5cm}

\vspace{0.5cm}
\noindent
\bf Opgave:\rm
\newline
\small
Laat zien dat het stelsel (5) voor het speciaal geval met een 
equidistant rooster en uniforme coefficient $k$ identiek is aan het
in de voorgaande sectie afgeleide stelsel 2.1(13).
Hoe verhouden de rechterlidvectoren zich ?
\normalsize
\vspace{0.5cm}

\noindent
Het stelsel vergelijkingen in matrix vorm geschreven is
\begin{equation}
\bf {A T } = \bf F
\end{equation}

\vspace{0.25in}
\noindent
\bf Opgave:\rm
\newline
\small
Toon aan dat de matrix $\bf A$ symmetrisch en tri-diagonaal is, d.w.z.
$A_{ij}\ =\ A_{ji}$, $A_{ij}\ =\ 0, |i\ -\ j| > 1$.
\normalsize

\vspace{0.25in}
\noindent
\bf Opgave:\rm
\newline
\small
Ga na dat de bovenstaande formulering ook nog toepasbaar is als we de
coefficient van de differentiaalvergelijking $k(z)$ definieren d.m.v.
een rij i.h.a. verschillende rooster segmentwaarden d.w.z. als we 
definieren,
\begin{equation}
k(z) \ =\ k_i ,\ z_{p_i} < z < z_{p_{i+1}}
\end{equation}
Hiermee kunnen gelaagde modellen met discontinuiteiten in de 
conductiviteit worden gemodelleerd.
\normalsize

\vspace{0.25in}
\noindent
\bf Opgave:\rm
\newline
\small
Als we de warmteproduktie coefficient $H(z)$ op dezelfde wijze 
definieren als in de geleidingscoefficient $k(z)$ in bovenstaande opgave
(stukwijs constant)
dan kan voor de rechterlidvector $\bf F$ in (4) worden afgeleid (ga na)
\begin{equation}
F_i \ =\ 
\int_{z_{m_{i-1}}} ^{z_{m_{i}}}
    H(z) dz
\ =\ 
\frac {h_{i-1}}{2} H(z_{m_{i-1}} ) \ +\ 
\frac {h_{i}}  {2} H(z_{m_{i}} )
\end{equation}
\normalsize

\vspace{0.25in}
\noindent
\bf Opgave:\rm
\newline
\small
Ga na: wanneer de warmteproduktie geconcentreerd is in een vlak
$z\ =\ z_s$, $H(z)\ =\ H_s \delta (z\ -\ z_s )$
met $z_{m_{k-1}}\ <\ z_s \ <\ z_{m_{k}}$ - d.w.z.
$z_s\ \in \ I_k$,
dan geldt voor de rechterlidvector in (4)
$F_i\ =\ H_s \delta_{ik}$
met $\delta_{ik}$ het Kronecker delta symbool.
\normalsize
%--------------------------------------------------------------------
\subsubsection{Behandeling van randvoorwaarden}
De behandeling van essentiele randvoorwaarden gaat op een vergelijkbare
manier als bij de discretisatie methode van sectie 2.1.
Essentiele randvoorwaarden geven een bijdrage tot de rechterlidvector
en ze reduceren het aantal vrijheidsgraden.
We behandelen hier de implementatie van een natuurlijke of Neumann 
randvoorwaarde die optreedt als de warmtestroomdichtheid op de
rand is voorgeschreven.
We stellen $k dT/dz |_{z=z_{N+1}} = q_{N+1}$.
De behandeling verschilt van de in sectie 2.1.1 voor een equidistant
rooster beschreven methode.
We werken de behandeling uit van de randvoorwaarde in roosterpunt
$p_{N+1}$ - zie Fig. 2.3.

%\begin{figure}[hb]
%\epsffile{fig3.eps}
%Figuur 3:\ \it 
%Onderdeel van een 1-D rooster met twee rooster segmenten, een 
%inwendig integratie interval $I_N$ en een randinterval
%$I_{N+1}$.
%\end{figure}

In dit geval is de temperatuur in het randpunt $p_{N+1}$ ook een
vrijheidsgraad.
Teneinde een volledig stel vergelijkingen te krijgen moet de 
discretisatie 2.2(1) ook in het roosterpunt $p_{N+1}$ geevalueerd
worden.
We werken de evaluatie uit voor de roosterpunten $p_{N}$ en $p_{N+1}$.

Voor $p_N$ krijgen we m.b.v. 2.2.1(5)
\begin{equation}
- \frac {k_{N-1}} {h_{N-1}} T_{N-1}
+ \left [
          \frac {k_{N-1}} {h_{N-1}}
          +
          \frac {k_{N  }} {h_{N  }}
  \right ]
T_{N}
- \frac {k_N} {h_N} T_{N+1}
= F_N
\end{equation}
De randtemperatuur $T_{N+1}$ treedt nu als vrijheidsgraad van het
probleem in het linkerlid op.

Voor het randpunt met voorgeschreven warmtestroomdichtheid $p_{N+1}$
gaan we uit van integraaluitdrukking 2.2(1) en we kiezen het
integratieinterval $I_{N+1} = [z_{m_N} , z_{p_{N+1}} ]$,
\begin{eqnarray}
\int_{z_{m_N}} ^{z_{p_{N+1}}}
\left [ \frac{d}{dz} k \frac{d}{dz} T + H \right ] dz
= \nonumber \\
k_N   \frac {dT}{dz} |_{z_{p_{N+1}}} -
k_N \frac {dT}{dz} |_{z_{m_{N  }}} +
\int_{z_{m_N}} ^{z_{{p_N+1}}} H(z) dz
\approx \nonumber \\
q_{N+1} - 
k_N \frac{ T_{N+1} - T_{N}} {h_N} + F_{N+1}
= 0
\end{eqnarray}
Waarmee we voor de vergelijking voor roosterpunt $p_{N+1}$ vinden
\begin{equation}
- \frac{k_N}{h_N} T_{N  }
+ \frac{k_N}{h_N} T_{N+1}
=
F_{N+1} +
q_{N+1}
\end{equation}
We zien dat de voorgeschreven warmtestroomdichtheid $q_{N+1}$
als extra bijdrage in de rechterlid vector verschijnt.




