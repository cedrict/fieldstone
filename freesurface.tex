\index{Free Surface}


When carrying out global models, i.e.  mantle convection in our case, the effect of the free surface
is often neglected/negligeable: topography ranges from $\sim$ 10km depth to $\sim$ 10km height, which 
is very small to the depth of the mantle ($\sim$ 3000km). 

However, it has long been regognised that there is a feedback between topography and crust/lithosphere
deformation: the surface of the Earth reflects the deeper processes, from orogeny, back-arc basins, 
rifts, mid-ocean ridges, etc ...

Free surface flows are not unique to Earth sciences, and their modelling has given rise to many studies 
and textbooks. A typical free-surface flow problem in the CFD literature is the so-called 'dam break' 
problem \cite{moeb99,bacp07,liir07,lemx08,homa09,anco09}. Other occurrences involve 
sea waves, flow over structures, flow around ships, mould filling, flow with bubbles \cite{liir07}.

 
What distinguishes geodynamics free surface modelling from its engineering 
counterpart is the absence of surface tension, the fact that the fluids under consideration are
Stokesian, and that their rheology is complex (the elastic and plastic components can be 
dominant at the surface).

%There are to main modelling approaches employed in Computational Geodynamics: the so-called 
%'sticky air' approach and the Arbitrary-Lagrangian approach.

%.......................................
\subsubsection{The fully Lagrangian approach}

In this case the mesh is deformed with the velocity (or displacement) computed on its nodes. 
It is sometimes called 'body fitting' \cite{crsg12} or 'boundary fitted'. 
In the case large deformation occurs (which is rather frequent in geodynamics - 
think about subduction or rifting processes where materials end up moving across the whole mesh),
it leads to highly deformed elements, and in some case even bow-tied. 
In the mildest cases it yields low accuracy calculations, especially when angles between edges 
become small or large. 
One way to overcome this problem is to remesh, i.e. generate a better mesh based on the 
available information on the deformed one. In 2D this is routinely done, especially when 
triangular elements are used. In 3D, multiple remeshing are very costly and it is generally
avoided.  
Note also that re-meshing often involves some form of interpolation and therefore some unwanted 
numerical diffusion. 
When deformation is reasonably small, fully lagrangian methods work and have been used in the 
geodynamics literature \cite{hach96b,mera80,labp00}.

\begin{center}
\includegraphics[width=7cm]{images/freesurface/labp00}\\
{\small Taken from \cite{labp00}. Note that the bottom and the top surface are deformed.}
\end{center}



GET:
Crook et al. 2006, and references therein \cite{crwy06})
Beaumont et al. 1994;  \cite{befh94}


%.......................................
\subsubsection{The Eulerian approach: using sticky air}
\index{Sticky Air}

Sticky air is the default option for numerical methods which mesh 
cannot be deformed (typically the finite difference method).
In this case, the air above the crust/sediments is modelled as a zero-density fluid with 
very low viscosity (see for instance the early article by Zaleski and Julien \cite{zaju92}). 
One problem quickly arises when one realises that the viscosity of the 
air ($\sim 18.5\cdot10^{-6}$ Pa$\cdot$s\footnote{\url{https://en.wikipedia.org/wiki/Viscosity}})
is almost 25-30 orders of magnitude lower than the (effective) viscosity of Earth materials. 
Real air viscosity cannot therefore be used because of 1) round-off errors, 2) extremely 
poorly-conditioned matrices. Low viscosities around $10^{16}-10^{19}$Pa$\cdot$s are then 
commonly used as they are still negligible next to those of the (plastic) crust, and the 
flow of air parallel to Earth materials only generates extremely small shear and normal stress values
(thereby approaching the true nature of a free surface). 
This approach is the one employed in all the papers based on the I2/I3(EL)VIS code (see Appendix~\ref{app:codes})
and has been benchmarked in Crameri et al. \cite{crsg12}.

This approach has a few advantages:
\begin{enumerate}
\item it is simple to implement 
\item it is compatible with all the standard numerical methods (FEM, FDM,FVM)
\item it avoids (potentially complicated) remeshing
\end{enumerate}
and quite a few drawbacks:
\begin{enumerate}
\item it increases the size of the computational domain, thereby adding more unknowns to the linear system: in \cite{scbe08} the air layer is set to 50km while the lithospheric domain underneath is 700km thick;
\item it requires the use of averaging all along the free-surface
where very large viscosity contrasts are present. Here is what Poliakov and Podlachikov \cite{popo92}
say about the sticky air method:
"Zaleski \& Julien \cite{zaju92} used a top layer with a very low
viscosity and density to represent air or water above the
surface. This allows a simple representation of the free
surface. However, due to the very high viscosity and density
contrast and diffusion between the top layer and the
underlying layers, calculations sometimes become unstable
and give significant errors."

\item it can showcase air entrainment:
\begin{center}
\includegraphics[width=8cm]{images/freesurface/scbe08}\\
{\small Taken from \cite{scbe08}. Details of the entrainment and lubrication of the soft surface layer. 
Light blue particles are sticky air particle and are found to greatly alter the viscosity
of the subduction channel.}
\end{center}
\item it is not clear how thick the air layer must be
\item it often requires to ascribe thermal parameters to the air;
\item it makes the implementation of Dirichlet or Neuman boundary conditions for temperature at the surface less
obvious.
\item it makes the coupling with surface processes codes less straightforward.
\item its accuracy depends on the method used to track materials in the rest of the code (markers, level sets, ...). If markers are used, the free surface position is then known up to the average distance between markers.
\item it negatively impacts the condition number of the matrix.
\end{enumerate}

The sticky air approach is employed by various codes in the subduction benchmark study \cite{scbe08}

The term 'sticky water' is sometimes employed too. The dynamic viscosity of water is about 
$10^{-3}$Pa$\cdot$s so that it is also negligible compared to the viscosity of Earth materials
and the same reasonng as air applies. However, in such a case a density of about 1000kg/m$^3$ 
is then assigned to the layer. REF?

In conclusion, as stated in \cite{crsg12}: "the sticky air method is a good way to
simulate a free surface for Eulerian approaches, provided that its
parameters are chosen carefully."


%..........................................................
\subsubsection{The Arbitrary Lagrangian Eulerian (ALE) approach}
\index{Arbitrary Lagrangian Eulerian} \index{ALE}

It is a very widely used approach in FEM-based geodynamics codes but originates in the field of 
CFD \cite{hiac74,hulz81} and is described at length in \cite{sozo01,dohp04,dohu03}.
To put is very simply, the key idea in the ALE formulation is
the introduction of a computational mesh which can move and deform with a velocity 
independent of the velocity carried by the material particles.

\paragraph{The simple approach in \cite{thie11}.}
What follows is written with a 2D Cartesian model in mind ($Q_1\times P_0$ elements are used).
The computational domain is a rectangle of size $L_x \times  L_y$ 
and a nnx $\times$ nny rectangular grid spanning the simulation
domain is generated.
The grid points constituting the top row of the grid define the
discrete free surface of the domain. Once the Eulerian velocity field
has been computed on these, their position is first updated using a
simple Eulerian advection step (see a,b on figure hereunder):
\[
\vec{r}_i'(t+\delta t) = \vec{r}_i(t) + \vec{v}_i \cdot \delta t
\qquad\qquad
i=1,\dots nnx
\]
The other boundaries of the system remain fixed at locations
$x=0$, $x=L_x$ and $y=0$. Even though the Eulerian grid must conform
to the current domain shape, only vertical motion of grid nodes is
allowed. It is therefore necessary to resample the predicted free
surface given by $\vec{r}_i'$ at equidistant positions between $x=0$ and $x=L_x$.
The resampling is carried out either with Spline functions or a 
moving least square algorithm. 
Finally, the vertical position of all the nodes corresponding
to column $i\in [1,nnx]$ is recalculated so that they are equidistant, 
as sketched in Figure d. This has the advantage of keeping
the mesh distortion to a minimum in the case of large
deformation.

\begin{center}
\includegraphics[width=9cm]{images/freesurface/ale2d}\\
 {\small The ALE algorithm of \cite{thie11} in 2D. 
(a) Grid and free surface at a given time $t$; 
(b) advection of the free surface; 
(c) resampling of the free surface at equidistant abscissae; 
(d) vertical adjustment of grid nodes in each column at equidistant ordinates.}
\end{center}

The ALE method is used in the SOPALE, SULEC, FANTOM, ELEFANT, and ASPECT codes to name a few
(see Appendix~\ref{app:codes}).

\paragraph{The not-so-simple but rather elegant approach of ASPECT}

What follows is mostly borrowed from Rose et al \cite{robh17}. Their approach 
has the advantage that it does not presuppose a geometry (Cartesian, Spherical, ...)
nor a number of dimensions. It is also designed to work in parallel and on octree-based
meshes, and with various combinations of boundary conditions.
Note that the authors specify that "for moderate mesh deformation, the mesh stays smooth and well
conditioned, though it breaks down for large deformations".

The mesh velocity in normal direction at the free surface (with
unit normal $\vec{n}$) has to be consistent with the velocity of the Stokes
velocity solution $\vec{\upnu}(t)$:
\begin{equation}
\vec{\upnu}_{\text{mesh}}(t)\cdot \vec{n} = \vec{\upnu}(t)\cdot \vec{n} 
\qquad
\text{on}
\quad
\Gamma_F
\end{equation}
In ALE calculations the internal mesh velocity is usually undetermined, 
but one wants to smoothly deform the mesh so as to preserve its regularity, 
avoiding inverted or otherwise poorly conditioned cells. 
The mesh deformation can be calculated in many different ways, icluding algebraic 
(as mentioned in the previous paragraph) and PDE based approaches.
The latter is chosen here. 
The Laplace equation is solved where the unknown in the mesh velocity, i.e. 
on must solve:
\begin{equation}
\Delta \vec{\upnu}_{\text{mesh}} = 0\label{eq:fsaspect1}
\end{equation}
subjected to the following boundary conditions:
\begin{eqnarray}
\vec{\upnu}_{\text{mesh}} &=& \vec{0} \qquad\qquad \text{on } \Gamma_0 \\
\vec{\upnu}_{\text{mesh}} &=& (\vec{\upnu}\cdot\vec{n})\vec{n} \qquad \text{on } \Gamma_F \\
\vec{\upnu}_{\text{mesh}}\cdot \vec{n} &=& 0 \qquad\qquad \text{on } \Gamma_{FS} \label{eq:fsaspect2}
\end{eqnarray}
where $\Gamma_{FS}$ is the part of the boundary with free slip boundary conditions, 
$\Gamma_0$ is the no-slip part and $\Gamma_{FS}$ is the free slip part.

Once the mesh velocity has been obtained for all mesh points, these can be moved with 
said velocity. However, it must be noted that the multiple occurences of the normal vector
in the above equations is not without problem as the normal vectors are not well defined on the
mesh vertices, which is where the mesh velocity is defined.

The authors list two simple methods of computing the normals:
\begin{itemize}
\item one can take $\vec{n}$ as the direction of the local vertical,
\item one could compute $\vec{n}$ as some weighted average of the cell normals adjacent to a given
vertex
\end{itemize}
but conclude that they have found that these schemes do not necessarily
have good mass conservation properties.

A better approach is proposed in the form of an $L_2$ projection of the 
normal velocity $\vec{v}\cdot\vec{n}$ onto the free surface $\Gamma_F$. 
Multiplying the boundary conditions 
\[
\vec{\upnu}_{\text{mesh}} = (\vec{\upnu}\cdot\vec{n})\vec{n} 
\]
by a test function $\vec{w}$ and integrating over the free surface part of the boundary, we find:
\begin{equation}
\int_{\Gamma_F} \vec{w}\cdot\vec{\upnu}_{\text{mesh}} d\Gamma 
=
\int_{\Gamma_F} \vec{w}\cdot (\vec{\upnu}\cdot\vec{n})\vec{n} d\Gamma
=
\int_{\Gamma_F} (\vec{w}\cdot\vec{n}) (\vec{\upnu}\cdot\vec{n}) d\Gamma \label{eq:fsaspect3}
\end{equation}
When discretized, this forms a linear system which can be
solved for the mesh velocity $\vec{\upnu}_{\text{mesh}}$ 
at the free surface. 
This system, being nonzero over only the
free surface, is relatively computationally inexpensive to solve.

This yields what the author coin the 'quasi-implicit' scheme 
(we have so far neglected any kind of stabilisation):
\begin{enumerate}
\item Solve the Stokes system;
\item Solve for the surface mesh velocity using Equation~\ref{eq:fsaspect3};
\item Solve for the internal mesh velocity using Equations~\ref{eq:fsaspect1}, \ref{eq:fsaspect2}; 
\item Advect the mesh forward in time using displacements determined by
the forward Euler scheme: $\vec{x}(t^{n+1} ) = \vec{x}(t^n ) + \vec{\upnu}_{\text{mesh}} \delta t$.
\end{enumerate}

Note that \cite{robh17} go further than this, propose a 'nonstandard finite difference scheme' and 
make a link with the stabilisation presented in \cite{kamm10}.

\vspace{1cm}
TOTAL WORK IN PROGRESS. Need to look at those papers:
\cite{huli88}
\cite{dumg11}
\cite{dumy16} 
\cite{anmp15}
\cite{krwd12}
\cite{stcl10}
\cite{maie12}

stabilisation \cite{kamm10,qube11,dumg11}
