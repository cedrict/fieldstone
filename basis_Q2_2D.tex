\begin{flushright} {\tiny {\color{gray} \tt basis\_Q2\_2D.tex}} \end{flushright}
%~~~~~~~~~~~~~~~~~~~~~~~~~~~~~~~~~~~~~~~~~~~~~~~~~~~~~~~~~~~~~~~~~~~~~~~~~~~~~~~~~~~~~~~~~~~~~~~~~~

This element is part of the so-called Lagrange family \cite{raki00}. 
Inside an element the local numbering of the nodes is as follows\footnote{I have adopted here 
a numbering scheme starting at zero! Also, it is a numbering among many other possible choices!}:

\input{tikz/tikz_q22d}

Note that this numbering is also employed in Li \cite[p56]{li06}.
The polynomial representation of the function $\phi$ over this element is then taken to be biquadratic:
\[
\phi^h(r,s) = a + br + cs + drs + er^2 + fs^2 + gr^2s + hrs^2 + i r^2s^2 = \sum_{i=0}^8 \bN_i(r,s) \phi_i
\]
and one can show that the basis functions are:
\begin{mdframed}[backgroundcolor=blue!5]
\begin{eqnarray}
\bN_0(r,s)&=& \frac{1}{2}r(r-1)  \frac{1}{2}s(s-1)\nonumber\\
\bN_1(r,s)&=& \frac{1}{2}r(r+1)  \frac{1}{2}s(s-1)\nonumber\\
\bN_2(r,s)&=& \frac{1}{2}r(r+1)  \frac{1}{2}s(s+1)\nonumber\\
\bN_3(r,s)&=& \frac{1}{2}r(r-1)  \frac{1}{2}s(s+1)\nonumber\\
\bN_4(r,s)&=&     (1-r^2)  \frac{1}{2}s(s-1)\nonumber\\
\bN_5(r,s)&=& \frac{1}{2}r(r+1)      (1-s^2)\nonumber\\
\bN_6(r,s)&=&     (1-r^2)  \frac{1}{2}s(s+1)\nonumber\\
\bN_7(r,s)&=& \frac{1}{2}r(r-1)      (1-s^2)\nonumber\\
\bN_8(r,s)&=&     (1-r^2)      (1-s^2)\nonumber
\end{eqnarray}
\end{mdframed}
Note that we have $\bN_i(r_j,s_j)=\delta_{ij}$ and then obviously $\bN_i(r_i,s_i)=1$. 

\begin{center}
\includegraphics[width=4cm]{images/basis_Q2_2D/N1}
\includegraphics[width=4cm]{images/basis_Q2_2D/N2}
\includegraphics[width=4cm]{images/basis_Q2_2D/N3}\\
\includegraphics[width=4cm]{images/basis_Q2_2D/N4}
\includegraphics[width=4cm]{images/basis_Q2_2D/N5}
\includegraphics[width=4cm]{images/basis_Q2_2D/N6}\\
\includegraphics[width=4cm]{images/basis_Q2_2D/N7}
\includegraphics[width=4cm]{images/basis_Q2_2D/N8}
\includegraphics[width=4cm]{images/basis_Q2_2D/N9}\\
{\captionfont Surface representation of the basis functions on the reference element.
{\color{gray} in images/basis\_Q2\_2D/ }}
\end{center}
Their derivatives are given by:
\begin{mdframed}[backgroundcolor=blue!5]
\begin{align}
\frac{\partial \bN_0}{\partial r}&= \frac{1}{2}(2r-1)  \frac{1}{2}s(s-1) & 
\frac{\partial \bN_0}{\partial s}&= \frac{1}{2}r(r-1)  \frac{1}{2}(2s-1)\nonumber\\
\frac{\partial \bN_1}{\partial r}&= \frac{1}{2}(2r+1)  \frac{1}{2}s(s-1) &
\frac{\partial \bN_1}{\partial s}&= \frac{1}{2}r(r+1)  \frac{1}{2}(2s-1)\nonumber\\
\frac{\partial \bN_2}{\partial r}&= \frac{1}{2}(2r+1)  \frac{1}{2}s(s+1) &
\frac{\partial \bN_2}{\partial s}&= \frac{1}{2}r(r+1)  \frac{1}{2}(2s+1)\nonumber\\
\frac{\partial \bN_3}{\partial r}&= \frac{1}{2}(2r-1)  \frac{1}{2}s(s+1) &
\frac{\partial \bN_3}{\partial s}&= \frac{1}{2}r(r-1)  \frac{1}{2}(2s+1)\nonumber\\
\frac{\partial \bN_4}{\partial r}&=       (-2r)  \frac{1}{2}s(s-1) &
\frac{\partial \bN_4}{\partial s}&=     (1-r^2)  \frac{1}{2}(2s-1)\nonumber\\
\frac{\partial \bN_5}{\partial r}&= \frac{1}{2}(2r+1)     (1-s^2)&
\frac{\partial \bN_5}{\partial s}&= \frac{1}{2}r(r+1)        (-2s)\nonumber\\
\frac{\partial \bN_6}{\partial r}&=       (-2r)  \frac{1}{2}s(s+1)&
\frac{\partial \bN_6}{\partial s}&=     (1-r^2)  \frac{1}{2}(2s+1)\nonumber\\
\frac{\partial \bN_7}{\partial r}&= \frac{1}{2}(2r-1)     (1-s^2)&
\frac{\partial \bN_7}{\partial s}&= \frac{1}{2}r(r-1)        (-2s)\nonumber\\
\frac{\partial \bN_8}{\partial r}&=       (-2r)     (1-s^2)&
\frac{\partial \bN_8}{\partial s}&=     (1-r^2)        (-2s)\nonumber
\end{align}
\end{mdframed}
These basis functions are used for example in \stone 18.
