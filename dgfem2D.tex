We start from the steady state (no advection) energy equation:
\[
\vec\nabla\cdot (k \vec\nabla T) + Q = 0
\]
where $Q$ is a source term.
As we did in 1D we tranform this equation by reintroducing the heat flux
\footnote{\url{https://en.wikipedia.org/wiki/Thermal_conduction}} $\vec{q}$ 
\begin{eqnarray}
\vec q &=& - k \vec\nabla T \\
\vec\nabla \cdot \vec q &=& Q
\end{eqnarray}
We then multiply these two equations with the test functions $\vec{N}_q$ (this one 
is vector valued) and 
$N_T$ respectively and integrate over the element $e$ under consideration:
\begin{eqnarray}
\int_{\Omega_e} \vec{N}_q \cdot  \vec q  \; dV &=& - \int_{\Omega_e} \vec{N}_q \cdot k \vec\nabla T\; dV \\
\int_{\Omega_e} N_T \vec\nabla \cdot \vec q \;  dV &=& \int_{\Omega_e} N_T Q \; dV
\end{eqnarray}
We then use integration by parts to make fluxes appear (and assume $k$ is constant within 
the element). We have
\begin{eqnarray}
\int_{\Omega_e} \vec{N}_q \cdot k \vec\nabla T \; dV 
&=& \int_{\Gamma_e} kT \;  \vec{N}_q \cdot \vec{n} \; dS 
- \int _{\Omega_e}  kT \;  \vec\nabla\cdot \vec{N}_q  \;  dV \\
\int_{\Omega_e} N_T \vec\nabla \cdot \vec q \; dV 
&=& \int_{\Gamma_e} {N}_T \vec{q} \cdot \vec{n}  \; dS
- \int_{\Omega_e} \vec\nabla N_T \cdot \vec q \; dV 
\end{eqnarray}
We then finally obtain the weak forms that are to be discretised:
\begin{eqnarray}
\int_{\Omega_e} \vec{N}_q \cdot  \vec q \; dV &=& 
-\int_{\Gamma_e} kT \;  \vec{N}_q \cdot \vec{n} \; dS 
+ \int _{\Omega_e}  kT \;  \vec\nabla\cdot \vec{N}_q  \;  dV \\
\int_{\Gamma_e}    {N}_T  \vec{q}\cdot \vec{n}  \; dS
- \int_{\Omega_e} \vec\nabla N_T \cdot \vec q \; dV 
&=& \int_{\Omega_e} N_T Q \; dV
\end{eqnarray}

We seek to approximate the exact solution $(\vec{q},T)$ with functions $(\vec{q}_h,T_h)$ inside the element so we 
now have 
\begin{eqnarray}
\int_{\Omega_e} \vec{N}_q \cdot  \vec q_h \; dV &=& 
-\int_{\Gamma_e} k \hat{T}_h \;  \vec{N}_q \cdot \vec{n} \; dS 
+ \int _{\Omega_e}  k T_h \;  \vec\nabla\cdot \vec{N}_q  \;  dV \\
\int_{\Gamma_e}    {N}_T \hat{\vec{q}}_h \cdot \vec{n}  \; dS
- \int_{\Omega_e} \vec\nabla N_T \cdot \vec{q}_h \; dV 
&=& \int_{\Omega_e} N_T Q \; dV
\end{eqnarray}
where the numerical fluxes are the approximations to $(\vec{q},T)$ on the boundary of element $e$.
As before we must now specify these fluxes inside the domain and also on the boundaries. These
can obviously not be chosen at will since they must at least render the discontinuous formulation stable.

The following table is taken from Li \cite{li06} and lists the numerical fluxes 
that are considered consistent and stable for the solution of the steady state heat conduction 
problems \cite{arbc02,cacp00}:
\begin{center}
\begin{tabular}{lll}
\hline
Method & $\hat{\vec{q}}_h$ & $\hat{T}_h$ \\
\hline
LDG \cite{cosh98} & $\{ \vec{q}_h \}$  & \\
DG  \cite{cacp00} & \\
Brezzi et al (2000) \cite{brmm00} &\\
IP \cite{dodu76} & \\
Bassi-Rebay \cite{barm97} &\\
NIPG \cite{riwg99} & \\
\hline
\end{tabular}
\end{center}
where the $C$ coefficients are matrices ?!

I NEED a DRAWING HERE

\index{general}{Average Operator}
\index{general}{Jump Operator}
The average operator and jump operators in the table are defined as follows:
let $\Gamma_{12}$ be an interior edge shared by elements $1$ and $2$ and let us 
define the unit normal vectors $\vec{n}_1$ and $\vec{n}_2$ on $\Gamma_{12}$ 
pointing exterior to element 1 and 2, respectively.

In what follows we adopt the 'DG' formulation (which seems to be the most versatile one
since its fluxes can be modified by zeroing some coefficients to obtain other formulations)
so that:
\begin{eqnarray}
\int_{\Omega_e} \vec{N}_q \cdot  \vec q_h dV &=& 
-\int_{\Gamma_e} k \left[    \right] \;  \vec{N}_q \cdot \vec{n} \; dS 
+ \int _{\Omega_e}  k T_h \;  \vec\nabla\cdot \vec{N}_q  \;  dV \\
\int_{\Gamma_e}    {N}_T \left[   \right] \cdot \vec{n}  \; dS
- \int_{\Omega_e} \vec\nabla N_T \cdot \vec{q}_h \; dV 
&=& \int_{\Omega_e} N_T Q dV
\end{eqnarray}


\begin{eqnarray}
\{ \vec{q} \} &=& \frac{1}{2} (\vec{q}_1 + \vec{q}_2) \\
\llbracket \vec{q} \rrbracket  &=& \vec{q}_1 \cdot\vec{n}_1 + \vec{q}_2\cdot\vec{n}_2 \\
\{ T \} &=& \frac{1}{2} (T_1 + T_2) \\
\llbracket T \rrbracket &=& T_1\cdot\vec{n}_1 + T_2\cdot\vec{n}_2 
\end{eqnarray}
Note that the jump $ \llbracket \vec{q} \rrbracket$ is a scalar while 
$\llbracket T \rrbracket $ is actually a vector. 

Inside the element the functions $T_h$ and $\vec{q}_h$ are represented by means 
of the shape functions such that
\[
T_h(\vec{r}) = \sum_{k=1}^m N_k(\vec{r}) T_k = \vec{N} \cdot \vec{T}
\]
\[
\vec{q}_x|_h(\vec{r}) = \sum_{k=1}^m N_k(\vec{r}) q_x|k = \vec{N}\cdot\vec{q}_x
\]
\[
\vec{q}_y|_h(\vec{r}) = \sum_{k=1}^m N_k(\vec{r}) q_y|k = \vec{N}\cdot\vec{q}_y
\]

