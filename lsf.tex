\begin{flushright} {\tiny {\color{gray} lsf.tex}} \end{flushright}
%~~~~~~~~~~~~~~~~~~~~~~~~~~~~~~~~~~~~~~~~~~~~~~~~~~~~~~~~~~~~~~~~~~~~~~~~~~~~~~~~~~~~~~~~~~~~~~~~~~

This method was developed in the 80's by Stanley Osher and James Sethian \cite{lofo06}

The Level-set Method (LSM), as it is commonly used in Computational Fluid Dynamics -- and especially 
in Computational Geodynamics -- represents a close curve $\Gamma$ (say, in our case, the 
interface between two fluids or layers) by means of a function $\phi$ (called the level-set function, or LSF).
$\Gamma$ is then the zero level-set of $\phi$:
\begin{equation}
\Gamma = \left\{ (x,y) \; |\; \phi(x,y)=0 \right\}
\end{equation}
The convention is that $\phi>0$ inside the region delimited by $\Gamma$ and $\phi<0$ outside.
The function value indicates on which side of the
interface a point is located (negative or positive) and this is
used to identify materials. 

Furthermore, if the curve $\Gamma$ moves with a velocity $\vec \upnu$, 
then it satisfies the following equation:
\begin{equation}
\frac{\partial \phi}{\partial t} + \vec\upnu \cdot \vec\nabla \phi = 0 
\end{equation}

The level set function is generally chosen to
be a signed distance function, i.e. $|\vec\nabla \phi| = 1$ everywhere 
and its value is also the distance to the interface.
The function value indicates on which side of the interface a
point is located (negative or positive) and this is used to identify materials.

As explained in \cite{hitg14}, the level-set function $\phi$ is advected 
with the velocity $\vec\upnu$ which is obtained by solving the Stokes equations.
This velocity does not guarantee that after an advection step the signed 
distance quality of the LSF is preserved. 
The LSF then needs to be corrected, which is also called reinitialisation. 
Finally, solving the advection equation must be done in an accurate manner both in time and space,
so that so-called ENO (essentially non-oscillatory) schemes are often employed for the 
space derivative \cite{ossh91,saev10}.


The level set method has not often been used in the geodynamics 
community with some notable exceptions.
Bourgouin et al use this method combined with Finite Differences to model 
lava flows \cite{bomh06,bomh07,habm07,grbh07}.
Braun \etal use a so-called particle based level set methodology in their 
FEM code in conjunction with AMR \cite{brtf08}.
Zlotnik \etal coupled the X-FEM method with level set functions to model 
slab break-off and Rayleigh-Taylor Diapirism \cite{zlfd08}.
This same particle level sets are studied by Samuel and Evonuk and applied to geophysical flows \cite{saev10}. 
In Suckale \etal (2010) \cite{sunh10,suhe10} the authors investigate simulating buoyancy‐driven flow in the presence of large viscosity contrasts.
Hale \etal (2010) \cite{hagr10} use the LSM in 3D and sudy the dynamics of slab tear faults.
An overview of the method and applications can
be found in \cite{osfe01}.

Several improvements upon the original LSM have been proposed, 
such as for instance the conservative level set of \cite{zhbl14}.
The most notable difference between CLS method originally proposed by Olsson \etal \cite{olkr05,olkz07}
and standard LS method lies in the choice of LS function. Instead of the signed distance function, the
CLS methods employ the Heaviside function $H(\phi)$ 
\[
H(\phi)=
\left\{
\begin{array}{ll}
1 & \phi>0 \\
1/2 & \phi=0 \\
0 & \phi<0
\end{array}
\right.
\]
where $\phi$ is the signed distance function as in the LSM. 
In practice, a hyperbolic tangent function is used:
\[
H(\phi) = \frac{1}{2} (1+\tan (\phi/2\epsilon))
\]
where $\epsilon$ defines the spreading width of $H$. In the case where there are only 
two fluids (i.e. a single level set is sufficient), the material properties such as density and viscosity
are computed as follows:
\[
\rho=\rho_1+(\rho_2-\rho_1)H(\phi)
\]
\[
\eta=\eta_1+(\eta_2-\eta_1)H(\phi)
\]

\Literature: \cite{vasv05,vasv08,migi07,vasv05b}. 
\begin{itemize}
\item Review of level-set methods \cite{gifo18}
\item Interactive 3-D computation of fault surfaces using level sets \cite{kadt08}
\end{itemize}





