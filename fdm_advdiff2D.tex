
So far, we have mainly focused on the diffusion equation in a non-moving flow 
(relevant for the case of a dike intrusion cooling off 
or for a lithosphere which remains undeformed). 

We now want to consider problems where material moves during the time period under 
consideration and takes temperature anomalies with it (e.g. a plume rising 
through a convecting mantle). 
If the numerical grid remains fixed in the background, the hot temperatures should 
be moved to different grid points at each time step. 

We start again from the heat transport equation of Section~\ref{ss:hte}:
\begin{equation}
\rho C_p \left( \frac{\partial T}{\partial t} + \vec\upnu \cdot \vec\nabla T  \right)=
\vec\nabla \cdot k \vec\nabla T + Q 
\end{equation}
We have previously dealt with the one-dimensional Cartesian coordinates equation:
\begin{equation}
\rho C_p \left( \frac{\partial T}{\partial t}  
+ u \frac{\partial T}{\partial x} \right)= 
\frac{\partial }{\partial x} \left(  k  \frac{\partial T}{\partial x} \right)+ Q
\end{equation}
and we now turn to the two-dimensional equation:
\begin{equation}
\rho C_p \left( \frac{\partial T}{\partial t}   
+ u \frac{\partial T}{\partial x}  
+ v \frac{\partial T}{\partial y} \right) 
=
\frac{\partial }{\partial x} \left(  k  \frac{\partial T}{\partial x} \right)
+
\frac{\partial }{\partial y} \left(  k  \frac{\partial T}{\partial y} \right)
+Q
\end{equation}
As before, asumming that $k$ is constant in space we can rewrite the equation as 
a function of the heat diffusivity $\kappa$:
\begin{equation}
\frac{\partial T}{\partial t}   
+ u \frac{\partial T}{\partial x}  
+ v \frac{\partial T}{\partial y} 
=
\kappa \left( 
\frac{\partial^2 T}{\partial x^2} 
+ \frac{\partial^2 T}{\partial y^2} \right) +Q
\end{equation}
Since we have already seen how to deal with 'pure' diffusion equations in the 
previous section, let us now turn to 'pure' advection equations:

\begin{equation}
\frac{\partial T}{\partial t}  + u \frac{\partial T}{\partial x} = 0
\end{equation}
or
\begin{equation}
\frac{\partial T}{\partial t}  + u \frac{\partial T}{\partial x} + v \frac{\partial T}{\partial y}= 0
\end{equation}
where we assume $\vec\upnu=(u,v)$ known. 

Even though the equations appear simple, it is quite tricky to solve them accurately, 
more so than for the diffusion problem. 
This is particularly the case if there are large gradients in the quantity that is to be advected. 

We have seen how to deal with the time derivative (explicit, implicit) 
and with the first order space derivative (forward, backward or central).
Let us consider again the FTCS scheme (Forward in Time, Central in Space).
\begin{equation}
\frac{T_{\color{teal}i,j}^{n+1}-T^n_{\color{teal}i,j}}{\delta t} 
+ u_{i,j} \frac{T^n_{\color{teal}i+1,j} - T^n_{{\color{teal}i-1,j}}}{2h_x} 
+ v_{i,j} \frac{T^n_{\color{teal}i,j+1} - T^n_{{\color{teal}i,j-1}}}{2h_y} =0 
\end{equation}
The fully implicit version is then as follows:
\begin{equation}
\frac{T_{\color{teal}i,j}^{n+1}-T^n_{\color{teal}i,j}}{\delta t} 
+ u_{i,j} \frac{T^{n+1}_{\color{teal}i+1,j} - T^{n+1}_{{\color{teal}i-1,j}}}{2h_x} 
+ v_{i,j} \frac{T^{n+1}_{\color{teal}i,j+1} - T^{n+1}_{{\color{teal}i,j-1}}}{2h_y} =0 
\end{equation}
or, 
\[
T_{\color{teal}i,j}^{n+1}   
+\frac{u_{i,j} \delta t}{2 h_x}
(  T^{n+1}_{\color{teal}i+1,j} - T^{n+1}_{{\color{teal}i-1,j}}  )
+\frac{v_{i,j} \delta t}{2 h_y}
(  T^{n+1}_{\color{teal}i,j+1} - T^{n+1}_{{\color{teal}i,j-1}} )
=
T^n_{\color{teal}i,j}
\]
The terms on the left will form five diagonals in the matrix while the term on the right 
is the right hand side.

The Crank-Nicolson approach is then easily derived by taking:
\begin{equation}
\frac{T_{\color{teal}i,j}^{n+1}-T^n_{\color{teal}i,j}}{\delta t} 
+\frac{u_{i,j}}{2} \frac{T^{n}_{\color{teal}i+1,j} - T^{n}_{{\color{teal}i-1,j}}}{2h_x} 
+\frac{v_{i,j}}{2} \frac{T^{n}_{\color{teal}i,j+1} - T^{n}_{{\color{teal}i,j-1}}}{2h_y} 
+\frac{u_{i,j}}{2} \frac{T^{n+1}_{\color{teal}i+1,j} - T^{n+1}_{{\color{teal}i-1,j}}}{2h_x} 
+\frac{v_{i,j}}{2} \frac{T^{n+1}_{\color{teal}i,j+1} - T^{n+1}_{{\color{teal}i,j-1}}}{2h_y} 
=0 
\end{equation}




%-/-/-/-/-/-/-/-/-/-/-/-/-/-/-/-/-/-/-/-/-/-/-/-/
\begin{center}
\begin{minipage}[t]{0.77\textwidth}
\par\noindent\rule{\textwidth}{0.4pt}
\begin{center}
\includegraphics[width=0.8cm]{images/garftr} \\
{\color{orange} Exercise FDM-11}
\end{center}

We wish to compute the advection of a product-cosine hill
in a prescribed velocity field. The initial temperature is:
\[
T_0(x,y)=
\left\{
\begin{array}{cc}
\frac{1}{4}
\left(1+\cos \pi\frac{x-x_c}{\sigma}\right)
\left(1+\cos \pi\frac{y-y_c}{\sigma}\right)
& \text{if } (x-x_c)^2+(y-y_c)^2\leq \sigma^2 \\
0 & \text{otherwise}
\end{array} 
\right.
\]
The boundary conditions are $T(x,y)=0$ on all four sides of the unit square domain. 
In what follows we set $x_c=y_c=2/3$ and $\sigma=0.2$.  
The velocity field is analytically prescribed: $\vec\upnu=(-(y-L_y/2),+(x-L_x/2))$.
Resolution is set to $31\times31$ nodes.

The timestep is set to $\delta t=2\pi/200$ and we wish to carry out 200
timesteps so that the cone does a $2\pi$ rotation.

See Stone 43 for results/figures of this experiment obtained 
with Finite Elements.

Implement this with the FTCS method. What do you observe? What happens when you decrease 
the value of $\delta t$? 

Bonus: Lax method, Crank-Nicolson method.

\par\noindent\rule{\textwidth}{0.4pt}
\end{minipage}
\end{center}

\vspace{1cm}

%-/-/-/-/-/-/-/-/-/-/-/-/-/-/-/-/-/-/
\begin{center}
\begin{minipage}[t]{0.77\textwidth}
\par\noindent\rule{\textwidth}{0.4pt}

\begin{center}
\includegraphics[width=0.8cm]{images/garftr} \\
{\color{orange}Exercise FDM-12}
\end{center}

NOT FOR 2020! 

Redo exercise FDM-6 in a unit square domain. 
The temperature field at $t=0$ is 
given by $T(x,y)=1$ for $x<0.25$ and $T(x,y)=0$ otherwise. The prescribed 
velocity is $\vec\upnu=(1,0)$ and we set $nnx=nny=51$.
Boundary conditions are $T=1$ at $x=0$ and $T=0$ at $x=1$.

Program the above FTCS method. Run the model for 250 time steps with $\delta t=0.002$. 
Compare the 2D solution with the previously obtained 1D solution of exercise FDM-6.

Make sure the code works in the $y$-direction too by rotating the initial temperature 
by $90\degree$ anti-clockwise, set $\vec{\upnu}=(0,1)$ and change boundary conditions accordingly. 

Bonus: Lax method, Crank-Nicolson method.

\par\noindent\rule{\textwidth}{0.4pt}
\end{minipage}
\end{center}

Note to self:
- CFL missing still
- stensils next to boundaries missing too
- add many more visual aids










