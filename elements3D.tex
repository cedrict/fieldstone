\begin{flushright} {\tiny {\color{gray} elements3D.tex}} \end{flushright}

%%%%%%%%%%%%%%%%%%%%%%%%%%%%%%%%%%%%%%%%%%%%%%%%%%%%%%%%%%%%%%%%%%%%%%%%%%%%%%%
\subsection{Linear basis functions in tetrahedra ($P_1$)}
\index{general}{$P_1$}

\input{basis_p1_3D}

%%%%%%%%%%%%%%%%%%%%%%%%%%%%%%%%%%%%%%%%%%%%%%%%%%%%%%%%%%%%%%%%%%%%%%%%%%%%%%%
\subsection{Enriched linear in tetrahedra($P_1^+$)}
\index{general}{$P_1^+$}

\input{basis_p1p_3D}

%%%%%%%%%%%%%%%%%%%%%%%%%%%%%%%%%%%%%%%%%%%%%%%%%%%%%%%%%%%%%%%%%%%%%%%%%%%%%%%
\subsection{Triquadratic basis functions in 3D ($Q_2$)}
\index{general}{$Q_2$}

\begin{flushright} {\tiny {\color{gray} basis\_q2\_3D.tex}} \end{flushright}
%~~~~~~~~~~~~~~~~~~~~~~~~~~~~~~~~~~~~~~~~~~~~~~~~~~~~~~~~~~~~~~~~~~~~~~~~~~~~~~~~~~~~~~~~~~~~~~~~~~

\begin{center}
\input{tikz/tikz_q2}
\end{center}
Note that the figure above depicts an element in $x,y,z$ space, not the reference
element in $r,s,t$ space which is centered around the origin.

\begin{eqnarray}
\bN_{1} &=& 0.5r(r-1)  \cdot 0.5s(s-1) \cdot 0.5t(t-1) \nn\\
\bN_{2} &=& (1-r^2)    \cdot 0.5s(s-1) \cdot 0.5t(t-1) \nn\\
\bN_{3} &=& 0.5r(r+1)  \cdot 0.5s(s-1) \cdot 0.5t(t-1) \nn\\
\bN_{4} &=&  0.5r(r-1) \cdot (1-s^2)   \cdot 0.5t(t-1) \nn\\
\bN_{5} &=&  (1-r^2)   \cdot (1-s^2)   \cdot 0.5t(t-1) \nn\\
\bN_{6} &=& 0.5r(r+1)  \cdot (1-s^2)   \cdot 0.5t(t-1) \nn\\
\bN_{7} &=&  0.5r(r-1) \cdot 0.5s(s+1) \cdot 0.5t(t-1) \nn\\
\bN_{8} &=&  (1-r^2)   \cdot 0.5s(s+1) \cdot 0.5t(t-1) \nn\\
\bN_{9} &=& 0.5r(r+1)  \cdot 0.5s(s+1) \cdot 0.5t(t-1) \nn\\
\bN_{10}&=&  0.5r(r-1) \cdot 0.5s(s-1) \cdot (1-t^2) \nn\\
\bN_{11}&=&  (1-r^2)   \cdot 0.5s(s-1) \cdot (1-t^2) \nn\\
\bN_{12}&=& 0.5r(r+1)  \cdot 0.5s(s-1) \cdot (1-t^2) \nn\\
\bN_{13}&=&  0.5r(r-1) \cdot (1-s^2)   \cdot (1-t^2) \nn\\
\bN_{14}&=&  (1-r^2)   \cdot (1-s^2)   \cdot (1-t^2) \nn\\
\bN_{15}&=& 0.5r(r+1)  \cdot (1-s^2)   \cdot (1-t^2) \nn\\
\bN_{16}&=&  0.5r(r-1) \cdot 0.5s(s+1) \cdot (1-t^2) \nn\\
\bN_{17}&=&  (1-r^2)   \cdot 0.5s(s+1) \cdot (1-t^2) \nn\\
\bN_{18}&=& 0.5r(r+1)  \cdot 0.5s(s+1) \cdot (1-t^2) \nn\\
\bN_{19}&=&  0.5r(r-1) \cdot 0.5s(s-1) \cdot 0.5t(t+1) \nn\\
\bN_{20}&=&  (1-r^2)   \cdot 0.5s(s-1) \cdot 0.5t(t+1) \nn\\
\bN_{21}&=& 0.5r(r+1)  \cdot 0.5s(s-1) \cdot 0.5t(t+1) \nn\\
\bN_{22}&=&  0.5r(r-1) \cdot (1-s^2)   \cdot 0.5t(t+1) \nn\\
\bN_{23}&=&  (1-r^2)   \cdot (1-s^2)   \cdot 0.5t(t+1) \nn\\
\bN_{24}&=& 0.5r(r+1)  \cdot (1-s^2)   \cdot 0.5t(t+1) \nn\\
\bN_{25}&=&  0.5r(r-1) \cdot 0.5s(s+1) \cdot 0.5t(t+1) \nn\\
\bN_{26}&=&  (1-r^2)   \cdot 0.5s(s+1) \cdot 0.5t(t+1) \nn\\
\bN_{27}&=& 0.5r(r+1)  \cdot 0.5s(s+1) \cdot 0.5t(t+1) \nn
\end{eqnarray}








%%%%%%%%%%%%%%%%%%%%%%%%%%%%%%%%%%%%%%%%%%%%%%%%%%%%%%%%%%%%%%%%%%%%%%%%%%%%%%%
\subsection{Enriched quadratic basis functions in tetrahedra ($P_2^+$)}
\index{general}{$P_2^+$}

\input{basis_p2p_3D}


%%%%%%%%%%%%%%%%%%%%%%%%%%%%%%%%%%%%%%%%%%%%%%%%%%%%%%%%%%%%%%%%%%%%%%%%%%%%%%%
%.....................................................................
\subsection{Linear basis functions for hexahedra ($P_1$)} \label{ss:lbfh3D}
\index{general}{$P_1$}

This is the ${\bm Q}_2\times P_{-1}$ element. 
I choose the reduced coordinates of the pressure nodes to be :

\begin{tabular}{cccc}
\hline
point & r & s & t \\
\hline
1& 1/2 &-1/2 &-1/2\\
2& -1/2 &1/2 &-1/2\\
3& -1/2 &-1/2& 1/2\\
4& 1/2 &1/2& 1/2 \\
\hline
\end{tabular}

Inside the element the pressure is given as a linear function of the reduced coordinates $r,s,t$:
\[
p(r,s,t)=a+br+cs+dt
\]
This expression must exactly interpolate the pressure at all four pressure nodes:
\begin{eqnarray}
p_1 
&=& p(r_1,s_1,t_1) 
= a+br_1+cs_1+dt_1 
= a+b/2-c/2-d/2\nonumber\\
p_2
&=& p(r_2,s_2,t_2)
= a+br_2+cs_2+dt_2
= a-b/2+c/2-d/2\nonumber\\
p_3
&=& p(r_3,s_3,t_3) 
= a+br_3+cs_3+dt_3 
= a-b/2-c/2+d/2\nonumber\\
p_4
&=& p(r_4,s_4,t_4) 
= a+br_4+cs_4+dt_4
= a+b/2+c/2+d/2\nonumber
\end{eqnarray}
or,
\begin{equation}
\left(
\begin{array}{cccc}
1 & 1/2 & -1/2 & -1/2 \\
1 & -1/2 & +1/2 & -1/2 \\
1 & -1/2 & -1/2 & +1/2 \\
1 & 1/2 & +1/2 & +1/2 
\end{array}
\right)
\left(
\begin{array}{c}
a\\b\\c\\d
\end{array}
\right)=
\left(
\begin{array}{c}
p_1\\p_2\\p_3\\p_4
\end{array}
\right)
\nonumber
\end{equation}

The matrix is invertible and we get:
\[
\left(
\begin{array}{c}
a\\b\\c\\d
\end{array}
\right)=
\left(
\begin{array}{cccc}
1/4 & 1/4 & 1/4 & 1/4 \\
1/2 & -1/2 & -1/2 & 1/2 \\
-1/2 & 1/2 & -1/2 & 1/2 \\
-1/2 & -1/2 & 1/2 & 1/2
\end{array}
\right)
\left(
\begin{array}{c}
p_1\\p_2\\p_3\\p_4
\end{array}
\right)
\]

so 
\begin{eqnarray}
p(r,s,t)
&=& a+br+cs+dt \nonumber\\
&=& \frac{1}{4}(p_1+p_2+p_3+p_4)
+\frac{1}{2}(p_1-p_2-p_3+p_4)r
+\frac{1}{2}(-p_1+p_2-p_3+p_4)s
+\frac{1}{2}(-p_1-p_2+p_3+p_4)t\nonumber\\
&=&
\frac{1}{4}(1+2r-2s-2t)p_1+
\frac{1}{4}(1-2r+2s-2t)p_2+
\frac{1}{4}(1-2r-2s+2t)p_3+
\frac{1}{4}(1+2r+2s+2t)p_4 \nonumber\\
&=& \sum_{i=1}^4 N_i(r,s,t) p_i
\end{eqnarray}
with
\begin{eqnarray}
N_1(r,s,t) &=& \frac{1}{4}(1+2r-2s-2t)\nonumber\\
N_2(r,s,t) &=& \frac{1}{4}(1-2r+2s-2t)\nonumber\\
N_3(r,s,t) &=& \frac{1}{4}(1-2r-2s+2t)\nonumber\\
N_4(r,s,t) &=& \frac{1}{4}(1+2r+2s+2t)\nonumber
\end{eqnarray}

\vspace{.6cm}

I could also have chosen 

\begin{tabular}{cccc}
\hline
point & r & s & t \\
\hline
1& 0 & 0 &0 \\
2& 1 & 0 &0 \\
3& 0 & 1 &0 \\
4& 0 & 0 &1 \\
\hline
\end{tabular}

This expression must exactly interpolate the pressure at all four pressure nodes:
\begin{eqnarray}
p_1  &=& p(r_1,s_1,t_1) = a+br_1+cs_1+dt_1 = a\nonumber\\
p_2  &=& p(r_2,s_2,t_2) = a+br_2+cs_2+dt_2 = a+b\nonumber\\
p_3  &=& p(r_3,s_3,t_3) = a+br_3+cs_3+dt_3 = a+c\nonumber\\
p_4  &=& p(r_4,s_4,t_4) = a+br_4+cs_4+dt_4 = a+d\nonumber
\end{eqnarray}
i.e.
\[
a=p_1
\qquad
b=p_2-p1
\qquad
c=p_3-p1
\qquad
d=p_4-p1
\]
or, 
\[
p^h(r,s)=a+br+cs+dt=p_1 + (p_2-p_1)r + (p_3-p_1)r + (p_4-p_1)t  = p_1(1-r-s-t) + r p_2 + s p_3 + t p_4
\]
so 
\begin{mdframed}[backgroundcolor=blue!5]
\begin{eqnarray}
N_1(r,s,t) &=& 1-r-s-t \\ 
N_2(r,s,t) &=& r \\
N_3(r,s,t) &=& s \\
N_4(r,s,t) &=& t
\end{eqnarray}
\end{mdframed}






%%%%%%%%%%%%%%%%%%%%%%%%%%%%%%%%%%%%%%%%%%%%%%%%%%%%%%%%%%%%%%%%%%%%%
\subsection{20-node serendipity basis functions in 3D ($Q_2^{(20)}$)}
\index{general}{$Q_2^{(20)}$} \index{general}{Serendipity element}

The serendipity elements are those rectangular elements which have no
interior nodes \cite[p91]{reddybook2}.

\begin{verbatim}
   t
   |
   .--s
  /
 r
                                    05=====20=====08 
                                    |             |  
                                    |             |  
                  17 - - - - - -19  13            16
                  .              .  |             |  
                  .              .  |             |  
06=====18=====07  .              .  01=====12=====04 @ r=-1
|             |   .              . 
|             |   .              .  
14            15  09 - - - - - -11 @ r=0
|             |   
|             |  
02=====10=====03 @ r=+1
\end{verbatim}

\todo[inline]{find/build basis functions!}


%..........................................................................---------------------------
\subsection{Enriched linear basis functions in quadrilaterals ($Q_1^+$) -WIP} \label{ss:quadmini3D}
\index{general}{$Q_1^+$}

\input{lamichhane3D}


%-----------------------------------------------------------------
\subsection{The rotated $Q_1$} \label{ss:rq1_3D}
\index{general}{$\tilde{Q}_1$}

The nodes are not on the corners of the element but in the middle of the
element faces:

\begin{center}
\includegraphics[width=4cm]{images/rannacherturek/elt3D}\\
{\captionfont Node numbering and connectivity pattern of the reference element. Taken from \cite{gekm08}}
\end{center}

We have $\tilde{Q}_1=span \{1,r,s,t,r^2-s^2,s^2-t^2\}$.

%.............................................
\paragraph{The Middle Point (MP) variant}. 

The basis functions are given by (see Georgiev \etal (2008) \cite{gekm08}):
\begin{eqnarray}
N_1(r,s,t) &=& \frac{1}{6}(1-3r+2r^2-s^2-t^2 ) \\
N_2(r,s,t) &=& \frac{1}{6}(1+3r+2r^2-s^2-t^2 ) \\
N_3(r,s,t) &=& \frac{1}{6}(1-r^2-3s+2s^2-t^2 ) \\
N_4(r,s,t) &=& \frac{1}{6}(1-r^2+3s+2s^2-t^2 ) \\
N_5(r,s,t) &=& \frac{1}{6}(1-r^2-s^2-3t+2t^2 ) \\
N_6(r,s,t) &=& \frac{1}{6}(1-r^2-s^2+3t+2t^2 ) 
\end{eqnarray}

\begin{eqnarray}
\frac{\partial N_1}{\partial r} &=& \frac{1}{6} (-3+4r )\\
\frac{\partial N_2}{\partial r} &=& \frac{1}{6} (3+4r )\\
\frac{\partial N_3}{\partial r} &=& \frac{1}{6} (-2r) \\
\frac{\partial N_4}{\partial r} &=& \frac{1}{6} (-2r) \\
\frac{\partial N_5}{\partial r} &=& \frac{1}{6} (-2r) \\
\frac{\partial N_6}{\partial r} &=& \frac{1}{6} (-2r) 
\end{eqnarray}

\begin{eqnarray}
\frac{\partial N_1}{\partial s} &=& \frac{1}{6} (-2s)\\ 
\frac{\partial N_2}{\partial s} &=& \frac{1}{6} (-2s) \\
\frac{\partial N_3}{\partial s} &=& \frac{1}{6} (-3+4s )\\
\frac{\partial N_4}{\partial s} &=& \frac{1}{6} (3+4s )\\
\frac{\partial N_5}{\partial s} &=& \frac{1}{6} (-2s) \\
\frac{\partial N_6}{\partial s} &=& \frac{1}{6} (-2s) 
\end{eqnarray}

\begin{eqnarray}
\frac{\partial N_1}{\partial t} &=& \frac{1}{6}(-2t) \\ 
\frac{\partial N_2}{\partial t} &=& \frac{1}{6}(-2t) \\ 
\frac{\partial N_3}{\partial t} &=& \frac{1}{6}(-2t) \\ 
\frac{\partial N_4}{\partial t} &=& \frac{1}{6}(-2t) \\ 
\frac{\partial N_5}{\partial t} &=& \frac{1}{6}(-3+4t) \\ 
\frac{\partial N_6}{\partial t} &=& \frac{1}{6}(3+4t)  
\end{eqnarray}


%......................................
\paragraph{The Mid Value (MV) variant}. 

\begin{eqnarray}
N_1(r,s,t) &=& \frac{1}{12}(2-6r+6r^2-3s^2-3t^2) \\
N_2(r,s,t) &=& \frac{1}{12}(2+6r+6r^2-3s^2-3t^2) \\
N_3(r,s,t) &=& \frac{1}{12}(2-3r^2-6s+6s^2-3t^2) \\
N_4(r,s,t) &=& \frac{1}{12}(2-3r^2+6s+6s^2-3t^2) \\
N_5(r,s,t) &=& \frac{1}{12}(2-3r^2-3s^2-6t+6t^2) \\
N_6(r,s,t) &=& \frac{1}{12}(2-3r^2-3s^2+6t+6t^2)
\end{eqnarray}

\begin{eqnarray}
\frac{\partial N_1}{\partial r} &=& \frac{1}{12}(-6+12r) = \frac{1}{2}(-1+2r)\\
\frac{\partial N_2}{\partial r} &=& \frac{1}{12}(6+12r) = \frac{1}{2}(1+2r)\\
\frac{\partial N_3}{\partial r} &=& \frac{1}{12}(-6r) = -\frac{1}{2}r \\
\frac{\partial N_4}{\partial r} &=& \frac{1}{12}(-6r) = -\frac{1}{2}r \\
\frac{\partial N_5}{\partial r} &=& \frac{1}{12}(-6r) = -\frac{1}{2}r \\
\frac{\partial N_6}{\partial r} &=& \frac{1}{12}(-6r) = -\frac{1}{2}r 
\end{eqnarray}

\begin{eqnarray}
\frac{\partial N_1}{\partial s} &=& \frac{1}{12} (-6s) = -\frac{1}{2}s \\
\frac{\partial N_2}{\partial s} &=& \frac{1}{12} (-6s) = -\frac{1}{2}s \\
\frac{\partial N_3}{\partial s} &=& \frac{1}{12} (-6+12s) = \frac{1}{2} (-1+2s) \\ 
\frac{\partial N_4}{\partial s} &=& \frac{1}{12} (6+12s) = \frac{1}{2} (1+2s) \\ 
\frac{\partial N_5}{\partial s} &=& \frac{1}{12} (-6s) = -\frac{1}{2}s \\
\frac{\partial N_6}{\partial s} &=& \frac{1}{12} (-6s) = -\frac{1}{2}s 
\end{eqnarray}

\begin{eqnarray}
\frac{\partial N_1}{\partial t} &=& \frac{1}{12} (-6t) = -\frac{1}{2}t \\
\frac{\partial N_2}{\partial t} &=& \frac{1}{12} (-6t) = -\frac{1}{2}t \\
\frac{\partial N_3}{\partial t} &=& \frac{1}{12} (-6t) = -\frac{1}{2}t \\
\frac{\partial N_4}{\partial t} &=& \frac{1}{12} (-6t) = -\frac{1}{2}t \\
\frac{\partial N_5}{\partial t} &=& \frac{1}{12} (-6+12t) = \frac{1}{2} (-1+2t) \\ 
\frac{\partial N_6}{\partial t} &=& \frac{1}{12} (6+12t) = \frac{1}{2} (1+2t) 
\end{eqnarray}






%-----------------------------------------------------------------------------
\subsection{The 3D enriched $Q_1^+\times P_0$ of Fortin} \label{ss:Q1pP03D}

This element is mentioned on p249 of Cuvelier, Segal \& van Steenhoven \cite{cuss86}:
"The enriched trilinear velocity-constant pressure element is probably the simplest admissible 3D element."
Fortin \cite{fort81} designed a simple LBB-stable $Q_1$ element to which mid-face nodes are added, 
i.e. a 'bubble' $Q_2$ function is added on each face.
However, only $\vec\upnu\cdot \vec{n}$ is present on these mid-face nodes: 

\input{tikz/tikz_q1pp0}

Fortin states: "this element satisfies the B.B. condition and is probably the 
simplest 3-D element to do So. This
unfortunately does not mean that it is more accurate (at least on regular meshes)." and 
"the element satisfies the B.B. condition. It can therefore be used in a non-regular mesh 
without fear. The number of
degrees of freedom is approximately double with respect to the $Q_1\times P_0$ element and this is
reflected by an increased number of vortices and a reduction of their size. However, there
seems to be a qualitative deficiency of these vortices since they do not easily assemble into
complex flows. Only numerical experiments can give the final answer."
This element is mentioned/used in \cite{rota87b,begt92,vadv03}.

Considering a single element, we have 
\begin{itemize}
\item $Q_1$: $2\times 2\times 2\times 3=24$ velocity dofs
\item $Q_1^+$: $2\times 2\times 2\times 3+6 = 30$ velocity dofs: 
\[
\vec{V}^T=(\underbrace{u_1,v_1,w_1,\dots,u_8,v_8,w_8}_{Q_1\; dofs},
\underbrace{u_9,v_9,w_9,u_{10},v_{10},w_{10}}_{bubble \; dofs})
\]
The big difference with all other elements so far is the fact that the 
dofs $u_9,v_9,w_9$ are not colocated (same
for the other three). $u_9$ lives in the middle of the $r=-1$ face, $v_9$ lives in 
the middle of the $s=-1$ face and 
$w_9$ lives in the middle of the $t=-1$ face.

\item $Q_2$: $3\times 3\times 3\times 3=81$ velocity dofs 
\end{itemize}

Considering a 3D mesh composed of $nel=nelx\times nely\times nelz$ elements:
\begin{itemize}
\item $Q_1$: the total number of Velocity dofs is $NfemV=(nelx+1)\times(nely+1)\times(nelz+1)\times 3$
\item $Q_1^+$:  the total number of nodes is 
\[NfemV=(nelx+1)\times(nely+1)\times(nelz+1)\times 3 
+ (nelx+1)\times nely\times nelz
+ nelx\times (nely+1)\times nelz
+ nelx\times nely \times (nelz+1)
\]
\item $Q_2$: the total number of Velocity dofs is 
$NfemV=(2nelx+1)\times (2nely+1)\times (2nelz+1)\times 3$
\end{itemize}

When $nelx=nely=nelz=n>>1$ then the numbers above converge to 
$3n^3$, $6n^3$ and $24n^3$ respectively. This means that for large meshes 
the enriched $Q_1$ uses twice as many dofs as the standard $Q_1$ while the 
$Q_2$ element uses 8 times more. 



\paragraph{$x$-component of velocity} The polynomial representation of the velocity in the element is given by 
\[
u^h(r,s,t) = a + br +c s + d t +e rs + f rt + g st + h rst
+ k b_9(r,s,t) + l b_{10}(r,s,t)
\]
where the two bubble functions are:
\[
b_9^u(r,s,t)=\frac{1}{2}(1-r)(1-s^2)(1-t^2)
\qquad
b_{10}^u(r,s,t)=\frac{1}{2}(1+r)(1-s^2)(1-t^2)
\]
The coordinates of the $u_9$ dof is (-1,0,0) and the coordinate of the $u_{10}$ dof is $(1,0,0)$.
We see that the bubble functions are 1 at their nodes and zero at all other nodes.
We can actually use a different basis for ${1,r,s,t,rs,rt,st,rst}$ and we 
instead choose the standard $Q_1$ functions so that $u^h$ becomes:
\[
u^h(r,s,t) = aN_1 + b N_2 + cN_3 +dN_4 + eN_5 + fN_6 + gN_7 + hN_8 
+ k b_9(r,s,t) + l b_{10}(r,s,t)
\]
We then must find the set of coefficients $\{a \dots l\}$ and we will do so by 
requiring that $u^h(r_i,s_i,t_i)=u_i$ for $i=1,10$. 

The coordinates of all 10 nodes and the values of basis functions at these locations are:

\begin{center}
\begin{tabular}{c|ccc|cccccccc|cc}
\hline
node $\#$  & $r$ & $s$ & $t$ & $N_1$ & $N_2$ & $N_3$ & $N_4$ & $N_5$ & $N_6$ & $N_7$ & $N_8$ & $b_9^u$ & $b_{10}^u$\\
\hline\hline
1 & -1 & -1 & -1 & 1 & 0 & 0 & 0 & 0 & 0 & 0 & 0 & 0 & 0\\
2 & +1 & -1 & -1 & 0 & 1 & 0 & 0 & 0 & 0 & 0 & 0 & 0 & 0\\
3 & +1 & +1 & -1 & 0 & 0 & 1 & 0 & 0 & 0 & 0 & 0 & 0 & 0\\
4 & -1 & +1 & -1 & 0 & 0 & 0 & 1 & 0 & 0 & 0 & 0 & 0 & 0\\
5 & -1 & -1 & +1 & 0 & 0 & 0 & 0 & 1 & 0 & 0 & 0 & 0 & 0\\
6 & +1 & -1 & +1 & 0 & 0 & 0 & 0 & 0 & 1 & 0 & 0 & 0 & 0\\
7 & +1 & +1 & +1 & 0 & 0 & 0 & 0 & 0 & 0 & 1 & 0 & 0 & 0\\
8 & -1 & +1 & +1 & 0 & 0 & 0 & 0 & 0 & 0 & 0 & 1 & 0 & 0\\
9 & -1 & 0 & 0 & 1/4 & 0   & 0 & 1/4 & 1/4& 0   & 0 & 1/4 & 1 & 0\\
10& +1 & 0 & 0 & 0   & 1/4 & 1/4 & 0  &0 & 1/4 & 1/4 & 0 & 0 & 1\\
\hline
\end{tabular}
\end{center}

We then have the following ten equations:
\begin{eqnarray}
u_1=u^h(r_1,s_1,t_1) &=& a \nn \\
u_2=u^h(r_1,s_1,t_1) &=& b \nn \\
u_3=u^h(r_1,s_1,t_1) &=& c \nn \\
u_4=u^h(r_1,s_1,t_1) &=& d \nn \\
u_5=u^h(r_1,s_1,t_1) &=& e \nn \\
u_6=u^h(r_1,s_1,t_1) &=& f \nn \\
u_7=u^h(r_1,s_1,t_1) &=& g \nn \\
u_8=u^h(r_1,s_1,t_1) &=& h \nn \\
u_9=u^h(r_9,s_9,t_9) &=& \frac{1}{4}(a+d+e+h)+k \nn \\
u_{10}=u^h(r_{10},s_{10},t_{10}) &=& \frac{1}{4}(b+c+f+g)+l \nn
\end{eqnarray}
or, 
\[
\left(
\begin{array}{cccccccccc}
 1 & 0 & 0 & 0 & 0 & 0 & 0 & 0 & 0 & 0\\
 0 & 1 & 0 & 0 & 0 & 0 & 0 & 0 & 0 & 0\\
 0 & 0 & 1 & 0 & 0 & 0 & 0 & 0 & 0 & 0\\
 0 & 0 & 0 & 1 & 0 & 0 & 0 & 0 & 0 & 0\\
 0 & 0 & 0 & 0 & 1 & 0 & 0 & 0 & 0 & 0\\
 0 & 0 & 0 & 0 & 0 & 1 & 0 & 0 & 0 & 0\\
 0 & 0 & 0 & 0 & 0 & 0 & 1 & 0 & 0 & 0\\
 0 & 0 & 0 & 0 & 0 & 0 & 0 & 1 & 0 & 0\\
 1/4 & 0   & 0 & 1/4 & 1/4& 0   & 0 & 1/4 & 1 & 0\\
 0   & 1/4 & 1/4 & 0  &0 & 1/4 & 1/4 & 0 & 0 & 1
\end{array}
\right)
\cdot
\left(
\begin{array}{c}
a \\ b\\ c\\ d\\ e\\ f\\ g\\ h\\ k\\ l
\end{array}
\right)
=
\left(
\begin{array}{c}
u_1 \\ u_2 \\ u_3 \\ u_4 \\ u_5 \\ u_6 \\ u_7 \\ u_8 \\ u_9 \\ u_{10}
\end{array}
\right)
\]

This yields
\begin{eqnarray}
a &=& u_1 \nn\\
b &=& u_2 \nn\\
c &=& u_3 \nn\\
d &=& u_4 \nn\\
e &=& u_5 \nn\\
f &=& u_6 \nn\\
g &=& u_7 \nn\\
h &=& u_8 \nn\\
k &=& u_9-\frac{1}{4}(u_1 + u_4 + u_5 + u_8) \nn\\
l &=& u_{10}-\frac{1}{4}(u_2 + u_3 + u_6 + u_7) \nn 
\end{eqnarray}

and then 
\begin{eqnarray}
u^h(r,s,t)  
&=& aN_1 + b N_2 + cN_3 +dN_4 + eN_5 + fN_6 + gN_7 + hN_8 
+ k b_9^u(r,s,t) + l b_{10}^u(r,s,t) \nn\\
&=& u_1N_1 + u_2 N_2 + u_3N_3 +u_4N_4 + u_5N_5 + u_6N_6 + u_7N_7 + u_8N_8 \nn\\
&& +
\left[u_9-\frac{1}{4}(u_1 + u_4 + u_5 + u_8)\right] b_9^u(r,s,t) +
\left[u_{10}-\frac{1}{4}(u_2 + u_3 + u_6 + u_7)\right] b_{10}^u(r,s,t) \nn\\
&=& 
\left(u_1-\frac{1}{4}b_9\right)N_1 +
\left(u_2-\frac{1}{4}b_{10}\right)N_2 +
\left(u_3-\frac{1}{4}b_{10}\right)N_3 +
\left(u_4-\frac{1}{4}b_9\right)N_4 + \nn\\
&&
\left(u_5-\frac{1}{4}b_9\right)N_5 +
\left(u_6-\frac{1}{4}b_{10}\right)N_6 +
\left(u_7-\frac{1}{4}b_{10}\right)N_7 +
\left(u_8-\frac{1}{4}b_9\right)N_8 + \nn\\
&& b_9^u(r,s,t) u_9+  b_{10}^u(r,s,t) u_{10} \nn
\end{eqnarray}

Finally, we can write the basis functions for the $u$ field:

\begin{mdframed}[backgroundcolor=blue!5]
\begin{eqnarray}
{N}_1^u(r,s,t) &=&  N_1(r,s,t) - \frac{1}{4} b_9^u(r,s,t)\nn\\
{N}_2^u(r,s,t) &=&  N_2(r,s,t) - \frac{1}{4} b_{10}^u(r,s,t)\nn\\
{N}_3^u(r,s,t) &=&  N_3(r,s,t) - \frac{1}{4} b_{10}^u(r,s,t)\nn\\
{N}_4^u(r,s,t) &=&  N_4(r,s,t) - \frac{1}{4} b_9^u(r,s,t)\nn\\
{N}_5^u(r,s,t) &=&  N_5(r,s,t) - \frac{1}{4} b_9^u(r,s,t)\nn\\
{N}_6^u(r,s,t) &=&  N_6(r,s,t) - \frac{1}{4} b_{10}^u(r,s,t)\nn\\
{N}_7^u(r,s,t) &=&  N_7(r,s,t) - \frac{1}{4} b_{10}^u(r,s,t)\nn\\
{N}_8^u(r,s,t) &=&  N_8(r,s,t) - \frac{1}{4} b_9^u(r,s,t)\nn\\
{N}_9^u(r,s,t) &=&  b_9^u(r,s,t)\nn\\
{N}_{10}^u(r,s,t) &=&  b_{10}^u(r,s,t) \nn
\end{eqnarray}
\end{mdframed}
And it is easy to verify that  
\[
\sum_{i=1}^{10} {N}_i^u(r,s,t) = 1  \qquad \forall r,s,t
\]
During the implementation phase we will need the derivatives of the basis functions, 
which are trivial for the standard $Q_1$ basis functions $N_i$. Remain then 


\begin{eqnarray}
\partial_r b_9^u(r,s,t) 
&=& \frac{\partial }{\partial r}  \left( \frac{1}{2}(1-r)(1-s^2)(1-t^2) \right) 
=  -\frac{1}{2}(1-s^2)(1-t^2)  \nn\\
\partial_s b_9^u(r,s,t) 
&=& \frac{\partial }{\partial s}  \left( \frac{1}{2}(1-r)(1-s^2)(1-t^2) \right) 
=  -(1-r)s(1-t^2)  \nn\\
\partial_t b_9^u(r,s,t) 
&=& \frac{\partial }{\partial t}  \left( \frac{1}{2}(1-r)(1-s^2)(1-t^2) \right) 
=  -(1-r)(1-s^2)t  \nn\\ \nn\\
\partial_r b_{10}^u(r,s,t) 
&=& \frac{\partial }{\partial r}  \left( \frac{1}{2}(1+r)(1-s^2)(1-t^2) \right) 
=  \frac{1}{2}(1-s^2)(1-t^2)  \nn\\
\partial_s b_{10}^u(r,s,t) 
&=& \frac{\partial }{\partial s}  \left( \frac{1}{2}(1+r)(1-s^2)(1-t^2) \right) 
=  -(1+r)s(1-t^2)  \nn\\
\partial_t b_{10}^u(r,s,t) 
&=& \frac{\partial }{\partial t}  \left( \frac{1}{2}(1+r)(1-s^2)(1-t^2) \right) 
=  -(1+r)(1-s^2)t  \nn
\end{eqnarray}

%.........................................
\paragraph{$y$-component of velocity} 
The bubbles are given by 
\[
b_9^v(r,s,t)=\frac{1}{2}(1-r^2)(1-s)(1-t^2)
\qquad
b_{10}^v(r,s,t)=\frac{1}{2}(1-r^2)(1+s)(1-t^2)
\]
The coordinates of all 10 nodes and the values of basis functions at these locations are:
\begin{center}
\begin{tabular}{c|ccc|cccccccc|cc}
\hline
node $\#$  & $r$ & $s$ & $t$ & $N_1$ & $N_2$ & $N_3$ & $N_4$ & $N_5$ & $N_6$ & $N_7$ & $N_8$ & $b_9^v$ & $b_{10}^v$\\
\hline\hline
1 & -1 & -1 & -1 & 1 & 0 & 0 & 0 & 0 & 0 & 0 & 0 & 0 & 0\\
2 & +1 & -1 & -1 & 0 & 1 & 0 & 0 & 0 & 0 & 0 & 0 & 0 & 0\\
3 & +1 & +1 & -1 & 0 & 0 & 1 & 0 & 0 & 0 & 0 & 0 & 0 & 0\\
4 & -1 & +1 & -1 & 0 & 0 & 0 & 1 & 0 & 0 & 0 & 0 & 0 & 0\\
5 & -1 & -1 & +1 & 0 & 0 & 0 & 0 & 1 & 0 & 0 & 0 & 0 & 0\\
6 & +1 & -1 & +1 & 0 & 0 & 0 & 0 & 0 & 1 & 0 & 0 & 0 & 0\\
7 & +1 & +1 & +1 & 0 & 0 & 0 & 0 & 0 & 0 & 1 & 0 & 0 & 0\\
8 & -1 & +1 & +1 & 0 & 0 & 0 & 0 & 0 & 0 & 0 & 1 & 0 & 0\\
9 &  0 & -1 &  0 & 1/4 & 1/4 & 0 & 0 & 1/4 & 1/4 & 0 & 0 & 1& 0\\
10&  0 & +1 &  0 & 0 & 0 & 1/4 & 1/4 & 0 & 0 & 1/4 & 1/4 & 0& 1\\
\hline
\end{tabular}
\end{center}

Then

\begin{mdframed}[backgroundcolor=blue!5]
\begin{eqnarray}
{N}_1^v(r,s,t) &=&  N_1(r,s,t) - \frac{1}{4} b_9^v(r,s,t)\nn\\
{N}_2^v(r,s,t) &=&  N_2(r,s,t) - \frac{1}{4} b_9^v(r,s,t)\nn\\
{N}_3^v(r,s,t) &=&  N_3(r,s,t) - \frac{1}{4} b_{10}^v(r,s,t)\nn\\
{N}_4^v(r,s,t) &=&  N_4(r,s,t) - \frac{1}{4} b_{10}^v(r,s,t)\nn\\
{N}_5^v(r,s,t) &=&  N_5(r,s,t) - \frac{1}{4} b_9^v(r,s,t)\nn\\
{N}_6^v(r,s,t) &=&  N_6(r,s,t) - \frac{1}{4} b_9^v(r,s,t)\nn\\
{N}_7^v(r,s,t) &=&  N_7(r,s,t) - \frac{1}{4} b_{10}^v(r,s,t)\nn\\
{N}_8^v(r,s,t) &=&  N_8(r,s,t) - \frac{1}{4} b_{10}^v(r,s,t)\nn\\
{N}_9^v(r,s,t) &=&  b_9^v(r,s,t)\nn\\
{N}_{10}^v(r,s,t) &=&  b_{10}^v(r,s,t)
\end{eqnarray}
\end{mdframed}

%.........................................
\paragraph{$z$-component of velocity} 



The bubbles are given by 
\[
b_9^w(r,s,t)=\frac{1}{2}(1-r^2)(1-s^2)(1-t)
\qquad
b_{10}^w(r,s,t)=\frac{1}{2}(1-r^2)(1-s^2)(1+t)
\]
The coordinates of all 10 nodes and the values of basis functions at these locations are:
\begin{center}
\begin{tabular}{c|ccc|cccccccc|cc}
\hline
node $\#$  & $r$ & $s$ & $t$ & $N_1$ & $N_2$ & $N_3$ & $N_4$ & $N_5$ & $N_6$ & $N_7$ & $N_8$ & $b_9^w$ & $b_{10}^w$\\
\hline\hline
1 & -1 & -1 & -1 & 1 & 0 & 0 & 0 & 0 & 0 & 0 & 0 & 0 & 0\\
2 & +1 & -1 & -1 & 0 & 1 & 0 & 0 & 0 & 0 & 0 & 0 & 0 & 0\\
3 & +1 & +1 & -1 & 0 & 0 & 1 & 0 & 0 & 0 & 0 & 0 & 0 & 0\\
4 & -1 & +1 & -1 & 0 & 0 & 0 & 1 & 0 & 0 & 0 & 0 & 0 & 0\\
5 & -1 & -1 & +1 & 0 & 0 & 0 & 0 & 1 & 0 & 0 & 0 & 0 & 0\\
6 & +1 & -1 & +1 & 0 & 0 & 0 & 0 & 0 & 1 & 0 & 0 & 0 & 0\\
7 & +1 & +1 & +1 & 0 & 0 & 0 & 0 & 0 & 0 & 1 & 0 & 0 & 0\\
8 & -1 & +1 & +1 & 0 & 0 & 0 & 0 & 0 & 0 & 0 & 1 & 0 & 0\\
9 &  0 &  0 & -1 & 1/4 & 1/4 & 1/4 & 1/4 & 0 & 0 & 0 & 0&  1& 0\\
10&  0 &  0 & +1 & 0 & 0 &0 & 0 & 1/4 & 1/4  & 1/4 & 1/4 & 0& 1\\
\hline
\end{tabular}
\end{center}


\begin{mdframed}[backgroundcolor=blue!5]
\begin{eqnarray}
{N}_1^w(r,s,t) &=&  N_1(r,s,t) - \frac{1}{4} b_9^w(r,s,t)\nn\\
{N}_2^w(r,s,t) &=&  N_2(r,s,t) - \frac{1}{4} b_9^w(r,s,t)\nn\\
{N}_3^w(r,s,t) &=&  N_3(r,s,t) - \frac{1}{4} b_9^w(r,s,t)\nn\\
{N}_4^w(r,s,t) &=&  N_4(r,s,t) - \frac{1}{4} b_9^w(r,s,t)\nn\\
{N}_5^w(r,s,t) &=&  N_5(r,s,t) - \frac{1}{4} b_{10}^w(r,s,t)\nn\\
{N}_6^w(r,s,t) &=&  N_6(r,s,t) - \frac{1}{4} b_{10}^w(r,s,t)\nn\\
{N}_7^w(r,s,t) &=&  N_7(r,s,t) - \frac{1}{4} b_{10}^w(r,s,t)\nn\\
{N}_8^w(r,s,t) &=&  N_8(r,s,t) - \frac{1}{4} b_{10}^w(r,s,t)\nn\\
{N}_9^w(r,s,t) &=&  b_9^w(r,s,t)\nn\\
{N}_{10}^w(r,s,t) &=&  b_{10}^w(r,s,t) \nn
\end{eqnarray}
\end{mdframed}


%.............................................. 
\paragraph{A word about the ${\bm B}$ matrix} We have  
\begin{eqnarray}
u^h(r,s,t) = \sum_{i=1}^{10} N_i^u(r,s,t) u_i \\
v^h(r,s,t) = \sum_{i=1}^{10} N_i^v(r,s,t) v_i \\
w^h(r,s,t) = \sum_{i=1}^{10} N_i^w(r,s,t) w_i 
\end{eqnarray}

Normally we do not make a distinction between the basis functions 
associated to $u,v,w$ but because of the bubbles on the faces we now have to. 

We have previously established that the strain rate 
vector $\vec{\dot \varepsilon}$ is: 
\begin{equation}
\vec{\dot\varepsilon}=
\left(
\begin{array}{c}
\frac{\partial u}{\partial x} \\ \\
\frac{\partial v}{\partial y} \\ \\
\frac{\partial w}{\partial z} \\ \\
\frac{\partial u}{\partial y}\! +\! \frac{\partial v}{\partial x} \\ \\
\frac{\partial u}{\partial z}\! +\! \frac{\partial w}{\partial x} \\ \\
\frac{\partial v}{\partial z}\! +\! \frac{\partial w}{\partial y} 
\end{array}
\right)
=
\left(
\begin{array}{c}
\sum\limits_i \frac{\partial N_i^u}{\partial x} u_i \\ \\
\sum\limits_i \frac{\partial N_i^v}{\partial y} v_i \\ \\
\sum\limits_i \frac{\partial N_i^w}{\partial z} w_i \\ \\
\sum\limits_i (\frac{\partial N_i^u}{\partial y} u_i\! +\! \frac{\partial N_i^v}{\partial x} v_i) \\ \\
\sum\limits_i (\frac{\partial N_i^u}{\partial z} u_i\! +\! \frac{\partial N_i^w}{\partial x} w_i) \\ \\
\sum\limits_i (\frac{\partial N_i^v}{\partial z} v_i\! +\! \frac{\partial N_i^w}{\partial y} w_i) 
\end{array}
\right)
=
\underbrace{
\left(
\begin{array}{ccccccccccc}
\frac{\partial N_1^u}{\partial x} & 0 & 0 &  \cdots  & \frac{\partial N_{10}^u}{\partial x} & 0 & 0 \\ \\
0 & \frac{\partial N_1^v}{\partial y} & 0 & \cdots & 0 & \frac{\partial N_{10}^v}{\partial y} & 0 \\ \\
0 & 0 & \frac{\partial N_1^w}{\partial z} & \cdots & 0 & 0 & \frac{\partial N_{10}^w}{\partial z}
\\ \\
\frac{\partial N_1^u}{\partial y} &  \frac{\partial N_1^v}{\partial x} &  
0 & \cdots  &\frac{\partial N_{10}^u}{\partial x} 
& \frac{\partial N_{10}^v}{\partial x} & 0 \\ \\
\frac{\partial N_1^u}{\partial z} & 0 & \frac{\partial N_1^w}{\partial x} & \cdots &
\frac{\partial N_{10}^u}{\partial z} & 0 & \frac{\partial N_{m_v}^w}{\partial x} \\  \\
0 &  \frac{\partial N_1^v}{\partial z}  & \frac{\partial N_1^w}{\partial y} & \cdots &
0 &  \frac{\partial N_{10}^v}{\partial z}  & \frac{\partial N_{10}^w}{\partial y} 
\end{array}
\right) 
}_{\bm B}
\!
\cdot
\!
\underbrace{
\left(
\begin{array}{c}
u_1 \\ v_1 \\ w_1 \\ u_2 \\ v_2 \\ w_2 \\ u_3 \\ v_3 \\ \dots \\ u_{10} \\ v_{10} \\ w_{10}
\end{array}
\right)
}_{\vec V} \nn
\end{equation}

\newpage
%-----------------------------------------------------------------------------
\subsection{The $Q_1\times Q_1$ + 2 bubbles} \label{ss:Q1Q1bb_3D}

\begin{flushright} {\tiny {\color{gray} \tt q1q13D\_2bubbles.tex}} \end{flushright}
%------------------------------------------------------------------------------

This element is implemented in \stone 82.
The two bubble functions are given in Karabelas \etal (2020) \cite{kahp20}:
\begin{eqnarray}
b_9(r,s,t) &=& \left(\frac{27}{32}\right)^3 (1-r^2)(1-s^2)(1-t^2) \cdot (1-r)(1-s)(1-t) 
= \beta_9(r)\cdot\beta_9(s) \cdot \beta_9(t) \nonumber\\
b_{10}(r,s,t) &=& \left(\frac{27}{32}\right)^3 (1-r^2)(1-s^2)(1-t^2) \cdot (1+r)(1+s)(1+t) 
= \beta_{10}(r)\cdot\beta_{10}(s) \cdot \beta_{10}(t)  \nonumber
\end{eqnarray}
where I have chosen nodes 1 ($\vec{r}_1=(-1,-1,-1)$) and 7 ($\vec{r}_7=(+1,+1,+1)$) 
as diagonally opposed nodes (a requirement from the paper), 
and with
\[
\beta_9(x)=\frac{27}{32} (1-x^2) (1-x)
\qquad
\beta_{10}(x)=\frac{27}{32} (1-x^2) (1+x)
\]

I have added the $(27/32)^3$ coefficients so that these functions are exactly 1 a their 
corresponding nodes.
The term $(1-r^2)(1-s^2)(1-t^2)$ makes sure that the two bubbles are conforming and exactly zero 
on the 6 faces of the element.
In what follows $\tilde{\bN}_{1..8}$ are the standard $Q_1$ basis functions.

\begin{center}
\includegraphics[width=8cm]{images/MINI3D/bubbles.pdf}\\
{\captionfont Representation of bubbles $\beta_9(x)$ and $\beta_{10}(x)$}
\end{center}

\begin{remark}
Bubble function 9 is not zero at node 10 and vice versa!
\end{remark}

The authors state: "This also allows for a straightforward inclusion in combination 
with existing finite element codes since all required
implementations are purely on the element level". This is especially true 
if static condensation is used (the authors explain static condensation 
for the bubbles in the appendix of the paper).

The ten nodes are the standard 8 corners of the $Q_1$ element as well  
as $\vec{r}_9=(-1/3,-1/3,-1/3)$ for $b_9$ and 
$\vec{r}_{10}=(1/3,1/3,1/3)$ for $b_{10}$.
We have the following approximation of function $f$ inside the element:
\[
f^h(r,s,t) = \sum_{i=1}^{8} a_i \tilde{\bN}_i(r,s,t) + a_9 b_9(r,s,t) + a_{10} b_{10}(r,s,t)
\]
We notice that bubble functions are exactly zero at the corners of the reference element and
we can compute the values of the ten polynomials ($\tilde{N}_{1-8}(r,s,t),b_9(r,s,t),b_{10}(r,s,t)$) 
at the ten nodes:
\[
\begin{array}{c|cccccccccc}
 & \tilde{\bN}_1 & \tilde{\bN}_2 & \tilde{\bN}_3 & \tilde{\bN}_4 & \tilde{\bN}_5 
 & \tilde{\bN}_6 & \tilde{\bN}_7 & \tilde{\bN}_8 & b_9 & b_{10}\\
 \hline\hline
\vec{r}_1=(-1,-1,-1)    &1 &0 &0 &0 &0 &0 &0 &0 &0 &0 \\
\vec{r}_2=(+1,-1,-1)    &0 &1 &0 &0 &0 &0 &0 &0 &0 &0 \\
\vec{r}_3=(+1,+1,-1)    &0 &0 &1 &0 &0 &0 &0 &0 &0 &0 \\
\vec{r}_4=(-1,+1,-1)    &0 &0 &0 &1 &0 &0 &0 &0 &0 &0 \\
\vec{r}_5=(-1,-1,+1)    &0 &0 &0 &0 &1 &0 &0 &0 &0 &0 \\
\vec{r}_6=(+1,-1,+1)    &0 &0 &0 &0 &0 &1 &0 &0 &0 &0 \\
\vec{r}_7=(+1,+1,+1)    &0 &0 &0 &0 &0 &0 &1 &0 &0 &0 \\
\vec{r}_8=(-1,+1,+1)    &0 &0 &0 &0 &0 &0 &0 &1 &0 &0 \\
\vec{r}_9=(-\frac13,-\frac13,-\frac13)    &8/27 & 4/27 & 2/27  & 4/27 & 4/27 & 2/27& 1/27& 2/27 &1 & 1/8\\
\vec{r}_{10}=(+\frac13,+\frac13,+\frac13) &1/27 & 2/17 & 4/27 & 2/27 & 2/27& 4/27& 8/27& 4/27& 1/8  &1 \\
\end{array}
\]




We then require that the polynomial representation of $f^h$ of $f$ inside the element
is such that $f^h(\vec{r}_i)=f_i$, i.e.:
\begin{eqnarray}
f_1 = f^h(r_1,s_1,t_1) &=& a_1  \nonumber\\
f_2 = f^h(r_2,s_1,t_1) &=& a_2  \nonumber\\
f_3 = f^h(r_3,s_1,t_1) &=& a_3  \nonumber\\
f_4 = f^h(r_4,s_1,t_1) &=& a_4  \nonumber\\
f_5 = f^h(r_5,s_1,t_1) &=& a_5  \nonumber\\
f_6 = f^h(r_6,s_1,t_1) &=& a_6  \nonumber\\
f_7 = f^h(r_7,s_1,t_1) &=& a_7  \nonumber\\
f_8 = f^h(r_8,s_1,t_1) &=& a_8  \nonumber\\
f_9 = f^h(r_9,s_9,t_9) &=&  \frac{1}{27} (8a_1 + 4a_2 + 2a_3 +4a_4 + 4a_5 + 2a_6 + a_7 +2a_8)  + a_9 + a_{10}/8 
\nonumber\\
f_{10} = f^h(r_{10},s_{10},t_{10}) &=& \frac{1}{27} (a_1 + 2a_2 + 4a_3 +2a_4 + 2a_5 + 4a_6 + 8a_7 +4a_8)  + a_9/8 + a_{10} \nonumber
\end{eqnarray}
or,
\[
\left(
\begin{array}{cccccccccc}
1 &0 &0 &0 &0 &0 &0 &0 &0 &0 \\
0 &1 &0 &0 &0 &0 &0 &0 &0 &0 \\
0 &0 &1 &0 &0 &0 &0 &0 &0 &0 \\
0 &0 &0 &1 &0 &0 &0 &0 &0 &0 \\
0 &0 &0 &0 &1 &0 &0 &0 &0 &0 \\
0 &0 &0 &0 &0 &1 &0 &0 &0 &0 \\
0 &0 &0 &0 &0 &0 &1 &0 &0 &0 \\
0 &0 &0 &0 &0 &0 &0 &1 &0 &0 \\
8/27 & 4/27 & 2/27  & 4/27 & 4/27 & 2/27& 1/27& 2/27 &1 & 1/8\\
1/27 & 2/17 & 4/27 & 2/27 & 2/27& 4/27& 8/27& 4/27& 1/8  &1 
\end{array}
\right)
\cdot
\left(
\begin{array}{c}
a_1 \\ a_2 \\ a_3 \\ a_4 \\ a_5 \\ a_6 \\ a_7 \\ a_8 \\ a_9 \\ a_{10}
\end{array}
\right)
=
\left(
\begin{array}{c}
f_1 \\ f_2 \\ f_3 \\ f_4 \\ f_5 \\ f_6 \\ f_7 \\ f_8 \\ f_9 \\ f_{10}
\end{array}
\right)
\]
which yields $a_i=f_i$ for $i=1,...8$ and 
\begin{eqnarray}
a_9 + a_{10}/8 &=& f_9\underbrace{-\frac{1}{27} (8f_1 + 4f_2 + 2f_3 +4f_4 + 4f_5 + 2f_6 + f_7 +2f_8)}_{\tilde{f}_9}   
\nonumber\\
a_9/8 + a_{10} &=& f_{10}
\underbrace{-\frac{1}{27} (f_1 + 2f_2 + 4f_3 +2f_4 + 2f_5 + 4f_6 + 8f_7 +4f_8)}_{\tilde{f}_{10}} 
\nonumber
\end{eqnarray}

\begin{eqnarray}
8a_9 + a_{10} &=& 8f_9 + 8\tilde{f}_9  \nonumber\\
a_9 + 8a_{10} &=& 8f_{10} +8\tilde{f}_{10}
\nonumber
\end{eqnarray}
and then 
\begin{eqnarray}
a_{9}
&=&\frac{1}{63}(64 {f}_{9}-8{f}_{10}) + \frac{1}{63}(64 \tilde{f}_{9}-8\tilde{f}_{10}) \nonumber\\
&=& \frac{8}{63}(8{f}_{9}-{f}_{10})+\frac{8}{63}(8\tilde{f}_{9}-\tilde{f}_{10}) \nonumber\\
&=& \frac{8}{63}(8{f}_{9}-{f}_{10}) -\frac{1}{27}\frac{8}{63}
\left[ 8 (8f_1 + 4f_2 + 2f_3 +4f_4 + 4f_5 + 2f_6 + f_7 +2f_8) \right. \nonumber\\
&& \left. - (f_1 + 2f_2 + 4f_3 +2f_4 + 2f_5 + 4f_6 + 8f_7 +4f_8) \right] \nonumber\\
&=& \frac{8}{63}(8{f}_{9}-{f}_{10}) -\frac{1}{27}\frac{8}{63} 
( 63 f_1 +30 f_2 +12 f_3  +30f_4 +30f_5 +12f_6 +12f_8)\nonumber\\ \nonumber \\
a_{10} 
&=& \frac{1}{63}(64 {f}_{10}-8{f}_9) + \frac{1}{63}(64 \tilde{f}_{10}-8\tilde{f}_9) \nonumber\\
&=& \frac{8}{63}(8 {f}_{10}-{f}_9) + \frac{8}{63}(8 \tilde{f}_{10}-\tilde{f}_9) \nonumber\\
&=& \frac{8}{63}(8 {f}_{10}-{f}_9) -\frac{1}{27}\frac{8}{63} \left[
8(f_1 + 2f_2 + 4f_3 +2f_4 + 2f_5 + 4f_6 + 8f_7 +4f_8) \right. \nonumber\\
&& \left. -(8f_1 + 4f_2 + 2f_3 +4f_4 + 4f_5 + 2f_6 + f_7 +2f_8)
\right] \nonumber\\
&=&\frac{8}{63}(8 {f}_{10}-{f}_9) -\frac{1}{27}\frac{8}{63} 
(12f_2 + 30f_3 + 12f_4 +12f_5 + 30f_6 + 63f_7 +30f_8      ) \nonumber
\end{eqnarray}

We can then write

\begin{eqnarray}
f^h(r,s,t) 
&=& f_1 \bN_1(r,s,t) + f_2 \bN_2(r,s,t) + f_3 \bN_3(r,s,t) + f_4 \bN_4(r,s,t) \nonumber\\
&+& f_5 \bN_5(r,s,t) + f_6 \bN_6(r,s,t) + f_7 \bN_7(r,s,t) + f_8 \bN_8(r,s,t) \nonumber\\
&+& \left[\frac{8}{63}(8{f}_{9}-{f}_{10})-\frac{1}{27}\frac{8}{63}  ( 63 f_1 +30 f_2 +12 f_3  +30f_4 +30f_5 +12f_6 +12f_8) \right] b_9(r,s,t) \nonumber\\
&&\left[ \frac{8}{63}(8 {f}_{10}-{f}_9)  -\frac{1}{27}\frac{8}{63} (12f_2 + 30f_3 + 12f_4 +12f_5 + 30f_6 + 63f_7 +30f_8) \right] b_{10}(r,s,t) \nonumber\\
%&=& \left(N_1(r,s,t) - \frac{8}{27}b_9(r,s,t) \right) f_1 \nonumber\\
%&+& \left(N_2(r,s,t) - \frac{30}{27}\frac{8}{63} b_9(r,s,t) - \frac{12}{27}\frac{8}{63} b_{10}(r,s,t)  \right) f_2 \nonumber\\
%&+&\left( N_3(r,s,t)  -\frac{12}{27}\frac{8}{63} b_9r,s,t) -\frac{30}{27}\frac{8}{63} b_{10}(r,s,t)  \right)f_3 \nonumber\\
%&+& \left( N_4(r,s,t) -\frac{30}{27}\frac{8}{63} b_{9}r,s,t) -\frac{12}{27}\frac{8}{63} b_{10}r,s,t)\right)f_4 \nonumber\\ 
%&+&\left( N_5(r,s,t)- \frac{30}{27}\frac{8}{63} b_{9}r,s,t) -\frac{12}{27}\frac{8}{63} b_{10}r,s,t) \right) f_5 \nonumber\\
%&+&\left( N_6(r,s,t) -\frac{12}{27}\frac{8}{63} b_9r,s,t) -\frac{30}{27}\frac{8}{63} b_{10}r,s,t) \right) f_6 \nonumber\\
%&+&\left( N_7(r,s,t) - \frac{8}{27}b_{10}(r,s,t) \right) f_7 \nonumber\\
%&+&\left(N_8(r,s,t) -\frac{12}{27}\frac{8}{63} b_9(r,s,t) -\frac{30}{27}\frac{8}{63} b_{10}(r,s,t) \right) f_8 \nonumber\\
%&+& \left(  \frac{64}{63}b_9(r,s,t) -\frac{8}{63} b_{10}(r,s,t) \right) f_9 
%+ \left( -\frac{8}{63} b_{9}(r,s,t)  + \frac{64}{63}b_{10}(r,s,t) \right) f_{10} \nonumber\\
&=& \left(\bN_1(r,s,t) - \frac{2^3}{3^3}b_9(r,s,t) \right) f_1 \nonumber\\
&+& \left(\bN_2(r,s,t) - \frac{2^3}{3^3}\frac{10}{21} b_9(r,s,t) - \frac{2^3}{3^3}\frac{4}{21} b_{10}(r,s,t)  \right) f_2 \nonumber\\
&+&\left( \bN_3(r,s,t)  -\frac{2^3}{3^3}\frac{4}{21} b_9r,s,t) -\frac{2^3}{3^3}\frac{10}{21} b_{10}(r,s,t)  \right)f_3 \nonumber\\
&+& \left( \bN_4(r,s,t) -\frac{2^3}{3^3}\frac{10}{21} b_{9}r,s,t) -\frac{2^3}{3^3}\frac{4}{21} b_{10}(r,s,t)\right)f_4 \nonumber\\ 
&+&\left( \bN_5(r,s,t)- \frac{2^3}{3^3}\frac{10}{21} b_{9}r,s,t) -\frac{2^3}{3^3}\frac{4}{21} b_{10}(r,s,t) \right) f_5 \nonumber\\
&+&\left( \bN_6(r,s,t) -\frac{2^3}{3^3}\frac{4}{21} b_9(r,s,t) -\frac{2^3}{3^3}\frac{10}{21} b_{10}(r,s,t) \right) f_6 \nonumber\\
&+&\left( \bN_7(r,s,t) - \frac{2^3}{3^3}b_{10}(r,s,t) \right) f_7 \nonumber\\
&+&\left( \bN_8(r,s,t) -\frac{2^3}{3^3}\frac{4}{21} b_9(r,s,t) -\frac{2^3}{3^3}\frac{10}{21} b_{10}(r,s,t) \right) f_8 \nonumber\\
&+& \left(  \frac{64}{63}b_9(r,s,t) -\frac{8}{63} b_{10}(r,s,t) \right) f_9 
+ \left( -\frac{8}{63} b_{9}(r,s,t)  + \frac{64}{63}b_{10}(r,s,t) \right) f_{10} 
\end{eqnarray}

and finally arrive at the basis functions:
\begin{mdframed}[backgroundcolor=blue!5]
\begin{eqnarray}
\bN_1(r,s,t)    &=&  \tilde{\bN}_1(r,s,t) - \frac{2^3}{3^3}b_9(r,s,t)      \nonumber\\
\bN_2(r,s,t)    &=&  \tilde{\bN}_2(r,s,t) - \frac{2^3}{3^3}\frac{10}{21} b_9(r,s,t) - \frac{2^3}{3^3}\frac{4}{21} b_{10}(r,s,t)      \nonumber\\
\bN_3(r,s,t)    &=&  \tilde{\bN}_3(r,s,t)  -\frac{2^3}{3^3}\frac{4}{21} b_9(r,s,t) -\frac{2^3}{3^3}\frac{10}{21} b_{10}(r,s,t)      \nonumber\\
\bN_4(r,s,t)    &=&  \tilde{\bN}_4(r,s,t) -\frac{2^3}{3^3}\frac{10}{21} b_{9}(r,s,t) -\frac{2^3}{3^3}\frac{4}{21} b_{10}r,s,t)      \nonumber\\
\bN_5(r,s,t)    &=&  \tilde{\bN}_5(r,s,t)- \frac{2^3}{3^3}\frac{10}{21} b_{9}(r,s,t) -\frac{2^3}{3^3}\frac{4}{21} b_{10}(r,s,t)      \nonumber\\
\bN_6(r,s,t)    &=&  \tilde{\bN}_6(r,s,t) -\frac{2^3}{3^3}\frac{4}{21} b_9(r,s,t) -\frac{2^3}{3^3}\frac{10}{21} b_{10}(r,s,t)      \nonumber\\
\bN_7(r,s,t)    &=&  \tilde{\bN}_7(r,s,t) - \frac{2^3}{3^3}b_{10}(r,s,t)      \nonumber\\
\bN_8(r,s,t)    &=&  \tilde{\bN}_8(r,s,t) -\frac{2^3}{3^3}\frac{4}{21} b_9(r,s,t) -\frac{2^3}{3^3}\frac{10}{21} b_{10}(r,s,t)      \nonumber\\
\bN_9(r,s,t)    &=&  \frac{64}{63}b_9(r,s,t) -\frac{8}{63} b_{10}(r,s,t)      \nonumber\\
\bN_{10}(r,s,t) &=&  -\frac{8}{63} b_{9}(r,s,t)  + \frac{64}{63}b_{10}(r,s,t)       \nonumber
\end{eqnarray}
\end{mdframed}

These are somewhat complex forms for the basis functions so we wish to verify the simple property $\sum \bN_i(r,s,t) =1$ 
for all $(r,s,t)$ inside the element:
\begin{eqnarray}
\sum_{i=1}^{10} \bN_i(r,s,t)  
&=& \sum_{i=1}^{8} \tilde{\bN}_i(r,s,t) \nonumber\\
&+& \left[\frac{2^3}{3^3} \left(-1 -\frac{10}{21} -\frac{4}{21} -\frac{10}{21} -\frac{10}{21} -\frac{4}{21} -\frac{4}{21}\right) + \frac{64}{63} -\frac{8}{63} \right] b_9(r,s,t) \nonumber\\
&+& \left[\frac{2^3}{3^3} \left(-\frac{4}{21} -\frac{10}{21} -\frac{4}{21} -\frac{4}{21} -\frac{10}{21} -1 -\frac{10}{21} \right) -\frac{8}{63} + \frac{64}{63} \right] b_{10}(r,s,t) \nonumber\\
&=& 1 + \left[\frac{2^3}{3^3} (-1 -42/21) +\frac{56}{63}  \right] b_9(r,s,t) 
+ \left[\frac{2^3}{3^3} (-42/21 -1)  + \frac{56}{63} \right] b_{10}(r,s,t) \nonumber\\
&=& 1 + \left[\frac{2^3}{3^3} (-3) + \frac{8}{9} \right] b_9(r,s,t) 
+ \left[\frac{2^3}{3^3} (-3)  + \frac{8}{9} \right] b_{10}(r,s,t) \nonumber\\
&=& 1
\end{eqnarray}
Let us move to first order consistency with $f(r)=r$:
\begin{eqnarray}
f^h(r,s,t)
=\sum_{i=1}^{10} \bN_i(r,s,t) f_i
=\sum_{i=1}^{10} \bN_i(r,s,t) r_i 
\end{eqnarray}
It has been established for the $\tilde{\bN}_i$ functions so we are left with
\begin{eqnarray}
f^h(r,s,t)
&=& \underbrace{\sum_{i=1}^8 \tilde{\bN}_i r_i }_{=r} \nn\\
&& -\frac{2^3}{3^3}b_9(r,s,t) (-1)  \nn\\
&& -\frac{2^3}{3^3}\frac{10}{21} b_9(r,s,t) (+1) - \frac{2^3}{3^3}\frac{4}{21} b_{10}(r,s,t) (+1)  \nn\\
&& -\frac{2^3}{3^3}\frac{4}{21} b_9(r,s,t)  (+1)-\frac{2^3}{3^3}\frac{10}{21} b_{10}(r,s,t) (+1)     \nn\\
&&-\frac{2^3}{3^3}\frac{10}{21} b_{9}(r,s,t) (-1) -\frac{2^3}{3^3}\frac{4}{21} b_{10}r,s,t) (-1)     \nn\\
&&- \frac{2^3}{3^3}\frac{10}{21} b_{9}(r,s,t) (-1) -\frac{2^3}{3^3}\frac{4}{21} b_{10}(r,s,t)  (-1)    \nn\\
&& -\frac{2^3}{3^3}\frac{4}{21} b_9(r,s,t) (+1) -\frac{2^3}{3^3}\frac{10}{21} b_{10}(r,s,t) (+1)  \nn\\
&& - \frac{2^3}{3^3}b_{10}(r,s,t) (+1)     \nonumber\\
&&-\frac{2^3}{3^3}\frac{4}{21} b_9(r,s,t) (-1) -\frac{2^3}{3^3}\frac{10}{21} b_{10}(r,s,t) (-1)      \nonumber\\
&& +  \frac{64}{63}b_9(r,s,t)(-1/3) -\frac{8}{63} b_{10}(r,s,t) (-1/3)     \nn\\
&&  -\frac{8}{63} b_{9}(r,s,t)(+1/3)  + \frac{64}{63}b_{10}(r,s,t)   (+1/3) \nn\\
&=& r + 
b_9(r,s,t) \left( 
\frac{8}{27} -\frac{8}{27}\frac{10}{21} -\frac{8}{27}\frac{4}{21} 
+\frac{8}{27}\frac{10}{21}
+\frac{8}{27}\frac{10}{21} 
-\frac{8}{27}\frac{4}{21}  
+\frac{8}{27}\frac{4}{21}
-\frac{64}{189}-\frac{8}{189}
\right) \nn\\ 
&+& b_{10}(r,s,t) \left(
-\frac{8}{27} \frac{4}{21} 
-\frac{8}{27}\frac{10}{21} 
+ \frac{8}{27}\frac{4}{21} 
+ \frac{8}{27}\frac{4}{21} 
-\frac{8}{27}\frac{10}{21} 
-\frac{8}{27}
+\frac{8}{27}\frac{10}{21}
+\frac{8}{189}
+\frac{64}{189}
\right) \nn\\
&=& r +  
b_9(r,s,t) \frac{8}{27} 
\underbrace{\left( 1 - \frac{10}{21} - \frac{4}{21} + \frac{10}{21} + \frac{10}{21} 
-\frac{9}{7} \right)}_{=0}
+b_{10}(r,s,t) \frac{8}{27} 
\underbrace{\left( 
- \frac{4}{21} -\frac{10}{21} + \frac{4}{21} + \frac{4}{21} - \frac{10}{21} -1 +  \frac{10}{21} + \frac{9}{7}
\right)}_{=0} \nn\\
&=& r 
\end{eqnarray}
which proves first-order consistency.





The derivatives of the $\tilde{\bN}_i$ basis functions are already established so we only focus on the spatial derivatives of the bubble functions:
\begin{eqnarray}
\frac{\partial b_9}{\partial r} 
%&=& \left(\frac{27}{32}\right)^3 (1-s^2)(1-t^2) (1-s)(1-t) 
%\frac{\partial }{\partial r} (1-r-r^2+r^3  ) \nonumber\\
&=& \left(\frac{27}{32}\right)^3 (1-s^2)(1-t^2) (1-s)(1-t) (-1-2r+3r^2  )\nonumber\\
\frac{\partial b_9}{\partial s}
&=& \left(\frac{27}{32}\right)^3 (1-r^2)(1-t^2) (1-r)(1-t) (-1-2s+3s^2  )\nonumber\\
\frac{\partial b_9}{\partial t}
&=& \left(\frac{27}{32}\right)^3 (1-r^2)(1-s^2) (1-r)(1-s) (-1-2t+3t^2  )\nonumber\\
\frac{\partial b_{10}}{\partial r} 
%&=& \left(\frac{27}{32}\right)^3 (1-s^2)(1-t^2) (1+s)(1+t) 
%\frac{\partial }{\partial r} (1+r-r^2-r^3  )\nonumber\\
&=& \left(\frac{27}{32}\right)^3 (1-s^2)(1-t^2) (1+s)(1+t) (1-2r-3r^2  )\nonumber\\
\frac{\partial b_{10}}{\partial s}
&=& \left(\frac{27}{32}\right)^3 (1-r^2)(1-t^2) (1+r)(1+t) (1-2s-3s^2  )\nonumber\\
\frac{\partial b_{10}}{\partial t} 
&=& \left(\frac{27}{32}\right)^3 (1-r^2)(1-s^2) (1+r)(1+s) (1-2t-3t^2  ) \nonumber
\end{eqnarray}



\Literature: \textcite{fofo85} (1985), \textcite{sofo87} (1987)

Bishnu talksabout Nitsche bc ? press error near boundary in fofo85


















\newpage
%-----------------------------------------------------------------------------
\subsection{The DSSY element} \label{ss:dssy_3D}
\input{dssy3D}


