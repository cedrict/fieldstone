\begin{flushright} {\tiny {\color{gray} fdm\_adv1D.tex}} \end{flushright}

%TODO to improve notes:
%use http://farside.ph.utexas.edu/teaching/329/lectures/node91.html
%


The 1D hyperbolic advection equation is:
\begin{equation}
\rho C_p \left( \frac{\partial T}{\partial t}  
+ u \frac{\partial T}{\partial x} \right)=0 
\end{equation}
or simply
\begin{equation}
\frac{\partial T}{\partial t} + u \frac{\partial T}{\partial x}=0 
\end{equation}
We have seen how to deal with the time derivative (explicit, implicit) 
and with the first order space derivative (forward, backward or central).
Let us consider the FTCS scheme (Forward in Time, Central in Space).
\[
\frac{T_{\color{teal}i}^{n+1}-T^n_{\color{teal}i}}{\delta t} 
+ u_i \frac{T^n_{\color{teal}i+1} - T^n_{{\color{teal}i-1}}}{2h} =0 
\]
or,
\[
T_i^{n+1} = T_i^n - \frac{u_i \delta t}{2 h} (T_{i+1}^n-T_{i-1}^n)
\]
Note that although the velocity $u$ is prescribed, it can vary in space, hence
the subscript $i$. 



There is however a major problem: 
the FTCS method is in this case {\bf unconditionally} {\bf un}stable (see Section 6.2.1 
of \cite{hoch}, section 4.3.1 of \cite{pell08}), i.e., it blows up for any $\delta t$.
The instability is related to the fact that this scheme produces negative diffusion, 
which is numerically unstable.


Finally, The Crank-Nicolson implicit scheme for solving the diffusion equation 
can be adapted to solve the advection equation:

\[
\frac{T_{\color{teal}i}^{n+1}-T^n_{\color{teal}i}}{\delta t} 
=- u_i 
\frac12 \left(
\frac{T^n_{\color{teal}i+1} - T^n_{{\color{teal}i-1}}}{2h} 
+
\frac{T^{n+1}_{\color{teal}i+1} - T^{n+1}_{{\color{teal}i-1}}}{2h} 
\right)
\]
or,
\[
T_{\color{teal}i}^{n+1} + \frac{u_i \delta t}{4h} (T_{{\color{teal}i+1}}^{n+1}-T_{{\color{teal}i-1}}^{n+1}) 
= T_{\color{teal}i}^n - \frac{u \delta t}{4h} (T_{{\color{teal}i+1}}^{n}-T_{{\color{teal}i-1}}^{n}) 
\]
which then makes the method implicit.








We could also consider the FTFS method:
\[
\frac{T_{\color{teal}i}^{n+1}-T^n_{\color{teal}i}}{\delta t} 
+ u_i \frac{T^n_{\color{teal}i+1} - T^n_{{\color{teal}i}}}{h} =0 
\]
but it is also {\bf unconditionally} {\bf un}stable (see Section 6.2.1 of \cite{hoch}).

We will now look at to methods which alleviate this problem:

\index{general}{Lax-Friedrichs method}
\begin{itemize}
\item The {\color{olive} Lax-Friedrichs method}\footnote{Named after Peter Lax 
and Kurt O. Friedrichs. \url{https://en.wikipedia.org/wiki/Lax-Friedrichs_method}} consists of replacing the $T_{\color{teal}i}^n$ 
in the time derivative term with $(T_{{\color{teal}i+1}}^n + T_{{\color{teal}i-1}}^n)/2$
(see for instance Section 4.3.1 of \cite{pell08} in the context of surface processes). 
The resulting equation is
\[
\frac{T_{\color{teal}i}^{n+1}-  (T_{{\color{teal}i+1}}^n + T_{{\color{teal}i-1}}^n)/2 }{\delta t} 
= - u_i \frac{T^n_{{\color{teal}i+1}}-T^n_{{\color{teal}i-1}}}{2 h}
\]
or, 
\[
T_{\color{teal}i}^{n+1} = \frac{1}{2} (T_{{\color{teal}i+1}}^n + T_{{\color{teal}i-1}}^n)  
- \frac{u_i \delta t}{h}  \frac{1}{2} (T^n_{{\color{teal}i+1}}-T^n_{{\color{teal}i-1}})
\]
von Neumann stability analysis indicates that this method is stable
when $C=u \delta t/h \leq 1$ where $C$ is the Courant number.
\item In the {\color{olive}Streamline upwind} method the spatial finite difference scheme 
depends on the sign of the velocity:
\[
\frac{T_{\color{teal}i}^{n+1}-  (T_{{\color{teal}i+1}}^n + T_{{\color{teal}i-1}}^n)/2   }{\delta t} =
\left\{
\begin{array}{l}
 - u_i \frac{T^n_{{\color{teal}i}}-T^n_{{\color{teal}i-1}}}{h_x}  \quad\quad  {\rm if} \quad u_i<0\\ \\
 - u_i \frac{T^n_{{\color{teal}i+1}}-T^n_{{\color{teal}i}}}{h_x}  \quad\quad  {\rm if} \quad u_i>0
\end{array}
\right.
\]
In fact, we have replaced central with forward or backward derivatives, depending on the flow direction. 
This method is stable when $C=u \delta t/h \leq 1$. 

\item {\color{olive} Lax-Wendroff method}\footnote{Named after Peter Lax 
and Burton Wendroff \url{https://en.wikipedia.org/wiki/Lax-Wendroff_method}} \cite{hoch}
is second-order accurate in both space and time. 
This method is an example of explicit time integration where the function that defines the governing equation is evaluated at the current time. 
\[
T^{n+1}_i = T_i^n - \frac{u \delta t}{2 h} (T_{i+1}^n-T_{i-1}^n)
+ \frac{u^2 dt^2}{2 h^2} (T_{i+1}^n-2 T_i^n+T_{i-1}^n)
\]








\end{itemize}
These are not the only possibilities, see for instance 
the {\color{olive} leapfrog method} or the
MacCormack method\footnote{\url{https://en.wikipedia.org/wiki/MacCormack_method}} (well suited for nonlinear equations).  
\index{general}{Leapfrog method}
\index{general}{Lax-Wendroff method}




%-/-/-/-/-/-/-/-/-/-/-/-/-/-/-/-/-/-/-/
\begin{center}
\begin{minipage}[t]{0.77\textwidth}
\par\noindent\rule{\textwidth}{0.4pt}

\begin{center}
\includegraphics[width=0.8cm]{images/garftr} \\
{\color{orange}Exercise FDM-6}
\end{center}

Let us consider the domain $[0,1]$. The temperature field at $t=0$ is 
given by $T=1$ for $x<0.25$ and $T=0$ otherwise. The prescribed 
velocity is $u=1$ and we set $nnx=51$.
Boundary conditions are $T=1$ at $x=0$ and $T=0$ at $x=1$.

\begin{center}
\input{tikz/tikz_fdm1Df}
\end{center}

Program the above FTCS method. Run the model for 250 time steps with $\delta t=0.002$. 
Program the Lax-Friedrichs method by modifying the previous code.\\
Bonus: Program the upwind method and/or the Crank-Nicolson method. 

\par\noindent\rule{\textwidth}{0.4pt}
\end{minipage}
\end{center}
%-/-/-/-/-/-/-/-/-/-/-/-/-/-/-/-/-/-/



