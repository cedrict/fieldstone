\begin{flushright} {\tiny {\color{gray} \tt pair\_crouzeixraviart.tex}} \end{flushright}
%~~~~~~~~~~~~~~~~~~~~~~~~~~~~~~~~~~~~~~~~~~~~~~~~~~~~~~~~~~~~~~~~~~~~~~~~~~~~~~~~~~~~~~~~~~~~~~~~~~

Since the ${\bm P}_2\times P_{-1}$ pair is not LBB stable \cite[p179]{reddybook2}, 
(see also table 3.13-1 of \textcite{grsa})
it is enhanced by a cubic bubble and is therefore called ${\bm P}_2^+\times P_{-1}$. 

This element was first introduced in \textcite{crra73} (1973).
It is a seven-node triangle (in 2D) with quadratic velocity shape 
functions enhanced by a cubic bubble function and discontinuous linear interpolation for 
the pressure field \cite{cuss86}. 
This element is LBB stable and no additional stabilization techniques are required \cite{elsw}.
The '+' in its name stands for the bubble while the '-' stands for the discontinuous
character of the pressure field: it is $P_1$ over the element, but discontinuous
across element edges.

\begin{remark}
Cuvelier \etal, 1986 \cite{cuss86} recommend a 6-point or 7-point quadrature rule for this element.
\end{remark}

\begin{remark}
Segal \cite{segal} explains 
for output purposes (printing, plotting etc.) the discontinuous pressures are averaged 
in vertices for all the adjoining elements. See also Fig. 7.3 of \cite{cuss86}.
\end{remark}

\begin{remark}
The simplest Crouzeix-Raviart element is the non-conforming linear triangle 
with constant pressure \cite{cuss86}, see Section~\ref{ss:p1ncp0}.
\end{remark}

It is worth noting that this element has more degrees of freedom  than the 
Taylor-Hood element for the same order of accuracy. However, since the 
bubble can be eliminated, one can design a modified version of this element.
\todo[inline]{Check Cuvelier book chapter 8 for modified element}

\begin{remark}
I have once asked the (main) author of MILAMIN why he chose this element, for 
example over the ${\bm P}_2\times P_1$. His answer is as follows:
{\color{darkgray} ``Elements with continuous pressure  are incapable of converging in the Linf 
norm for mechanical problems exhibiting pressure jumps such as the inclusion-host setup. 
During my MSc and PhD I was focusing on sharp heterogeneities, so this is why I decided 
to choose ${\bm P}_2^+\times P_{-1}$. 
You will see that it is also easy to invert the pressure mass matrix for such elements, 
which is really useful (both for the augmentation and preconditioning).''}
\end{remark}

This element is used by \textcite{popo92} (1992) to study the deformation of 
the surface above a rising diapir. Note that they actually use a 
{\color{darkgray} ``13 point integration formula (Hughes 1987) for calculation of the stiffness matrix was 
used in order to conserve detailed information from the marker field in the coarse FEM mesh''}. 
It is the element used in the MILAMIN code, see \textcite{daks08} (2008).
It is also featured in \textcite{thba24} (2024).
It is also used in \textcite{anmp15} (2015) in the context of a new free-surface stabilization scheme. 
It is the element used in LaCoDe, see \textcite{demh19} (2019).
It is mentioned in Section~6.2 in \textcite{bobf08} (2008).
It is compared to the ${\bm P}_2\times P_1$ element for the Navier-Stokes equations in 
\textcite{krba05} (2005).

\begin{center}
\includegraphics[width=6cm]{images/pair_cr/cr}\\
{\captionfont Taken from \textcite{begt92} (1992).}
\end{center}

This element counts $(4+6+4+1)*3=45$ vdofs and 4 pdofs.

The 3D version of this element is presented in \textcite{begt92} (1992). 
In their Table III we find the following node coordinates and basis functions:
\[
\begin{array}{lll}
\text{\# node} & \text{coordinates} & \text{basis function} \\
1  & (0,0,0) & \bN_1(r,s,t)=(1-r-s-t) -\frac12 (\bN_5+\bN_7+\bN_8)-\frac13(\bN_{11}+\bN_{12}+\bN_{13})-\bN_{15}/4 \\ 
2  & (1,0,0) & \bN_2(r,s,t)=r-\frac12(\bN_5+\bN_6+\bN_9)    -\frac13(\bN_{11}+\bN_{12}+\bN_{14}) -\bN_{15}/4 \\
3  & (0,1,0) & \bN_3(r,s,t)=s-\frac12(\bN_6+\bN_7+\bN_{10}) -\frac13(\bN_{11}+\bN_{13}+\bN_{14}) -\bN_{15}/4 \\
4  & (0,0,1) & \bN_4(r,s,t)=t-\frac12(\bN_8+\bN_9+\bN_{10}) -\frac13(\bN_{12}+\bN_{13}+\bN_{14}) -\bN_{15}/4 \\
5  & (\frac12,0,0)             &  \bN_5(r,s,t)=4(1-r-s-t)r-\frac49(\bN_{11}+\bN_{12})-\bN_{15}/4  \\
6  & (\frac12,\frac12,0)       &  \bN_6(r,s,t)=4rs-\frac49(\bN_{11}+\bN_{14})-\bN_{15}/4  \\
7  & (0,\frac12,0)             &  \bN_7(r,s,t)=4(1-r-s-t)s-\frac49(\bN_{11}+\bN_{13})-\bN_{15}/4  \\
8  & (0,0,\frac12)             &  \bN_8(r,s,t)=4(1-r-s-t)t-\frac49(\bN_{12}+\bN_{13})-\bN_{15}/4  \\
9  & (\frac12,0,\frac12)       &  \bN_9(r,s,t)=4rt-\frac49(\bN_{12}+\bN_{14})-\bN_{15}/4  \\
10 & (0,\frac12,\frac12)       &  \bN_{10}(r,s,t)= 4st-\frac49(\bN_{13}+\bN_{14})-\bN_{15}/4 \\
11 & (\frac13,\frac13,0)       &  \bN_{11}(r,s,t)= 27(1-r-s-t)rs-\frac{108}{256}\bN_{15} \\
12 & (\frac13,0,\frac13)       &  \bN_{12}(r,s,t)= 27(1-r-s-t)rt-\frac{108}{256}\bN_{15} \\
13 & (0,\frac13,\frac13)       &  \bN_{13}(r,s,t)= 27(1-r-s-t)st-\frac{108}{256}\bN_{15} \\
14 & (\frac13,\frac13,\frac13) &  \bN_{14}(r,s,t)= 27rst-\frac{108}{256}\bN_{15} \\
15 & (\frac14,\frac14,\frac14) &  \bN_{15}(r,s,t)= 256(1-r-s-t)rst \\
\end{array}
\]

We have
\begin{eqnarray}
\bN_1+\bN_2+\bN_3+\bN_4 
&=& (1-r-s-t) -\frac12 (\bN_5+\bN_7+\bN_8)-\frac13(\bN_{11}+\bN_{12}+\bN_{13})-\bN_{15}/4 \nn\\
&+& r-\frac12(\bN_5+\bN_6+\bN_9)    -\frac13(\bN_{11}+\bN_{12}+\bN_{14}) -\bN_{15}/4 \\
&+& s-\frac12(\bN_6+\bN_7+\bN_{10}) -\frac13(\bN_{11}+\bN_{13}+\bN_{14}) -\bN_{15}/4 \\
&+& t-\frac12(\bN_8+\bN_9+\bN_{10}) -\frac13(\bN_{12}+\bN_{13}+\bN_{14}) -\bN_{15}/4 \\
&=& 1 -\frac12(1+1)\bN_5 -\frac12(1+1)\bN_6 -\frac12(1+1)\bN_7 -\frac12(1+1)\bN_8 -\frac12(1+1)\bN_9 -\frac12(1+1)\bN_{10}    \nn\\
&&  -\frac13(1+1+1)\bN_{11} -\frac13(1+1+1)\bN_{12} -\frac13(1+1+1)\bN_{13} -\frac13(1+1+1)\bN_{14} 
-\frac{1}{4}(1+1+1+1)\bN_{15} \nn\\
&=& 1 -\bN_5 -\bN_6 -\bN_7 -\bN_8 -\bN_{9} -\bN_{10}
      -\bN_{11} -\bN_{12} -\bN_{13} -\bN_{14} -\bN_{15} 
\end{eqnarray}
So in the end
\[
\sum_{i=1}^{15} \bN_i = (\bN_1+\bN_2+\bN_3+\bN_4)  + \sum_{i=5}^{15} \bN_i = 1
\]

\begin{center}
\url{https://defelement.com/elements/conforming-crouzeix-raviart.html}
\end{center}



