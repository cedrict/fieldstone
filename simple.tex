\index{general}{SIMPLE} 
\begin{flushright} {\tiny {\color{gray} simple.tex}} \end{flushright}
%~~~~~~~~~~~~~~~~~~~~~~~~~~~~~~~~~~~~~~~~~~~~~~~~~~~~~~~~~~~~~~~~~~~~~~~~~~~~~~~~~~~~~~~~~~~~~~~~~~


What follows is borrowed from \fullcite{john16}, page 666. 


The SIMPLE method (Semi-Implicit Method for Pressure-Linked Equations)
has been introduced by Patankar \& Spalding (1972) \cite{pasp72} as an iterative method to solve
the finite volume discretized incompressible Navier-Stokes equations. 

The algorithm
is based on the following steps:
\begin{itemize}
\item First the pressure is assumed to be known from the previous iteration.
\item Then the velocity is solved from the momentum equations. The newly obtained
velocities do not satisfy the continuity equation since the pressure is only a
guess.
\item In the next substeps the velocities and pressures are corrected in order to
satisfy the discrete continuity equation.
\end{itemize}

SIMPLE relies on the block LU decomposition
\begin{equation}
\left(
\begin{array}{cc}
\K & \G \\ \G^T & -\C 
\end{array}
\right)
\cdot
\left(
\begin{array}{c}
\vec{\cal V} \\ \vec{\cal P}
\end{array}
\right)
=
\left(
\begin{array}{cc}
\K & 0 \\ \G^T & -\SSS
\end{array}
\right)
\cdot
\left(
\begin{array}{cc}
{\bm I} & \K^{-1} \cdot \G \\
0 & {\bm I} 
\end{array}
\right)
\cdot
\left(
\begin{array}{c}
\vec{\cal V} \\ \vec{\cal P}
\end{array}
\right)
=
\left(
\begin{array}{c}
\vec{f} \\ \vec{h}
\end{array}
\right)
\end{equation}

The approximation $\K^{-1}$ as ${\bm D}_\K^{-1} = (diag(\K))^{-1}$ leads to the 
SIMPLE algorithm. In this case the approximation of the Schur complement matrix is given by
$\tilde{\SSS} = \G^T\cdot  {\bm D}_\K^{-1} \cdot \G$  and the decomposition looks like
\[
\left(
\begin{array}{cc}
\K & \G \\ \G^T & -\C 
\end{array}
\right)
\simeq
\left(
\begin{array}{cc}
\K & 0 \\ 
\G^T & -\tilde{\SSS}
\end{array}
\right)
\cdot
\left(
\begin{array}{cc}
{\bm I} & {\bm D}_\K^{-1} \cdot \G \\
0 & {\bm I} 
\end{array}
\right)
\]
Thus one iteration of SIMPLE solves the following system:
\[
\left(
\begin{array}{cc}
\K & \G \\ \G^T & -\C 
\end{array}
\right)
\simeq
\left(
\begin{array}{cc}
\K & 0 \\ 
\G^T & -\tilde{\SSS}
\end{array}
\right)
\cdot
\left(
\begin{array}{cc}
{\bm I} & {\bm D}_\K^{-1} \cdot \G \\
0 & {\bm I} 
\end{array}
\right)
\cdot
\left(
\begin{array}{c}
\vec{\cal V} \\ \vec{\cal P}
\end{array}
\right)
=
\left(
\begin{array}{c}
\vec{f} \\ \vec{h}
\end{array}
\right)
\]





Before we can write out the SIMPLE algorithm, we must first take a small detour via so-called
distributive iterative methods \cite{vusb00,tack10}. \index{general}{Distributive Iterative Method}
Let us consider the linear system 
\[
{\bm A}\cdot \vec{x}=\vec{b}
\] 
A stationary iterative method is defined as follows:
\[
\vec{x}^{k+1}= {\bm B}\cdot \vec{x}^{k}+ \vec{c}
\]
where $\vec{c}=({\bm I}-{\bm B})\cdot {\bm A}^{-1}\cdot \vec{b}$. 
Left-multiplying all terms by $({\bm I}-{\bm B})^{-1}$ first and then left-multiplying again 
by ${\bm A}$  we arrive at:
\[
{\bm A}\cdot ({\bm I}-{\bm B})^{-1}\cdot \vec{x}^{k+1}
={\bm A}\cdot ({\bm I}-{\bm B})^{-1}\cdot {\bm B}\cdot 
\vec{x}^{k} + {\bm A}\cdot ({\bm I}-{\bm B})^{-1} \cdot \vec{c}
\]
We define ${\bm M}={\bm A}\cdot ({\bm I}-{\bm B})^{-1} $ so that now
\[
{\bm M}\cdot\vec{x}^{k+1}={\bm M}\cdot {\bm B}\cdot \vec{x}^{k}+\vec{b} 
\]
We define ${\bm N}={\bm M}\cdot {\bm B}$
and finally 
\[
{\bm M}\cdot\vec{x}^{k+1}={\bm N}\cdot \vec{x}^{k}+\vec{b} 
\]
Note that ${\bm M}-{\bm N}={\bm M}-{\bm M}\cdot {\bm B}
= {\bm M}\cdot  ({\bm I}-{\bm B}) 
= {\bm A}\cdot ({\bm I}-{\bm B})^{-1}\cdot ({\bm I}-{\bm B}) 
= {\bm A}$. 
Let us now write the original system 
${\bm A}\cdot \vec{x}=\vec{b}$ as $({\bm A}\cdot {\bm B})\cdot ({\bm B}^{-1}\cdot \vec{x})=\vec{b}$
or, $ \underline{\bm A}\cdot  \underline{\vec{x}}=\vec{b} $
with 
$\vec{x}={\bm B}\cdot \underline{\vec{x}}$
and 
$\underline{\bm A}={\bm A}\cdot {\bm B}$.
Splitting $\underline{\bm A}={\bm M}-{\bm N}$ again yields 
\[
{\bm M}\cdot \underline{\vec{x}}^{k+1}={\bm N}\cdot \underline{\vec{x}}^{k}+\vec{b} 
\]
Using $\vec{x}={\bm B}\cdot \underline{\vec{x}}$, we get 
\[
{\bm M}\cdot {\bm B}^{-1}\cdot \vec{x}^{k+1} = {\bm N}\cdot {\bm B}^{-1}\cdot  \vec{x}^{k}+\vec{b} 
\]
We then have 
\begin{eqnarray}
\vec{x}^{k+1}
&=&  {\bm B}\cdot {\bm M}^{-1} \cdot[ {\bm N} \cdot  {\bm B}^{-1}  \cdot \vec{x}^{k}+ \vec{b}  ] \\
&=&  {\bm B}\cdot {\bm M}^{-1} \cdot[ ({\bm M} - \underline{\bm A}) \cdot  {\bm B}^{-1}\cdot \vec{x}^{k}+\vec{b}  ]\\
&=&  {\bm B}\cdot {\bm M}^{-1} \cdot[ ({\bm M} - {\bm A}\cdot {\bm B}) \cdot  {\bm B}^{-1}\cdot  \vec{x}^{k}+ \vec{b}  ]\\
&=&  {\bm B}\cdot {\bm M}^{-1} \cdot[ {\bm M}\cdot {\bm B}^{-1} \cdot \vec{x}^{k} - {\bm A}\cdot {\bm B}\cdot  {\bm B}^{-1} \cdot \vec{x}^{k}+\vec{b}  ]\\
&=&  {\bm B}\cdot {\bm M}^{-1} \cdot[ {\bm M}\cdot {\bm B}^{-1} \cdot \vec{x}^{k} - {\bm A}\cdot \vec{x}^{k}+ \vec{b}  ]\\
&=&  \vec{x}^k + {\bm B}\cdot {\bm M}^{-1}\cdot [ \vec{b}    - {\bm A} \cdot \vec{x}^{k}  ]
\end{eqnarray}
Finally, we have the following recursion:
\begin{equation}
\boxed{\vec{x}^{k+1} = \vec{x}^k +{\bm B} \cdot {\bm M} ^{-1}\cdot (\vec{b} -{\bm A}\cdot \vec{x}^{k}  ) }
\label{eq:simplerec}
\end{equation}
Coming back to the SIMPLE algorithm, we start from 
\[
{\bm A}=
\left(
\begin{array}{cc}
\K & \G \\
\G^T & 0
\end{array}
\right)
\]
The matrix ${\bm B}$ is then chosen to be 
\[
{\bm B}=
\left(
\begin{array}{cc}
{\bm I} & -\K^{-1} \G \\
0 & {\bm I}
\end{array}
\right)
\]
We then have 
\[
{\bm A}\cdot  {\bm B} = 
\left(
\begin{array}{cc}
\K & \G \\
\G^T & 0
\end{array}
\right)
\cdot 
\left(
\begin{array}{cc}
{\bm I} & -\K^{-1} \G \\
0 & {\bm I}
\end{array}
\right)
=
\left(
\begin{array}{cc}
\K & 0 \\
\G^T & -\SSS
\end{array}
\right)
\]
where $\SSS=\G^T \cdot \K^{-1} \cdot \G$.
Let us recall that we define ${\bm D}_\K =diag(\K)$ and $\hat{\SSS}=\G^T \cdot {\bm D}_\K^{-1} \cdot \G$. 
We further define 
\[
{\bm M}=
\left(
\begin{array}{cc}
\K & 0 \\
\G^T & -\hat{\SSS}
\end{array}
\right)
\]
and ${\bm N}$ follows from the splitting ${\bm A}\cdot {\bm B}= {\bm M} - {\bm N}$. 
(Note that we do not need to form nor use ${\bm N}$).

The standard SIMPLE algorithm also replaces $\K^{-1}$  by  ${\bm D}_\K^{-1}$ in ${\bm B}$ so that 
${\bm B}$ is approximated by:
\[
{\bm B}=
\left(
\begin{array}{cc}
{\bm I} & -{\bm D}_\K^{-1} \G \\
0 & {\bm I}
\end{array}
\right)
\]
in the iterations.
We can define 
\[
\vec{r}^k=
\vec{b}-{\bm A}\cdot \vec{x}^k = 
\left(
\begin{array}{c}
\vec{f} \\ \vec{h}
\end{array}
\right)
-
\left(
\begin{array}{cc}
\K & \G \\
\G^T & 0
\end{array}
\right)
\cdot
\left(
\begin{array}{c}
\vec{\cal V}^k \\ \vec{\cal P}^k
\end{array}
\right)
=
\left(
\begin{array}{c}
\vec{r}_{\cal V}^k \\ \vec{r}_{\cal P}^k
\end{array}
\right)
\]

The iteration loop of Eq.~\eqref{eq:simplerec} then takes the form 
\[
\left(
\begin{array}{c}
\vec{\cal V}^{k+1} \\ 
\vec{\cal P}^{k+1}
\end{array}
\right)
=
\left(
\begin{array}{c}
V^k \\ P^k
\end{array}
\right)
+ 
{\bm B}  {\bm M} ^{-1}
\left(
\begin{array}{c}
r_V^k \\ r_P^k
\end{array}
\right)
=
\left(
\begin{array}{c}
V^k \\ P^k
\end{array}
\right)
+ 
\left(
\begin{array}{c}
\delta V^k \\ \delta P^k
\end{array}
\right)
\quad\quad
\textrm{with}\quad\quad
\left(
\begin{array}{c}
\delta V^k \\ \delta P^k
\end{array}
\right)
=
{\bm B}  {\bm M} ^{-1}
\left(
\begin{array}{c}
\vec{r}_{\cal V}^k \\ 
\vec{r}_{\cal P}^k
\end{array}
\right)
\]
This last equation can be rewritten\footnote{Remember 
that $({\bm A}\cdot {\bm B})^{-1}={\bm B}^{-1}\cdot {\bm A}^{-1}$}:
\[
{\bm M} \cdot 
\left[ {\bm B}^{-1} \cdot 
\left(
\begin{array}{c}
\delta \vec{\cal V}^k \\ 
\delta \vec{\cal P}^k
\end{array}
\right)
\right]
=
\left(
\begin{array}{c}
\vec{r}_{\cal V}^k \\ 
\vec{r}_{\cal P}^k
\end{array}
\right)
\]
We then have to solve 
\begin{equation}
{\bm M} 
\cdot
\left(
\begin{array}{c}
\delta^\star \vec{\cal V}^k \\ 
\delta^\star \vec{\cal P}^k
\end{array}
\right)
=
\left(
\begin{array}{cc}
\K & 0 \\
\G^T & -\hat{\SSS}
\end{array}
\right)
\cdot
\left(
\begin{array}{c}
\delta^\star \vec{\cal V}^k \\ 
\delta^\star \vec{\cal P}^k
\end{array}
\right)
=
\left(
\begin{array}{c}
\vec{r}_{\cal V}^k \\ 
\vec{r}_{\cal P}^k
\end{array}
\right)
\label{simple1aa}
\end{equation}
and then compute
\begin{equation}
\left(
\begin{array}{c}
\delta \vec{\cal V}^k \\ 
\delta \vec{\cal P}^k
\end{array}
\right)
=
{\bm B}  
\left(
\begin{array}{c}
\delta^\star \vec{\cal V}^k \\ 
\delta^\star \vec{\cal P}^k
\end{array}
\right)
\label{simple2aa}
\end{equation}
Fortunately Eq.~\eqref{simple1aa} translates into:
\begin{eqnarray}
\K \cdot \delta^\star \vec{\cal V}^k &=&  \vec{r}_{\cal V}^k   \\
\hat{\SSS} \cdot  \delta^\star \vec{\cal P}^k &=&  - \vec{r}_P^k + \G^T \cdot \delta^\star \vec{V}^k 
\end{eqnarray}
and Eq.~\eqref{simple2aa} translates into:
\[
\left(
\begin{array}{c}
\delta \vec{\cal V}^k \\ 
\delta \vec{\cal P}^k
\end{array}
\right)
=
\left(
\begin{array}{cc}
{\bm I} & -{\bm D}_\K^{-1}\cdot \G \\
0 & {\bm I}
\end{array}
\right)
\cdot
\left(
\begin{array}{c}
\delta^\star \vec{\cal V}^k \\ 
\delta^* \vec{\cal P}^k
\end{array}
\right)
\]
or, 
\begin{eqnarray}
\delta \vec{\cal V}_k &=& \delta^\star \vec{\cal V}^k 
-{\bm D}_\K^{-1}\cdot \G \cdot\delta^\star \vec{\cal P}_k \\
\delta \vec{\cal P}_k &=& \delta^\star \vec{\cal P}^k
\end{eqnarray}


The final algorithm will then look as follows:

\begin{mdframed}[backgroundcolor=blue!5]
\begin{enumerate}
\item compute the residuals 
\begin{eqnarray}
\vec{r}_{\cal V} &=& \vec{f} - \K \cdot \vec{\cal V}^{(k)} - \G \cdot \vec{\cal P}^{(k)} \nn\\
\vec{r}_{\cal P} &=& \vec{h} - \G^T \cdot \vec{\cal V}^{(k)}
\end{eqnarray}
\item Solve $\K  \cdot \delta^\star \vec{\cal V}^k =  \vec{r}_{\cal V}^k  $
\item Solve $\hat{\SSS} \cdot \delta^\star \vec{\cal P}^k =  \vec{r}_{\cal P}^k - \G^T \cdot  \delta^\star V^k $
\item Compute $\delta \vec{\cal V}^k = \delta^\star \vec{\cal V}^k -{\bm D}_\K^{-1} \cdot \G \cdot \delta^\star \vec{\cal P}_k $
\item Update $\delta \vec{\cal P}^k = \delta^\star \vec{\cal P}^k$
\item Update 
\begin{eqnarray}
\vec{\cal V}^{(k+1)} &=& \vec{\cal V}^{(k)} + \omega_{\cal V} \delta \vec{\cal V}^{(k)} \nn\\
\vec{\cal P}^{(k+1)} &=& \vec{\cal P}^{(k)} + \omega_{\cal P} \delta \vec{\cal P}^{(k)} 
\end{eqnarray}
\end{enumerate}
\end{mdframed}
where the parameters $\omega_{\cal V}$ and $\omega_{\cal P}$ are between 0 and 1. 

Note that SIMPLE can be used as left and as right preconditioner, see page 669 of \textcite{john16}.

Also, John states that:
``SIMPLE is easily to implement, which makes
it attractive. It relies on the already assembled matrix blocks. Only the approximation 
$\hat{\SSS}$ of the Schur complement matrix has to be computed. This
matrix couples pressure degrees of freedom that are usually not coupled in finite
element approximations of the diffusion operator, but it is still a sparse matrix.
The efficiency of SIMPLE depends on how good $\K^{-1}$ is approximated by its
diagonal.''

\vspace{.5cm}

\Literature: 
\begin{itemize}
\item \fullcite{eche13}, 
\item \fullcite{brsa97b}, 
\item \fullcite{urvs09} for SIMPLE(R) algorithm, 
\item \fullcite{tack10}
\end{itemize}




