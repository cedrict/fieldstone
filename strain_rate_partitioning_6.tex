

\begin{center}
\input{tikz/tikz_vvp2}
\end{center}

The algorithm goes then as follows:
\begin{enumerate}
\item Assume we know $\dot\varepsilon_T$ (from previous iteration), 
as well as the plasticity parameters $Y$ (a constant in the case of von Mises, or a pressure-dependent 
quantity otherwise) and $\eta_m$.
\item We start by assuming that the plasticity 'block' is not active ($\dot{\varepsilon}_{vp}=0$): we have then three dampers in series. 
We need their associated strain rates 
$\dot{\varepsilon}_{df}$ and $\dot{\varepsilon}_{ds}$ which are such that 
\[
\dot\varepsilon_T = 
\dot\varepsilon_v + \dot\varepsilon_{ds} + \dot\varepsilon_{df} 
\]
with
\begin{eqnarray}
\dot\varepsilon_v &=& \frac{\tau}{2 \eta_v} \label{sr_v} \\
\dot\varepsilon_{ds}  &=& A_{ds} \tau^n \exp \left(-\frac{Q_{ds}+pV_{ds}}{RT}\right) \label{sr_ds} \\
\dot\varepsilon_{df}  &=& A_{df} \tau   \exp \left(-\frac{Q_{df}+pV_{df}}{RT}\right) \label{sr_df} 
\end{eqnarray}
such that we are in fact looking for the stress value $\tau$ so that 
\[
\dot\varepsilon_T = 
A_{ds} \tau^n \exp \left(-\frac{Q_{ds}+p V_{ds}}{RT}\right) 
+
A_{df} \tau   \exp \left(-\frac{Q_{df}+p V_{df}}{RT}\right) 
+
\frac{\tau}{2 \eta_v}
\]
or, we must find the zero of the function ${\cal F}$: 
\[
{\cal F}(\tau) = 
\dot\varepsilon_T 
- A_{ds} \tau^n \exp \left(-\frac{Q_{ds}+p V_{ds}}{RT}\right) 
- A_{df} \tau   \exp \left(-\frac{Q_{df}+p V_{df}}{RT}\right) 
- \frac{\tau}{2 \eta_v} 
\]
This equation can be solved with a Newton-Raphson algorithm
and the iterations will be of the form:
\[
\tau_{n+1} = \tau_n - \frac{{\cal F}(\tau_n)}{{\cal F}'(\tau_n)}
\]
where the derivative of the function ${\cal F}$ with respect to $\tau$ reads:
\[
{\cal F}'(\tau)=\frac{\partial {\cal F}}{\partial \tau}=
- A_{df} \exp\left(-\frac{Q_{df}+pV_{df}}{RT}\right) 
- A_{ds} n \tau^{n-1} \exp\left(-\frac{Q_{ds}+pV_{ds}}{RT}\right)
- \frac{1}{2\eta_v}
\]
Once the value of $\tau$ is found, 
the strain rate values of Eqs. (\ref{sr_ds}), (\ref{sr_df}) and (\ref{sr_v}) 
can be computed and so can the respective effective viscosities:
\begin{eqnarray}
\eta_{ds} 
&=& \frac{1}{2} A_{ds}^{1/n} \dot\varepsilon_{ds}^{\frac{1}{n}-1} \exp \left(\frac{Q_{ds}+pV_{ds}}{nRT}\right) \\
\eta_{df} 
&=& \frac{1}{2} A_{df}^{1/n}  \exp \left(\frac{Q_{df}+pV_{df}}{RT}\right) 
\end{eqnarray}
Their average effective viscosity $\tilde{\eta}_{eff}$ is given by 
\[
\tilde{\eta}_{eff} = \left( \frac{1}{\eta_{ds}} + \frac{1}{\eta_{df}} + \frac{1}{\eta_v} \right)^{-1}
\]


\item if $\tau =2 \tilde{\eta}_{eff} \dot\varepsilon_T < Y$ the stress is below the yield stress value 
and the plasticity element is indeed not active. Use $\tilde{\eta}_{eff}$ in the material model.

\item if $\tau=2 \tilde{\eta}_{eff} \dot\varepsilon_T > Y$ the stress is above the yield value, which is not 
allowed. In this case the plastic element must be present and active and the viscous dampers are then 
in series with the (visco)plastic element. The formers deform 
with a strain rate $\dot{\varepsilon}_v$, $\dot\epsilon_{ds}$ and $\dot{\epsilon}_{df}$ 
while the latter with $\dot\epsilon_{vp}$ (all under the same tress $\tau$) 
and we have  $\dot\varepsilon_T = \dot{\varepsilon}_v + \dot\varepsilon_{ds} + \dot\varepsilon_{df} + \dot\varepsilon_{vp}$ so:

\begin{eqnarray}
\dot\varepsilon_T - \dot{\varepsilon}_v(\tau) - \dot\varepsilon_{ds}(\tau) - \dot\varepsilon_{df}(\tau)  
&=& \dot\varepsilon_{vp}  \nonumber\\
&=& \frac{\tau}{2 \left( \frac{Y}{2\dot\varepsilon_{vp}} + \eta_m  \right)} 
\nonumber\\
%&=&  \frac{\tau}{2 \left( \frac{Y}{2 (\dot\varepsilon_T -\dot{\varepsilon}_v(\tau) -\dot\varepsilon_{ds}(\tau) 
%+ \dot\varepsilon_{df}(\tau)   )} + \eta_m  \right)} 
%\nonumber\\
\dot\varepsilon_T -  \dot{\varepsilon}_v(\tau) -\dot\varepsilon_{ds}(\tau) - \dot\varepsilon_{df}(\tau) 
&=&
\frac{\tau}{2 \left( \frac{Y}{2 (\dot\varepsilon_T -\dot{\varepsilon}_v(\tau) -\dot\varepsilon_{ds}(\tau) 
+ \dot\varepsilon_{df}(\tau)   )} + \eta_m  \right)} \nonumber
\\
2 [\dot\varepsilon_T -\dot{\varepsilon}_v(\tau) -\dot\varepsilon_{ds}(\tau) - \dot\varepsilon_{df}(\tau) ]
 \left( \frac{Y}{2 (\dot\varepsilon_T -\dot{\varepsilon}_v(\tau) -\dot\varepsilon_{ds}(\tau) + \dot\varepsilon_{df}(\tau)   )} + \eta_m  \right) &=& \tau 
\nonumber\\
Y + 2 (\dot\varepsilon_T - \dot{\varepsilon}_v(\tau) -\dot\varepsilon_{ds}(\tau) - \dot\varepsilon_{df}(\tau) ) \eta_m  &=& \tau \nonumber 
\end{eqnarray}
As before, we must find the zero of the function ${\cal F}$: 
\begin{eqnarray}
{\cal F}(\tau) 
&=& Y + 2 [\dot\varepsilon_T -\dot{\varepsilon}_v(\tau)- \dot\varepsilon_{ds}(\tau) -\dot\varepsilon_{df}(\tau) ]\eta_m -\tau \nonumber\\
&=& Y + 2 \left[ 
\dot\varepsilon_T - \frac{\tau}{2\eta_v}-A_{ds} \tau^n \exp \left(-\frac{Q_{ds}+pV_{ds}}{RT}\right) 
- A_{df} \tau   \exp \left(-\frac{Q_{df}+pV_{df}}{RT}\right)  
\right]\eta_m -\tau \nn
\end{eqnarray}
Because dislocation creep involves the $n$-th power of the stress we will here also need 
to find the zero by means of a Newton-Raphson algorithm. 

We have:
\begin{eqnarray}
\frac{\partial {\cal F}}{\partial \tau} &=&
\left[
- \frac{1}{\eta_v} 
-2 \frac{\partial  \dot\varepsilon_{ds}(\tau) }{\partial \tau} 
-2 \frac{\partial  \dot\varepsilon_{df}(\tau) }{\partial \tau} 
\right] \eta_m - 1
\end{eqnarray}

\begin{eqnarray}
{\cal F}(\tau) / 2\eta_m 
&=& 
\frac{Y}{2\eta_m}  + \dot\varepsilon_T 
- \frac{\tau}{2\eta_v}
- A_{ds}(p,T) \tau^n 
- A_{df}(p,T) \tau  
- \frac{\tau}{2\eta_m} \\
&=& 
- A_{ds}(p,T) \tau^n - \left(A_{df}(p,T) +  \frac{1}{2\eta_v} -\frac{1}{2\eta_m} \right) \tau
+ 
\left( \frac{Y}{2\eta_m}  + \dot\varepsilon_T \right) =0
\end{eqnarray}



\end{enumerate}

Note that when $\eta_m=0$ we logically recover $\tau=Y$ as the stress cannot exceed the yield strength $Y$.

Although this approach is probably the most consistent in terms of physics, the presence 
of the Newton-Raphson iterations makes it very expensive since this procedure is to be repeated 
for every quadrature point or every particle.

Let us consider a concrete example: we set $Y=20\si{\mega\pascal}$, $\eta_v=10^{25}\si{pascal}$, 
$\eta_m=10^{20}\si{pascal}$. The domain is one-dimensional of depth $660\si{km}$. The density is
assumed to be constant at $3300\si{\kg\per\cubic\metre}$. Dislocation and diffusion creep parameters
are taken from Karato \& Wu (1993) \cite{kawu93}. The temperature is linear is $20\si{\celsius}$ 
at the surface, $550\si{\celsius}$ at $30\si{km}$ depth, $1330\si{celsius}$ at $90\si{\km}$ depth 
and $1380\si{\celsius}$ at the bottom. Pressure is assumed to be lithostatic. 
The python program and the gnuplot script are in {\sl images/rheology/example}.

In the code I consider two cases: 'old' and 'new'. The latter is described above. 
'old' goes as follows: loop over total strain rate values. 
Compute dislocation and diffusion creep viscosities with it. Compute harmonic 
average of these with linear viscosity. Compute deviatoric stress value. 
use it in dislocation and diffusion formulae to arrive at respective strainrates. 

\newpage
\begin{center}
\includegraphics[width=5.5cm]{images/rheology/example/map_sr_df_old-1}
\includegraphics[width=5.5cm]{images/rheology/example/map_sr_df_new-1}
\includegraphics[width=5.5cm]{images/rheology/example/map_sr_df_diff-1}\\
\includegraphics[width=5.5cm]{images/rheology/example/map_sr_ds_old-1}
\includegraphics[width=5.5cm]{images/rheology/example/map_sr_ds_new-1}
\includegraphics[width=5.5cm]{images/rheology/example/map_sr_ds_diff-1}\\
\includegraphics[width=5.5cm]{images/rheology/example/map_sr_v_old-1}
\includegraphics[width=5.5cm]{images/rheology/example/map_sr_v_new-1}
\includegraphics[width=5.5cm]{images/rheology/example/map_sr_v_diff-1}\\
\includegraphics[width=5.5cm]{images/rheology/example/map_etaeff_old-1}
\includegraphics[width=5.5cm]{images/rheology/example/map_etaeff_new-1}
\includegraphics[width=5.5cm]{images/rheology/example/map_etaeff_diff-1}\\
\includegraphics[width=5.5cm]{images/rheology/example/map_tau_old-1}
\includegraphics[width=5.5cm]{images/rheology/example/map_tau_new-1}
\includegraphics[width=5.5cm]{images/rheology/example/map_tau_diff-1}\\
\includegraphics[width=5.5cm]{images/rheology/example/profile_sr-1}
\includegraphics[width=5.5cm]{images/rheology/example/profile_tau-1}
\includegraphics[width=5.5cm]{images/rheology/example/profile_etaeff-1}\\
\includegraphics[width=5.5cm]{images/rheology/example/map_isplast_old-1}
\includegraphics[width=5.5cm]{images/rheology/example/map_isplast_new-1}\\
{\captionfont Viscous branch: (ds+df+v) 'old' stands 
for the old approach when $\dot{\varepsilon}_T$
was used for all mechanisms. 'new' stands for the new approach and the right strain rate 
decomposition.}
\end{center}



\newpage
\begin{center}
\includegraphics[width=5.5cm]{images/rheology/example/map_etaeff_old_pl-1}
\includegraphics[width=5.5cm]{images/rheology/example/map_etaeff_new_pl-1}
\includegraphics[width=5.5cm]{images/rheology/example/map_etaeff_diff_pl-1}\\
\includegraphics[width=5.5cm]{images/rheology/example/map_tau_old_pl-1}
\includegraphics[width=5.5cm]{images/rheology/example/map_tau_new_pl-1}
\includegraphics[width=5.5cm]{images/rheology/example/map_tau_diff_pl-1}\\
\includegraphics[width=5.5cm]{images/rheology/example/profile_sr_pl-1}
\includegraphics[width=5.5cm]{images/rheology/example/profile_tau_pl-1}
\includegraphics[width=5.5cm]{images/rheology/example/profile_etaeff_pl-1}\\
{\captionfont Visco-viscoplastic rheology: (ds+df+v+vp)} 
\end{center}





\begin{remark}
Chenin \etal (2019) \cite{chmd19}, 
base their rheological model on the additive decomposition of the following
deviatoric strain rate tensor ${\bm \varepsilon}^d$:
\[
{\bm \varepsilon}^d =
{\bm \varepsilon}^{el}+
{\bm \varepsilon}^{pl}+
{\bm \varepsilon}^{ds}+
{\bm \varepsilon}^{df}+
{\bm \varepsilon}^{pe}
\]
where the five strain rate terms correspond respectively to the elastic, plastic, 
and viscous creep (dislocation, diffusion, peierls) contributions. 
This implies that all these elements are in series and the associated 
viscosities are then averaged with an harmonic mean. 
Rather interestingly, it is then stated that "this strain rate equation is nonlinear
and solved locally on cell centroids and vertices in order to define the current effective viscosity 
and stress \cite{poso08}."
\end{remark}

\Literature: \textcite{hoor89,lopr90,homo90,scps01,lova01,anpa19,elga10}





