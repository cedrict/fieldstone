
\index{general}{$\tilde{Q}_1\times P_0$}
\index{general}{Korn's inequality}
\index{general}{Rannacher-Turek element}
\index{general}{Nonconforming element}

This element is the natural quadrilateral analogue
of the well-known triangular $P_1^{nc} $Stokes element of Crouzeix-Raviart \cite{crra73}.
This element is sometimes called ${\bm Q}_1^{rot} \times Q_0$ or the Rannacher-Turek element 
\cite[Section 3.6.5]{john16} (see also Appendix~B.4, example B.53 of \textcite{john16}).
This rectangular nonconforming \cite{crfa89} element is termed the rotated ${\bm Q}_1$ element 
because of the fact that $r^2-s^2$ can be generated from $rs$ (occurring in the bilinear $Q_1$ 
element) by a rotation of 45$\degree$ \cite[p93]{chen}.
The velocity approximation is achieved by rotated dim-linear functions that have 
continuous degrees of freedom on
the faces of the mesh cells as we have seen in Section~\ref{ss:rq1}.
This element was introduced in Rannacher \& Turek (1992) \cite{ratu92} 
has been proven to satisfy the inf-sup condition. It has been studied comprehensively in Schieweck 
(1997)\footnote{Habilitation thesis in German}, \cite{shzh06} and in Turek \cite{ture94,ture96}.
Superconvergence properties have also been reported \cite{misx06,misx07}.
It has been used in 2D \cite{maky17} and 3D \cite{klll96,gekm08} and forms the basis of the FeatFlow 
software\footnote{\url{http://www.featflow.de/en/index.html}}. 
It is used in the PhD thesis of Gastaldo \cite{gast07} and Ouazzi \cite{ouaz05}.
It has been 
successfully coupled to multigrid solvers \cite{chos98,tuos02}.
This element has been compared to the stabilised ${\bm Q}_1\times P_0$ element \cite{lisi13}.
It is mentioned in \cite{hans11}

It essentially comes in two flavours, the Middle Point (MP) and the Mid Value (MV) one.

\begin{remark} 
John \cite{john16} explains that: "For the point-value-oriented non-conforming finite element spaces (MP), 
the value of the Dirichlet boundary
condition in the barycenter of the faces at the boundary is taken. Using the mean-
value-oriented spaces (MV), one computes the integrals of the boundary condition on
these faces and normalizes with the area of the faces to set the boundary values.
In the case of homogeneous Dirichlet boundary conditions, the boundary values
computed in both ways are zero."
\end{remark}

\begin{remark} 
John also makes a very important point: "There are also unmapped (non-parametric) versions of 
these finite element spaces, which define the polynomials directly on the mesh cell K. It is shown in Rannacher
and Turek (1992) \cite{ratu92} that these versions are inf-sup stable on more general meshes than
the mapped (parametric) version of the ${\bm Q}_1^{rot}\times Q_0$ finite element, e.g., on strongly
nonuniform meshes. Considering all four types of ${\bm Q}_1^{rot}\times Q_0$ finite elements, the
optimal order of convergence on perturbed meshes is achieved only by the mean-
value-oriented version of the unmapped ${\bm Q}_1^{rot}\times Q_0$   finite element.
\end{remark}

Mahmood \etal \cite{maky17} mention a very important fact: "The chosen nonconforming element requires
additional stabilization for handling the deformation tensor formulation due to missing Korn's inequality 
\cite{horg95,knob00,bren04}.
To this end we employ the standard edge oriented stabilization \cite{tuos02,tuou07} in our simulations."
This is a rather unfortunate fact that although LBB stable this element needs an additional 
term in the weak form (see Turek \etal (2002) \cite{tuos02}) 
so as to suppress parasitic velocity modes when the div-grad formulation 
of the Stokes equation is used (as opposed to the Laplace formulation -- see \cite[Section 6.5.2]{dohu03}).

This element is used in \textcite{hala01} (2001) in the context of near incompressible elasticity. 
It is mentioned that it does not fulfill the discrete Korn's inequality. It is then stabilised 
in a discontinuous Galerkin framework.

\Literature \textcite{shee20} (2020), \textcite{chen92} (1992) , \textcite{chen93} (1993) 

\url{https://defelement.com/elements/rannacher-turek.html}




