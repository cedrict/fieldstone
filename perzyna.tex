\begin{flushright} {\tiny {\color{gray} perzyna.tex}} \end{flushright}
%~~~~~~~~~~~~~~~~~~~~~~~~~~~~~~~~~~~~~~~~~~~~~~~~~~~~~~~~~~~~~~~~~~~~~~~~~~~~~~~~~~~~~~~~~~~~~~~~~~


The Perzyna formulation is mentioned in \url{
https://en.wikipedia.org/wiki/Viscoplasticity}.
It is a rate-dependent plasticity formulation that was proposed in 1966 by Piotr Perzyna \cite{perz66}.
See also \url{https://neml.readthedocs.io/en/dev/vp_flow/perzyna.html}.



In what follows I make use of the approach and notations of Zienkiewicz (1974) \cite{zico74} (and all 
the 1974-75 papers that follow) and the book by Owen \& Hinton \cite{owhi}.

The total strain (rate) is divided into two parts\footnote{Zienkiewicz \cite{zico74} 
adds a third term ${\bm \varepsilon}^0$ which stands for initial/autogenous strain such as due 
to temperature changes but I neglect it in what follows.}:
\[
\dot{\bm \varepsilon} = \dot{\bm \varepsilon}^e + \dot{\bm \varepsilon}^{vp}  
\]
where ${\bm \varepsilon}^e$ stands for the elastic strain tensor and 
${\bm \varepsilon}^{vp}$ stands for the visco-plastic strain tensor.

%Since the tensors are symmetric only 6 of the 9 components are independent and 
%the above relationship is often re-written
%\[
%\vec{\varepsilon} = \vec{\varepsilon}^e + \vec{\varepsilon}^{vp}  
%\]
%with
%\[
%\vec{\varepsilon}=
%\left(
%\begin{array}{c}
%\varepsilon_{xx} \\
%\varepsilon_{yy} \\
%\varepsilon_{zz} \\
%\varepsilon_{xy} \\
%\varepsilon_{xz} \\
%\varepsilon_{yz} 
%\end{array}
%\right)
%\]
%For a linear elastic material 
%\[
%{\vec \varepsilon}^e = {\bm D}^{-1} \cdot {\vec \sigma}
%\]
%where ${\bm D}^{-1}$ is a symmetric elasticity matrix (compliance matrix).




The yield condition is given as 
\[
{\FFF}({\bm \sigma},\kappa) 
= \Psi(\bm\sigma,\dot{\bm \varepsilon}) -Y(\kappa) = 0
\]
with ${\FFF}<0$ denoting the purely elastic region, $\kappa$ is a 
history-dependent hardening/softening parameter and $Y(\kappa)$ is a static yield stress.
$\Psi$ is a function of the stress and/or strain rate invariants.

We borrow from classical viscoplasticity theory (Perzyna (1966) \cite{perz66}, Perzyna (1988) \cite{perz88}) 
the idea of a plastic potential defined as $\QQQ({\bm \sigma})$ and write
\begin{equation}
{\dot{\bm \varepsilon}}^{vp} 
= \gamma \Big\langle
\phi\left( {\FFF} \right) 
\Big\rangle
\frac{\partial \QQQ}{\partial \bm\sigma}
\end{equation}
where $\gamma$ is a positive, possibly time-dependent fluidity parameter. 
Note that sometimes the pseudo-viscosity $\bar{\eta}=\gamma^{-1}$ is defined \cite{zigo74}
so that the equation above writes:
\begin{equation}
{\dot{\bm \varepsilon}}^{vp} 
= \frac{1}{\bar{\eta}} \Big\langle
\phi\left( {\FFF} \right) 
\Big\rangle
\frac{\partial \QQQ}{\partial \bm\sigma}
\end{equation}
${\FFF}$ represents the plastic yield condition.
$\phi(x)$ is a positive scalar-valued monotonic increasing function in the range 
$x>0$ such that $\phi^{-1}(x)$ exists and possess similar properties in the same range. 
The notation $\langle \rangle$ denotes the Macaulay 
brackets\footnote{\url{https://en.wikipedia.org/wiki/Macaulay_brackets}} and stands 
for\footnote{there is a difference between 
\cite{zico74}(1974) and \cite{zico74b}(1974) wrt $>$ and $\ge$, and also 
a difference with wikipedia!} 
\begin{eqnarray}
\langle \phi(x) \rangle = \phi(x) & {\rm if} & x>0 \nonumber\\
\langle \phi(x) \rangle = 0 & {\rm if} & x\le 0 \nonumber
\end{eqnarray}
If ${\QQQ}={\FFF}$ then we speak of an {\it associative} law and if ${\QQQ} \neq {\FFF}$ 
we have a {\it non-associative} situation. 
The tensor $\frac{\partial \QQQ}{\partial \bm\sigma}$ represents the direction
of plastic flow and when ${\FFF}={\QQQ}$ it is a vector directed normal to the yield surface
at the stress point under consideration. This is potentially problematic in the 
case of the Tresca and Mohr-Coulomb yield surfaces since the normal is not well defined
along the apices of the surfaces (see Section~7.6 of \cite{owhi}).
In the non-associative case, the direction of plastic flow in the 
principal  stress space during plastic flow is not the same
as the direction of the vector normal to the yield surface.

In what follows we concentrate our attention on isotropic materials for which 
both ${\FFF}$ and ${\QQQ}$ can be defined in terms of stress invariants.

According to Zienkiewicz et al (1975) \cite{zihl75}:
\begin{displayquote}
{\color{darkgray}
One of the main stumbling blocks of the 
classical plasticity theory lay in the universal
assumption, based on Drucker's postulates (Drucker and Prager, 1952), that the plastic 
behaviour is `associated'. With the use of Mohr-Coulomb type yield envelopes to define the
limit between states of elasticity and of continuing irreversible deformation,
the associated behaviour manifestly contradicted observation and gave excessive dilation.
It became necessary therefore to extend plasticity ideas to a `non-associated'
form in which the plastic potential and yield surfaces are defined separately.}
\end{displayquote}
At the same time, it is worth remembering that these early studies mostly dealt
with plasticity in metals, and later soils, but not kilometer-scale crustal layers.

Also, the Perzyna model is not the only one, see for instance
the Duvaut-Lions viscoplastic model or the Consistency model \cite{wasd97,hesd02}.

We therefore need to look into the derivative of the plastic potential ${\QQQ}$
with respect to the stress tensor. Since the potential 
is expressed as a function of the stress invariants ${\III}_1(\bm\sigma)$,
${\III}_2(\bm\tau)$ and $\uptheta_L(\bm\tau)$, we then have\footnote{
The derivative of the Lod\'e angle was obtained in Section~\ref{ss:lode}}:

\begin{eqnarray}
\frac{\partial \QQQ}{\partial \bm\sigma}
&=&
\frac{\partial }{\partial \bm\sigma} \QQQ\left({\III}_1(\bm\sigma),{\III}_2(\bm\tau),\uptheta_{\rm L}(\bm\tau)\right)\nn\\
&=&
\frac{\partial \QQQ}{\partial {\III}_1(\bm\sigma)} 
\frac{\partial {\III}_1(\bm\sigma)}{\partial \bm\sigma} 
+
\frac{\partial \QQQ}{\partial \sqrt{{\III}_2(\bm\tau)}} 
\frac{\partial \sqrt{ {\III}_2(\bm\tau)}   }{\partial {\III}_2(\bm\tau)} 
\frac{\partial {\III}_2(\bm\tau)}{\partial \bm\sigma} 
+
\frac{\partial \QQQ}{\partial \uptheta_{\rm L}(\bm\tau)} 
\frac{\partial \uptheta_{\rm L}(\bm\tau)}{\partial \bm\sigma} \nn\\
&=&
\frac{\partial \QQQ}{\partial {\III}_1(\bm\sigma)} 
\frac{\partial {\III}_1(\bm\sigma)}{\partial \bm\sigma} 
+
\frac{\partial \QQQ}{\partial \sqrt{{\III}_2(\bm\tau)}} 
\frac{1}{2 \sqrt{ {\III}_2(\bm\tau)}   }
\frac{\partial {\III}_2(\bm\tau)}{\partial \bm\sigma} 
\nn\\
&&
-
\frac{\partial \QQQ}{\partial \uptheta_{\rm L}(\bm\tau)} 
\frac{\sqrt{3}}{2\cos 3\uptheta_{\rm L}}
\left[
-\frac32  \frac{ {\III}_3(\bm\tau)   }{ {\III}_2(\bm\tau)^{5/2}}
\; \frac{\partial {\III}_2(\bm\tau)}{\partial \bm\sigma} 
+  \frac{1}{{\III}_2(\bm\tau)^{3/2}} 
\; \frac{\partial {\III}_3(\bm\tau)}{\partial \bm\sigma} 
\right] \nn\\
&=&
\frac{\partial \QQQ}{\partial {\III}_1(\bm\sigma)} 
\; \frac{\partial {\III}_1(\bm\sigma)}{\partial \bm\sigma} 
+  
\left(\frac{\partial \QQQ}{\partial \sqrt{{\III}_2(\bm\tau)}} 
\frac{1}{2 \sqrt{ {\III}_2(\bm\tau)}   }   
+
\frac{\partial \QQQ}{\partial \uptheta_{\rm L}(\bm\tau)} 
\frac{\sqrt{3}}{2\cos 3\uptheta_{\rm L}}
\frac32  \frac{ {\III}_3(\bm\tau)   }{ {\III}_2(\bm\tau)^{5/2}}
\right)
\; \frac{\partial {\III}_2(\bm\tau)}{\partial \bm\sigma} \nn\\
&&
-
\frac{\partial \QQQ}{\partial \uptheta_{\rm L}(\bm\tau)} 
\frac{\sqrt{3}}{2\cos 3\uptheta_{\rm L}}
\frac{1}{{\III}_2(\bm\tau)^{3/2}} 
\; \frac{\partial {\III}_3(\bm\tau)}{\partial \bm\sigma} \nn\\
&=& 
C_1  \frac{\partial {\III}_1(\bm\sigma)}{\partial \bm\sigma} 
+
C_2  \frac{\partial {\III}_2(\bm\tau)}{\partial \bm\sigma} 
+
C_3  \frac{\partial {\III}_3(\bm\tau)}{\partial \bm\sigma} 
\end{eqnarray}
i.e.
\begin{mdframed}[backgroundcolor=blue!5]
\begin{eqnarray}
\frac{\partial \QQQ}{\partial \bm\sigma}
&=& 
C_1 {\bm a}_1 +
C_2 {\bm a}_2 +
C_3 {\bm a}_3 
=
C_1  \frac{\partial {\III}_1(\bm\sigma)}{\partial \bm\sigma} 
+
C_2  \frac{\partial {\III}_2(\bm\tau)}{\partial \bm\sigma} 
+
C_3  \frac{\partial {\III}_3(\bm\tau)}{\partial \bm\sigma} 
\end{eqnarray}
\end{mdframed}
where the $C_{1,2,3}$ coefficients depend on the plastic potential $\QQQ$
and the stress invariants as follows:
\begin{eqnarray}
C_1 &=&  \frac{\partial \QQQ}{\partial {\III}_1(\bm\sigma)} \\
C_2 
&=& \frac{\partial \QQQ}{\partial \sqrt{{\III}_2(\bm\tau)}} 
\frac{1}{2 \sqrt{ {\III}_2(\bm\tau)}   }   
+
\frac{\partial \QQQ}{\partial \uptheta_{\rm L}(\bm\tau)} 
\frac{\sqrt{3}}{2\cos 3\uptheta_{\rm L}}
\frac32  \frac{ {\III}_3(\bm\tau)   }{ {\III}_2(\bm\tau)^{5/2}} \nn\\
&=& 
\frac{\partial \QQQ}{\partial \sqrt{{\III}_2(\bm\tau)}} 
\frac{1}{2 \sqrt{ {\III}_2(\bm\tau)}   }   
-
\frac12
\frac{\tan 3\uptheta_{\rm L}}{ {\III}_2(\bm\tau)}
\frac{\partial \QQQ}{\partial \uptheta_{\rm L}(\bm\tau)}  \nn\\
&=& 
\frac{1}{2 \sqrt{ {\III}_2(\bm\tau)}   }   
\left(
\frac{\partial \QQQ}{\partial \sqrt{{\III}_2(\bm\tau)}} 
-
\frac{\tan 3\uptheta_{\rm L}}{\sqrt {{\III}_2(\bm\tau)}}
\frac{\partial \QQQ}{\partial \uptheta_{\rm L}(\bm\tau)}  
\right) \nn
\\
C_3 &=&  
-
\frac{\sqrt{3}}{2\cos 3\uptheta_{\rm L}}
\frac{1}{{\III}_2(\bm\tau)^{3/2}} 
\frac{\partial \QQQ}{\partial \uptheta_{\rm L}(\bm\tau)} 
\end{eqnarray}
These are identical to those of Eq.~(7.71) in Owen \& Hinton\footnote{This is 
not exactly true: the factor  $\frac{1}{2 \sqrt{ {\III}_2(\bm\tau)} }$
is absent in their Eq.~(7.71) but it is to be found in their Eq.~(7.70).}:
\begin{center}
\fbox{\includegraphics[width=12cm]{images/perzyna/owenhinton2}}
\end{center}

\noindent Note that we already have established (see Section~\ref{ss:recapInv}) that  
\begin{eqnarray}
{\bm a}_1 =\frac{\partial {\III}_1(\bm\sigma)}{\partial \bm\sigma} &=& {\bm 1} \nn\\
{\bm a}_2 =\frac{\partial {\III}_2(\bm\tau)}{\partial \bm\sigma} &=& {\bm \tau} \nn\\
{\bm a}_3 =\frac{\partial {\III}_3(\bm\tau)}{\partial \bm\sigma} 
&=& \bm\tau\cdot\bm\tau -\frac23  {\III}_2(\bm\tau){\bm 1} \nn
\end{eqnarray}
with (we will need these soon)
\begin{eqnarray}
\textrm{Tr}[{\bm a}_1]=
{\rm tr}\left[ \frac{\partial {\III}_1(\bm\sigma)}{\partial \bm\sigma} \right] &=& 3 \\  
\textrm{Tr}[{\bm a}_2]=
{\rm tr}\left[ \frac{\partial {\III}_2(\bm\tau)}{\partial \bm\sigma}   \right] &=& 0 \\
\textrm{Tr}[{\bm a}_3]=
{\rm tr}\left[ \frac{\partial {\III}_3(\bm\tau)}{\partial \bm\sigma}   \right] &=& 
{\rm tr}[\bm\tau\cdot\bm\tau] - 2  {\III}_2(\bm\tau) = 2  {\III}_2(\bm\tau) -2  {\III}_2(\bm\tau) = 0 
\end{eqnarray}

The generic form of the plastic potential derivative then reads 
\begin{mdframed}[backgroundcolor=blue!5]
\begin{eqnarray}
\frac{\partial \QQQ}{\partial \bm\sigma}
&=&
C_1 {\bm 1} 
+
C_2 {\bm \tau} 
+
C_3  \left( \bm\tau\cdot\bm\tau - \frac23  {\III}_2(\bm\tau)  {\bm 1} \right)
\end{eqnarray}
\end{mdframed}

The momentum conservation equation that we solve is 
\[
-\vec\nabla p + \vec\nabla \cdot \left[
2 \eta \dot{\bm\varepsilon}^d(\vec\upnu) 
\right]+ \rho \vec g = \vec 0
\]
so we need the {\it deviatoric} strain rate tensor. 
We here assume for simplicity that there is only a visco-plastic element in the system
(or that the other mechanisms are deviatoric), 
i.e. $\dot{\bm\varepsilon}=\dot{\bm\varepsilon}^{vp}$.
Then 
\begin{eqnarray}
\dot{\bm\varepsilon}^d 
&=& \dot{\bm\varepsilon}^{vp} - \frac13 {\rm tr}[ \dot{\bm\varepsilon}^{vp}] {\bm 1} \nn\\
&=& \gamma \langle \phi(\FFF) \rangle 
\left\{
\left(C_1 {\bm a}_1 + C_2 {\bm a}_2 + C_3 {\bm a}_3 \right)
-\frac13 {\rm tr} \left(C_1 {\bm a}_1 + C_2 {\bm a}_2 + C_3 {\bm a}_3 \right) {\bm 1}
\right\} \nn\\
&=& \gamma \langle \phi(\FFF) \rangle 
\left\{
\left(C_1 {\bm a}_1 + C_2 {\bm a}_2 + C_3 {\bm a}_3 \right)
-\frac13 
C_1 3  {\bm 1}
\right\} \nn\\
&=& \gamma \langle \phi(\FFF) \rangle 
\left\{
\left(C_1 \bm 1 + C_2 \bm\tau + C_3 (\bm\tau\cdot\bm\tau-\frac23 {\III}_2(\bm\tau)\bm 1)\right)
-C_1 \bm 1
\right\} \nn\\
&=& \gamma \langle \phi(\FFF) \rangle 
\left(C_2 \bm\tau + C_3 (\bm\tau\cdot\bm\tau-\frac23 {\III}_2(\bm\tau)\bm 1)\right)
\end{eqnarray}
The $C_{1,2,3}$ coefficients have been computed in 
Sections~\ref{sec:vMcriterion}, \ref{sec:trcriterion}, \ref{sec:dpcriterion} and 
\ref{sec:mccriterion}, and are summarized below: 

\begin{center}
\begin{footnotesize}
\begin{tabular}{lccc}
\hline
& $C_1$ & $C_2(\times \frac{1}{2 \sqrt{{\III}_2({\bm \tau})}})$ & $C_3$ \\
              \hline\hline
Tresca         &0 & $2 \cos\uptheta_{\rm L} ( 1 + {\color{teal}2} \tan\uptheta_{\rm L}  \tan 3\uptheta_{\rm L})$ &
$\frac{\sqrt{3}}{{\III}_2({\bm \tau}) } \frac{\sin\uptheta_{\rm L} }{\cos 3\uptheta_{\rm L}}$
\\ \\
von Mises      &0& $\sqrt{3}$ & 0 \\ \\ 
Mohr-Coulomb   & $\frac13 \sin\phi$ & 
$
\cos \uptheta_{\rm L} \left[
(1 +  {\color{teal}2}\tan \uptheta_{\rm L}   \tan 3\uptheta_{\rm L})
+\frac{1}{\sqrt{3}} \sin\phi
( {\color{teal}2}\tan 3\uptheta_{\rm L} - \tan\uptheta_{\rm L}) \right]$
&
$\frac{\sqrt{3}\sin\uptheta_{\rm L} +  \sin \phi \cos \uptheta_{\rm L}}
{2 {\III}_2({\bm \tau}) \cos 3\uptheta_{\rm L}}$ 
\\ \\
Drucker-Prager & $\alpha$ & 1 & 0 \\  
\hline
\end{tabular}
\end{footnotesize}
\end{center}
The difference with the table below  taken from Owen \& Hinton \cite{owhi} is highlighted in blue.

\begin{center}
\fbox{\includegraphics[width=12cm]{images/perzyna/owenhinton1}}\\
{\captionfont Taken from \cite{owhi}.
This table supposedly presents all three $C_{1,2,3}$ coefficients 
for all four plastic potentials/yield functions (associative plasticity).
This is however not the case: the $C_2$ column is not $C_2$ but 
$\partial \FFF/\partial\sqrt I_2$!
}
\end{center}


Also, the continuity equation for incompressible flow contains the divergence 
of the velocity field, and in this case 
\begin{eqnarray}
\vec\nabla\cdot\vec\upnu 
&=& {\rm tr}[\dot{\bm\varepsilon}^{vp}]  \nn\\
&=& {\rm tr}\left[ 
\gamma \Big\langle
\phi\left( {\FFF} \right) 
\Big\rangle
\frac{\partial \QQQ}{\partial \bm\sigma}
\right] \nn\\
&=& 
\gamma \Big\langle
\phi\left( {\FFF} \right) 
\Big\rangle
{\rm tr}\left[
C_1  \frac{\partial {\III}_1(\bm\sigma)}{\partial \bm\sigma} 
+
C_2  \frac{\partial {\III}_2(\bm\tau)}{\partial \bm\sigma} 
+
C_3  \frac{\partial {\III}_3(\bm\tau)}{\partial \bm\sigma} 
\right] \nn\\
&=&
\gamma \Big\langle
\phi\left( {\FFF} \right) 
\Big\rangle
{\rm tr}\left[
C_1 {\bm 1} 
+
C_2 {\bm \tau} 
+
C_3  \left( \bm\tau\cdot\bm\tau - \frac23  {\III}_2(\bm\tau)  {\bm 1} \right)
\right] \nn\\
&=&  \gamma \langle \phi(\FFF) \rangle  3 C_1
\end{eqnarray}

%If $C_3=0$, as in the vM or DP case, then we have a relationship between
%deviatoric strain rate and stress that involves a scalar. 
%If not, then a 4th order tensor is needed. This is bad news. 
%Then, vM and Tresca do not modify the continuity equation as $C_1=0$.

I find it difficult to wrap my head around this as the continuity equation 
is usually derived by other means. 
If $C_1$ is not zero, then dilation occurs, the material is not incompressible
so density should also change...  
On the other hand this is a 'problem' only when the bracket is nonzero, i.e. when
plasticity is activated. If the Macaulay bracket is zero, then we recover 
the zero divergence constraint.



