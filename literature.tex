\documentclass[a4paper,12pt]{report}

\usepackage[utf8]{inputenc}
\usepackage{graphicx}
\usepackage{xcolor} 
\usepackage[cm]{fullpage}
\usepackage{bold-extra} % to get rid of some warnings
\usepackage{upgreek}
\usepackage{bm}

\usepackage{hyperref}
\hypersetup{
colorlinks,
citecolor=black,
filecolor=black,
linkcolor=violet,
urlcolor=black}

\usepackage[colorinlistoftodos,prependcaption,textsize=tiny]{todonotes}

\newcommand{\aspect}{{\textsc{Aspect~}{}}}
\newcommand{\elefant}{{\textsc{Elefant~}{}}}
\newcommand{\citcoms}{{\textsc{CitcomS~}{}}}
\newcommand{\citcomsve}{{\textsc{CitcomSVE~}{}}}
\newcommand{\fantom}{{\textsc{Fantom~}{}}}
\newcommand{\sulec}{{\textsc{Sulec~}{}}}
\newcommand{\sopale}{{\textsc{Sopale~}{}}}
\newcommand{\douar}{{\textsc{Douar~}{}}}
\newcommand{\ghost}{{\textsc{Ghost~}{}}}
\newcommand{\fluidity}{{\textsc{Fluidity~}{}}}
\newcommand{\sepran}{{\textsc{Sepran~}{}}}
\newcommand{\stone}{{\color{teal} {\textsc{stone~}}}}
\newcommand{\etal}{{\it et al.~}}
\newcommand{\nn}{\nonumber}
\newcommand{\A}{{\mathbb{A}}}
\newcommand{\K}{{\mathbb{K}}}
\newcommand{\J}{{\mathbb{J}}}
\newcommand{\G}{{\mathbb{G}}}
\newcommand{\Z}{{\mathbb{Z}}}
\newcommand{\C}{{\mathbb{C}}}
\newcommand{\W}{{\mathbb{W}}}
\newcommand{\R}{{\mathbb{R}}}
\newcommand{\M}{{\mathbb{M}}}
\newcommand{\N}{{\mathbb{N}}}
\newcommand{\LLL}{{\mathbb{L}}}
\newcommand{\SSS}{{\mathbb{S}}}
\newcommand{\HH}{{\mathbb{H}}}
\newcommand{\Literature}{\includegraphics[height=4mm]{images/lit} {\sffamily Relevant Literature}}
\newcommand{\bscthesis}{{(\bf BSc Thesis)}}
\newcommand{\mscthesis}{{(\bf MSc Thesis)}}
\newcommand{\captionfont}{\tiny}
\newcommand{\pythonfile}{\color{blue} \sffamily }
\newcommand{\shellscriptfile}{\color{purple} \sffamily }
\newcommand{\asciifile}{\color{olive} \sffamily }
\newcommand{\OK}{{\bf OK}}
\newcommand{\filenamefont }{\sl }
\newcommand{\foldernamefont }{\it }
\newcommand{\codefont}{\bfseries\ttfamily}
\newcommand{\Ranb}{{\mathsf{Ra}}}
\newcommand{\Nunb}{{\mathsf{Nu}}}
\newcommand{\Prnb}{{\mathsf{Pr}}}
\newcommand{\Penb}{{\mathsf{Pe}}}
\newcommand{\Dinb}{{\mathsf{Di}}}
\newcommand{\python}{\color{darkgray} \sffamily }
\newcommand{\bN}{{\mathcal{N}}}
\newcommand{\qx}{\underset{\tiny x}{q}}
\newcommand{\qy}{\underset{\tiny y}{q}}
\newcommand{\nx}{\underset{\tiny x}{n}}
\newcommand{\ny}{\underset{\tiny y}{n}}
\newcommand{\qhx}{\underset{\tiny x}{q_h}}
\newcommand{\qhy}{\underset{\tiny y}{q_h}}
\newcommand{\qix}{\underset{\tiny x}{q_i}}
\newcommand{\qiy}{\underset{\tiny y}{q_i}}
\newcommand{\Tx}{\underset{\tiny x}{T}}
\newcommand{\Ty}{\underset{\tiny y}{T}}
\newcommand{\blueqx}{  {\color{blue} \underset{\tiny x}{q}  }  }
\newcommand{\blueqy}{  {\color{blue} \underset{\tiny y}{q}  }  }
\newcommand{\blueT}{  {\color{blue} T}  }
\newcommand{\brownT}{  {\color{brown} T}  }
\newcommand{\brownqx}{  {\color{brown} \underset{\tiny x}{q}  }  }
\newcommand{\brownqy}{  {\color{brown} \underset{\tiny y}{q}  }  }
\newcommand\norm[1]{\left\lVert#1\right\rVert}


%Bibliography stuff
\usepackage[maxnames=6]{biblatex}
\addbibresource{biblio_geosciences.bib}

\title{Literature}
\author{C. Thieulot}

%%%%%%%%%%%%%%%%%%%%%%%%%%%%%%%%%%%%%%%%%%%%%%%%%%%%%%%%%%%
\begin{document}

\thispagestyle{empty}

\begin{center}
{\large Literature}
\end{center}

\newpage

%\maketitle
\tableofcontents

\newpage
This is a {\it very} rough attempt at classifying my somewhat extensive 
bibliography per theme/topic.
It goes without saying that this cannot be extensive and that since I 
started computational geodynamics myself around 2006. 
The provided lists are biaised towards the last 2 decades or so. 
In retrospect, the categories I have chosen could have been subdivided
into narrower fields. I understand that having 100+ references 
for 'subduction'  or 'mantle convection' is not particularly useful, 
but it means that all these papers show up in the bibliography section 
of this book, and the titles of said papers are then searchable per keyword.

\chapter{Review papers} 

%====================
\section{Subduction}

   \begin{itemize}
   \item [\nineteeneightytwo] Controls of subduction geometry, location of magmatic arcs,
                              and tectonics of arc and back-arc regions \cite{crpi82}
   \item [\nineteenninetyfive] Subduction dynamics: From the trench
                               to the core-mantle boundary \cite{kinc95}
   \item [\twothousandone] Stagnant slabs in the upper and lower mantle transition zone \cite{fuwo01}
   \item [\twothousandone] Role of subduction dynamics for regional and global plate motions \cite{befa09}
   \item [\twothousandtwo] Subduction zones \cite{ster02}
   \item [\twothousandeight]Modeling the dynamics of subducting slabs \cite{bill08}
   \item [\twothousandnine] Exhumation of oceanic blueschists and eclogites in subduction zones: Timing and mechanisms \cite{agyj09}
   \item [\twothousandnine] Role of subduction dynamics for regional and global plate motions \fullcite{befa09}
   \item [\twothousandnine] Stagnant slabs \cite{fuon09}
   \item [\twothousandten]  Slab dynamics in the transition zone \cite{bill10}
   \item [\twothousandeleven] Future directions in subduction modeling \cite{gery11}
   \item [\twothousandthirteen] Subduction Zones \cite{bufv13}
   \item [\twothousandfourteen] Rheological and geodynamic controls on the mechanisms of subduction
and HP/UHP exhumation of crustal rocks during continental collision \cite{bufa14}
   \item [\twothousandfourteen] Mechanisms of continental subduction and exhumation of HP and UHP rocks \cite{bufy14b}
   \item [\twothousandsixteen]  Continental versus oceanic subduction zones \cite{zhch16}
   \item [\twothousandseventeen] Subduction-transition zone interaction \cite{goav17}
   \item [\twothousandeighteen] Slab breakoff \cite{garm18}
   \item [\twothousandeighteen] Subduction initiation in nature and models \cite{stge18}
   \item [\twothousandtwentyone] Subduction initiation from the earliest stages to self-sustained subduction \cite{laar21}
   \item [\twothousandtwentyone] When plateau meets subduction zone \cite{lidl21}
   \item [\twothousandtwentytwo] Numerical modeling of subduction \cite{gery22}
   \item [\twothousandtwentytwo] Make subductions diverse again \cite{chmm22}
   \item [\twothousandtwentytwo] Subduction initiation triggered by collision \cite{yang22} 
   \item [\twothousandtwentythree] The thermal structure of subduction zones \cite{vawi23,wiva23}

   \end{itemize}

%==================
\section{Orogeny}
   \begin{itemize}
   \item [\nineteenseventy] Mountain Belts and the New Global Tectonics  \cite{debi70}
   \item [\nineteeneightyeight] Support, structure, and evolution of mountain belts \cite{moly88}
   \item [\twothousandtwelve] Experimental modelling of orogenic wedges: A review \cite{grmd12} 
   \item [\twothousandtwelve] Thermal–mechanical evolution of crustal orogenic belts at convergent plate boundaries \cite{vand12}
   \item [\twothousandthirteen] the origin of orogens \cite{jabe13}
   \end{itemize}

%==========================
\section{Mantle convection}

   \begin{itemize}
   \item \fullcite{tuox72}
   \item [\nineteenseventyfour] Convection in the earth’s mantle: towards a numerical simulation \cite{mcrw74}
   \item [\nineteenninetytwo] Geophysical and geochemical observations in the mantle \cite{dari92}
   \item [\nineteenninetyeight] The scales of mantle convection \cite{ande98}
   \item [\twothousandfive] Numerical and laboratory studies of mantle convection \cite{taxn05}
   \item [\twothousandeight] Mantle convection: a review \cite{ogaw08}
   \item [\twothousandtwelve] Dynamics and evolution of the deep mantle  \cite{tack12}
   \item [\twothousandeighteen] Crustal evolution and mantle dynamics through Earth history \cite{kore18}
   \item [\twothousandnineteen] Deep Mantle Water Cycle Based on the Numerical Modeling of Subducted Slabs and Global-Scale Mantle Dynamics \cite{nana19}
   \item [\twothousandtwenty] Mantle Convection in Terrestrial Planets \cite{mube20}
   \end{itemize}

%=========================
\section{Mantle \& plates}

   \begin{itemize}
   \item [\twothousandthree] The generation of plate tectonics from mantle convection \cite{berc03}
   \item [\twothousandnine] Supercontinent-superplume coupling, true polar wander and plume mobility \cite{lizh09}
   \item [\twothousandeleven] Mantle convection models featuring plate tectonic behavior \cite{lowm11}
   \item [\twothousandtwelve] Interior dynamics and long term evolution of habitable planets \cite{taab12}
   \item [\twothousandfourteen] Mantle dynamics in the Mediterranean \cite{faba14}
   \item [\twothousandfifteen] Rapid Plate Motion Variations Through Geological Time \cite{iabu15}
   \item [\twothousandseventeen] A mantle convection perspective on global tectonics \cite{cogu17}
   \end{itemize}

%=========================
\section{Mantle \& cores}

   \begin{itemize}
   \item[\twothousandtwenty] Coupled core-mantle evolution \cite{naka20}
   \end{itemize}


%===========================
\section{Plate tectonics and/or Wilson cycle}

   \begin{itemize}
   \item [\nineteeneightyeight] The Supercontinent Cycle \cite{nawm88}
   \item [\twothousandeleven] Plate Tectonics, the Wilson Cycle, LLSVPs and Mantle Plumes \cite{burk11}
   \item [\twothousandfourteen] Review of Wilson Cycle plate margins \cite{buto14}
   \item [\twothousandfourteen] The supercontinent cycle \cite{nams14}
   \item [\twothousandeighteen] The diversity of tectonic modes and thoughts about transitions between them \cite{lena18}
   \item [\twothousandnineteen] Mantle plumes and mantle dynamics in the Wilson cycle \cite{hero19}
   \item [\twothousandnineteen] Fifty years of the Wilson Cycle concept in plate tectonics \cite{wihb19}
   \item [\twothousandnineteen] Supercontinents: myths, mysteries, and milestones \cite{panm19}
   \item [\twothousandtwentytwo] Tectonic evolution of convergent plate margins \cite{zhcc22} 
   \item [\twothousandtwentythree] Deconstructing plate tectonic reconstructions \cite{sewd23}
   \item [\twothousandtwentythree] Plate tectonics in the twenty-first century \cite{zhen23}
   \item [\twothousandtwentythree] A tectonic manifesto \cite{stgt23}
   \end{itemize}

%===========================
\section{Mantle structure}

   \begin{itemize}
   \item [\nineteeneightysix] Temperature distribution in crust and mantle \cite{jemo86}
   \item [\twothousand] Heterogeneity of the lowermost mantle \cite{garn00}
   \item [\twothousandone] 5 page review of Earth's mantle structure \cite{hewo01}
   \item [\twothousandtwo] Mantle mixing: the generation, preservation, and destruction of chemical heterogeneities \cite{vahb02}
   \item [\twothousandthree] Whole-mantle convection and the transition-zone water filter \cite{beka03}
   \item [\twothousandseven] Thermo-chemical structure of the lower mantle \cite{dett07}
   \item [\twothousandtwelve] Geophysics of Chemical Heterogeneity in the Mantle \cite{stli12}
   \item [\twothousandthirteen] Caveats on tomographic images \cite{fopa13}
   \item [\twothousandfifteen] Thermally Dominated Deep Mantle LLSVPs: A Review \cite{dagl15}
   \item [\twothousandnineteen] What lies beneath? thoughts on the lower mantle. \cite{hega19}
   \end{itemize}


%=======================
\section{Plumes}

   \begin{itemize}
   \item[\nineteenseventyseven] Old paper with very funny cartoons \cite{hovo77}
   \item[\twothousandtwentyone] Mantle plumes and their role in Earth processes \cite{kobj21}
   \end{itemize}


%====================================
\section{Computational geodynamics}

   \begin{itemize}
   \item [\nineteenninetyseven] Quantification of uncertainty in computational fluid dynamics \cite{roac97}
   \item [\twothousand] Modelling plate tectonics and convection in the mantle \cite{mogz00}
   \item [\twothousandone] Overview of numerical methods for Earth simulations \cite{momd01}
   \item [\twothousandtwo] Uncertainty Quantification for Multiscale Simulations \cite{degg02}
   \item [\twothousandfive] Numerical solution of saddle point problems \cite{begl05}
   \item [\twothousandeight] Recent advances in computational geodynamics: Theory, numerics and applications \cite{kags08}
   \item [\twothousandthirteen] Overview of adaptive finite element analysis in computational geodynamics \cite{masm13}
   \item [\twothousandthirteen] What makes computational open source software libraries successful? \cite{bahe13}
   \item [\twothousandfourteen] Advances and challenges in geotectonic modelling \cite{bufy14}
   \item [\twothousandfifteen] Attributes of a community computer code \cite{comc15}
   \item [\twothousandfifteen] Attributes of a community lithospheric modeling computer code \cite{comc15}
   \item [\twothousandfifteen] Moving lithospheric modeling forward: Attributes of a community computer code \cite{comc15}
   \item [\twothousandseventeen] Software and the Scientist: Coding and Citation Practices in Geodynamics \cite{hwfs17}
   \item [\twothousandnineteen] Impact of Outreach through Software Citation for Community Software \cite{hwpc19}
   \item [\twothousandnineteen] The Role of Scientific Communities in Creating Reusable Software \cite{kehg19}
   \item [\twothousandtwenty] On the cause of continental breakup \cite{niu20}
   \item [\twothousandtwentytwo] Eighty Years of the Finite Element Method: Birth, Evolution, and Future \cite{lilp22}
   \end{itemize}

%=========================
\section{Extensional systems}

   \begin{itemize}
   \item [\twothousandsixteen] Fault linkage and relay structures in extensional settings \cite{foro16}
   \item [\twothousandseventeen] Rifted margin architecture and crustal rheology: Reviewing 
                Iberia-Newfoundland, Central South Atlantic, and South China Sea \cite{brhc17}
   \item [\twothousandnineteen] Rifted Margins: State of the Art and Future Challenges \cite{pema19}
   \item [\twothousandtwentythree] Geodynamics of continental rift initiation and evolution \cite{brko23}
   \end{itemize}

%=========================
\section{Rheology}
   \begin{itemize}
   \item [\nineteeneightythree] Rheology of the lithosphere \cite{kirb83}
   \item [\nineteeneightyseven] Rheology of the Lithosphere \cite{kikr87} \cite{ramu87}
   \item [\nineteenninetynine] The yield stress - a review \cite{barn99}
   \item [\twothousandtwo] The Origins of Rheology: A Short Historical Excursion \cite{dora02}
   \item [\twothousandthree] Modeling shear zones: solid- and fluid-thermal-mechanical approaches \cite{reyu03}
   \item [\twothousandeight] Rheology of the Lower Crust and Upper Mantle \cite{budr08}
   \item [\twothousandeight] Tectonic pressure: Theoretical concepts and modelled examples \cite{manc08}
   \item [\twothousandten] Rheology of deep upper mantle \cite{kara10}
   \item [\twothousandeleven] Rheology and strength of the lithosphere \cite{buro11}
   \item [\twothousandtwelve] Serpentine in active subduction zones \cite{reyn12}
   \item [\twothousandfourteen] Plate tectonics on terrestrial planets: From the view-point of mineral physics \cite{kara14}
   \item [\twothousandfourteen] Yielding to Stress: Recent Developments in Viscoplastic Fluid Mechanics \cite{bafo14}
   \item [\twothousandfifteen] Tectonic significance of serpentinites \cite{gusr15}
   \item [\twothousandtwentyone] Clarification of terminology conflicts \cite{wang21} 
   \item [\twothousandtwentyone] Fold geometry and folding \cite{nafo21} 
   \end{itemize}


%=========================
\section{The lithosphere}
   \begin{itemize}
   \item [\twothousandfive] Evolution of the continental lithosphere \cite{slee05}
   \item [\twothousandten] Lithosphere tectonics and thermo-mechanical properties: An integrated modelling
         approach for Enhanced Geothermal Systems exploration in Europe \cite{clvz10}
   \item [\twothousandthirteen] The behavior of the lithosphere on seismic to geologic timescales \cite{wazh13}
   \item [\twothousandfourteen] Continental transforms \cite{noto14}
   \item [\twothousandseventeen] The structural evolution of the deep continental lithosphere \cite{comm17}
   \end{itemize}

%====================================
\section{Gravity \& Geoid studies}
   \begin{itemize}
   \item Long wavelength gravity and topography anomalies \cite{wada81}
   \item The geological significance of the geoid \cite{chas85}
   \item Observing Global Mass Transport to Understand Global Change and to benefit society \cite{pabb15}
   \item Understandin deep earth dynamics: a numerical modelling approach \cite{siag17}
   \item Gravity observations and 3D structure of the Earth \cite{ricl06}
   \item A Brief Tour into the History of Gravity: From Emocritus to Einstein \cite{pamo13}
   \end{itemize}

%=========================
\section{Salt tectonics}
   \begin{itemize}
   \item Salt tectonics at passive margins: Geology versus models \cite{brfo11}
   \item The Role of Salt Tectonics in the Energy Transition \cite{duhp23}
   \end{itemize}

%=================================
\section{Miscellaneous}
\begin{itemize}
\item Planetary Magnetic Fields and Fluid Dynamos \cite{jone11}
\item Analogue modelling: historical outline \cite{koyi97}; Approaches, scaling, materials and quantification, with an application to subduction experiments \cite{scst16}
\item Exhumation of (ultra-)high-pressure terranes: concepts and mechanisms \cite{warr13}
\item Paradigms, new and old, for ultra-high-pressure tectonism \cite{hage13}
\item The role of solid-solid phase transitions in mantle convection \cite{fada17}
\item Verification, validation and confirmation of numerical models \cite{orsb94}
\item Structure and dynamics of the mantle wedge \cite{vank03}
\item Mountain building, observations and models of dynamic topgraphy \cite{flgm13,fabc13}
\item Reconciling laboratory and observational models of mantle rheology in geodynamic modelling \cite{king16}
\item Controlling parameters, surface expressions and the future directions in delamination modeling \cite{goue18}
\item Structural dynamics of salt systems \cite{javs94}
\item Crustal versus mantle core complexes \cite{brst18}
\item Precambrian geodynamics: concepts and models \cite{gery14}
\item A review of brittle compressional wedge models \cite{buit12}
\item accreted terranes: a compilation of island arcs, oceanic
      plateaus, submarine ridges, seamounts, and continental fragments \cite{tebu14}
\item Hotspot swells \cite{kiad14}
\item Theory of scale models as applied to the study of geologic structures \cite{hubb37}
\item Dynamic Topography and Ice Age Paleoclimate \cite{miac20}
\item Coupled surface to deep Earth processes: Perspectives from TOPO-EUROPE
with an emphasis on climate- and energy-related societal challenges \cite{clsk23}
\item How to efficiently debug computational solid mechanics models so
you can enjoy the beauty of simulations \cite{copb23}
\item The solid Earth's influence on sea level \cite{conr13}  
\item Vening Meinesz \cite{vlaa89}
\item The geoscience of coupled deep Earth-surface processes in Europe \cite{clzb07}
\end{itemize}







\chapter{Geodynamics} 
\begin{flushright} {\tiny {\color{gray} topics.tex}} \end{flushright}

This is a {\it very} rough attempt at classifying my somewhat extensive 
bibliography per theme/topic.
It goes without saying that this cannot be extensive and that since I 
started computational geodynamics myself around twothousandsix these lists are 
biaised towards the last 2 decades or so. 
In retrospect, the categories I have chosen could have been subdivided
into narrower fields. I understand that having 100+ references 
for 'subduction'  or 'mantle convection' is not particularly useful, 
but it means that all these papers show up in the bibliography section 
of this book, and the titles of said papers are then searchable per keyword.



%--------------------------------------------------------------------
%--------------------------------------------------------------------
\subsection{Big review papers - very good for students}
%--------------------------------------------------------------------
%--------------------------------------------------------------------

\begin{itemize}

\item Subduction
   \begin{itemize}
   \item [\nineteeneightytwo] Controls of subduction geometry, location of magmatic arcs, 
         and tectonics of (back-)arc regions \cite{crpi82}
   \item [\nineteenninetyfive] From the trench to the core-mantle boundary \cite{kinc95}
   \item [\twothousandone] Stagnant slabs in the upper and lower mantle transition region \cite{fuwo01}
   \item [\twothousandone] A Review of the Role of Subduction Dynamics for Regional and Global Plate Motions \cite{befa09}
   \item [\twothousandtwo] Subduction zones \cite{ster02}
   \item [\twothousandeight] Modeling the subduction dynamics \cite{bill08}
   \item [\twothousandnine] Exhumation of oceanic blueschists and eclogites in subduction zones \cite{agyj09}
   \item [\twothousandnine] A review of the role of subduction dynamics for regional and global plate motions \cite{befa09}
   \item [\twothousandnine] Stagnant Slab: A Review \cite{fuon09}
   \item [\twothousandten] Slab dynamics in the transition zone \cite{bill10}
   \item [\twothousandeleven] Future directions in subduction modeling \cite{gery11}
   \item [\twothousandthirteen] Introduction to the special issue on ``Subduction Zones'' of Solid Earth \cite{bufv13}
   \item [\twothousandfourteen] Rheological and geodynamic controls on the mechanisms of subduction and HP/UHP exhumation 
                of crustal rocks during continental collision \cite{bufa14,bufy14b}
   \item [\twothousandseventeen] Subduction-transition zone interaction: A review \cite{goav17}
   \item [\twothousandeighteen] Slab breakoff: A critical appraisal of a geological theory as applied in space and time \cite{garm18}
   \item [\twothousandeighteen] Subduction initiation in nature and models \cite{stge18}
   \end{itemize}

\item Orogeny:
   \begin{itemize}
   \item [\nineteenseventy] Mountain Belts and the New Global Tectonics  \cite{debi70}
   \item [\nineteeneightyeight] Support, structure, and evolution of mountain belts \cite{moly88}
   \item [\twothousandtwelve] Experimental modelling of orogenic wedges: A review \cite{grmd12} 
   \item [\twothousandthirteen] the origin of orogens \cite{jabe13}
   \end{itemize}

\item Mantle convection 

   \begin{itemize}
   \item [\nineteenninetytwo] Geophysical and geochemical observations in the mantle \cite{dari92}
   \item [\nineteenninetyeight] The scales of mantle convection \cite{ande98}
   \item [\twothousandfive] Numerical and laboratory studies of mantle convection \cite{taxn05}
   \item [\twothousandeight] Mantle convection: a review \cite{ogaw08}
   \item [\twothousandtwelve] Dynamics and evolution of the deep mantle  \cite{tack12}
   \item [\twothousandeighteen] Crustal evolution and mantle dynamics through Earth history \cite{kore18}
   \item [\twothousandtwenty] Mantle Convection in Terrestrial Planets \cite{mube20}
   \end{itemize}

\item Mantle \& plates:
   \begin{itemize}
   \item [\twothousandthree] The generation of plate tectonics from mantle convection \cite{berc03}
   \item [\twothousandnine] Supercontinent-superplume coupling, true polar wander and plume mobility \cite{lizh09}
   \item [\twothousandeleven] Mantle convection models featuring plate tectonic behavior \cite{lowm11}
   \item [\twothousandtwelve] Interior dynamics and long term evolution of habitable planets \cite{taab12}
   \item [\twothousandfourteen] Mantle dynamics in the Mediterranean \cite{faba14}
   \item [\twothousandfifteen] Rapid Plate Motion Variations Through Geological Time \cite{iabu15}
   \item [\twothousandseventeen] A mantle convection perspective on global tectonics \cite{cogu17}
   \end{itemize}

\item Plate tectonics and/or Wilson cycle
   \begin{itemize}
   \item [\twothousandeleven] Plate Tectonics, the Wilson Cycle, LLSVPs and Mantle Plumes \cite{burk11}
   \item [\twothousandfourteen] Review of Wilson Cycle plate margins \cite{buto14}
   \item [\twothousandeighteen] The diversity of tectonic modes and thoughts about transitions between them \cite{lena18}
   \item [\twothousandnineteen] Fifty years of the Wilson Cycle concept in plate tectonics \cite{wihb19}
   \end{itemize}

\item Mantle structure
   \begin{itemize}
   \item [\nineteeneightysix] Temperature distribution in crust and mantle \cite{jemo86}
   \item [\twothousand] Heterogeneity of the lowermost mantle \cite{garn00}
   \item [\twothousandone] 5 page review of Earth's mantle structure \cite{hewo01}
   \item [\twothousandtwo] Mantle mixing: the generation, preservation, and destruction of chemical heterogeneities \cite{vahb02}
   \item [\twothousandthree] Whole-mantle convection and the transition-zone water filter \cite{beka03}
   \item [\twothousandseven] Thermo-chemical structure of the lower mantle \cite{dett07}
   \item [\twothousandtwelve] Geophysics of Chemical Heterogeneity in the Mantle \cite{stli12}
   \item [\twothousandthirteen] Caveats on tomographic images \cite{fopa13}
   \end{itemize}

\item Plumes
   \begin{itemize}
   \item[\nineteenseventyseven] Old paper with very funny cartoons \cite{hovo77}
   \item[\twothousandtwentyone] Mantle plumes and their role in Earth processes \cite{kobj21}
   \end{itemize}


\item Computational geodynamics
   \begin{itemize}
   \item [\nineteenninetyseven] Quantification of uncertainty in computational fluid dynamics \cite{roac97}
   \item [\twothousand] Modelling plate tectonics and convection in the mantle \cite{mogz00}
   \item [\twothousandone] Overview of numerical methods for Earth simulations \cite{momd01}
   \item [\twothousandtwo] Uncertainty Quantification for Multiscale Simulations \cite{degg02}
   \item [\twothousandfive] Numerical solution of saddle point problems \cite{begl05}
   \item [\twothousandeight] Recent advances in computational geodynamics: Theory, numerics and applications \cite{kags08}
   \item [\twothousandthirteen] Overview of adaptive finite element analysis in computational geodynamics \cite{masm13}
   \item [\twothousandthirteen] What makes computational open source software libraries successful? \cite{bahe13}
   \item [\twothousandfourteen] Advances and challenges in geotectonic modelling \cite{bufy14}
   \item [\twothousandfifteen] Attributes of a community computer code \cite{comc15}
   \item [\twothousandfifteen] Attributes of a community lithospheric modeling computer code \cite{comc15}
   \item [\twothousandfifteen] Moving lithospheric modeling forward: Attributes of a community computer code \cite{comc15}
   \item [\twothousandseventeen] Software and the Scientist: Coding and Citation Practices in Geodynamics \cite{hwfs17}
   \item [\twothousandnineteen] Impact of Outreach through Software Citation for Community Software \cite{hwpc19}
   \item [\twothousandnineteen] The Role of Scientific Communities in Creating Reusable Software \cite{kehg19}
   \item [\twothousandtwenty] On the cause of continental breakup \cite{niu20}
   \end{itemize}

\item Extensional systems
   \begin{itemize}
   \item [\twothousandsixteen] Fault linkage and relay structures in extensional settings \cite{foro16}
   \item [\twothousandseventeen] Rifted margin architecture and crustal rheology: Reviewing 
                Iberia-Newfoundland, Central South Atlantic, and South China Sea \cite{brhc17}
   \item [\twothousandnineteen] Rifted Margins: State of the Art and Future Challenges \cite{pema19}\\
   \end{itemize}

\item Rheology \index{topics}{Rheology}
   \begin{itemize}
   \item [\nineteeneightythree] Rheology of the lithosphere \cite{kirb83}
   \item [\nineteeneightyseven] Rheology of the Lithosphere \cite{kikr87} \cite{ramu87}
   \item [\nineteenninetynine] The yield stress - a review \cite{barn99}
   \item [\twothousandtwo] The Origins of Rheology: A Short Historical Excursion \cite{dora02}
   \item [\twothousandthree] Modeling shear zones: solid- and fluid-thermal-mechanical approaches \cite{reyu03}
   \item [\twothousandeight] Rheology of the Lower Crust and Upper Mantle \cite{budr08}
   \item [\twothousandeight] Tectonic pressure: Theoretical concepts and modelled examples \cite{manc08}
   \item [\twothousandten] Rheology of deep upper mantle \cite{kara10}
   \item [\twothousandeleven] Rheology and strength of the lithosphere \cite{buro11}
   \item [\twothousandtwelve] Serpentine in active subduction zones \cite{reyn12}
   \item [\twothousandfourteen] Plate tectonics on terrestrial planets: From the view-point of mineral physics \cite{kara14}
   \item [\twothousandfifteen] Tectonic significance of serpentinites \cite{gusr15}
   \item [\twothousandtwentyone] Clarification of terminology conflicts \cite{wang21} 
   \end{itemize}

\item Miscellaneous
   \begin{itemize}
   \item The solid Earth's influence on sea level \cite{conr13}  \index{topics}{Sea Level}
   \item Vening Meinesz \cite{vlaa89}
   \item The geoscience of coupled deep Earth-surface processes in Europe \cite{clzb07}
   \end{itemize}

\item The lithosphere
   \begin{itemize}
   \item [\twothousandfive] Evolution of the continental lithosphere \cite{slee05}
   \item [\twothousandten] Lithosphere tectonics and thermo-mechanical properties: An integrated modelling
         approach for Enhanced Geothermal Systems exploration in Europe \cite{clvz10}
   \item [\twothousandthirteen] The behavior of the lithosphere on seismic to geologic timescales \cite{wazh13}
   \item [\twothousandfourteen] Continental transforms \cite{noto14}
   \item [\twothousandseventeen] The structural evolution of the deep continental lithosphere \cite{comm17}
   \end{itemize}

\item Gravity \& Geoid studies
   \begin{itemize}
   \item Long wavelength gravity and topography anomalies \cite{wada81}
   \item The geological significance of the geoid \cite{chas85}
   \item Observing Global Mass Transport to Understand Global Change and to benefit society \cite{pabb15}
   \end{itemize}

\item Planetary Magnetic Fields and Fluid Dynamos \cite{jone11}


\item Analogue modelling: historical outline \cite{koyi97}; Approaches, scaling, materials and quantification, with an application to subduction experiments \cite{scst16}
\item Exhumation of (ultra-)high-pressure terranes: concepts and mechanisms \cite{warr13}
\item Paradigms, new and old, for ultra-high-pressure tectonism \cite{hage13}
\item The role of solid-solid phase transitions in mantle convection \cite{fada17}
\item Verification, validation and confirmation of numerical models \cite{orsb94}
\item Experimental modelling of orogenic wedges \cite{grmd12}
\item Structure and dynamics of the mantle wedge \cite{vank03}
\item Mountain building, observations and models of dynamic topgraphy \cite{flgm13,fabc13}
\item Reconciling laboratory and observational models of mantle rheology in geodynamic modelling \cite{king16}
\item Controlling parameters, surface expressions and the future directions in delamination modeling \cite{goue18}
\item Salt tectonics at passive margins: Geology versus models \cite{brfo11}
\item Structural dynamics of salt systems \cite{javs94}
\item Crustal versus mantle core complexes \cite{brst18}
\item Precambrian geodynamics: concepts and models \cite{gery14}
\item A review of brittle compressional wedge models \cite{buit12}
\item accreted terranes: a compilation of island arcs, oceanic
      plateaus, submarine ridges, seamounts, and continental fragments \cite{tebu14}
\item Hotspot swells \cite{kiad14}
\item Theory of scale models as applied to the study of geologic structures \cite{hubb37}
\item Dynamic Topography and Ice Age Paleoclimate \cite{miac20}
\end{itemize}

%------------------------------------------------------------------------------
%------------------------------------------------------------------------------
\subsection{Analogue modelling}
\index{topics}{Analogue Modelling}
%------------------------------------------------------------------------------
%------------------------------------------------------------------------------

\begin{scriptsize}
\begin{itemize}
\item[\nineteenseventyfive]
\textcite{dixo75} \citetitle{dixo75}\\
\item[\nineteeneightytwo]
\textcite{tapl82} \citetitle{tapl82}\\
\item[\nineteeneightyeight] 
\textcite{peta88} \citetitle{peta88}\\
\textcite{crud88} \citetitle{crud88}\\
\item[\nineteenninety]
\textcite{mccl90} \citetitle{mccl90}\\ 
\textcite{jodc90} \citetitle{jodc90}\\
\item[\nineteenninetyone]
\textcite{daco91}
\item[\nineteenninetytwo]
\textcite{salt92}
\item[\nineteenninetythree]
\textcite{nabr93}
\textcite{shem93}
\item[\nineteenninetyseven] 
\textcite{vank97}
\item[\nineteenninetyeight] 
\textcite{bubr98}
\item[\nineteenninetynine] 
\textcite{dava99}
\textcite{befo99}
\textcite{fagd99}
\textcite{nagg99}
\item[\twothousand]
\textcite{sche00}
\textcite{sobm00}
\textcite{chlb00}  
\textcite{mime00}
\item[\twothousandone] 
\textcite{haki01}
\textcite{chys01}  
\textcite{lirc01}
\item[\twothousandtwo] 
\textcite{dagl02}
\item[\twothousandthree] 
\textcite{smbs03}
\textcite{muso03}
\textcite{nagv03}
\item[\twothousandfour] 
\textcite{sche04}
\textcite{sche04b}
\item[\twothousandfive] 
\textcite{jujb05}
\textcite{sche05}
\textcite{sobb05}
\item[\twothousandsix] 
\textcite{scbb06}
\textcite{tibs06}
\textcite{crnp06}
\textcite{lemm06}
\textcite{pabs06}
\textcite{malm06}
\item[\twothousandseven] 
\textcite{socb07}
\item[\twothousandeight] 
\textcite{clbz08}
\textcite{fufh08}
\textcite{esfm08}
\item[\twothousandnine] 
\textcite{pina09}
\textcite{bonn09}
\item[\twothousandeleven] 
\textcite{dalt11}
\textcite{gopc11}
\textcite{grhd11}
\item[\twothousandtwelve] 
\textcite{grmd12}
\textcite{iadc12}
\item[\twothousandthirteen] 
\textcite{luws13}
\textcite{vadv13}
\textcite{guhf13}
\textcite{mibg13}
\textcite{mesc13}
\textcite{dusc13}
\textcite{kern13} 
\textcite{wakk13}
\item[\twothousandfifteen] 
\textcite{casw15}
\textcite{rods15}
\textcite{kiff15}
\textcite{chsd15}
\item[\twothousandsixteen] 
\textcite{scbb16}
\textcite{chss16}
\item[\twothousandseventeen]
\textcite{casw17}
\item[\twothousandeighteen] 
\textcite{pirf18}
\textcite{bews18} 
\item[\twothousandnineteen] 
\textcite{mocb19}
\textcite{sccs19}
\textcite{muwm19}
\textcite{fegb19}
\item[\twothousandtwenty] 
\textcite{zwsr20} \citetitle{zwsr20}\\ 
\textcite{kiph20} \citetitle{kiph20}\\
\textcite{daro20} \citetitle{daro20}\\
\end{itemize}
\end{scriptsize}

%------------------------------------------------------------------------------
%------------------------------------------------------------------------------
\subsection{Archean tectonics, Hadean Earth, early Earth}
\index{topics}{Archean tectonics}
\index{topics}{Early Earth}
\index{topics}{Hedean Earth}
%------------------------------------------------------------------------------
%------------------------------------------------------------------------------

\begin{scriptsize}
\begin{itemize}
\item[\nineteeneightyfour]   
\textcite{boas84} \citetitle{boas84}\\
\item[\nineteeneightynine]   
\textcite{cagh89} \citetitle{cagh89}\\
\item[\nineteenninetyfour]   
\textcite{vlvv94} \citetitle{vlvv94}\\
\item[\nineteenninetysix]    
\textcite{kafo96} \citetitle{kafo96}\\
\item[\twothousand]          
\textcite{devv00b} \citetitle{devv00b}\\
\item[\twothousandfour]      
\textcite{vavv04} \citetitle{vavv04}\\
\textcite{vavv04b} \citetitle{vavv04b}\\
\item[\twothousandsix]       
\textcite{reho06} \citetitle{reho06}\\
\item[\twothousandeight]     
\textcite{vava08} \citetitle{vava08}\\
\item[\twothousandten]       
\textcite{grpy10} \citetitle{grpy10}\\
\item[\twothousandfifteen]   
\textcite{maha15} \citetitle{maha15}\\
\item[\twothousandsixteen]   
\textcite{onlw16} \citetitle{onlw16}\\ 
\textcite{fige16} \citetitle{fige16}\\
\item[\twothousandseventeen] 
\textcite{onmz17} \citetitle{onmz17}\\
\item[\twothousandeighteen]  
\textcite{fole18} \citetitle{fole18}\\
\item[\twothousandnineteen]  
\textcite{canc19} \citetitle{canc19}\\ 
\textcite{gery19} \citetitle{gery19}\\
\item[\twothousandtwenty]    
\textcite{chcg20} \citetitle{chcg20}\\ 
\textcite{grco20} \citetitle{grco20}\\
\textcite{canc20} \citetitle{canc20}\\
\textcite{gumc20} \citetitle{gumc20}\\
\textcite{fole20} \citetitle{fole20}\\
\item[\twothousandtwentyone]
\textcite{pegz21} \citetitle{pegz21}    
\end{itemize}
\end{scriptsize}


%------------------------------------------------------------------------------
%------------------------------------------------------------------------------
\subsection{Asymmetry}
\label{sec:topics:asymmetry}
\index{topics}{Asymmetry}
%------------------------------------------------------------------------------

\begin{scriptsize}
\begin{itemize}
\item[1989]
\textcite{brbe89b} \citetitle{brbe89b}\\
\item[1993]
\textcite{gowo93} \citetitle{gowo93}\\
\item[2003]
\textcite{hube03} \citetitle{hube03}\\
\item[2006]
\textcite{coma06} \citetitle{coma06}\\
\item[2008]
\textcite{vanv08} \citetitle{vanv08}\\
\textcite{naht08} \citetitle{naht08}\\
\item[2011]
\textcite{vanj11} \citetitle{vanj11}\\
\item[2014]
\textcite{buge14} \citetitle{buge14}\\
\textcite{flgw14} \citetitle{flgw14}\\
\item[2015]
\textcite{svlh15} \citetitle{svlh15}\\
\item[2016]
\textcite{frsc16} \citetitle{frsc16}\\
\end{itemize}
\end{scriptsize}


%------------------------------------------------------------------------------
%------------------------------------------------------------------------------
\subsection{Anisotropy, Lattice/Crystal preferred orientation, SKS splitting}
\label{sec:topics:anisotropy}
\index{topics}{Anisotropy} 
\index{topics}{LPO/CPO} 
\index{topics}{SKS splitting}
%------------------------------------------------------------------------------
%------------------------------------------------------------------------------

\begin{scriptsize}
\begin{itemize}
\item[\nineteeneightynine] 
\cite{ribe89b}
\cite{ribe89c}
\item[\nineteenninetyone] 
\cite{riyu91}
\item[\nineteenninetytwo] 
\cite{ribe92b}
\item[\nineteeneightythree] 
\cite{zhhj93}
\item[\twothousandtwo] 
\cite{mudm02} 
\cite{mcvk02} 
\cite{kari02}
\item[\twothousandthree] 
\cite{mumc03} 
\cite{mcvk03}
\cite{beke03}
\item[\twothousandfour] 
\cite{mumc04} 
\cite{karb04}
\item[\twothousandsix] 
\cite{besb06}
\cite{lafh06}
\item[\twothousandseven] 
\cite{cobs07} \cite{cobs07}\\
\cite{rimb07} \cite{rimb07}\\
\cite{lopk07} \cite{lopk07}\\
\item[\twothousandeight] 
\textcite{beke08} \citetitle{beke08}\\
\textcite{beck08} \citetitle{beck08}\\
\item[\twothousandnine] 
\textcite{tokv09} \citetitle{tokv09}\\
\item[\twothousandten] 
\textcite{cobe10} \citetitle{cobe10}\\
\textcite{jabi10a} \citetitle{jabi10a}\\
\item[\twothousandeleven] 
\textcite{obbh11} \citetitle{obbh11}\\
\textcite{scbb11} \citetitle{scbb11}\\
\item[\twothousandtwelve] 
\textcite{mibe12} \citetitle{mibe12}\\
\textcite{ruma12} \citetitle{ruma12}\\
\item[\twothousandthirteen] 
\textcite{faca13} \citetitle{faca13}\\
\textcite{almb13} \citetitle{almb13}\\
\item[\twothousandfourteen] 
\textcite{facc14} \citetitle{facc14} \\
\textcite{diwl14} \citetitle{diwl14}\\
\textcite{lidr14} \citetitle{lidr14}\\
\item[\twothousandfifteen] 
\textcite{ealw15} \citetitle{ealw15}\\
\textcite{gorc15} \citetitle{gorc15}\\
\item[\twothousandseventeen] 
\textcite{majf17} \citetitle{majf17}\\ 
\textcite{hegd17} \citetitle{hegd17}\\
\item[\twothousandeighteen] 
\textcite{peka18} \citetitle{peka18}\\
\item[\twothousandnineteen] 
\textcite{mats19} \citetitle{mats19}\\
\textcite{stff19} \citetitle{stff19}\\
\textcite{fefs19} \citetitle{fefs19}\\
\item[\twothousandtwentyone] 
\textcite{hafw21} \citetitle{hafw21}\\
\textcite{mabh21} \citetitle{mabh21}\\
\textcite{mota21} \citetitle{mota21}\\ 
\textcite{frbi21} \citetitle{frbi21}\\ 
\end{itemize}
\end{scriptsize}

%------------------------------------------------------------------------------
%------------------------------------------------------------------------------
\subsection{Benchmark, analytical solutions, code comparisons, methodology, num. methods, theory}
%------------------------------------------------------------------------------
%------------------------------------------------------------------------------

{\color{red} this category makes little sense ... should be split? removed? }

\begin{scriptsize}
\nineteenseventyfour: Hirt \etal \cite{hiac74}\\
\nineteenseventyfive: Wakiya \cite{waki75a,waki75b}\\
\nineteeneightyfour: Yuen \& Sabadini \cite{yusa84}, Smolarkiewicz \cite{smol84}\\
\nineteeneightynine: Blankenbach \etal \cite{blbc89}\\
\nineteenninety: Travis \etal \cite{trab90}\\
\nineteenninetythree: Lenardic \& Kaula \cite{leka93}\\
\nineteenninetyfour: Braun \& Sambridge \cite{brsa94}\\
\nineteenninetyfive: Braun \& Sambridge \cite{brsa95}, Moresi \& Solomatov \cite{moso95}, 
                     Fullsack \cite{full95}\\
\nineteenninetysix: Zhong \cite{zhon96}, Moresi \etal \cite{mozg96}\\
\nineteenninetyseven: Ristow \cite{rist97}\\
\nineteenninetynine: Lindgren \cite{lind99}, Bird \cite{bird99}\\
\twothousandone: Moresi \etal\cite{modm01}, van Keken \cite{vank01}\\
\twothousandtwo: M{\"u}hlhaus \etal \cite{mudm02}\\
\twothousandthree: \cite{taki03}\cite{modm03}\cite{geyu03}\cite{geyu03b}\cite{taxi03}\cite{scpo03}\\
\twothousandfour: \cite{kaps04}\cite{kasa04}\cite{kaks08}\cite{mumc04}\\
\twothousandfive: \cite{mure05}\\
\twothousandsix: \cite{kapo06}\cite{more06}\cite{onmm06}\cite{mudm06}\cite{tact06}\\
\twothousandseven: \cite{toma07}\cite{chcc07}, Kaus \& Becker \cite{kabe07}, \cite{kaks07}\cite{moql07}\cite{geyu07}\cite{dadh07}
      \cite{zldf07}\\
\twothousandeight: \cite{zhmt08}\cite{deka08}\cite{trub08}\cite{krdp08}\cite{mamo08}\cite{gepd98}
      \cite{vack08}\cite{heta08}\cite{brtf08}\cite{daks08}\cite{chzy08}\cite{tack08}\cite{hust08b}\\
\twothousandnine: King \cite{king09}, Geenen \etal \cite{geum09}, Velic \etal \cite{vemm09}, 
                  Quinteros \etal \cite{qurj09}\\
\twothousandten: \cite{kaus10}\cite{kamm10}\cite{egat10}\cite{kilv10}\\
\twothousandeleven: \cite{dumg11}\cite{uibb11}\cite{hegc11}\cite{muso11}\cite{dawk11}\cite{lemm11}\\
\twothousandtwelve: \cite{crsg12}\cite{chgv12}\cite{krwd12}\cite{may12}\cite{gerb12}\cite{asmo12}\\
\twothousandthirteen: \cite{chtl13}\cite{kemk13}\cite{gemd13}\cite{hutm13}\\
\twothousandfourteen: \cite{thmk14}\cite{mabl14}\cite{lopp14}\cite{stlh14}\\
\twothousandfifteen: \cite{lelk15}\cite{rumi15}\cite{chpe15}\cite{mabl15}\\
\twothousandsixteen: \cite{dumy16}\cite{blmp16}\\
\twothousandseventeen: \cite{robh17}\cite{wisv17}\cite{majc17}\\
\twothousandeighteen: Meriaux \etal \cite{memm18}, Crameri \cite{cram18}, Wieczorek \& Meshede \cite{wime18}\\
\twothousandnineteen: \cite{liki19}\cite{demh19}\cite{galb19}\cite{frtv19}\cite{yuwa19}\cite{ropu19}\\
\twothousandtwenty: \cite{homb20}\cite{trlb20}\cite{gadb20}\cite{jaca20a,jaca20b} 
\twothousandtwentyone: Clevenger \& Heister \cite{clhe21}
\end{scriptsize}

%------------------------------------------------------------------------------
%------------------------------------------------------------------------------
\subsection{Continental crust} 
\index{topics}{Continental Crust}
%------------------------------------------------------------------------------
%------------------------------------------------------------------------------

\begin{scriptsize}
\begin{itemize}
\item[\nineteeneightysix] Chapman \cite{chap86}, Barton \cite{bart86}
\item[\nineteeneightynine] Ord \& Hobbs \cite{ord89}
\item[\nineteenninetyfour] Sawyer \cite{sawy94}
\item[\twothousandone] Doin \& Henry \cite{dohe01}
\item[\twothousandfour] Gerya \etal \cite{gepm04}
\item[\twothousandthirteen] Castro \etal \cite{cavg13}, Tirel \etal \cite{tibb13}
\item[\twothousandnineteen] Schmeling \etal \cite{scmw19}
\end{itemize}
\end{scriptsize}


%------------------------------------------------------------------------------
%------------------------------------------------------------------------------
\subsection{Oceanic crust} 
\index{topics}{Oceanic Crust}

\begin{scriptsize}
\begin{itemize}
\item[\nineteeneightyeight] 
\textcite{mofo88} \citetitle{mofo88}\\
\item[\nineteenninetyfour] 
\textcite{chho94} \citetitle{chho94}\\
\item[\nineteenninetysix] 
\textcite{vaky96} \citetitle{vaky96}\\
\item[\twothousandfour] 
\textcite{vavv04b} \citetitle{vavv04b}\\
\item[\twothousandseven] 
\textcite{brva07b} \citetitle{brva07b}\\
\item[\twothousandeight] 
\textcite{gomm08} \citetitle{gomm08}\\
\item[\twothousandthirteen] 
\textcite{limc13} \citetitle{limc13}\\
\textcite{yosh13} \citetitle{yosh13}\\
\item[\twothousandfifteen] 
\textcite{rula15} \citetitle{rula15}\\
\item[\twothousandseventeen] 
\textcite{taac17} \citetitle{taac17}\\
\item[\twothousandtwenty] 
\textcite{mugu20} \citetitle{mugu20}\\
\textcite{yabt20} \citetitle{yabt20}\\
\end{itemize}
\end{scriptsize}


%------------------------------------------------------------------------------
%------------------------------------------------------------------------------
\subsection{Core dynamics, core formation, CMB temperature/heat flux}
\index{topics}{Core Dynamics} 
\index{topics}{CMB}
%------------------------------------------------------------------------------
%------------------------------------------------------------------------------

\begin{scriptsize}
\begin{itemize}
\item[\nineteenninetysix] Hansen \& Yuen \cite{hayu96}, Boehler \cite{boeh96}
\item[\nineteenninetyeight] van den Berg \& Yuen \cite{vayu98}
\item[\twothousandfour] Nakagawa \& Tackley \cite{nata04c}
\item[\twothousandseven] Petford \etal \cite{pery07}
\item[\twothousandeight] Lay \etal \cite{lahb08}, Golabek \etal \cite{gost08}, Samuel \& Tackley \cite{sata08}
\item[\twothousandnine] King \etal \cite{kisn09}\\
\item[\twothousandten] Nakagawa \& Tackley \cite{nata10}, Lassak \etal \cite{lamg10}, 
                       Samuel \etal \cite{sate10}
\item[\twothousandeleven] Zhang \& Zhong  \cite{zhzh11}, Deguen \& Cardin \cite{deca11}
\item[\twothousandtwelve] Cottaar \& Buffett  \cite{cobu12}
\item[\twothousandtwelve] Truemper \etal  \cite{trbh12}
\item[\twothousandthirteen] Nakagawa \& Tackley  \cite{nata13}
\item[\twothousandeighteen] Langemeyer \etal  \cite{lalt18}
\item[\twothousandnineteen] Yin \etal  \cite{yiym19}, Bouffard \etal \cite{bocl19}
\item[\twothousandtwenty] Heyn \etal \cite{hect20}, Lesher \etal \cite{ledb20}
\end{itemize}
\end{scriptsize}

%------------------------------------------------------------------------------
%------------------------------------------------------------------------------
\subsection{Compressible flow}
\index{topics}{Compressible Flow}

\begin{scriptsize}
\begin{itemize}
\item[\nineteensixty] Spiegel \& Veronis \cite{spve60}
\item[\nineteeneighty] Jarvis \& McKenzie \cite{jamc80}
\item[\nineteeneightyseven]  Yuen \etal \cite{yuqh87}
\item[\nineteeneightyeight] Glatzmaier \cite{glat88}, Yuen \etal \cite{yuzl88} 
\item[\nineteeneightynine] Machetel \& Yuen \cite{mayu89} 
\item[\nineteenninetytwo] Bercovici \etal \cite{besg92}, Balachandar \etal \cite{bayr92}
\item[\nineteenninetysix] Tackley \cite{tack96}, Zhang \& Yuen \cite{zhyu96}
\item[\nineteenninetyeight] Mittal \& Tezduyar \cite{mite98} 
\item[\twothousandfour] Nakagawa \& Tackley \cite{nata04}
\item[\twothousandfive] Hauke \etal \cite{halg05a,halg05b}
\item[\twothousandseven] Feistauer \& Ku{\v{c}}era \cite{feku07} 
\item[\twothousandeight] Tackley \cite{tack08}, Leng \& Zhong \cite{lezh08}, Trubitsyn \cite{trub08}
\item[\twothousandnine] Taliadorou \etal \cite{tagm09} 
\item[\twothousandten] King \etal \cite{kilv10}
\item[\twothousandeleven] Tan \etal \cite{talz11}
\item[\twothousandtwelve] Bollada \& Philips \cite{boph12}
\item[\twothousandthirteen] \cite{lizh13}, Shahraki \& Schmeling \cite{shsc13}
\item[\twothousandfifteen] Kameyama \etal \cite{kamo15}
\item[\twothousandsixteen] Ghelichkhan \& Bunge \cite{ghbu16}
\item[\twothousandeighteen] Colli \etal \cite{cogb18}, Ghelichkhan \& Bunge \cite{ghbu18}
\item[\twothousandnineteen] Curbelo \etal \cite{cuda19}, de Montserrat \etal \cite{demh19}
\item[\twothousandtwenty] Gassm{\"o}ller \etal \cite{gadb20}
\end{itemize}
\end{scriptsize}

%------------------------------------------------------------------------------
%------------------------------------------------------------------------------
\subsection{Computational Structural geology}
\index{topics}{Structural Geology}
%------------------------------------------------------------------------------
%------------------------------------------------------------------------------

\begin{scriptsize}
\begin{itemize}
\item[\nineteenseventyone] \cite{stbe71}
\item[\nineteenninetytwo] Barr \& Houseman \cite{baho92}
\item[\nineteenninetythree] Ildefonse \& Mancktelow \cite{ilma93}
\item[\nineteenninetyfive] \cite{fige95}
\item[\nineteenninetysix] Barr \& Houseman \cite{baho96}, Herrmann \etal \cite{hept96}
\item[\twothousand] \cite{acgf00}\cite{trla00}
\item[\twothousandone] \cite{masc01}
\item[\twothousandsix] Crook \etal \cite{crwy06}
\item[\twothousandeight] \cite{manc08}\cite{scsf08}
\item[\twothousandeleven] \cite{frem11}
\item[\twothousandthirteen] \cite{soma13}\cite{lehl13}
\item[\twothousandfourteen] \cite{olbm14}
\item[\twothousandfifteen] \cite{pevp15}\cite{jalr15}
\item[\twothousandseventeen] Nabavi \etal \cite{naam17}, \cite{scdu17}
\item[\twothousandeighteen] Nabavi \etal \cite{naam18}, Webber \etal \cite{weef18}
\item[\twothousandnineteen] \cite{llor19}\cite{yada19}\cite{sogh19}
\end{itemize}
\end{scriptsize}

%------------------------------------------------------------------------------
%------------------------------------------------------------------------------
\subsection{Channel flow model} 
\index{topics}{Channel Flow}
%------------------------------------------------------------------------------
%------------------------------------------------------------------------------

\begin{scriptsize}
\begin{itemize}
\item[\twothousand] 
\textcite{clro00} \citetitle{clro00}\\
\item[\twothousandfour] 
\textcite{bejn04} \citetitle{bejn04}\\
\textcite{jabm04} \citetitle{jabm04}\\
\item[\twothousandsix] 
\textcite{jabn06} \citetitle{jabn06}\\
\textcite{mebe06} \citetitle{mebe06}\\
\textcite{benj06} \citetitle{benj06}\\
\item[\twothousandseven] 
\textcite{jabn07} \citetitle{jabn07}\\
\item[\twothousandeleven] 
\textcite{jabe11} \citetitle{jabe11}\\
\end{itemize}
\end{scriptsize}

%------------------------------------------------------------------------------
%------------------------------------------------------------------------------
\subsection*{Continental collision} 
\index{topics}{Continental Collision}
%------------------------------------------------------------------------------
%------------------------------------------------------------------------------

\begin{scriptsize}
\begin{itemize}
\item[\nineteenseventyfive] 
\textcite{mota75} \citetitle{mota75}
\item[\nineteeneightytwo] 
\textcite{enmc82} \citetitle{enmc82} 
\item[\nineteeneightysix] 
\textcite{hoen86a} \citetitle{hoen86a}
\item[\nineteenninetyeight] 
\textcite{elbj98} \citetitle{elbj98} 
\textcite{bubr98} \citetitle{bubr98}
\item[\nineteenninetynine] 
\textcite{elbe99} \citetitle{elbe99} 
\textcite{will99b} \citetitle{will99b}
\item[\twothousand] 
\textcite{sobm00} \citetitle{sobm00}
\item[\twothousandthree] 
\textcite{refm03} \citetitle{refm03}
\item[\twothousandfive] 
\textcite{sobb05} \citetitle{sobb05}
\item[\twothousandnine] 
\textcite{sckb09} \citetitle{sckb09}
\item[\twothousandeleven] 
\textcite{lemk11} \citetitle{lemk11}
\item[\twothousandtwelve] 
\textcite{mavf12} \citetitle{mavf12}
\item[\twothousandthirteen] 
\textcite{scpo13} \citetitle{scpo13}
\item[\twothousandfourteen] 
\textcite{lesh14} \citetitle{lesh14}
\item[\twothousandfifteen] 
\textcite{puka15} \citetitle{puka15}
\item[\twothousandeighteen] 
\textcite{masg18} \citetitle{masg18} 
\textcite{gesr18} \citetitle{gesr18}
\item[\twothousandtwentyone] 
\textcite{scvg21} \citetitle{scvg21}
\end{itemize}
\end{scriptsize}

%------------------------------------------------------------------------------
%------------------------------------------------------------------------------
\subsection{Core complexes}
\index{topics}{Metamorphic Core Complex}
%------------------------------------------------------------------------------
%------------------------------------------------------------------------------

\begin{scriptsize}
\begin{itemize}
\item[\twothousandseven] 
\textcite{gewm07} \citetitle{gewm07}
\item[\twothousandeight] 
\textcite{tibb08} \citetitle{tibb08}
\item[\twothousandnine] 
\textcite{tigv09} \citetitle{tigv09} 
\textcite{retw09} \citetitle{retw09}
\item[\twothousandten] 
\textcite{olbt10} \citetitle{olbt10}
\item[\twothousandeleven] 
\textcite{retk11} \citetitle{retk11}
\item[\twothousandtwelve] 
\textcite{lehm12} \citetitle{lehm12} 
\textcite{scgb12} \citetitle{scgb12}
\item[\twothousandfifteen] 
\textcite{pebu15} \citetitle{pebu15}
\item[\twothousandseventeen] 
\textcite{esmp17} \citetitle{esmp17}
\item[\twothousandeighteen] 
\textcite{brst18} \citetitle{brst18}
\item[\twothousandnineteen] 
\textcite{biem19} \citetitle{biem19}
\end{itemize}
\end{scriptsize}


%------------------------------------------------------------------------------
%------------------------------------------------------------------------------
\subsection{GPS}
\index{topics}{GPS} 
%------------------------------------------------------------------------------
%------------------------------------------------------------------------------



%------------------------------------------------------------------------------
%------------------------------------------------------------------------------
\subsection{Tomography, deep Earth structure}
\index{topics}{Mantle Tomography}
%------------------------------------------------------------------------------
%------------------------------------------------------------------------------
\begin{scriptsize}
\begin{itemize}
\item[\nineteeneightyone] 
\textcite{dzan81} \citetitle{dzan81}\\
\item[\nineteenninetyone] 
\textcite{spak91} \citetitle{spak91}\\
\item[\nineteenninetythree] 
\textcite{kara93} \citetitle{kara93}\\
\item[\twothousandnine] 
\textcite{scbr09} \citetitle{scbr09}\\
\item[\twothousandthree] 
\textcite{pimo03} \citetitle{pimo03}\\
\item[\twothousandten] 
\textcite{sifb10} \citetitle{sifb10}\\ 
\item[\twothousandeleven]
\textcite{ridv11} \citetitle{ridv11}\\


\cite{fopa13} \cite{fopa13}\\ 

\item[\twothousandsixteen] 
\textcite{moek16} \citetitle{moek16}\\
\item[\twothousandeighteen] 
\textcite{homs18} \citetitle{homs18}\\
\end{itemize}
\end{scriptsize}

%------------------------------------------------------------------------------
%------------------------------------------------------------------------------
\subsection{Heat flow}
\index{topics}{Heat Flow}
%------------------------------------------------------------------------------
%------------------------------------------------------------------------------
\begin{scriptsize}
\begin{itemize}
\item[\nineteensixtyseven]
\textcite{mcke67} \citetitle{mcke67}
\item[\twothousandtwenty]
\textcite{moku20} \citetitle{moku20}
\end{itemize}
\end{scriptsize}

%------------------------------------------------------------------------------
%------------------------------------------------------------------------------
\subsection{Gravity, GRACE, GOCE}
\index{topics}{GRACE} 
\index{topics}{GOCE} 
\index{topics}{Gravity} 
%------------------------------------------------------------------------------
%------------------------------------------------------------------------------
\begin{scriptsize}
\begin{itemize}
\item[\nineteensixtyfour] 
\textcite{runc64} \citetitle{runc64}\\
\item[\nineteensixtyfive]
\textcite{morg65} \citetitle{morg65}\\
\item[\nineteensixtyseven] 
\textcite{mcke67} \citetitle{mcke67}\\
\item[\nineteeneightytwo] 
\textcite{clau82} \citetitle{clau82}\\
\item[\nineteeneightythree]
\textcite{kawa83} \citetitle{kawa83}\\
\item[\nineteeneightysix] 
\textcite{mequ86} \citetitle{mequ86}\\
\textcite{camq86} \citetitle{camq86}\\
\item[\nineteenninety]
\textcite{lips90} \citetitle{lips90}\\
\item[\twothousand] 
\textcite{zhmr00} \citetitle{zhmr00}\\
\item[\twothousandsix] 
\textcite{saad06} \citetitle{saad06}\\
\item[\twothousandeight]
\textcite{stdm08} \citetitle{stdm08}\\
\item[\twothousandten]
\textcite{katc10} \citetitle{katc10}\\
\item[\twothousandeleven]
\textcite{ruys11} \citetitle{ruys11}\\
\textcite{furu11} \citetitle{furu11}\\
\item[\twothousandfifteen]
\textcite{rotv15} \citetitle{rotv15}\\
\item[\twothousandeighteen] 
\textcite{zhmc18} \citetitle{zhmc18}\\
\textcite{ghmc18} \citetitle{ghmc18}\\
\item[\twothousandtwenty] 
\textcite{root20} \citetitle{root20}\\ 
\textcite{szes20} \citetitle{szes20}\\
\textcite{lerm20} \citetitle{lerm20}\\
\textcite{rovb20} \citetitle{rovb20}\\
\textcite{hasm20} \citetitle{hasm20}\\
\item[\twothousandtwentyone]
\textcite{fulm21} \citetitle{fulm21}\\
\end{itemize}
\end{scriptsize}

%----------------------------------------
HEAT FLUX
\begin{scriptsize}
\begin{itemize}
\item[\twothousandten] 
\textcite{dada10} \citetitle{dada10}
\end{itemize}
\end{scriptsize}


%%%%%%%%%%%%%%%%%%%%%%%%%%%%%%%%%%%%%%%%%%%%%%
 
\begin{scriptsize}
\nineteenseventyseven: Romanowicz \& Lambeck \cite{rola77}\\
\nineteenninetytwo: Gordon \& Stein \cite{gost92}\\
\nineteenninetyeight: Wahr \etal \cite{wamb98}\\
\nineteenninetynine: Ritsema \etal \cite{rivw99}, Smith \etal \cite{smst99}\\ 
\twothousand: Braitenberg \etal \cite{brzf00}\\
\twothousandone: Bunge \& Davies \cite{buda01}\\
\twothousandtwo: Becker \& Boschi \cite{bebo02}\\
\twothousandthree: Kreemer \etal \cite{krhh03}, Song \& Simons \cite{sosi03}, 
, Vermeersen \cite{verm03}\\
\twothousandfour: Tapley \etal \cite{tabr04}, Ritsema \etal \cite{rivw04}, Boschi \etal \cite{boek04},
                  Wahr \etal \cite{wasz04}\\
\twothousandfive: Chen \etal \cite{chrw05}, Trampert \& van der Hilst \cite{trva05}\\
\twothousandsix: Marotta \etal \cite{masr06}, Artemieva \cite{arte06}, 
                 Swenson \& Wahr \cite{swwa06}, Crosby \etal \cite{crms06}\\
\twothousandseven: Mickus \etal \cite{mitk07}, Loyd \etal \cite{lobc07}, Ritsema \etal \cite{rimb07}, 
                   Rangelova \etal \cite{ravb07}, Tamisiea \etal \cite{tamd07}, 
                   Heck \& Seitz \cite{hese07}\\
\twothousandeight: Zhou \cite{zhou08}, Romanowicz \cite{roma08}, 
                   Tesauro \etal \cite{tekc08}, van der Wal \cite{vaws08}, 
\twothousandtwelve: Hayes \etal \cite{hawj12}, Reguzzoni \& Sampietro \cite{resa12},
                    \cite{fesw12}\cite{simj12}\cite{beck12}\cite{pahk12}, 
                    Save \etal \cite{sabt12}, Sasgen \etal \cite{sakm12}, 
                    Mandea \etal \cite{mapl12}, Jacob \etal \cite{jawp12},
                    Hirt \etal \cite{hick12}\\
\twothousandthirteen: \cite{ress13}\cite{ebbf13}\cite{davi13}\cite{scle13}\cite{waja13}, 
\twothousandfourteen: \cite{paml14}\cite{ebbf14}\cite{krbk14}\cite{licl14}\cite{aubb14}, 
                      Cadio \& Korenaga \cite{cako14}, Schrama \etal \cite{scwr14}\\
\twothousandfifteen: Bouman \etal \cite{boem15}, Broerse \etal \cite{brrs15}, 
                     Fullea \etal \cite{furc15}, Pail \etal \cite{pabb15},
                     Mandea \etal \cite{manp15}, 
                     Fecher \etal \cite{fepg15}, van der Meijde \etal \cite{vapb15},
                     Reguzzoni \& Sampietro \cite{resa15}\\
\twothousandsixteen: Koelemeijer \etal \cite{kord16}
                     Root \etal \cite{rond16}, 
                     Dubey \& Tiwari \cite{duti16}, Colli \etal \cite{cogb16}\\
\twothousandseventeen: Root \etal \cite{roev17}\\
\twothousandeighteen: Panet \etal \cite{pabn18}, Hayes \etal \cite{hamp18}
                      Richards \etal \cite{rihc18}\\
\twothousandnineteen: Soler \etal \cite{sopg19}, Shulgin \& Artemieva \cite{shar19}, 
                      Afonso \etal \cite{afss19}, 
                      Saraswati \cite{sacm19}, Szwillus \etal \cite{szae19}\\
\end{scriptsize}



%------------------------------------------------------------------------------
%------------------------------------------------------------------------------
\subsection{Discontinuous Galerkin (DG)}
\index{topics}{Discontinuous Galerkin Method} 
\index{topics}{DG-FEM}
%------------------------------------------------------------------------------
%------------------------------------------------------------------------------

\begin{scriptsize}
\nineteenseventythree: \textcite{rehi73}\\
\nineteenninetyseven: \textcite{bare97}\\
\nineteenninetyeight: \textcite{cosh98}\\
\nineteenninetynine: \textcite{riwg99}\\
\twothousand: \textcite{coks00}\textcite{brmm00}\textcite{cacp00}\\
\twothousandtwo: \textcite{cacp02}\textcite{coks02}\textcite{arbc02}\textcite{gurw02}\\
\twothousandthree: \textcite{cock03}\\
\twothousandfour: \textcite{coks04}\\
\twothousandfive: \textcite{cacs05}\textcite{coks05}\textcite{cogo05a}\textcite{cogo05b}\textcite{cogo05c}\\
\twothousandseven: \textcite{coks07}\textcite{feku07}\\
\twothousandeight: \textcite{kans08}\textcite{mofh08}\textcite{dole08}\textcite{pepe08}\\
\twothousandnine: \textcite{coks09}\textcite{cogo09}\textcite{cogl09}\textcite{ngpc09}\textcite{shu09}\textcite{codg08}\textcite{cogw09}\\
\twothousandten: \textcite{ngpc10}\textcite{conp10}\textcite{mofp10}\textcite{kari10}\textcite{cogs10}\\
\twothousandeleven: \textcite{geor11}\textcite{ngpc11}\\
\twothousandtwelve: \textcite{kauf12}\textcite{ngpe12}\textcite[chapt. 31]{lomw12}\\
\twothousandthirteen: \textcite{vyrc13}\textcite{rhcv13}, \textcite{klwh13}\\
\twothousandfifteen: \textcite{lelk15}\textcite{kalc15}\\
\twothousandsixteen: \textcite{cock16}\textcite{makc16}\\
\twothousandseventeen: \textcite{fewk17}\textcite{iglo17},
                       \textcite{hepb17}\textcite{chll17},
                       \textcite{sclu17a}\textcite{sclu17b}
                       \textcite{sclu17c}\textcite{zhan17}\\
\twothousandeighteen: \textcite{puth18}\textcite{wogu18}\textcite{fakr18}\textcite{muwy18}
\end{scriptsize}

%------------------------------------------------------------------------------
%------------------------------------------------------------------------------
\subsection{Dynamo}
\index{topics}{Dynamo}
%------------------------------------------------------------------------------
%------------------------------------------------------------------------------

\begin{scriptsize}
\begin{itemize}
\item[\twothousandfive] Harder \& Hansen \cite{haha05}
\item[\twothousandnine] Roberts \etal \cite{rolm09}
\item[\twothousandeleven] Jones \cite{jone11}
\item[\twothousandthirteen] Ernst-Hullermann \etal \cite{erhh13}, van Summeren \etal \cite{vagc13}
\item[\twothousandsixteen] Choblet \etal \cite{chah16}
\end{itemize}
\end{scriptsize}

%------------------------------------------------------------------------------
%------------------------------------------------------------------------------
\subsection{(role of) Elasticity in geodynamics modelling}
\index{topics}{Elasticity in Geodynamics}
%------------------------------------------------------------------------------
%------------------------------------------------------------------------------

\begin{scriptsize}
\begin{itemize}
\item[\nineteenseventy] 
\textcite{walc70} \citetitle{walc70} \\
\item[\nineteenseventyseven]
\textcite{debr77} \citetitle{debr77}\\
\item[\nineteeneightyfour]
\textcite{yusa84} \citetitle{yusa84}\\
\item[\nineteeneightysix] 
\textcite{sayp86} \citetitle{sayp86}\\
\item[\nineteeneightyseven] 
\textcite{brbe87} \citetitle{brbe87}\\
\item[\nineteenninetyfive]
\textcite{budi95} \citetitle{budi95}\\
\textcite{hamy95} \citetitle{hamy95}\\
\item[\nineteenninetysix] 
\textcite{hach96b} \citetitle{hach96b}\\
\textcite{chri96b} \citetitle{chri96b}\\
\textcite{mitr96} \citetitle{mitr96}\\
\item[\nineteenninetyseven] 
\textcite{hajc97} \citetitle{hajc97}\\
\item[\nineteenninetyeight] 
\textcite{copo98} \citetitle{copo98}\\
\textcite{reyu98} \citetitle{reyu98}\\
\item[\twothousandone] 
\textcite{vapy01} \citetitle{vapy01}\\
\textcite{modm01} \citetitle{modm01}\\
\item[\twothousandtwo]
\textcite{mumh02} \citetitle{mumh02}\\
\textcite{modm02} \citetitle{modm02}\\
\item[\twothousandthree] 
\textcite{hukm03}\citetitle{hukm03}\\
\textcite{wabu03}\citetitle{wabu03}\\
\item[\twothousandfive] 
\textcite{mure05}\citetitle{mure05}\\
\item[\twothousandsix] 
\textcite{kapo06}\citetitle{kapo06}\\
\textcite{mudm06}\citetitle{mudm06}\\
\item[\twothousandseven] 
\textcite{kabe07}\citetitle{kabe07}\\
\item[\twothousandeight] 
\textcite{baso08}\citetitle{baso08}\\
\textcite{fukk08}\citetitle{fukk08}\\
\textcite{thpo08}\citetitle{thpo08}\\
\item[\twothousandnine] 
\textcite{qurj09}\citetitle{qurj09}\\
\item[\twothousandten] 
\textcite{bepo10}\citetitle{bepo10}\\
\item[\twothousandtwelve] 
\textcite{gerb12}\citetitle{gerb12}\\
\textcite{kasc12}\citetitle{kasc12}\\
\item[\twothousandthirteen] 
\textcite{wahd13}\citetitle{wahd13}\\
\item[\twothousandfourteen] 
\textcite{famc14}\citetitle{famc14}\\
\textcite{fogm14}\citetitle{fogm14}\\
\textcite{olbe14}\citetitle{olbe14}\\
\textcite{hepk14}\citetitle{hepk14}\\
\item[\twothousandfifteen] 
\textcite{thkp15}\citetitle{thkp15}\\
\item[\twothousandsixteen] 
\textcite{bafl16}\citetitle{bafl16}\\
\textcite{jads16}\citetitle{jads16}\\
\textcite{olbm16}\citetitle{olbm16}\\
\item[\twothousandseventeen] 
\textcite{pact17} \citetitle{pact17}\\
\item[\twothousandeighteen] 
\textcite{dusd18} \citetitle{dusd18}\\
\textcite{mosp18} \citetitle{mosp18}\\
\item[\twothousandnineteen] 
\textcite{pact19} \citetitle{pact19}\\
\item[\twothousandtwenty] 
\textcite{sams20} \citetitle{sams20}\\
\textcite{lahh20} \citetitle{lahh20}\\
\end{itemize}
\end{scriptsize}

%------------------------------------------------------------------------------
%------------------------------------------------------------------------------
\subsection{(Geodynamics+) surface processes, erosion, sedimentation, topography evolution}
\index{topics}{Surface Processes}
\index{topics}{Erosion} 
\index{topics}{Sedimentation}
\index{topics}{Topography Evolution}
\index{topics}{Landscape Evolution}
%------------------------------------------------------------------------------
%------------------------------------------------------------------------------

\begin{scriptsize}
\begin{itemize}
\item [1953]                
\textcite{lema53} \citetitle{lema53}\\
\item [\nineteensixty]      
\textcite{cull60} \citetitle{cull60}\\
\item [\nineteenninety] 
\textcite{moen90} \citetitle{moen90}\\
\textcite{enmo90} \citetitle{enmo90}\\
\item [\nineteenninetytwo] 
\textcite{befh92} \citetitle{befh92}\\
\textcite{chas92} \citetitle{chas92}\\
\item [\nineteenninetythree] 
\textcite{povp93} \citetitle{povp93}\\
\textcite{wibf93} \citetitle{wibf93}\\
\item [\nineteenninetyfour] 
\textcite{howa94} \citetitle{howa94}\\ 
\textcite{koon94} \citetitle{koon94}\\ 
\textcite{kobe94} \citetitle{kobe94}\\
\textcite{gikb94} \citetitle{gikb94}\\ 
\textcite{whme04} \citetitle{whme04}\\
\item [\nineteenninetyfive] 
\textcite{chmm95} \citetitle{chmm95}\\
\textcite{koon95} \citetitle{koon95}\\
\item [\nineteenninetysix] 
\textcite{avbu96} \citetitle{avbu96}\\
\textcite{bekh96} \citetitle{bekh96}\\
\textcite{kobe96} \citetitle{kobe96}\\
\textcite{whme06} \citetitle{whme06}\\
\item [\nineteenninetyseven] 
\textcite{brsa97} \citetitle{brsa97}\\
\textcite{gaft97} \citetitle{gaft97}\\
\textcite{babr97} \citetitle{babr97}\\
\item [\nineteenninetyeight] 
\textcite{deea98} \citetitle{deea98}\\
\textcite{vabr98} \citetitle{vabr98}\\
\item [\nineteenninetynine] 
\textcite{will99a} \citetitle{will99a}\\
\textcite{bupi99} \citetitle{bupi99}\\
\textcite{babr99} \citetitle{babr99}\\
\textcite{tobr99} \citetitle{tobr99}\\
\item [\twothousandone] 
\textcite{zemk01} \citetitle{zemk01}\\
\textcite{tulg01} \citetitle{tulg01}\\
\textcite{brsh01} \citetitle{brsh01}\\
\textcite{bupo01} \citetitle{bupo01}\\
\textcite{coul01} \citetitle{coul01}\\
\textcite{crda01} \citetitle{crda01}\\
\textcite{moln01} \citetitle{moln01}\\
\item [\twothousandtwo] 
\textcite{wibr02} \citetitle{wibr02}\\ 
\textcite{mobr02} \citetitle{mobr02}\\
\textcite{garc02} \citetitle{garc02}\\ 
\textcite{whtu02} \citetitle{whtu02}\\ 
\textcite{tuwh02} \citetitle{tuwh02}\\
\item [\twothousandthree] 
\textcite{brau03} \citetitle{brau03}\\
\item [\twothousandfour] 
\textcite{fijj04} \citetitle{fijj04}\\
\textcite{gocl04} \citetitle{gocl04}\\
\textcite{simp04} \citetitle{simp04}\\ 
\textcite{skdi04} \citetitle{skdi04}\\
\item [\twothousandfive] 
\textcite{lave05} \citetitle{lave05}\\
\textcite{will05} \citetitle{will05}\\
\textcite{lahd05} \citetitle{lahd05}\\
\item [\twothousandsix] 
\textcite{rosw06} \citetitle{rosw06}\\ 
\textcite{brau06} \citetitle{brau06}\\
\textcite{bocr06} \citetitle{bocr06}\\ 
\textcite{simp06} \citetitle{simp06}\\
\textcite{stwr06} \citetitle{stwr06}\\ 
\textcite{golc06} \citetitle{golc06}\\
\item [\twothousandseven] 
\textcite{buto07} \citetitle{buto07}\\ 
\textcite{sebp07} \citetitle{sebp07}\\
\textcite{tomk07} \citetitle{tomk07}\\ 
\textcite{strw07} \citetitle{strw07}\\
\item [\twothousandeight] 
\textcite{alle08} \citetitle{alle08}\\ 
\textcite{rowf08} \citetitle{rowf08}\\
\item [\twothousandnine]  
\textcite{whip09} \citetitle{whip09}\\ 
\textcite{kuhe09} \citetitle{kuhe09}\\
\textcite{makh09} \citetitle{makh09}\\ 
\textcite{pina09} \citetitle{pina09}\\
\textcite{dala09} \citetitle{dala09}\\ 
\textcite{bonn09} \citetitle{bonn09}\\
\item [\twothousandten] 
\textcite{will10} \citetitle{will10}\\ 
\textcite{tuha10} \citetitle{tuha10}\\
\textcite{brau10} \citetitle{brau10}\\ 
\textcite{brya10} \citetitle{brya10}\\
\textcite{crmw10} \citetitle{crmw10}\\
\item [\twothousandeleven] 
\textcite{robr11} \citetitle{robr11}\\
\textcite{grhd11} \citetitle{grhd11}\\
\item [\twothousandtwelve]  
\textcite{kiwh12} \citetitle{kiwh12}\\
\textcite{brvv12} \citetitle{brvv12}\\
\item [\twothousandthirteen] 
\textcite{vehc13} \citetitle{vehc13}\\ 
\textcite{brwi13} \citetitle{brwi13}\\
\textcite{fihv13a}\citetitle{fihv13a}\\
\textcite{fihv13b}\citetitle{fihv13b}\\
\textcite{brrs13} \citetitle{brrs13}\\ 
\textcite{chgz13} \citetitle{chgz13}\\
\textcite{tuva13} \citetitle{tuva13}\\ 
\textcite{caya13} \citetitle{caya13}\\
\item [\twothousandfourteen] 
\textcite{mehn14} \citetitle{mehn14}\\ 
\textcite{crbr14} \citetitle{crbr14}\\
\textcite{cokm14} \citetitle{cokm14}\\ 
\textcite{erhv14} \citetitle{erhv14}\\
\textcite{stsc14} \citetitle{stsc14}\\ 
\textcite{olbm14} \citetitle{olbm14}\\
\item [\twothousandfifteen]  
\textcite{uewg15} \citetitle{uewg15}\\
\textcite{fohk15} \citetitle{fohk15}\\
\textcite{cofk15} \citetitle{cofk15}\\
\textcite{erhv15} \citetitle{erhv15}\\
\item [\twothousandsixteen]  
\textcite{coyc16} \citetitle{coyc16}\\
\textcite{schr16} \citetitle{schr16}\\
\item [\twothousandeighteen] 
\textcite{jolp18} \citetitle{jolp18}\\
\item [\twothousandnineteen] 
\textcite{anpa19} \citetitle{anpa19}\\
\textcite{sall19} \citetitle{sall19}\\
\item [\twothousandtwenty]  
\textcite{ster20} \citetitle{ster20}\\
\textcite{diho20} \citetitle{diho20}\\
\textcite{fabe20} \citetitle{fabe20}\\
\textcite{behu20} \citetitle{behu20}\\
\textcite{grco20} \citetitle{grco20}\\
\textcite{stmj21} \citetitle{stmj21}\\
\end{itemize}
\end{scriptsize}

%------------------------------------------------------------------------------
%------------------------------------------------------------------------------
\subsection{Geotechnics}
\index{topics}{Geotechnics}
%------------------------------------------------------------------------------
%------------------------------------------------------------------------------

\begin{scriptsize}
\nineteenninetynine: \textcite{ster99}\\
\twothousandthree: \textcite{gora03}\textcite{zhll03}\\
\twothousandfour: \textcite{gour04}\\
\twothousandsix: \textcite{gork06}\\
\twothousandfourteen: \textcite{bufy14}
\end{scriptsize}

%------------------------------------------------------------------------------
%------------------------------------------------------------------------------
\subsection{Glacier dynamics, ice sheets, ice flow, ice rheology}
\index{topics}{Glacier Dynamics} 
\index{topics}{Ice Sheets} 
\index{topics}{Ice flow} 
\index{topics}{Ice Rheology}
%------------------------------------------------------------------------------
%------------------------------------------------------------------------------

\begin{scriptsize}
\begin{itemize}
\item[\nineteeneightynine] Budd \& Jacka \cite{buja89}
\item[\nineteenninety] van der Ween \& Whillans \cite{vawh90}
\item[\nineteenninetyfour] Wilson \& Zhang \cite{wizh94}
\item[\nineteenninetyseven] Greve \cite{grev97}
\item[\twothousandone] Goldsby \& Kohlstedt \cite{goko01}
\item[\twothousandfour] Freeman \etal \cite{frmm04}
\item[\twothousandsix] \cite{asbl06}, Freeman \etal \cite{frmm06}
\item[\twothousandseven] Sulsky \etal \cite{susp07}, Zwinger \etal \cite{zwgg07}
\item[\twothousandeleven] Zhang \etal \cite{zhjg11}
\item[\twothousandtwelve] Pollard \& DeConto \cite{pode12}
\item[\twothousandthirteen] Rasmussen \etal \cite{raab13}
\item[\twothousandfourteen] Leng \etal \cite{lejx14}, Montagnat \etal \cite{moad14}
\item[\twothousandfifteen] Isaac \etal \cite{issg15}, Frehner \etal \cite{frlg15}
\item[\twothousandsixteen] Krabbendam \cite{krab16}, Dansereau \etal \cite{daws16}
\item[\twothousandseventeen] Logan \etal \cite{lolc17}, Goelzer \etal \cite{gors17}
\item[\twothousandeighteen] Helanow \& Ahlkrona \cite{heah18}, Minchew \etal \cite{mimr18}
\item[\twothousandnineteen] Kuiper \etal \cite{kudd19,kuwd19}, Kuiper PhD thesis \cite{kuiper19}
\end{itemize}
\end{scriptsize}


also ian\_hewitt\_karthaus\_rheology.pdf

%------------------------------------------------------------------------------
%------------------------------------------------------------------------------
\subsection{(use of) Inverse methods, inversion, adjoint methods, assimilation}
\index{topics}{Inverse Methods} 
\index{topics}{Adjoint Methods} 
\index{topics}{Data Assimilation}
%------------------------------------------------------------------------------
%------------------------------------------------------------------------------

What Is an Adjoint Model? \cite{erri97}

\begin{scriptsize}
\begin{itemize}
\item[\nineteenninetysix] 
\textcite{fomi96}  \citetitle{fomi96}\\ 
\item[\nineteenninetyeight] 
\textcite{cava98}  \citetitle{cava98}\\
\item[\nineteenninetynine] 
\textcite{samb99}  \citetitle{samb99} \\
\textcite{samb99b} \citetitle{samb99b}\\
\item[\twothousandone] 
\textcite{bomo01} \citetitle{bomo01}\\ 
\textcite{kapo01} \citetitle{kapo01}\\ 
\textcite{kasc01} \citetitle{kasc01}\\
\item[\twothousandtwo] 
\textcite{shri02} \citetitle{shri02}\\
\textcite{burb02} \citetitle{burb02}\\
\item[\twothousandthree] 
\textcite{buht03} \citetitle{buht03}\\
\item[\twothousandfour] 
\textcite{isst04} \citetitle{isst04}\\ 
\textcite{mifo04} \citetitle{mifo04}\\
\item[\twothousandsix] 
\textcite{sifg06} \citetitle{sifg06}\\
\item[\twothousandseven] 
\textcite{isks07} \citetitle{isks07}\\
\item[\twothousandeight] 
\textcite{splg08} \citetitle{splg08}\\
\textcite{ligu08} \citetitle{ligu08}\\
\item[\twothousandnine] 
\textcite{wama09} \citetitle{wama09}\\
\textcite{splg09} \citetitle{splg09}\\
\textcite{sifg09} \citetitle{sifg09}\\
\item[\twothousandtwelve] 
\textcite{naco12} \citetitle{naco12}\\
\item[\twothousandfourteen] 
\textcite{wosp14} \citetitle{wosp14}\\
\textcite{hobo14} \citetitle{hobo14}\\
\textcite{licl14} \citetitle{licl14}\\ 
\textcite{bakp14} \citetitle{bakp14}\\
\textcite{glfo14} \citetitle{glfo14}\\ 
\textcite{gran14} \citetitle{gran14}\\
\item[\twothousandfifteen] 
\textcite{wahg15} \citetitle{wahg15}\\
\textcite{cobs15} \citetitle{cobs15}\\
\textcite{vybu15} \citetitle{vybu15}\\
\textcite{sobd15} \citetitle{sobd15}\\
\textcite{rasg15} \citetitle{rasg15}\\
\item[\twothousandsixteen] 
\textcite{ghbu16} \citetitle{ghbu16}\\ 
\textcite{bocf16} \citetitle{bocf16}\\
\textcite{yagu16} \citetitle{yagu16}\\
\textcite{baum16} \citetitle{baum16}\\
\textcite{pric16} \citetitle{pric16}\\
\item[\twothousandseventeen] 
\textcite{ligs17} \citetitle{ligs17}\\
\textcite{zhli17} \citetitle{zhli17}\\
\item[\twothousandeighteen] 
\textcite{bofc18} \citetitle{bofc18}\\
\textcite{ghbu18} \citetitle{ghbu18}\\
\textcite{cogb18} \citetitle{cogb18}\\
\textcite{ghmc18} \citetitle{ghmc18}\\
\textcite{prda18} \citetitle{prda18}\\
\textcite{repk18} \citetitle{repk18}\\
\textcite{fupc18} \citetitle{fupc18}\\
\textcite{shyp18} \citetitle{shyp18}\\
\item[\twothousandnineteen]
\textcite{mamr19} \citetitle{mamr19}\\ 
\item[\twothousandtwenty] 
\textcite{rehp20} \citetitle{rehp20}\\ 
\textcite{lufs20} \citetitle{lufs20}\\
\textcite{ruml20} \citetitle{ruml20}\\
\textcite{resi20} \citetitle{resi20}\\
\textcite{orza20} \citetitle{orza20}\\
\textcite{moku20} \citetitle{moku20}\\
\item[\twothousandtwentyone] 
\textcite{mabh21} \citetitle{mabh21}\\
\textcite{reub21} \citetitle{reub21}\\
\end{itemize}
\end{scriptsize}

%------------------------------------------------------------------------------
%------------------------------------------------------------------------------
\subsection{Large scale mantle-plate interaction, whole Earth models}
%------------------------------------------------------------------------------
%------------------------------------------------------------------------------

\begin{scriptsize}
\nineteeneightyfive: Yuen \& Fleitout \cite{yufl85}\\
\nineteenninetythree: Lowman \& Jarvis \cite{loja93}\\
\nineteenninetyfive: Lowman \& Jarvis \cite{loja95}\\
\nineteenninetysix: Lowman \& Jarvis \cite{loja96}\\
\nineteenninetyeight: Pysklywec \& Mitrovica \cite{pymi98}\\
\nineteenninetynine: Lowman \& Jarvis \cite{loja99}\\
\twothousand: Goes \etal \cite{golw00}
\twothousandone: Lowman \etal \cite{lokg01} \\
\twothousandthree: Lowman \etal \cite{lokg03} \\
\twothousandfour: Lowman \etal \cite{lokg04} \\
\twothousandsix: Conrad \& Lithgow-Bertelloni \cite{coli06}\\
\twothousandeight: Goes \etal \cite{gocm08}, Takaku \& Fukao \cite{tafu08}\\
\twothousandten: \cite{wamg10}\cite{stgb10}\cite{cobe10}\\
\twothousandeleven: Lowman \etal \cite{lokt11}\\
\twothousandtwelve: \cite{algs12}\cite{roct12}\cite{crtm12}\\
\twothousandthirteen: \cite{ghbh13}\cite{yahb13}\\
\twothousandsixteen: \cite{macs16}\\
\twothousandeighteen: \cite{hulz18}\cite{osss18b}\\
\twothousandnineteen: Flament \cite{flam19}
\end{scriptsize}

%------------------------------------------------------------------------------
%------------------------------------------------------------------------------
\subsection{Crust/Lithosphere modelling, plate motion, plate stress}
%------------------------------------------------------------------------------
%------------------------------------------------------------------------------
\index{topics}{Plate motion modelling}

\begin{scriptsize}
\begin{itemize}
\item[1914]
\textcite{barr14} \citetitle{barr14}\\
\item[\nineteenseventy]
\textcite{walc70} \citetitle{walc70}\\
\item[\nineteenseventyseven] 
\textcite{crou77} \citetitle{crou77}\\
\item[\nineteeneightyone]
\textcite{brpo81} \citetitle{brpo81}\\
\item[\nineteeneightythree]
\textcite{mcja83} \citetitle{mcja83}\\
\item[\nineteeneightyfour]
\textcite{kupa84} \citetitle{kupa84}\\
\textcite{riff84} \citetitle{riff84}\\
\item[\nineteeneightysix]
\textcite{stbb86} \citetitle{stbb86}\\
\item[\nineteeneightyeight] 
\textcite{daco88} \citetitle{daco88}\\
\textcite{coda88} \citetitle{coda88}\\
\item[\nineteeneightynine]
\textcite{jabe89} \citetitle{jabe89}\\
\item[\nineteenninety]
\textcite{chmo90} \citetitle{chmo90}\\
\item[\nineteenninetyone]
\textcite{chbv91} \citetitle{chbv91}\\
\textcite{daco91} \citetitle{daco91}\\
\item[\nineteenninetytwo]
\textcite{moln92} \citetitle{moln92}\\
\textcite{budi92} \citetitle{budi92}\\
\textcite{kigw92} \citetitle{kigw92}\\
\item[\nineteenninetythree]
\textcite{nefo93} \citetitle{nefo93}\\
\textcite{brau93} \citetitle{brau93}\\
\textcite{grma93} \citetitle{grma93}\\
\textcite{berc93} \citetitle{berc93}\\
\item[\nineteenninetyfour] 
\textcite{buso94} \citetitle{buso94}\\
\textcite{befh94} \citetitle{befh94}\\
\item[\nineteenninetyfive] 
\textcite{belg95} \citetitle{belg95}\\
\textcite{brbe95} \citetitle{brbe95}\\
\textcite{kian95} \citetitle{kian95}\\
\textcite{budi95} \citetitle{budi95}\\
\textcite{elfb95} \citetitle{elfb95}\\
\textcite{zhgu95b} \citetitle{zhgu95b}\\
\item[\nineteenninetysix] 
\cite{bekh96}
\cite{berc96}
\cite{jabh96}\\
\item[\nineteenninetyseven] 
\cite{thsj97}
\cite{babr97}
\cite{bucl97}
\cite{mole97}\\
\item[\nineteenninetyeight] 
\cite{bird98}
\cite{lecd98}
\cite{kian98}
\cite{mafs98}
\cite{madu98}
\cite{gumm98}
\cite{berc98}
\cite{madu98}\\
\item[\nineteenninetynine] 
\cite{will99b}
\cite{bird99}
\cite{clbp99}
\cite{fugo99}
\cite{mole99}
\cite{lemo99}
\cite{gebp99}\\
\item[\twothousand] 
\cite{hanl00}
\cite{labp00}
\cite{lemm00}
\cite{gumm00}
\cite{lemo00}
\cite{pepo00}
\cite{scys00b}\\
\item[\twothousandone] 
\cite{homo01}
\cite{beoc01}
\cite{kapo01}\\
\item[\twothousandtwo] 
\cite{labu02}
\cite{coli02}
\cite{bast02}
\cite{gedh02}
\cite{kilg02}\\
\item[\twothousandthree] 
\cite{wipo03}
\cite{wabu03}
\cite{geur03}
\cite{upke03}
\cite{vamf03}
\cite{bupf03}
\cite{lemm03}
\cite{onmo03}\\
\item[\twothousandfour] 
\cite{tibb04}
\cite{gewi04}
\cite{colm04}
\textcite{coli04} \citetitle{coli04}\\
\textcite{pybe04} \citetitle{pybe04}\\
\item[\twothousandfive] 
\cite{vazs05}
\cite{hagu05}
\cite{wiwg05}
\cite{mcjp05}\\
\item[\twothousandsix] 
\cite{bube06}
\cite{basv06}
\cite{kasc06}
\cite{fuwb06}
\cite{colm06}
\cite{pabs06}
\cite{crnp06} 
\cite{sahm06}\\
\item[\twothousandseven] 
\cite{afrf07}
\cite{kore07}
\cite{gewm07}
\cite{jabn07}\\
\item[\twothousandeight] 
\cite{affr08}
\cite{tibb08}
\cite{hapo08}
\cite{busc08}
\cite{clbz08}
\cite{chlg08}
\cite{kasb08}
\cite{fabs08}
\cite{chgu08}
\cite{buit08}
\cite{onlg08}\\
\item[\twothousandnine]
\cite{bupb09}
\cite{plmg09}
\cite{rigo09}
\cite{bubg09}
\cite{coco09}\\
\item[\twothousandten]
\textcite{hamo10} \citetitle{hamo10}\\
\textcite{fasm10} \citetitle{fasm10}\\
\textcite{grpy10} \citetitle{grpy10}\\
\textcite{vago10} \citetitle{vago10}\\
\textcite{plmf10} \citetitle{plmf10}\\
\textcite{spgs10a} \citetitle{spgs10a}\\
\textcite{pygp10} \citetitle{pygp10}\\
\textcite{jabw10} \citetitle{jabw10}\\
\item[\twothousandeleven]
\textcite{rera11} \citetitle{rera11}\\
\textcite{chss11} \citetitle{chss11}\\
\item[\twothousandtwelve]
\textcite{wagw12} \citetitle{wagw12}\\
\textcite{vacl12} \citetitle{vacl12}\\
\textcite{buit12} \citetitle{buit12}\\
\textcite{kogp12} \citetitle{kogp12}\\
\textcite{gohg12} \citetitle{gohg12}\\
\textcite{trub12} \citetitle{trub12}\\
\item[\twothousandthirteen]
\textcite{wazh13} \citetitle{wazh13}\\
\textcite{krcu13} \citetitle{krcu13}\\
\textcite{frbm13} \citetitle{frbm13}\\
\textcite{wagw13} \citetitle{wagw13}\\
\textcite{duyp13} \citetitle{duyp13}\\
\textcite{rugb13} \citetitle{rugb13}\\
\textcite{scdg13} \citetitle{scdg13}\\
\item[\twothousandfourteen]
\textcite{kava14} \citetitle{kava14}\\
\textcite{dusp14} \citetitle{dusp14}\\
\textcite{wavp14} \citetitle{wavp14}\\
\textcite{whbb14} \citetitle{whbb14}\\
\textcite{scml14} \citetitle{scml14}\\
\textcite{mals14} \citetitle{mals14}\\
\textcite{gupm14} \citetitle{gupm14}\\
\textcite{gahs14} \citetitle{gahs14}\\
\textcite{mutg14} \citetitle{mutg14}\\
\item[\twothousandfifteen] 
\textcite{wavp15} \citetitle{wavp15}\\
\textcite{thkp15} \citetitle{thkp15}\\
\textcite{mags15} \citetitle{mags15}\\
\textcite{duys15} \citetitle{duys15}\\
\textcite{dusp15} \citetitle{dusp15}\\
\item[\twothousandsixteen] 
\textcite{wahz16} \citetitle{wahz16}\\
\textcite{heps16} \citetitle{heps16}\\
\item[\twothousandseventeen] 
\textcite{rugb17} \citetitle{rugb17}\\  
\textcite{ozgw17} \citetitle{ozgw17}\\
\textcite{vomc17} \citetitle{vomc17}\\  
\textcite{taac17} \citetitle{taac17}\\
\textcite{ithc17} \citetitle{ithc17}\\  
\textcite{liwg17} \citetitle{liwg17}\\
\item[\twothousandeighteen]
\textcite{wavp18} 
\textcite{nigw18} 
\textcite{bemc18}  
\textcite{neew18} 
\textcite{stbe18}\\
\item[\twothousandnineteen] 
\textcite{koen19}
\textcite{kipd19}
\textcite{crcm19}
\textcite{pedm19}
\textcite{mazz19}
\textcite{chch19}\\
\item[\twothousandtwenty] 
\textcite{yamq20} 
\textcite{miko20}
\end{itemize}
\end{scriptsize}


%------------------------------------------------------------------------------
%------------------------------------------------------------------------------
\subsection{Delamination, edge driven convection, gravitational instability, mantle unrooting, lithosphere thinning, small scale convection} 
\index{topics}{Delamination} 
\index{topics}{Edge Driven Convection}
\index{topics}{Mantle Unrooting}
\index{topics}{Lithosphere Thinning}
%------------------------------------------------------------------------------
%------------------------------------------------------------------------------

\begin{scriptsize}
\begin{itemize}
\item[\nineteenseventynine] Bird \cite{bird79}
\item[\nineteeneightyone] Houseman \etal \cite{homm81}
\item[\nineteeneightyfive] Yuen \& Fleitout \etal \cite{yufl85}
\item[\nineteeneightysix] Fleitout \etal \cite{flfy86}
\item[\nineteenninetythree] Kay \& Kay \cite{kaka93} 
\item[\nineteenninetyfive] King \& Anderson \cite{kian95}
\item[\nineteenninetyseven] Houseman \& Molnar \cite{homo97}
\item[\nineteenninetyeight] King \& Anderson \cite{kian98}, Schott \& Schmeling \cite{scsc98}, 
                            Marotta \& al \cite{mafs98}, Meissner \& Mooney \cite{memo98}
\item[\nineteenninetynine] Schott \etal \cite{scys99} 
\item[\twothousand] King \& Ritsema \cite{kiri00}, Schott \etal \cite{scys00}, 
                    Houseman \etal \cite{honk00}
\item[\twothousandone] Jull \& Kelemen \cite{juke01}
\item[\twothousandthree] Korenaga \& Jordan \cite{kojo03} 
\item[\twothousandfour] Morency \& Doin \cite{modo04}
\item[\twothousandsix] Le Pourhiet \etal \cite{legs06}
\item[\twothousandseven] Elkins-Tanton \cite{elki07}
\item[\twothousandeight] Valera \etal \cite{vanv08}, Gogus \& Pyslklywec \cite{gopy08}, 
                         van Wijk \etal \cite{vavg08}
\item[\twothousandten] van Wijk \etal \cite{vabv10}
\item[\twothousandeleven] Levander \etal \cite{lesm11}, Valera \etal \cite{vanj11}
\item[\twothousandtwelve] Bajolet \etal \cite{bagf12}
\item[\twothousandthirteen] Krystopowicz \& Currie \cite{krcu13}, Stern \etal \cite{sths13}
\item[\twothousandfourteen] Bao \etal \cite{baeg14}, Kaislaniemi \& van Hunen \cite{kava14}
\item[\twothousandfifteen] Wang \etal \cite{wahz15,wavp15}
\item[\twothousandsixteen] Dalaison \& Davies \cite{dada16}
\item[\twothousandseventeen] Beall \etal \cite{bems17}
\item[\twothousandeighteen] Perry-Houts \& Karlstrom \cite{peka18}
\item[\twothousandnineteen] Lei \etal \cite{lell19}
\item[\twothousandtwenty] Magni \& Kirali \cite{maki20}
\item[\twothousandtwentyone] Qi \etal \cite{qizx21}, Comeau \etal \cite{cosb21}
\end{itemize}
\end{scriptsize}


%------------------------------------------------------------------------------
%------------------------------------------------------------------------------
\subsection{Detachment faults} 
\index{topics}{Detachment faults}
%------------------------------------------------------------------------------
%------------------------------------------------------------------------------

\begin{scriptsize}
\begin{itemize}
\item[\twothousandseven]     
\textcite{werr07} \citetitle{werr07}
\item[\twothousandten]       
\textcite{jaml10} \citetitle{jaml10}
\item[\twothousandeleven]    
\textcite{rera11} \citetitle{rera11}
\item[\twothousandfifteen]   
\textcite{matv15} \citetitle{matv15}
\item[\twothousandnineteen]  
\textcite{gubg19} \citetitle{gubg19}
\item[\twothousandtwentyone] 
\textcite{sabg21} \citetitle{sabg21}
\end{itemize}
\end{scriptsize}

%------------------------------------------------------------------------------
%------------------------------------------------------------------------------
\subsection{Dynamic topography} 
\index{topics}{Dynamic Topography}
%------------------------------------------------------------------------------
%------------------------------------------------------------------------------

\begin{scriptsize}
\begin{itemize}
\item[\nineteeneightyfive] Hager \etal \cite{hacr85}
\item[\nineteeneightyseven] Revenaugh \& Parsons \cite{repa87}
\item[\nineteenninetytwo] Kiefer \& Hager \cite{kiha92}
\item[\nineteenninetythree] Gurnis \cite{gurn93,gurn93b}
\item[\nineteenninetyseven] Pysklywec \& Mitrovica \cite{pymi97}
\item[\nineteenninetynine] Burgess \& Moresi \cite{bumo99}
\item[\twothousandthree] Conrad \& Gurnis \cite{cogu03}
\item[\twothousandnine] Conrad \& Husson \cite{cohu09}
\item[\twothousandten] Boschi \etal \cite{bofb10}, Braun \cite{brau10},
                       Stein \etal \cite{stfh10}, Shephard \etal \cite{shml10}
\item[\twothousandeleven] Ramsay \& Pysklywec \cite{rapy11}
\item[\twothousandtwelve] Shephard \etal \cite{shlm12}, Zhang \etal \cite{zhzf12}
\item[\twothousandthirteen] Braun \etal \cite{brrs13}, Flament \etal \cite{flgm13}
\item[\twothousandfifteen] Austermann \etal \cite{aupm15}, Kiraly \etal \cite{kiff15},
                           Davila \& Lithgow-Bertelloni \cite{dali15}
\item[\twothousandsixteen] Hoggard \etal \cite{howa16}, Gvirtzman \etal \cite{gvfb16},
                           Yang \& Gurnis \cite{yagu16}, Steinberger \cite{stei16},
                           Colli \etal \cite{cogb16}
\item[\twothousandseventeen] Yang \etal \cite{yamm17}, Austermann \etal \cite{aumh17},
                             Greff-Lefftz \etal \cite{grrb17}
\item[\twothousandeighteen] Osei Tutu \etal \cite{osss18}, Vibe \etal \cite{vibc18}
\item[\twothousandnineteen] Deschamps \& Li \cite{deli19}, Davies \etal \cite{davk19}, 
                            Bodur \& Rey \cite{bore19}
\item[\twothousandtwenty] Briaud \etal \cite{braf20}, Mitrovica \etal \cite{miac20}
\end{itemize}
\end{scriptsize}


%------------------------------------------------------------------------------
%------------------------------------------------------------------------------
\subsection{Cratons}
\index{topics}{Cratons}

\begin{scriptsize}
\begin{itemize}
\item[\nineteenninetyseven] Burgess \etal \cite{bugm97}
\item[\nineteenninetynine] Drury \etal \cite{drdv99}, Lenardic \& Moresi \cite{lemo99}
\item[\twothousand] King \& Ritsema \cite{kiri00}, Lenardic \etal \cite{lemm00},
                    Schoofs \etal \cite{scth00}
\item[\twothousandone] Braun \& Shaw \cite{brsh01}, Drury \etal \cite{drvc01}
\item[\twothousandthree] Lenardic \etal \cite{lemm03}, Weinberg \etal \cite{wemv03}
\item[\twothousandeight] O'Neill \cite{onlg08}
\item[\twothousandnine] Cooper \& Conrad \cite{coco09}, Keranen \etal \cite{kekj09}
\item[\twothousandeleven] Huismans \& Beaumont \cite{hube11}
\item[\twothousandtwelve] Gorczyk \etal \cite{gohg12}, Guillou-Frottier \etal \cite{gubc12},
                          Miller \& Becker \cite{mibe12}, Polyansky \etal \cite{pokb12}
\item[\twothousandthirteen] Francois \etal \cite{frbm13}, Gac \etal \cite{gahs13},
                            Liao \etal \cite{ligw13}
\item[\twothousandfourteen] Ganne \etal \cite{gagb14}, Wang \etal \cite{wavp14},
                            Bao \etal \cite{baeg14}, Liao \& Gerya \cite{lige14}
\item[\twothousandfifteen] Wang \etal \cite{wahz15,wazh15}, Taramon \etal \cite{tarn15}
\item[\twothousandsixteen] Wang \etal \cite{wahz16}, Koptev \etal \cite{kobc16}
\item[\twothousandseventeen] Liao \etal \cite{liwg17}
\item[\twothousandeighteen] Ran \etal \cite{rabw18}, Beall \etal \cite{bemc18},
                            Gorczyk \etal \cite{gomb18}, Wenker \& Beaumont \cite{webe18b},
                            Wang \etal \cite{wavp18}, Paul \etal \cite{pagc19}
\item[\twothousandtwenty] Capitanio \etal \cite{canc20}, Celli \etal \cite{cels20},
                          Perchuk \etal \cite{pegz20}, Paul \& Ghosh \cite{pagh20}
\end{itemize}
\end{scriptsize}




%------------------------------------------------------------------------------
%------------------------------------------------------------------------------
\subsection{Lattice Boltzmann Method}
\index{topics}{Lattice Boltzmann Method}

Mora \& Yuen \cite{moyu17}
Mora \& Yuen \cite{moyu18}


%------------------------------------------------------------------------------
%------------------------------------------------------------------------------
\subsection{Lithospheric stress, intra-plate stress, intra-plate deformation}
\index{topics}{Lithospheric Stress}
\index{topics}{Intraplate Stress}
\index{topics}{Global Stress Field}
%------------------------------------------------------------------------------
%------------------------------------------------------------------------------

\begin{scriptsize}
\begin{itemize}
\item[\nineteenseventyfive] 
\textcite{fouy75}\cite{fouy75}\\
\textcite{sosr75}\cite{sosr75}\\
\item[\nineteenseventysix] 
\textcite{riss76}\cite{riss76}\\
\item[\nineteenseventyseven] 
\textcite{chtu77}\cite{chtu77}\\
\item[\nineteenseventynine] 
\textcite{riss79}\cite{riss79}\\
\item[\nineteeneightynine] 
\textcite{boww89}\cite{boww89}\\
\item[\nineteenninetyone] 
\textcite{worg91}\cite{worg91}\\
\item[\nineteenninetytwo] 
\textcite{rich92}\cite{rich92}\\
\textcite{wuvr92}\cite{wuvr92}\\
\textcite{zoba92}\cite{zoba92}\\
\textcite{clko92}\cite{clko92}\\
\item[\twothousandone] 
\cite{stsm01}
\item[\twothousandtwo] 
\cite{jack02}
\item[\twothousandfour] 
\cite{ligu04}
\item[\twothousandfive] 
\cite{timr05}
\item[\twothousandseven] 
\cite{hert07}
\item[\twothousandeight] 
\cite{bilr08}
\cite{ghhw08}
\cite{netv08}
\item[\twothousandnine] 
\cite{ghhf09}
\cite{nacl09}
\item[\twothousandten] 
\cite{bepo10}
\cite{yosh10}
\item[\twothousandtwelve] 
\cite{nalr12}
\cite{ghho12}
\cite{wagw12}
\item[\twothousandthirteen] 
\cite{ghhw13}\cite{wagw13}
\item[\twothousandfourteen] 
\cite{vagw14}
\item[\twothousandseventeen] 
\cite{grrb17}
\item[\twothousandeighteen] 
\cite{osss18} 
\cite{magu18}
\item[\twothousandnineteen] 
\cite{tamg19}
\end{itemize}
\end{scriptsize}


%--------------------------------------------------------------------
%------------------------------------------------------------------------------
\subsection{Paleomagnetism} 
\index{topics}{Paleomagnetism}
%--------------------------------------------------------------------
%------------------------------------------------------------------------------

TODO

Rolf \& Pesonen \cite{rope18}



%--------------------------------------------------------------------
%------------------------------------------------------------------------------
\subsection{Passive margins} 
\index{topics}{Passive Margins}
%--------------------------------------------------------------------
%------------------------------------------------------------------------------

\begin{scriptsize}
\nineteeneightytwo: \cite{clwv82}\\
\nineteeneightysix: \cite{lies86}\\
\twothousandfive: Gemmer \etal \cite{gebi05}\\
\twothousandeight: \cite{clbz08}\cite{kasb08}\\
\twothousandten: \cite{fasm10}\cite{nigm10}\\
\twothousandeleven: \cite{rapy11}\cite{nigm11}\cite{brfo11}\\
\twothousandthirteen: \cite{mana13}\cite{yahb13}\\
\twothousandfourteen: \cite{macg14}\\
\twothousandfifteen: \cite{gebw15}\cite{nigo15}\\
\twothousandsixteen: \cite{dupm16}\\
\twothousandeighteen: \cite{sahf18}\cite{mube18}\cite{tebu18}\\
\twothousandnineteen: Zhong \& Li \cite{zhli19}
\end{scriptsize}


%--------------------------------------------------------------------
%------------------------------------------------------------------------------
\subsection{Folding, buckling} 
\index{topics}{Folding} 
\index{topics}{Buckling}
%--------------------------------------------------------------------
%------------------------------------------------------------------------------

\todo[inline]{separate buckling from folding}

\begin{scriptsize}
\begin{itemize}
\item[\nineteenseventy] 
\textcite{ramb70} \citetitle{ramb70}\\
\item[\nineteenseventyone] 
\textcite{ramb71} \citetitle{ramb71}\\
\item[\nineteenseventyeight] 
\textcite{wilz78} \citetitle{wilz78}\\
\item[\nineteenninetyone] 
\textcite{flet91} \citetitle{flet91}\\
\item[\nineteenninetythree] 
\textcite{zhhj93} \citetitle{zhhj93}\\
\item[\nineteenninetyfive] 
\textcite{flet95} \citetitle{flet95}\\
\item[\nineteenninetysix] 
\textcite{zhho96} \citetitle{zhho96}\\
\item[\nineteenninetynine] 
\textcite{nagg99} \citetitle{nagg99}\\
\textcite{bupo99} \citetitle{bupo99}\\
\textcite{scpo99} \citetitle{scpo99}\\
\item[\twothousandone] 
\textcite{scpo01} \citetitle{scpo01}\\
\textcite{scpo01b} \citetitle{scpo01b}\\
\item[\twothousandtwo] 
\textcite{mumh02} \citetitle{mumh02}\\
\item[\twothousandthree] 
\textcite{ribe03} \citetitle{ribe03}\\
\textcite{nagv03} \citetitle{nagv03}\\
\item[\twothousandsix] 
\textcite{frsc06} \citetitle{frsc06}\\
\item[\twothousandseven] 
\textcite{risr07} \citetitle{risr07}\\
\item[\twothousandeight] 
\textcite{schm08} \citetitle{schm08}\\
\textcite{manc08} \citetitle{manc08}\\
\textcite{scdk08} \citetitle{scdk08}\\
\item[\twothousandnine] 
\textcite{simp09} \citetitle{simp09}\\
\item[\twothousandten] 
\textcite{resb10} \citetitle{resb10}\\
\item[\twothousandeleven] 
\textcite{freh11} \citetitle{freh11}\\
\item[\twothousandtwelve] 
\textcite{reds12} \citetitle{reds12}\\
\textcite{grsc12} \citetitle{grsc12}\\
\textcite{scsc12} \citetitle{scsc12}\\
\item[\twothousandthirteen] 
\textcite{regc13} \citetitle{regc13}\\
\item[\twothousandfourteen] 
\textcite{freh14} \citetitle{freh14}\\ 
\textcite{frex14} \citetitle{frex14}\\
\item[\twothousandsixteen] 
\textcite{frsc16} \citetitle{frsc16}\\
\end{itemize}
\end{scriptsize}

%--------------------------------------------------------------------
%------------------------------------------------------------------------------
\subsection{Geoid}
\index{topics}{Geoid}
%--------------------------------------------------------------------
%------------------------------------------------------------------------------

\begin{scriptsize}
\begin{itemize}
\item[\nineteeneightyfour] Davies \cite{davi84}, Hager \cite{hage84},
                           \cite{riff84}, \cite{riha84},
                           Watts \& Ribe \cite{wari84}
\item[\nineteeneightyfive] Hager \etal \cite{hacr85}, Chase \cite{chas85}
\item[\nineteeneightysix] Davies \cite{davi86}
\item[\nineteeneightyeight] Bercovici \etal \cite{besz88}, Forte \& Peltier \cite{fope88}
\item[\nineteeneightynine] Ricard \etal \cite{rivf89}
\item[\nineteenninetytwo] Zhong \& Gurnis \cite{zhgu92}, King \& Hager \cite{kiha92}, 
                          Ribe \cite{ribe92}
\item[\nineteenninetythree] Zhang \& Christensen \cite{zhch93}, Ricard \etal \cite{rirl93}
\item[\nineteenninetyfour] King \& Hager \cite{kiha94}
\item[\nineteenninetyfive] King \cite{king95}, Moresi \& Parsons \cite{mopa95}
\item[\nineteenninetysix] Moresi \& Gurnis \cite{mogu96}
\item[\nineteenninetyseven] Wen \& Anderson \cite{wean97a}, King \cite{king97} 
\item[\nineteenninetyeight] Cadek \& van den Berg \cite{cava98}, Chen \& King \cite{chki98},
                            Kiefer \& Kellogg \cite{kike98}
\item[\twothousandone] Zhong \cite{zhon01}
\item[\twothousandseven] Kaban \etal \cite{kart07}
\item[\twothousandeight] Metivier \& Conrad \cite{meco08}
\item[\twothousandnine] King \cite{king09}, Tosi \etal \cite{tocm09},
                        Yoshida \& Nakakuki \cite{yona09}
\item[\twothousandten] \cite{ghbz10}\cite{spgs10b}
\item[\twothousandeleven] Cadio \etal \cite{capd11}
\item[\twothousandtwelve] Hines \& Billen \cite{hibi12}, Cadio \etal \cite{cabp12}, 
                          Kaban \& Trubitsyn \cite{katr12}
\item[\twothousandthirteen] Shahraki \& Schmeling \cite{shsc13}, Chaves \& Ussami \cite{chus13}
\item[\twothousandfourteen] Cadio \& Korenaga \cite{cako14}, Kaban \etal \cite{kaps14}
\item[\twothousandfifteen] Liu \& Zhong \cite{lizh15}
\item[\twothousandsixteen] Nerlich \etal \cite{necg16}
\item[\twothousandseventeen] \cite{grab17}\\
\item[\twothousandeighteen] King \cite{king18}
\end{itemize}
\end{scriptsize}

%--------------------------------------------------------------------
%------------------------------------------------------------------------------
\subsection{Geothermal Energy} 
\index{topics}{Geothermal Energy}
%--------------------------------------------------------------------
%------------------------------------------------------------------------------

\begin{scriptsize}
Quenette \etal \cite{quxm15}, Renaud \etal \cite{revf19}
\end{scriptsize}

%--------------------------------------------------------------------
%------------------------------------------------------------------------------
\subsection{Grain size (evolution) \& influence on geodynamics}
\label{sec:topics:gsev}
\index{topics}{Grain Size (Evolution)}
\index{topics}{Grain Damage}
%--------------------------------------------------------------------
%------------------------------------------------------------------------------

\begin{scriptsize}
\nineteeneightyfour: Karato \cite{kara84}\\
\nineteenninetysix: Solomatov \cite{solo96}\\
\nineteenninetyseven: Kameyama \etal \cite{kayf97}\\
\nineteeneightynine: \cite{brcp99}\\
\twothousandone: de Bresser \etal \cite{dets01}, Solomatov \cite{solo01}\\
\twothousandtwo: Solomatov \etal \cite{soet02}\\
\twothousandthree: \cite{hapa03}, Regenauer-Lieb \& Yuen \cite{reyu03}\\
\twothousandeight: Solomatov \& Reese \cite{sore08}\\
\twothousandnine: Behn \etal \cite{behe09}\\
\twothousandeleven: \cite{rorb11}\\
\twothousandthirteen: \cite{beri13}\\
\twothousandfourteen: \cite{besr14}, Foley \& Bercovici \cite{fobe14} \\
\twothousandfifteen: \cite{thrk15}\cite{tukb15}\cite{pevp15}\cite{glfa15}\\
\twothousandseventeen: \cite{ceww17}\cite{daef17}\cite{mube17}\cite{scdu17}\\
\twothousandeighteen: Bercovici \& Mulyukova \cite{bemu18}, Bellas \etal \cite{bezb18},
                      Mulyukova \& Bercovici \cite{mube18}, Jain \etal \cite{jakk18},
                      Foley \cite{fole18} \\
\twothousandnineteen: Mulyukova \& Bercovici \cite{mube19}\\
\twothousandtwenty: 
\textcite{mube20} \citetitle{mube20}\\
\textcite{scrt20} \citetitle{scrt20}\\
\textcite{sctr20} \citetitle{sctr20}\\
\textcite{thsc20} \citetitle{thsc20}\\
\textcite{fole20} \citetitle{fole20}\\
\textcite{sctp20} \citetitle{sctp20}\\
\end{scriptsize}

%--------------------------------------------------------------------
%------------------------------------------------------------------------------
\subsection{Numerical hardware, GPU}
\label{sec:topics:hardware}
\index{topics}{Numerical Hardware}
\index{topics}{GPU}
%--------------------------------------------------------------------

\begin{scriptsize}
\twothousandsix: Oeser \etal \cite{oebm06}\\
\twothousandthirteen: Knepley \& Yuen \cite{knyu13}, Galvan \& Miller \cite{gami13}, 
                      Kl\"ockner \etal \cite{klwh13}, Sanchez \etal \cite{sagy13}\\
\twothousandfourteen: Zheng \etal{} \cite{zhzg14}\\
\twothousandfifteen: Ta \etal \cite{tact15}\\
\end{scriptsize}

%--------------------------------------------------------------------
%------------------------------------------------------------------------------
\subsection{LLSVP, ULVZ, CMB layer, thermo-chemical piles, D'' layer}
%------------------------------------------------------------------------------
%--------------------------------------------------------------------
\index{topics}{LLSVP}
\index{topics}{D'' layer}
\index{topics}{Thermo-chemical pile}
\index{topics}{Postperovskite Phase Transition}

\begin{center}
\includegraphics[width=5cm]{images/burk11}\cite{burk11}
\end{center}

\begin{scriptsize}
\begin{itemize}
\item[\nineteeneighty]       
\textcite{yupe80} \citetitle{yupe80}\\
\item[\nineteeneightysix]    
\textcite{dagu86} \citetitle{dagu86}\\
\item[\nineteeneightyeight]  
\textcite{hayu88} \citetitle{hayu88}\\
\item[\nineteeneightynine]   
\textcite{hayu89} \citetitle{hayu89}\\
\item[\nineteenninetyfour]   
\textcite{ride94} \citetitle{ride94}\\
\item[\nineteenninetysix]    
\textcite{boeh96} \citetitle{boeh96}\\
\item[\nineteenninetyseven]  
\textcite{kell97} \citetitle{kell97}\\
\item[\nineteenninetyeight]  
\textcite{tack98b} \citetitle{tack98b}\\
\item[\twothousandone]       
\textcite{soga01} \citetitle{soga01}\\
\item[\twothousandtwo]       
\textcite{somo02} \citetitle{somo02}\\
\textcite{tagh02} \citetitle{tagh02}\\
\item[\twothousandfour]      
\textcite{mczh04} \citetitle{mczh04}\\
\textcite{nata04} \citetitle{nata04}\\
\item[\twothousandfive]      
\textcite{nata05} \citetitle{nata05}\\
\textcite{wyso05} \citetitle{wyso05}\\
\textcite{mczh05a} \citetitle{mczh05a}\\
\textcite{nata05b} \citetitle{nata05b}\\
\item[\twothousandsix]       
\textcite{nata06} \citetitle{nata06}\\
\item[\twothousandseven]     
\textcite{heta07} \citetitle{heta07}\\
\textcite{moyu07} \citetitle{moyu07}\\
\textcite{pelt07} \citetitle{pelt07}\\
\textcite{hibl07} \citetitle{hibl07}\\
\textcite{yumc07} \citetitle{yumc07}\\
\item[\twothousandeight]     
\textcite{gamc08} \citetitle{gamc08}\\
\textcite{nata08} \citetitle{nata08}\\
\textcite{stho08} \citetitle{stho08}\\
\item[\twothousandnine]
\textcite{bumr09} \citetitle{bumr09}\\
\item[\twothousandten]
\textcite{stto10} \citetitle{stto10}\\
\textcite{mcgr10} \citetitle{mcgr10}\\
\textcite{nata10} \citetitle{nata10}\\
\textcite{vady10} \citetitle{vady10} \\
\textcite{toyc10} \citetitle{toyc10}\\
\item[\twothousandeleven]    
\textcite{bowg11} \citetitle{bowg11}\\
\textcite{talz11} \citetitle{talz11} \\ 
\textcite{vayj11} \citetitle{vayj11}\\
\textcite{dekt11} \citetitle{dekt11}\\
\textcite{burk11} \citetitle{burk11}\\
\item[\twothousandtwelve]    
\textcite{stto12} \citetitle{stto12}\\
\textcite{dagd12} \citetitle{dagd12}\\
\textcite{dect12} \citetitle{dect12}\\
\item[\twothousandthirteen]  
\textcite{limc13} \citetitle{limc13}\\
\textcite{bogs13a} \citetitle{bogs13a}\\
\textcite{bogs13b} \citetitle{bogs13b}\\
\item[\twothousandfourteen]  
\textcite{budt14} \citetitle{budt14}\\
\textcite{lidt14} \citetitle{lidt14}\\
\textcite{tovd14} \citetitle{tovd14}\\
\item[\twothousandfifteen]   
\textcite{musd15} \citetitle{musd15}\\
\textcite{hafg15} \citetitle{hafg15}\\
\textcite{delt15} \citetitle{delt15}\\
\textcite{wilm15} \citetitle{wilm15}\\
\textcite{lidt15} \citetitle{lidt15}\\
\textcite{sobd15} \citetitle{sobd15}\\
\item[\twothousandsixteen]   
\textcite{dost16} \citetitle{dost16}\\
\textcite{tosa16} \citetitle{tosa16}\\
\item[\twothousandseventeen] 
\textcite{hish17} \citetitle{hish17}\\
\textcite{lizh17} \citetitle{lizh17}\\
\item[\twothousandeighteen]  
\textcite{daga18} \citetitle{daga18}\\
\textcite{lizo18} \citetitle{lizo18}\\
\textcite{hect18} \citetitle{hect18}\\
\textcite{dert18} \citetitle{dert18}\\
\item[\twothousandnineteen]  
\textcite{hebo19} \citetitle{hebo19}\\
\textcite{rejv19} \citetitle{rejv19}\\
\textcite{mcna19} \citetitle{mcna19}\\
\item[\twothousandtwenty]    
\textcite{cilw20} \citetitle{cilw20}\\
\textcite{szes20} \citetitle{szes20}\\
\textcite{scrt20} \citetitle{scrt20}\\
\textcite{daro20} \citetitle{daro20}\\
\textcite{hect20b} \citetitle{hect20b}\\
\item[\twothousandtwentyone]
\textcite{cafb21} \citetitle{cafb21}\\    
\item[\twothousandtwentytwo] 
\textcite{limc22} \citetitle{limc22}\\ 
\end{itemize}
\end{scriptsize}

%--------------------------------------------------------------------
%------------------------------------------------------------------------------
\subsection{Magma ocean}
\index{topics}{Magma Oceans}
%--------------------------------------------------------------------
%------------------------------------------------------------------------------

\begin{scriptsize}
\begin{itemize}
\item[\nineteenninetythree] 
\textcite{sost93a} \citetitle{sost93a}\\
\textcite{sost93b} \citetitle{sost93b}\\
\item[\twothousandtwo] 
\textcite{elvh02} \citetitle{elvh02}\\
\item[\twothousandsix] 
\textcite{hosh06} \citetitle{hosh06}\\
\item[\twothousandseven] 
\textcite{solo07} \citetitle{solo07}\\
\item[\twothousandten] 
\textcite{devv10} \citetitle{devv10}\\
\item[\twothousandtwelve] 
\textcite{ullc12} \citetitle{ullc12}\\
\item[\twothousandthirteen] 
\textcite{plth13} \citetitle{plth13} \\
\textcite{moha13} \citetitle{moha13}\\
\item[\twothousandfifteen] 
\textcite{maha15} \citetitle{maha15}\\
\item[\twothousandtwenty] 
\textcite{bobm20} \citetitle{bobm20}\\
\textcite{agml20} \citetitle{agml20}\\
\end{itemize}
\end{scriptsize}

%--------------------------------------------------------------------
%------------------------------------------------------------------------------
\subsection{Magma transport / melting / two phase flow/ (intra-plate) volcanism / lava flow/ 
continental flood basalt}
\index{topics}{Magma Transport}
\index{topics}{Melting}
\index{topics}{Melt Migration}
\index{topics}{Two Phase Flow}
%------------------------------------------------------------------------------
%--------------------------------------------------------------------

\begin{scriptsize}
\begin{itemize}
\item[\nineteeneightyfour] 
\textcite{scst84} \citetitle{scst84}\\
\textcite{mcke84} \citetitle{mcke84}\\
\item[\nineteeneightyfive] 
\textcite{ribe85} \citetitle{ribe85}\\
\textcite{ribe85b} \citetitle{ribe85b}\\
\item[\nineteeneightysix] 
\textcite{scst86} \citetitle{scst86}\\
\textcite{ribe86} \citetitle{ribe86}\\
\item[\nineteeneightyseven] 
\textcite{hayu87} \citetitle{hayu87}\\
\textcite{spmc87} \citetitle{spmc87}\\
\textcite{rism87} \citetitle{rism87}\\
\textcite{ribe87} \citetitle{ribe87}\\
\item[\nineteeneightyeight] 
\textcite{scot88} \citetitle{scot88}\\
\textcite{ribe88b} \citetitle{ribe88b}\\
\item[\nineteenninety] 
\textcite{hayu90} \citetitle{hayu90}\\
\item[\nineteenninetythree] 
\textcite{spie93} \citetitle{spie93}\\
\textcite{tast93} \citetitle{tast93}\\
\item[\nineteenninetyfour] 
\textcite{jhpp94} \citetitle{jhpp94}\\
\textcite{sawy94} \citetitle{sawy94}\\
\item[\nineteenninetyfive] 
\textcite{bisc95} \citetitle{bisc95}\\
\textcite{crks95} \citetitle{crks95}\\
\textcite{ahwk95} \citetitle{ahwk95}\\
\item[\nineteenninetysix] 
\textcite{laki96} \citetitle{laki96}\\
\item[\nineteenninetyeight] 
\textcite{rabg98} \citetitle{rabg98}\\
\item[\nineteenninetynine] 
\textcite{devv99} \citetitle{devv99}\\
\textcite{momo99} \citetitle{momo99}\\
\item[\twothousand] 
\textcite{elha00} \citetitle{elha00}\\
\item[\twothousandone] 
\textcite{bers01} \citetitle{bers01}\\
\item[\twothousandtwo] 
\textcite{sobo02} \citetitle{sobo02}\\
\item[\twothousandthree] 
\textcite{beri03} \citetitle{beri03}\\
\item[\twothousandfive] 
\textcite{onml05} \citetitle{onml05}\\
\item[\twothousandsix] 
\textcite{onmm06} \citetitle{onmm06}\\
\item[\twothousandseven] 
\textcite{srrb07} \citetitle{srrb07}\\
\textcite{mohb07} \citetitle{mohb07}\\
\textcite{elki07} \citetitle{elki07}\\
\textcite{copb07} \citetitle{copb07}\\
\item[\twothousandeight] 
\textcite{hets08} \citetitle{hets08}\\
\textcite{hest08} \citetitle{hest08}\\
\item[\twothousandnine] 
\textcite{bavi09} \citetitle{bavi09}\\
\item[\twothousandten] 
\textcite{baiv10} \citetitle{baiv10}\\
\textcite{habl10} \citetitle{habl10}\\
\textcite{cows10} \citetitle{cows10}\\
\textcite{dekc10} \citetitle{dekc10}\\
\item[\twothousandeleven] 
\textcite{baiv11} \citetitle{baiv11}\\
\textcite{zhgy11} \citetitle{zhgy11}\\
\textcite{zhgh11} \citetitle{zhgh11}\\
\textcite{bics11} \citetitle{bics11}\\
\textcite{mobh11} \citetitle{mobh11}\\
\item[\twothousandtwelve] 
\textcite{yatd12} \citetitle{yatd12}\\
\textcite{kasc12b} \citetitle{kasc12b}\\
\textcite{ullc12} \citetitle{ullc12}\\
\item[\twothousandthirteen] 
\textcite{kemk13} \citetitle{kemk13}\\
\textcite{mofm13} \citetitle{mofm13}\\
\textcite{mowe13} \citetitle{mowe13}\\
\item[\twothousandfourteen] 
\textcite{kast14} \citetitle{kast14}\\
\item[\twothousandfifteen] 
\textcite{tukb15} \citetitle{tukb15}\\
\textcite{moba15} \citetitle{moba15}\\
\textcite{rerl15} \citetitle{rerl15}\\
\textcite{riag15} \citetitle{riag15}\\
\textcite{rey15} \citetitle{rey15}\\
\textcite{yadm15} \citetitle{yadm15}\\
\item[\twothousandsixteen] 
\textcite{keka16} \citetitle{keka16}\\
\textcite{vade16} \citetitle{vade16}\\
\textcite{mesj16} \citetitle{mesj16}\\
\textcite{dalg16} \citetitle{dalg16}\\
\textcite{porb16} \citetitle{porb16}\\
\item[\twothousandseventeen] 
\textcite{dilc17} \citetitle{dilc17}\\
\item[\twothousandeighteen] 
\textcite{lorg18} \citetitle{lorg18}\\
\textcite{scmo18} \citetitle{scmo18}\\
\item[\twothousandnineteen] 
\textcite{dagg19} \citetitle{dagg19}\\
\textcite{scmw19} \citetitle{scmw19}\\
\item[\twothousandtwenty] 
\textcite{siss20} \citetitle{siss20}\\
\textcite{zhbp20} \citetitle{zhbp20}\\
\textcite{rubk20} \citetitle{rubk20}\\
\textcite{rukb20} \citetitle{rukb20}\\
\textcite{cobd20} \citetitle{cobd20}\\
\textcite{lerm20} \citetitle{lerm20}\\
\item[\twothousandtwentyone] 
\textcite{dudm21} \citetitle{dudm21}\\ 
\end{itemize}
\end{scriptsize}

%--------------------------------------------------------------------
%--------------------------------------------------------------------
\subsection{Magma chambers}
\index{topics}{Magnma Chamber}
%--------------------------------------------------------------------
%--------------------------------------------------------------------

\begin{scriptsize}
\begin{itemize}
\item[\nineteeneightytwo] 
\textcite{spyk82} \citetitle{spyk82}\\
\item[\nineteeneightyseven] 
\textcite{hayu87} \citetitle{hayu87}\\
\item[\twothousandfour] 
\textcite{geys04} \citetitle{geys04}\\
\item[\twothousandtwelve] 
\textcite{gerb12} \citetitle{gerb12}\\ 
\textcite{gech12} \citetitle{gech12}\\
\item[\twothousandfourteen] 
\textcite{cuwi14} \citetitle{cuwi14}\\
\item[\twothousandeighteen] 
\textcite{gehn18} \citetitle{gehn18}\\
\end{itemize}
\end{scriptsize}

%--------------------------------------------------------------------
%------------------------------------------------------------------------------
\subsection{Mantle convection/dynamics, whole Earth models, plate interaction}
\index{topics}{Mantle Convection}
%--------------------------------------------------------------------
%------------------------------------------------------------------------------

\begin{scriptsize}
\begin{itemize}
\item[\nineteensixtyseven] 
\textcite{tuox67} \citetitle{tuox67}\\
\item[\nineteenseventyone] 
\textcite{totu71} \citetitle{totu71}\\
\item[\nineteenseventytwo] 
\textcite{pelt72} \citetitle{pelt72}\\
\item[\nineteenseventyfour] 
\textcite{youn74} \citetitle{youn74}\\
\textcite{mcrw74} \citetitle{mcrw74}\\
\item[\nineteenseventyfive] 
\textcite{hemw75} \citetitle{hemw75}\\
\textcite{buss75} \citetitle{buss75}\\
\item[\nineteenseventysix] 
\textcite{mcri76} \citetitle{mcri76}\\
\textcite{sath76} \citetitle{sath76}\\
\item[\nineteenseventyseven] 
\textcite{yusc77} \citetitle{yusc77}\\
\item[\nineteenseventyeight] 
\textcite{mahz78} \citetitle{mahz78}\\ 
\textcite{hsui78} \citetitle{hsui78}\\
\textcite{haoc78} \citetitle{haoc78}\\
\textcite{pamc78} \citetitle{pamc78}\\
\textcite{rimc78} \citetitle{rimc78}\\
\item[\nineteenseventynine] 
\textcite{ludt79} \citetitle{ludt79}\\ 
\textcite{buss79} \citetitle{buss79}\\
\textcite{shpe79} \citetitle{shpe79}\\
\textcite{phiv79} \citetitle{phiv79}\\
\item[\nineteeneighty] 
\textcite{olco80} \citetitle{olco80}\\
\textcite{jamc80} \citetitle{jamc80}\\
\textcite{scsc80} \citetitle{scsc80}\\
\textcite{zess80} \citetitle{zess80}\\
\textcite{daly80} \citetitle{daly80}\\
\item[\nineteeneightyone] 
\textcite{yups81} \citetitle{yups81}\\
\textcite{buss81} \citetitle{buss81}\\
\textcite{jasc81} \citetitle{jasc81}\\
\textcite{haoc81} \citetitle{haoc81}\\
\textcite{cotu81} \citetitle{cotu81}\\
\item[\nineteeneightytwo] 
\textcite{jape82} \citetitle{jape82}\\
\textcite{homc82} \citetitle{homc82}\\
\textcite{buri82} \citetitle{buri82}\\
\item[\nineteeneightythree] 
\textcite{hous83} \citetitle{hous83}\\
\textcite{hous83b} \citetitle{hous83b}\\
\textcite{chri83} \citetitle{chri83}\\
\textcite{mcke83} \citetitle{mcke83}\\
\textcite{chri83b} \citetitle{chri83b}\\
\textcite{zesd83} \citetitle{zesd83}\\
\item[\nineteeneightyfour] 
\textcite{olyb84} \citetitle{olyb84}\\
\textcite{jarv84} \citetitle{jarv84}\\
\textcite{haeb84} \citetitle{haeb84}\\
\textcite{haeb84b} \citetitle{haeb84b}\\
\textcite{harp84} \citetitle{harp84}\\
\textcite{davi84} \citetitle{davi84}\\
\textcite{boas84} \citetitle{boas84}\\
\textcite{chri84} \citetitle{chri84}\\
\textcite{chri84b} \citetitle{chri84b}\\
\textcite{moca84} \citetitle{moca84}\\
\textcite{flyu84} \citetitle{flyu84}\\
\textcite{flyu84b} \citetitle{flyu84b}\\
\item[\nineteeneightyfive] 
\textcite{jarv85} \citetitle{jarv85}\\
\textcite{baum85} \citetitle{baum85}\\
\textcite{chri85} \citetitle{chri85}\\
\textcite{csra85} \citetitle{csra85}\\
\textcite{scan85} \citetitle{scan85}\\
\item[\nineteeneightysix] 
\textcite{davi86} \citetitle{davi86}\\
\textcite{guda86} \citetitle{guda86}\\
\textcite{quys86} \citetitle{quys86}\\
\textcite{crmc86} \citetitle{crmc86}\\
\item[\nineteeneightyseven] 
\textcite{yuqh87} \citetitle{yuqh87}\\
\item[\nineteeneightyeight] 
\textcite{haeb88} \citetitle{haeb88}\\
\textcite{glat88} \citetitle{glat88}\\
\textcite{gurn88} \citetitle{gurn88}\\
\textcite{viyu88} \citetitle{viyu88}\\
\textcite{whit88} \citetitle{whit88}\\
\textcite{davi88} \citetitle{davi88}\\
\textcite{grpa98} \citetitle{grpa98}\\
\item[\nineteeneightynine] 
\textcite{weoy89} \citetitle{weoy89}\\
\textcite{chyu89} \citetitle{chyu89}\\
\textcite{besg89} \citetitle{besg89}\\
\textcite{schm89} \citetitle{schm89}\\
\textcite{sthe89} \citetitle{sthe89}\\
\textcite{rivi89} \citetitle{rivi89}\\
\textcite{davi89} \citetitle{davi89}\\
\item[\nineteenninety] 
\textcite{trab90} \citetitle{trab90}\\
\textcite{gurn90} \citetitle{gurn90}\\
\textcite{ketu90} \citetitle{ketu90}\\
\textcite{sope90} \citetitle{sope90}\\
\item[\nineteenninetyone] 
\textcite{jarv91} \citetitle{jarv91}\\
\textcite{chha91} \citetitle{chha91}\\
\textcite{mawe91} \citetitle{mawe91}\\
\textcite{gaot91} \citetitle{gaot91}\\
\textcite{vayv91} \citetitle{vayv91}\\
\textcite{hayk91} \citetitle{hayk91}\\
\textcite{leys91} \citetitle{leys91}\\
\textcite{mayu91} \citetitle{mayu91}\\
\item[\nineteenninetytwo] 
\textcite{dari92} \citetitle{dari92}\\
\textcite{besg92} \citetitle{besg92}\\
\textcite{vayv92} \citetitle{vayv92}\\
\textcite{chri92} \citetitle{chri92}\\
\textcite{haym92} \citetitle{haym92}\\
\textcite{rien92} \citetitle{rien92}\\
\textcite{hayk92} \citetitle{hayk92}\\
\textcite{mayw92} \citetitle{mayw92}\\
\textcite{mayu92} \citetitle{mayu92}\\
\item[\nineteenninetythree] 
\textcite{bayr93} \citetitle{bayr93}\\
\textcite{zhch93} \citetitle{zhch93}\\
\textcite{jarv93} \citetitle{jarv93}\\
\textcite{tack93} \citetitle{tack93}\\
\textcite{carm93} \citetitle{carm93}\\
\textcite{vavy93} \citetitle{vavy93}\\
\textcite{tasg93} \citetitle{tasg93}\\
\textcite{zhgu93} \citetitle{zhgu93}\\
\textcite{mamc93} \citetitle{mamc93}\\
\textcite{zebi93} \citetitle{zebi93}\\
\textcite{vayv93} \citetitle{vayv93}\\
\textcite{hayk93} \citetitle{hayk93}\\
\textcite{hayu93} \citetitle{hayu93}\\
\textcite{hoyb93} \citetitle{hoyb93}\\
\textcite{hoby93} \citetitle{hoby93}\\
\item[\nineteenninetyfour] 
\textcite{yurb94} \citetitle{yurb94}\\
\textcite{bayu94} \citetitle{bayu94}\\
\textcite{haeb94} \citetitle{haeb94}\\
\textcite{bucc94} \citetitle{bucc94}\\
\textcite{chho94} \citetitle{chho94}\\
\textcite{tasg94} \citetitle{tasg94}\\
\textcite{itki94} \citetitle{itki94}\\
\textcite{leka94} \citetitle{leka94}\\
\textcite{scha94} \citetitle{scha94}\\
\item[\nineteenninetyfive] 
\textcite{styu95} \citetitle{styu95}\\
\textcite{bayr95} \citetitle{bayr95}\\
\textcite{bayr95b} \citetitle{bayr95b}\\
\textcite{zhgu95} \citetitle{zhgu95}\\
\textcite{vayv95} \citetitle{vayv95}\\
\textcite{buba95} \citetitle{buba95}\\
\textcite{rasz95} \citetitle{rasz95}\\
\textcite{berc95} \citetitle{berc95}\\
\textcite{puhj95} \citetitle{puhj95}\\
\textcite{pujh95} \citetitle{pujh95}\\
\textcite{solo95} \citetitle{solo95}\\
\textcite{vayu95} \citetitle{vayu95}\\
\textcite{matb95} \citetitle{matb95}\\
\textcite{thmc95} \citetitle{thmc95}\\
\item[\nineteenninetysix] 
\textcite{laym96} \citetitle{laym96}\\
\textcite{zhyu96} \citetitle{zhyu96}\\
\textcite{hond96} \citetitle{hond96}\\
\textcite{rytr96a} \citetitle{rytr96a}\\
\textcite{rytr96b} \citetitle{rytr96b}\\
\textcite{tack96} \citetitle{tack96}\\
\textcite{trbo96} \citetitle{trbo96}\\
\textcite{birg96} \citetitle{birg96}\\
\textcite{burb96} \citetitle{burb96}\\
\textcite{kafo96} \citetitle{kafo96}\\
\textcite{guez96} \citetitle{guez96}\\
\textcite{vayu96} \citetitle{vayu96}\\
\textcite{rasz96} \citetitle{rasz96}\\
\textcite{rasz96b} \citetitle{rasz96b}\\
\textcite{leka96} \citetitle{leka96}\\
\textcite{iwas96} \citetitle{iwas96}\\
\textcite{buri96} \citetitle{buri96}\\
\textcite{schh96} \citetitle{schh96}\\
\textcite{trha96} \citetitle{trha96}\\
\item[\nineteenninetyseven] 
\textcite{deja97} \cite{deja97} \\
\textcite{hond97} \cite{hond97}\\
\textcite{iwho97} \cite{iwho97} \\
\textcite{burb97} \cite{burb97}\\
\textcite{mole97} \cite{mole97} \\
\textcite{somo97} \cite{somo97}\\
\textcite{rats97} \cite{rats97} \\
\textcite{cicv97} \cite{cicv97}\\
\textcite{vayu97} \cite{vayu97} \\
\textcite{laym97} \cite{laym97}\\
\textcite{mebr97} \cite{mebr97} \\
\textcite{csyu97} \cite{csyu97}\\
\item[\nineteenninetyeight] 
\textcite{ande98} \cite{ande98}\\
\textcite{iwho98} \cite{iwho98}\\
\textcite{devv98} \cite{devv98}\\
\textcite{tack98} \cite{tack98}\\
\textcite{tack98b} \cite{tack98b}\\
\textcite{trha98b} \cite{trha98b}\\
\textcite{trha98} \cite{trha98}\\
\textcite{burl98} \cite{burl98}\\
\textcite{mokm98} \cite{mokm98}\\
\textcite{lena98} \cite{lena98}\\
\textcite{vayu98} \cite{vayu98}\\
\textcite{wema98} \cite{wema98}\\
\item[\nineteenninetynine] 
\textcite{resb99} \cite{resb99}\\
\textcite{duyr99} \cite{duyr99}\\
\textcite{vazh99} \cite{vazh99}\\
\textcite{dava99} \cite{dava99}\\
\textcite{tabg99} \cite{tabg99}\\
\textcite{como99} \cite{como99}\\
\textcite{cicv99} \cite{cicv99}\\
\textcite{trrj99} \cite{trrj99}\\
\textcite{loga99} \cite{loga99}\\
\textcite{momo99} \cite{momo99}\\
\item[\twothousand] 
\textcite{albe00} \cite{albe00}\\
\textcite{hayu00} \cite{hayu00}\\
\textcite{devv00b} \cite{devv00b}\\
\textcite{tack00} \cite{tack00}\\
\textcite{tack00b} \cite{tack00b}\\
\textcite{tack00c} \cite{tack00c}\\
\textcite{tack00d} \cite{tack00d}\\
\textcite{zhzm00} \cite{zhzm00}\\
\textcite{legm00} \cite{legm00}\\
\textcite{conr00} \cite{conr00}\\
\textcite{somo00} \cite{somo00}\\
\textcite{duyu00} \cite{duyu00}\\
\textcite{duyy00} \cite{duyy00}\\
\item[\twothousandone] 
\textcite{vank01} \cite{vank01}\\
\textcite{riyb01} \cite{riyb01}\\
\textcite{lemo01} \cite{lemo01}\\
\textcite{vays01} \cite{vays01}\\
\textcite{moqu01} \cite{moqu01}\\
\textcite{zhon01} \cite{zhon01}\\
\textcite{burm01} \cite{burm01}\\
\textcite{dabu01} \cite{dabu01}\\
\item[\twothousandtwo] 
\textcite{tasu02} \cite{tasu02}\\
\textcite{modm02} \cite{modm02}\\
\textcite{tack02} \cite{tack02}\\
\textcite{vaya02} \cite{vaya02}\\
\textcite{vayu02} \cite{vayu02}\\
\textcite{taxi02} \cite{taxi02}\\
\textcite{scbh02} \cite{scbh02}\\
\textcite{strb02} \cite{strb02}\\
\textcite{duyr02} \cite{duyr02}\\
\textcite{hiys02} \cite{hiys02}\\
\item[\twothousandthree] 
\textcite{hapa03} \cite{hapa03}\\
\textcite{lemo03} \cite{lemo03}\\
\textcite{mumc03} \cite{mumc03}\\
\textcite{fasa03} \cite{fasa03}\\
\textcite{heta03} \cite{heta03}\\
\textcite{sibu03} \cite{sibu03}\\
\textcite{ogaw03} \cite{ogaw03}\\
\textcite{ogaw03b} \cite{ogaw03b}\\
\textcite{kore03} \cite{kore03}\\
\item[\twothousandfour] 
\textcite{thkl04} \cite{thkl04} \\
\textcite{vavv04b} \cite{vavv04b}\\
\textcite{xita04b} \cite{xita04b}\\
\textcite{xita04} \cite{xita04}\\
\textcite{nata04b} \cite{nata04b}\\
\textcite{vayr04} \cite{vayr04}\\
\textcite{brws04} \cite{brws04}\\
\textcite{stsh04} \cite{stsh04}\\
\textcite{scbh04} \cite{scbh04} \\
\textcite{leda04} \cite{leda04}\\
\textcite{leda04b} \cite{leda04b}\\ 
\item[\twothousandfive]
\cite{resb05}
\cite{taxn05}
\cite{bupc05}
\cite{grlt05}
\cite{lemj05}
\cite{kogk05}
\cite{mczh05b}
\cite{vary05}
\cite{nata05}
\cite{nabu05}
\cite{chob05}
\cite{phbu05}
\cite{hosh05}
\item[\twothousandsix] 
\cite{soba06} 
\cite{beck06} 
\cite{nake06} 
\cite{losh06} 
\cite{sthh06} 
\cite{yoka06}
\item[\twothousandseven] 
\cite{ghja07}
\cite{nake07} 
\cite{mayu07}
\cite{brva07a}
\cite{brva07b}
\cite{grlt07}
\cite{grlt07b}
\cite{huda07}
\cite{tanh07} 
\cite{tagu07} 
\cite{jalo07} 
\cite{galo07}
\cite{galo07b} 
\cite{nelo07} 
\cite{soba07}
\item[\twothousandeight] Ghias \& Jarvis \cite{ghja08}, Tackley \cite{tack08,tack08b},
                   Chiu-Webster \etal \cite{chhl08}, Brandenburg \etal \cite{brhv08},
                   Deschamps \& Tackley \cite{deta08}, Plank \& van Keken \cite{plva08},
                   Hoink \& Lenardic \cite{hole08}, van Heck \& Tackley \cite{vata08},
                   Trubitsyn \etal \cite{trkr08}, Shahnas \etal \cite{shlj08},
                   Stein \& Hansen \cite{stha08}, Yoshida \cite{yosh08}, Gait \etal \cite{galg08}
\item[\twothousandnine] Wolstencroft \etal \cite{wodd09}, Foley \& Becker \cite{fobe09},
                  Gottschaldt \etal \cite{gows09}, Deschamps \& Tackley \cite{deta09},
                  \cite{onlj09}\cite{wazh09},
                  \cite{vavv09}, Breuer \& Hansen \cite{brha09},
                  \cite{scbs09b}, Oeser \etal \cite{oebm09},
                  Fujita \& Ogawa \cite{fuog09}
\item[\twothousandten] O'Farrell \& Lowman \cite{oflo10}, \cite{bumb10}
                 \cite{detn10}\cite{yayh10}
                 \cite{nata10}\cite{hole10}
                 \cite{zhzl10}\cite{vayb10}
                 \cite{brmw10}
\item[\twothousandeleven] Yuen \etal \cite{yutc11}, Lowman \cite{lowm11},
                    Rolf \& Tackley \cite{rota11}, Wolstencroft \& Davies \cite{woda11},
                    Lenardic \etal \cite{lemj11}, Becker \& Faccenna \cite{befa11},
                    Petschel \etal \cite{pewb11}, Androvandi \etal \cite{andl11}
\item[\twothousandtwelve] Biggin \etal \cite{bisa12}, Coltice \etal \cite{cort12b}
                    Deschamps \etal \cite{deyt12}, Solomatov \cite{solo12}, 
                    Weller \& Lenardic \cite{wele12}
\item[\twothousandthirteen] \cite{holj13}\cite{dadb13}, 
                      Tosi \etal \cite{toyd13}, Bower \etal \cite{bogs13a},
                      Burstedde \etal \cite{busa13}, Miyauchi \& Kameyama \cite{mika13},
                      Faccenna \etal \cite{fabc13}, Coltice \cite{cosr13},
                      Cooper \etal \cite{coml13}, Conrad \etal \cite{cost13},
                      Stein \& Hansen \cite{stha13}, Plesa \etal \cite{plth13},
                      O'Farrell \etal \cite{oflb13}, Whitehead \etal \cite{whch13}
\item[\twothousandfourteen] \cite{arfw14}\cite{helo14}\cite{crta14}\cite{flgw14}
                      \cite{roct14}\cite{cort14}\cite{becr14}
                      \cite{nata14}\cite{stha14}\cite{stlh14}\cite{ogaw14}

\item[\twothousandfifteen]   
\textcite{zhru15}  
\textcite{wegg15}
\textcite{bect15}  
\textcite{pesw15}  
\textcite{khfh15}
\item[\twothousandsixteen]   
\textcite{frbs16}  
\textcite{sisc16}
\textcite{boba16}  
\textcite{wele16}
\textcite{welm16}  
\textcite{vade16}
\textcite{chah16}  
\textcite{woso16b}

\item[\twothousandseventeen] 
\textcite{badw17}
\textcite{ghts17}
\textcite{civj17}

\item[\twothousandeighteen] Guerrero \etal \cite{guld18}, Coltice \etal \cite{cold18}, 
                            Arnould \etal \cite{arcf18}, Coltice \& Sheppard \cite{cosh18}, 
                            Weller \& Lenardic \cite{wele18}, Richards \& Lenardic \cite{rile18}

\item[\twothousandnineteen] 
\textcite{gult19} \citetitle{gult19}\\
\textcite{mazh19} \citetitle{mazh19}\\
\textcite{cohf19} \citetitle{cohf19}\\
\textcite{lewh19} \citetitle{lewh19}\\
\textcite{ulcw19} \citetitle{ulcw19}\\
\textcite{boba19} \citetitle{boba19}\\
\textcite{fube19} \citetitle{fube19}\\
\textcite{plju19} \citetitle{plju19}\\

\item[\twothousandtwenty] 
\textcite{lalt20} \citetitle{lalt20}\\
\textcite{gugb20} \citetitle{gugb20}\\
\textcite{yabt20} \citetitle{yabt20}\\
\textcite{yosy20} \citetitle{yosy20}\\
\textcite{arcf20} \citetitle{arcf20}\\
\textcite{babd20} \citetitle{babd20}\\
\textcite{lorb20} \citetitle{lorb20}\\
\textcite{loru20} \citetitle{loru20}\\

\item[\twothousandtwentyone] \textcite{lalt21}, \textcite{khmo21}
\end{itemize}
\end{scriptsize}

%--------------------------------------------------------------------
%------------------------------------------------------------------------------
\subsection{Mantle rheology, phase transitions, stratification, (temperature) profile}
\index{topics}{Phase Transition} 
\index{topics}{Phase Diagram} 
\index{topics}{Mantle Rheology} 
\index{topics}{Mantle Viscosity} 
\index{topics}{Mantle Stratification} 
\index{topics}{Mantle Structure} 
%------------------------------------------------------------------------------
%--------------------------------------------------------------------

\begin{scriptsize}
\begin{itemize}
\item[1923] Williamson \& Adams \cite{wiad23}
\item[1952] Birch \cite{birc52}
\item[\nineteenseventysix] O'Connell \cite{ocon76}
\item[\nineteenseventyseven] Stacey \cite{stac77}
\item[\nineteeneightytwo] Yuen \etal \cite{yusb82}, Christensen \cite{chri82}
\item[\nineteeneightyfive] Christensen \& Yuen \cite{chyu85}
\item[\nineteeneightysix] Yuen \cite{yuen86} 
\item[\nineteeneightynine] Ito \& Katsura \cite{itka89} 
\item[\nineteenninetyone] Forte \etal \cite{fopd91} 
\item[\nineteenninetytwo] Zhao \etal \cite{zhyh92}
\item[\nineteenninetythree] Tackley \etal \cite{tasg93}, Bercovici \etal \cite{best93}, 
                      Kiefer \cite{kief93}, Steinbach \etal \cite{styz93},
                      Yuen \etal \cite{yucc93}, Honda \etal \cite{hoby93}, 
                      Daessler \& Yuen \cite{dayu93} 
\item[\nineteenninetyfour] Cadek \etal \cite{cays94}, \cite{vayv94}
                    \cite{zhgu94b}\cite{styu94}, Solheim \& Peltier \cite{sope94},
                    Podladchikov \etal \cite{popy94}
\item[\nineteenninetyfive] King \& Ita \cite{kiit95}, Zhang \& yuen \cite{zhyu95}, 
                     Christensen \cite{chri95}, Schubert \& Tackley \cite{scta95},
                     Tackley \cite{tack95}
\item[\nineteenninetysix] Peltier \cite{pelt96}, Mitrovica \cite{mitr96}, Tackley \cite{tack96b}
\item[\nineteenninetyseven] \cite{mifo97}, Peltier \etal \cite{pebs97}
\item[\nineteenninetyeight] Cadek \& van den Berg\cite{cava98}, Kennett \cite{kenn98}
\item[\nineteenninetynine] Sidorin \etal \cite{sigh99}, Kellogg \etal \cite{kehv99}, 
                     van der Hilst \& Karason \cite{vaka99}
\item[\twothousandone] Romanowicz \cite{roma01}
\item[\twothousandthree] Bercovici \& Karato \cite{beka03} 
\item[\twothousandfive] \cite{hett05}\cite{nata05b}\cite{nabu05}\cite{stli05}\cite{stli05b}
\item[\twothousandsix] Jacobs \etal \cite{javd06}, Steinberger \& Calderwood \cite{stca06}
\item[\twothousandseven] 
Paulson \etal \cite{pazw07}, 
Moucha \etal \cite{mofm07}, 
Tackley ey al \cite{tanh07}, 
Stixrude \& Lithgow-Bertelloni \cite{stli07}, 
Litasov \& Ohtani \cite{lioh07},
Jacobs \& de Jong \cite{jade07},
Piazzoni \etal \cite{pisb07}, 
Kaban \etal \cite{kart07}
\textcite{conn09} \citetitle{conn09}
\item[\twothousandnine] Nakagawa \etal \cite{natd09}
\item[\twothousandten] Katsura \etal \cite{kayy10}
\item[\twothousandeleven] Matyska \etal \cite{mayw11}, Jacobs \& van den Berg \cite{java11}, 
                    Faul \etal \cite{faff11}, Nakagawa \& Tackley \cite{nata11}, 
                    van den Berg \etal \cite{vayj11}, Stixrude \& Lithgow-Bertelloni \cite{stli11}
\item[\twothousandtwelve] Tackley \cite{tack12}, Samuel \& Tosi \cite{sato12}, 
                    Nakagawa \etal \cite{natd12}, Stixrude \& Lithgow-Bertelloni \cite{stli12}
\item[\twothousandthirteen] Farla \etal \cite{fakc13}, Tackley \etal \cite{taab13}, Jacobs \etal \cite{jasv13}
\item[\twothousandfifteen] Ballmer \etal \cite{basn15}, Glisovic \etal \cite{glfa15}, Amodeo \etal \cite{amsb15}
\item[\twothousandsixteen] Tirone \cite{tiro16}, Benesova \& Ciskova \cite{beci16}
\item[\twothousandseventeen] van der Meer \etal \cite{vavs17}, Jacobs \etal \cite{jasv17}, 
                             Ballmer \etal \cite{bahh17}, Shahnas \etal \cite{shyp17,shpj17}
\item[\twothousandeighteen] Mao \& Zhong \cite{mazh18}, Nakada \etal \cite{naoi18}, Rolf \etal \cite{roct18}
\item[\twothousandnineteen] Jacobs \etal \cite{jasv19}
\item[\twothousandtwenty] Houser \etal \cite{hohv20}, Lu \etal{} \cite{lufs20}, Wang \& Li \cite{wali20},
                          Rudolph \etal \cite{ruml20}
\item[\twothousandtwentyone] Pokorny \etal \cite{pocv21}, Vesterholt \etal \cite{vepn21},
                             Adam \etal \cite{adkc21}, Liu \etal \cite{ligl21b}
\end{itemize}
\end{scriptsize}


%--------------------------------------------------------------------
%------------------------------------------------------------------------------
\subsection{Mantle wedge} 
\index{topics}{Mantle Wedge}
%------------------------------------------------------------------------------
%--------------------------------------------------------------------

\begin{scriptsize}
\begin{itemize}
\item[\nineteensixtynine] 
\textcite{mcke69} \citetitle{mcke69}\\
\item[\nineteenseventyone] 
\textcite{tomj71} \citetitle{tomj71}\\
\item[\nineteenseventyeight] 
\textcite{tosl78} \citetitle{tosl78}\\
\item[\nineteenseventynine] 
\textcite{bobo79} \citetitle{bobo79}\\
\item[\nineteeneightyfive] 
\textcite{hond85} \citetitle{hond85}\\
\item[\nineteenninetytwo] 
\textcite{dast92} \citetitle{dast92}\\
\item[\nineteenninetythree] 
\textcite{furu93} \citetitle{furu93}\\
\item[\nineteenninetynine] 
\textcite{pewa99} \citetitle{pewa99}\\
\item[\twothousandone] 
\textcite{bigu01} \citetitle{bigu01}\\
\textcite{haki01} \citetitle{haki01}\\
\item[\twothousandtwo]
\textcite{vakp02} \citetitle{vakp02}\\
\item[\twothousandthree]
\textcite{vank03} \citetitle{vank03}
\item[\twothousandfour]
\textcite{enwi04} \citetitle{enwi04}\\
\textcite{cuwh04} \citetitle{cuwh04}\\
\item[\twothousandsix] 
\textcite{abvk06} \citetitle{abvk06}\\
\textcite{gogc06} \citetitle{gogc06}\\
\textcite{gecy06} \citetitle{gecy06}\\
\textcite{syab06} \citetitle{syab06}\\
\textcite{lafh06} \citetitle{lafh06}\\
\item[\twothousandseven] 
\textcite{gogc07} \citetitle{gogc07}\\
\textcite{knvk07} \citetitle{knvk07}\\
\textcite{lohd07} \citetitle{lohd07}\\
\item[\twothousandeight]
\textcite{knva08} \citetitle{knva08}\\ 
\textcite{cage08} \citetitle{cage08}\\
\textcite{vack08} \citetitle{vack08}\\
\textcite{wawh08} \citetitle{wawh08}\\
\item[\twothousandnine] 
\textcite{leki09} \citetitle{leki09}\\
\textcite{heaa09} \citetitle{heaa09}\\
\textcite{wawa09} \citetitle{wawa09}\\
\item[\twothousandten]
\textcite{roms10} \citetitle{roms10}\\
\textcite{hogz10} \citetitle{hogz10}\\
\item[\twothousandeleven] 
\textcite{zhgh11} \citetitle{zhgh11}\\
\item[\twothousandfourteen]
\textcite{ledg14} \citetitle{ledg14}\\
\textcite{mabv14} \citetitle{mabv14}\\
\item[\twothousandfourteen]
\textcite{wahh15} \citetitle{wahh15}\\
\item[\twothousandsixteen]
\textcite{dalg16} \citetitle{dalg16}\\
\item[\twothousandseventeen]
\textcite{rerm17} \citetitle{rerm17}\\
\item[\twothousandeighteen]
\textcite{pltv18} \citetitle{pltv18}\\
\item[\twothousandtwentyone]
\textcite{wada21} \citetitle{wada21}\\
\end{itemize}
\end{scriptsize}

%--------------------------------------------------------------------
%------------------------------------------------------------------------------
\subsection{Mixing, stirring, degassing} 
\index{topics}{Mixing}
\index{topics}{Stirring}
%------------------------------------------------------------------------------
%--------------------------------------------------------------------

\begin{scriptsize}
\begin{itemize}
\item[\nineteeneightyfour] 
\textcite{olyb84} \citetitle{olyb84}\\
\item[\nineteenninety] 
\textcite{ketu90} \citetitle{ketu90}\\
\item[\nineteenninety] 
\textcite{davi90} \citetitle{davi90}\\
\item[\nineteenninetysix] 
\textcite{pelt96} \citetitle{pelt96}\\
\item[\nineteenninetynine] 
\textcite{cori99} \citetitle{cori99}\\
\item[\twothousandone] 
\textcite{huke01} \citetitle{huke01}\\
\item[\twothousandtwo] 
\textcite{vahb02} \citetitle{vahb02}\\
\item[\twothousandthree] 
\textcite{fasa03} \citetitle{fasa03}\\
\textcite{vabh03} \citetitle{vabh03}\\
\item[\twothousandfive] 
\textcite{colt05} \citetitle{colt05}\\
\item[\twothousandseven] 
\textcite{gogc07} \citetitle{gogc07}\\
\textcite{nake07} \citetitle{nake07}\\
\textcite{vabh07} \citetitle{vabh07}\\
\item[\twothousandeleven] 
\textcite{lemj11} \citetitle{lemj11}\\
\textcite{saad11} \citetitle{saad11}\\
\item[\twothousandeighteen] 
\textcite{onzh18} \citetitle{onzh18}\\
\end{itemize}
\end{scriptsize}

%--------------------------------------------------------------------
%------------------------------------------------------------------------------
\subsection{Obduction, ophiolites}
%------------------------------------------------------------------------------
%--------------------------------------------------------------------
\index{topics}{Obduction} \index{topics}{Ophiolites}

\begin{scriptsize}
\begin{itemize}
\item[\nineteenninety] 
\textcite{hack90} \citetitle{hack90}\\
\item[\nineteenninetyone] 
\textcite{hack91} \citetitle{hack91}\\
\item[\nineteenninetyseven] 
\textcite{rabh97} \citetitle{rabh97}\\
\item[\twothousand] 
\textcite{mokd00} \citetitle{mokd00}\\
\item[\twothousandfourteen] 
\textcite{agzf14} \citetitle{agzf14}\\
\item[\twothousandsixteen] 
\textcite{duay16} \citetitle{duay16}\\
\item[\twothousandtwenty] 
\textcite{rohb20} \citetitle{rohb20}\\
\item[\twothousandtwentyone] 
\textcite{pody21} \citetitle{pody21}\\
\end{itemize}
\end{scriptsize}

%--------------------------------------------------------------------
%--------------------------------------------------------------------
\subsection{Oceanic Lithosphere}
%--------------------------------------------------------------------
%--------------------------------------------------------------------
\index{topics}{Oceanic Lithosphere}

\begin{scriptsize}
\begin{itemize}
\item[\nineteenseventysix] Schubert \etal \cite{scfy76}\\
\item[\nineteenseventyseven] De Bremaecker \cite{debr77}\\
\item[\nineteeneightythree] 
\textcite{cobe83} \citetitle{cobe83}\\
\item[\nineteeneightyfour] Yuen \& Fleitout \cite{yufl84}\\
\item[\nineteeneightyeight] Morgan \& Forsyth \cite{mofo88}\\
\item[\nineteenninety] Ogawa \cite{ogaw90} \\
\item[\twothousand] Tetzlaff \& Schmeling \cite{tesc00}\\
\item[\twothousandone] Kaban \& Schwintzer \cite{kasc01}\\
\item[\nineteeneightyfour] Fleitout \& Yuen \cite{flyu84} \\
\item[\nineteenninetyeight] Buck \& Poliakov \cite{bupo98}\\
\item[\twothousandseven] 
\textcite{afrf07} \citetitle{afrf07}
\textcite{kore07} \citetitle{kore07}
\textcite{macl07} \citetitle{macl07}
\item[\twothousandeight] 
\textcite{chgu08} \citetitle{chgu08}
\item[\twothousandtwelve] 
\textcite{trub12} \citetitle{trub12} 
\item[\twothousandsixteen]  
\textcite{koko16} \citetitle{koko16}
\item[\twothousandeighteen] 
\textcite{rihc18} \citetitle{rihc18}
\end{itemize}
\end{scriptsize}

%--------------------------------------------------------------------
%--------------------------------------------------------------------
\subsection{Onset of convection}
%--------------------------------------------------------------------
%--------------------------------------------------------------------
\index{topics}{Onset of convection}

\begin{scriptsize}
\begin{itemize}
\item[\nineteeneightytwo] 
\textcite{homc82} \citetitle{homc82}\\
\item[\nineteenninety] 
\textcite{sope90} \citetitle{sope90}\\
\item[\twothousand] 
\textcite{scth00} \citetitle{scth00}\\
\item[\twothousandsix] 
\textcite{soba06} \citetitle{soba06}\\
\item[\twothousandtwo] 
\textcite{kojo02} \citetitle{kojo02}\\
\item[\twothousandtwo] 
\textcite{kojo03} \citetitle{kojo03}\\
\item[\twothousandseven] 
\textcite{soba07} \citetitle{soba07}\\
\item[\twothousandfifteen] 
\textcite{kamo15} \citetitle{kamo15}\\
\end{itemize}
\end{scriptsize}

%--------------------------------------------------------------------
%------------------------------------------------------------------------------
\subsection{Plate motion and mantle, plate tectonic reconstruction}
%------------------------------------------------------------------------------
%--------------------------------------------------------------------
\index{topics}{(Absolute) Plate Motion}
\index{topics}{Plate Tectonics Reconstruction}
\index{topics}{Plate Kinematics}
\index{topics}{True Polar Wander}

\nineteensixtysix: Wilson \cite{wils66}\\
\nineteensixtyseven: McKenzie \& Parker \cite{mcpa67}\\
\nineteensixtyeight: Isacks \etal \cite{isos68} \\ 
\nineteenseventythree: McKenzie \& Selater \cite{mcse73}\\
\nineteenseventyfour: \cite{sosl74}\\
\nineteenseventyfive: \cite{harp75}\\
\nineteenninety: \cite{dega90}\\
\nineteenninetytwo: \cite{zieg92a}, Gordon \& Stein \cite{gost92}\\
\nineteenninetyfour: \cite{guto94}\\
\nineteenninetyseven: \cite{wean97b}\\
\nineteenninetyeight: Zhong \etal \cite{zhgm98}, Lithgow-Bertelloni \& Richards \cite{liri98}\\
\nineteenninetynine: \cite{ribr99}\\
\twothousandone: \cite{yohk01}\\
\twothousandtwo: \cite{stoc02}\\
\twothousandthree: \cite{evan03}\cite{reta03}\\
\twothousandseven: \cite{zhzl07}\\
\twothousandnine: \cite{lizh09}\cite{vasv09}\cite{iabu09}\cite{scbs09}\\
\twothousandten: \cite{stto10}\cite{dega10}\\
\twothousandtwelve: \cite{huss12}\cite{gutz12}\cite{qumm12}\cite{holr12}\cite{dost12}\cite{shbs12}\\
\twothousandthirteen: \cite{mosq13}\cite{cost13}\\
\twothousandfourteen: Rudoplph \& Zhong \cite{ruzh14} \\
\twothousandfifteen: Yoshida \& Hamano \cite{yoha15}\\
\twothousandsixteen: \cite{pric16}\\
\twothousandseventeen: Stotz \etal \cite{stid17}\\
\twothousandnineteen: Tetley \etal \cite{tewg19}, Wessel \& Conrad \cite{weco19}, 
                      Flament \cite{flam19}\\
\twothousandtwenty: Semple \& Lenardic \cite{sele20}\\
\twothousandtwentyone: 
\textcite{cafm21},
\textcite{atco21}



%--------------------------------------------------------------------
%------------------------------------------------------------------------------
\subsection{Plume dynamics}
\index{topics}{Plume Dynamics}
%------------------------------------------------------------------------------
%--------------------------------------------------------------------

\begin{scriptsize}
\begin{itemize}
\item[\nineteenseventyone] Morgan \cite{morg71}
\item[\nineteenseventythree] Tozer \cite{toze73}
\item[\nineteenseventyfive] Parmentier \etal \cite{patt75}
\item[\nineteenseventyseven] Holden \& Vogt \cite{hovo77}
\item[\nineteeneighty] Yuen \& Peltier \cite{yupe80}
\item[\nineteeneightyseven] Zhao \& Yuen \cite{zhyu87}, Ribe \& Smooke \cite{rism87}
\item[\nineteenninety] Davies \cite{davi90}
\item[\nineteenninetyone] Kellogg \cite{kell91}, Griffiths \& Campbell \cite{grca91b}
\item[\nineteenninety] Griffiths \& Campbell \cite{grca90}
\item[\nineteenninetythree] Kellogg \& King \cite{keki93}, Malevsky \& Yuen \cite{mayu93}
\item[\nineteenninetyfour] Nakakuki \etal \cite{nasf94}, Farnetani \& Richards \cite{fari94},
                           Lenardic \& Kaula \cite{leka94b}, Hansen \& Yuen \cite{hayu94},
                           Matyska \etal \cite{mamy94}
\item[\nineteenninetyfive] Farnetani \& Richards \cite{fari95}
\item[\nineteenninetysix] Leitch \etal \cite{lesy96} 
\item[\nineteenninetyseven] van Keken \cite{vank97}, Kellogg \& King\cite{keki97},
                            Larsen \etal \cite{laym97}, Larsen \& Yuen \cite{layu97,layu97b},
                            Manga \cite{mang97}, King \cite{king97} 
\item[\nineteenninetyeight] Thompson \& Tackley \cite{thta98}, Steinberger \& O'Connell \cite{stoc98}
\item[\nineteenninetynine] Larsen \etal \cite{lays99}
\item[\twothousand] Cserepes \& Yuen \cite{csyu00}, Brunet \& Yuen \cite{bryu00}
\item[\twothousandone] Lithgow-Bertelloni \cite{lirc01}
\item[\twothousandtwo] Farnetani \etal \cite{falt02}, Davaille \etal \cite{dagl02},
                       Ni \etal \cite{nitg02}, Tan \etal \cite{tagh02}
\item[\twothousandthree] Samuel \& Farnetani \cite{safa03}
\item[\twothousandfour] Goes \etal \cite{goch04}, Schubert \etal \cite{scmo04}, Lowman \etal \cite{lokg04},
                        Ke \& Solomatov \cite{keso04} 
\item[\twothousandfive] Tan \& Gurnis \cite{tagu05}, Bunge \cite{bung05}, Zhong \cite{zhon05}, 
                        Lin \& van Keken \cite{liva05}, Matyska \& Yuen \cite{mayu05}
\item[\twothousandsix] Ismail-Zadeh \etal \cite{isst06}, Lin \& van Keken \cite{liva06a,liva06b}, 
                       Zhong \cite{zhon06}, Mittelstaedt \& Tackley \cite{mita06},
                       Nolet \etal \cite{nokm06}, Quere \& Forte \cite{qufo06}, 
                       Ke \& Solomatov \cite{keso06}, Campbell \& Davies \cite{cada06}
\item[\twothousandseven] Yuen \etal \cite{yumh07}, Ogawa \cite{ogaw07}
\item[\twothousandeight] Lowman \etal \cite{logg08} 
\item[\twothousandnine] Vatteville \etal \cite{vavl09}, Bower \etal \cite{bogj09},
                        Farnetani \& Hofmann \cite{faho09}, Schuberth \etal \cite{scbs09b},
                        Leng \& Zhong \cite{lezh09}
\item[\twothousandeleven] Tosi \& Yuen \cite{toyu11}, Tan \etal \cite{talz11},
                          Burke \cite{burk11}, Meriaux \etal \cite{memm11}, 
                          Davaille \etal \cite{dalt11}, Trubitsyn \etal \cite{tree11},
\item[\twothousandtwelve] Vincent \etal{} \cite{viym12}
\item[\twothousandthirteen] Davaille \etal \cite{dagm13}, Massmeyer \etal \cite{madd13},
                            Anderson \cite{ande13}, van Keken \etal \cite{vadv13}, 
                            Bossmann \& van Keken \cite{bova13}
\item[\twothousandfourteen] Glisovic \& Forte \cite{glfo14} 
\item[\twothousandfifteen] Dannberg \& Sobolev \cite{daso15}, Hassan \etal \cite{hafg15}, 
                           Heron \etal \cite{hels15}
\item[\twothousandsixteen] Kiefer \& Li \cite{kili16}, Dannberg PhD thesis \cite{dannbergphd}, 
                           Jones \etal \cite{jodc16}, Shahnas \etal \cite{shpy16}
\item[\twothousandseventeen] \cite{moyu17}\cite{lizh17}
\item[\twothousandeighteen] Davaille \etal \cite{dacc18}, Trubitsyn \& Evseev \cite{trev18}, 
                            Zhang \& Li \cite{zhli18}, Mora \& Yuen \cite{moyu18}
\item[\twothousandnineteen] Arnould \etal \cite{argc19}, Li \& Zhong \cite{lizh19}
\item[\twothousandtwenty] G{\"u}lcher \etal \cite{gugm20}, Ribe \etal \cite{rits20},
                          Heyn \etal \cite{hect20b}
\item[\twothousandtwentyone] Koppers \etal \cite{kobj21}, Xiang \etal \cite{xiwk21}
\end{itemize}
\end{scriptsize}

%------------------------------------------------------------------------------
%------------------------------------------------------------------------------
\subsection{Plume-Lithosphere interaction, LIP, hotspots}
\index{topics}{Plume-Lithosphere Interaction}
\index{topics}{Large Igneous Provinces}
\index{topics}{Hotspots}
%------------------------------------------------------------------------------
%------------------------------------------------------------------------------

\begin{scriptsize}
\begin{itemize}
\item[\nineteenninety] Davies \cite{davi90}
\item[\nineteenninetyone] Griffiths \& Campbell \cite{grca91}
\item[\nineteenninetytwo] Hill \etal \cite{hicd92}, Campbell \& Griffiths \cite{cagr92}, 
                          Saunders \etal \cite{sask92}
\item[\nineteenninetyfour] Ribe \& Christensen \cite{rich94}, Farnetani \& Richards \cite{fari94},
                           Ribe \& de Valpine \cite{ride94}, Davies \cite{davi94}
\item[\nineteenninetyfive] White \& McKenzie \cite{whmc95}, Farnetani \& Richards \cite{fari95},
                           Ribe \etal \cite{rict95}
\item[\nineteenninetysix] Zhong \etal \cite{zhgm96}, Ribe \cite{ribe96}
\item[\nineteenninetyeight] Moore \etal \cite{most98}, Ribe \& Delattre \cite{ride98}
\item[\nineteenninetynine] Moore \etal \cite{most99}, Sheth \cite{shet99},
                           Bijwaard \& Spakman \cite{bisp99}
\item[\twothousand] Lowry \etal \cite{lors00}
\item[\twothousandone] Vasilyev \etal \cite{vapy01}
\item[\twothousandtwo] Foulger \cite{foul02}
\item[\twothousandthree] van Hunen \& Zhong \cite{vazh03}
\item[\twothousandfour] Yoshida \& Ogawa \cite{yoog04}
\item[\twothousandfive] Burov \& Guillou-Frottier \cite{bugu05}, Farnetani \& Samuel \cite{fasa05}, 
                        Yoshida \& Ogawa \cite{yoog05}, Campbell \cite{camp05}
\item[\twothousandsix] Davies \& Bunge \cite{dabu06}, Thoraval \etal \cite{thtd06}
\item[\twothousandseven] Steiner \& Conrad \cite{stco07}
\item[\twothousandeight] Ueda \etal \cite{uegs08}, Sleep \cite{slee08}
\item[\twothousandnine] Burov \& Cloetingh \cite{bucl09}, Zhu \etal \cite{zhgy09},
                        Ballmer \etal \cite{baiv10}, Tarduno \etal \cite{tabs09}
                        Manea \etal \cite{maml09}
\item[\twothousandten] Faccenna \etal \cite{fabl10}, Leng \& Zhong \cite{lezh10}
\item[\twothousandeleven] Sobolev \etal \cite{sosk11}, van Hinsbergen \etal \cite{vasd11},
                          Koppers \cite{kopp11}
\item[\twothousandtwelve] Husson \& Conrad \cite{huco12}, Guillou-Frottier \etal \cite{gubc12},
                          Betts \etal \cite{bemm12}
\item[\twothousandthirteen] Brune \etal \cite{brps13}
\item[\twothousandfourteen] Burov \& Gerya \cite{buge14}, Gerya \cite{gery14b},
                            Buiter \& Torsvik \cite{buto14}, Buiter \cite{buit14},
                            Lee \& Lim \cite{leli14}, Agrusta \etal \cite{agat13}
\item[\twothousandfifteen] Betts \etal \cite{bemm15}, Gerya \etal \cite{gesb15},
                           Koptev \etal \cite{kocb15}, Meriaux \etal \cite{meds15},
                           Lim \& Lee \cite{lile15}, Meriaux \etal \cite{medd15},
                           French \& Romanowicz \cite{frro15}
\item[\twothousandsixteen] Fischer \& Gerya \cite{fige16}, Gassmoeller \etal \cite{gadb16},
                           Koptev \etal \cite{kobc16}
\item[\twothousandseventeen] Barnett-Moore \etal \cite{bahf17}, Bredow \etal \cite{brsg17},
                             Beniest \etal \cite{bekb17}, Koptev \etal \cite{kocb17},
                             Eguchi \etal \cite{egim17}
\item[\twothousandeighteen] Dannberg \& Gassm\"oller \cite{daga18}, Francois \etal \cite{frkc18},
                            Friedrich \etal \cite{frbr18}, Gorczyk \etal \cite{gomb18}
\item[\twothousandnineteen] Koptev \etal \cite{kobg19}, Steinberger \etal \cite{stbl19},
                            Bono \etal \cite{botb19}
\item[\twothousandtwenty] Baes \etal \cite{basg20,basg20b}, Dang \etal \cite{dazl20},
                          Piccolo \etal \cite{pikw20}
\item[\twothousandtwentyone] Cloetingh \etal \cite{clkk21}, Rodriguez \etal \cite{roac21},
                             van Hinsbergen \etal \cite{vasg21}, Baes \etal \cite{basg21},
                             Wang \& Li \cite{wali21}
\end{itemize}
\end{scriptsize}

%------------------------------------------------------------------------------
%------------------------------------------------------------------------------
\subsection{Porous media} 
\index{topics}{Porous Media}
%------------------------------------------------------------------------------
%------------------------------------------------------------------------------

\begin{scriptsize}
\begin{itemize}
\item[\nineteeneightysix] Scott \& Stevenson \cite{scst86}
\item[\nineteeneightyeight] Scott \cite{scot88}
\item[\nineteenninetythree] Spiegelman \cite{spie93}
\item[\twothousand] Schoofs \etal \cite{scth00b}
\item[\twothousandthirteen] Dymkova \& Gerya \cite{dyge13}
\item[\twothousandnineteen] Eichheimer \etal \cite{eitp19}
\item[\twothousandtwenty] Eichheimer \etal \cite{eitf20}
\end{itemize}
\end{scriptsize}

%------------------------------------------------------------------------------
%------------------------------------------------------------------------------
\subsection{Precambrian tectonics}
\index{topics}{Precambrian Tectonics}
%------------------------------------------------------------------------------
%------------------------------------------------------------------------------

\begin{scriptsize}
\begin{itemize}
\item[\nineteenninetyfour] 
\textcite{guto94} \citetitle{guto94}\\
\item[\twothousandthree] 
\textcite{wemv03} \citetitle{wemv03}\\
\item[\twothousandten] 
\textcite{sigb10} \citetitle{sigb10}\\
\item[\twothousandeleven] 
\textcite{pege11} \citetitle{pege11}\\
\item[\twothousandfourteen] 
\textcite{gery14} \citetitle{gery14}\\
\textcite{gagb14} \citetitle{gagb14}\\
\textcite{sigb14} \citetitle{sigb14}\\
\item[\twothousandtwenty] 
\textcite{poyd20} \citetitle{poyd20}\\
\end{itemize}
\end{scriptsize}

%------------------------------------------------------------------------------
%------------------------------------------------------------------------------
\subsection{Preconditioner business}
%------------------------------------------------------------------------------
%------------------------------------------------------------------------------

\begin{scriptsize}
\cite{benz02}
\cite{bewa08}
\cite{urvs08}
\end{scriptsize}

%------------------------------------------------------------------------------
%------------------------------------------------------------------------------
\subsection{Reservoir modelling}
\index{topics}{Reservoir Modelling}
%------------------------------------------------------------------------------
%------------------------------------------------------------------------------

\begin{scriptsize}
\twothousandthirteen: \textcite{orwa13}
\end{scriptsize}

%--------------------------------------------------------------------
\subsection{Regenauer-Lieb}
%--------------------------------------------------------------------

{\scriptsize
\twothousand: \cite{reyu98}\\
\twothousand: \cite{reyu00}\\
\twothousandthree: \cite{reyu03}\\
\twothousandfour: \cite{reyu04}\\
\twothousandsix: \cite{rehy06}\cite{rewr06}\\
\twothousandnine: \cite{reps09}\\
\twothousandthirteen: \cite{revp13}
}

%------------------------------------------------------------------------------
%------------------------------------------------------------------------------
\subsection{Restoration, Dynamic Reverse Modelling, Inversion tectonics}
\index{topics}{Restoration}
\index{topics}{Dynamic Reverse Modelling}
\index{topics}{Inversion Tectonics}
%------------------------------------------------------------------------------
%------------------------------------------------------------------------------

\begin{scriptsize}
\begin{itemize}
\item[\twothousandone] 
\textcite{istv01} \citetitle{istv01}\\
\item[\twothousandfour] 
\textcite{istt04} \citetitle{istt04}\\
\item[\twothousandfive] 
\textcite{koma05} \citetitle{koma05}\\
\item[\twothousandtwelve] 
\textcite{lofg12} \citetitle{lofg12}\\
\item[\twothousandeighteen] 
\textcite{lojm18} \citetitle{lojm18}\\
\item[\twothousandtwenty] 
\textcite{sctc20} \citetitle{sctc20}\\
\textcite{taas20} \citetitle{taas20}\\
\end{itemize}
\end{scriptsize}

%--------------------------------------------------------------------
%------------------------------------------------------------------------------
\subsection{Rheology, material parameters, rock mechanics}
\index{topics}{Rheology}
%--------------------------------------------------------------------
%------------------------------------------------------------------------------

\begin{scriptsize}
\begin{itemize}
\item[1951] 
\textcite{druc51}\citetitle{druc51}\\
\textcite{hafn51}\citetitle{hafn51}\\
\item[1952] 
\textcite{drpr52}\citetitle{drpr52}\\
\item[\nineteensixtyeight] 
\textcite{byer68} \citetitle{byer68}\\
\item[\nineteensixtynine] 
\textcite{hand69} \citetitle{hand69}\\
\item[\nineteenseventytwo] 
\textcite{carr72} \citetitle{carr72}\\
\item[\nineteenseventyfour] 
\textcite{kogo74} \citetitle{kogo74}\\
\item[\nineteenseventynine] 
\textcite{goev79} \citetitle{goev79}\\
\textcite{evgo79} \citetitle{evgo79}\\
\item[\nineteeneighty] 
\textcite{brko80} \citetitle{brko80}\\
\item[\nineteeneightyone] 
\textcite{delo81} \citetitle{delo81}\\
\item[\nineteeneightyfour] 
\textcite{rafi84} \citetitle{rafi84}\\
\textcite{chpa84} \citetitle{chpa84}\\
\textcite{vede84} \citetitle{vede84}\\
\item[\nineteeneightysix] 
\textcite{kapf86} \citetitle{kapf86}\\
\item[\nineteeneightyseven] 
\textcite{kikr87} \citetitle{kikr87}\\ 
\textcite{ramu87} \citetitle{ramu87}\\
\textcite{cats87} \citetitle{cats87}\\
\item[\nineteenninety] 
\textcite{wica90} \citetitle{wica90}\\
\item[\nineteenninetytwo] 
\textcite{bako92} \citetitle{bako92}\\
\textcite{chbo92} \citetitle{chbo92}\\
\textcite{kali92} \citetitle{kali92}\\
\textcite{kohl92} \citetitle{kohl92}\\
\item[\nineteenninetythree] 
\textcite{kawu93} \citetitle{kawu93}\\
\item[\nineteenninetyfour] 
\textcite{fran94} \citetitle{fran94}\\
\item[\nineteenninetyfive] 
\textcite{koem95} \citetitle{koem95}\\
\textcite{gltu95} \citetitle{gltu95}\\
\item[\nineteenninetysix] 
\textcite{wasd96} \citetitle{wasd96}\\
\textcite{hiko96} \citetitle{hiko96}\\
\item[\nineteenninetyseven] 
\textcite{eshe97a} \citetitle{eshe97a}\\
\textcite{eshe97b} \citetitle{eshe97b}\\
\item[\nineteenninetyeight] 
\textcite{copo98} \citetitle{copo98}\\
\textcite{mazk98} \citetitle{mazk98}\\
\item[\nineteenninetynine] 
\textcite{kayk99} \citetitle{kayk99}\\
\item[\twothousand] 
\textcite{rydr00} \citetitle{rydr00}\\ 
\textcite{rana00} \citetitle{rana00}\\ 
\textcite{meko00a} \citetitle{meko00a}\\
\textcite{meko00b} \citetitle{meko00b}\\
\item[\twothousandone] 
\textcite{lova01} \citetitle{lova01}\\ 
\textcite{kary01} \citetitle{kary01}\\
\item[\twothousandtwo] 
\textcite{hirt02} \citetitle{hirt02}\\
\item[\twothousandthree] 
\textcite{hiko03} \citetitle{hiko03}\\
\textcite{kaju03} \citetitle{kaju03}\\
\textcite{mohi03} \citetitle{mohi03}\\
\item[\twothousandfive] 
\textcite{didr05} \citetitle{didr05}\\
\textcite{drur05} \citetitle{drur05}\\
\item[\twothousandsix] 
\textcite{rygw06} \citetitle{rygw06}\\
\textcite{buwa06} \citetitle{buwa06}\\
\textcite{momu06} \citetitle{momu06}\\
\textcite{liwr06} \citetitle{liwr06}\\
\item[\twothousandseven] 
\textcite{hirw07} \citetitle{hirw07}\\
\textcite{kohl07} \citetitle{kohl07}\\
\textcite{faja07} \citetitle{faja07}\\
\item[\twothousandeight] 
\textcite{lemm08} \citetitle{lemm08}\\
\textcite{budr08} \citetitle{budr08}\\
\textcite{koka08} \citetitle{koka08}\\
\textcite{gird08} \citetitle{gird08}\\
\item[\twothousandnine] 
\textcite{kayk09} \citetitle{kayk09}\\
\textcite{kako09} \citetitle{kako09}\\
\item[\twothousandeleven] 
\textcite{lell11} \citetitle{lell11}\\
\textcite{kemk11} \citetitle{kemk11}\\
\textcite{hazk11} \citetitle{hazk11}\\
\item[\twothousandtwelve] 
\textcite{reyn12} \citetitle{reyn12}\\
\item[\twothousandthirteen] 
\textcite{lepo13} \citetitle{lepo13}\\ 
\textcite{miam13} \citetitle{miam13}\\ 
\textcite{mont13} \citetitle{mont13}\\
\item[\twothousandfourteen] 
\textcite{codb14} \citetitle{codb14}\\
\item[\twothousandfifteen] 
\textcite{chpe15} \citetitle{chpe15}\\ 
\textcite{ohkh15} \citetitle{ohkh15}\\
\item[\twothousandseventeen] 
\textcite{bocc17} \citetitle{bocc17}\\
\item[\twothousandnineteen] 
\textcite{rejv19} \citetitle{rejv19}\\ 
\textcite{hakt19} \citetitle{hakt19}\\
\textcite{gocg19} \citetitle{gocg19}\\
\end{itemize}
\end{scriptsize}

%--------------------------------------------------------------------
\subsection{Rifting, seafloor spreading, mid-ocean ridges, pull-apart basins, extension}
\index{topics}{Rifting} 
\index{topics}{Seafloor spreading} 
\index{topics}{Extension}
\index{topics}{Mid-Ocean Ridge}
\index{topics}{Ocean floor}
%--------------------------------------------------------------------

{\color{red} this should be split into oceanic, continental, 2D, 3D ...}
add oceanic transforms as separate topic?

\begin{scriptsize}
\begin{itemize}
\item[\nineteensixtyeight] 
\textcite{lepi68} \citetitle{lepi68}\\
\item[\nineteenseventytwo] 
\textcite{lath72}\citetitle{lath72}\\
\item[\nineteenseventythree] 
\textcite{froi73} \citetitle{froi73}\\
\item[\nineteenseventyseven] 
\textcite{pasc77} \citetitle{pasc77}\\
\item[\nineteenseventyeight] 
\textcite{stei78} \citetitle{stei78}\\
\textcite{mcke78} \citetitle{mcke78}\\
\item[\nineteeneighty] 
\textcite{bran80} \citetitle{bran80}\\
\textcite{roke80} \citetitle{roke80}\\
\item[\nineteeneightytwo] 
\textcite{bekb82} \citetitle{bekb82}\\
\item[\nineteeneightythree] 
\textcite{engl83} \citetitle{engl83}\\
\item[\nineteeneightyfour] 
\textcite{poay84} \citetitle{poay84}\\
\item[\nineteeneightyfive] 
\textcite{bosw85} \citetitle{bosw85}\\
\item[\nineteeneightysix] 
\textcite{hoen86b} \citetitle{hoen86b}\\
\textcite{zupf86} \citetitle{zupf86} \\
\textcite{zupa86} \citetitle{zupa86} \\
\textcite{mofr86} \citetitle{mofr86}\\
\textcite{mcke86} \citetitle{mcke86} \\
\textcite{buck86} \citetitle{buck86}\\
\item[\nineteeneightyseven] 
\textcite{spmc87} \citetitle{spmc87} \\
\textcite{brbe87} \citetitle{brbe87}\\
\item[\nineteeneightyeight] 
\textcite{bums88} \citetitle{bums88}\\
\textcite{ribe88b} \citetitle{ribe88b}\\
\item[\nineteeneightynine] 
\textcite{mewi89} \citetitle{mewi89} \\
\textcite{brbe89} \citetitle{brbe89}\\
\textcite{brbe89b} \citetitle{brbe89b}\\
\textcite{brbe89c} \citetitle{brbe89c}\\
\textcite{ismb89} \citetitle{ismb89} \\
\textcite{soen89} \citetitle{soen89}\\
\item[\nineteenninety] 
\textcite{fara90} \citetitle{fara90}\\
\textcite{lipa90} \citetitle{lipa90}\\
\textcite{mccl90} \citetitle{mccl90}\\
\textcite{chmo90} \citetitle{chmo90}\\
\textcite{chmo90b} \citetitle{chmo90b}\\
\item[\nineteenninetyone] 
\textcite{trbr91} \citetitle{trbr91}\\
\textcite{buck91} \citetitle{buck91}\\
\item[\nineteenninetytwo] 
\textcite{zieg92b} \citetitle{zieg92b}\\
\textcite{egan92} \citetitle{egan92}\\
\textcite{chld92} \citetitle{chld92}\\
\item[\nineteenninetythree] 
\textcite{gowo93} \citetitle{gowo93}\\
\item[\nineteenninetyfour] 
\textcite{trca94} \citetitle{trca94}\\
\textcite{jhpp94} \citetitle{jhpp94}\\
\textcite{popy94} \citetitle{popy94}\\
\item[\nineteenninetyfive] 
\textcite{gowo95} \citetitle{gowo95}\\
\textcite{katl95} \citetitle{katl95}\\
\item[\nineteenninetysix] 
\textcite{dusa96} \citetitle{dusa96}\\
\textcite{beda96} \citetitle{beda96}\\
\textcite{mada96} \citetitle{mada96}\\
\item[\nineteenninetyeight] 
\textcite{rafm98} \citetitle{rafm98}\\
\item[\nineteenninetynine] 
\textcite{brun99} \citetitle{brun99}\\
\textcite{bulp99} \citetitle{bulp99}\\
\textcite{gowo99} \citetitle{gowo99}\\
\item[\twothousand] 
\textcite{mime00} \citetitle{mime00}\\
\textcite{scth00} \citetitle{scth00}\\
\item[\twothousandone] 
\textcite{hupc01} \citetitle{hupc01}\\ 
\textcite{hupc01b} \citetitle{hupc01b}\\
\textcite{frbr01} \citetitle{frbr01}\\ 
\textcite{frnb01a} \citetitle{frnb01a}\\
\textcite{frnb01b} \citetitle{frnb01b}\\
\item[\twothousandtwo] 
\textcite{hube02} \citetitle{hube02}\\ 
\textcite{hani02} \citetitle{hani02}\\
\textcite{dabm02} \citetitle{dabm02}\\
\textcite{vacl02} \citetitle{vacl02}\\
\textcite{belz02} \citetitle{belz02}\\
\textcite{hupc02} \citetitle{hupc02}\\
\textcite{hube02b} \citetitle{hube02b}\\
\textcite{labu02} \citetitle{labu02}\\

\item[\twothousandthree] 
\textcite{hube03} \citetitle{hube03}\\ 
\textcite{hani03} \citetitle{hani03}\\
\textcite{covb03} \citetitle{covb03}\\
\textcite{wibm03} \citetitle{wibm03}\\

\item[\twothousandfour] 
\textcite{hier04} \citetitle{hier04}\\
\textcite{sees04} \citetitle{sees04}\\

\item[\twothousandfive] 
\textcite{hubb05} \citetitle{hubb05}\\ 
\textcite{coub05} \citetitle{coub05}\\
\textcite{vanw05} \citetitle{vanw05}\\
\textcite{vabl05} \citetitle{vabl05}\\

\item[\twothousandsix] 
\textcite{tibs06} \citetitle{tibs06}\\ 
\textcite{coma06} \citetitle{coma06}\\
\textcite{crwy06} \citetitle{crwy06}\\
\textcite{peso06} \citetitle{peso06}\\
\textcite{lemm06} \citetitle{lemm06}\\
\textcite{malm06} \citetitle{malm06}\\
\textcite{crms06} \citetitle{crms06}\\

\item[\twothousandseven] 
\textcite{huha07} \citetitle{huha07}\\ 
\textcite{macl07} \citetitle{macl07}\\
\textcite{vabl07} \citetitle{vabl07}\\
\textcite{dyrm07} \citetitle{dyrm07}\\
\textcite{hube07} \citetitle{hube07}\\
\textcite{buto07} \citetitle{buto07}\\
\textcite{socb07} \citetitle{socb07}\\
\textcite{werr07} \citetitle{werr07}\\
\textcite{nabu07} \citetitle{nabu07}\\
\item[\twothousandeight] 
\textcite{cort08} \citetitle{cort08}\\ 
\textcite{gumb08} \citetitle{gumb08}\\
\textcite{buhb08} \citetitle{buhb08}\\
\textcite{hube08} \citetitle{hube08}\\
\textcite{peso08} \citetitle{peso08}\\
\textcite{rerw08} \citetitle{rerw08}\\
\textcite{codh08} \citetitle{codh08}\\
\item[\twothousandnine] 
\textcite{agcz09} \citetitle{agcz09}\\
\textcite{kekj09} \citetitle{kekj09}\\
\textcite{sihb09} \citetitle{sihb09}\\
\item[\twothousandten] 
\textcite{aubh10} \citetitle{aubh10}\\
\textcite{fosr10} \citetitle{fosr10}\\
\textcite{gerya2010} \citetitle{gerya2010}\\
\item[\twothousandeleven] 
\textcite{alht11} \citetitle{alht11}\\
\textcite{ellw11} \citetitle{ellw11}\\
\textcite{hube11} \citetitle{hube11}\\
\item[\twothousandtwelve] 
\textcite{alht12} \citetitle{alht12}\\
\textcite{brps12} \citetitle{brps12}\\
\textcite{bein12} \citetitle{bein12}\\
\item[\twothousandthirteen] 
\textcite{alhf13} \citetitle{alhf13}\\ 
\textcite{brau13} \citetitle{brau13}\\
\textcite{chbe13} \citetitle{chbe13}\\
\textcite{knak13} \citetitle{knak13}\\
\textcite{kern13} \citetitle{kern13}\\
\textcite{mipf13} \citetitle{mipf13}\\
\textcite{wabd13} \citetitle{wabd13}\\
\textcite{ligw13} \citetitle{ligw13}\\
\textcite{gery13c} \citetitle{gery13c}\\
\textcite{gery13} \citetitle{gery13}\\
\textcite{ebvk13} \citetitle{ebvk13}\\
\textcite{beha13} \citetitle{beha13}\\
\item[\twothousandfourteen] 
\textcite{hebr14} \citetitle{hebr14}\\ 
\textcite{lige14} \citetitle{lige14}\\
\textcite{lige14b} \citetitle{lige14b}\\
\textcite{brun14} \citetitle{brun14}\\
\textcite{kobf14} \citetitle{kobf14}\\
\textcite{ebva14} \citetitle{ebva14}\\
\textcite{puge14} \citetitle{puge14}\\
\textcite{hube14} \citetitle{hube14}\\
\textcite{gogu14} \citetitle{gogu14}\\
\textcite{cosb14} \citetitle{cosb14}\\
\textcite{pokb14} \citetitle{pokb14}\\
\item[\twothousandfifteen] 
\textcite{nabu15} \citetitle{nabu15}\\
\textcite{clbq15} \citetitle{clbq15}\\
\textcite{huyb15} \citetitle{huyb15}\\
\textcite{wulc15} \citetitle{wulc15}\\
\textcite{shmj15} \citetitle{shmj15}\\
\textcite{svlh15} \citetitle{svlh15}\\
\textcite{olbi15} \citetitle{olbi15}\\
\textcite{pean15} \citetitle{pean15}\\
\item[\twothousandsixteen] 
\textcite{olbm16} \citetitle{olbm16}\\
\textcite{jekm16} \citetitle{jekm16}\\
\textcite{zwsn16} \citetitle{zwsn16}\\
\textcite{jala16} \citetitle{jala16}\\
\item[\twothousandseventeen] 
\textcite{lemh17} \citetitle{lemh17}\\
\textcite{brcr17} \citetitle{brcr17}\\
\textcite{bekb17} \citetitle{bekb17}\\
\textcite{nabp17} \citetitle{nabp17}\\
\item[\twothousandeighteen] 
\textcite{chsm18} \citetitle{chsm18}\\
\textcite{brwm18} \citetitle{brwm18}\\
\textcite{brun18} \citetitle{brun18}\\
\textcite{tebu18} \citetitle{tebu18}\\
\textcite{jebu18} \citetitle{jebu18}\\
\textcite{sahf18} \citetitle{sahf18}\\
\textcite{pesn18} \citetitle{pesn18}\\
\textcite{mord18} \citetitle{mord18}\\
\textcite{webe18} \citetitle{webe18}\\
\textcite{webe18b} \citetitle{webe18b}\\
\textcite{gebu18} \citetitle{gebu18}\\
\textcite{marc18} \citetitle{marc18}\\
\textcite{bews18} \citetitle{bews18}\\
\item[\twothousandnineteen] 
\textcite{lisp19} \citetitle{lisp19}\\
\textcite{zwsb19} \citetitle{zwsb19}\\
\textcite{anpa19} \citetitle{anpa19}\\
\textcite{dual19} \citetitle{dual19}\\
\textcite{mocb19} \citetitle{mocb19}\\
\textcite{chmd19} \citetitle{chmd19}\\
\textcite{thhu19} \citetitle{thhu19}\\
\textcite{jala19} \citetitle{jala19}\\
\textcite{hooi19} \citetitle{hooi19}\\
\textcite{lapk19} \citetitle{lapk19}\\
\textcite{jolm19} \citetitle{jolm19}\\
\textcite{hepm19} \citetitle{hepm19}\\
\item[\twothousandtwenty] 
\textcite{niu20} \citetitle{niu20}\\ 
\textcite{cump20} \citetitle{cump20}\\ 
\textcite{pena20} \citetitle{pena20}\\
\textcite{ster20} \citetitle{ster20}\\
\textcite{fahm20} \citetitle{fahm20}\\
\textcite{siss20} \citetitle{siss20}\\
\textcite{zwsr20} \citetitle{zwsr20}\\
\textcite{glbs20} \citetitle{glbs20}\\
\textcite{lial20} \citetitle{lial20}\\
\textcite{duhm20} \citetitle{duhm20}\\
\textcite{nagb20} \citetitle{nagb20}\\
\textcite{jolm20} \citetitle{jolm20}\\
\textcite{chsm20} \citetitle{chsm20}\\
\textcite{grrm21} \citetitle{grrm21}\\
\textcite{yosy20b} \citetitle{yosy20b}\\
\item[\twothousandtwentyone] 
\textcite{kotr21} \citetitle{kotr21}\\
\textcite{lalt21} \citetitle{lalt21}\\
\textcite{hebg21} \citetitle{hebg21}\\
\textcite{nebg21} \citetitle{nebg21}\\
\textcite{qill21} \citetitle{qill21}\\
\textcite{luhu21} \citetitle{luhu21}\\
\textcite{gona21} \citetitle{gona21}\\
\textcite{manp21} \citetitle{manp21}\\
\end{itemize}
\end{scriptsize}

%--------------------------------------------------------------------
\subsection{Critical Wedges}
\index{topics}{Critical Wedge}
\index{topics}{Critical Taper}

\begin{scriptsize}
\begin{itemize}
\item[\nineteenninetyfour] 
\textcite{koon94}\citetitle{koon94}\\
\item[\twothousandsix] 
\textcite{rosw06}\citetitle{rosw06}\\
\item[\twothousandeight] 
\textcite{rowf08}\citetitle{rowf08}\\
\item[\twothousandthirteen] 
\textcite{cass13}\citetitle{cass13}\\
\end{itemize}
\end{scriptsize}

%--------------------------------------------------------------------
%--------------------------------------------------------------------
\subsection{Salt tectonics, Shale tectonics}
\index{topics}{Salt Tectonics}
\index{topics}{Shale Tectonics}
%--------------------------------------------------------------------

\begin{scriptsize}
\begin{itemize}
\item[\nineteenseventyeight] 
\textcite{woid78} \cite{woid78}\\
\item[\nineteenninetyone] 
\textcite{tars91} \cite{tars91}\\
\item[\nineteenninetytwo] 
\textcite{zaju92} \cite{zaju92}\\
\textcite{veja92} \cite{veja92}\\
\item[\nineteenninetythree]
\textcite{nabr93} \cite{nabr93}\\ 
\textcite{vasv93} \cite{vasv93}\\
\textcite{wejv93} \cite{wejv93}\\
\textcite{wein93} \cite{wein93}\\
\item[\nineteenninetysix] 
\textcite{maar96} \cite{maar96}\\
\item[\nineteenninetyeight] 
\textcite{giju98} \cite{giju98}\\
\item[\twothousandfour] 
\textcite{istt04} \cite{istt04}\\
\textcite{geim04} \cite{geim04}\\
\textcite{mcmg04} \cite{mcmg04}\\
\item[\twothousandfive] 
\textcite{gebi05} \cite{gebi05}\\
\item[\twothousandsix] 
\textcite{maqs06} \cite{maqs06}\\
\item[\twothousandseven] 
\textcite{huja07} \cite{huja07}\\
\textcite{maqs07} \cite{maqs07}\\
\item[\twothousandeight] 
\textcite{chks08} \cite{chks08}\\ 
\item[\twothousandnine] 
\textcite{grba09} \cite{grba09}\\
\textcite{hujs09} \cite{hujs09}\\ 
\item[\twothousandten] 
\textcite{albe10} \cite{albe10}\\
\textcite{albi10} \cite{albi10}\\
\textcite{inbe10} \cite{inbe10}\\
\textcite{inbe10b} \cite{inbe10b}\\ 
\textcite{albs10} \cite{albs10}\\
\item[\twothousandeleven] 
\textcite{brfo11} \cite{brfo11}\\
\item[\twothousandtwelve] 
\textcite{fejr12} \cite{fejr12}\\
\textcite{liqi12} \cite{liqi12}\\
\textcite{grbe12} \cite{grbe12}\\
\textcite{albe12} \cite{albe12}\\
\textcite{grbi12} \cite{grbi12}\\
\textcite{goib12} \cite{goib12}\\
\textcite{rukb12} \cite{rukb12}\\
\item[\twothousandthirteen] 
\textcite{gobi13} \cite{gobi13}\\
\textcite{nipc13} \cite{nipc13}\\
\textcite{wakk13} \cite{wakk13}\\
\item[\twothousandfourteen] 
\textcite{bakp14} \cite{bakp14}\\
\textcite{feka14a} \cite{feka14a}\\
\textcite{feka14b} \cite{feka14b}\\
\textcite{ghbu14} \cite{ghbu14}\\
\textcite{nifh14} \cite{nifh14}\\
\textcite{peel14} \cite{peel14}\\
\item[\twothousandfifteen] 
\textcite{feka15} \cite{feka15}\\
\textcite{cofk15} \cite{cofk15}\\
\item[\twothousandsixteen] 
\textcite{masg16} \cite{masg16}\\
\textcite{albe16} \cite{albe16}\\
\item[\twothousandseventeen] 
\textcite{grbe17} \cite{grbe17}\\
\textcite{henf17} \cite{henf17}\\
\item[\twothousandnineteen] 
\textcite{hadv19} \cite{hadv19}\\
\textcite{clcc19} \cite{clcc19}\\
\end{itemize}
\end{scriptsize}

%-------------------------------------------------------------------
%--------------------------------------------------------------------
\subsection{Sea Level evolution, GIA}
\index{topics}{Sea Level}
%--------------------------------------------------------------------

\begin{scriptsize}
\begin{itemize}
\item[\nineteenseventyeight] 
\textcite{pefc78} \citetitle{pefc78}\\
\item[\twothousandseven] 
\textcite{pazw07} \citetitle{pazw07}\\
\item[\twothousandnine] 
\textcite{cohu09} \citetitle{cohu09}\\
\item[\twothousandthirteen] 
\textcite{conr13} \citetitle{conr13}\\
\textcite{ivjw13} \citetitle{ivjw13}\\
\item[\twothousandfourteen] 
\textcite{larp14} \citetitle{larp14}\\
\item[\twothousandeighteen] 
\textcite{makv18} \citetitle{makv18}\\
\item[\twothousandtwenty]
\end{itemize}
\end{scriptsize}

%-------------------------------------------------------------------
%--------------------------------------------------------------------
\subsection{Segregated methods to solve the Stokes system}
%-------------------------------------------------------------------

\begin{scriptsize}
\cite{raju91}
\cite{haeh93}
\cite{leru95}
\cite{duto98}
\cite{wade03}
\cite{wade04}
\cite{utne08}
\end{scriptsize}

%-------------------------------------------------------------------
%--------------------------------------------------------------------
\subsection{Seismo-tectonics, subduction earthquakes}
\index{topics}{Seismo-tectonics}
%--------------------------------------------------------------------

\begin{scriptsize}
\begin{itemize}
\item[\nineteenninetyeight] Huc \etal \cite{huhc98}
\item[\twothousandthree] Bonini \etal \cite{bocs03}
\item[\twothousandtwelve] Wang \etal \cite{wahh12}
\item[\twothousandthirteen] van Dinther \etal \cite{vagd13a,vagd13b}, Mikhailov \cite{milp13},
                            Myhill \cite{myhi13}
\item[\twothousandfourteen] van Dinther \etal \cite{vamd14}
\item[\twothousandfifteen] Herrendorfer \etal \cite{hevg15}
\item[\twothousandeighteen] Govers \etal \cite{gofv18}, Herman \etal \cite{hefg18}, 
                            Herrendorfer \etal \cite{hegv18}, Dal Zilio \etal \cite{davg18}
\item[\twothousandnineteen] van Zelst \etal \cite{vawg19} van Zelst phd thesis \cite{vanzelst},
                            van Dinther \etal \cite{vakf19}
\item[\twothousandtwenty] Brizzi \etal \cite{brvf20}, Petrini \etal \cite{pegy20}, 
                          D'acquisto \etal \cite{dadm20}, Madden \etal \cite{mabb20} 
\item[\twothousandtwentyone] Jackson \etal \cite{jamp21}, Behr \etal \cite{begc21}
\end{itemize}
\end{scriptsize}

%--------------------------------------------------------------------
%--------------------------------------------------------------------
\subsection{Stagnant lid} 
\index{topics}{Stagnant Lid}
%--------------------------------------------------------------------

\begin{scriptsize}
\begin{itemize}
\item[\nineteenninetysix] Solomatov \& Moresi \cite{somo96}
\item[\nineteenninetyseven] Solomatov \& Moresi \cite{somo97}
\item[\nineteenninetyeight] Reese \etal \cite{resm98}
\item[\nineteenninetynine] Reese \etal \cite{resm99}, Reese \etal \cite{resb99}
\item[\twothousand] Solomatov \& Moresi \cite{somo00}
\item[\twothousandtwo] Reese \& Solomatov \cite{reso02}
\item[\twothousandfour] Freeman \etal \cite{frmm04}
\item[\twothousandfive] Reese \etal \cite{resb05}
\item[\twothousandnine] King \cite{king09}
\item[\twothousandten] Sramek \& Zhong \cite{srzh10}
\item[\twothousandeleven] Orth \& Solomatov \cite{orso11}
\item[\twothousandfourteen] Yao \etal \cite{yadl14}
\item[\twothousandsixteen] Wong \& Solomatov \cite{woso16b}, Crameri \& Tackley \cite{crta16}
\item[\twothousandseventeen] Patocka \etal \cite{pact17}
\end{itemize}
\end{scriptsize}

%--------------------------------------------------------------------
%--------------------------------------------------------------------
\subsection{Stream Function} 
\index{topics}{Stream Function}
%--------------------------------------------------------------------

\begin{scriptsize}
\noindent
\nineteeneightynine: Machetel \& yuen \cite{mayu89} \\
\nineteenninetysix: Larsen \etal \cite{laym96} \\
\end{scriptsize}

%--------------------------------------------------------------------
%--------------------------------------------------------------------
\subsection{Subduction} 
\index{topics}{Subduction}
%--------------------------------------------------------------------
This category should be subdivided into continental collision, subduction 2D \& 3D...

{\color{red} needs sorting: what are the major subtopics ? plate contact/trench? bending ? 
angle? } 

\begin{scriptsize}
\begin{itemize}
\item[\nineteenseventy] 
\textcite{mito70} \citetitle{mito70}\\
\item[\nineteenseventyeight] 
\textcite{haoc78} \citetitle{haoc78}\\
\textcite{yufs78} \citetitle{yufs78}\\
\item[\nineteeneighty] 
\textcite{mera80} \citetitle{mera80}\\
\item[\nineteeneightytwo] 
\textcite{crpi82} \citetitle{crpi82}\\
\item[\nineteeneightyfive] 
\textcite{thar85} \citetitle{thar85}\\
\item[\nineteeneightysix] 
\textcite{jarr86} \citetitle{jarr86}\\
\item[\nineteeneightyseven] 
\textcite{peac87b} \citetitle{peac87b}\\
\item[\nineteeneightyeight] 
\textcite{guha88} \citetitle{guha88}\\
\item[\nineteeneightynine] 
\textcite{boww89} \citetitle{boww89}\\
\textcite{mibj89} \citetitle{mibj89}\\
\textcite{hesw89} \citetitle{hesw89}\\
\item[\nineteenninety] 
\textcite{hstt90} \citetitle{hstt90}\\
\textcite{kiha90} \citetitle{kiha90}\\
\item[\nineteenninetytwo] 
\textcite{zhgu92} \citetitle{zhgu92}\\ 
\textcite{whbw92} \citetitle{whbw92}\\
\textcite{gurn92} \citetitle{gurn92}\\
\textcite{taoc92} \citetitle{taoc92}\\
\item[\nineteenninetythree] 
\textcite{jope93} \citetitle{jope93}\\
\textcite{dvnm93} \citetitle{dvnm93}\\
\textcite{wibf93} \citetitle{wibf93}\\
\textcite{shem93} \citetitle{shem93}\\
\item[\nineteenninetyfour] 
\textcite{zhgu94} \citetitle{zhgu94}\\
\textcite{wibe94} \citetitle{wibe94}\\
\textcite{wdbo94a} \citetitle{wdbo94a}\\
\textcite{wdbo94b} \citetitle{wdbo94b}\\
\textcite{bequ94} \citetitle{bequ94}\\
\textcite{gaha94} \citetitle{gaha94}\\
\item[\nineteenninetyfive] 
\textcite{masa95} \citetitle{masa95}\\
\item[\nineteenninetysix] 
\textcite{chri96} \citetitle{chri96}\\
\textcite{gisb96} \citetitle{gisb96}\\
\textcite{wabe96} \citetitle{wabe96}\\
\textcite{mipb96} \citetitle{mipb96}\\
\textcite{zhgu96} \citetitle{zhgu96}\\
\item[\nineteenninetyseven] 
\textcite{hajc97} \citetitle{hajc97}\\
\textcite{kisa97} \citetitle{kisa97}\\
\textcite{olwh97} \citetitle{olwh97}\\
\textcite{nesg97} \citetitle{nesg97}\\
\textcite{hogu97} \citetitle{hogu97}\\
\textcite{hajc97} \citetitle{hajc97}\\
\item[\nineteenninetyeight] 
\textcite{itki98} \citetitle{itki98}\\
\textcite{buwg98} \citetitle{buwg98}\\
\textcite{brmy98} \citetitle{brmy98}\\
\textcite{jabf98} \citetitle{jabf98}\\
\textcite{wabb98} \citetitle{wabb98}\\
\item[\nineteenninetynine] 
\textcite{hagu99} \citetitle{hagu99}\\
\textcite{befo99} \citetitle{befo99}\\
\textcite{bumo99} \citetitle{bumo99}\\
\textcite{roda99} \citetitle{roda99}\\
\textcite{elbp99} \citetitle{elbp99}\\
\textcite{scmr99} \citetitle{scmr99}\\
\textcite{elbe99} \citetitle{elbe99}\\
\textcite{beep99} \citetitle{beep99}\\
\textcite{nesb99} \citetitle{nesb99}\\
\textcite{coha99} \citetitle{coha99}\\
\item[\twothousand] 
\textcite{tesc00} \citetitle{tesc00}\\
\textcite{brky00} \citetitle{brky00}\\
\textcite{bemh00} \citetitle{bemh00}\\
\textcite{chlb00} \citetitle{chlb00}\\
\item[\twothousandone] 
\textcite{bujl01} \citetitle{bujl01}\\
\textcite{bugw01} \citetitle{bugw01}\\
\textcite{chys01} \citetitle{chys01}\\
\textcite{coha01} \citetitle{coha01}\\
\textcite{kary01} \citetitle{kary01}\\
\item[\twothousandtwo] 
\textcite{civv02} \citetitle{civv02}\\
\textcite{gesp02} \citetitle{gesp02}\\
\textcite{ster02} \citetitle{ster02}\\
\textcite{jabn02} \citetitle{jabn02}\\
\item[\twothousandthree] 
\textcite{refm03} \citetitle{refm03}\\
\textcite{fumr03} \citetitle{fumr03}\\
\textcite{gehd03} \citetitle{gehd03}\\
\textcite{bigs03} \citetitle{bigs03}\\
\item[\twothousandfour] 
\textcite{toba04} \citetitle{toba04}\\
\textcite{bocj04} \citetitle{bocj04}\\
\textcite{bejn04} \citetitle{bejn04}\\
\textcite{tobj04} \citetitle{tobj04}\\
\textcite{sche04} \citetitle{sche04}\\
\textcite{sche04b} \citetitle{sche04b}\\
\textcite{enwi04} \citetitle{enwi04}\\
\textcite{geys04} \citetitle{geys04}\\
\item[\twothousandfive] 
\textcite{jalo05} \citetitle{jalo05}\\
\textcite{lahb05} \citetitle{lahb05}\\
\textcite{gowo05} \citetitle{gowo05}\\
\textcite{enbs05} \citetitle{enbs05}\\
\textcite{artd05} \citetitle{artd05}\\
\textcite{gowo05} \citetitle{gowo05}\\
\textcite{mage05} \citetitle{mage05}\\
\textcite{stge05} \citetitle{stge05}\\
\textcite{sche05} \citetitle{sche05}\\
\textcite{lahb05} \citetitle{lahb05}\\
\item[\twothousandsix] 
\textcite{degw06} \citetitle{degw06}\\
\textcite{rohu06} \citetitle{rohu06}\\
\textcite{masr06} \citetitle{masr06}\\
\textcite{gest06} \citetitle{gest06}\\
\textcite{fump06} \citetitle{fump06}\\
\textcite{pibf06} \citetitle{pibf06}\\
\textcite{stfs06} \citetitle{stfs06}\\
\textcite{libi06} \citetitle{libi06}\\
\textcite{hapf06} \citetitle{hapf06}\\
\textcite{sobk06} \citetitle{sobk06}\\
\textcite{syab06} \citetitle{syab06}\\
\textcite{cuhy06} \citetitle{cuhy06}\\
\item[\twothousandseven] 
\textcite{tank07} \citetitle{tank07}\\ 
\textcite{artd07} \citetitle{artd07}\\
\textcite{yaab07} \citetitle{yaab07}\\
\textcite{cubh07} \citetitle{cubh07}\\
\textcite{civv07} \citetitle{civv07}\\
\textcite{masp07} \citetitle{masp07}\\
\textcite{camg07} \citetitle{camg07}\\
\textcite{scfs07} \citetitle{scfs07}\\
\textcite{gogg07} \citetitle{gogg07}\\
\textcite{gowg07} \citetitle{gowg07}\\
\textcite{magu07} \citetitle{magu07}\\
\textcite{moct07} \citetitle{moct07}\\
\textcite{onlm07} \citetitle{onlm07}\\
\textcite{lohd07} \citetitle{lohd07}\\
\textcite{zldf07} \citetitle{zldf07}\\
\textcite{bihi07} \citetitle{bihi07}\\
\item[\twothousandeight] 
\textcite{yaba08} \citetitle{yaba08}\\
\textcite{ozrs08} \citetitle{ozrs08}\\
\textcite{wabj08} \citetitle{wabj08}\\
\textcite{wabj08b} \citetitle{wabj08b}\\
\textcite{boht08a} \citetitle{boht08a}\\
\textcite{boht08b} \citetitle{boht08b}\\
\textcite{migb08} \citetitle{migb08}\\
\textcite{baso08} \citetitle{baso08}\\
\textcite{fagc08} \citetitle{fagc08}\\
\textcite{gecy08} \citetitle{gecy08}\\
\textcite{fufh08} \citetitle{fufh08}\\
\textcite{buya08} \citetitle{buya08}\\
\textcite{degw08} \citetitle{degw08}\\
\textcite{degw08b} \citetitle{degw08b}\\
\textcite{gepb08} \citetitle{gepb08}\\
\textcite{nigc08} \citetitle{nigc08}\\
\textcite{sebp08} \citetitle{sebp08}\\
\textcite{cuhb08} \citetitle{cuhb08}\\
\textcite{wuch08} \citetitle{wuch08}\\
\textcite{divf08} \citetitle{divf08}\\
\textcite{naht08} \citetitle{naht08}\\
\item[\twothousandnine] 
\textcite{yahb09} \citetitle{yahb09}\\
\textcite{bill09} \citetitle{bill09}\\
\textcite{fagb09} \citetitle{fagb09}\\
\textcite{bejb09} \citetitle{bejb09}\\
\textcite{kecw09} \citetitle{kecw09}\\
\textcite{gecm09} \citetitle{gecm09}\\
\textcite{gefc09} \citetitle{gefc09}\\
\textcite{famg09} \citetitle{famg09}\\
\textcite{lige09} \citetitle{lige09}\\
\textcite{moct09} \citetitle{moct09}\\
\textcite{lohb09} \citetitle{lohb09}\\
\textcite{befa09} \citetitle{befa09}\\
\textcite{agyj09} \citetitle{agyj09}\\
\textcite{yamb09} \citetitle{yamb09}\\
\textcite{huby09} \citetitle{huby09}\\
\item[\twothousandten] 
\textcite{hagr10} \citetitle{hagr10}\\ 
\textcite{lobh10} \citetitle{lobh10}\\
\textcite{mamb10} \citetitle{mamb10}\\
\textcite{camg10} \citetitle{camg10}\\
\textcite{casm10} \citetitle{casm10}\\
\textcite{ligb10} \citetitle{ligb10}\\
\textcite{stfc10} \citetitle{stfc10}\\
\textcite{moyb10} \citetitle{moyb10}\\
\textcite{zhst10} \citetitle{zhst10}\\
\textcite{qusp10} \citetitle{qusp10}\\
\textcite{moht10} \citetitle{moht10}\\
\textcite{leki10} \citetitle{leki10}\\
\textcite{sigb10} \citetitle{sigb10}\\
\textcite{stsf10} \citetitle{stsf10}\\
\textcite{syva10} \citetitle{syva10}\\
\textcite{nati10} \citetitle{nati10}\\
\item[\twothousandeleven] 
\textcite{lixg11} \citetitle{lixg11}\\ 
\textcite{list11} \citetitle{list11}\\
\textcite{bubj11} \citetitle{bubj11}\\ 
\textcite{bagw11b} \citetitle{bagw11b}\\
\textcite{cafz11} \citetitle{cafz11}\\ 
\textcite{geme11} \citetitle{geme11}\\
\textcite{qube11} \citetitle{qube11}\\ 
\textcite{blgg11} \citetitle{blgg11}\\
\textcite{gery11b} \citetitle{gery11b}\\ 
\textcite{leki11} \citetitle{leki11}\\
\textcite{scsf11} \citetitle{scsf11}\\
\textcite{gopc11} \citetitle{gopc11}\\
\textcite{gocm11} \citetitle{gocm11}\\
\item[\twothousandtwelve] 
\textcite{anwb12} \citetitle{anwb12}\\
\textcite{jahu12} \citetitle{jahu12}\\
\textcite{jabi12} \citetitle{jabi12}\\
\textcite{jabk12} \citetitle{jabk12}\\
\textcite{lixg12} \citetitle{lixg12}\\
\textcite{grpy12} \citetitle{grpy12}\\
\textcite{grpy12b} \citetitle{grpy12b}\\
\textcite{ronb12} \citetitle{ronb12}\\
\textcite{tebu12} \citetitle{tebu12}\\
\textcite{thka12} \citetitle{thka12}\\
\textcite{bova12} \citetitle{bova12}\\
\textcite{civs12} \citetitle{civs12}\\
\textcite{camo12} \citetitle{camo12}\\
\textcite{cafa12} \citetitle{cafa12}\\
\textcite{gebk12} \citetitle{gebk12}\\
\textcite{liri12} \citetitle{liri12}\\
\textcite{beva12} \citetitle{beva12}\\
\textcite{uegb12} \citetitle{uegb12}\\
\textcite{bija12} \citetitle{bija12}\\
\textcite{sigb12} \citetitle{sigb12}\\
\textcite{vogc12} \citetitle{vogc12}\\
\textcite{buqm12} \citetitle{buqm12}\\
\textcite{yoth12} \citetitle{yoth12}\\
\textcite{gigh12} \citetitle{gigh12}\\
\textcite{vakn12} \citetitle{vakn12}\\
\textcite{rosm12} \citetitle{rosm12}\\
\textcite{talv12} \citetitle{talv12}\\

\item[\twothousandthirteen]  
\textcite{nabg13} \citetitle{nabg13}\\ 
\textcite{hage13} \citetitle{hage13}\\ 
\textcite{moho13} \citetitle{moho13}\\ 
\textcite{ancv13} \citetitle{ancv13}\\ 
\textcite{namu13} \citetitle{namu13}\\ 
\textcite{yosh13} \citetitle{yosh13}\\ 
\textcite{zhgt13} \citetitle{zhgt13}\\ 
\textcite{lixg13} \citetitle{lixg13}\\ 
\textcite{jabr13} \citetitle{jabr13}\\ 
\textcite{izht13} \citetitle{izht13}\\ 
\textcite{luws13} \citetitle{luws13}\\ 
\textcite{dusc13} \citetitle{dusc13}\\ 
\textcite{tibb13} \citetitle{tibb13}\\ 
\textcite{bubj13} \citetitle{bubj13}\\ 
\textcite{scmo13} \citetitle{scmo13}\\ 
\textcite{fuob13} \citetitle{fuob13}\\ 
\textcite{magc13} \citetitle{magc13}\\ 
\textcite{musi13} \citetitle{musi13}\\ 
\textcite{mibg13} \citetitle{mibg13}\\ 
\textcite{grpy13} \citetitle{grpy13}\\ 
\textcite{cavg13} \citetitle{cavg13}\\ 
\textcite{vocg13} \citetitle{vocg13}\\ 
\textcite{qula13} \citetitle{qula13}\\ 
\textcite{bugu13} \citetitle{bugu13}\\ 
\textcite{myhi13} \citetitle{myhi13}\\ 
\textcite{mesc13} \citetitle{mesc13}\\ 
\textcite{cibi13} \citetitle{cibi13}\\ 
\textcite{scra13} \citetitle{scra13}\\ 
\textcite{rems13} \citetitle{rems13}\\ 
\textcite{vagd13a} \citetitle{vagd13a}\\ 
\textcite{vagd13b} \citetitle{vagd13b}\\ 

\item[\twothousandfourteen]  
\textcite{hond14} \citetitle{hond14}\\
\textcite{ronc14} \citetitle{ronc14}\\
\textcite{mobm14} \citetitle{mobm14}\\
\textcite{famc14} \citetitle{famc14}\\
\textcite{fogm14} \citetitle{fogm14}\\
\textcite{frba14} \citetitle{frba14}\\
\textcite{gagd14} \citetitle{gagd14}\\
\textcite{lidr14} \citetitle{lidr14}\\
\textcite{bocj04} \citetitle{bocj04}\\
\textcite{bagb14} \citetitle{bagb14}\\
\textcite{stjm14} \citetitle{stjm14}\\
\textcite{basc14} \citetitle{basc14}\\
\textcite{vamd14} \citetitle{vamd14}\\
\textcite{kile14} \citetitle{kile14}\\
\textcite{jahm14} \citetitle{jahm14}\\
\textcite{bufa14} \citetitle{bufa14}\\
\textcite{chsv14} \citetitle{chsv14}\\
\textcite{chsg14} \citetitle{chsg14}\\
\textcite{sigb14} \citetitle{sigb14}\\
\textcite{shjm14} \citetitle{shjm14}\\
\textcite{mova14} \citetitle{mova14}\\
\textcite{olpr14} \citetitle{olpr14}\\
\textcite{mafv14} \citetitle{mafv14}\\
\textcite{voge14} \citetitle{voge14}\\
\textcite{voge14b} \citetitle{voge14b}\\
\textcite{paml14b} \citetitle{paml14b}\\ 
\textcite{bufy14b} \citetitle{bufy14b}\\
\textcite{robn14} \citetitle{robn14}\\
\item[\twothousandfifteen]   
\textcite{bemm15} \citetitle{bemm15}\\
\textcite{bomv15} \citetitle{bomv15}\\
\textcite{bogf15} \citetitle{bogf15}\\
\textcite{ceag15} \citetitle{ceag15}\\
\textcite{kifr15} \citetitle{kifr15}\\
\textcite{vami15} \citetitle{vami15}\\
\textcite{dali15} \citetitle{dali15}\\
\textcite{mami15} \citetitle{mami15}\\
\textcite{rula15} \citetitle{rula15}\\
\textcite{chsd15} \citetitle{chsd15}\\
\textcite{dusc15} \citetitle{dusc15}\\
\textcite{yotr15} \citetitle{yotr15}\\
\textcite{cibi15} \citetitle{cibi15}\\
\textcite{hobb15} \citetitle{hobb15}\\
\textcite{carr15} \citetitle{carr15}\\
\textcite{mori15} \citetitle{mori15}\\
\item[\twothousandsixteen]   
\textcite{tomy16} \citetitle{tomy16}\\
\textcite{gukt16} \citetitle{gukt16}\\
\textcite{robn16} \citetitle{robn16}\\
\textcite{mavm16} \citetitle{mavm16}\\
\textcite{magc16} \citetitle{magc16}\\
\textcite{marl16} \citetitle{marl16}\\
\textcite{mesj16} \citetitle{mesj16}\\
\textcite{jada16} \citetitle{jada16}\\
\textcite{jada16b} \citetitle{jada16b}\\ 
\textcite{liku16} \citetitle{liku16}\\
\textcite{chss16} \citetitle{chss16}\\
\textcite{agys16} \citetitle{agys16}\\
\item[\twothousandseventeen] 
\textcite{kicf17} \citetitle{kicf17}\\ 
\textcite{sche17} \citetitle{sche17}\\
\textcite{pest17} \citetitle{pest17}\\
\textcite{vomc17} \citetitle{vomc17}\\
\textcite{majf17} \citetitle{majf17}\\
\textcite{yabr17} \citetitle{yabr17}\\
\textcite{shwl17} \citetitle{shwl17}\\
\textcite{hobe17} \citetitle{hobe17}\\
\textcite{rerm17} \citetitle{rerm17}\\
\textcite{crlt17} \citetitle{crlt17}\\
\textcite{fidd17} \citetitle{fidd17}\\

\item[\twothousandeighteen] 
\textcite{yamz18} \citetitle{yamz18}\\
\textcite{crli18} \citetitle{crli18}\\
\textcite{spcv18} \citetitle{spcv18}\\
\textcite{chss18} \citetitle{chss18}\\
\textcite{yagz18} \citetitle{yagz18}\\
\textcite{mazh18} \citetitle{mazh18}\\
\textcite{pukp18} \citetitle{pukp18}\\
\textcite{masg18} \citetitle{masg18}\\
\textcite{biar18} \citetitle{biar18}\\

\item[\twothousandnineteen] 
\textcite{magn19} \citetitle{magn19}\\
\textcite{mavb19} \citetitle{mavb19}\\
\textcite{scvm19} \citetitle{scvm19}\\
\textcite{cakc19} \citetitle{cakc19}\\
\textcite{samo19} \citetitle{samo19}\\
\textcite{sihf19} \citetitle{sihf19}\\
\textcite{meag19} \citetitle{meag19}\\
\textcite{vaws19} \citetitle{vaws19}\\
\textcite{bokg19} \citetitle{bokg19}\\
\textcite{vawg19} \citetitle{vawg19}\\
\textcite{cibi19} \citetitle{cibi19}\\
\textcite{pust19} \citetitle{pust19}\\
\textcite{kani19} \citetitle{kani19}\\

\item[\twothousandtwenty] 
\textcite{algg20} \citetitle{algg20}\\
\textcite{braf20} \citetitle{braf20}\\
\textcite{vamg20} \citetitle{vamg20}\\
\textcite{dawl20} \citetitle{dawl20}\\
\textcite{meag20} \citetitle{meag20}\\
\textcite{bedh20} \citetitle{bedh20}\\
\textcite{heyg20} \citetitle{heyg20}\\
\textcite{kicd20} \citetitle{kicd20}\\
\textcite{mugu20} \citetitle{mugu20}\\
\textcite{gatt20} \citetitle{gatt20}\\
\textcite{pust20} \citetitle{pust20}\\
\textcite{bill20} \citetitle{bill20}\\
\textcite{rozr20} \citetitle{rozr20}\\
\textcite{relr20} \citetitle{relr20}\\
\textcite{tacm20} \citetitle{tacm20}\\
\textcite{kiph20} \citetitle{kiph20}\\
\textcite{sams20} \citetitle{sams20}\\
\textcite{grlc20} \citetitle{grlc20}\\
\textcite{perz20} \citetitle{perz20}\\
\textcite{crmd20} \citetitle{crmd20}\\
\textcite{pegz20} \citetitle{pegz20}\\
\textcite{aslr20} \citetitle{aslr20}\\
\textcite{abvw20} \citetitle{abvw20}\\
\textcite{gumc20} \citetitle{gumc20}\\
\textcite{grlc20} \citetitle{grlc20}\\
\textcite{tska20} \citetitle{tska20}\\
\textcite{sche20} \citetitle{sche20}\\
\textcite{nemc20} \citetitle{nemc20}\\
\textcite{scwh20} \citetitle{scwh20}\\
\textcite{with20} \citetitle{with20}\\

\item[\twothousandtwentyone] 
\textcite{sugm21} \citetitle{sugm21}\\
\textcite{befd21} \citetitle{befd21}\\
\textcite{chcg21} \citetitle{chcg21}\\
\textcite{kifc21} \citetitle{kifc21}\\
\textcite{zhle21} \citetitle{zhle21}\\
\textcite{bafu21} \citetitle{bafu21}\\
\textcite{kekg21} \citetitle{kekg21}\\
\textcite{enma21} \citetitle{enma21}\\
\textcite{hoco21} \citetitle{hoco21}\\
\textcite{resr21} \citetitle{resr21}\\
\textcite{gupg21b} \citetitle{gupg21b}\\
\textcite{ligl21b} \citetitle{ligl21b}\\

\end{itemize}
\end{scriptsize}

%--------------------------------------------------------------------
\subsection{Subduction - slab detachment, break-off, sinking velocity}
\index{topics}{Slab Detachment} 
\index{topics}{Slab Break-off}
\index{topics}{Slab Sinking velocity}
%--------------------------------------------------------------------

\begin{scriptsize}
\begin{itemize}
\item[\nineteeneightyfive] 
\textcite{futo85} \citetitle{futo85}\\
\item[\nineteenninetytwo] 
\textcite{wosp92} \citetitle{wosp92}\\
\item[\nineteenninetyfive] 
\textcite{yowo95} \citetitle{yowo95}\\
\textcite{voda95} \citetitle{voda95}\\
\textcite{davo95} \citetitle{davo95}\\
\item[\nineteenninetyseven] 
\textcite{wowo97} \citetitle{wowo97}\\
\item[\nineteenninetyeight] 
\textcite{desw98} \citetitle{desw98}\\
\textcite{caws98} \citetitle{caws98}\\
\item[\twothousand] 
\textcite{wosp00} \citetitle{wosp00}\\
\item[\twothousandtwo] 
\textcite{bugw02} \citetitle{bugw02}\\
\item[\twothousandfour] 
\textcite{geym04} \citetitle{geym04}\\
\item[\twothousandfive] 
\textcite{mozl05} \citetitle{mozl05}\\
\item[\twothousandsix] 
\textcite{fabm06} \citetitle{fabm06}\\
\item[\twothousandeight] 
\textcite{zlfd08} \citetitle{zlfd08}\\
\item[\twothousandnine] 
\textcite{anbi09} \citetitle{anbi09}\\
\textcite{bubi09} \citetitle{bubi09}\\
\textcite{vasv09} \citetitle{vasv09}\\
\item[\twothousandten] 
\textcite{bubi10} \citetitle{bubi10}\\
\textcite{bagc10} \citetitle{bagc10}\\
\textcite{hagr10} \citetitle{hagr10}\\
\item[\twothousandeleven] 
\textcite{dugm11} \citetitle{dugm11}\\
\textcite{vaal11} \citetitle{vaal11}\\
\textcite{schm11} \citetitle{schm11}\\
\item[\twothousandtwelve] 
\textcite{dugk12} \citetitle{dugk12}\\
\textcite{dusg12} \citetitle{dusg12}\\
\item[\twothousandthirteen] 
\textcite{care13} \citetitle{care13}\\
\textcite{mafv13} \citetitle{mafv13}\\
\textcite{ghbu13} \citetitle{ghbu13}\\
\textcite{duge13} \citetitle{duge13}\\
\textcite{lixg13} \citetitle{lixg13}\\
\item[\twothousandfourteen] 
\textcite{dugs14} \citetitle{dugs14}\\
\textcite{besr14} \citetitle{besr14}\\
\textcite{vosd14} \citetitle{vosd14}\\
\textcite{butm14} \citetitle{butm14}\\
\item[\twothousandfifteen] 
\textcite{vosc15} \citetitle{vosc15}\\
\textcite{fohk15} \citetitle{fohk15}\\
\item[\twothousandseventeen] 
\textcite{frbm17} \citetitle{frbm17}\\
\textcite{maav17} \citetitle{maav17}\\
\item[\twothousandeighteen] 
\textcite{garm18} \citetitle{garm18}\\
\textcite{bezb18} \citetitle{bezb18}\\
\item[\twothousandnineteen] 
\textcite{beml19} \citetitle{beml19}\\
\textcite{fegb19} \citetitle{fegb19}\\
\item[\twothousandtwenty] 
\textcite{thsc20} \citetitle{thsc20}\\
\item[\twothousandtwentyone] 
\textcite{erhf21} \citetitle{erhf21}\\
\end{itemize}
\end{scriptsize}

%--------------------------------------------------------------------
\subsection{Subduction + water (fluids), mantle dynamics + water}
\index{topics}{Subduction+fluids}
%--------------------------------------------------------------------

\begin{scriptsize}
\begin{itemize}
\item[\nineteeneightyseven]
\textcite{peac87a} \citetitle{peac87a}\\
\item[\nineteenninety]
\textcite{peac90a} \citetitle{peac90a}\\
\textcite{peac90b} \citetitle{peac90b}\\
\item[\nineteenninetyone]
\textcite{peac91} \citetitle{peac91}\\
\item[\nineteenninetyeight]
\textcite{scpo98} \citetitle{scpo98}\\
\item[\twothousandtwo] 
\textcite{vakp02} \citetitle{vakp02}\\
\item[\twothousandfour] 
\textcite{didb04} \citetitle{didb04}\\
\item[\twothousandsix] 
\textcite{abvk06} \citetitle{abvk06}\\
\item[\twothousandeight] 
\textcite{vary08} \citetitle{vary08}\\
\textcite{wawh08} \citetitle{wawh08}\\
\item[\twothousandten] 
\textcite{roms10} \citetitle{roms10}\\
\item[\twothousandeleven] 
\textcite{geme11} \citetitle{geme11}\\
\textcite{vahs11} \citetitle{vahs11}\\
\item[\twothousandtwelve] 
\textcite{fagm12} \citetitle{fagm12}\\
\item[\twothousandfourteen] 
\textcite{qubu14} \citetitle{qubu14}\\
\textcite{mabv14} \citetitle{mabv14}\\ 
\textcite{malg14} \citetitle{malg14}\\ 
\textcite{wisv14} \citetitle{wisv14}\\
\item[\twothousandfifteen] 
\textcite{bomv15} \citetitle{bomv15}\\
\textcite{nani15} \citetitle{nani15}\\
\item[\twothousandseventeen] 
\textcite{ceww17} \citetitle{ceww17}\\
\textcite{wewv17} \citetitle{wewv17}\\
\item[\twothousandeighteen] 
\textcite{fade18} \citetitle{fade18}\\
\item[\twothousandnineteen] 
\textcite{ceww19} \citetitle{ceww19}\\ 
\textcite{meag19} \citetitle{meag19}\\
\textcite{ligc19} \citetitle{ligc19}\\
\textcite{prdp19} \citetitle{prdp19}\\
\item[\twothousandtwentytwo] 
\textcite{li22} \citetitle{li22}\\
\end{itemize}
\end{scriptsize}

%--------------------------------------------------------------------
\subsection{Subduction/plate tectonics initiation}
\index{topics}{Subduction Initiation}
%--------------------------------------------------------------------

\todo[inline]{split between Induced (ISI) and Spontaneous (SSI)}

\begin{scriptsize}
\begin{itemize}
\item[\nineteenseventyeight] 
\textcite{bird78} \citetitle{bird78}\\
\item[\nineteeneightytwo] 
\textcite{clwv82} \citetitle{clwv82}\\
\item[\nineteeneightyfour] 
\textcite{cade84} \citetitle{cade84}\\
\item[\nineteeneightynine] 
\textcite{clwv89} \citetitle{clwv89}\\
\item[\nineteenninety] 
\textcite{ogaw90} \citetitle{ogaw90}\\ 
\item[\nineteenninetyone] 
\textcite{muph91} \citetitle{muph91}\\
\item[\nineteenninetytwo] 
\textcite{stbl92} \citetitle{stbl92}\\
\item[\nineteenninetysix] 
\textcite{kest96} \citetitle{kest96}\\
\item[\nineteenninetyeight] 
\textcite{togu98} \citetitle{togu98}\\
\item[\nineteenninetynine] 
\textcite{fagd99} \citetitle{fagd99}\\
\item[\twothousand] 
\textcite{pybf00} \citetitle{pybf00}\\
\item[\twothousandone] 
\textcite{dohe01} \citetitle{dohe01}\\
\textcite{reyb01} \citetitle{reyb01}\\
\textcite{brry01} \citetitle{brry01}\\
\item[\twothousandthree] 
\textcite{hags03} \citetitle{hags03}\\
\textcite{niop03} \citetitle{niop03}\\
\item[\twothousandfour] 
\textcite{ster04} \citetitle{ster04}\\
\textcite{guhl04} \citetitle{guhl04}\\
\textcite{solo04} \citetitle{solo04}\\
\item[\twothousandfive] 
\textcite{bihi05} \citetitle{bihi05}\\
\textcite{hyne05} \citetitle{hyne05}\\
\item[\twothousandseven] 
\textcite{kore07} \citetitle{kore07}\\
\item[\twothousandeight] 
\textcite{uegs08} \citetitle{uegs08}\\
\item[\twothousandten] 
\textcite{nigm10} \citetitle{nigm10}\\
\textcite{bucl10} \citetitle{bucl10}\\
\item[\twothousandeleven] 
\textcite{bagw11} \citetitle{bagw11}\\ 
\textcite{nigm11} \citetitle{nigm11}\\
\textcite{legu11} \citetitle{legu11}\\
\item[\twothousandtwelve] 
\textcite{stri12} \citetitle{stri12}\\ 
\textcite{thka12} \citetitle{thka12}\\
\textcite{lega12} \citetitle{lega12}\\ 
\textcite{shch12} \citetitle{shch12}\\
\item[\twothousandthirteen] 
\textcite{dyge13} \citetitle{dyge13}\\
\textcite{mana13} \citetitle{mana13}\\
\textcite{kore13} \citetitle{kore13}\\
\textcite{mibg13} \citetitle{mibg13}\\
\item[\twothousandfourteen] 
\textcite{recf14} \citetitle{recf14}\\ 
\textcite{macg14} \citetitle{macg14}\\
\textcite{crta14} \citetitle{crta14}\\
\textcite{beri14} \citetitle{beri14}\\
\item[\twothousandfifteen] 
\textcite{woso15} \citetitle{woso15}\\ 
\textcite{matv15} \citetitle{matv15}\\
\textcite{pebu15} \citetitle{pebu15}\\
\textcite{vapm15} \citetitle{vapm15}\\
\textcite{legu15} \citetitle{legu15}\\
\textcite{gesb15} \citetitle{gesb15}\\
\item[\twothousandsixteen] 
\textcite{woso16a} \citetitle{woso16a}\\ 
\textcite{crta16} \citetitle{crta16}\\
\textcite{maka16} \citetitle{maka16}\\
\textcite{bags16} \citetitle{bags16}\\
\textcite{heps16} \citetitle{heps16}\\
\item[\twothousandseventeen] 
\textcite{magm17} \citetitle{magm17}\\
\textcite{baso17} \citetitle{baso17}\\
\item[\twothousandeighteen] 
\textcite{zhlg18} \citetitle{zhlg18}\\ 
\textcite{basq18} \citetitle{basq18}\\ 
\textcite{stge18} \citetitle{stge18}\\ 
\textcite{hall18} \citetitle{hall18}\\
\item[\twothousandnineteen] 
\textcite{begb19} \citetitle{begb19}\\
\textcite{gubg19} \citetitle{gubg19}\\
\textcite{ulcw19} \citetitle{ulcw19}\\
\textcite{zhli19} \citetitle{zhli19}\\
\item[\twothousandtwenty] 
\textcite{arla20} \citetitle{arla20}\\
\textcite{zhlg20} \citetitle{zhlg20}\\
\textcite{mapg20} \citetitle{mapg20}\\
\textcite{tawm20} \citetitle{tawm20}\\
\textcite{basg20b} \citetitle{basg20b}\\ 
\textcite{auwy20} \citetitle{auwy20}\\
\item[\twothousandtwentyone] 
\textcite{kndc21} \citetitle{kndc21}\\ 
\textcite{roac21} \citetitle{roac21}\\
\textcite{vasg21} \citetitle{vasg21}\\
\textcite{basg21} \citetitle{basg21}\\
\textcite{zhwa21} \citetitle{zhwa21}\\
\textcite{zhzl21} \citetitle{zhzl21}\\
\textcite{auwy21} \citetitle{auwy21}\\
\end{itemize}
\end{scriptsize}

%--------------------------------------------------------------------
\subsection{Subduction - flat/low angle/horizontal subduction}
\index{topics}{Flat/low angle subduction}
%--------------------------------------------------------------------

\begin{scriptsize}
\begin{itemize}
\item[\twothousand] van Hunen \etal \cite{vavv00}
\item[\twothousandone] van Hunen \etal \cite{vavv01}
\item[\twothousandtwo] van Hunen \etal \cite{vavv02,vavv02b}
\item[\twothousandfour] van Hunen \etal \cite{vavv04d}
\item[\twothousandeight] Perez-Campos \etal \cite{pekh08}, Espurt \etal \cite{esfm08}
\item[\twothousandeleven] Currie \& Beaumont \cite{cube11}
\item[\twothousandtwelve] Manea \etal \cite{mapm12}, Rodriguez-Gonzalez \etal \cite{ronb12}
\item[\twothousandfifteen] Gerault \etal \cite{gehm15}, Taramon \etal \cite{tarn15},
                           Eakin \etal \cite{ealw15}
\item[\twothousandsixteen] Chiarabba \etal \cite{chdf16}, Huangfu \etal \cite{huwc16}, Hu \etal \cite{hulh16}
\item[\twothousandnineteen] Siravo \etal \cite{sifg19}, Sandiford \etal \cite{sams19b},
                            Ma \etal \cite{malg19}
\item[\twothousandtwenty] Dai \etal \cite{dawl20}, Schellart \cite{sche20}
\end{itemize}
\end{scriptsize}


%--------------------------------------------------------------------
\subsection{Subduction - slab rollback} 
\index{topics}{Slab rollback}
%--------------------------------------------------------------------

\begin{scriptsize}
\begin{itemize}
\item[2006] 
\textcite{stfs06} \citetitle{stfs09}\\
\item[2009] 
\textcite{huby09} \citetitle{huby09}\\
\item[2010] 
\textcite{spha10} \citetitle{spha10}\\
\item[2012] 
\textcite{mapm12} \citetitle{mapm12}\\
\item[2013] 
\textcite{namu13} \citetitle{namu13}\\
\textcite{cibi13} \citetitle{cibi13}\\
\item[2014]
\textcite{stjm14} \citetitle{stjm14}\\
\textcite{vavs14} \citetitle{vavs14}\\
\item[2015]
\textcite{medd15} \citetitle{medd15}\\
\item[2020]
\textcite{dawl20} \citetitle{dawl20}\\
\end{itemize}
\end{scriptsize}



%--------------------------------------------------------------------
\subsection{Teaching} 
\index{topics}{Teaching}
%--------------------------------------------------------------------

\begin{scriptsize}
\begin{itemize}
\item[2011] \textcite{grap11}\citetitle{grap11}\\
\item[2014] \textcite{kerh14}\citetitle{kerh14}\\
\item[2019] \textcite{bemg19}\citetitle{bemg19}\\
\end{itemize}
\end{scriptsize}

%--------------------------------------------------------------------
\subsection{Tethys} 
\index{topics}{Tethys}
%--------------------------------------------------------------------

\begin{scriptsize}
\begin{itemize}
\item[\nineteenninetynine] 
\textcite{vasb99} \citetitle{vasb99}\\
\item[\twothousand] 
\textcite{mokd00} \citetitle{mokd00}\\
\item[\twothousandeleven] 
\textcite{befa11} \citetitle{befa11}\\
\item[\twothousandthirteen]
\textcite{wagw13} \citetitle{wagw13}\\
\item[\twothousandsixteen] 
\textcite{necg16} \citetitle{necg16}\\
\item[\twothousandeighteen] 
\textcite{marc18} \citetitle{marc18}\\
\item[\twothousandtwentyone] 
\textcite{gupg21b} \citetitle{gupg21b} \\
\end{itemize}
\end{scriptsize}

%--------------------------------------------------------------------
\subsection{Transform faults} 
\index{topics}{Transform faults}
%--------------------------------------------------------------------

\begin{scriptsize}
\begin{itemize}
\item[\nineteenseventytwo] Lachenbruch \& Thompson \cite{lath72}
\item[\nineteenseventyeight] Yuen \etal \cite{yufs78}
\item[\twothousandseven] \cite{macl07}
\item[\twothousandten] \cite{gerya2010}
\item[\twothousandtwelve] Shervais \& Choi \cite{shch12}
\item[\twothousandthirteen] Gerya \cite{gery13c}
\item[\twothousandeighteen] Zhou \etal \cite{zhlg18}
\item[\twothousandtwenty] Arcay \etal \cite{arla20}, Schierjott \etal \cite{sctr20}
\end{itemize}
\end{scriptsize}

%--------------------------------------------------------------------
\subsection{Wilson cycle, supercontinent cycles}
\index{topics}{Wilson cycle}
\index{topics}{Supercontinent Formation}
\index{topics}{Supercontinent Cycle}
\index{topics}{Supercontinent Breakup}
%--------------------------------------------------------------------

\begin{scriptsize}
\begin{itemize}
\item[\nineteenninetyfive] Trubitsyn \& Rykov \cite{trry95}
\item[\nineteenninetynine] Lowman \& Jarvis \cite{loja99}
\item[\twothousandthree] Evans \cite{evan03}
\item[\twothousandseven] Zhong \etal \cite{zhzl07}, Coltice \etal \cite{copb07}, 
                         Phillips \& Bunge \cite{phbu07}
\item[\twothousandnine] Zhang \etal \cite{zhzm09}, O'Neill \etal \cite{onlj09}
\item[\twothousandten] Heron \& Lowman \cite{helo10}
\item[\twothousandeleven] Lenardic \etal \cite{lemj11}, Burke \cite{burk11}, Heron \& Lowman \cite{helo11}
\item[\twothousandfourteen] Buiter \& Torsvik \cite{buto14}, Heron \& Lowman \cite{helo14}, 
                            Rolf \etal \cite{roct14}
\item[\twothousandfifteen] Heron \etal \cite{hels15}
\item[\twothousandsixteen] Trim \& Lowman \cite{trlo16}
\item[\twothousandseventeen] 
\textcite{woda17} \citetitle{woda17}\\ 
\textcite{kaha17} \citetitle{kaha17}\\
\textcite{baso17} \citetitle{baso17}\\
\item[\twothousandeighteen] 
\textcite{panm18} \citetitle{panm18}\\
\textcite{hall18} \citetitle{hall18}\\
\item[\twothousandnineteen] Beaussier \etal \cite{begb19}, Wilson \etal \cite{wihb19}, 
                            Huang \etal \cite{huzl19} 
\item[\twothousandtwenty] Heron \etal \cite{hemn20}
\item[\twothousandtwentyone] Facenna \etal \cite{fabh21}
\end{itemize}
\end{scriptsize}

%------------------------------------------------------------------------------
\subsection{Meshless methods (SPH, RKPM, DEM, FPM, ...)}
\index{topics}{Smoothed Particle Hydrodynamics} 
\index{topics}{SPH}
\index{topics}{Discrete Element Method} 
\index{topics}{DEM}
%------------------------------------------------------------------------------

\begin{scriptsize}
\nineteenseventyseven: \cite{lucy77}\\
\nineteeneightyfive:   \cite{mona85}\\
\nineteenninetytwo:    \cite{mona92}\\
\nineteenninetysix:    \cite{beko96}\\
\nineteenninetyseven:  \cite{mofz97}\\
\nineteenninetynine:   \cite{zhfm99}, \cite{ogsa99}\\
\twothousand:          \cite{begl00}, \cite{lihl00}, \cite{juim00}\\
\twothousandone:       \cite{idso01}\\
\twothousandtwo:       \cite{lilr02}, \cite{lill02}, \cite{lili02}\\
\twothousandthree:     \cite{lill03}, \cite{mamo03}\\
\twothousandfour:      \cite{hufl04}, \cite{wali04}\\
\twothousandfive:      \cite{febh05}\cite{lixl05}\cite{thes05}\cite{thje05a}\cite{thje05b}\\
\twothousandsix:       \cite{lili06}\cite{yabm06}\\
\twothousandseven:     \cite{busf07}\\
\twothousandeight: Bui \etal \cite{bufs08}, Lee \etal \cite{lemx08}\\
\twothousandten: Das \& Cleary \cite{dacl10}\\
\twothousandeleven: \cite{prcl11}\cite{kukg11}
                    \cite{kadm11}\cite{szpt11}
                    \cite{howt11}, Beuth \etal \cite{bewv11},
\twothousandtwelve: \cite{szpm12}\\
\twothousandthirteen: \cite{koau13}\cite{viau13}\\
\twothousandfourteen: \cite{dazs14}\cite{lekb14}\\
\twothousandfifteen: \cite{nifs15}\\
\twothousandsixteen: Violeau \& Rogers \cite{viro16}\\
\twothousandeighteen: \cite{krrk18}\cite{goej18}\\
\twothousandnineteen: \cite{meho19}\cite{meho19b}
\end{scriptsize}

%------------------------------------------------------------------------------
%------------------------------------------------------------------------------
\subsection{Element Free Galerkin Method}
\index{topics}{EFGM} 
\index{topics}{Element Free Galerkin Method}
%------------------------------------------------------------------------------
%------------------------------------------------------------------------------

\begin{scriptsize}
\begin{itemize}
\item[1994]
\textcite{begl94b} \citetitle{begl94b}\\
\item[1995]
\textcite{belg95a} \citetitle{belg95a}\\
\textcite{belg95b} \citetitle{belg95b}\\
\item[1996]
\textcite{bekf96} \citetitle{bekf96}\\
\textcite{como96} \citetitle{como96}\\
\item[1997]
\textcite{bekk97} \citetitle{bekk97}\\
\item[1998]
\textcite{pobe98} \citetitle{pobe98}\\
\textcite{zhat98} \citetitle{zhat98}\\
\item[2003]
\textcite{hans03} \citetitle{hans03}\\
\item[2004]
\textcite{katf04} \citetitle{katf04}\\
\textcite{huvv04} \citetitle{huvv04}\\
\item[2010]
\textcite{yiha10} \citetitle{yiha10}\\
\textcite{libe10} \citetitle{libe10}\\
\end{itemize}
\end{scriptsize}

%------------------------------------------------------------------------------
%------------------------------------------------------------------------------
\subsection{Planetary accretion, exoplanets, planet formation, segregation}
\index{topics}{Planetary Accretion} 
\index{topics}{Planet Formation}
\index{topics}{Exo-planets}
%------------------------------------------------------------------------------
%------------------------------------------------------------------------------

\begin{scriptsize}
\begin{itemize}
\item[\twothousandeight] Lenardic \etal \cite{lejm08} 
\item[\twothousandnine] Lin \etal \cite{ligt09}, Golabek \etal \cite{gogk09}
\item[\twothousandten] van den Berg \etal \cite{vayb10}
\item[\twothousandeleven] Lin \etal \cite{ligt11}, van Summeren \etal \cite{vacg11}
\item[\twothousandthirteen] van Summeren \etal \cite{vagc13}
\item[\twothousandfourteen] Golabek \etal \cite{gobg14}, Yao \etal \cite{yadl14}
\item[\twothousandnineteen] Neumann \cite{neum19}, van den Berg \etal \cite{vayu19}
\item[\twothousandtwenty] O'Neill \etal \cite{onlw20}
\end{itemize}
\end{scriptsize}

%------------------------------------------------------------------------------
%------------------------------------------------------------------------------
\subsection{Accretionary wedges, nappes, thrust wedges, orogenic wedge, fold-thrust belt} 
\index{topics}{Accretionary Wedge}
\index{topics}{Orogenic Wedge}
\index{topics}{Accretionary Prism}
\index{topics}{Thrust Wedge}
\index{topics}{Fold-Thrust Belt}
%------------------------------------------------------------------------------
%------------------------------------------------------------------------------

\begin{scriptsize}
\begin{itemize}
\item[\nineteeneightythree] Stockmal \cite{stoc83}, Davis \etal \cite{dasd83}
\item[\nineteeneightyfour] Dahlen \cite{dahl84}, Dahlen \etal \cite{dasd84}
\item[\nineteenninety] Dahlen \cite{dahl90}
\item[\nineteenninetyfour] Koons \cite{koon94}
\item[\nineteenninetyfour] Chalaron \etal \cite{chmm95} 
\item[\nineteenninetynine] Vanbrabant \etal \cite{vajh99}
\item[\twothousandthree] \cite{wiep03}\cite{smbs03}\cite{muso03}\cite{vamf03}
\item[\twothousandsix] \cite{simp06}\cite{yabm06}
\item[\twothousandtwelve] Ruh \etal \cite{rukb12}
\item[\twothousandthirteen] Ruh \etal \cite{rugb13}
\item[\twothousandsixteen] Mannu \etal \cite{mauw16}
\item[\twothousandseventeen] Mannu \etal \cite{mauw17}, Ruh \etal \cite{rugb17}
\item[\twothousandeighteen] 
\cite{weib18}
\item[\twothousandnineteen] 
\cite{elgb19}
\cite{meho19}
\cite{meho19b}
\item[\twothousandtwenty] 
\textcite{spsk20} \citetitle{spsk20}\\
\textcite{spbe20} \citetitle{spbe20}\\
\textcite{kids20} \citetitle{kids20}\\
\textcite{hube20} \citetitle{hube20}\\
\textcite{ruh20}  \citetitle{ruh20}\\
\item[\twothousandtwentyone] 
\textcite{cadm21} \citetitle{cadm21}\\
\textcite{anmg21} \citetitle{anmg21}\\
\end{itemize}
\end{scriptsize}

%------------------------------------------------------------------------------
%------------------------------------------------------------------------------
\subsection{Thrust-wrench fault} 
\index{topics}{Thrust-Wrench Fault}
%------------------------------------------------------------------------------
%------------------------------------------------------------------------------

\begin{scriptsize}
\twothousandfifteen: Rosas \etal \cite{rods15}
\end{scriptsize}

%------------------------------------------------------------------------------
%------------------------------------------------------------------------------
\subsection{Thrust fault} 
\index{topics}{Thrust Fault}
%------------------------------------------------------------------------------
%------------------------------------------------------------------------------

\begin{scriptsize}
\nineteenninety: Molnar \& England \cite{moen90b} (putain)\\
\nineteenninetytwo: Molnar \cite{moln92} \\
\twothousandfourteen: Steer \etal \cite{stsc14}
\end{scriptsize}

%------------------------------------------------------------------------------
%------------------------------------------------------------------------------
\subsection{Transpressional systems} 
\index{topics}{Transpressional system}
%------------------------------------------------------------------------------
%------------------------------------------------------------------------------

\begin{scriptsize}
\begin{itemize}
\item[\nineteenninetyfour] Tikoff \& Teyssier \cite{tite94}
\item[\nineteenninetyseven] Thompson \etal \cite{thsj97}
\item[\twothousandthree] Koons \etal \cite{konc03}, Upton \etal \cite{upke03}
\item[\twothousandeleven] Leever \etal \cite{legs11}
\item[\twothousandseventeen] Nabavi \etal \cite{naam17}, Ruh \etal \cite{rugb17}
\item[\twothousandeighteen] Nabavi \etal \cite{naam18}
\end{itemize}
\end{scriptsize}

%------------------------------------------------------------------------------
%------------------------------------------------------------------------------
\subsection{Urey ratio}
\index{topics}{Urey Ratio}
%------------------------------------------------------------------------------
%------------------------------------------------------------------------------

\begin{scriptsize}
\begin{itemize}
\item[\twothousandeight] 
\textcite{kore08} \citetitle{kore08}
\item[\twothousandtwelve] 
\textcite{nata12} \citetitle{nata12}
\end{itemize}
\end{scriptsize}

%------------------------------------------------------------------------------
%------------------------------------------------------------------------------
\subsection{Intrusions, diapirism, Rayleigh-Taylor instability}
\index{topics}{Intrusions}
\index{topics}{Diapirism}
\index{topics}{Rayleigh-Taylor Instability}
%------------------------------------------------------------------------------
%------------------------------------------------------------------------------

See EGU blog article: 
\url{https://blogs.egu.eu/divisions/gd/2021/02/17/rayleigh-taylor-instability-in-geodynamics/}

\begin{scriptsize}
\begin{itemize}
\item[\nineteensixtyfive] Biot \& Ode \cite{biod65}
\item[\nineteensixtyeight] Ramberg \cite{ramb68}
\item[\nineteenseventytwo] Berner \etal \cite{bers72}
\item[\nineteenseventyfive] Dixon \cite{dixo75}
\item[\nineteenseventyeight] Woidt \cite{woid78}
\item[\nineteeneighty] Ramberg \cite{ramb80}, Woidt \& Neubebauer \cite{wone80}
\item[\nineteeneightyone] Bridwell \& Potzick \cite{brpo81}
\item[\nineteeneightythree] Ribe \cite{ribe83}
\item[\nineteeneightysix] Weijermars \& Schmeling  \cite{wesc86}
\item[\nineteeneightyseven] Schmeling  \cite{schm87}
\item[\nineteeneightyeight] Schmeling \etal \cite{sccm88}, Whitehead \cite{whit88b}  
\item[\nineteenninetytwo] van Keken \etal \cite{vayv92}, Zaleski \& Julien \cite{zaju92}, 
                    Weinberg \cite{wein92}, Weinberg \& Schmeling \cite{wesc92},
                    Vendeville \& Jackson \cite{veja92}\cite{pepp92}
\item[\nineteenninetythree] Nalpas \& Brun \cite{nabr93}, van Keken \etal \cite{vayv93,vasv93}
                            Podlachikov \etal \cite{potp93}, Poliakov \etal \cite{povp93,pocp93},
                            Weinberg \cite{wein93}
\item[\nineteenninetyfour] Weinberg \& Podlachikov \cite{wepo94}, Daudre \& Cloetingh \cite{dacl94}
\item[\nineteenninetyfive] Weinberg \& Podlachikov \cite{wepo95}, Bittner \& Schmeling \cite{bisc95},
                           Cruden \etal \cite{crks95}
\item[\nineteenninetyseven] Weinberg \cite{wein97}
\item[\nineteenninetyeight] Molnar \etal \cite{mohc98}
\item[\nineteenninetynine] Drury \etal \cite{drdv99}
\item[\twothousandone] Kaus \& Podlachikov \cite{kapo01}, Drury \etal \cite{drvc01}
\item[\twothousandthree] Gerya \etal \cite{geur03}, van Thienen \etal \cite{vavd03}
\item[\twothousandfour] Gerya \etal \cite{gepm04,geur04}, Ismail-Zadeh \etal \cite{istt04},
                        Burg \etal \cite{bukp04}
\item[\twothousandseven] Gerya \& Burg \cite{gebu07}
\item[\twothousandeight] Burg \& Gerya \cite{buge08}, Zlotnik \etal \cite{zlfd08},
\item[\twothousandeleven] Ellis \etal \cite{ellw11}, Perchuk \& Gerya \cite{pege11},
                          Fuchs \etal \cite{fusk11}
\item[\twothousandtwelve] Polyansky \etal \cite{pokb12}
\item[\twothousandthirteen] Fuchs \& Schmeling \cite{fusc13}
\item[\twothousandfourteen] Fernandez \& Kaus \cite{feka14b}
\item[\twothousandfifteen] Fernandez \& Kaus \cite{feka15}, Fuchs \etal \cite{fuks15}
\item[\twothousandsixteen] Cao \etal \cite{cakp16}, Polyansky \etal \cite{porb16}
\item[\twothousandeighteen] Gerbault \etal \cite{gesr18}
\item[\twothousandtwenty] Louis-Napoleon \etal \cite{logb20}, Schuh-Senlis \etal \cite{sctc20}
\end{itemize}
\end{scriptsize}

%--------------------------------------------------------------------
%------------------------------------------------------------------------------
\subsection{Visualization, rendering}
\index{topics}{Visualization}
%------------------------------------------------------------------------------
%--------------------------------------------------------------------

\begin{scriptsize}
\cite{faha}\\
2004: Rudolph \etal \cite{rugy04}\\
2005: Rudolph \etal \cite{rugy05}\\
2008: Chen \etal \cite{chzy08}, Stegman \etal \cite{stmt08}
      Billen \etal \cite{bikh08}, Kadlec \etal \cite{kadt08}\\
2012: May \cite{may12}\\
2017: \cite{krke17}\\
\twothousandeighteen: Crameri \cite{cram18}\\
\twothousandtwenty: Crameri \etal \cite{crsh20}
\end{scriptsize}

%--------------------------------------------------------------------
%------------------------------------------------------------------------------
\subsection{Solving Stokes Saddle Point problem}
\index{topics}{Uzawa-type algorithms}
%------------------------------------------------------------------------------
%------------------------------------------------------------------------------

\begin{scriptsize}
\cite{laqu86}
\cite{rotf90}
\cite{frha93}
\cite{elgo94}
\cite{cheb96}\cite{elma96}
\cite{brpv97}
\cite{lixu01}
\cite{dogs06}\cite{lica06}
\cite{hoow17}
\end{scriptsize}

%--------------------------------------------------------------------
%------------------------------------------------------------------------------
\subsection{Celestial bodies}
%------------------------------------------------------------------------------
%------------------------------------------------------------------------------

\begin{itemize}

\item Europa \index{topics}{Europa}
\begin{scriptsize}
\begin{itemize}
\item[\twothousandfour] \cite{shha04}
\item[\twothousandfive] \cite{shha05}, \cite{mish05}
\item[\twothousandeight] \cite{hash08}
\item[\twothousandten] \cite{hash10}
\item[\twothousandeleven] \cite{hash11}
\item[\twothousandfourteen] \cite{kast14}
\item[\twothousandnineteen] \cite{almc19}
\end{itemize}
\end{scriptsize}

\item Moon \index{topics}{Moon}
\begin{scriptsize}
\begin{itemize}
\item[\twothousandtwo] Elkins-Tanton \etal \cite{elvh02}
\item[\twothousandfour] Elkins-Tanton \etal \cite{elhg04}
\item[\twothousandten] de Vries \etal \cite{devv10}
\item[\twothousandthirteen] de Vries \etal \cite{dejv13} 
\item[\twothousandnineteen] Zhao \etal \cite{zhdv19}
\end{itemize}
\end{scriptsize}
 
\item Venus \index{topics}{Venus}

\begin{scriptsize}
\begin{itemize}
\item[\nineteenninety] 
\textcite{scbg90} \citetitle{scbg90} \\
\textcite{sozh90} \citetitle{sozh90} \\
\item[\nineteenninetyone] 
\textcite{lekb91} \citetitle{lekb91} \\
\textcite{leyu91} \citetitle{leyu91} \\
\item[\nineteenninetytwo] 
\textcite{kiha92} \citetitle{kiha92} \\
\textcite{sqjs92} \citetitle{sqjs92} \\
\item[\nineteenninetythree] 
\textcite{kief93} \citetitle{kief93} \\
\textcite{lekb93} \citetitle{lekb93} \\
\textcite{ogaw93} \citetitle{ogaw93}
\item[\nineteenninetyfive] 
\textcite{lekb95} \citetitle{lekb95} \\
\textcite{mopa95} \citetitle{mopa95} \\
\item[\nineteenninetysix] 
\textcite{somo96} \citetitle{somo96} \\
\item[\nineteenninetyseven] 
\textcite{mang97} \citetitle{mang97} \\
\item[\nineteenninetyeight] 
\textcite{mazk98} \citetitle{mazk98} \\
\textcite{resm98} \citetitle{resm98} \\
\textcite{moso98} \citetitle{moso98} \\
\textcite{phha98} \citetitle{phha98} \\
\item[\nineteenninetynine] 
\textcite{resm99} \citetitle{resm99} \\
\item[\twothousand] 
\textcite{ogaw00} \citetitle{ogaw00} \\
\item[\twothousandthree] 
\textcite{vesh03} \citetitle{vesh03} \\
\item[\twothousandfour] 
\textcite{vesb04} \citetitle{vesb04} \\
\item[\twothousandfive] 
\textcite{vavv05} \citetitle{vavv05} \\
\item[\twothousandseven] 
\textcite{reso07} \citetitle{reso07} \\
\item[\twothousandten] 
\textcite{stfh10} \citetitle{stfh10} \\
\textcite{stwt10} \citetitle{stwt10} \\
\item[\twothousandeleven] 
\textcite{orso11} \citetitle{orso11} \\
\item[\twothousandtwelve] 
\textcite{arta12} \citetitle{arta12} \\
\textcite{orso12} \citetitle{orso12} \\
\textcite{nobs12} \citetitle{nobs12} \\
\item[\twothousandthirteen] 
\textcite{huyz13} \citetitle{huyz13} \\
\item[\twothousandfourteen] 
\textcite{gita14} \citetitle{gita14} \\
\textcite{gery14b} \citetitle{gery14b} \\
\item[\twothousandseventeen] 
\textcite{cram17} \citetitle{cram17} \\
\textcite{dast17} \citetitle{dast17} \\
\item[\twothousandeighteen] 
\textcite{king18} \citetitle{king18} \\
\textcite{ross18} \citetitle{ross18} \\
\item[\twothousandtwenty] 
\textcite{weki20} \citetitle{weki20} \\
\textcite{gugm20} \citetitle{gugm20} \\
\textcite{uprc20} \citetitle{uprc20} \\
\textcite{kacc20} \citetitle{kacc20} \\
\item[\twothousandtwentyone] 
\end{itemize}
\end{scriptsize}


%....................................
\item Mars \index{topics}{Mars}\\
\begin{scriptsize}
\begin{itemize}
\item[\nineteeneightytwo] 
\textcite{baps82} \citetitle{baps82}\\
\textcite{witu82} \citetitle{witu82}\\
\textcite{sohe82} \citetitle{sohe82}\\
\item[\nineteenninety] 
\textcite{scbg90} \citetitle{scbg90}\\

\item[\nineteenninetyone] 
\textcite{spoh91} \citetitle{spoh91}\\
\textcite{jaer91} \citetitle{jaer91}\\

\item[\nineteenninetyfour] 
\textcite{slee94},\citetitle{slee94}\\

\item[\nineteenninetysix] 
\textcite{hach96} \citetitle{hach96}\\
\textcite{brzy96} \citetitle{brzy96}\\
\textcite{kibn96} \citetitle{kibn96}\\
\textcite{mema96} \citetitle{mema96}\\

\item[\nineteenninetyseven] 
\textcite{brys97} \citetitle{brys97}\\

\item[\nineteenninetyeight] 
\textcite{resm98} \citetitle{resm98}\\
\textcite{brys98} \citetitle{brys98}\\

\item[\nineteenninetynine] 
\textcite{smst99} \citetitle{smst99}\\

\item[\twothousand] 
\textcite{hard00} \citetitle{hard00}\\ 

\item[\twothousandone] 
\textcite{nist01} \citetitle{nist01}\\
\textcite{zube01} \citetitle{zube01}\\
\textcite{scvy01} \citetitle{scvy01}\\

\item[\twothousandtwo] 
\textcite{resb02} \citetitle{resb02}\\
\textcite{zhon02} \citetitle{zhon02}\\
\textcite{haph02} \citetitle{haph02}\\
\textcite{mcby02} \citetitle{mcby02}\\
\textcite{scvy02} \citetitle{scvy02}\\

\item[\twothousandthree] 
\textcite{zhro03} \citetitle{zhro03}\\
\textcite{lozh03} \citetitle{lozh03}\\
\textcite{kief03} \citetitle{kief03}\\

\item[\twothousandfour] 
\textcite{lenm04} \citetitle{lenm04}\\ 
\textcite{vavv04c} \citetitle{vavv04c}\\ 
\textcite{resb04} \citetitle{resb04}\\
\textcite{reki04} \citetitle{reki04}\\
\textcite{rozh04} \citetitle{rozh04}\\

\item[\twothousandfive] 
\textcite{vavv05} \citetitle{vavv05}\\ 
\textcite{onml05} \citetitle{onml05}\\
\textcite{belw05} \citetitle{belw05}\\

\item[\twothousandsix] 
\textcite{reso06} \citetitle{reso06}\\
\textcite{losh06} \citetitle{losh06}\\
\textcite{rozh06} \citetitle{rozh06}\\
\textcite{keso06} \citetitle{keso06}\\
\textcite{koys06} \citetitle{koys06}\\
\textcite{brsp06} \citetitle{brsp06}\\

\item[\twothousandseven] 
\textcite{rozh07} \citetitle{rozh07}\\ 
\textcite{reso07b} \citetitle{reso07b}\\
\textcite{liki07} \citetitle{liki07}\\

\item[\twothousandeight] 
\textcite{loha08} \citetitle{loha08}\\
\textcite{winm08} \citetitle{winm08}\\

\item[\twothousandnine] 
\textcite{keta09} \citetitle{keta09}\\
\textcite{zhon09} \citetitle{zhon09}\\
\textcite{rolm09} \citetitle{rolm09}\\
\textcite{keso09} \citetitle{keso09}\\
\textcite{smzt09} \citetitle{smzt09}\\
\textcite{habg09} \citetitle{habg09}\\

\item[\twothousandten] 
\textcite{srzh10} \citetitle{srzh10}\\ 
\textcite{reos10} \citetitle{reos10}\\
\textcite{reso10} \citetitle{reso10}\\
\textcite{stwt10} \citetitle{stwt10}\\
\textcite{grbr10} \citetitle{grbr10}\\

\item[\twothousandeleven] 
\textcite{gokg11} \citetitle{gokg11}\\
\textcite{reos11} \citetitle{reos11}\\
\textcite{jizl11} \citetitle{jizl11}\\
\textcite{koaf11} \citetitle{koaf11}\\
\textcite{nasc11} \citetitle{nasc11}\\

\item[\twothousandtwelve] 
\textcite{srzh12} \citetitle{srzh12}\\
\textcite{roar12} \citetitle{roar12}\\
\textcite{hick12} \citetitle{hick12}\\
\textcite{belr12} \citetitle{belr12}\\

\item[\twothousandthirteen] 
\textcite{ruts13} \citetitle{ruts13}\\
\textcite{ruts13b} \citetitle{ruts13b}\\

\item[\twothousandfourteen] 
\textcite{seki14}\citetitle{seki14}\\
\textcite{chki14}\citetitle{chki14}\\

\item[\twothousandfifteen] 
\textcite{kifs15} \citetitle{kifs15}\\

\item[\twothousandsixteen] 
\textcite{zhon16} \citetitle{zhon16}\\
\textcite{kili16} \citetitle{kili16}\\
\textcite{gegl16} \citetitle{gegl16}\\
\textcite{bobm16} \citetitle{bobm16}\\

\item[\twothousandseventeen] 
\textcite{rubr17} \citetitle{rubr17}
\textcite{hema17} \citetitle{hema17}
\textcite{azka17} \citetitle{azka17}

\item[\twothousandeighteen] 
\textcite{cimt18} \citetitle{cimt18}\\
\textcite{goej18} \citetitle{goej18}\\
\textcite{scmo18} \citetitle{scmo18}\\
\textcite{khlr18} \citetitle{khlr18}\\
\textcite{domk18} \citetitle{domk18}\\

\item[\twothousandnineteen] 
\textcite{smls19} \citetitle{smls19}\\
\textcite{cahe19} \citetitle{cahe19}\\
\textcite{dilg19} \citetitle{dilg19}\\

\item[\twothousandtwenty] 
\textcite{lobp20} \citetitle{lobp20}\\ 
\textcite{gilb20} \citetitle{gilb20}\\
\textcite{agtb20} \citetitle{agtb20}\\
\textcite{geno20} \citetitle{geno20}\\
\textcite{basb20} \citetitle{basb20}\\
\textcite{tajh20} \citetitle{tajh20}\\

\item[\twothousandtwentyone] 
\textcite{khcv21} \citetitle{khcv21}\\ 
\textcite{stkb21} \citetitle{stkb21}\\
\textcite{knpb21} \citetitle{knpb21}\\
\textcite{vand21} \citetitle{vand21}\\
\textcite{kobj21} \citetitle{kobj21}\\
\textcite{ribc21} \citetitle{ribc21}\\
\textcite{topa21} \citetitle{topa21}\\

\end{itemize}
\end{scriptsize}

%....................................
\item Mercury \index{topics}{Mercury}

\begin{scriptsize}
\begin{itemize}
\item[\twothousandseven] 
\textcite{reki07} \citetitle{reki07}
\item[\twothousandeight] 
\textcite{king08} \citetitle{king08}
\item[\twothousandtwelve] 
\textcite{roba12} \citetitle{roba12} 
\item[\twothousandtwentyone] 
\textcite{gult21} \citetitle{gult21}
\end{itemize}
\end{scriptsize}

%....................................
\item Pluto \index{topics}{Pluto}

\begin{scriptsize}
\twothousandsixteen \textcite{mcnw16} \citetitle{mcnw16}
\end{scriptsize}

%....................................
\item Super-Earths \& exoplanets \index{topics}{Super-Earths}
   \begin{scriptsize}
   \cite{stfl11}\cite{vata11}
   \cite{stlh13}
   \cite{welo15}\cite{miko15}\cite{kamo15}
   \end{scriptsize}

\item Icy satellites  \index{topics}{Icy satellites}
   \begin{scriptsize}
   \cite{kasc12b}
   \end{scriptsize}

\item Enceladus  \index{topics}{Enceladus}
   \begin{scriptsize}
   \cite{roni08},
   \cite{betc10},
   \cite{hats12},
   \cite{robg14}
   \end{scriptsize}

\item Io  \index{topics}{Io}
   \begin{scriptsize}
   \cite{tasg01}, \cite{tack01}
   \end{scriptsize}
\end{itemize}

%--------------------------------------------------------------------
%--------------------------------------------------------------------
\subsection{Locations}
%--------------------------------------------------------------------
%--------------------------------------------------------------------

\begin{itemize}
%..........................
\item South America, Andes, Andean orogeny 
\index{topics}{Andes}
\index{topics}{South America}

\begin{scriptsize}
\begin{itemize}
\item[\nineteenninetyfour] Wdowinski \& Bock \cite{wdbo94b}
\item[\twothousand] Gutscher \etal \cite{gusb00}
\item[\twothousandtwo] van Hunen \etal \cite{vavv02b}
\item[\twothousandfive] Babeyko \& Sobolev \cite{baso05}, Sobolev \& Babeyko \cite{soba05}
\item[\twothousandsix] Babeyko \etal \cite{basv06}, Medvedev \etal \cite{meph06},
                       Iaffaldano \etal \cite{iabd06}, book \cite{oncf06}, Sobolev \etal \cite{sobk06}
\item[\twothousandseven] Iaffaldano \etal \cite{iabb07}
\item[\twothousandeight] Espurt \etal \cite{esfm08}, Heidbach \etal \cite{heib08}, 
                         Iaffaldano \& Bunge \cite{iabu08}, Gonzalez \etal \cite{gogm08}
\item[\twothousandnine] Keppie \etal \cite{kecw09}, Gerbault \etal \cite{gecm09}
\item[\twothousandtwelve] Husson \etal \cite{hucf12}, Shephard \etal \cite{shlm12},
                          Iaffaldano \cite{iadc12}
\item[\twothousandthirteen] van der Meijde \& Assumpcao \cite{waja13}
\item[\twothousandfifteen] Currie \etal \cite{cudd15}, Eakin \etal \cite{ealw15}
\item[\twothousandsixteen] Rodriguez-Gonzalez \etal \cite{robn16}, Martinod \etal \cite{marl16}, 
                           Chiarabba \etal \cite{chdf16}, Hu \etal \cite{hulh16}
\item[\twothousandseventeen] Schellart \cite{sche17}
\item[\twothousandnineteen] Yang \etal \cite{yamg19}
\item[\twothousandtwenty] Sch{\"u}tt \& Whipp \cite{scwh20}, Withers \cite{with20}
\item[\twothousandtwentyone] Barrionuevo \etal \cite{balm21}, Strak \& Schellart \cite{stsc21}
\end{itemize}
\end{scriptsize}

%..........................
\item North America \index{topics}{North America}

\begin{scriptsize}
\nineteenseventythree: \cite{sabu73}\\
\nineteenninety: \cite{huha90}\\
\nineteenninetyseven: \cite{bugm97}\\
\twothousandone: Ch\'ery \etal \cite{chzh01} \\
\twothousandsix: \cite{besb06}\\
\twothousandeight: Spasojevic \etal \cite{splg08}\\
\twothousandnine: Spasojevic \etal \cite{splg09}\\
\twothousandtwelve: \cite{beck12}\\
\twothousandthirteen: \cite{ghbh13}\cite{simi13}\\
\twothousandfifteen: \cite{riag15}
\twothousandtwentyone: Saxena \etal \cite{sacp21}
\end{scriptsize}

%..........................
\item Apennines \index{topics}{Apennines}

\begin{scriptsize}
\nineteenninetyeight: \cite{buwg98}
\twothousandseven: \cite{shpy07}
\twothousandnine: \cite{rohu09}
\twothousandfifteen: \cite{vami15}
\end{scriptsize}

%..........................
\item the Netherlands \index{topics}{Netherlands}

\begin{scriptsize}
\twothousandtwo Crombaghs \etal \cite{crdv02}
\twothousandtwenty Bekesi \etal \cite{besb20}
\end{scriptsize}

\item Gulf of Aden \index{topics}{Gulf of Aden}

\begin{scriptsize}
\twothousandthree Hubert-Ferrari \etal \cite{hukm03}\\
\twothousandthirteen Bellahsen \etal \cite{beha13}, Brune \& Autin \cite{brau13},
                     Watremez \etal \cite{wabd13}\\
\twothousandtwenty Duclaux \etal \cite{duhm20} 
\end{scriptsize}

%..........................
\item Banda \index{topics}{Banda Arc}

\begin{scriptsize}
Royden \& Husson \cite{rohu09}
Spakman \& Hall \cite{spha10}
Schliffke \etal \cite{scvg21}
\end{scriptsize}


%..........................
\item Alps \index{topics}{Alps}

\begin{scriptsize}
\nineteenninetysix: \cite{beeh96}\\
\nineteenninetyseven: \cite{repe97}\\
\nineteenninetyeight: \cite{desw98}\\
\twothousand: \cite{pfeb00}\\
\twothousandone: \cite{bujl01}\\
\twothousandtwo: \cite{pfsb02}\\
\twothousandthree: \cite{pimo03}\\
\twothousandfive: \cite{buge05}\\
\twothousandseven: \cite{masp07}\\
\twothousandeight: \cite{vifj08}\\
\twothousandtwelve: Roda \etal \cite{rosm12}\\
\twothousandthirteen: \cite{luws13}\cite{baes13}\cite{bubj13}\\
\twothousandfourteen: \cite{bubj14}\\
\twothousandfifteen: \cite{scdu15}\cite{fohk15}\\
\twothousandeighteen: Marotta \etal \cite{marc18}\\
\twothousandnineteen: Roda \etal \cite{rors19}\\
\twothousandtwenty: Kiss \etal \cite{kids20}, Roda \etal \cite{rozr20}, Regorda \etal \cite{relr20},
                    Assanelli \etal \cite{aslr20}
\twothousandtwentyone: G{\"u}n \etal \cite{gupg21}
\end{scriptsize}

%..........................
\item Mediterranean region \index{topics}{Mediterranean Region}

\begin{scriptsize}
\begin{itemize}
\item[\nineteenninetyseven] \cite{pimo97}\cite{nesg97}
\item[\nineteenninetynine] \cite{nesb99}
\item[\twothousand] \cite{wosp00}
\item[\twothousandthree] \cite{pimo03}
\item[\twothousandfour] Spakman \& Wortel \cite{spwo04}, Boschi \etal \cite{boek04}
\item[\twothousandnine] Wortel \etal \cite{wogs09}
\item[\twothousandten] Boschi \etal \cite{bofb10}, Faccenna \& Becker \cite{fabe10}
\item[\twothousandfourteen] Chertova \etal \cite{chsv14,chsg14}, van Hinsbergen \etal \cite{vavs14},
                            Magni \etal \cite{mafv14}
\item[\twothousandsixteen] Menant \etal \cite{mesj16}
\item[\twothousandeighteen] Spakman \etal \cite{spcv18}
\item[\twothousandnineteen] Gueydan \etal \cite{gumt19}
\item[\twothousandtwenty] Blom \etal \cite{blgf20}, Faccenna \& Becker \cite{fabe20}, 
                          van den Broek \& Gaina \cite{vaga20}, Negredo \etal \cite{nemc20}
\item[\twothousandtwentyone] Erdos \etal \cite{erhf21}
\end{itemize}
\end{scriptsize}

%..........................
\item {New Zealand} \index{topics}{New Zealand}

\begin{scriptsize}
\begin{itemize}
\item[\nineteenninety] Koons \cite{koon90}
\item[\nineteenninetyfive] Braun \& Beaumont \cite{brbe95}
\item[\nineteenninetysix] Beaumont \etal \cite{bekh96}
\item[\nineteenninetyeight] Waschbusch \etal \cite{wabb98}
\item[\nineteenninetynine] Batt \& Braun \cite{babr99}
\item[\twothousandtwo] Gerbault \etal \cite{gedh02}, Pysklywec \etal \cite{pybf02}
\item[\twothousandthree] Gerbault \etal \cite{gehd03}, Koons \etal \cite{konc03}, Upton \etal \cite{upke03}
\item[\twothousandsix] Liu \& Bird \cite{libi06}
\item[\twothousandseven] Upton \& Koons \cite{upko07}
\item[\twothousandnine] Upton \etal \cite{upkc09}
\item[\twothousandten] Pysklywec \etal \cite{pyeg10}, Spasojevic \etal \cite{spgs10a}
\item[\twothousandtwelve] Grigull \etal \cite{grel12}
\item[\twothousandthirteen] Stern \etal \cite{sths13}
\item[\twothousandsixteen] Ellis \etal \cite{elwr16}
\end{itemize}
\end{scriptsize}

%..........................
\item {Zagros} \index{topics}{Zagros}

\begin{scriptsize}
\cite{rabh97}
\cite{mozl05}
\cite{vech06}
\cite{hamo10}
\cite{yakm11}
\cite{nipc13}
\cite{frba14}
\cite{ghbu14}
\cite{coyc16}
Ruh \etal \cite{rugb17}
\end{scriptsize}

%..........................
\item {Himalayan region, Tibetan plateau, India collision} 
\index{topics}{Himalayan region}
\index{topics}{Tibetan plateau}

\begin{scriptsize}
\begin{itemize}
\item[\nineteenseventyfive] 
\cite{mota75}
\item[\nineteenseventyseven]  
\cite{mota77}
\item[\nineteenseventyeight] 
\cite{bird78}
\item[\nineteeneightytwo] 
\cite{tapl82}
\cite{vidm82}
\cite{engl82}
\item[\nineteeneightyfour] 
\cite{vidm84}
\item[\nineteeneightysix] 
\cite{vimd86}
\cite{moln86} 
\cite{enho86}
\item[\nineteeneightyseven] 
\cite{zhyu87b} 
\item[\nineteeneightyeight] 
\cite{peta88}
\cite{daco88} 
\cite{coda88}
\item[\nineteeneightynine] 
\cite{moln89}
\item[\nineteenninety] 
\cite{jodc90}
\item[\nineteenninetythree] 
\cite{moem93}
\cite{hoen93}
\cite{avta93}
\item[\nineteenninetyfour] Willett \& Beaumont \cite{wibe94}
\item[\nineteenninetyfive] Chalaron \etal \cite{chmm95}, Lenardic \etal \cite{leka95}
\item[\nineteenninetyseven] Royden \etal \cite{robk97}, England \& Molnar \cite{enmo97}, 
                            Neil \& Houseman \cite{neho97}
\item[\nineteenninetyeight] McCaffrey \& Nabelek \cite{mcna98}, Hodges \cite{hodg98}
\item[\nineteenninetynine] van der Voo \etal \cite{vasb99}, Burg \& Podlachikov \cite{bupo99}
\item[\twothousand] Chen \etal \cite{chbl00}, Clark \& Royden \cite{clro00}, 
                    Holt \cite{holt00}, Braitenberg \etal \cite{brzf00}
\item[\twothousandone] Beaumont \etal \cite{bejn01}, Lav{\'e} \& Avouac \cite{laav01}, 
                       Zeitler \etal \cite{zemk01}, Tapponnier \etal \cite{tazr01}
\item[\twothousandtwo] Koons \etal \cite{kozc02}, Jackson \cite{jack02}
\item[\twothousandthree] Replumaz \& Tapponnier \cite{reta03}
\item[\twothousandfour] Beaumont \& al \cite{bejn04}, Jamieson \etal \cite{jabm04}, 
                        Zhang \etal \cite{zhsw04}, Replumaz \& al \cite{rekv04}, 
                        Kapp \& Guynn \cite{kagu04}, Berger \etal \cite{bejh04}
\item[\twothousandfive] Clark \etal \cite{clbr05}, Robl \& Stuwe \cite{rost05a,rost05b}
\item[\twothousandsix] Clark \etal \cite{clrw06}, Jamieson \etal \cite{jabn06}, 
                       Godard \etal \cite{golc06}, Jimenez-Munt \& Platt \cite{jipl06}
\item[\twothousandseven] Meade \cite{mead07}, Hetenyi \etal \cite{hecb07},
                         Ismail-Zadeh \etal \cite{isls07}
\item[\twothousandeight] Burg \& Schmalholz \cite{busc08}, Stuwe \etal \cite{strh08},
                         Cook \& Royden \cite{coro08}
\item[\twothousandten] 
\textcite{hamo10} \citetitle{hamo10}\\
\textcite{joha10} \citetitle{joha10}\\
\textcite{luli10} \citetitle{luli10}\\
\item[\twothousandeleven] 
\textcite{befa11} \citetitle{befa11}\\ 
\textcite{zhxy11} \citetitle{zhxy11}\\
\textcite{vasd11} \citetitle{vasd11}\\
\textcite{jabe11} \citetitle{jabe11}\\
\textcite{iahb11} \citetitle{iahb11}\\
\textcite{seep11} \citetitle{seep11}\\
\item[\twothousandtwelve] 
\textcite{zams12} \citetitle{zams12}\\ 
\textcite{vald12} \citetitle{vald12}\\
\item[\twothousandthirteen] 
\textcite{care13} \citetitle{care13}\\
\textcite{chgz13} \citetitle{chgz13}\\
\textcite{chgz13b} \citetitle{chgz13b}\\
\item[\twothousandfourteen] 
\textcite{whbb14} \citetitle{whbb14}\\ 
\textcite{mutg14} \citetitle{mutg14}\\
\textcite{stjm14} \citetitle{stjm14}\\
\textcite{lesh14} \citetitle{lesh14}\\
\item[\twothousandfifteen] 
\textcite{puka15} \citetitle{puka15}\\
\textcite{jarh15} \citetitle{jarh15}\\
\textcite{yoha15} \citetitle{yoha15}\\
\item[\twothousandsixteen] 
\textcite{kebb16} \citetitle{kebb16}\\
\textcite{staj16} \citetitle{staj16}\\
\textcite{fezl16} \citetitle{fezl16}\\
\item[\twothousandseventeen] 
\textcite{bube17} \citetitle{bube17}\\
\item[\twothousandeighteen] 
\textcite{pirf18} \citetitle{pirf18}\\
\textcite{pukp18} \citetitle{pukp18}\\
\textcite{flbb18} \citetitle{flbb18}\\
\textcite{jofb18} \citetitle{jofb18}\\
\item[\twothousandnineteen] 
\textcite{sccs19} \citetitle{sccs19}\\
\textcite{scvm19} \citetitle{scvm19}\\
\textcite{wazg19} \citetitle{wazg19}\\
\textcite{scdh19} \citetitle{scdh19}\\
\item[\twothousandtwenty] 
\textcite{livn20} \citetitle{livn20}\\
\textcite{chlc20} \citetitle{chlc20}\\
\textcite{pust20} \citetitle{pust20}\\
\textcite{yakl20} \citetitle{yakl20}\\
\textcite{ghbm20} \citetitle{ghbm20}\\
\item[\twothousandtwentyone] 
\textcite{famu21} \citetitle{famu21}\\
\textcite{pels21} \citetitle{pels21}\\
\textcite{pirc21} \citetitle{pirc21}\\
\textcite{cull21} \citetitle{cull21}\\
\end{itemize}
\end{scriptsize}

%..........................
\item {Pyrenees} \index{topics}{Pyrenees}

\begin{scriptsize}
\begin{itemize}
\item[\nineteenninetyone]   \textcite{chvd91} 
\item[\nineteenninetytwo]   \textcite{chou92}
\item[\nineteenninetythree] \textcite{qubh93}
\item[\nineteenninetyeight] \textcite{giju98}
\item[\twothousand]         \textcite{bemh00}
\item[\twothousandfour] McClay \etal \cite{mcmg04}, Sibuet \etal \cite{siss04}
\item[\twothousandten] Jammes \etal \cite{jaml10}
\item[\twothousandtwelve] Vissers \& Meijer \cite{vime12}
\item[\twothousandthirteen] Fillon \etal \cite{fihv13b}
\item[\twothousandfourteen] Jammes \etal \cite{jahm14}
\item[\twothousandnineteen] Duretz \etal \cite{dual19}, Jourdon \etal \cite{jolm19}
\end{itemize}
\end{scriptsize}

%..........................
\item{Caribbean} \index{topics}{Caribbean region}

\begin{scriptsize}
\begin{itemize}
\item[\twothousandten] van Benthem \& Govers \cite{vago10}
\item[\twothousandthirteen] van Benthem \etal \cite{vags13}
\item[\twothousandfourteen] Boschman \etal \cite{bovt14}, van Benthem \etal \cite{vagw14},
                            Nerlich \etal \cite{necb14}
\item[\twothousandfifteen] Hodges \& Miller \cite{homi15}, Nerlich \etal \cite{necb15}
\item[\twothousandtwenty] Philippon \etal \cite{phvb20}, Munch \etal \cite{mugu20}
\item[\twothousandtwentyone] Gomez-Garcia \etal\cite{gols21}, Braszus \etal \cite{brga21},
                             Chen \etal \cite{chcb21}
\end{itemize}
\end{scriptsize}

%..................................................
\item{East mediterranean - Aegean region, Turkey} 
\index{topics}{Aegean region}
\index{topics}{Turkey}
\index{topics}{Anatolia}

\begin{scriptsize}
\begin{itemize}
\item[\nineteenseventyeight] McKenzie \cite{mcke78b}
\item[\nineteenninetynine] Gautier \etal \cite{gabm99}
\item[\twothousandthree] Provost \etal \cite{prch03}
\item[\twothousandten] Capitanio \etal \cite{cazf10}
\item[\twothousandeleven] Endrun \etal \cite{enlm11}
\item[\twothousandthirteen] Jolivet \etal \cite{jofh13}
\item[\twothousandseventeen] Ozbakir \etal \cite{ozgw17}
\item[\twothousandtwenty] Rolland \etal \cite{rohb20}, Fernandez-Blanco \etal \cite{femb20}
\item[\twothousandtwentyone] 
\textcite{femc21} \citetitle{femc21}\\
\textcite{sepg21} \citetitle{sepg21}
\end{itemize}
\end{scriptsize}

%...............................................................
\item{Ethiopian and Afar rift, Malawi Rift, East African rift} 
\index{topics}{Afar rift}
\index{topics}{Malawi rift}
\index{topics}{East African rift}

\begin{scriptsize}
\begin{itemize}
\item[\twothousandseven] Mickus \etal \cite{mitk07}
\item[\twothousandeight] Corti \cite{cort08}
\item[\twothousandnine] Keranen \etal \cite{kekj09}
\item[\twothousandten] Beutel \etal \cite{beve10}
\item[\twothousandfourteen] Philippon \etal \cite{phcs14}, Saria \etal \cite{sacs14}
\item[\twothousandfifteen] Fadel \etal \cite{favk15}
\item[\twothousandseventeen] Brune \etal \cite{brcr17} 
\item[\twothousandnineteen] Corti \etal \cite{cocf19}, La Rosa \etal \cite{lapk19}, Njinju \etal \cite{njas19}
\item[\twothousandtwenty] Glerum \etal \cite{glbs20}, Stamps \etal \cite{stkf20}, 
                          Petrunin \etal \cite{peke20}, Muluneh \etal \cite{mubi20},
                          Chang \etal \cite{chkd20}
\item[\twothousandtwentyone] Njinju \etal \cite{njsn21} 
\end{itemize}
\end{scriptsize}

%......................................................
\item{Alaskan region} \index{topics}{Alaskan region}

\begin{scriptsize}
\begin{itemize}
\item[\twothousandten] Koons \etal \cite{kohp10}, Jadamec \& Billen \cite{jabi10a}
\item[\twothousandtwelve] Jadamec \& Billen \cite{jabi12}
\item[\twothousandthirteen] Jadamec \etal \cite{jabr13}
\item[\twothousandseventeen] Haynie \& Jadamec \cite{haja17}
\item[\twothousandeighteen] Miller \& Moresi \cite{mimo18}
\end{itemize}
\end{scriptsize}

%..........................
\item{Farallon plate} \index{topics}{Farallon plate}
{\scriptsize
\textcite{lisg08} \citetitle{lisg08}
\textcite{list11} \citetitle{list11}
\textcite{list12} \citetitle{list12}
\textcite{licu16} \citetitle{licu16}
}

%..........................
\item{Japan, Izu-Bonin} 
\index{topics}{Japan} 
\index{topics}{Izu-Bonin}
\index{topics}{Tohoku-Hokkaido}

\begin{scriptsize}
\begin{itemize}
\item[\nineteeneightyfive]
\textcite{hond85} \citetitle{hond85}\\
\item[\nineteenninetytwo]
\textcite{stbl92} \citetitle{stbl92}\\
\item[\twothousandseven]
\textcite{lohd07} \citetitle{lohd07}\\
\item[\twothousandnine]
\textcite{obyf09} \citetitle{obyf09}\\
\item[\twothousandtwelve]
\textcite{vakn12} \citetitle{vakn12}\\
\item[\twothousandthirteen]
\textcite{musi13} \citetitle{musi13}\\
\textcite{moho13} \citetitle{moho13}\\
\item[\twothousandfourteen]
\textcite{leli14} \citetitle{leli14}\\
\textcite{kigk14} \citetitle{kigk14}\\
\textcite{mova14} \citetitle{mova14}\\
\textcite{hond14} \citetitle{hond14}\\
\item[\twothousandfifteen]
\textcite{kilk15} \citetitle{kilk15}\\
\textcite{arib15} \citetitle{arib15}\\
\item[\twothousandseventeen]
\textcite{yagz17} \citetitle{yagz17}\\
\item[\twothousandnineteen]
\textcite{yamg19} \citetitle{yamg19}\\
\item[\twothousandtwenty]
\textcite{mapg20} \citetitle{mapg20}\\
\item[\twothousandtwentyone]
\textcite{mota21} \citetitle{mota21}\\ 
\end{itemize}
\end{scriptsize}

%..........................
\item{Tonga-Kermadec subduction zone, Fiji} \index{topics}{Tonga-Kermadec subduction}

\begin{scriptsize}
\begin{itemize}
\item[\twothousandthree] 
\textcite{bigs03}, \textcite{bigu03}
\item[\twothousandsix] 
\textcite{zhpy06}
\item[\twothousandsixteen] 
\textcite{chff16}
\item[\twothousandseventeen] 
\textcite{wewv17}
\item[\twothousandtwentyone] 
\textcite{ligl21}
\end{itemize}
\end{scriptsize}

%..........................
\item{Western United States}


\begin{scriptsize}
\begin{itemize}
\item[\nineteenninetytwo]
\textcite{stbl92} \citetitle{stbl92}\\
\item[\twothousand]
\cite{honk00}
\cite{lors00}
\item[\twothousandsix]
\cite{besb06}
\cite{legs06}
\item[\twothousandtwelve]
\cite{luli12}
\item[\twothousandeight]
\cite{pehu18}
\item[\twothousandtwentyone]
\cite{chap21}
\end{itemize}
\end{scriptsize}


%..........................
\item{Southeastern United States}

\begin{scriptsize}
Heron \etal \cite{heps19}
\end{scriptsize}


%..........................
\item Australian plate \index{topics}{Australian plate}
{\scriptsize
\cite{himu03}\cite{wemv03}\cite{pymi03}\cite{onml03}
\cite{onmj05}
\cite{hazs10}\cite{dimg10}
\cite{mahg11}\cite{digm11}
\cite{gosk14}
\cite{scsp15}
Mather \etal (2019) \cite{mamr19}
}
%..........................
\item Barents sea \index{topics}{Barents sea}
{\scriptsize
\cite{buto07b}
\cite{gahs13}
\cite{gahs14}
}
%..........................
\item Carpathians \index{topics}{Carpathians}
{\scriptsize
\cite{clbm04}
\cite{isms05}
\cite{nehe06}
\cite{sepg19}
}
%..........................
\item African continent \index{topics}{African Continent}
{\scriptsize
\cite{gikb94}
Pysklywec \& Mitrovica \cite{pymi99},
\cite{vabt11}
\cite{busm12}
\cite{gagb14}
\cite{wakc17}
Guillocheau \etal \cite{gusb18},
\cite{cels20}
}
%..........................
\item Hawaii \index{topics}{Hawaii}

\begin{scriptsize}
\begin{itemize}
\item[\nineteeneightyeight] Ribe \cite{ribe88}
\item[\nineteenninetysix] Richards \& Lithgow-Bertelloni \cite{rili96}
\item[\nineteenninetyeight] Moore \etal \cite{most98}
\item[\nineteenninetynine] Ribe \& Christensen \cite{rich99}
\item[\twothousand] Cserepes \etal \cite{cscr00} 
\item[\twothousandthree] van Hunen \& Zhong \cite{vazh03}
\item[\twothousandfour] Ribe \cite{ribe04}
\item[\twothousandeight] Got \etal \cite{gomm08}
\item[\twothousandnine] Tarduno \etal \cite{tabs09}
\item[\twothousandeleven] Asaadi \etal \cite{asrs11}
\item[\twothousandtwelve] Cadio \etal \cite{cabp12}
\item[\twothousandthirteen] Zhong \& Watts \cite{zhwa13}, Plattner \etal \cite{plab13}
\item[\twothousandnineteen] Bono \etal \cite{botb19}
\item[\twothousandtwenty] Wei \etal \cite{wesl20}
\end{itemize}
\end{scriptsize}

%..........................
\item Hellenic zone/ Greece \index{topics}{Greece/Hellenic area} 
{\scriptsize
\cite{spwv88}
\cite{guhf13}
\cite{olpr14}
}
%..........................
\item Gibraltar zone \index{topics}{Gibraltar area}
\begin{scriptsize}
\begin{itemize}
\item[\twothousandtwo] Gutscher \etal \cite{gumr02}, Negredo \etal \cite{nebs02}
\item[\twothousandeight] Valera \etal \cite{vanv08}
\item[\twothousandten] Fullea \etal \cite{fufa10}
\item[\twothousandthirteen] Miller \etal \cite{miab13}, Alpert \etal \cite{almb13}
\item[\twothousandfifteen] Meriaux \etal \cite{medd15}, Fullea \etal \cite{furc15}
\item[\twothousandsixteen] Neres \etal \cite{necf16}
\item[\twothousandnineteen] Capella \etal \cite{casv19}, Jimenez-Munt \etal \cite{jitf19}
\item[\twothousandtwentyone] Fullea \etal \cite{func21}
\end{itemize}
\end{scriptsize}
%..........................
\item Norway \index{topics}{Norway}
{\scriptsize
\cite{soma13}
\cite{bubj15}
}
%..........................
\item Canyonlands \index{topics}{Canyonlands}
{\scriptsize
\cite{trca94}
\cite{scwa02}
\cite{grsk03}
}
%..........................
\item Dead Sea \index{topics}{Dead Sea}
{\scriptsize
\cite{sopg05},
Deves \etal \cite{dekk11}
}
%..........................
\item Canada \index{topics}{Canada}
{\scriptsize
Royden \& Keen \cite{roke80}
\cite{brbw93}
\cite{pelj99}
}
%..........................
\item Basin and Range \index{topics}{Basin and Range}
{\scriptsize
\cite{brbe89c}
\cite{wefr09}
}
%..........................
\item Yellowstone \index{topics}{Yellowstone}
{\scriptsize
Chaves \& Ussami \cite{chus13},
Reuber \etal \cite{rekp18}
}

%..........................
\item China, South China Sea
\index{topics}{China}
\index{topics}{South China Sea}
{\scriptsize
\cite{zhst10}
\cite{wazh15}
\cite{guyr16}
\cite{lixs19}
\cite{dawl20}
Li \etal \cite{lisy20},
Qi \etal (2021) \cite{qill21}
}
%..........................
\item Arabian plate \index{topics}{Arabian plate}
{\scriptsize
\cite{rerl15}
}
%..........................
\item Scotia plate \index{topics}{Scotia Plate}
{\scriptsize
\cite{necb13}
\cite{vaga20}
\cite{vasv21}
}
%..........................
\item Cantabria \& North-Iberian margin \index{topics}{Cantabria \& North-Iberian margin}
{\scriptsize
\cite{clbb02}
\cite{peap15}
}
%..........................
\item South East Asia \index{topics}{South East Asia}
{\scriptsize
Lesne \etal \cite{lecd00}\\
\cite{rekv04}
\cite{yotr15}\cite{hasp15}\cite{meds15}
\cite{necg16}
}
%..........................
\item Colorado plateau \index{topics}{Colorado plateau}
{\scriptsize
Bird (1979) \cite{bird79}
\cite{vabv10}
\cite{lesm11}
}
%..........................
\item Antarctica  \index{topics}{Antarctica}

\begin{scriptsize}
Huerta \& Harry \cite{huha07},
Whitehouse \etal \cite{whbl12},
Bredow \& Steinberger \cite{brst21}
\end{scriptsize}

%..........................
\item Greenland  \index{topics}{Greenland}
{\scriptsize
\cite{stsj15}\cite{heps15}\cite{stbl19}
}
%..........................
\item Atlas, Morroco  \index{topics}{Atlas, Morroco}
{\scriptsize
\cite{mica12}
\cite{kava14}
}
%..........................
\item Taiwan  \index{topics}{Taiwan}
{\scriptsize
Chemenda \etal \cite{chys01}, Fuller \etal \cite{fuwf06}, Lin \& Kuo \cite{liku16},
Wang \etal \cite{wakz19}
}
%..........................
\item Madagascar \index{topics}{Madagascar}
\begin{scriptsize}
\twothousandtwenty \textcite{rasf20} 
\end{scriptsize}

%..........................
\item Mariana Trench  \index{topics}{Mariana Trench}
{\scriptsize
\textcite{zhlb15}
}

%..........................
\item Pannonian Basin \index{topics}{Pannonian Basin}

\begin{scriptsize}
Huismans \etal \cite{hupc01b},
Huismans \etal \cite{hupc02},
Koptev \etal \cite{kock21}
\end{scriptsize}

%..........................
\item Scandinavia  
\index{topics}{Scandinavia}
\index{topics}{Scandinavian Caledonides}
{\scriptsize
\cite{ramb80}
\cite{bovc14}
}
%..........................
\item Iran
\index{topics}{Iran}

\begin{scriptsize}
Bonini \etal \cite{bocs03},
Vernant \&  Chery \cite{vech06},
Hatzfeld \& Molnar \cite{hamo10},
Yamato \etal \cite{yakm11},
Nilfouroushan \etal \cite{nipc13},
Francois \etal \cite{frba14},
Collignon \etal \cite{coyc16},
Mousavi \& Fullea \cite{mofu20}
\end{scriptsize} 
 

%..........................
\item Iceland
\index{topics}{Iceland}

\begin{scriptsize}
White \cite{whit89},
Jull \& McKenzie (1996) \cite{jumc96},
Bijwaard \& Spakman  (1999) \cite{bisp99},
Ritsema \etal (1999) \cite{rivw99},
Acocella \etal  (2000) \cite{acgf00},
Koptev \etal (2017) \cite{kocb17},
Barnett-Moore \etal (2017) \cite{bahf17},
Steinberger \etal (2019) \cite{stbl19},
Ribe \etal (2020) \cite{rits20}
\end{scriptsize} 

%..........................
\item Pacific 
\index{topics}{Pacific}

\begin{scriptsize}
\nineteensixtyseven: McKenzie \& Parker \cite{mcpa67}\\
\nineteeneighty: Watts \etal \cite{wabr80}\\
\nineteeneightytwo: Ribe \& Watts \cite{riwa82}\\ 
\nineteenninety: Jolivet \etal \cite{jodc90}\\
\twothousandfive: van Hunen \etal \cite{vazs05}, McNamara \& Zhong \cite{mczh05a}\\
\twothousandten: Zhu \etal \cite{zhst10}\\
\twothousandeleven: Cadio \etal \cite{capd11}\\
\twothousandthirteen: Nerlich \etal \cite{necb13}, Key \etal \cite{kecl13}, Ballmer \etal \cite{bacs13}\\
\twothousandfifteen: Seton \etal \cite{sefw15}, Nerlich \etal \cite{necb15}\\
\twothousandseventeen: Stotz \etal \cite{stid17}, Tondi \etal \cite{togr17}, Egushi \cite{egim17}\\
\twothousandeighteen: Yang \etal \cite{yamz18}\\
\twothousandnineteen: Wessel \& Conrad \cite{weco19}, Schellart \etal \cite{sccs19}\\
\twothousandtwenty: Maunder \cite{mapg20}
\end{scriptsize}



%..........................
\item Variscan 
\index{topics}{Variscan}

\begin{scriptsize}
\nineteenninetynine: Vanbrabant \etal \cite{vajh99}\\
\twothousandfour: Fischer \etal \cite{fijj04} \\
\twothousandseven: Marotta \& Spalla \cite{masp07} \\
\twothousandthirteen: Regorda \etal \cite{rems13} \\
\twothousandseventeen: \cite{regorda} \\
\twothousandeighteen: Gerbault \etal \cite{gesr18} \\
\twothousandtwenty: Regorda \etal \cite{relr20}\\
\twothousandtwentyone: Maierova \etal \cite{mass21}
\end{scriptsize}


\end{itemize}








\chapter{Celestial bodies} %\section{Celestical bodies}


%....................................
\section{Mercury}

\begin{small}
\begin{itemize}
\item[\twothousandseven] 
\fullcite{reki07} 
\item[\twothousandeight] 
\fullcite{king08} 
\item[\twothousandtwelve] 
\fullcite{roba12} 
\item[\twothousandtwentyone] 
\fullcite{gult21} 
\item[\twothousandtwentytwo] 
\fullcite{xihz22} 
\end{itemize}
\end{small}

 
%....................................
\section{Venus}

\begin{small}
\begin{itemize}
\item[\nineteensixtynine]
\textbullet\fullcite{scto69} 
\item[\nineteenninety] 
\textbullet\fullcite{scbg90}\\ 
\textbullet\fullcite{sozh90} 
\item[\nineteenninetyone] 
\textbullet\fullcite{lekb91} \\
\textbullet\fullcite{leyu91} 
\item[\nineteenninetytwo] 
\textbullet\fullcite{kiha92} \\
\textbullet\fullcite{sqjs92} \\
\textbullet\fullcite{mcfj92} 
\item[\nineteenninetythree] 
\textbullet\fullcite{kief93} \\
\textbullet\fullcite{lekb93} \\
\textbullet\fullcite{ogaw93} 
\item[\nineteenninetyfive] 
\textbullet\fullcite{lekb95} \\
\textbullet\fullcite{kaul95} \\
\textbullet\fullcite{mopa95}  \\
\textbullet\fullcite{scsa95} 
\item[\nineteenninetysix] 
\textbullet\fullcite{somo96} \\ 
\textbullet\fullcite{foob96} 
\item[\nineteenninetyseven] 
\textbullet\fullcite{mang97} 
\item[\nineteenninetyeight] 
\textbullet\fullcite{mazk98}  \\
\textbullet\fullcite{resm98}  \\
\textbullet\fullcite{moso98}  \\
\textbullet\fullcite{phha98} 
\item[\nineteenninetynine] 
\textbullet\fullcite{resm99} 
\item[\twothousand] 
\textbullet\fullcite{ogaw00} 
\item[\twothousandthree] 
\textbullet\fullcite{vesh03} 
\item[\twothousandfour] 
\textbullet\fullcite{vesb04} 
\item[\twothousandfive] 
\textbullet\fullcite{vavv05} 
\item[\twothousandseven] 
\textbullet\fullcite{reso07} 
\item[\twothousandten] 
\textbullet\fullcite{stfh10}  \\
\textbullet\fullcite{stwt10} 
\item[\twothousandeleven] 
\textbullet\fullcite{orso11} 
\item[\twothousandtwelve] 
\textbullet\fullcite{arta12}  \\
\textbullet\fullcite{orso12}  \\
\textbullet\fullcite{nobs12} 
\item[\twothousandthirteen] 
\textbullet\fullcite{huyz13} 
\item[\twothousandfourteen] 
\textbullet\fullcite{gita14}  \\
\textbullet\fullcite{gery14b} 
\item[\twothousandfifteen] 
\textbullet\fullcite{ghai15} 
\item[\twothousandseventeen] 
\textbullet\fullcite{cram17}  \\
\textbullet\fullcite{dast17} 
\item[\twothousandeighteen] 
\textbullet\fullcite{king18}  \\
\textbullet\fullcite{ross18} 
\item[\twothousandtwenty] 
\textbullet\fullcite{weki20}  \\
\textbullet\fullcite{gugm20}  \\
\textbullet\fullcite{uprc20}  \\
\textbullet\fullcite{kacc20} 
\item[\twothousandtwentyone] 
\textbullet\fullcite{macg21}  \\
\textbullet\fullcite{bygs21} 
\item[\twothousandtwentytwo]
\textbullet\fullcite{adss22}  \\
\textbullet\fullcite{bamo22}  \\
\textbullet\fullcite{mawi22}  \\
\textbullet\fullcite{rowg22} 
\item[\twothousandtwentythree]
\textbullet\fullcite{smoo23}  \\
\textbullet\fullcite{lour23}  \\
\textbullet\fullcite{titl23}  \\
\textbullet\fullcite{mawp23}  \\
\textbullet\fullcite{adsm23}  \\
\textbullet\fullcite{guyg23}  \\
\textbullet\fullcite{hanm23} 
\item[\twothousandtwentyfour]
\end{itemize}
\fullcite{caks24} \\
\fullcite{vamp24} 
\end{small}

%....................................
\section{Moon}

\begin{small}
\begin{itemize}
\item[\nineteenseventy]
\fullcite{tuox70}
\item[\nineteenseventytwo]
\fullcite{tuht72}
\item[\nineteenseventythree]
\fullcite{care73}
\item[\nineteenseventyfour]
\fullcite{care74}
\item[\nineteenseventynine]
\fullcite{carg79}
\item[1998]
\fullcite{alpa98}
\item[\twothousandone] 
\fullcite{spkb01}
\item[\twothousandtwo] 
\fullcite{elvh02} 
\item[\twothousandthree] 
\fullcite{stjz03} 
\item[\twothousandfour] 
\fullcite{elhg04} 
\item[\twothousandten] 
\fullcite{devv10} 
\item[\twothousandtwelve] 
\fullcite{zhqa12} 
\item[\twothousandthirteen] 
\fullcite{dejv13} 
\item[\twothousandsixteen] 
\fullcite{qizw16} 
\item[\twothousandseventeen] 
\fullcite{jaal17} 
\item[\twothousandeighteen] 
\fullcite{qizp18} 
\item[\twothousandnineteen] 
\fullcite{zhdv19} 
\item[\twothousandtwentytwo]
\fullcite{javs22} \\ 
\fullcite{faab22} 
\item[\twothousandtwentythree]
\fullcite{zhzl23} \\
\fullcite{yuld23} \\
\fullcite{ukog23}
\item[\twothousandtwentyfour]
\fullcite{fizm24}
\end{itemize}
\end{small}





%....................................
\section{Mars}

Mars fact sheet: \url{https://nssdc.gsfc.nasa.gov/planetary/factsheet/marsfact.html}

\begin{small}
\begin{itemize}
\item[\nineteensixtynine]
\fullcite{scto69} 
\item[\nineteeneightytwo] 
\fullcite{baps82}  \\
\fullcite{witu82}  \\
\fullcite{sohe82} 
\item[\nineteenninety] 
\fullcite{scbg90}  \\
\fullcite{thsc90} 
\item[\nineteenninetyone] 
\fullcite{spoh91}  \\
\fullcite{jaer91} 
\item[\nineteenninetyfour] 
\fullcite{slee94}
\item[\nineteenninetysix] 
\fullcite{hach96}  \\
\fullcite{brzy96}  \\
\fullcite{kibn96}  \\
\fullcite{mema96} 
\item[\nineteenninetyseven]  
\fullcite{brys97} 
\item[\nineteenninetyeight] 
\fullcite{resm98}  \\
\fullcite{hard98}  \\
\fullcite{befe98}  \\
\fullcite{wuha98}  \\
\fullcite{brys98}  
\item[\nineteenninetynine] 
\fullcite{smst99} 
\item[\twothousand] 
\fullcite{hard00} 
\item[\twothousandone] 
\fullcite{nist01}  \\
\fullcite{zube01}  \\
\fullcite{scvy01} 
\item[\twothousandtwo] 
\fullcite{resb02}  \\
\fullcite{zhon02}  \\
\fullcite{haph02}  \\
\fullcite{mcby02}  \\
\fullcite{scvy02} 
\item[\twothousandthree] 
\fullcite{zhro03}  \\
\fullcite{lozh03}  \\
\fullcite{kief03} 
\item[\twothousandfour] 
\fullcite{lenm04}  \\
\fullcite{vavv04c}  \\
\fullcite{resb04}  \\
\fullcite{reki04}  \\
\fullcite{rozh04} 
\item[\twothousandfive]  
\fullcite{vavv05}  \\
\fullcite{elzp05}  \\
\fullcite{onml05}  \\
\fullcite{belw05} 
\item[\twothousandsix] 
\fullcite{reso06}  \\
\fullcite{losh06}  \\
\fullcite{rozh06}  \\
\fullcite{keso06}  \\
\fullcite{koys06}  \\
\fullcite{brsp06} 
\item[\twothousandseven]
\fullcite{rozh07}  \\
\fullcite{reso07b} \\
\fullcite{liki07} 
\item[\twothousandeight] 
\fullcite{loha08}  \\
\fullcite{winm08} 
\item[\twothousandnine]
\fullcite{keta09}  \\
\fullcite{zhon09}  \\
\fullcite{rolm09}  \\
\fullcite{keso09}  \\
\fullcite{smzt09}  \\
\fullcite{habg09} 
\item[\twothousandten] 
\fullcite{srzh10}  \\
\fullcite{reos10}  \\
\fullcite{reso10}  \\
\fullcite{stwt10}  \\
\fullcite{wabh10}  \\
\fullcite{grbr10} 
\item[\twothousandeleven] 
\fullcite{gokg11}  \\
\fullcite{reos11}  \\
\fullcite{jizl11}  \\
\fullcite{koaf11}  \\
\fullcite{nasc11} 
\item[\twothousandtwelve] 
\fullcite{srzh12}  \\
\fullcite{roar12}  \\
\fullcite{hick12}  \\
\fullcite{belr12} 
\item[\twothousandthirteen] 
\fullcite{pltb13}  \\
\fullcite{ruts13}  \\
\fullcite{ruts13b} 
\item[\twothousandfourteen] 
\fullcite{seki14} \\
\fullcite{chki14} \\
\fullcite{letg14}
\item[\twothousandfifteen] 
\fullcite{kifs15} 
\item[\twothousandsixteen] 
\fullcite{zhon16}  \\
\fullcite{kili16}  \\
\fullcite{gegl16}  \\
\fullcite{bobm16} 
\item[\twothousandseventeen] 
\fullcite{rubr17}  \\
\fullcite{hema17}  \\
\fullcite{azka17} 
\item[\twothousandeighteen] 
\fullcite{cimt18}  \\
\fullcite{goej18}  \\
\fullcite{scmo18}  \\
\fullcite{khlr18}  \\
\fullcite{domk18}  \\
\fullcite{plpt18} 
\item[\twothousandnineteen] 
\fullcite{smls19}  \\
\fullcite{cahe19}  \\
\fullcite{dilg19} 
\item[\twothousandtwenty] 
\fullcite{lobp20}  \\
\fullcite{gilb20}  \\
\fullcite{agtb20}  \\
\fullcite{geno20}  \\
\fullcite{basb20}  \\
\fullcite{tajh20}  \\
\fullcite{brfi20} 
\item[\twothousandtwentyone] 
\fullcite{khcv21}  \\
\fullcite{stkb21}  \\
\fullcite{knpb21}  \\
\fullcite{vand21}  \\
\fullcite{ribc21}  \\
\fullcite{topa21}  \\
\fullcite{sabp21} 
\item[\twothousandtwentytwo]
\fullcite{wibm22}  \\
\fullcite{watk22}  \\
\fullcite{plwk22}  \\
\fullcite{bran22} 
\item[\twothousandtwentythree]
\fullcite{bajg23} \\
\fullcite{khhd23} \\
\fullcite{sadr23}
\item[\twothousandtwentyfour]
\fullcite{muki24}\\
\fullcite{chrg24}\\
\fullcite{drsg24}
\end{itemize}
\end{small}



%....................................
\section{Pluto}

\begin{small}
\begin{itemize}
\item[\twothousandsixteen] 
\fullcite{mcnw16} 
\end{itemize}
\end{small}

%....................................
\section{Super-Earths, Giant planets \& exoplanets}

\begin{small}
\begin{itemize}
\item[\twothousandsix]
\fullcite{evgl06}
\item[\twothousandeleven]
\fullcite{stfl11}  \\
\fullcite{vata11} 
\item[\twothousandtwelve]
\fullcite{evsa12} 
\item[\twothousandthirteen]
\fullcite{stlh13} 
\item[\twothousandfifteen] 
\fullcite{welo15}  \\
\fullcite{miko15}  \\
\fullcite{evon15}  \\
\fullcite{kamo15} 
\item[\twothousandtwentyone]
\fullcite{mebl21} 
\item[\twothousandtwentythree] 
\fullcite{shpy23} 
\end{itemize}
\end{small}

%....................................
\section{Icy satellites, icy moons}

Icy moons are a class of natural satellites with surfaces composed mostly of ice. 
An icy moon may harbor an ocean underneath the surface, and possibly include a rocky 
core of silicate or metallic rocks.
\url{https://en.wikipedia.org/wiki/Icy_moon}

\begin{small}
\begin{itemize}
\item[1987]
\fullcite{thsc87}
\item[1988]
\fullcite{thsq88}
\item[\twothousandone] 
\fullcite{deso01} 
\item[\twothousandtwelve] 
\fullcite{kasc12b} 
\item[\twothousandseventeen] 
\fullcite{chts17} 
\item[\twothousandnineteen] 
\fullcite{wefb19} 
\item[\twothousandtwenty] 
\fullcite{hadc20} 
\item[\twothousandtwentyone]
\fullcite{goju21} \\ 
\fullcite{cawj21}
\item[\twothousandtwentythree]
\fullcite{lelm23} 
\end{itemize}
\end{small}

%..........................................................
\section{Europa}

The Galilean satellites were first seen by the Italian astronomer 
Galileo Galilei in 1610. Io is closest, followed by Europa, Ganymede, 
and Callisto. It has a smooth and bright surface, with a layer of 
water surrounding the mantle of the planet, thought to be 100 kilometers thick.

\begin{small}
\begin{itemize}
\item[1986]
\fullcite{thsc86}
\item[\twothousandtwo] 
\fullcite{husw02} 
\item[\twothousandfour] 
\fullcite{shha04} 
\item[\twothousandfive] 
\fullcite{shha05}\\ 
\fullcite{hash05}\\ 
\fullcite{mish05} 
\item[\twothousandeight] 
\fullcite{hash08} 
\item[\twothousandten] 
\fullcite{hash10} 
\item[\twothousandeleven] 
\fullcite{hash11} 
\item[\twothousandfourteen] 
\fullcite{kast14} \\
\fullcite{awzh14} 
\item[\twothousandnineteen] 
\fullcite{almc19} 
\item[\twothousandtwentyone] 
\fullcite{cawj21}
\item[\twothousandtwentytwo] 
\fullcite{wohw22b}
\end{itemize}
\end{small}



%....................................
\section{Ceres}

\url{https://en.wikipedia.org/wiki/Ceres_(dwarf_planet)}
The robotic NASA spacecraft Dawn approached Ceres for its orbital mission in 2015.
and found Ceres's surface to be a mixture of water ice, and hydrated minerals such as carbonates and clay. 

\begin{small}
\begin{itemize}
\item[\twothousandtwentytwo] 
\fullcite{kibm22} 
\end{itemize}
\end{small}

%....................................
\section{Enceladus}

Enceladus is the sixth-largest moon of Saturn (19th largest in the Solar System). 
It is about 500 kilometers in diameter, about a tenth of that of Saturn's largest moon, Titan. 
Enceladus is mostly covered by fresh, clean ice, making it one of the most reflective bodies 
of the Solar System. 
\url{https://en.wikipedia.org/wiki/Enceladus}

\begin{small}
\begin{itemize}
\item[\twothousandeight] 
\fullcite{roni08} 
\item[\twothousandnine]
\fullcite{stfm09} 
\item[\twothousandten]
\fullcite{betc10} 
\item[\twothousandtwelve] 
\fullcite{hats12} 
\item[\twothousandthirteen] 
\fullcite{shhh13} 
\item[\twothousandfourteen]
\fullcite{robg14} 
\item[\twothousandtwentytwo] 
\fullcite{wohw22b}
\end{itemize}
\end{small}

%..........................................................
\section{Callisto}

The Galilean satellites were first seen by the Italian astronomer Galileo Galilei in 1610. 
Io is closest, followed by Europa, Ganymede, and Callisto (1.9 million km or
26.4 $R_J$ from Jupiter). Callisto has the lowest mean density of all Galilean satellites.

\begin{small}
\begin{itemize}
\item[1988]
\fullcite{mumc88} 
\item[\twothousandfour]
\fullcite{nabs04}
\item[\twothousandfive]
\fullcite{kukr05}
\item[\twothousandsix]
\fullcite{free06}
\end{itemize}
\end{small}

%..........................................................
\section{Ganymede}

The Galilean satellites were first seen by the Italian astronomer Galileo Galilei in 1610. 
Io is closest, followed by Europa, Ganymede, and Callisto.

\begin{small}
\begin{itemize}
\item[1988]
\fullcite{mumc88} \\ 
\fullcite{thsc88} 
\item[1990]
\fullcite{thsq90}
\item[\twothousandsix]
\fullcite{free06}
\item[\twothousandfourteen]
\fullcite{awzh14} 
\end{itemize}
\end{small}


%....................................
\section{Io}

The Galilean satellites were first seen by the Italian astronomer Galileo Galilei in 1610. 
Io is closest, followed by Europa, Ganymede, and Callisto.
With a diameter of 3642 kilometers, it is the fourth-largest moon in the Solar System, 
and is only marginally larger than Earth's moon.

\begin{small}
\begin{itemize}
\item[\twothousandone]
\fullcite{tasg01} \\ 
\fullcite{tack01} \\
\fullcite{mcsd01} 
\item[\twothousandthirteen] 
\fullcite{shpp13} 
\item[\twothousandtwenty] 
\fullcite{sthh20} \\ 
\fullcite{spkh20} \\
\fullcite{spkh20b} 
\item[\twothousandtwentytwo] 
\fullcite{ketc22} 
\end{itemize}
\end{small}

%....................................
\section{Planetesimals}

\begin{small}
\begin{itemize}
\item[\twothousandfourteen]
\fullcite{gobg14}
\item[\twothousandnineteen]
\fullcite{likk19} \\
\fullcite{neum19}
\item[\twothousandtwentyone]
\fullcite{goju21} 
\end{itemize}
\end{small}




\chapter{Geological areas on Earth} %%%%%%%%%%%%%%%%%%%%%%%%%%%%%%%%%%%%%%%%%%%%%%%%%%%%%%%%%%%%%%%%%%%%%%%%%%%%%%%
%AAAAAAAAAAAAAAAAAAAAAAAAAAAAAAAAAAAAAAAAAAAAAAAAAAAAAAAAAAAAAAAAAAAAAAAAAAAAAA
%%%%%%%%%%%%%%%%%%%%%%%%%%%%%%%%%%%%%%%%%%%%%%%%%%%%%%%%%%%%%%%%%%%%%%%%%%%%%%%

\section{Aegean region, Anatolia, Turkey, East mediterranean}

\begin{small}
\begin{itemize}
\item[\nineteenseventyeight] 
\fullcite{mcke78b} 
\item[\nineteenninetynine] 
\fullcite{gabm99} 
\item[\twothousandthree] 
\fullcite{prch03} 
\item[\twothousandten] 
\fullcite{cazf10} 
\item[\twothousandeleven] 
\fullcite{enlm11} 
\item[\twothousandthirteen] 
\fullcite{jofh13}\\ 
\fullcite{fabj13} 
\item[\twothousandseventeen] 
\fullcite{ozgw17} 
\item[\twothousandtwenty] 
\fullcite{rohb20} \\
\fullcite{femb20} 
\item[\twothousandtwentyone] 
\fullcite{femc21} \\
\fullcite{segp21} 
\item[\twothousandtwentythree]
\fullcite{bogb23} 
\item[\twothousandtwentyfour]
\fullcite{angp24} \\
\fullcite{seps24} 
\end{itemize}
\end{small}

\section{African continent}

\begin{small}
\begin{itemize}
\item[\nineteenninetyfour]
\fullcite{gikb94} 
\item[\nineteenninetynine]
\fullcite{pymi99} 
\item[\twothousandeleven]
\fullcite{vabt11} 
\item[\twothousandtwelve]
\fullcite{busm12} 
\item[\twothousandfourteen]
\fullcite{gagb14} 
\item[\twothousandseventeen]
\fullcite{wakc17} 
\item[\twothousandeighteen]
\fullcite{gusb18} 
\item[\twothousandtwenty]
\fullcite{cels20} 
\end{itemize}
\end{small}

\section{African-Eurasia (west of Iberia)}

\begin{small}
\begin{itemize}
\item[\twothousandthree]
\fullcite{jine03}
\item[\twothousandten]
\fullcite{jifv10}
\item[\twothousandnineteen]
\fullcite{argc19} 
\end{itemize}
\end{small}

\section{Algeria}

\begin{small}
\begin{itemize}
\item[\twothousandthree]
\fullcite{jine03}
\item[\twothousandfifteen]
\fullcite{hapa15} 
\item[\twothousandeighteen]
\fullcite{hapl18} 
\item[\twothousandtwentyone]
\fullcite{kufv21}
\end{itemize}
\end{small}

\section{Arabian plate}

\begin{small}
\begin{itemize}
\item[\twothousandthirteen] 
\fullcite{fabj13}
\item[\twothousandfifteen] 
\fullcite{rerl15}
\item[\twothousandeighteen] 
\fullcite{barj18} 
\item[\twothousandtwenty] 
\fullcite{pekk20}
\item[\twothousandtwentytwo] 
\fullcite{aryt22}
\end{itemize}
\end{small}

\section{Antarctica}

\begin{small}
\begin{itemize}
\item[\nineteenninetyeight]
\fullcite{gumm98} 
\item[\twothousandseven]
\fullcite{huha07} 
\item[\twothousandten]
\fullcite{spgs10a} 
\item[\twothousandtwelve]
\fullcite{whbl12} \\ 
\fullcite{pode12} 
\item[\twothousandthirteen]
\fullcite{awzh13} \\
\fullcite{ivjw13} 
\item[\twothousandfifteen]
\fullcite{aupm15} 
\item[\twothousandeighteen]
\fullcite{mimr18} 
\item[\twothousandtwentyone]
\fullcite{brst21} 
\item[\twothousandtwentythree]
\fullcite{stgl23} 
\item[2024]
\fullcite{goyp24}
\end{itemize}
\end{small}

\section{Arctic region}

\begin{small}
\begin{itemize}
\item[2021]
\fullcite{lora21}
\item[2023]
\fullcite{zhzz23}
\item[2024] 
\fullcite{hesc24}\\
\fullcite{lobb24}
\end{itemize}
\end{small}

\section{Atlas, Morroco}

\begin{small}
\begin{itemize}
\item[\twothousandtwelve]
\fullcite{mica12} 
\item[\twothousandfourteen]
\fullcite{kava14} 
\item[\twothousandtwentythree]
\fullcite{lafn23} 
\end{itemize}
\end{small}

\section{Atlantic ocean, opening}

\begin{small}
\begin{itemize}
\item[1966]
\fullcite{wils66}
\item[1989]
\fullcite{brbe89c}
\fullcite{whit89}
\item[1990]
\fullcite{lips90}
\item[1993]
\fullcite{nefo93}
\item[1999]
\fullcite{lays99}
\fullcite{fagd99}
\item[2009]
\fullcite{arhm09}
\item[2010]
\fullcite{albe10}
\fullcite{albs10}
\fullcite{fufa10}
\item[2011]
\fullcite{nigm11}
\fullcite{rapy11}
\item[2012]
\fullcite{hucf12}
\item[2013]
\fullcite{durt13}
\item[2014]
\fullcite{ebbf14}
\fullcite{cosb14}
\fullcite{hebr14}
\fullcite{flgw14}
\item[2015]
\fullcite{furc15}
\item[2016]
\fullcite{oles16}
\item[2017]
\fullcite{bekb17}
\fullcite{brhc17}
\fullcite{taac17}
\item[2017]
\fullcite{esmp17}
\item[2018]
\fullcite{dusr18}
\fullcite{vifb18}
\fullcite{cogb18}
\item[2019]
\fullcite{shar19}
\fullcite{stbl19}
\item[2020]
\fullcite{cump20}
\fullcite{peaa20}
\item[2021]
\fullcite{luhu21}
\item[2023]
\fullcite{scsb23}
\end{itemize}
\end{small}

\section{Alaskan region} 

\begin{small}
\begin{itemize}
\item[1996]
\fullcite{bird96}
\item[\twothousandten] 
\fullcite{kohp10}  \\
\fullcite{jabi10} 
\item[\twothousandtwelve] 
\fullcite{jabi12} 
\item[\twothousandthirteen] 
\fullcite{jabr13} 
\item[\twothousandfifteen] 
\fullcite{fifr15} 
\item[\twothousandseventeen] 
\fullcite{haja17} 
\item[\twothousandeighteen] 
\fullcite{mimo18} 
\end{itemize}
\end{small}

\section{Apennines}

\begin{small}
\begin{itemize}
\item[\nineteenninetyeight] 
\fullcite{buwg98} 
\item[\twothousandseven] 
\fullcite{shpy07} 
\item[\twothousandnine] 
\fullcite{rohu09} 
\item[\twothousandfourteen] 
\fullcite{fabm14} 
\item[\twothousandfifteen] 
\fullcite{vami15} 
\item[\twothousandtwenty] 
\fullcite{dadm20} 
\end{itemize}
\end{small}

\section{Alps}

\begin{small}
\begin{itemize}
\item[\nineteenninetysix] 
\fullcite{beeh96} 
\item[\nineteenninetyseven] 
\fullcite{repe97} 
\item[\nineteenninetyeight] 
\fullcite{desw98} 
\item[\twothousand] 
\fullcite{pfeb00} 
\item[\twothousandone] 
\fullcite{bujl01} 
\item[\twothousandtwo] 
\fullcite{pfsb02} 
\item[\twothousandthree] 
\fullcite{pimo03} 
\item[\twothousandfive] 
\fullcite{buge05} \\
\fullcite{jign05}
\item[\twothousandseven] 
\fullcite{masp07} 
\item[\twothousandeight] 
\fullcite{vifj08} 
\item[\twothousandtwelve] 
\fullcite{rosm12} 
\item[\twothousandthirteen] 
\fullcite{luws13}  \\
\fullcite{baes13}  \\
\fullcite{bubj13} 
\item[\twothousandfourteen] 
\fullcite{bubj14} 
\item[\twothousandfifteen] 
\fullcite{scdu15}  \\
\fullcite{fohk15} 
\item[\twothousandeighteen] 
\fullcite{marc18} 
\item[\twothousandnineteen] 
\fullcite{rors19}  \\
\fullcite{stsh19} 
\item[\twothousandtwenty] 
\fullcite{kids20}  \\
\fullcite{rozr20}  \\
\fullcite{relr20}  \\
\fullcite{aslr20} 
\item[\twothousandtwentyone] 
\fullcite{gupg21} 
\item[\twothousandtwentytwo] 
\fullcite{vavw22} 
\end{itemize}
\end{small}

\section{Gulf of Aden}

\begin{small}
\begin{itemize}
\item[\twothousandthree] 
\fullcite{hukm03} 
\item[\twothousandthirteen] 
\fullcite{beha13}  \\
\fullcite{brau13}  \\
\fullcite{wabd13} 
\item[\twothousandtwenty] 
\fullcite{duhm20} 
\item[\twothousandtwentytwo] 
\fullcite{bors22} 
\end{itemize}
\end{small}

%%%%%%%%%%%%%%%%%%%%%%%%%%%%%%%%%%%%%%%%%%%%%%%%%%%%%%%%%%%
\section{Australian plate}

\begin{small}
\begin{itemize}
\item[\twothousandthree]
\fullcite{himu03} \\
\fullcite{wemv03} \\
\fullcite{pymi03} \\
\fullcite{onml03} 
\item[\twothousandfive]
\fullcite{onmj05} 
\item[\twothousandten]
\fullcite{hazs10} \\
\fullcite{dimg10} 
\item[\twothousandeleven]
\fullcite{mahg11} \\
\fullcite{digm11} 
\item[\twothousandfourteen]
\fullcite{gosk14} 
\item[\twothousandfifteen]
\fullcite{scsp15} 
\item[\twothousandsixteen]
\fullcite{hepy16} 
\item[\twothousandnineteen]
\fullcite{mamr19} \\
\fullcite{smbc19} 
\item[2020]
\fullcite{fiog20} \\
\fullcite{onmb20}
\item[\twothousandtwentytwo]
\fullcite{pafl22}\\ 
\fullcite{dodl22} 
\item[\twothousandtwentythree]
\fullcite{ropr23} \\
\fullcite{rich23}
\end{itemize}
\end{small}

%%%%%%%%%%%%%%%%%%%%%%%%%%%%%%%%%%%%%%%%%%%%%%%%%%%%%%%%%%%%%%%%%%%%%%%%%%%%%%%
%BBBBBBBBBBBBBBBBBBBBBBBBBBBBBBBBBBBBBBBBBBBBBBBBBBBBBBBBBBBBBBBBBBBBBBBBBBBBBB
%%%%%%%%%%%%%%%%%%%%%%%%%%%%%%%%%%%%%%%%%%%%%%%%%%%%%%%%%%%%%%%%%%%%%%%%%%%%%%%

\section{Barents sea}

\begin{small}
\begin{itemize}
\item[\twothousandseven]
\fullcite{buto07b} 
\item[\twothousandthirteen]
\fullcite{gahs13} 
\item[\twothousandfourteen]
\fullcite{gahs14} 
\item[\twothousandfifteen]
\fullcite{rotv15} 
\end{itemize}
\end{small}

\section{Basin and Range}

\begin{small}
\begin{itemize}
\item[\nineteeneightynine]
\fullcite{brbe89c} 
\item[\twothousandnine]
\fullcite{wefr09} 
\end{itemize}
\end{small}

\section{Banda, Molucca subduction zone}

\begin{small}
\begin{itemize}
\item[\twothousandnine]
\fullcite{rohu09} 
\item[\twothousandten]
\fullcite{spha10} 
\item[\twothousandtwentyone]
\fullcite{scvg21} 
\item[\twothousandtwentytwo]
\fullcite{hura22} 
\item[\twothousandtwentyfour]
\fullcite{yuwz24}
\end{itemize}
\end{small}


\section{Brazil}

\begin{small}
\begin{itemize}
\item[2013]
\fullcite{assa13}
\item[2004] 
\fullcite{wesm04}
\item[2019]
\fullcite{sisa19}
\item[2022]
\fullcite{saup22}
\end{itemize}
\end{small}


%%%%%%%%%%%%%%%%%%%%%%%%%%%%%%%%%%%%%%%%%%%%%%%%%%%%%%%%%%%%%%%%%%%%%%%%%%%%%%%
%CCCCCCCCCCCCCCCCCCCCCCCCCCCCCCCCCCCCCCCCCCCCCCCCCCCCCCCCCCCCCCCCCCCCCCCCCCCCCC
%%%%%%%%%%%%%%%%%%%%%%%%%%%%%%%%%%%%%%%%%%%%%%%%%%%%%%%%%%%%%%%%%%%%%%%%%%%%%%%

\section{Cascadia}

\begin{small}
\begin{itemize}
\item[2004]
\fullcite{cuwh04}
\item[2009]
\fullcite{luli09}
\item[2017] 
\fullcite{mova17}
\item[2025] 
\fullcite{frbn25}
\end{itemize}
\end{small}

\section{Canary Islands}

\begin{small}
\begin{itemize}
\item[\twothousandfifteen]
\fullcite{fucn15}\\
\fullcite{medd15}
\item[\twothousandtwentytwo]
\fullcite{maba22} \\
\fullcite{nevr22}
\item[\twothousandtwentythree]
\fullcite{jink23}
\end{itemize}
\end{small}


\section{Carpathians}

\begin{small}
\begin{itemize}
\item[\twothousand]
\fullcite{wosp00} 
\item[\twothousandfour]
\fullcite{clbm04} 
\item[\twothousandfive]
\fullcite{isms05} 
\item[\twothousandsix]
\fullcite{nehe06} 
\item[\twothousandnineteen]
\fullcite{sepg19} 
\end{itemize}
\end{small}

\section{Canyonlands}

\begin{small}
\begin{itemize}
\item[\nineteenninetyfour]
\fullcite{trca94} 
\item[\twothousandtwo]
\fullcite{scwa02} 
\item[\twothousandthree]
\fullcite{grsk03} 
\end{itemize}
\end{small}

\section{Canada}

\begin{small}
\begin{itemize}
\item[\nineteeneighty]
\fullcite{roke80} 
\item[\nineteenninetythree]
\fullcite{brbw93} \\ 
\fullcite{bakp93}
\item[\nineteenninetyeight]
\fullcite{elbj98} 
\item[\nineteenninetynine]
\fullcite{pelj99} \\
\fullcite{elbe99} 
\item[\twothousandten]
\fullcite{jabw10} \\
\fullcite{albe10} \\
\fullcite{albs10} 
\item[\twothousandthirteen]
\fullcite{awzh13} 
\item[\twothousandthirteen]
\fullcite{baeg14} 
\item[\twothousandtwenty]
\fullcite{hube20} 
\item[\twothousandtwentythree]
\fullcite{hepm23} 
\end{itemize}
\end{small}

\section{China, South China Sea, East China Sea, North China Craton}

\begin{small}
\begin{itemize}
\item[\twothousandten] 
\fullcite{zhst10} 
\item[\twothousandfifteen] 
\fullcite{wazh15} 
\item[\twothousandsixteen] 
\fullcite{guyr16} \\ 
\fullcite{wahz16} 
\item[\twothousandeighteen] 
\fullcite{lecd18} \\
\fullcite{lidl18} 
\item[\twothousandnineteen] 
\fullcite{lixs19} 
\item[\twothousandtwenty] 
\fullcite{dawl20} \\
\fullcite{peaa20} \\
\fullcite{lisy20} 
\item[\twothousandtwentyone] 
\fullcite{qill21} \\
\fullcite{yalz21} 
\item[\twothousandtwentytwo] 
\fullcite{lilg22} \\ 
\fullcite{wuwh22} \\
\fullcite{maly22} 
\item[\twothousandtwentythree] 
\fullcite{zhzw23} \\
\fullcite{su__23} 
\item[\twothousandtwentyfour] 
\fullcite{ficd24} \\ 
\fullcite{lizm24} \\ 
\fullcite{suzl24} \\ 
\fullcite{libe24} \\ 
\fullcite{licc24}
\item[\twothousandtwentyfive] 
\fullcite{pasp25} \\
\fullcite{yays25}
\end{itemize}
\end{small}

\section{Colorado plateau}

\begin{small}
\begin{itemize}
\item[\nineteenseventynine]
\fullcite{bird79} 
\item[\twothousandten]
\fullcite{vabv10} 
\item[\twothousandeleven]
\fullcite{lesm11}
\item[\twothousandsixteen]
\fullcite{rogj16} 
\item[\twothousandtwentythree] 
\fullcite{heka23}
\end{itemize}
\end{small}

\section{Caribbean region, plate} 

\begin{small}
\begin{itemize}
\item[\twothousandfour] 
\fullcite{nejv04} 
\item[\twothousandeight] 
\fullcite{caon08} 
\item[\twothousandten] 
\fullcite{vago10} 
\item[\twothousandthirteen] 
\fullcite{vags13} 
\item[\twothousandfourteen] 
\fullcite{bovt14} \\
\fullcite{vagw14} \\
\fullcite{necb14} 
\item[\twothousandfifteen] 
\fullcite{homi15} \\
\fullcite{necb15} 
\item[\twothousandtwenty] 
\fullcite{phvb20} \\
\fullcite{mugu20} 
\item[\twothousandtwentyone] 
\fullcite{gols21} \\
\fullcite{brga21} \\
\fullcite{chcb21} \\
\fullcite{ceha21} 
\item[\twothousandtwentythree] 
\fullcite{rida23} 
\item[\twothousandtwentyfour] 
\fullcite{sawb24} \\ 
\fullcite{cofh24} 
\end{itemize}
\end{small}

%================================================
\section{Central America, Mexico, Guld of Mexico}

\begin{small}
\begin{itemize}
\item[\twothousand] 
\fullcite{gacn00}
\item[\twothousandeight] 
\fullcite{pekh08}
\item[\twothousandnine] 
\fullcite{grba09}\\
\fullcite{ladg09}
\item[\twothousandtwelve] 
\fullcite{grbe12}
\item[\twothousandfifteen] 
\fullcite{gehm15}
\item[\twothousandsixteen] 
\fullcite{naoo16}
\item[\twothousandseventeen] 
\fullcite{grbe17}
\item[\twothousandtwentythree] 
\fullcite{bamm23}
\end{itemize}
\end{small}

%==============================
\section{Chile Triple junction}

\begin{small}
\begin{itemize}
\item[2008] \fullcite{gogm08}
\item[2012] \fullcite{mapm12}
\item[2021] \fullcite{gusw21}
\item[2023] \fullcite{gusw23}
\end{itemize}
\end{small}

%%%%%%%%%%%%%%%%%%%%%%%%%%%%%%%%%%%%%%%%%%%%%%%%%%%%%%%%%%%%%%%%%%%%%%%%%%%%%%%
%DDDDDDDDDDDDDDDDDDDDDDDDDDDDDDDDDDDDDDDDDDDDDDDDDDDDDDDDDDDDDDDDDDDDDDDDDDDDDD
%%%%%%%%%%%%%%%%%%%%%%%%%%%%%%%%%%%%%%%%%%%%%%%%%%%%%%%%%%%%%%%%%%%%%%%%%%%%%%%

\section{Dead Sea} 

\begin{small}
\begin{itemize}
\item[\twothousandfive]
\fullcite{sopg05} 
\item[\twothousandeleven]
\fullcite{dekk11} 
\item[\twothousandtwentyfour]
\fullcite{hebg24} 
\end{itemize}
\end{small}

\section{Dinarides, Pannonian region} 

\begin{small}
\begin{itemize}
\item[2001]
\fullcite{hupc01b}
\item[2002]
\fullcite{hupc02}
\item[2019]
\fullcite{hulf19}
\item[2022]
\fullcite{zhjt22}
\item[2024]
\fullcite{zhjt24}
\item[2024]
\fullcite{bevp25}
\end{itemize}
\end{small}

%%%%%%%%%%%%%%%%%%%%%%%%%%%%%%%%%%%%%%%%%%%%%%%%%%%%%%%%%%%%%%%%%%%%%%%%%%%%%%%
%EEEEEEEEEEEEEEEEEEEEEEEEEEEEEEEEEEEEEEEEEEEEEEEEEEEEEEEEEEEEEEEEEEEEEEEEEEEEEE
%%%%%%%%%%%%%%%%%%%%%%%%%%%%%%%%%%%%%%%%%%%%%%%%%%%%%%%%%%%%%%%%%%%%%%%%%%%%%%%

\section{(North) East Asia}

\begin{small}
\begin{itemize}
\item[\twothousandfifteen]
\fullcite{kilk15} \\
\item[2017]
\fullcite{ryle17}
\item[\twothousandeighteen]
\fullcite{yamz18}
\item[\twothousandtwentyone] 
\fullcite{lora21}
\item[\twothousandtwentytwo] 
\fullcite{wuwh22}
\end{itemize}
\end{small}

\section{(South) East Asia}

\begin{small}
\begin{itemize}
\item[\twothousand]
\fullcite{lecd00} 
\item[\twothousandfour]
\fullcite{rekv04} 
\item[\twothousandfifteen]
\fullcite{kilk15} \\
\fullcite{yotr15} \\
\fullcite{hasp15} \\
\fullcite{meds15} 
\item[\twothousandsixteen]
\fullcite{necg16}
\item[2017]
\fullcite{ryle17}
\item[\twothousandtwentytwo] 
\fullcite{brcb22}
\end{itemize}
\end{small}

%%%%%%%%%%%%%%%%%%%%%%%%%%%%%%%%%%%%%%%%%%%%%%%%%%%%%%%%%%%%%%%%%
\section{Ethiopian and Afar rift, Malawi Rift, East African rift} 

\begin{small}
\begin{itemize}
\item[\twothousandfive] 
\fullcite{likc05} 
\item[\twothousandseven] 
\fullcite{mitk07} 
\item[\twothousandeight] 
\fullcite{cort08} 
\item[\twothousandnine] 
\fullcite{kekj09} 
\item[\twothousandten] 
\fullcite{beve10} 
\item[\twothousandthirteen] 
\fullcite{fabj13} 
\item[\twothousandfourteen] 
\fullcite{phcs14}\\ 
\fullcite{sacs14} 
\item[\twothousandfifteen] 
\fullcite{favk15} 
\item[\twothousandseventeen] 
\fullcite{brcr17} 
\item[\twothousandnineteen] 
\fullcite{cocf19} \\
\fullcite{lapk19} \\
\fullcite{njas19} 
\item[\twothousandtwenty] 
\fullcite{glbs20} \\
\fullcite{stkf20} \\
\fullcite{peke20} \\
\fullcite{mubi20} \\
\fullcite{chkd20} 
\item[\twothousandtwentyone] 
\fullcite{njsn21} \\
\fullcite{ribr21} \\
\fullcite{rasn21} 
\item[\twothousandtwentytwo] 
\fullcite{cond22} \\
\fullcite{mabc22} \\
\fullcite{kaam22} 
\item[\twothousandtwentythree]
\fullcite{njsa23} \\ 
\fullcite{chkr23} 
\item[\twothousandtwentyfour]
\fullcite{xulw24} 
\fullcite{mubp24}
\item[2025]
\fullcite{puld25} 
\end{itemize}
\end{small}

%%%%%%%%%%%%%%%%%%%%%%%%%%%%%%%%%%%%%%%%%%%%%%%%%%%%%%%%%%%%%%%%%%%%%%%%%%%%%%%
%FFFFFFFFFFFFFFFFFFFFFFFFFFFFFFFFFFFFFFFFFFFFFFFFFFFFFFFFFFFFFFFFFFFFFFFFFFFFFF
%%%%%%%%%%%%%%%%%%%%%%%%%%%%%%%%%%%%%%%%%%%%%%%%%%%%%%%%%%%%%%%%%%%%%%%%%%%%%%%

\section{Farallon plate} 

\begin{small}
\begin{itemize}
\item[\twothousandeight]
\fullcite{lisg08} 
\item[\twothousandeleven]
\fullcite{list11} 
\item[\twothousandtwelve]
\fullcite{list12} 
\item[\twothousandsixteen]
\fullcite{licu16} 
\item[\twothousandtwentyfour]
\fullcite{cofh24} 
\end{itemize}
\end{small}

%%%%%%%%%%%%%%%%%%%%%%%%%%%%%%%%%%%%%%%%%%%%%%%%%%%%%%%%%%%%%%%%%%%%%%%%%%%%%%%
%GGGGGGGGGGGGGGGGGGGGGGGGGGGGGGGGGGGGGGGGGGGGGGGGGGGGGGGGGGGGGGGGGGGGGGGGGGGGGG
%%%%%%%%%%%%%%%%%%%%%%%%%%%%%%%%%%%%%%%%%%%%%%%%%%%%%%%%%%%%%%%%%%%%%%%%%%%%%%%

\section{Gibraltar zone, Azores region}

\begin{small}
\begin{itemize}
\item[\twothousandone] 
\fullcite{jibf01} \\
\fullcite{jift01}
\item[\twothousandtwo] 
\fullcite{gumr02} \\
\fullcite{nebs02} 
\item[\twothousandfive] 
\fullcite{zeaf05} 
\item[\twothousandeight] 
\fullcite{vanv08} 
\item[\twothousandten] 
\fullcite{fufa10} 
\item[\twothousandtwelve] 
\fullcite{gudw12} 
\item[\twothousandthirteen] 
\fullcite{miab13} \\
\fullcite{almb13} \\
\fullcite{durt13} 
\item[\twothousandfifteen] 
\fullcite{medd15} \\
\fullcite{furc15} 
\item[\twothousandsixteen] 
\fullcite{necf16} 
\item[\twothousandnineteen] 
\fullcite{casv19} \\
\fullcite{jitf19} 
\item[\twothousandtwentyone] 
\fullcite{func21} 
\item[\twothousandtwentythree]
\fullcite{genm23} 
\item[\twothousandtwentyfour] 
\fullcite{durr24} \\
\fullcite{malb24} 
\item[\twothousandtwentyfive] 
\fullcite{mabg25}
\end{itemize}
\end{small}

\section{Greenland}

\begin{small}
\begin{itemize}
\item[\nineteenninetyseven]
\fullcite{grev97} 
\item[\twothousandeleven]
\fullcite{scwo11} 
\item[\twothousandtwelve]
\fullcite{sakm12} 
\item[\twothousandthirteen]
\fullcite{raab13} 
\item[\twothousandfourteen]
\fullcite{moad14} 
\item[\twothousandfifteen]
\fullcite{stsj15} \\
\fullcite{heps15} 
\item[\twothousandseventeen]
\fullcite{gors17} 
\item[\twothousandnineteen]
\fullcite{stbl19} \\
\fullcite{kudd19} \\
\fullcite{kuwd19} 
\item[\twothousandtwentyfour] 
\fullcite{weco24} 
\end{itemize}
\end{small}

%%%%%%%%%%%%%%%%%%%%%%%%%%%%%%%%%%%%%%%%%%%%%%%%%%%%%%%%%%%%%%%%%%%%%%%%%%%%%%%
%HHHHHHHHHHHHHHHHHHHHHHHHHHHHHHHHHHHHHHHHHHHHHHHHHHHHHHHHHHHHHHHHHHHHHHHHHHHHHH
%%%%%%%%%%%%%%%%%%%%%%%%%%%%%%%%%%%%%%%%%%%%%%%%%%%%%%%%%%%%%%%%%%%%%%%%%%%%%%%

%==============================
\section{Hellenic zone/ Greece}

\begin{small}
\begin{itemize}
\item[\nineteeneightyeight] 
\fullcite{spwv88} 
\item[\twothousandthirteen]
\fullcite{guhf13} 
\item[\twothousandfourteen]
\fullcite{olpr14} 
\end{itemize}
\end{small}

%===============
\section{Hawaii}

\begin{small}
\begin{itemize}
\item[\nineteeneightyeight] 
\fullcite{ribe88} 
\item[\nineteenninetysix] 
\fullcite{rili96} 
\item[\nineteenninetyeight] 
\fullcite{most98} 
\item[\nineteenninetynine] 
\fullcite{rich99} 
\item[\twothousand] 
\fullcite{cscr00} 
\item[\twothousandthree] 
\fullcite{vazh03} 
\item[\twothousandfour] 
\fullcite{ribe04} 
\item[\twothousandeight] 
\fullcite{gomm08} 
\item[\twothousandnine] 
\fullcite{tabs09} 
\item[\twothousandeleven] 
\fullcite{asrs11} 
\item[\twothousandtwelve] 
\fullcite{cabp12} 
\item[\twothousandthirteen] 
\fullcite{zhwa13} \\ 
\fullcite{plab13} 
\item[\twothousandnineteen] 
\fullcite{botb19} 
\item[\twothousandtwenty] 
\fullcite{wesl20} \\ 
\fullcite{bezw20} 
\item[\twothousandtwentyone] 
\fullcite{bezh21a} 
\item[2025] 
\fullcite{doib25} 
\end{itemize}
\end{small}

%===========================================================
\section{Himalayan region, Tibetan plateau, India collision} 

\begin{small}
\begin{itemize}
\item[\nineteenseventyfive] 
\fullcite{mota75} 
\item[\nineteenseventyseven]  
\fullcite{mota77} 
\item[\nineteenseventyeight] 
\fullcite{bird78} 
\item[\nineteeneightytwo] 
\fullcite{tapl82} \\
\fullcite{vidm82} \\
\fullcite{engl82} 
\item[\nineteeneightyfour] 
\fullcite{vidm84} 
\item[\nineteeneightysix] 
\fullcite{vimd86} \\
\fullcite{moln86} \\
\fullcite{enho86} 
\item[\nineteeneightyseven] 
\fullcite{zhyu87b} 
\item[\nineteeneightyeight] 
\fullcite{peta88} \\
\fullcite{daco88} \\
\fullcite{coda88} 
\item[\nineteeneightynine] 
\fullcite{moln89} 
\item[\nineteenninety] 
\fullcite{jodc90} 
\item[\nineteenninetythree] 
\fullcite{moem93} \\
\fullcite{hoen93} \\
\fullcite{avta93} 
\item[\nineteenninetyfour] 
\fullcite{wibe94} \\ 
\fullcite{kobi94} 
\item[\nineteenninetyfive] 
\fullcite{chmm95} \\
\fullcite{leka95} 
\item[\nineteenninetyseven] 
\fullcite{robk97} \\
\fullcite{enmo97} \\
\fullcite{neho97} 
\item[\nineteenninetyeight] 
\fullcite{mcna98} \\
\fullcite{hodg98} 
\item[\nineteenninetynine] 
\fullcite{vasb99} \\
\fullcite{bupo99} 
\item[\twothousand] 
\fullcite{chbl00} \\
\fullcite{clro00} \\
\fullcite{holt00} \\
\fullcite{brzf00} 
\item[\twothousandone] 
\fullcite{bejn01} \\
\fullcite{laav01} \\
\fullcite{zemk01} \\
\fullcite{tazr01} 
\item[\twothousandtwo] 
\fullcite{kozc02} \\
\fullcite{jack02} 
\item[\twothousandthree] 
\fullcite{reta03} 
\item[\twothousandfour] 
\fullcite{bejn04} \\
\fullcite{jabm04} \\
\fullcite{zhsw04} \\
\fullcite{rekv04} \\
\fullcite{kagu04} \\
\fullcite{bejh04} 
\item[\twothousandfive] 
\fullcite{clbr05} \\
\fullcite{rost05a} \\
\fullcite{rost05b} 
\item[\twothousandsix] 
\fullcite{clrw06} \\
\fullcite{jabn06} \\
\fullcite{golc06} \\
\fullcite{jipl06} 
\item[\twothousandseven] 
\fullcite{mead07} \\
\fullcite{hecb07} \\
\fullcite{isls07} 
\item[\twothousandeight] 
\fullcite{busc08} \\
\fullcite{strh08} \\
\fullcite{coro08} 
\item[\twothousandten] 
\fullcite{hamo10} \\
\fullcite{joha10} \\
\fullcite{luli10} 
\item[\twothousandeleven] 
\fullcite{befa11} \\
\fullcite{zhxy11} \\
\fullcite{vasd11} \\
\fullcite{jabe11} \\
\fullcite{iahb11} \\
\fullcite{seep11} 
\item[\twothousandtwelve] 
\fullcite{zams12} \\
\fullcite{vald12} 
\item[\twothousandthirteen] 
\fullcite{care13} \\
\fullcite{chgz13} \\
\fullcite{chgz13b} \\
\fullcite{barl13} 
\item[\twothousandfourteen] 
\fullcite{whbb14} \\
\fullcite{mutg14} \\
\fullcite{stjm14} \\
\fullcite{lesh14} \\
\fullcite{recg14} 
\item[\twothousandfifteen] 
\fullcite{puka15} \\
\fullcite{jarh15} \\
\fullcite{yoha15} 
\item[\twothousandsixteen] 
\fullcite{kebb16} \\
\fullcite{staj16} \\
\fullcite{huwl16} \\
\fullcite{fezl16} 
\item[\twothousandseventeen] 
\fullcite{bube17} \\
\fullcite{chcl17} 
\item[\twothousandeighteen] 
\fullcite{pirf18} \\
\fullcite{pukp18} \\
\fullcite{flbb18} \\
\fullcite{yalg18} \\
\fullcite{jofb18} \\
\fullcite{jolp18b} \\
\fullcite{zhzz18} 
\item[\twothousandnineteen] 
\fullcite{sccs19} \\
\fullcite{scvm19} \\
\fullcite{wazg19} \\
\fullcite{scdh19} \\
\fullcite{sigh19} \\
\fullcite{hulf19} \\
\fullcite{bifl19} \\
\fullcite{davg19} 
\item[\twothousandtwenty] 
\fullcite{livn20} \\
\fullcite{chlc20} \\
\fullcite{pust20} \\
\fullcite{yakl20} \\
\fullcite{ghbm20} \\
\fullcite{sigh20} \\
\fullcite{capi20} \\ 
\fullcite{kebb20} 
\item[\twothousandtwentyone] 
\fullcite{famu21} \\
\fullcite{pels21} \\
\fullcite{pirc21} \\
\fullcite{cull21} \\
\fullcite{xicx21} \\ 
\fullcite{orsh21} \\ 
\fullcite{mars21} 
\item[\twothousandtwentytwo] 
\fullcite{wawf22} \\ 
\fullcite{pazs22} \\ 
\fullcite{shsb22} \\ 
\fullcite{kebj22} 
\item[\twothousandtwentythree] 
\fullcite{lass23} \\ 
\fullcite{lilm23} \\ 
\fullcite{chho23} \\ 
\fullcite{pirt23} \\ 
\fullcite{jitd23} \\ 
\fullcite{liyq23} \\ 
\fullcite{xicy23} \\ 
\fullcite{boss23} 
\item[\twothousandtwentyfour] 
\fullcite{xicc24} \\
\fullcite{chzy24} \\
\fullcite{vapc24} \\
\fullcite{xicm24} \\ 
\fullcite{xibg24} \\
\fullcite{stxs24} \\
\fullcite{zhhz24} \\
\fullcite{lill24} \\
\fullcite{ghsm24} \\
\fullcite{xuss24} 
\item[\twothousandtwentyfive] 
\fullcite{zhen25}
\end{itemize}
\end{small}

%%%%%%%%%%%%%%%%%%%%%%%%%%%%%%%%%%%%%%%%%%%%%%%%%%%%%%%%%%%%%%%%%%%%%%%%%%%%%%%
%IIIIIIIIIIIIIIIIIIIIIIIIIIIIIIIIIIIIIIIIIIIIIIIIIIIIIIIIIIIIIIIIIIIIIIIIIIIIII
%%%%%%%%%%%%%%%%%%%%%%%%%%%%%%%%%%%%%%%%%%%%%%%%%%%%%%%%%%%%%%%%%%%%%%%%%%%%%%%

\section{Italy}
See also: Apennines

\begin{small}
\begin{itemize}
\item[1999] \fullcite{necb99} \\ \fullcite{dins99} \\ \fullcite{nebc99}
\item[2005] \fullcite{canv05}
\item[2008] \fullcite{vifj08}
\item[2014] \fullcite{fabm14}
\item[2022] \fullcite{zhjt22}
\item[2024] \fullcite{zhjt24} \\ \fullcite{lasf24}
\item[2025] \fullcite{devf25}
\end{itemize}
\end{small}

%=============================================================
\section{Iberian peninsula, Cantabria \& North-Iberian margin}

\begin{small}
\begin{itemize}
\item[1990]
\fullcite{fetz90} 
\item[1995]
\fullcite{tofc95} 
\item[1994]
\fullcite{zefe94} 
\item[\nineteenninetysix]
\fullcite{grri96}
\item[\twothousandtwo]
\fullcite{clbb02} 
\item[\twothousandfour]
\fullcite{ayma04} \\
\fullcite{femt04} 
\item[\twothousandeleven]
\fullcite{paca11} 
\item[\twothousandfifteen]
\fullcite{peap15} \\
\fullcite{cafj15} \\
\fullcite{caft15}
\item[\twothousandtwentyone]
\fullcite{kufv21} 
\item[\twothousandtwentythree]
\fullcite{tojn23} 
\end{itemize}
\end{small}

%==============================================================================
\section{Iran, Zagros} 

\begin{small}
\begin{itemize}
\item[1978]
\fullcite{bird78b}
\item[\nineteenninetyseven]
\fullcite{rabh97} 
\item[\twothousandthree] 
\fullcite{bocs03} 
\item[\twothousandfive]
\fullcite{mozl05} 
\item[\twothousandsix] 
\fullcite{vech06} 
\item[\twothousandten] 
\fullcite{hamo10} 
\item[\twothousandeleven] 
\fullcite{yakm11} 
\item[\twothousandthirteen] 
\fullcite{nipc13} 
\item[\twothousandfourteen] 
\fullcite{ghbu14} \\ 
\fullcite{frba14} 
\item[\twothousandsixteen] 
\fullcite{coyc16} 
\item[\twothousandseventeen] 
\fullcite{rugb17} \\ 
\fullcite{waeh17} 
\item[\twothousandeighteen] 
\fullcite{ruvb18} \\ 
\fullcite{ruve18} \\ 
\fullcite{barj18} 
\item[\twothousandtwenty] 
\fullcite{herv20} \\ 
\fullcite{mofu20} 
\item[\twothousandtwentyone]
\fullcite{nabr21} 
\item[\twothousandtwentythree] 
\fullcite{gacy23} 
\item[\twothousandtwentyfour]
\fullcite{gorn24}
\end{itemize}
\end{small} 

%==============================================================================
\section{Indian Ocean} 

\begin{small}
\begin{itemize}
\item[\nineteenseventythree]
\fullcite{mcse73}
\item[\twothousandtwelve]
\fullcite{dost12}
\item[\twothousandseventeen]
\fullcite{ghts17}
\item[\twothousandtwentytwo] 
\fullcite{ghpa22}
\item[\twothousandtwentythree] 
\fullcite{pagh23}
\item[\twothousandtwentyfour] 
\fullcite{lulz24}
\end{itemize}
\end{small}

\section{Iceland, Reykjanes Ridge, Iceland plume}

\begin{small}
\begin{itemize}
\item[\nineteeneightynine]
\fullcite{whit89} 
\item[\nineteenninetysix]
\fullcite{jumc96} 
\item[\nineteenninetynine]
\fullcite{bisp99}\\ 
\fullcite{rivw99} 
\item[\twothousand]
\fullcite{acgf00} 
\item[\twothousandfour]
\fullcite{masc04}
\item[\twothousandseventeen]
\fullcite{kocb17} \\
\fullcite{bahf17} 
\item[\twothousandnineteen]
\fullcite{stbl19} 
\item[\twothousandtwenty]
\fullcite{rits20} 
\item[\twothousandtwentytwo]
\fullcite{zhlz22} 
\end{itemize}
\end{small} 

%%%%%%%%%%%%%%%%%%%%%%%%%%%%%%%%%%%%%%%%%%%%%%%%%%%%%%%%%%%%%%%%%%%%%%%%%%%%%%%
%JJJJJJJJJJJJJJJJJJJJJJJJJJJJJJJJJJJJJJJJJJJJJJJJJJJJJJJJJJJJJJJJJJJJJJJJJJJJJJ
%%%%%%%%%%%%%%%%%%%%%%%%%%%%%%%%%%%%%%%%%%%%%%%%%%%%%%%%%%%%%%%%%%%%%%%%%%%%%%%

\section{Japan, Izu-Bonin, Sea of Japan} 

\begin{small}
\begin{itemize}
\item[\nineteeneightyfive]
\fullcite{hond85} 
\item[\nineteenninetytwo]
\fullcite{stbl92} 
\item[\twothousandseven]
\fullcite{homo07} \\ 
\fullcite{lohd07} 
\item[\twothousandnine]
\fullcite{obyf09} 
\item[2011]
\fullcite{moho11}
\item[\twothousandtwelve]
\fullcite{vakn12} 
\item[\twothousandthirteen]
\fullcite{musi13}\\ 
\fullcite{moho13} 
\item[\twothousandfourteen]
\fullcite{leli14} \\
\fullcite{kigk14} \\
\fullcite{mova14} \\
\fullcite{hond14} 
\item[\twothousandfifteen]
\fullcite{kilk15} \\
\fullcite{arib15} 
\item[\twothousandsixteen]
\fullcite{leli16b}
\item[\twothousandseventeen]
\fullcite{mova17} \\
\fullcite{yagz17} 
\item[\twothousandeighteen]
\fullcite{fahb18} 
\item[\twothousandnineteen]
\fullcite{yamg19} 
\item[\twothousandtwenty]
\fullcite{mapg20} \\
\fullcite{yole20}
\item[\twothousandtwentyone]
\fullcite{mota21} \\
\fullcite{lewa21} \\
\fullcite{leki21} \\
\fullcite{zhjq21} 
\item[\twothousandtwentytwo]
\fullcite{mori22} \\
\fullcite{yuhl22} 
\item[\twothousandtwentythree]
\fullcite{izhy23} \\ 
\fullcite{dosk23} \\ 
\fullcite{yole23} \\ 
\fullcite{leki23} \\ 
\fullcite{ligu23b} \\
\fullcite{momm23} \\
\fullcite{hapl23} \\
\fullcite{dhmo23}
\item[\twothousandtwentyfive]
\fullcite{gigh25}
\end{itemize}
\end{small}

%%%%%%%%%%%%%%%%%%%%%%%%%%%%%%%%%%%%%%%%%%%%%%%%%%%%%%%%%%%%%%%%%%%%%%%%%%%%%%%
%KKKKKKKKKKKKKKKKKKKKKKKKKKKKKKKKKKKKKKKKKKKKKKKKKKKKKKKKKKKKKKKKKKKKKKKKKKKKKK
%%%%%%%%%%%%%%%%%%%%%%%%%%%%%%%%%%%%%%%%%%%%%%%%%%%%%%%%%%%%%%%%%%%%%%%%%%%%%%%

\section{Korea, Korean Peninsula} 

\begin{small}
\begin{itemize}
\item[2022]
\fullcite{less22}\\
\fullcite{kilk22}
\item[2023]
\fullcite{lesh23}
\end{itemize}
\end{small}

\section{Kamchatka}

\begin{small}
\begin{itemize}
\item[2007]
\fullcite{zwgg07}
\item[2024]
\fullcite{erpz24} 
\end{itemize}
\end{small}



%%%%%%%%%%%%%%%%%%%%%%%%%%%%%%%%%%%%%%%%%%%%%%%%%%%%%%%%%%%%%%%%%%%%%%%%%%%%%%%
%LLLLLLLLLLLLLLLLLLLLLLLLLLLLLLLLLLLLLLLLLLLLLLLLLLLLLLLLLLLLLLLLLLLLLLLLLLLLLL
%%%%%%%%%%%%%%%%%%%%%%%%%%%%%%%%%%%%%%%%%%%%%%%%%%%%%%%%%%%%%%%%%%%%%%%%%%%%%%%


%%%%%%%%%%%%%%%%%%%%%%%%%%%%%%%%%%%%%%%%%%%%%%%%%%%%%%%%%%%%%%%%%%%%%%%%%%%%%%%
%MMMMMMMMMMMMMMMMMMMMMMMMMMMMMMMMMMMMMMMMMMMMMMMMMMMMMMMMMMMMMMMMMMMMMMMMMMMMMM
%%%%%%%%%%%%%%%%%%%%%%%%%%%%%%%%%%%%%%%%%%%%%%%%%%%%%%%%%%%%%%%%%%%%%%%%%%%%%%%

\section{Mediterranean region}

\begin{small}
\begin{itemize}
\item[\nineteenninetyseven] 
\fullcite{pimo97}  \\
\fullcite{nesg97} 
\item[\nineteenninetynine] 
\fullcite{nesb99}  \\
\fullcite{neft99}  \\
\fullcite{dins99} 
\item[\twothousand] 
\fullcite{wosp00} 
\item[\twothousandthree] 
\fullcite{pimo03} 
\item[\twothousandfour] 
\fullcite{spwo04}  \\
\fullcite{boek04} 
\item[\twothousandnine] 
\fullcite{wogs09} 
\item[\twothousandten] 
\fullcite{bofb10}  \\
\fullcite{fabe10} 
\item[\twothousandfourteen] 
\fullcite{chsv14}  \\
\fullcite{chsg14}  \\
\fullcite{vavs14}  \\
\fullcite{mafv14} 
\item[\twothousandsixteen] 
\fullcite{mesj16} 
\item[\twothousandeighteen] 
\fullcite{spcv18} 
\item[\twothousandnineteen] 
\fullcite{gumt19} 
\item[\twothousandtwenty] 
\fullcite{blgf20}  \\
\fullcite{fabe20}  \\
\fullcite{vaga20}  \\
\fullcite{nemc20} 
\item[\twothousandtwentyone]
\fullcite{kufv21}  \\
\fullcite{erhf21}  \\
\fullcite{lofy21} 
\item[\twothousandtwentytwo] 
\fullcite{cobf22}  \\
\fullcite{pefv22}  \\
\fullcite{bafg22} 
\item[\twothousandtwentyfive] 
\fullcite{sckl25}
\end{itemize}
\end{small}

\section{Madagascar}

\begin{small}
\begin{itemize}
\item[\twothousandtwenty]
\fullcite{rasf20} 
\end{itemize}
\end{small}

\section{Mayotte}

\begin{small}
\begin{itemize}
\item[\twothousandtwentyfour]
\fullcite{derg24} 
\end{itemize}
\end{small}

\section{Mariana Trench}

\begin{small}
\begin{itemize}
\item[1978]
\fullcite{bird78c}
\item[\nineteenninetytwo]
\fullcite{stbl92}
\item[\twothousandfifteen]
\fullcite{yotr15}\\
\fullcite{arib15}\\
\fullcite{zhlb15}
\item[\twothousandeighteen]
\fullcite{fahb18}
\item[\twothousandtwentythree]
\fullcite{quzj23}\\
\fullcite{chzl23}
\end{itemize}
\end{small}

%%%%%%%%%%%%%%%%%%%%%%%%%%%%%%%%%%%%%%%%%%%%%%%%%%%%%%%%%%%%%%%%%%%%%%%%%%%%%%%
%NNNNNNNNNNNNNNNNNNNNNNNNNNNNNNNNNNNNNNNNNNNNNNNNNNNNNNNNNNNNNNNNNNNNNNNNNNNNNN
%%%%%%%%%%%%%%%%%%%%%%%%%%%%%%%%%%%%%%%%%%%%%%%%%%%%%%%%%%%%%%%%%%%%%%%%%%%%%%%

\section{New Zealand} 

\begin{small}
\begin{itemize}
\item[\nineteenninety] 
\fullcite{koon90} 
\item[\nineteenninetyfive] 
\fullcite{brbe95} 
\item[\nineteenninetysix] 
\fullcite{bekh96} 
\item[\nineteenninetyeight] 
\fullcite{wabb98} 
\item[\nineteenninetynine] 
\fullcite{babr99} 
\item[\twothousandtwo] 
\fullcite{gedh02}  \\
\fullcite{libi02b} \\
\fullcite{pybf02} 
\item[\twothousandthree] 
\fullcite{gehd03}  \\
\fullcite{konc03}  \\
\fullcite{upke03} 
\item[\twothousandsix] 
\fullcite{libi06} 
\item[\twothousandseven] 
\fullcite{upko07} 
\item[\twothousandnine] 
\fullcite{upkc09} 
\item[\twothousandten] 
\fullcite{pyeg10}  \\
\fullcite{spgs10a} 
\item[\twothousandtwelve] 
\fullcite{grel12} 
\item[\twothousandthirteen] 
\fullcite{sths13} 
\item[\twothousandsixteen] 
\fullcite{elwr16} 
\item[\twothousandtwentyfour] 
\fullcite{ligu24}
\item[\twothousandtwentyfive] 
\fullcite{donf25}
\end{itemize}
\end{small}

\section{Norway} 

\begin{small}
\begin{itemize}
\item[\twothousandthirteen] 
\fullcite{soma13} 
\item[\twothousandfourteen] 
\fullcite{soda14} 
\item[\twothousandfifteen] 
\fullcite{bubj15} 
\item[\twothousandtwentytwo] 
\fullcite{pefb22} 
\end{itemize}
\end{small}

\section{North America}

\begin{small}
\begin{itemize}
\item[\nineteenseventythree] 
\fullcite{sabu73} 
\item[\nineteenninety] 
\fullcite{huha90} 
\item[\nineteenninetyseven] 
\fullcite{bugm97} 
\item[\twothousandone] 
\fullcite{chzh01} 
\item[\twothousandtwo] 
\fullcite{libi02} 
\item[\twothousandsix] 
\fullcite{besb06} 
\item[\twothousandeight] 
\fullcite{splg08} 
\item[\twothousandnine] 
\fullcite{splg09} 
\item[\twothousandtwelve] 
\fullcite{beck12} 
\item[\twothousandthirteen]
\fullcite{ghbh13}  \\
\fullcite{simi13} 
\item[\twothousandfifteen]
\fullcite{riag15}  \\
\fullcite{belf15} 
\item[\twothousandsixteen]
\fullcite{afry16}
\item[\twothousandnineteen]
\fullcite{wabe19} 
\item[\twothousandtwentyone]
\fullcite{sacp21}  \\
\fullcite{arpb21} 
\item[\twothousandtwentytwo]
\fullcite{liki22} 
\item[\twothousandtwentyfour]
\fullcite{wisa24} 
\end{itemize}
\end{small}

\section{the Netherlands}

\begin{small}
\begin{itemize}
\item[\twothousandtwo]  
\fullcite{crdv02} 
\item[\twothousandtwenty]  
\fullcite{besb20} 
\end{itemize}
\end{small}

%%%%%%%%%%%%%%%%%%%%%%%%%%%%%%%%%%%%%%%%%%%%%%%%%%%%%%%%%%%%%%%%%%%%%%%%%%%%%%%
%OOOOOOOOOOOOOOOOOOOOOOOOOOOOOOOOOOOOOOOOOOOOOOOOOOOOOOOOOOOOOOOOOOOOOOOOOOOOOO
%%%%%%%%%%%%%%%%%%%%%%%%%%%%%%%%%%%%%%%%%%%%%%%%%%%%%%%%%%%%%%%%%%%%%%%%%%%%%%%

%%%%%%%%%%%%%%%%%%%%%%%%%%%%%%%%%%%%%%%%%%%%%%%%%%%%%%%%%%%%%%%%%%%%%%%%%%%%%%%
%PPPPPPPPPPPPPPPPPPPPPPPPPPPPPPPPPPPPPPPPPPPPPPPPPPPPPPPPPPPPPPPPPPPPPPPPPPPPPP
%%%%%%%%%%%%%%%%%%%%%%%%%%%%%%%%%%%%%%%%%%%%%%%%%%%%%%%%%%%%%%%%%%%%%%%%%%%%%%%

\section{Pamir-Hindu Kush region}

\begin{small}
\begin{itemize}
\item[\twothousandseven]
\fullcite{nerv07} 
\item[\twothousandsixteen]
\fullcite{schr16} 
\end{itemize}
\end{small}

\section{Pannonian Basin}

\begin{small}
\begin{itemize}
\item[\twothousandone]
\fullcite{hupc01b} 
\item[\twothousandtwo]
\fullcite{hupc02} 
\item[\twothousandeleven]
\fullcite{jabm11} 
\item[\twothousandtwentyone]
\fullcite{kock21} 
\end{itemize}
\end{small}

\section{Pacific}

\begin{small}
\begin{itemize}
\item[\nineteensixtyseven] 
\fullcite{mcpa67} 
\item[\nineteeneighty] 
\fullcite{wabr80} 
\item[\nineteeneightytwo] 
\fullcite{riwa82} 
\item[\nineteenninety] 
\fullcite{jodc90} 
\item[\twothousandfive] 
\fullcite{vazs05} \\ 
\fullcite{mczh05a} 
\item[\twothousandten] 
\fullcite{zhst10} 
\item[\twothousandeleven] 
\fullcite{capd11} 
\item[\twothousandthirteen] 
\fullcite{necb13} \\ 
\fullcite{kecl13} \\
\fullcite{bacs13} 
\item[\twothousandfifteen] 
\fullcite{sefw15} \\ 
\fullcite{necb15} 
\item[\twothousandseventeen] 
\fullcite{stid17} \\ 
\fullcite{togr17} \\
\fullcite{egim17} 
\item[\twothousandeighteen] 
\fullcite{yamz18} 
\item[\twothousandnineteen] 
\fullcite{weco19} \\ 
\fullcite{sccs19} 
\item[\twothousandtwenty] 
\fullcite{mapg20} 
\item[\twothousandtwentyone] 
\fullcite{moma21} 
\item[\twothousandtwentytwo]  
\fullcite{pafl22} \\ 
\fullcite{licw22} 
\item[\twothousandtwentythree]
\fullcite{lihh23}  
\item[\twothousandtwentyfive]
\fullcite{yawz25}  
\end{itemize}
\end{small}

\section{Philippine plate, Manila Trench}

\begin{small}
\begin{itemize}
\item[\twothousandfifteen]
\fullcite{cibi15}
\item[\twothousandsixteen]
\fullcite{gukt16}
\item[\twothousandeighteen]
\fullcite{horb18}
\item[\twothousandtwenty]
\fullcite{bicc20}
\item[\twothousandtwentythree]
\fullcite{momm23}
\item[\twothousandtwentyfour]
\fullcite{dohx24}
\end{itemize}
\end{small}

\section{Pyrenees} 

\begin{small}
\begin{itemize}
\item[\nineteenninetyone]   
\fullcite{chvd91} 
\item[\nineteenninetytwo]   
\fullcite{chou92} 
\item[\nineteenninetythree] 
\fullcite{qubh93} 
\item[\nineteenninetyeight]
\fullcite{giju98} 
\item[\twothousand]         
\fullcite{bemh00} 
\item[\twothousandfour] 
\fullcite{mcmg04}  \\
\fullcite{siss04} 
\item[\twothousandten] 
\fullcite{jaml10} 
\item[\twothousandtwelve] 
\fullcite{vime12} 
\item[\twothousandthirteen] 
\fullcite{fihv13b} 
\item[\twothousandfourteen] 
\fullcite{jahm14} 
\item[\twothousandnineteen] 
\fullcite{dual19}  \\
\fullcite{jolm19} 
\end{itemize}
\end{small}

%%%%%%%%%%%%%%%%%%%%%%%%%%%%%%%%%%%%%%%%%%%%%%%%%%%%%%%%%%%%%%%%%%%%%%%%%%%%%%%
%QQQQQQQQQQQQQQQQQQQQQQQQQQQQQQQQQQQQQQQQQQQQQQQQQQQQQQQQQQQQQQQQQQQQQQQQQQQQQQ
%%%%%%%%%%%%%%%%%%%%%%%%%%%%%%%%%%%%%%%%%%%%%%%%%%%%%%%%%%%%%%%%%%%%%%%%%%%%%%%

%%%%%%%%%%%%%%%%%%%%%%%%%%%%%%%%%%%%%%%%%%%%%%%%%%%%%%%%%%%%%%%%%%%%%%%%%%%%%%%
%RRRRRRRRRRRRRRRRRRRRRRRRRRRRRRRRRRRRRRRRRRRRRRRRRRRRRRRRRRRRRRRRRRRRRRRRRRRRRR
%%%%%%%%%%%%%%%%%%%%%%%%%%%%%%%%%%%%%%%%%%%%%%%%%%%%%%%%%%%%%%%%%%%%%%%%%%%%%%%

\section{Reunion island/volcano}

\begin{small}
\begin{itemize}
\item[\twothousandseventeen]
\fullcite{brsg17} 
\item[\twothousandtwentytwo]
\fullcite{gefp22} 
\end{itemize}
\end{small}

%%%%%%%%%%%%%%%%%%%%%%%%%%%%%%%%%%%%%%%%%%%%%%%%%%%%%%%%%%%%%%%%%%%%%%%%%%%%%%%
%SSSSSSSSSSSSSSSSSSSSSSSSSSSSSSSSSSSSSSSSSSSSSSSSSSSSSSSSSSSSSSSSSSSSSSSSSSSSSS
%%%%%%%%%%%%%%%%%%%%%%%%%%%%%%%%%%%%%%%%%%%%%%%%%%%%%%%%%%%%%%%%%%%%%%%%%%%%%%%

\section{Scotia plate}

\begin{small}
\begin{itemize}
\item[\twothousandthirteen]
\fullcite{necb13} 
\item[\twothousandtwenty]
\fullcite{vaga20} 
\item[\twothousandtwentyone]
\fullcite{vasv21} 
\item[\twothousandtwentythree]
\fullcite{scsb23}
\end{itemize}
\end{small}

\section{Scandinavia}

\begin{small}
\begin{itemize}
\item[\nineteeneighty] 
\fullcite{ramb80} 
\item[\twothousandfive]
\fullcite{stka05} 
\item[\twothousandeight]
\fullcite{stdm08} 
\item[\twothousandthirteen]
\fullcite{vabs13} 
\item[\twothousandfourteen] 
\fullcite{bovc14} 
\item[\twothousandfifteen] 
\fullcite{rovn15} 
\item[\twothousand]
\fullcite{rovb20} 
\end{itemize}
\end{small}

\section{Sunda}

\begin{small}
\begin{itemize}
\item[\twothousandeighteen]
\fullcite{racr18} 
\item[\twothousandtwentytwo]
\fullcite{zugc22} 
\end{itemize}
\end{small}

%%%%%%%%%%%%%%%%%%%%%%%%%%%%%%%%%%%%%%%%%%%%%%
\section{South America, Andes, Andean orogeny}

\begin{small}
\begin{itemize}
\item[\nineteenninetyfour] 
\fullcite{wdbo94b} 
\item[\nineteenninetysix] 
\fullcite{zori96} 
\item[\twothousand] 
\fullcite{gusb00} 
\item[\twothousandtwo] 
\fullcite{vavv02b} 
\item[\twothousandfour] 
\fullcite{huri04} 
\item[\twothousandfive] 
\fullcite{baso05} \\ 
\fullcite{soba05} 
\item[\twothousandsix] 
\fullcite{basv06} \\
\fullcite{meph06} \\
\fullcite{iabd06} \\
\fullcite{oncf06} \\
\fullcite{sobk06} 
\item[\twothousandseven] 
\fullcite{iabb07} 
\item[\twothousandeight] 
\fullcite{esfm08} \\
\fullcite{heib08} \\
\fullcite{iabu08} \\
\fullcite{gogm08} 
\item[\twothousandnine] 
\fullcite{kecw09} \\
\fullcite{gecm09} \\
\fullcite{wahk09} \\
\fullcite{luli09} \\
\fullcite{luli09b} 
\item[\twothousandtwelve] 
\fullcite{hucf12} \\
\fullcite{shlm12} \\
\fullcite{iadc12} 
\item[\twothousandthirteen]
\fullcite{assa13} \\
\fullcite{wahk13} \\
\fullcite{waja13} 
\item[\twothousandfifteen] 
\fullcite{cudd15} \\
\fullcite{ealw15} \\
\fullcite{zeha15} 
\item[\twothousandsixteen] 
\fullcite{robn16} \\
\fullcite{marl16} \\
\fullcite{chdf16} \\
\fullcite{hulh16} 
\item[\twothousandseventeen] 
\fullcite{sche17} \\
\fullcite{wajr17} \\
\fullcite{faoh17} 
\item[\twothousandnineteen] 
\fullcite{sisa19} \\
\fullcite{yamg19} 
\item[\twothousandtwenty] 
\fullcite{scwh20} \\
\fullcite{with20} 
\item[\twothousandtwentyone] 
\fullcite{balm21} \\
\fullcite{stsc21} \\
\fullcite{hulg21} \\
\fullcite{wacd21} 
\item[\twothousandtwentyone] 
\fullcite{sgmd22} \\ 
\fullcite{hebe22} \\
\fullcite{rosb22} \\
\fullcite{saup22} \\ 
\fullcite{posl22} \\
\fullcite{lisb22} 
\item[\twothousandtwentythree]
\fullcite{pors23} \\ 
\fullcite{gusw23b} \\ 
\fullcite{sayb23} 
\item[\twothousandtwentyfour]
\fullcite{liwc24} \\
\fullcite{qucp24} 
%\item[\twothousandtwentyfive]
%\fullcite{wahl25} 
\end{itemize}
\end{small}

\section{Siberia}

\begin{small}
\begin{itemize}
\item[2000] \fullcite{elha00}
\item[2018] \fullcite{yaca18}
\end{itemize}
\end{small}

%%%%%%%%%%%%%%%%%%%%%%%%%%%%%%%%%%%%%%%%%%%%%%%%%%%%%%%%%%%%%%%%%%%%%%%%%%%%%%%
%TTTTTTTTTTTTTTTTTTTTTTTTTTTTTTTTTTTTTTTTTTTTTTTTTTTTTTTTTTTTTTTTTTTTTTTTTTTTTT
%%%%%%%%%%%%%%%%%%%%%%%%%%%%%%%%%%%%%%%%%%%%%%%%%%%%%%%%%%%%%%%%%%%%%%%%%%%%%%%

\section{Tonga-Kermadec subduction zone, Fiji }

\begin{small}
\begin{itemize}
\item[1978]
\fullcite{bird78c}
\item[\twothousandthree] 
\fullcite{bigs03} \\
\fullcite{bigu03} 
\item[\twothousandsix] 
\fullcite{zhpy06} 
\item[\twothousandsixteen] 
\fullcite{chff16} 
\item[\twothousandseventeen] 
\fullcite{wewv17} 
\item[2018]
\fullcite{kile18}
\item[\twothousandtwentyone] 
\fullcite{ligl21} 
\item[\twothousandtwentythree] 
\fullcite{pocb23}
\item[2024]
\fullcite{pest24}
\end{itemize}
\end{small}

\section{Taiwan}

\begin{small}
\begin{itemize}
\item[\twothousandone]
\fullcite{chys01} 
\item[\twothousandsix]
\fullcite{fuwf06} 
\item[\twothousandeight]
\fullcite{kasb08} 
\item[\twothousandnine]
\fullcite{yamb09} \\
\fullcite{kalb09} 
\item[\twothousandsixteen]
\fullcite{gukt16} \\ 
\fullcite{liku16} 
\item[\twothousandnineteen]
\fullcite{wakz19} 
\item[\twothousandtwenty]
\fullcite{tadl20}
\item[\twothousandtwentyone]
\fullcite{waky21}
\end{itemize}
\end{small}

\section{Tarim Basin, Tian Shan } 

\begin{small}
\begin{itemize}
\item[\nineteenninetyseven] 
\fullcite{neho97}
\item[\twothousandtwentytwo] 
\fullcite{wazm22}
\item[\twothousandtwentyfive] 
\fullcite{walc25}
\end{itemize}
\end{small}

%%%%%%%%%%%%%%%%%%%%%%%%%%%%%%%%%%%%%%%%%%%%%%%%%%%%%%%%%%%%%%%%%%%%%%%%%%%%%%%
%UUUUUUUUUUUUUUUUUUUUUUUUUUUUUUUUUUUUUUUUUUUUUUUUUUUUUUUUUUUUUUUUUUUUUUUUUUUUUU
%%%%%%%%%%%%%%%%%%%%%%%%%%%%%%%%%%%%%%%%%%%%%%%%%%%%%%%%%%%%%%%%%%%%%%%%%%%%%%%

\section{Western United States, San Andreas system}

\begin{small}
\begin{itemize}
\item[\nineteeneighty]
\fullcite{bipi80}
\item[\nineteeneightyfour]
\fullcite{biba84}
\item[\nineteeneightyseven]
\fullcite{vawo87}
\item[\nineteenninetytwo]
\fullcite{stbl92} 
\item[\nineteenninetyfour]
\fullcite{biko94} 
\item[\nineteenninetyeight]
\fullcite{rabg98} \\ 
\fullcite{sali98} 
\item[\twothousand]
\fullcite{honk00}\\ 
\fullcite{lors00} 
\item[\twothousandthree]
\fullcite{magf03} 
\item[\twothousandfour]
\fullcite{mojo04} 
\item[\twothousandsix]
\fullcite{besb06} \\
\fullcite{legs06} \\
\fullcite{scdm06}
\item[\twothousandeight]
\fullcite{plkb08} 
\item[\twothousandnine]
\fullcite{bibu09} 
\item[\twothousandtwelve]
\fullcite{luli12} 
\item[\twothousandthirteen]
\fullcite{plbe13} 
\item[\twothousandfourteen]
\fullcite{vanb14} 
\item[\twothousandseventeen]
\fullcite{petc17} 
\item[\twothousandeighteen]
\fullcite{pehu18}
\item[\twothousandtwenty]
\fullcite{iswa20}  
\item[\twothousandtwentyone]
\fullcite{chap21} 
\item[\twothousandtwentytwo]
\fullcite{bahf22} \\
\fullcite{baha22} 
\end{itemize}
\end{small}

\section{Southeastern United States}

\begin{small}
\begin{itemize}
\item[\twothousandnineteen] 
\fullcite{heps19} 
\end{itemize}
\end{small}

%%%%%%%%%%%%%%%%%%%%%%%%%%%%%%%%%%%%%%%%%%%%%%%%%%%%%%%%%%%%%%%%%%%%%%%%%%%%%%%
%VVVVVVVVVVVVVVVVVVVVVVVVVVVVVVVVVVVVVVVVVVVVVVVVVVVVVVVVVVVVVVVVVVVVVVVVVVVVVV
%%%%%%%%%%%%%%%%%%%%%%%%%%%%%%%%%%%%%%%%%%%%%%%%%%%%%%%%%%%%%%%%%%%%%%%%%%%%%%%

\section{Variscan}

\begin{small}
\begin{itemize}
\item[\nineteenninetynine] 
\fullcite{vajh99} 
\item[\twothousandfour] 
\fullcite{fijj04} 
\item[\twothousandseven] 
\fullcite{masp07} 
\item[\twothousandthirteen] 
\fullcite{rems13} 
\item[\twothousandseventeen] 
\fullcite{regorda} 
\item[\twothousandeighteen] 
\fullcite{gesr18} 
\item[\twothousandtwenty] 
\fullcite{relr20} 
\item[\twothousandtwentyone] 
\fullcite{mass21} 
\end{itemize}
\end{small}

%%%%%%%%%%%%%%%%%%%%%%%%%%%%%%%%%%%%%%%%%%%%%%%%%%%%%%%%%%%%%%%%%%%%%%%%%%%%%%%
%WWWWWWWWWWWWWWWWWWWWWWWWWWWWWWWWWWWWWWWWWWWWWWWWWWWWWWWWWWWWWWWWWWWWWWWWWWWWWW
%%%%%%%%%%%%%%%%%%%%%%%%%%%%%%%%%%%%%%%%%%%%%%%%%%%%%%%%%%%%%%%%%%%%%%%%%%%%%%%


%%%%%%%%%%%%%%%%%%%%%%%%%%%%%%%%%%%%%%%%%%%%%%%%%%%%%%%%%%%%%%%%%%%%%%%%%%%%%%%
%XXXXXXXXXXXXXXXXXXXXXXXXXXXXXXXXXXXXXXXXXXXXXXXXXXXXXXXXXXXXXXXXXXXXXXXXXXXXXX
%%%%%%%%%%%%%%%%%%%%%%%%%%%%%%%%%%%%%%%%%%%%%%%%%%%%%%%%%%%%%%%%%%%%%%%%%%%%%%%


%%%%%%%%%%%%%%%%%%%%%%%%%%%%%%%%%%%%%%%%%%%%%%%%%%%%%%%%%%%%%%%%%%%%%%%%%%%%%%%
%YYYYYYYYYYYYYYYYYYYYYYYYYYYYYYYYYYYYYYYYYYYYYYYYYYYYYYYYYYYYYYYYYYYYYYYYYYYYYY
%%%%%%%%%%%%%%%%%%%%%%%%%%%%%%%%%%%%%%%%%%%%%%%%%%%%%%%%%%%%%%%%%%%%%%%%%%%%%%%

\section{Yellowstone}

\begin{small}
\begin{itemize}
\item[\twothousandthirteen]
\fullcite{chus13} 
\item[\twothousandsixteen]
\fullcite{leli16} 
\item[\twothousandeighteen]
\fullcite{rekp18} 
\end{itemize}
\end{small}

%%%%%%%%%%%%%%%%%%%%%%%%%%%%%%%%%%%%%%%%%%%%%%%%%%%%%%%%%%%%%%%%%%%%%%%%%%%%%%%
%ZZZZZZZZZZZZZZZZZZZZZZZZZZZZZZZZZZZZZZZZZZZZZZZZZZZZZZZZZZZZZZZZZZZZZZZZZZZZZZ
%%%%%%%%%%%%%%%%%%%%%%%%%%%%%%%%%%%%%%%%%%%%%%%%%%%%%%%%%%%%%%%%%%%%%%%%%%%%%%%








%\chapter{Numerical methods} 

%------------------------------------------------------------------------------
%------------------------------------------------------------------------------
\section{Pseudo-transient method}
%------------------------------------------------------------------------------
%------------------------------------------------------------------------------

\fullcite{durp19}
\fullcite{raud22}

%------------------------------------------------------------------------------
%------------------------------------------------------------------------------
\section{Machine learning, Artificial intelligence, Deep learning, ...}
%------------------------------------------------------------------------------
%------------------------------------------------------------------------------

\begin{scriptsize}
\begin{itemize}
\item[\twothousandtwentyone] 
\fullcite{agtk21} 
\item[\twothousandtwentythree] 
\fullcite{liob23}
\end{itemize}
\end{scriptsize}

%------------------------------------------------------------------------------
%------------------------------------------------------------------------------
\section{Axisymmetric flow}
%------------------------------------------------------------------------------
%------------------------------------------------------------------------------

\begin{scriptsize}
\begin{itemize}
\item[\nineteenninetytwo] 
\fullcite{kiha92}
\item[\nineteenninetyseven] 
\fullcite{king97}
\item[\twothousandthree] 
\fullcite{kief03}
\item[\twothousandfour] 
\fullcite{reki04}
\item[\twothousandseven] 
\fullcite{liki07}
\item[\twothousandten] 
\fullcite{lezh10} 
\item[\twothousandtwelve] 
\fullcite{legu12} 
\end{itemize}
\end{scriptsize}

%------------------------------------------------------------------------------
%------------------------------------------------------------------------------
\section{Discontinuous Galerkin (DG)}
%------------------------------------------------------------------------------
%------------------------------------------------------------------------------

\begin{scriptsize}
\begin{itemize}
\item[\nineteenseventythree] 
\fullcite{rehi73}
\item[\nineteenninetyseven] 
\fullcite{bare97}
\item[\nineteenninetyeight] 
\fullcite{cosh98}
\item[\nineteenninetynine] 
\fullcite{riwg99}
\item[\twothousand] 
\fullcite{coks00}\\
\fullcite{brmm00}\\
\fullcite{cacp00}
\item[\twothousandone] 
\fullcite{hala01}
\item[\twothousandtwo] 
\fullcite{cacp02}\\
\fullcite{coks02}\\
\fullcite{arbc02}\\
\fullcite{gurw02}
\item[\twothousandthree] 
\fullcite{cock03}\\
\fullcite{hala03}
\item[\twothousandfour] 
\fullcite{coks04}
\item[\twothousandfive] 
\fullcite{cacs05}\\
\fullcite{coks05}\\
\fullcite{cogo05a}\\
\fullcite{cogo05b}\\
\fullcite{cogo05c}
\item[\twothousandseven] 
\fullcite{coks07}\\
\fullcite{feku07}
\item[\twothousandeight] 
\fullcite{kans08}\\
\fullcite{mofh08}\\
\fullcite{dole08}\\
\fullcite{pepe08}
\item[\twothousandnine] 
\fullcite{coks09}\\
\fullcite{cogo09}\\
\fullcite{cogl09}\\
\fullcite{ngpc09}\\
\fullcite{shu09}\\
\fullcite{codg08}\\
\fullcite{cogw09}
\item[\twothousandten] 
\fullcite{ngpc10}\\
\fullcite{conp10}\\
\fullcite{mofp10}\\
\fullcite{kari10}\\
\fullcite{cogs10}
\item[\twothousandeleven] 
\fullcite{geor11}\\
\fullcite{ngpc11}
\item[\twothousandtwelve] 
\fullcite{kauf12}\\
\fullcite{ngpe12}\\
\fullcite[chapt. 31]{lomw12}
\item[\twothousandthirteen] 
\fullcite{vyrc13}\\
\fullcite{rhcv13}\\
\fullcite{klwh13}
\item[\twothousandfourteen] 
\fullcite{cosh14}
\item[\twothousandfifteen] 
\fullcite{lelk15}\\
\fullcite{kalc15}
\item[\twothousandsixteen] 
\fullcite{cock16}\\
\fullcite{makc16}
\item[\twothousandseventeen] 
\fullcite{fewk17}\\
\fullcite{iglo17}\\
\fullcite{hepb17}\\
\fullcite{chll17}\\
\fullcite{sclu17a}\\
\fullcite{sclu17b}\\
\fullcite{sclu17c}\\
\fullcite{zhan17}
\item[\twothousandeighteen]
\fullcite{puth18}\\
\fullcite{wogu18}\\
\fullcite{fakr18}\\
\fullcite{muwy18}
\end{itemize}
\end{scriptsize}

%------------------------------------------------------------------------------
%------------------------------------------------------------------------------
\section{(use of) Inverse methods, inversion, adjoint methods, assimilation}
%------------------------------------------------------------------------------
%------------------------------------------------------------------------------


\begin{scriptsize}
\begin{itemize}
\item[\nineteenninetysix] 
\fullcite{fomi96}  
\item[\nineteenninetyeight] 
\fullcite{cava98}  
\item[\nineteenninetynine] 
\fullcite{samb99} \\ 
\fullcite{samb99b} 
\item[\twothousand] 
\fullcite{deso00}
\item[\twothousandone] 
\fullcite{bomo01} \\
\fullcite{kapo01} \\
\fullcite{kasc01} 
\item[\twothousandtwo] 
\fullcite{shri02} \\
\fullcite{burb02} 
\item[\twothousandfour] 
\fullcite{mifo04} 
\item[\twothousandsix] 
\fullcite{sifg06} \\
\fullcite{iskt06} 
\item[\twothousandseven] 
\fullcite{isks07} 
\item[\twothousandnine] 
\fullcite{sifg09} 
\item[\twothousandtwelve] 
\fullcite{naco12} 
\item[\twothousandfourteen] 
\fullcite{licl14} \\
\fullcite{bakp14} \\
\fullcite{glfo14} 
\item[\twothousandfifteen] 
\fullcite{wahg15} \\
\fullcite{cobs15} \\
\fullcite{sobd15} \\
\fullcite{baka15} 
\item[\twothousandsixteen] 
\fullcite{bocf16} \\
\fullcite{yagu16} \\
\fullcite{baum16} \\
\fullcite{pric16} 
\item[\twothousandseventeen] 
\fullcite{zhli17} 
\item[\twothousandeighteen] 
\fullcite{bofc18} \\
\fullcite{shyp18} 
\item[\twothousandnineteen]
\fullcite{wahg19} 
\item[\twothousandtwenty] 
\fullcite{lufs20} \\
\fullcite{ruml20} \\
\fullcite{orza20} \\
\fullcite{moku20} 
\item[\twothousandtwentyone] 
\fullcite{mabh21} \\
\fullcite{reub21} 
\item[\twothousandtwentytwo] 
\fullcite{dodl22} 
\item[\twothousandtwentythree] 
\fullcite{hogs23} 
\end{itemize}
\end{scriptsize}



%------------------------------------------------------------------------------
%------------------------------------------------------------------------------
\section{Adjoint methods (in geodynamics)}
%------------------------------------------------------------------------------
%------------------------------------------------------------------------------

What Is an Adjoint Model? \cite{erri97}

\begin{scriptsize}
\begin{itemize}
\item[\twothousandthree] 
\fullcite{buht03} 
\item[\twothousandfour] 
\fullcite{isst04} 
\item[\twothousandeight] 
\fullcite{splg08}\\ 
\fullcite{ligu08} 
\item[\twothousandnine] 
\fullcite{wama09} \\
\fullcite{splg09} 
\item[\twothousandthirteen] 
\fullcite{tona13} 
\item[\twothousandfourteen] 
\fullcite{wosp14} \\
\fullcite{hobo14} \\
\fullcite{gran14} 
\item[\twothousandfifteen] 
\fullcite{rasg15} \\
\fullcite{vybu15} 
\item[\twothousandsixteen] 
\fullcite{ghbu16} 
\item[\twothousandseventeen] 
\fullcite{ligs17} 
\item[\twothousandeighteen] 
\fullcite{ghbu18} \\
\fullcite{prda18} \\
\fullcite{cogb18} \\
\fullcite{repk18} \\
\fullcite{fupc18} \\
\fullcite{ghmc18} 
\item[\twothousandnineteen]
\fullcite{mamr19} \\
\fullcite{brad19} 
\item[\twothousandtwenty] 
\fullcite{orza20} \\
\fullcite{rehp20} \\
\fullcite{cobo20} \\
\fullcite{resi20} \\
\fullcite{rehr20} 
\item[\twothousandtwentyone] 
\fullcite{reub21} \\
\fullcite{ghbo21} 
\end{itemize}
\end{scriptsize}

%------------------------------------------------------------------------------
\section{Meshless methods in general}
%------------------------------------------------------------------------------

\begin{scriptsize}
\begin{itemize}
\item[\nineteenninetysix]
\fullcite{beko96}
\item[\twothousand]
\fullcite{begl00}\\ 
\fullcite{lihl00}\\
\fullcite{juim00}
\item[\twothousandone]
\fullcite{idso01}
\item[\twothousandtwo]
\fullcite{lili02}
\item[\twothousandfour]
\fullcite{hufl04}
\end{itemize}
\end{scriptsize}

%------------------------------------------------------------------------------
\section{Meshless methods: SPH}
%------------------------------------------------------------------------------

\begin{scriptsize}
\begin{itemize}
\item[\nineteenseventyseven]
\fullcite{lucy77}
\item[\nineteeneightyfive]   
\fullcite{mona85}
\item[\nineteenninetytwo]    
\fullcite{mona92}
\item[\nineteenninetyseven]
\fullcite{mofz97}
\item[\nineteenninetynine]
\fullcite{zhfm99}\\
\fullcite{ogsa99}
\item[\twothousandthree]
\fullcite{lill03}\\
\fullcite{mamo03}
\item[\twothousandtwo]
\fullcite{lill02} 
\item[\twothousandfourteen] 
\fullcite{lekb14}\\
\fullcite{dazs14}
\item[\twothousandfive]
\fullcite{thje05a}\\
\fullcite{thje05b}\\
\fullcite{febh05}\\
\fullcite{thes05}\\
\fullcite{lixl05}
\item[\twothousandsix]
\fullcite{lili06}
\item[\twothousandeight]
\fullcite{bufs08} \\
\fullcite{lemx08}
\item[\twothousandseven]
\fullcite{busf07}
\item[\twothousandten]
\fullcite{dacl10}
\item[\twothousandeleven]
\fullcite{prcl11}\\
\fullcite{kukg11}\\
\fullcite{kadm11}\\
\fullcite{szpt11}\\
\fullcite{howt11}
\item[\twothousandtwelve]
\fullcite{szpm12}
\item[\twothousandthirteen]
\fullcite{koau13}
\item[\twothousandfifteen]
\fullcite{nifs15}`
\item[\twothousandsixteen]
\fullcite{viro16}
\item[\twothousandeighteen]
\fullcite{krrk18}\\
\fullcite{goej18}
\item[\twothousandtwentythree]
\fullcite{bajg23}
\end{itemize}
\end{scriptsize}

%------------------------------------------------------------------------------
\section{Meshless methods: RKPM}
%------------------------------------------------------------------------------

\begin{scriptsize}
\begin{itemize}
\item[\twothousand]
\fullcite{lihl00}\\
\fullcite{juim00}
\item[\twothousandtwo]  
\fullcite{lilr02}
\item[\twothousandfour] 
\fullcite{wali04}
\end{itemize}
\end{scriptsize}

%------------------------------------------------------------------------------
\section{Meshless methods: Discrete Element Method (DEM)}
%------------------------------------------------------------------------------

\begin{scriptsize}
\begin{itemize}
\item[\twothousandsix] 
\fullcite{yabm06}
\item[\twothousandthirteen]
\fullcite{viau13}\\
\fullcite{koau13}
\item[\twothousandnineteen] 
\fullcite{meho19}\\
\fullcite{meho19b}
\item[\twothousandtwentytwo] 
\fullcite{mink22}
\end{itemize}
\end{scriptsize}

%------------------------------------------------------------------------------
%------------------------------------------------------------------------------
\section{Element Free Galerkin Method}
%------------------------------------------------------------------------------
%------------------------------------------------------------------------------

\begin{scriptsize}
\begin{itemize}
\item[\nineteenninetyfour] 
\fullcite{begl94b} 
\item[\nineteenninetyfive] 
\fullcite{belg95a} \\ 
\fullcite{belg95b} 
\item[\nineteenninetysix] 
\fullcite{bekf96} \\
\fullcite{como96} 
\item[\nineteenninetyseven] 
\fullcite{bekk97} 
\item[\nineteenninetyeight] 
\fullcite{pobe98} \\
\fullcite{zhat98} 
\item[\twothousandthree]
\fullcite{hans03} 
\item[\twothousandfour]
\fullcite{katf04} \\
\fullcite{huvv04} 
\item[\twothousandten]
\fullcite{yiha10} \\
\fullcite{libe10} 
\end{itemize}
\end{scriptsize}

%------------------------------------------------------------------------------
%------------------------------------------------------------------------------
%------------------------------------------------------------------------------
\section{Lattice Boltzmann Method, Lattice Gas Automata}

\begin{scriptsize}
\begin{itemize}
\item[\nineteeneightyseven]
\fullcite{kamz87} 
\item[\twothousandeight]
\fullcite{hupc08} 
\item[\twothousandseventeen]
\fullcite{moyu17} 
\item[\twothousandeighteen]
\fullcite{moyu18} 
\item[\twothousandtwentythree]
\fullcite{momy23} 
\end{itemize}
\end{scriptsize}

%--------------------------------------------------------------------
%------------------------------------------------------------------------------
\section{Solving Stokes Saddle Point problem}
%------------------------------------------------------------------------------
%------------------------------------------------------------------------------

\begin{scriptsize}
\fullcite{laqu86}
\fullcite{rotf90}
\fullcite{frha93}
\fullcite{elgo94}
\fullcite{cheb96}
\fullcite{elma96}
\fullcite{brpv97}
\fullcite{lixu01}
\fullcite{dogs06}
\fullcite{lica06}
\fullcite{hoow17}
\end{scriptsize}

%-------------------------------------------------------------------
%--------------------------------------------------------------------
\section{Segregated methods to solve the Stokes system}
%-------------------------------------------------------------------

\begin{scriptsize}
\fullcite{raju91}
\fullcite{haeh93}
\fullcite{leru95}
\fullcite{duto98}
\fullcite{wade03}
\fullcite{wade04}
\fullcite{utne08}
\end{scriptsize}

%------------------------------------------------------------------------------
%------------------------------------------------------------------------------
\section{Preconditioner business}
%------------------------------------------------------------------------------
%------------------------------------------------------------------------------

\begin{scriptsize}
\fullcite{benz02}
\fullcite{bewa08}
\fullcite{urvs08}
\end{scriptsize}

%------------------------------------------------------------------------------
%------------------------------------------------------------------------------
\section{Benchmark, analytical solutions, code comparisons, methodology, num. methods, theory}
%------------------------------------------------------------------------------
%------------------------------------------------------------------------------
{\color{red} this category makes little sense ... should be split? removed? }

\begin{scriptsize}
\nineteenseventyfour: Hirt \etal \cite{hiac74}\\
\nineteenseventyfive: Wakiya \cite{waki75a,waki75b}\\
\nineteeneightyfour: Yuen \& Sabadini \cite{yusa84}, Smolarkiewicz \cite{smol84}\\
\nineteeneightynine: Blankenbach \etal \cite{blbc89}\\
\nineteenninety: Travis \etal \cite{trab90}\\
\nineteenninetythree: Lenardic \& Kaula \cite{leka93}\\
\nineteenninetyfour: Braun \& Sambridge \cite{brsa94}\\
\nineteenninetyfive: Braun \& Sambridge \cite{brsa95}, Moresi \& Solomatov \cite{moso95}, 
                     Fullsack \cite{full95}\\
\nineteenninetysix: \cite{zhon96}, \cite{mozg96}\\
\nineteenninetyseven: \cite{rist97}\\
\nineteenninetynine: \cite{lind99}, \cite{bird99}\\
\twothousandone: \cite{modm01}, \cite{vank01}\\
\twothousandtwo: \cite{mudm02}\\
\twothousandthree: \cite{taki03}\cite{modm03}\cite{geyu03}\cite{geyu03b}\cite{taxi03}\cite{scpo03}\\
\twothousandfour: \cite{kaps04}\cite{kasa04}\cite{kaks08}\cite{mumc04}\\
\twothousandfive: \cite{mure05}\\
\twothousandsix: \cite{kapo06}\cite{more06}\cite{onmm06}\cite{mudm06}\cite{tact06}\\
\twothousandseven: 
\cite{toma07},
\cite{chcc07},
\cite{kabe07},
\cite{kaks07},
\cite{moql07},
\cite{geyu07},
\cite{dadh07},
\cite{zldf07}\\
\twothousandeight: \cite{zhmt08}\cite{deka08}\cite{trub08}\cite{krdp08}\cite{mamo08}\cite{gepd98}
      \cite{vack08}\cite{heta08}\cite{brtf08}\cite{daks08}\cite{chzy08}\cite{tack08}\cite{hust08b}\\
\twothousandnine: \cite{king09}, \cite{geum09}, \cite{vemm09}, 
                  \cite{qurj09}\\
\twothousandten: \cite{kaus10}\cite{kamm10}\cite{elga10}\cite{kilv10}\\
\twothousandeleven: \cite{dumg11}\cite{uibb11}\cite{hegc11}\cite{muso11}\cite{dawk11}\cite{lemm11}\\
\twothousandtwelve: \cite{crsg12}\cite{chgv12}\cite{krwd12}\cite{may12}\cite{gerb12}\cite{asmo12}\\
\twothousandthirteen: \cite{chtl13}\cite{kemk13}\cite{gemd13}\cite{hutm13}\\
\twothousandfourteen: \cite{thmk14}\cite{mabl14}\cite{lopp14}\cite{stlh14}\\
\twothousandfifteen: \cite{lelk15}\cite{rumi15}\cite{chpe15}\cite{mabl15}\\
\twothousandsixteen: \cite{dumy16}\cite{blmp16}\\
\twothousandseventeen: \cite{robh17}\cite{wisv17}\cite{majc17}\\
\twothousandeighteen: Meriaux \etal \cite{memm18}, Crameri \cite{cram18}, Wieczorek \& Meshede \cite{wime18}\\
\twothousandnineteen: \cite{liki19}\cite{demh19}\cite{galb19}\cite{frtv19}\cite{yuwa19}\cite{ropu19}\\
\twothousandtwenty: \cite{homb20}\cite{trlb20}\cite{gadb20}\cite{jaca20a,jaca20b} 
\twothousandtwentyone: Clevenger \& Heister \cite{clhe21}
\end{scriptsize}




%--------------------------------------------------------------------
\section{Stream Function} 
%--------------------------------------------------------------------

%\begin{scriptsize}
%\begin{itemize}
%\item[\nineteeneightynine]
%\fullcite{mayu89} 
%\item[\nineteenninetysix] 
%\fullcite{laym96} 
%\item[\nineteenninetyseven] 
%\fullcite{wahe97} 
%\item[\nineteenninetynine] 
%\fullcite{sola99}
%\end{itemize}
%\end{scriptsize}

%--------------------------------------------------------------------
%------------------------------------------------------------------------------
\section{Numerical hardware, GPU}
\label{sec:topics:hardware}
%--------------------------------------------------------------------

\begin{scriptsize}
\begin{itemize}
\item[\twothousandsix]
\fullcite{oebm06} 
\item[\twothousandthirteen]
\fullcite{knyu13} \\
\fullcite{gami13} \\
\fullcite{klwh13} \\
\fullcite{sagy13} 
\item[\twothousandfourteen]
\fullcite{zhzg14} 
\item[\twothousandfifteen]
\fullcite{tact15} 
\end{itemize}
\end{scriptsize}

%--------------------------------------------------------------------
%------------------------------------------------------------------------------
\section{Visualization, rendering}
%------------------------------------------------------------------------------
%--------------------------------------------------------------------

\begin{scriptsize}
\begin{itemize}
\item[\twothousandfour] 
\fullcite{rugy04} 
\item[\twothousandfive] 
\fullcite{rugy05} 
\item[\twothousandeight] 
\fullcite{chzy08} \\
\fullcite{stmt08} \\
\fullcite{bikh08} \\
\fullcite{kadt08} \\
\fullcite{dakk08} \\
\fullcite{faha}   
\item[\twothousandtwelve] 
\fullcite{may12} \\
\fullcite{scpo12}
\item[\twothousandfifteen] 
\fullcite{wiab15}
\item[\twothousandseventeen] 
\fullcite{krke17} \\
\fullcite{majc17} 
\item[\twothousandeighteen] 
\fullcite{cram18} 
\item[\twothousandnineteen] 
\fullcite{sutr19} 
\item[\twothousandtwenty] 
\fullcite{crsh20} 
\end{itemize}
\end{scriptsize}

%------------------------------------------------------------------------------
%------------------------------------------------------------------------------
\section{Rheology, plasticity}
%------------------------------------------------------------------------------
%------------------------------------------------------------------------------

\




\chapter{Codes in geodynamics } 
In what follows I make a quick inventory of the main codes of computational geodynamics, 
for crust, lithosphere and/or mantle modelling.

in order to find all CIG-codes citations go to: https://geodynamics.org/cig/news/publications-refbase/

\begin{itemize}

%------------
\item ABAQUS

\cite{gedh02}
\cite{fumr03}
\cite{camg07}
\cite{kuhe09}
\cite{camg10}
\cite{makh09}
\cite{nalr12}
\cite{pevp15}


%------------
\item ADELI

\noindent
1997: \cite{hajc97}\\
2004: \cite{gocl04}\\
2006: \cite{vech06} \\
2008: \cite{boht08a}\cite{boht08b}\\
2012: \cite{gech12}\cite{gigh12}\\
2013: \cite{wahd13}\\
2015: \cite{ceag15}\\
2018: \cite{cegm18}\cite{gehn18}

%------------
\item ASPECT

This code is hosted by CIG at \url{https://geodynamics.org/cig/software/aspect/}. 
It is an open source community code based on the finite element library deal.II. 
It is massively parallel, relies on the p4est library for adaptive mesh refinement,
uses the Trilinos solver library, and can deal with 2D and 3D geometries. 

\cite{bahk07}
\cite{krhb12}
\cite{aupm15}
\cite{tosn15}
\cite{dahe16}
\cite{gadb16}
\cite{zhon16}
\cite{hepb17}
\cite{daef17}
\cite{hedg17}
\cite{robh17}
\cite{robu17}
\cite{aumh17}
\cite{thie17}
\cite{brsg17}
\cite{onmz17}
\cite{tasm17}
\cite{zhli17}
\cite{daga18}
\cite{onzh18}
\cite{gltf18}
\cite{heps18}
\cite{galh18}
\cite{peka18}
\cite{puth18}
\cite{brst18b}
\cite{baba19}
\cite{stbl19}
\cite{cocf19}
\cite{liki19}

%-------------
\item BASIL/ELLE \url{http://elle.ws/}
\cite{bokj08}
\cite{llor19}

%------------
\item CHIC 
\cite{norv15}

%------------
\item CitcomS and CITCOMCU

These codes are hosted by CIG at \url{https://geodynamics.org/cig/software/citcomcu/}
and \url{https://geodynamics.org/cig/software/citcoms/}.

\noindent
1996: \cite{somo96}\\
1997: \cite{mole97}\\
1998: \cite{moso98}\cite{zhgm98}\cite{vazh99}\\
2000: \cite{zhzm00}\cite{gumr00}\\
2001: \cite{bigu01}\\
2002: \cite{tagh02}\\
2003: \cite{vazh03}\cite{cogu03}\cite{bigu03}\\
2004: \cite{solo04}\\
2005: \cite{bihi05}\\
2006: \cite{beck06}\cite{pibf06}\cite{tact06}\cite{besb06}\cite{coli06}\\
2007: \cite{bihi07}\cite{zhzl07}\cite{magu07}\cite{bavi07}\cite{rimb07}\cite{mofm07}\cite{cobs07}\\
2008: \cite{dihf08}\cite{gamc08}\cite{zhmt08}\cite{hole08}\\
2009: \cite{lizh09}\cite{arhm09}\cite{zhzm09}\cite{anbi09}\cite{fobe09}\cite{bubi09}\cite{befa09}\cite{lezh09}\\
2010: \cite{bumb10}\cite{vabv10}\cite{baiv10}\cite{bubi10}\cite{zhzl10}\cite{bill10}\cite{jabi10}\\
2011: \cite{befa11}\cite{lemj11}\cite{vaal11}\cite{legu11}\cite{list11}\\
2012: \cite{arbi12}\cite{jabi12}\cite{bija12}\cite{bova12}\cite{hucf12}\cite{zhym12}\cite{solo12}
\cite{hibi12}\cite{jabk12}\cite{mapm12}\\
2013: \cite{bacs13}\cite{bogs13a}\cite{bogs13b}\cite{jabr13}\cite{qula13}\cite{oldh13}\cite{arbi13}\cite{cost13}\\
2014: \cite{flgw14}\cite{budt14}\cite{kava14}\cite{arfw14}\cite{wavp14}\cite{seki14}\cite{agvg14}
\cite{mabv14}\cite{zhu14}\\
2015: \cite{bacs15}\cite{bogf15}\cite{bomv15}\cite{sefw15}\cite{daso15}\cite{vami15}\cite{wazh15}
\cite{wavp15}\cite{waav15}\cite{hafg15}\cite{tarn15}\cite{legu15}\\
2016: \cite{welm16}\cite{wele16}\cite{jada16}\cite{frbs16}\\
2017: \cite{aggv17}\cite{maav17}\cite{frbm17}\cite{haja17}\\
2018: \cite{hect18}\cite{king18}\cite{kavb18}\\
2019: \cite{mavb19}\cite{fube19}\cite{magn19}\cite{malg19}\cite{mazh19}


\todo[inline]{cross check with CIG database}


%------------
\item CONMAN
This code is hosted by CIG at \url{https://geodynamics.org/cig/software/conman/}

\cite{kirh90}
\cite{itki94}
\cite{kian95}
\cite{kian98}
\cite{itki98}
\cite{befo99}
\cite{nake07}
\cite{dadh07}
\cite{kifr15}


%------------
\item CONVRS 
\cite{yoth12}
\cite{yosh13} 
 



%------------
\item DOUAR

\cite{brtf08}
\cite{thfb08}
\cite{yahb09}
\cite{brya10}
\cite{lobh10}
\cite{mutg13}
\cite{whbb14}
\cite{neew18}
\cite{koen19}

%------------
\item DYNEARTHSOL
\cite{chtl13}


\item ELMER
Elmer is an open source multiphysical simulation software mainly developed by 
CSC - IT Center for Science (CSC). Elmer development was started 1995 in collaboration with 
Finnish Universities, research institutes and industry. Elmer includes physical models of 
fluid dynamics, structural mechanics, electromagnetics, heat transfer and acoustics, 
for example. These are described by partial differential equations which Elmer solves 
by the Finite Element Method (FEM). \url{https://www.csc.fi/web/elmer}

\cite{mals14}


%------------
\item M-DOODS, Duretz code
\cite{yatd12}
\cite{yahb13}
\cite{chmd19}

%------------
\item FENICS
\cite{alrk14}


%------------
\item GAIA

\cite{hutm13}

%------------
\item GALE

This code is hosted by CIG at \url{https://geodynamics.org/cig/software/gale/}

\cite{fabs08}
\cite{gotc08}
\cite{beve10}
\cite{cmwt10}
\cite{lehm12}\cite{liqi12}
\cite{arbi13}

%------------
\item (G)TECTON

\noindent
1980: \cite{mera80}\\
1981: \cite{mera81}\\
1993: \cite{gowo93}\\
1995: \cite{gowo95}\\
1996: \cite{guez96}\\
1999: \cite{gowo99}\\
2001: \cite{bugw01}\cite{gome01}\\
2002: \cite{bugw02}\\
2005: \cite{gowo05}\cite{vanw05}\cite{vabl05}\\
2006: \cite{degw06}\cite{libi06}\cite{scdm06}\\
2007: \cite{vabl07}\\
2009: \cite{ladg09}\\
2011: \cite{bagw11}\cite{bagw11b}\\
2015: \cite{mags15}

%------------
\item ELEFANT

\cite{tosn15}
\cite{matv15}
\cite{busa16}
\cite{latb17}
\cite{thie17}
\cite{pltv18}
\cite{wohu19}

%-------------
\item ELLIPSIS

\cite{modm03}
\cite{omma06} 
\cite{moql07}
\cite{dyrm07}
\cite{onlg08}
\cite{pyeg10}
\cite{legu11}
\cite{lega12}


%------------
\item FANTOM
\index{FANTOM}

\cite{thie11}
\cite{alht11}
\cite{alht12}
\cite{alhf13}
\cite{erhv14}
\cite{thsh14}
\cite{erhv15}
\cite{erhv19}

%------------
\index FDCON

\cite{enbs05}
\cite{fusc13}
\cite{fuks15}


%------------
\index{FLUIDITY}
\item FLUIDITY
\cite{dawk11}
\cite{gagd14}

\index{geoFLAC}
\item geoFLAC (based on PARAVOZ)
\cite{jala19}

%------------
\index{IFISS}
\item IFISS: Incompressible Flow Iterative Solution Solver is a
MATLAB package that is a very useful tool for people interested in
learning about solving PDE’s.
IFISS includes built-in software for 2D versions of:
the Poisson equation, the convection-diffusion equation, the Stokes equations
and the Navier-Stokes equations.\\
\url{https://personalpages.manchester.ac.uk/staff/david.silvester/ifiss/}



%------------
\item the I2(3)E(L)VIS code

2003: \cite{geyu03}\cite{geyu03b}\cite{geur03}\\
2004: \cite{geym04}\cite{geys04}\cite{gepm04}\\
2005: \cite{buge05}\\
2006: \cite{bbeg06}\cite{gest06}\cite{gogc06}\cite{gecy06}\\
2007: \cite{geyu07}\cite{gogc07}\\
2008: \cite{scbe08}\cite{gecy08}\cite{uegs08}\cite{fagc08}\cite{zhgy09}\\
2009: \cite{gefc09}\\
2010: \cite{gerya2010}\cite{nigm10}\\
2011: \cite{dugm11}\cite{dumg11}\cite{lixg11}\cite{gery11}\cite{geme11}\\
2012: \cite{crsg12}\cite{dugk12}\cite{lixg12}\cite{fagm12}\\
2013: \cite{lixg13}\cite{nabg13}\cite{magc13}\cite{vagd13a}\cite{vagd13b}\cite{zhgt13}\cite{dyge13}\cite{gemd13}\cite{mana13}\\
2014: \cite{dugs14}\cite{puge14}\cite{rugb14}\cite{voge14b}\cite{bagb14}\cite{lige14}\cite{stjm14}\cite{malg14}
\cite{buge14}\cite{gosk14}\cite{bagb14}\cite{vamd14}\\
2015: \cite{duay15}\cite{uewg15}\cite{rula15}\cite{gesb15}\cite{rula15}\\
2016: \cite{kobc16}\cite{magc16}\cite{fige16}\\
2019: \cite{kobg19}\cite{ligc19}

%------------
\item I3MG
\cite{facc14}

%------------
\item LAMEM
\cite{scbe08}
\cite{kamm10}
\cite{lemk11}
\cite{may12}
\cite{lesh14}
\cite{cokm14}
\cite{bakp14}
\cite{feka14a}
\cite{feka14b}
\cite{puka15}
\cite{feka15}
\cite{cofk15}
\cite{kapb16}

%------------
\item LAPEX2D (LAgrangian Particle EXplicit, based on the prototype code PAROVOZ) 
\cite{sopg05}
\cite{bbeg06}\cite{basv06}
\cite{baso08}
\cite{scbe08}
\cite{sosk11}


%-------------
\item LITMOD
\cite{afrf07}
\cite{affr08}
\cite{fuac09}
\cite{fufa10}


%------------
\item MARC
\cite{nesg97}
\cite{nesb99}


%------------
\item MILAMIN

MILAMIN is a finite element method implementation in native MATLAB that is capable of doing one million degrees of freedom per minute on a modern desktop computer. This includes pre-processing, solving, and post-processing. The MILAMIN strategies and package are applicable to a broad class of problems in Earth science. \url{http://milamin.org/}

\noindent
2008: \cite{daks08}\\
2010: \cite{krda10}\cite{kaus10}\\
2011: \cite{yakm11}\\
2012: \cite{gebk12}\\
2014: \cite{jobk14}\\
2015: \cite{lukz15}\cite{gehm15}\cite{thkp15}\cite{musd15}\\
2016: \cite{jads16}\cite{maka16}\\
2018: \cite{dusd18}\cite{jasc18}\cite{jadg18}\cite{comj18}\cite{jens18}\cite{rabw18}\cite{chsm18}\\
2019: \cite{anpa19}\cite{sifg19}\cite{baba19}


%------------
\item PARAVOZ/FLAMAR/FLAC

\noindent
1989: \cite{cund89}\\
1993: \cite{poli93}\\
1996: \cite{hach96}\\
1998: \cite{gepd98}\\
2000: \cite{labp00}\\
2001: \cite{bujl01}\cite{bupo01}\\
2002: \cite{bast02}\cite{clbb02}\\
2003: \cite{hags03}\cite{gehd03}\cite{upke03}\\
2004: \cite{guhl04}\cite{gewi04}\cite{toba04}\cite{tibb04}\\
2005: \cite{bugu05}\\
2007: \cite{yaab07}\cite{buto07}\\
2008: \cite{yaba08}\cite{tibb08}\\
2009: \cite{gecm09}\cite{yahb09}\cite{bucl09}\\
2012: \cite{anwb12}\cite{gech12}\cite{gubc12}\cite{gerb12}\\
2013: \cite{wabd13}\cite{frbm13}\\
2014: \cite{frba14}\cite{gagb14}\cite{bufa14}\\
2015: \cite{wulc15}\cite{marl15}\cite{gebw15}\cite{svlh15}\\




%------------
\item PINK3D
\cite{vosc15}


%------------
\item PLASTI
\cite{fuwb06}



%------------
\item pTatin3D: A nice succinct description of the code is given in Appendix B of \cite{lemh17}.

2013: \cite{phil13}\\
2014: \cite{mabl14}\\
2015: \cite{mabl15}\\
2017: \cite{lemh17}\\
2018: \cite{jolp18}\\
2019: \cite{jolm19}

%---------------------
\item Pylith

\cite{aakw13}


%------------
\item RHEA
\cite{bugg08}
\cite{stgb10}
\cite{algs12}
\cite{busa13}

%------------
\item SAMOVAR
\cite{egat10}

%------------
\item SEPRAN

1993: \cite{beky93}\cite{vavy93}\\
1994: \cite{vlvv94}\cite{vayv94}\\
1995: \cite{vayv95}\\
1996: \cite{vayu96}\\
2002: \cite{civv02}\cite{vavv02}\\
2003: \cite{vavs03}\\
2004: \cite{vavv04}\cite{vavv04b}\cite{vavv04c}\\
2005: \cite{vavv05}\cite{sepr05}\\
2006: \cite{liva06a}\cite{liva06b}\\
2007: \cite{vant07}\cite{civv07}\cite{brva07a}\cite{brva07b}\\
2008: \cite{plva08}\\
2009: \cite{vavl09}\\
2010: \cite{vahy10}\cite{syva10}\\
2011: \cite{vahs11}\\
2012: \cite{besy12}\cite{beva12}\cite{chgv12}\\
2013: \cite{ancv13}\\
2014: \cite{chsg14}\cite{mova14}\\
2015: \cite{vasy15}\\
2019: \cite{zhdv19}\cite{vayu19}

%------------
\item SISTER

\cite{olbm16}

%------------
\item SLIM3D

\cite{poso08}
\cite{qusp10}
\cite{brps12}
\cite{brps13}
\cite{brau13}
\cite{brun14}
\cite{hebr14}
\cite{kobf14}
\cite{clbq15}
\cite{brcr17}
\cite{basq18}

%------------
\item SLOMO
\cite{kaus05}

%------------
\index SNAC
\cite{chlg08}


%------------
\item SOPALE

1994: \cite{wibe94}\cite{befh94}\\
1995: \cite{full95}\cite{elfb95}\\
1996: \cite{bekh96}\\
1999: \cite{will99a}\cite{will99b}\\
2000: \cite{pybf00}\cite{bemh00}\\
2001: \cite{bejn01}\\
2002: \cite{hube02}\cite{pybf02}\\
2003: \cite{hube03}\cite{vamf03}\cite{wipo03}\cite{pymi03}\\
2004: \cite{bejn04}\cite{pycr04}\cite{pybe04}\cite{elsp04}\cite{geim04}\\
2005: \cite{gebi05}\cite{hubb05}\\
2006: \cite{pysk06}\cite{selz06}\\
2007: \cite{hube07}\cite{cubh07}\cite{mohb07}\\
2008: \cite{sebp08}\cite{wabj08}\cite{wabj08b}\cite{gopy08}\\
2009: \cite{kecw09}\cite{bejb09}\cite{bupb09}\cite{grba09}\cite{sihb09}\\
2010: \cite{albs10}\cite{albe10}\cite{grpy10}\cite{pygp10}\\
2011: \cite{cube11}\cite{bubj11}\cite{hube11}\\
2012: \cite{grpy12}\cite{grpy12b}\cite{kogp12}\cite{grbe12}\cite{jahu12}\\
2013: \cite{bubj13}\cite{chbe13}\cite{fihv13a}\cite{fihv13b}\cite{gobi13}\cite{grpy13}\cite{knak13}\cite{nipc13}\cite{jahm13}\\
2014: \cite{gogu14}\\
2015: \cite{albe15}\cite{bubj15}\cite{heps15}\\
2016: \cite{licu16}\\
2017: \cite{bube17}


%------------
\item STAGYY
\cite{rota11}
\cite{yadl14}
\cite{crta14}
\cite{cosh18}
\cite{gult19}

%------------
\item SUBMAR

\cite{masr06}
\cite{masp07}
\cite{roms10}


%------------
\item SULEC
SULEC is a finite element code that solves the incompressible Navier-Stokes equations 
for slow creeping flows. The code is developed by Susan Ellis 
(GNS Sciences, NZ) and Susanne Buiter (NGU). 
\url{http://www.geodynamics.no/buiter/sulec.html}

\noindent
2011: \cite{qube11}\cite{ellw11}\\
2012: \cite{buit12}\cite{tebu12}\cite{crsg12}\cite{grel12}\\
2013: \cite{ghbu13}\\
2014: \cite{ghbu14}\cite{qubu14}\\
2015: \cite{nabu15}\\
2016: \cite{zwsn16}\\
2017: \cite{nabp17}\\
2018: \cite{tebu18}











%------------
\item TERRA:
The computational grid is based on a projection of the regular icosahedron onto a 
sphere and successive dyadic refinements \cite{bafr85}.  Concentric copies of such  
spherical layers of nodes build the domain in radial direction.

\cite{baum83}
\cite{glat88}
\cite{buba95}
\cite{burb97}\cite{yang97}
\cite{burl98}
\cite{phbs09}\cite{wodd09}\cite{gows09}
\cite{woda11}
\cite{dadb13}
\cite{vade16}

\index{TerraFERMA}
\item TerraFERMA
\cite{wisv14}
\cite{wisv17}
\cite{spmw16}
\cite{ceww17}
\cite{ceww19}


%------------
\item YACC
\cite{tosn15}
\cite{tomy16}

%------------
\item UNDERWORLD 1\&2

\noindent
2006: \cite{stfs06}\\
2007: \cite{moql07}\cite{stfs07}\\
2008: \cite{lemm08}\cite{ozrs08}\cite{gotc08}\\
2010: \cite{casm10}\cite{mamb10}\cite{stsf10}\cite{stfc10}\cite{fasm10}\\
2011: \cite{memm11}\cite{cafz11}\\
2012: \cite{cafa12}\\
2013: \cite{bemm13}\cite{scmo13}\cite{faca13}\cite{care13}\\
2014: \cite{famc14}\\
2015: \cite{quxm15}\cite{bemm15}\cite{scsp15}\cite{shmj15}\\
2016: \cite{shmv16}\cite{onlw16}\cite{kicf16}\\
2018: \cite{memm18}\\
2019: \cite{samo19}\cite{yamg19}

\item VEMAN
\cite{bepo10}


\end{itemize}


%\chapter{My publications} 
\begin{itemize}
\item[\twothousandthree]
{\bf [01]} \fullcite{dewl03}\\
{\bf [02]} \fullcite{esth03}

\item[\twothousandfive]
{\bf [03]} \fullcite{thje05a}\\
{\bf [04]} \fullcite{thje05b}\\
{\bf [05]} \fullcite{thes05}

\item[\twothousandeight]
{\bf [06]}\fullcite{thfb08}\\ %6
{\bf [07]}\fullcite{thfb08} %7

\item[\twothousandnine]
{\bf [08]} \fullcite{yahb09} %8

\item[\twothousandten]
{\bf [09]} \fullcite{lobh10} %9

\item[\twothousandeleven]
{\bf [10]} \fullcite{thie11}\\ %10
{\bf [11]} \fullcite{alht11} %11

\item[\twothousandtwelve]
{\bf [12]} \fullcite{alht12}\\ %12

\item[\twothousandthirteen]
{\bf [13]} \fullcite{alhf13} %13

\item[\twothousandfourteen]
{\bf [14]} \fullcite{hitg14}\\ %14
{\bf [15]} \fullcite{thsh14} %16

\item[\twothousandfifteen]
{\bf [16]} \fullcite{erhv15}\\ %15
{\bf [17]} \fullcite{matv15}\\ %17
{\bf [18]} \fullcite{vapm15}\\ %18
{\bf [19]} \fullcite{tosn15} %19

\item[\twothousandsixteen]
{\bf [20]} \fullcite{busa16} %20

\item[\twothousandseventeen]
{\bf [21]} \fullcite{latb17}\\ %21
{\bf [22]} \fullcite{thie17} %23

\item[\twothousandeighteen]
{\bf [23]} \fullcite{gltf18}\\ %22
{\bf [24]} \fullcite{thie18}\\ %24
{\bf [25]} \fullcite{pltv18} %25

\item[\twothousandnineteen]
{\bf [26]} \fullcite{frbt19}\\ %26
{\bf [27]} \fullcite{frtv19} %27

\item[\twothousandtwenty]
{\bf [28]} \fullcite{logb20}\\ %28
{\bf [29]} \fullcite{sctc20} %29

\item[\twothousandtwentyone]
{\bf [30]} \fullcite{pirc21} %30

\item[\twothousandtwentytwo]
{\bf [31]} \fullcite{thba22}\\ %31
{\bf [32]} \fullcite{vacp22}\\ %32
{\bf [33]} \fullcite{ross22} %33

\item[\twothousandtwentythree]
{\bf [34]} \fullcite{retv23} %34

\end{itemize}




%\chapter{A \fantom, an \elefant and a \ghost} While a post-doctoral researcher at Bergen University I developed the FANTOM code. Here is what other people and I have published with it:

\begin{itemize}

\item {\it FANTOM : two- and three-dimensional numerical modelling of creeping flows for the solution of geological problems}, 
C. Thieulot, Physics of the Earth and Planetary Interiors, 188, 2011.

\begin{center}
\includegraphics[width=8cm]{images/mycodes/thie11_img}
\end{center}


\item {\it Three-dimensional numerical modeling of upper crustal extensional systems}, 
V. Allken, R.S. Huismans and C. Thieulot, JGR 116, 2011. \url{https://doi:10.1029/2011JB008319} 

\begin{center}
\includegraphics[height=3cm]{images/mycodes/alht11_img}
\end{center}


\item {\it Factors controlling the mode of rift interaction in brittle-ductile coupled systems: A 3D numerical study}, 
V. Allken, R.S. Huismans and C. Thieulot, Geochem. Geophys. Geosyst. 13(5), 2012.
\url{https://doi:10.1029/2012GC004077}

\begin{center}
\includegraphics[height=3cm]{images/mycodes/alht12_img}
\end{center}


\item {\it 3D numerical modelling of graben interaction and linkage: a case study of the Canyonlands grabens, Utah}, 
V. Allken, R.S. Huismans, Haakon Fossen and C. Thieulot, Basin Research, 25, 1-14, 2013.
\url{https://doi: 10.1111/bre.12010}

\begin{center}
\includegraphics[height=3cm]{images/mycodes/alhf13_img}
\end{center}


\item {\it Three-dimensional numerical simulations of crustal systems undergoing orogeny and subjected to surface processes}, 
C. Thieulot, P. Steer and R.S. Huismans, Geochem. Geophys. Geosyst., 15, 2014. doi:10.1002/2014GC005490

\item {\it Extensional inheritance and surface processes as controlling factors of mountain belt structure}, 
Z. Erd\"os, R.S. Huismans, P. van der Beek, and C. Thieulot, J. Geophys. Res. Solid Earth, 119, 2014. doi:10.1002/2014JB011408

\item {\it First-order control of syntectonic sedimentation on crustal-scale structure of mountain belts}, 
Z. Erd\"os, R.S. Huismans, P. van der Beek, J. Geophys. Res. Solid Earth, 120, 5362-5377, 2015. doi:10.1002/2014JB011785

\item {\it The Wilson Cycle and Effects of Tectonic Structural Inheritance
on Rifted Passive Margin Formation}, C.A. Salazar-Mora, R.S. Huismans, H. Fossen and M. Egydio-Silva, 
Tectonics, 37, 3085-3101, 2017. 01. doi:10.1029/2018TC004962 

\begin{center}
\includegraphics[height=5cm]{images/mycodes/sahf18_img}
\end{center}


\item {\it Control of increased sedimentation on orogenic fold-and-thrust belt structure - 
insights into the evolution of the Western Alps}, 
Z. Erd\"os, R.S. Huismans and P. van der Beek, Solid Earth, 10, 391-404, 2019.
\url{https://doi.org/10.5194/se-10-391-2019}

\begin{center}
\includegraphics[height=3cm]{images/mycodes/erhv19_img}
\end{center}

\item {\it Mountain building or backarc extension in ocean-continent subduction systems - a function of
backarc lithospheric strength and absolute plate velocities}, 
S.G. Wolf and R.S. Huismans, JGR, 2019. \url{https://doi.org/10.1029/2018JB017171}

\begin{center}
\includegraphics[height=3cm]{images/mycodes/wohu19_img}
\end{center}


\end{itemize}

Upon my arrival at Utrecht University in 2012 I started working an a more flexible code, called ELEFANT, which has since very much 
diverged from FANTOM.

\begin{itemize}
\item {\it The effect of obliquity on temperature in subduction zones: insights from 3-D numerical modeling}, 
A. Plunder, C. Thieulot and D.J.J. van Hinsbergen, Solid Earth 9, 759-776, 2018. \url{https://doi.org/10.5194/se-9-759-2018}

\begin{center}
\includegraphics[height=3.5cm]{images/mycodes/pltv18_img}
\end{center}


\item {\it Analytical solution for viscous incompressible Stokes flow in a spherical shell}, 
C. Thieulot, Solid Earth 8, 1181-1191, 2017. \url{https://doi.org/10.5194/se-8-1181-2017}

\begin{center}
\includegraphics[height=3cm]{images/mycodes/thie17}
\end{center}



\item  {\it Lithosphere erosion and continental breakup: interaction of extension, plume upwelling and melting}, 
A. Lavecchia, C. Thieulot, F. Beekman, S. Cloetingh and S. Clark, E.P.S.L. 467, p89-98, 2017.

\begin{center}
\includegraphics[height=3cm]{images/mycodes/latv17_img}
\end{center}


\item {\it Benchmarking numerical models of brittle thrust wedges}, 
Susanne J.H. Buiter, Guido Schreurs, Markus Albertz, Taras V. Gerya, Boris Kaus,
Walter Landry, Laetitia le Pourhiet, Yury Mishin, David L. Egholm, Michele Cooke,
Bertrand Maillot, Cedric Thieulot, Tony Crook, Dave May, Pauline Souloumiac, Christopher Beaumont
Journal of Structural Geology 92, p140-177, 2016. \url{https://doi:10.1016/j.jsg.2016.03.003}

\begin{center}
\includegraphics[height=1.8cm]{images/mycodes/busa16_img}
\end{center}


\item {\it A community benchmark for viscoplastic thermal convection in a 2-D square box}, 
N. Tosi, C. Stein, L. Noack, C. Huettig, P. Maierova, H. Samuel, D.R. Davies, C.R. Wilson, S.C. Kramer, C. Thieulot, A. Glerum, M. Fraters, W. Spakman, A. Rozel, P.J. Tackley, Geochem. Geophys. Geosyst. 16, doi:10.1002/2015GC005807, 2015.

\begin{center}
\includegraphics[height=3cm]{images/mycodes/tosn15_img}
\end{center}


\item {\it Dynamics of intraoceanic subduction initiation: 1. Oceanic detachment fault inversion and the formation of supra-subduction zone ophiolites}, M. Maffione, C. Thieulot, D.J.J. van Hinsbergen, A. Morris, O. Pluemper and W. Spakman, Geochem. Geophys. Geosyst. 16, p1753-1770, 2015.

\begin{center}
\includegraphics[height=1.8cm]{images/mycodes/matv15_img}
\end{center}

\item {\it The Geodynamic World Builder: a solution for complex initial conditions in numerical modelling},
M. Fraters, C. Thieulot, A. van den Berg and W. Spakman,
Solid Earth, \url{https://doi.org/10.5194/se-2019-24}, 2019.

\begin{center}
\includegraphics[height=2.8cm]{images/mycodes/frtv19_img}
\end{center}


\end{itemize}


\begin{itemize}
\item {\it GHOST: Geoscientific Hollow Sphere Tessellation}, 
C. Thieulot, Solid Earth, 9, 1169–1177, 2018. \url{https://doi.org/10.5194/se-9-1169-2018}

\begin{center}
\includegraphics[height=3cm]{images/mycodes/shell_HS06}
\includegraphics[height=3cm]{images/mycodes/shell_HS12}
\includegraphics[height=3cm]{images/mycodes/shell_HS20}
\end{center}

\end{itemize}



%\printbibliography
\end{document}
%%%%%%%%%%%%%%%%%%%%%%%%%%%%%%%%%%%%%%%%%%%%%%%%%%%%%%%%%%%
