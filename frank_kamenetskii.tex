\begin{flushright} {\tiny {\color{gray} frank\_kamenetskii.tex}} \end{flushright}
%~~~~~~~~~~~~~~~~~~~~~~~~~~~~~~~~~~~~~~~~~~~~~~~~~~~~~~~~~~~~~~~~~~~~~~~~~~~~~~~~~~~~~~~~~~~~~~~~~~

Many CITCOMs-based publications \cite{bumb10,budt14} 
have used the following (dimensionless) viscosity for the mantle:
\[
\eta(T,z) = \eta_r(r) \exp(A(0.5-T))
\]
where $\eta_r$ is a depth-dependent viscosity profile (usually defined as 
discontinuous linear profiles for various shells)

The non-dimensional activation coefficient is chosen to be $A=9.2103$ in 
\cite{budt14} which leads to a temperature-induced viscosity contrast of $10^4$ (for 
$T\in[0,1]$).

This is also called the Frank-Kamenetskii flow rule, as used in \cite{stha13,lemh17}:
\[
\eta' = \eta_0 \exp(-\theta T)
\]
where the the parameters $\eta_0$, $\theta$ account for the local chemical composition of the rock.
Note that the Frank-Kamenetskii approximation takes many forms in the literature \cite{nobr13}.
\index{general}{Frank-Kamenetskii}

Another temperature-dependent common expression is as follows \cite{flyu84}:
\[
\eta(T)=\eta_\infty \exp \left( \frac{Q}{R}(\frac{1}{T}-\frac{1}{T_\infty} ) \right)
\]
Also, following \cite{flyu84}: For studying transient convection in a non-
Newtonian rheological fluid, it is expedient from a
computational point of view to employ a law
which behaves linearly for low stresses initially
and becomes gradually non-Newtonian only after
a certain threshold stress level has been surpassed \cite{chri84,chyu84}:
\[
\eta(T,p,\tau_2) =\eta(T,p) \frac{1}{A_2 + A_3 \tau_2^2}
\]
where $A_2$ is a parameter describing the linear creep
at low stress levels and $A_3$ governs the transition
stress between Newtonian and non-Newtonian rheologies.

Coltice and Sheppard (2018) \cite{cosh18} use a depth- and temperature-dependent 
viscosity formulation:
\[
\eta(z,T)=\eta_0(z) \exp \frac{Q}{RT}
\]
Note that this expression is supplemented with a pseudo-plastic formulation \cite{roct12}.

\Literature: \cite{king16}





