\chapter{Tensors, coordinate systems and invariants}
\begin{flushright} {\tiny {\color{gray} chapter\_tensors.tex}} \end{flushright}


\section{The strain rate tensor in all coordinate systems}
\begin{flushright} {\tiny {\color{gray} strainrate\_tensor.tex}} \end{flushright}


The strain rate tensor $\dot{\bm\varepsilon}(\vec\upnu)$ is given by
\begin{equation}
\dot{\bm \varepsilon}({\vec \upnu}) 
= \frac{1}{2} \left( {\vec \nabla}{\vec \upnu}+ ({\vec \nabla}{\vec \upnu})^T \right) 
\end{equation}

%.....................................
\subsection{Cartesian coordinates}
\begin{eqnarray}
\dot\varepsilon_{xx} &=& \frac{\partial u}{\partial x} \\
\dot\varepsilon_{yy} &=& \frac{\partial v}{\partial y} \\
\dot\varepsilon_{zz} &=& \frac{\partial w}{\partial z} \\
\dot\varepsilon_{yx} =
\dot\varepsilon_{xy} &=& \frac{1}{2} \left( \frac{\partial u}{\partial y} + \frac{\partial v}{\partial x}  \right)\\
\dot\varepsilon_{zx} =
\dot\varepsilon_{xz} &=& \frac{1}{2} \left( \frac{\partial u}{\partial z} + \frac{\partial w}{\partial x}  \right)\\
\dot\varepsilon_{zy} =
\dot\varepsilon_{yz} &=& \frac{1}{2} \left( \frac{\partial v}{\partial z} + \frac{\partial w}{\partial y}  \right)
\end{eqnarray}

In the \aspect manual there is an interesting discussion about the strain rate tensor in the case of 
2D models: 
\begin{displayquote}
{\color{darkgray}
The notion we adopt here is to think of two-dimensional models in the following way: 
We assume that the domain we want to solve on is a two-dimensional
cross section (parameterized by x and y coordinates) that extends infinitely far in both negative and positive
z direction. Further, we assume that the velocity is zero in z direction and that all variables have no
variation in z direction. As a consequence, we ought to really think of these two-dimensional models as
three-dimensional ones in which the z component of the velocity is zero and so are all z derivatives.}
\end{displayquote}

This of course makes sense but it means that when the deviatoric strain rate tensor needs to be 
computed, then it is given by
\[
\dot{\bm \varepsilon}^d = \dot{\bm \varepsilon} 
- \frac{1}{\bm 3} (\vec\nabla\cdot\vec\upnu) {\bm 1}
=
\left(
\begin{array}{ccc}
\dot{\varepsilon}_{xx} & \dot{\varepsilon}_{xy} &  0 \\
\dot{\varepsilon}_{xy} & \dot{\varepsilon}_{yy} &  0 \\
0 &0 & 0
\end{array}
\right)
- \frac{1}{\bm 3} (\dot{\varepsilon}_{xx}+\dot{\varepsilon}_{yy}) {\bm 1}
=
\frac{1}{3}
\left(
\begin{array}{ccc}
2 \dot{\varepsilon}_{xx} - \dot{\varepsilon}_{yy} & 3\dot{\varepsilon}_{xy} &  0 \\
3\dot{\varepsilon}_{xy} & -\dot{\varepsilon}_{xx}+2 \dot{\varepsilon}_{yy} &  0 \\
0 &0 & -\dot{\varepsilon}_{xx}-\dot{\varepsilon}_{yy}
\end{array}
\right)
\]
As a consequence the shear heating term $\Phi$ is given by 
\begin{eqnarray}
\Phi = 2 \eta \dot{\bm \varepsilon}^d :\dot{\bm \varepsilon}^d 
&=& 2 \eta \frac19
\left[
(2\dot{\varepsilon}_{xx}-\dot{\varepsilon}_{yy})^2 +
(-\dot{\varepsilon}_{xx}+2\dot{\varepsilon}_{yy})^2 +
2\cdot 9\dot{\varepsilon}_{xy}^2
+( -\dot{\varepsilon}_{xx}-\dot{\varepsilon}_{yy})^2 \right] \nn\\
&=& 2\eta \frac19
\left[
4\dot{\varepsilon}_{xx}^2 - 4 \dot{\varepsilon}_{xx}\dot{\varepsilon}_{yy}
+\dot{\varepsilon}_{yy}^2
+ \dot{\varepsilon}_{xx}^2 - 4 \dot{\varepsilon}_{xx}\dot{\varepsilon}_{yy}
+ 4\dot{\varepsilon}_{yy}^2
+ 18 \dot{\varepsilon}_{xy}^2
+ \dot{\varepsilon}_{xx}^2 +2\dot{\varepsilon}_{xx}\dot{\varepsilon}_{yy}
+ \dot{\varepsilon}_{yy}^2 \right] \nn\\
&=& 2\eta \frac19
\left[ 6 \dot{\varepsilon}_{xx}^2 
+ 6 \dot{\varepsilon}_{yy}^2 
-6 \dot{\varepsilon}_{xx}\dot{\varepsilon}_{yy}
+ 18 \dot{\varepsilon}_{xy}^2 \right] \nn\\
&=& 2\eta \left[ \frac{2}{3} \dot{\varepsilon}_{xx}^2 
+ \frac23 \dot{\varepsilon}_{yy}^2 
-\frac23 \dot{\varepsilon}_{xx}\dot{\varepsilon}_{yy}
+ 2 \dot{\varepsilon}_{xy}^2 \right]
\end{eqnarray}


%.............................................
\subsection{Polar coordinates \label{ss:srpc}}

\begin{eqnarray}
\dot\varepsilon_{rr} 
&=& \frac{\partial \upnu_r}{\partial r} \nn\\
\dot\varepsilon_{\theta\theta} 
&=& \frac{\upnu_r}{r} + \frac{1}{r} \frac{\partial \upnu_\theta}{\partial \theta}  \nn\\
\dot\varepsilon_{\theta r} = \dot\varepsilon_{r\theta} 
&=& \frac{1}{2} \left(   \frac{\partial \upnu_\theta}{\partial r} - \frac{\upnu_\theta}{r} 
+\frac{1}{r} \frac{\partial \upnu_r}{\partial \theta}  \right)  \nn
\end{eqnarray}



%........................................................
\subsection{Cylindrical coordinates \label{ss:srcc}}

\begin{eqnarray}
\dot\varepsilon_{rr} 
&=& \frac{\partial \upnu_r}{\partial r} 
\\
\dot\varepsilon_{\theta\theta} 
&=& \frac{\upnu_r}{r} + \frac{1}{r} \frac{\partial \upnu_\theta}{\partial \theta}  
\\
\dot\varepsilon_{\theta r} = \dot\varepsilon_{r\theta} 
&=& \frac{1}{2} \left(   \frac{\partial \upnu_\theta}{\partial r} - \frac{\upnu_\theta}{r} 
+\frac{1}{r} \frac{\partial \upnu_r}{\partial \theta}  \right)
\\
\dot\varepsilon_{zz} 
&=& \frac{\partial \upnu_z}{\partial z} 
\\
\dot{\varepsilon}_{rz} = \dot{\varepsilon}_{zr} 
&=& \frac{1}{2}\left( \frac{\partial \upnu_r}{\partial z} + \frac{\partial \upnu_z}{\partial r}  \right) 
\\
\dot{\varepsilon}_{\theta z} = \dot{\varepsilon}_{z \theta} &=& \frac{1}{2}\left( 
\frac{1}{r} \frac{\partial \upnu_z}{\partial \theta} + \frac{\partial \upnu_\theta}{\partial z}  \right) 
\end{eqnarray}


The velocity divergence is given by
\begin{eqnarray}
\vec{\nabla}\cdot\vec\upnu 
&=& \dot{\varepsilon}_{rr} + \dot{\varepsilon}_{\theta\theta} + \dot{\varepsilon}_{zz} 
 = \cfrac{\partial \upnu_r}{\partial r} + \cfrac{1}{r}\left(\cfrac{\partial \upnu_\theta}{\partial \theta} 
+ \upnu_r \right)  + \cfrac{\partial \upnu_z}{\partial z}
\end{eqnarray} 


%......................................................
\subsection{Spherical coordinates \label{ss:srsc}}

\begin{eqnarray}
\dot\varepsilon_{rr} 
&=& \frac{\partial \upnu_r}{\partial r} \\
\dot\varepsilon_{\theta\theta} 
&=& \frac{\upnu_r}{r} + \frac{1}{r} \frac{\partial \upnu_\theta}{\partial \theta}  \\
\dot\varepsilon_{\phi\phi} 
&=& \frac{1}{r \sin\theta} \frac{\partial \upnu_\phi}{\partial \phi} +
\frac{\upnu_r}{r} +\frac{\upnu_\theta \cot \theta}{r} \\
\dot\varepsilon_{\theta r} = \dot\varepsilon_{r\theta}   
&=& \frac{1}{2} \left( r \frac{\partial}{\partial r} (\frac{\upnu_\theta}{r} ) 
+\frac{1}{r} \frac{\partial \upnu_r}{\partial \theta} \right) \\
\dot\varepsilon_{\phi r} = \dot\varepsilon_{r\phi}      
&=&  \frac{1}{2} \left(  \frac{1}{r \sin\theta} \frac{\partial \upnu_r}{\partial \phi} 
+ r \frac{\partial }{\partial r} (\frac{\upnu_\phi}{r}) \right)  \\
\dot\varepsilon_{\phi \theta} = \dot\varepsilon_{\theta\phi} 
&=& \frac{1}{2} \left( \frac{\sin \theta}{r} \frac{\partial }{\partial \theta} (\frac{\upnu_\phi}{\sin\theta}) + \frac{1}{r \sin\theta} \frac{\partial \upnu_\theta}{\partial \phi}    \right) 
\end{eqnarray}



%...............................................................................
\subsection{Relationship between Cartesian and polar coordinates expressions}

We can go from Cartesian to polar coordinates  via the $2\times 2$ transformation matrix:
\begin{equation}
{\cal P}=
\left(
\begin{array}{ccc}
\cos\theta & \sin\theta \\
-\sin\theta & \cos\theta
\end{array}
\right)
\end{equation}
The rows correspond to the components of $\vec{e}_r$ and $\vec{e}_\theta$ in the Cartesian basis.
A vector $\vec{\upnu}$ transforms from one orthonormal basis to another by multiplying it by 
the matrix ${\cal P}$. As we have seen before, this yields
\begin{eqnarray}
\upnu_r &=& u \cos\theta + v \sin\theta \\
\upnu_\theta &=& -u \sin\theta + v \cos\theta
\end{eqnarray}
A second-order tensor ${\bm a}$ is Cartesian coordinates transforms into ${\bm a}^\star$
in polar coordinates by 
\[
{\bm a}^\star = {\cal P} \cdot {\bm a} \cdot {\cal P}^T
\]
and obviously 
\[
{\bm a} = {\cal P}^T \cdot {\bm a}^\star \cdot {\cal P}
\]
We obtain for the strain rate tensor (or the stress tensor):
\begin{eqnarray}
\dot{\varepsilon}_{rr} 
&=& \dot{\varepsilon}_{xx} \cos^2\theta + \dot{\varepsilon}_{yy} \sin^2\theta 
+ 2 \dot{\varepsilon}_{xy} \sin\theta\cos\theta \nn\\
\dot{\varepsilon}_{\theta\theta}
&=& \dot{\varepsilon}_{xx} \sin^2\theta + \dot{\varepsilon}_{yy} \cos^2\theta 
- 2 \dot{\varepsilon}_{xy} \sin\theta\cos\theta \nn\\
\dot{\varepsilon}_{r\theta} 
&=& \dot{\varepsilon}_{xy} (\cos^2\theta-\sin^2\theta) + 
(\dot{\varepsilon}_{yy} - \dot{\varepsilon}_{xx})\sin\theta \cos\theta \nn
\end{eqnarray}
Using the trigonometric identities $\sin 2\theta = 2 \sin\theta\cos\theta$
and $\cos^2\theta-\sin^2\theta = \cos 2\theta$
, then 
we obtain 
\begin{eqnarray}
\dot{\varepsilon}_{rr} 
&=& \dot{\varepsilon}_{xx} \cos^2\theta + \dot{\varepsilon}_{yy} \sin^2\theta 
+  \dot{\varepsilon}_{xy} \sin 2\theta \nn\\
\dot{\varepsilon}_{\theta\theta}
&=& \dot{\varepsilon}_{xx} \sin^2\theta + \dot{\varepsilon}_{yy} \cos^2\theta 
-  \dot{\varepsilon}_{xy} \sin 2\theta \nn\\
\dot{\varepsilon}_{r\theta} 
&=& \dot{\varepsilon}_{xy} \cos 2\theta + 
\frac12(\dot{\varepsilon}_{yy} - \dot{\varepsilon}_{xx}) \sin 2\theta  \nn
\end{eqnarray}



and likewise:
\begin{eqnarray}
\dot{\varepsilon}_{xx} 
&=& \dot{\varepsilon}_{rr} \cos^2\theta + \dot{\varepsilon}_{\theta\theta} \sin^2\theta - 2 \dot{\varepsilon}_{r\theta} \sin\theta\cos\theta \\
\dot{\varepsilon}_{yy}
&=& \dot{\varepsilon}_{rr} \sin^2\theta + \dot{\varepsilon}_{\theta\theta} \cos^2\theta + 2 \dot{\varepsilon}_{r\theta} \sin\theta\cos\theta \\
\dot{\varepsilon}_{xy} 
&=& \dot{\varepsilon}_{r\theta} (\cos^2\theta-\sin^2\theta) + 
(\dot{\varepsilon}_{rr} - \dot{\varepsilon}_{\theta\theta})\sin\theta \cos\theta \label{ss:srboth}
\end{eqnarray}


















\newpage
%------------------------------------------------------
\section{Principal stress and principal invariants} \label{sec:princ_stress}
\begin{flushright} {\tiny {\color{gray} physics.tex}} \end{flushright}

\index{general}{Maximum Shear Stress} 
\index{general}{Principal Stress}

As seen before (see Section~\ref{sec:stresstensor}) 
the stress tensor is a symmetric $3\times3$ real matrix, and linear algebra tells us that it 
therefore has three mutually orthogonal unit-length eigenvectors $\vec{n}_{1}$, $\vec{n}_{2}$, 
$\vec{n}_{3}$ and three real eigenvalues $\lambda _{1},\lambda _{2},\lambda _{3}$ 
such that ${\bm \sigma}\!\cdot\! \vec{n}_i=\lambda_{i} \vec{n}_{i}$.

%from wiki stress 
As a consequence, in a coordinate system with axes $\vec{n}_{1},\vec{n}_{2},\vec{n}_{3}$, 
the stress tensor is a diagonal matrix, and has only the three normal components $\lambda _{1},\lambda _{2},\lambda _{3}$
i.e. the principal stresses. If the three eigenvalues are equal, the stress is an isotropic compression or tension, always perpendicular to any surface, there is no shear stress, and the tensor is a diagonal matrix in any coordinate frame.

\subsection{In two dimensions}

We are looking for the stress tensor eigenvector vector $\vec{n}=(n_x,n_y)$ associated to the
eigenvalue $\lambda$ such that 
\[
\left(
\begin{array}{cc}
\sigma_{xx} & \sigma_{xy} \\
\sigma_{xy} & \sigma_{yy} 
\end{array}
\right)
\cdot
\left(
\begin{array}{c}
n_x \\ n_y
\end{array}
\right)
=
\lambda
\left(
\begin{array}{c}
n_x \\ n_y
\end{array}
\right)
\]
or,
\[
\left(
\begin{array}{cc}
\sigma_{xx} & \sigma_{xy} \\
\sigma_{xy} & \sigma_{yy} 
\end{array}
\right)
\cdot
\left(
\begin{array}{c}
n_x \\ n_y
\end{array}
\right)
-
\left(
\begin{array}{cc}
\lambda & 0 \\ 
0 & \lambda 
\end{array}
\right)
\cdot
\left(
\begin{array}{c}
n_x \\ n_y
\end{array}
\right)
= \vec{0}
\]
i.e.,
\[
\left(
\begin{array}{cc}
\sigma_{xx}-\lambda  & \sigma_{xy} \\
\sigma_{xy} & \sigma_{yy} -\lambda 
\end{array}
\right)
\cdot
\left(
\begin{array}{c}
n_x \\ n_y
\end{array}
\right)
= \vec{0}
\]
which yields
\[
(\sigma_{xx}-\lambda)(\sigma_{yy}-\lambda)-\sigma_{xy}^2 =0
\]
or, 
\[
\lambda^2 - (\sigma_{xx}+\sigma_{yy}) \lambda   + (\sigma_{xx}\sigma_{yy}-\sigma_{xy}^2) =0
\]
The discriminant $\Delta$ is 
\begin{eqnarray}
\Delta 
&=& (\sigma_{xx}+\sigma_{yy})^2-4(\sigma_{xx}\sigma_{yy}-\sigma_{xy}^2)  \nn\\
&=& (\sigma_{xx}-\sigma_{yy})^2 +4\sigma_{xy}^2  \nn
\end{eqnarray}
The roots are given by:
\begin{eqnarray}
\lambda_\pm 
&=& \frac{ (\sigma_{xx}+\sigma_{yy}) \pm \sqrt{ (\sigma_{xx}-\sigma_{yy})^2 +4\sigma_{xy}^2 } }{2} \nn\\
&=& \frac{ \sigma_{xx}+\sigma_{yy}}{2} \pm \sqrt{ \left(\frac{\sigma_{xx}-\sigma_{yy}}{2}\right)^2 +\sigma_{xy}^2 } \nn
\end{eqnarray}
The two principal stresses are then:
\begin{mdframed}[backgroundcolor=blue!5]
\begin{eqnarray}
\sigma_1 &=& \frac{ \sigma_{xx}+\sigma_{yy}}{2} 
+ \sqrt{ \left(\frac{\sigma_{xx}-\sigma_{yy}}{2}\right)^2 +\sigma_{xy}^2 } \nn\\
\sigma_2 &=& \frac{ \sigma_{xx}+\sigma_{yy}}{2} 
- \sqrt{ \left(\frac{\sigma_{xx}-\sigma_{yy}}{2}\right)^2 +\sigma_{xy}^2 } \label{eq:princ_stress_2D} 
\end{eqnarray}
\end{mdframed}
with the convention $\sigma_1>\sigma_2$.
The maximum shear stress is defined as one-half the difference between the two principal 
stresses 
\begin{mdframed}[backgroundcolor=blue!5]
\begin{equation}
\tau_{\text max}=
\frac{\sigma_1-\sigma_2}{2}
=\sqrt{ \left(\frac{\sigma_{xx}-\sigma_{yy}}{2}\right)^2 +\sigma_{xy}^2 }
\label{eq:max_shear_stress_2D} 
\end{equation}
\end{mdframed}
The eigenvector $\vec{n}_1$ corresponding to $\sigma_1$ is obtained by solving 
\[
{\bm \sigma}\!\cdot\! \vec{n}_1 = \sigma_1 \vec{n}_1
\]
and same for the other eigenvalue/vector:
\[
{\bm \sigma} \!\cdot\! \vec{n}_2 = \sigma_2 \vec{n}_2
\]
Each is a system of two equations with two unknowns. These are not difficult to solve, 
but can prove cumbersome. Note that linear algebra tells us that $\vec{n}_1\cdot\vec{n}_2=0$, 
i.e. the eigenvectors form a basis of $\mathbb{R}^2$.

This is the reason why often people go another route. One can ask the question: what is the 
value of the angle $\theta_p$ which, if used to perform a rotation of the axis system, yields 
a stress tensor that is diagonal, with the principal stresses on the diagonal?
 
\begin{center}
\includegraphics[width=9cm]{images/princ_stress/PrincipalStress}\\
{\scriptsize Taken from \url{https://www.efunda.com/formulae/solid_mechanics/mat_mechanics/plane_stress_principal.cfm}}
\end{center}
The rotation matrix is 
\[
{\bm R}=
\left(
\begin{array}{cc}
\cos\theta_p & -\sin\theta_p \\
\sin\theta_p & \cos\theta_p
\end{array}
\right)
\]
and the image of ${\bm \sigma}$ by means of the axis rotation is 
${\bm \sigma}'= {\bm R}\cdot {\bm \sigma}\cdot {\bm R}^{-1}$, i.e.
\begin{eqnarray}
{\bm \sigma}' 
&=&
\left(
\begin{array}{cc}
\cos\theta_p & -\sin\theta_p \\
\sin\theta_p & \cos\theta_p
\end{array}
\right)
\cdot
\left(
\begin{array}{cc}
\sigma_{xx} & \sigma_{xy} \\
\sigma_{xy} & \sigma_{yy} 
\end{array}
\right)
\cdot
\left(
\begin{array}{cc}
\cos\theta_p & \sin\theta_p \\
-\sin\theta_p & \cos\theta_p
\end{array}
\right) \nn\\
&=&\left(
\begin{array}{cc}
\cos\theta_p & -\sin\theta_p \\
\sin\theta_p & \cos\theta_p
\end{array}
\right)
\cdot
\left(
\begin{array}{cc}
\sigma_{xx} \cos\theta_p - \sigma_{xy} \sin\theta_p  &
\sigma_{xx} \sin\theta_p + \sigma_{xy} \cos\theta_p  \\
\sigma_{xy} \cos\theta_p - \sigma_{yy} \sin\theta_p & 
\sigma_{xy} \sin\theta_p + \sigma_{yy} \cos\theta_p 
\end{array}
\right) \nn\\
&=&
\left(
\begin{array}{cc}
\dots & 
\cos\theta_p(\sigma_{xx} \sin\theta_p + \sigma_{xy} \cos\theta_p )-
\sin\theta_p(\sigma_{xy} \sin\theta_p + \sigma_{yy} \cos\theta_p ) \\
\dots & \dots 
\end{array}
\right) \nn 
\end{eqnarray}
In the matrix above I have only computed the off diagonal term since 
we are actually looking for $\theta_p$ such that $\sigma_{xy}'=0$, or
\begin{eqnarray}
\cos\theta_p(\sigma_{xx} \sin\theta_p + \sigma_{xy} \cos\theta_p )-
\sin\theta_p(\sigma_{xy} \sin\theta_p + \sigma_{yy} \cos\theta_p ) &=& 0 \nn\\
\sin\theta_p \cos\theta_p (\sigma_{xx}-\sigma_{yy}) + ( \cos^2\theta_p -\sin^2\theta_p )\sigma_{xy} &=& 0 \nn 
\end{eqnarray}
and then 
\[
\frac{ \sin\theta_p \cos\theta_p}{ \cos^2\theta_p -\sin^2\theta_p }
= \frac{\sigma_{xy}}{ \sigma_{xx}-\sigma_{yy} }
\]
The left hand term is actually a trigonometric 
identity\footnote{\url{https://en.wikipedia.org/wiki/List_of_trigonometric_identities}}:
\[
\frac{ \sin\theta_p \cos\theta_p}{ \cos^2\theta_p -\sin^2\theta_p } 
= \frac{\frac12 \sin 2\theta_p}{\cos 2\theta_p}
= \frac{1}{2} \tan 2\theta_p
\]
and finally:
\[
\tan 2\theta_p = \frac{ 2\sigma_{xy}}{ \sigma_{xx}-\sigma_{yy} }
\qquad
\text{or}
\qquad
\boxed{
\theta_p = \frac{1}{2} \tan^{-1} \frac{ 2\sigma_{xy}}{ \sigma_{xx}-\sigma_{yy} }
}
\]
Once $\theta_p$ has been found the other direction is given by $\theta_p +\pi/2$.

\vspace{.5cm}

\noindent \underline{Example}: Let us assume a diagonal stress tensor of the form 
\[
{\bm \sigma} = 
\left(
\begin{array}{cc}
a & 0 \\
0 & b
\end{array}
\right)
\]
then $\tan 2\theta_p = 0$, and then $\theta_p=0$. The principal directions are the horizontal and 
vertical directions, i.e. the Cartesian axis system, which is consistent.


\todo[inline]{add a remark that 2D does not exist and that plane strain incompressible actually is
what is going on above}


%...................................... 
\subsection{In three dimensions \label{ss:lode}}

We are looking for the stress tensor eigenvector vector $\vec{n}=(n_x,n_y,n_z)$ associated to the
eigenvalue $\lambda$ such that 
\[
\left(\begin{array}{ccc}
\sigma_{xx} & \sigma_{xy} & \sigma_{xz} \\
\sigma_{xy} & \sigma_{yy} & \sigma_{yz} \\
\sigma_{xz} & \sigma_{yz} & \sigma_{zz}
\end{array}\right)
\cdot
\left(\begin{array}{c}
n_x \\ n_y \\ n_z
\end{array}\right)
=
\lambda
\left(\begin{array}{c}
n_x \\ n_y \\ n_z
\end{array}\right)
\]
or,
\[
\left(\begin{array}{ccc}
\sigma_{xx} & \sigma_{xy} & \sigma_{xz} \\
\sigma_{xy} & \sigma_{yy} & \sigma_{yz} \\
\sigma_{xz} & \sigma_{yz} & \sigma_{zz}
\end{array}\right)
\cdot
\left(\begin{array}{c}
n_x \\ n_y \\ n_z
\end{array}\right)
-
\left(\begin{array}{ccc}
\lambda & 0 & 0\\ 
0 & \lambda  & 0 \\
0 & 0 & \lambda 
\end{array}\right)
\cdot
\left(\begin{array}{c}
n_x \\ n_y \\ n_z
\end{array}\right)
= \vec{0}
\]

\[
\left(\begin{array}{ccc}
\sigma_{xx}-\lambda & \sigma_{xy} & \sigma_{xz} \\
\sigma_{xy} & \sigma_{yy}-\lambda & \sigma_{yz} \\
\sigma_{xz} & \sigma_{yz} & \sigma_{zz} -\lambda
\end{array}\right)
\cdot
\left(\begin{array}{c}
n_x \\ n_y \\ n_z
\end{array}\right)
= \vec{0}
\]
Non-trivial solutions of this equation require 
\[
\left|  
\begin{array}{ccc}
\sigma_{xx}-\lambda & \sigma_{xy} & \sigma_{xz} \\
\sigma_{xy} & \sigma_{yy}-\lambda & \sigma_{yz} \\
\sigma_{xz} & \sigma_{yz} & \sigma_{zz} -\lambda
\end{array}
\right|
=0
\]
Expanding the determinant results in the following cubic equation:
\begin{eqnarray}
0 
&=&
(\sigma_{xx}-\lambda) [ ( \sigma_{yy}-\lambda )( \sigma_{zz} -\lambda)- \sigma_{yz}^2]
- \sigma_{xy} [ \sigma_{xy} ( \sigma_{zz} -\lambda) - \sigma_{yz} \sigma_{xz} ] 
+ \sigma_{xz} [ \sigma_{xy}  \sigma_{yz} - (\sigma_{yy}-\lambda )  \sigma_{xz} ]  \nn\\
&=& (\sigma_{xx}-\lambda) [ \sigma_{yy}\sigma_{zz} -\lambda (\sigma_{yy}+ \sigma_{zz})+ \lambda ^2- \sigma_{yz}^2]
- \sigma_{xy} [ \sigma_{xy} ( \sigma_{zz} -\lambda) - \sigma_{yz} \sigma_{xz} ] 
+ \sigma_{xz} [ \sigma_{xy}  \sigma_{yz} - (\sigma_{yy}-\lambda )  \sigma_{xz} ] \nn\\
&=& -\lambda^3
+ (\sigma_{xx} + \sigma_{yy} + \sigma_{zz} )\lambda^2
+ (-\sigma_{yy}\sigma_{zz} -\sigma_{xx}\sigma_{yy} -\sigma_{xx}\sigma_{zz} 
  +\sigma_{yz}^2 +\sigma_{xy}^2 + \sigma_{xz}^2 )\lambda
+ det({\bm \sigma}) \nn
\end{eqnarray}
or, after multiplying the last line by -1,
\begin{equation}
\lambda^3 - {\KKK}_1({\bm \sigma}) \lambda^2 + {\KKK}_2({\bm \sigma}) \lambda -{\KKK}_3({\bm \sigma})=0
\label{eq:prinv:KKK}
\end{equation}
with\footnote{Note that in the equation \eqref{eq:prinv:KKK} there is often a plus sign in front of ${\KKK}_2$ 
but not always. Be careful when reading literature!}:
\begin{eqnarray}
{\KKK}_1({\bm \sigma}) &=& \sigma_{xx}+\sigma_{yy}+\sigma_{zz}\nn\\
{\KKK}_2({\bm \sigma}) &=& \sigma_{xx}\sigma_{yy}+\sigma_{yy}\sigma_{zz}+\sigma_{xx}\sigma_{zz}
-\sigma_{xy}^2 -\sigma_{xz}^2 -\sigma_{yz}^2 \nn\\
{\KKK}_3({\bm \sigma}) 
&=& \det ({\bm \sigma}) \nn\\
&=& \sigma_{xx}\sigma_{yy}\sigma_{zz}-\sigma_{xx}\sigma_{yz}^2
-\sigma_{xy}^2\sigma_{zz}+\sigma_{xy}\sigma_{yz}\sigma_{xz}
+\sigma_{xz}\sigma_{xy}\sigma_{yz}-\sigma_{xz}^2\sigma_{yy} \nn\\
&=& \sigma_{xx}\sigma_{yy}\sigma_{zz}
+2\sigma_{xy}\sigma_{yz}\sigma_{xz}
-( \sigma_{xx}\sigma_{yz}^2 +  \sigma_{zz} \sigma_{xy}^2 + \sigma_{yy}\sigma_{xz}^2 )
\end{eqnarray}
\index{general}{Principal Invariants}

\noindent ${\KKK}_1$, ${\KKK}_2$ and ${\KKK}_3$ are called {\bf principal}
invariants\footnote{\url{https://en.wikipedia.org/wiki/Invariants_of_tensors}} 
(see also Appendix A.1 of Zienkiewicz \& Taylor \cite{zita2} or  Eq.~(6.4) of Freudenthal \& Geiringer \cite{frge58}). 
These invariants can be written in a coordinate-free manner\footnote{Proofs are in 
Appendix~\ref{app:invariants}}:
\begin{mdframed}[backgroundcolor=blue!5]
\begin{eqnarray}
{\KKK}_1({\bm \sigma}) &=& {\rm tr}({\bm \sigma})  \nn\\
{\KKK}_2({\bm \sigma}) &=& \frac{1}{2}(  {\rm tr}({\bm \sigma}) ^2 - {\rm tr}({\bm \sigma}^2)  ) \nn\\
{\KKK}_3({\bm \sigma}) &=& \det ({\bm \sigma}) \nn
\end{eqnarray}
\end{mdframed}
and if the stress tensor is diagonal, we have
\begin{eqnarray}
{\KKK}_1({\bm \sigma}) &=& \sigma_{1}+\sigma_{2}+\sigma_{3}\nn\\
{\KKK}_2({\bm \sigma}) &=& \sigma_{1}\sigma_{2}+\sigma_{2}\sigma_{3}+\sigma_{1}\sigma_{3} \nn\\
{\KKK}_3({\bm \sigma}) &=& \sigma_1\sigma_2\sigma_3 \nn
\end{eqnarray}
The principal invariants ${\KKK}_{\{1,2,3\}}$ are related to the {\bf moment} 
invariants ${\III}_{\{1,2,3\}}$ 
(see Section~\ref{sec:invariants}) as follows (Appendix A.2 of Zienkiewicz \& Taylor \cite{zita2}):
\begin{eqnarray}
{\III}_1({\bm \sigma})&=& {\KKK}_1({\bm \sigma}) \\ 
{\III}_2({\bm \sigma})&=& \frac{1}{2}{\KKK}_1({\bm \sigma})^2 -{\KKK}_2({\bm \sigma}) \label{eq:IK2}\\
{\III}_3({\bm \sigma})&=& \frac{1}{3}{\KKK}_1({\bm \sigma})^3 -{\KKK}_1({\bm \sigma}) 
{\KKK}_2({\bm \sigma}) + {\KKK}_3({\bm \sigma}) \label{eq:IK3}
\end{eqnarray}
\todo[inline]{write proofs in appendix}
Very often we will find ourselves interested in the principal components 
of the deviatoric stress tensor $\bm \tau$ so that we now have the following determinant to compute:
\[
\left|  
\begin{array}{ccc}
\tau_{xx}-\lambda & \tau_{xy} & \tau_{xz} \\
\tau_{xy} & \tau_{yy}-\lambda & \tau_{yz} \\
\tau_{xz} & \tau_{yz} & \tau_{zz} -\lambda
\end{array}
\right|
=0
\]
and therefore obtain the following cubic equation
\begin{equation}
\lambda^3 - {\KKK}_1({\bm \tau}) \lambda^2 + {\KKK}_2({\bm \tau}) \lambda -{\KKK}_3({\bm \tau})=0
\end{equation}
By definition of a deviatoric tensor we have ${\KKK}_1({\bm \tau})=0$ and 
then Eqs.~\eqref{eq:IK2} and \eqref{eq:IK3} become
\begin{eqnarray}
{\III}_2({\bm \tau}) &=&  - {\KKK}_2({\bm \tau}) \\
{\III}_3({\bm \tau}) &=&  {\KKK}_3({\bm \tau}) 
\end{eqnarray}
so that the cubic equation becomes
\begin{equation} 
\lambda^3 -  {\III}_2({\bm \tau}) \lambda -  {\III}_3({\bm \tau}) =0 \label{opopop}
\end{equation}
Noting the trigonometric identity\footnote{see section 7.4 of Owen \& Hinton \cite{owhi}}
\begin{equation}
\sin 3\theta = 3 \sin \theta - 4 \sin^3 \theta
\qquad
{\rm or,}
\qquad
\sin^3 \theta - \frac{3}{4}\sin \theta + \frac{1}{4} \sin 3\theta = 0\label{pc_eq2}
\end{equation}
and substituting $\lambda=r\sin \theta$ into (\ref{opopop}) we have\footnote{Note that $r$ and $\theta$ have nothing 
to do with polar, cylindrical or spherical coordinates.}
\begin{equation}
\sin^3 \theta -\frac{ {\III}_2({\bm \tau}) }{r^2} \sin \theta -\frac{ {\III}_3({\bm \tau})  }{r^3}=0\label{pc_eq3}
\end{equation}
Comparing (\ref{pc_eq2}) and (\ref{pc_eq3}) gives
\begin{eqnarray}
r&=&\frac{2}{\sqrt{3}}\sqrt{ {\III}_2({\bm \tau})  }\label{pc_eq4bis} \\
\sin 3 \theta &=& -\frac{4 {\III}_3({\bm \tau})  }{r^3}=-\frac{3\sqrt{3}}{2}\frac{ {\III}_3({\bm \tau}) }{ {\III}_2({\bm \tau}) ^{3/2}} \label{pc_eq4}
\end{eqnarray}

The so-called Lode angle  \cite{zico74} is then given by \index{general}{Lode Angle}
\begin{mdframed}[backgroundcolor=blue!5]
\begin{equation}
\uptheta_{\rm L}=\frac{1}{3} \sin^{-1} 
\left( -\frac{3\sqrt{3}}{2} \frac{{\III}_3({\bm \tau})}{{\III}_2({\bm \tau})^{3/2}} \right)
\label{eq:lodang}
\end{equation}
\end{mdframed}
with $-\pi/6 <\uptheta_{\rm L} <\pi/6 $. The very same equation is 
also found in Willett (1992) \cite{will92} for instance.

The first root of (\ref{pc_eq4}) with $\uptheta_{\rm L}$ determined for $3\uptheta_{\rm L}$ in the 
range $\pm \pi/2$ is a convenient alternative to the third invariant, ${\III}_3({\bm \tau})$. 
By noting the cyclic nature of $\sin (3\uptheta_{\rm L}+2n \pi)$ we have immediatly the three 
(and only three) possible values of $\sin \uptheta_{\rm L} $ which define the three principal stresses. 
The deviatoric principal stresses are given by $\lambda=r \sin \uptheta_{\rm L}$ on substitution 
of the three values of $\sin \uptheta_{\rm L}$ in turn. 

We then obtain 
\begin{equation}
\left\{
\begin{array}{c}
\tau_1 \\ \\
\tau_2 \\ \\
\tau_3
\end{array}
\right\}
= \frac{2  }{\sqrt{3}}\sqrt{ {\III}_2({\bm \tau})  }
\left\{
\begin{array}{c}
\sin (\uptheta_{\rm L} + 2\pi/3)  \\ \\
\sin \uptheta_{\rm L}   \\ \\
\sin (\uptheta_{\rm L} + 4\pi/3  )
\end{array}
\right\}
\end{equation}
with $\tau_1>\tau_2>\tau_3$ and $-\pi/6 \leq \uptheta_{\rm L} \leq \pi/6$. It is indeed easy to verify that 
for $-\pi/6 \leq \uptheta_{\rm L} \leq \pi/6$ we have  
$\sin (\uptheta_{\rm L} + 2\pi/3) > \sin \uptheta_{\rm L} > \sin (\uptheta_{\rm L} + 4\pi/3)$.

Finally, we wish to compute the principal stresses of the full stress tensor ${\bm \sigma}$.
In the right coordinate system both stress and deviatoric stress tensors are diagonal and 
${\bm \sigma}=-p {\bm 1} + {\bm \tau}$ writes:
\[
\left(
\begin{array}{ccc}
\sigma_1 &0 &0 \\
0& \sigma_2 &0 \\
0&0 & \sigma_3  
\end{array}
\right)
=
\left(
\begin{array}{ccc}
-p&0&0\\
0&-p&0\\
0&0&-p
\end{array}
\right)
+
\left(
\begin{array}{ccc}
\tau_1 & 0&0 \\
0& \tau_2 & 0\\
0&0 & \tau_3  
\end{array}
\right)
\]
so that (since $p=-\frac{1}{3}tr({\bm \sigma})=-\frac{1}{3}{\III}_1({\bm \sigma})$) 
\begin{eqnarray}
\sigma_1 &=& \tau_1 - p = \tau_1 + \frac{1}{3}{\III}_1({\bm \sigma})\\ 
\sigma_2 &=& \tau_2 - p = \tau_2 + \frac{1}{3}{\III}_1({\bm \sigma})\\ 
\sigma_3 &=& \tau_3 - p = \tau_3 + \frac{1}{3}{\III}_1({\bm \sigma}) 
\end{eqnarray}
and finally the total principal stresses are
\begin{equation}
\left\{
\begin{array}{c}
\sigma_1 \\ \\
\sigma_2 \\ \\
\sigma_3
\end{array}
\right\}
= \frac{2  }{\sqrt{3}}\sqrt{ {\III}_2({\bm \tau})  }
\left\{
\begin{array}{c}
\sin (\uptheta_{\rm L} + 2\pi/3)  \\ \\
\sin \uptheta_{\rm L}   \\ \\
\sin (\uptheta_{\rm L} + 4\pi/3  )
\end{array}
\right\}
+
\frac{{\III}_1({\bm \sigma})}{3}
\left\{
\begin{array}{c}
1 \\ \\
1 \\ \\
1
\end{array}
\right\}
\end{equation}
with $\sigma_1>\sigma_2>\sigma_3$ and $-\pi/6 \leq \uptheta_{\rm L} \leq \pi/6$. 
We have
\begin{eqnarray}
\sin (\uptheta_{\rm L} + 2\pi/3)  
&=& \sin \uptheta_{\rm L} \cos 2\pi/3 + \cos \uptheta_{\rm L} \sin 2\pi/3 \nn\\
&=& -\frac{1}{2}\sin \uptheta_{\rm L}  + \cos \uptheta_{\rm L} \frac{\sqrt{3}}{2} \\
\sin (\uptheta_{\rm L} + 4\pi/3)  
&=& \sin \uptheta_{\rm L} \cos 4\pi/3 + \cos \uptheta_{\rm L} \sin 4\pi/3  \nn\\
&=& -\frac{1}{2} \sin \uptheta_{\rm L} - \cos \uptheta_{\rm L} \frac{\sqrt{3}}{2} 
\end{eqnarray}
so that 
\begin{eqnarray}
\left\{
\begin{array}{c}
\sigma_1 \\ \\
\sigma_2 \\ \\
\sigma_3
\end{array}
\right\}
&=& \frac{2  }{\sqrt{3}}\sqrt{ {\III}_2({\bm \tau}) }
\left\{
\begin{array}{c}
-\frac{1}{2}\sin \uptheta_{\rm L}  + \cos \uptheta_{\rm L} \frac{\sqrt{3}}{2} \\ \\
\sin \uptheta_{\rm L}   \\ \\
-\frac{1}{2} \sin \uptheta_{\rm L} - \cos \uptheta_{\rm L} \frac{\sqrt{3}}{2} 
\end{array}
\right\}
+
\frac{{\III}_1({\bm \sigma})}{3}
\left\{
\begin{array}{c}
1 \\ \\
1 \\ \\
1
\end{array}
\right\} \\
&=& \sqrt{ {\III}_2({\bm \tau}) }
\left\{
\begin{array}{c}
-\frac{1}{\sqrt{3}}\sin \uptheta_{\rm L}  + \cos \uptheta_{\rm L} \\ \\
\frac{2}{\sqrt{3}} \sin \uptheta_{\rm L} \\ \\
-\frac{1}{\sqrt{3}}\sin \uptheta_{\rm L}  - \cos \uptheta_{\rm L} 
\end{array}
\right\}
+
\frac{{\III}_1({\bm \sigma})}{3}
\left\{
\begin{array}{c}
1 \\ \\
1 \\ \\
1
\end{array}
\right\}
\label{eq:sig123}
\end{eqnarray}





\begin{remark} The Lode angle is one of the Lode 
coordinates\footnote{\url{https://en.wikipedia.org/wiki/Lode_coordinates}},
or Haigh-Westergaard coordinates. 
\index{general}{Haigh-Westergaard Coordinates}
\index{general}{Lode Coordinates}
\end{remark}

\begin{remark} The Lode angle $\uptheta_{\rm L}$ is essentially similar to the 
Lode parameter \index{general}{Lode Parameter} defined by $-\sqrt{3}\tan\uptheta$ \cite{owhi}.
\end{remark}

\begin{remark}
There are 3 different Lode angles, as 
explained online\footnote{\url{https://en.wikipedia.org/wiki/Lode_coordinates}}:
\[
\sin 3\uptheta_s = -\sin 3 \bar{\uptheta}_s = \cos 3\uptheta_c = \frac{3\sqrt{3}}{2}\frac{{\III}_3({\bm \tau})}{({\III}_2({\bm \tau}))^{3/2}}
\]
and they are related by $\uptheta_s = \frac{\pi}{6}-\uptheta_c$ and $\uptheta_s = -\bar{\uptheta}_s$. 
The one used in this document is in fact the $\bar{\uptheta}_s$ above.
\label{rq:signs}
\end{remark}

%\newpage
To recap:
\begin{mdframed}[backgroundcolor=blue!5]
\begin{eqnarray}
\sigma_1 &=& \frac{{\III}_1({\bm \sigma})}{3} + \sqrt{{\III}_2({\bm \tau})} \left(-\frac{1}{\sqrt{3}}\sin \uptheta_{\rm L}  +\cos\uptheta_{\rm L} \right) \label{eq:sigma1} \\ 
\sigma_2 &=& \frac{{\III}_1({\bm \sigma})}{3} + \sqrt{{\III}_2({\bm \tau})} \left(\frac{2}{\sqrt{3}}\sin 
\uptheta_{\rm L}   \right)    \label{eq:sigma2} \\
\sigma_3 &=& \frac{{\III}_1({\bm \sigma})}{3} + \sqrt{{\III}_2({\bm \tau})} \left(-\frac{1}{\sqrt{3}}\sin \uptheta_{\rm L}  - \cos \uptheta_{\rm L} \right)    \label{eq:sigma3}
\end{eqnarray}
\end{mdframed}

We will later need $\sigma_1-\sigma_3$ and $\sigma_1+\sigma_3$ so we compute these
quantities hereafter:

\begin{eqnarray}
\sigma_1 -\sigma_3
&=&  \sqrt{{\III}_2({\bm \tau})} \left( 
-\frac{1}{\sqrt{3}}\sin \uptheta_{\rm L}  + \cos \uptheta_{\rm L} 
+\frac{1}{\sqrt{3}}\sin \uptheta_{\rm L}  + \cos \uptheta_{\rm L} \right) \nn\\
&=& 2 \cos \uptheta_{\rm L} \sqrt{{\III}_2({\bm \tau})} \\ 
\sigma_1 + \sigma_3 
&=&   
\frac{{\III}_1({\bm \sigma})}{3} + \sqrt{{\III}_2({\bm \tau})} \left(-\frac{1}{\sqrt{3}}\sin 
\uptheta_{\rm L}  + \cos \uptheta_{\rm L} \right)   
+\frac{{\III}_1({\bm \sigma})}{3} + \sqrt{{\III}_2({\bm \tau})} \left(-\frac{1}{\sqrt{3}}\sin 
\uptheta_{\rm L}  - \cos \uptheta_{\rm L} \right)   
 \nn\\
&=& 
\frac{2}{3} {\III}_1({\bm \sigma}) -\sqrt{ {\III}_2({\bm \tau})} \frac{2}{\sqrt{3}}\sin \uptheta_{\rm L} 
\end{eqnarray}
or, 
\begin{mdframed}[backgroundcolor=blue!5]
\begin{eqnarray}
\frac{\sigma_1 -\sigma_3}{2} &=&  \cos \uptheta_{\rm L} \sqrt{{\III}_2({\bm \tau})}  \label{eq:sig13a} \\
\frac{\sigma_1 + \sigma_3}{2} &=& \frac{1}{3} {\III}_1({\bm \sigma}) -\sqrt{{\III}_2({\bm \tau})} \frac{1}{\sqrt{3}}\sin \uptheta_{\rm L} \label{eq:sig13b}
\end{eqnarray}
\end{mdframed}

\begin{remark}
The expression for the Lode angle is different in \cite[p101]{book_zitf} than 
in \cite{zico74} or \cite[p62]{zita2}. They all look suspiciously wrong too.
\end{remark}

%.........................................................................
\subsection{About the 2nd principal invariant of the deviatoric stress}

\begin{eqnarray}
{\KKK}_2({\bm \tau}) 
&=& \frac{1}{2}[{\rm Tr}({\bm \tau}) ^2 - {\rm Tr}({\bm \tau}^2)] \nonumber\\
&=& \frac{1}{2}[ (\tau_{xx}+\tau_{yy})^2 - (\tau_{xx}^2+2\tau_{xy}^2+\tau_{yy}^2)] \nonumber\\
&=& \frac{1}{2}[ \tau_{xx}^2+2\tau_{xx}\tau_{yy} +\tau_{yy}^2 - \tau_{xx}^2-2\tau_{xy}^2-\tau_{yy}^2] \nonumber\\
&=& \frac{1}{2}[ 2\tau_{xx}\tau_{yy} -2\tau_{xy}^2] \nonumber\\
&=& \tau_{xx}\tau_{yy} -\tau_{xy}^2 \nonumber\\
&=& \left(\sigma_{xx}-\frac{\sigma_{xx}+\sigma_{yy}}{2}\right)
\left(\sigma_{yy}-\frac{\sigma_{xx}+\sigma_{yy}}{2}\right)
-\tau_{xy}^2 \nonumber\\
&=& \left(\sigma_{xx}-\frac{\sigma_{xx}+\sigma_{yy}}{2}\right)
\left(\sigma_{yy}-\frac{\sigma_{xx}+\sigma_{yy}}{2}\right)
-\tau_{xy}^2 \nonumber\\
&=& \left(\frac{\sigma_{xx}-\sigma_{yy}}{2}\right)
\left(\frac{-\sigma_{xx}+\sigma_{yy}}{2}\right)
-\tau_{xy}^2 \nonumber\\
&=& -\left(\frac{\sigma_{xx}-\sigma_{yy}}{2}\right)^2
-\tau_{xy}^2 \nonumber
\end{eqnarray}
Looking at Eq.~\eqref{eq:max_shear_stress_2D}, we can then write
\begin{mdframed}[backgroundcolor=blue!5]
\[
\tau_{max}= \sqrt{-{\KKK}_2(\bm\tau)}
=\sqrt{ \left(\frac{\sigma_{xx}-\sigma_{yy}}{2}\right)^2 +\sigma_{xy}^2 }
\]
\end{mdframed}


%%%%%%%%%%%%%%%%%%%%%%%%%%%%%%%%%%%%%%%%%%%%%%%%%%%%%%%%%%%%%%%%%%%%%%%%%%%%%%%
\section{Tensor (moment) invariants}\label{sec:invariants}
\begin{flushright} {\tiny {\color{gray} chapter\_tensors.tex}} \end{flushright}

\index{general}{Tensor Invariant}
\index{general}{Moment Invariant}

There are many different notations used in the literature for invariants 
and these can prove to be 
confusing\footnote{No kidding, true story.}. Note that we only consider symmetric tensors in what follows.
Given a tensor $\bm{T}$,  one can compute its (moment) invariants as follows 
(see \cite[p.339]{reddybook2}, or Appendix A.2 of \cite{zita2})

\begin{eqnarray}
{\III}_1({\bm T}) 
&=& {\rm tr}[\bm{T}] \\
&=& T_{xx} + T_{yy} + T_{zz} \\ 
{\III}_2({\bm T}) 
&=& \frac{1}{2} {\rm tr}[{\bm T}\cdot{\bm T}] \\
&=& \frac{1}{2} \sum_{ij} T_{ij} T_{ji} \\
&=& \frac{1}{2} (T_{xx}^2 + T_{yy}^2 + T_{zz}^2) + T_{xy}^2 + T_{xz}^2 + T_{yz}^2 \\
{\III}_3({\bm T}) 
&=& \frac{1}{3} {\rm tr}[{\bm T}\cdot{\bm T}\cdot {\bm T}]   \\
&=& \frac{1}{3}\sum_i\sum_j \sum_k T_{ij} T_{jk} T_{ki}  
%&=& \frac{1}{3} (T_{xx} ( T_{xx}T_{xx} + T_{xy}T_{xy} + T_{xz}T_{xz} )) \qquad (i=j=x,k=x,y,z)\nn\\ 
%&&+ \frac{1}{3} (T_{yy} ( T_{yx}T_{yx} + T_{yy}T_{yy} + T_{yz}T_{yz} )) \qquad (i=j=y,k=x,y,z)\nn\\ 
%&&+ \frac{1}{3} (T_{zz} ( T_{zx}T_{zx} + T_{zy}T_{zy} + T_{zz}T_{zz} )) \qquad (i=j=z,k=x,y,z)\nn\\ 
%&&+\frac{2}{3} (T_{xy} ( T_{xx}T_{yx} + T_{xy}T_{yy} + T_{xz}T_{yz} )) \qquad (i=x,j=y,k=x,y,z)\nn\\ 
%&&+ \frac{2}{3} (T_{xz} ( T_{xx}T_{zx} + T_{xy}T_{zy} + T_{xz}T_{zz} )) \qquad (i=x,j=z,k=x,y,z)\nn\\ 
%&&+ \frac{2}{3} (T_{yz} ( T_{yx}T_{zx} + T_{yy}T_{zy} + T_{yz}T_{zz} )) \qquad (i=y,j=z,k=x,y,z)\nn\\ 
\end{eqnarray}


\[
\begin{array}{ccccc}
i & j & k & T_{ij}T_{jk}T_{ki} & symm \\
\hline
x&x&x&  T_{xx}T_{xx}T_{xx}  & T_{xx}^3  \\
y&x&x&  T_{yx}T_{xx}T_{xy}  & T_{xx}T_{xy}^2 \\
z&x&x&  T_{zx}T_{xx}T_{xz}  & T_{xx}T_{xz}^2 \\
x&y&x&  T_{xy}T_{yx}T_{xx}  & T_{xx}T_{xy}^2 \\
y&y&x&  T_{yy}T_{yx}T_{xy}  & T_{yy}T_{xy}^2 \\
z&y&x&  T_{zy}T_{yx}T_{xz}  & T_{xy}T_{xz}T_{yz} \\
x&z&x&  T_{xz}T_{zx}T_{xx}  & T_{xx}T_{xz}^2 \\
y&z&x&  T_{yz}T_{zx}T_{xy}  & T_{xy}T_{xz}T_{yz} \\
z&z&x&  T_{zz}T_{zx}T_{xz}  & T_{zz}T_{xz}^2 \\
\hline
x&x&y&  T_{xx}T_{xy}T_{yx}  & T_{xx}T_{xy}^2 \\
y&x&y&  T_{yx}T_{xy}T_{yy}  & T_{yy}T_{xy}^2 \\
z&x&y&  T_{zx}T_{xy}T_{yz}  & T_{xy}T_{xz}T_{yz} \\
x&y&y&  T_{xy}T_{yy}T_{yx}  & T_{yy}T_{xy}^2 \\
y&y&y&  T_{yy}T_{yy}T_{yy}  & T_{yy}^3  \\
z&y&y&  T_{zy}T_{yy}T_{yz}  & T_{yy}T_{yz}^2 \\
x&z&y&  T_{xz}T_{zy}T_{yx}  & T_{xy}T_{xz}T_{yz} \\
y&z&y&  T_{yz}T_{zy}T_{yy}  & T_{yy}T_{yz}^2 \\
z&z&y&  T_{zz}T_{zy}T_{yz}  & T_{zz}T_{yz}^2 \\
\hline
x&x&z&  T_{xx}T_{xz}T_{zx}  & T_{xx}T_{xz}^2 \\
y&x&z&  T_{yx}T_{xz}T_{zy}  & T_{xy}T_{xz}T_{yz}\\
z&x&z&  T_{zx}T_{xz}T_{zz}  & T_{zz}T_{xz}^2\\
x&y&z&  T_{xy}T_{yz}T_{zx}  & T_{xy}T_{yz}T_{yz}\\
y&y&z&  T_{yy}T_{yz}T_{zy}  & T_{yy}T_{yz}^2\\
z&y&z&  T_{zy}T_{yz}T_{zz}  & T_{zz}T_{yz}^2\\
x&z&z&  T_{xz}T_{zz}T_{zx}  & T_{zz}T_{xz}^2\\
y&z&z&  T_{yz}T_{zz}T_{zy}  & T_{zz}T_{yz}^2\\
z&z&z&  T_{zz}T_{zz}T_{zz}  & T_{zz}^3  \\
\hline
\end{array}
\]
In the end 
\[
\sum_{i=x,y,z} \sum_{j=x,y,z} \sum_{k=x,y,z}
T_{ij}T_{jk}T_{ki}
= T_{xx}( T_{xx}^2 + 3T_{xy}^2 + 3T_{xz}^2)
+ T_{yy}(3T_{xy}^2 +  T_{yy}^2 + 3T_{yz}^2   )
+ T_{zz}(3T_{xz}^2 + 3T_{yz}^2 + T_{zz}^2  )
+6T_{xy}T_{yz}T_{yz}
\]
and then the third moment invariant of the symmetric tensor ${\bm T}$
is given by:
\begin{eqnarray}
{\III}_3({\bm T}) 
&=& \frac{1}{3} T_{xx} (  T_{xx}^2 + 3 T_{xy}^2 + 3 T_{xz}^2  )     \nonumber\\
&+& \frac{1}{3} T_{yy} (3 T_{xy}^2 +   T_{yy}^2 + 3 T_{yz}^2  )     \nonumber\\
&+& \frac{1}{3} T_{zz} (3 T_{xz}^2 + 3 T_{yz}^2 +   T_{zz}^2)       \nonumber\\
&+& 2 T_{xy} T_{xz} T_{yz} \nn\\
&=& \frac{1}{3} (T_{xx}^3+T_{yy}^3+T_{zz}^3) 
+T_{xx} ( T_{xy}^2 +  T_{xz}^2  ) 
+T_{yy} ( T_{xy}^2 +  T_{yz}^2  ) 
+T_{zz} ( T_{xz}^2 +  T_{yz}^2  ) + 2 T_{xy} T_{xz} T_{yz}  \nn
\end{eqnarray}

\newpage
%%%%%%%%%%%%%%%%%%%%%%%%%%%%%%%%%%%%%%%%%%%%%%%%%%%%%%%%%%%%%%%%%%%%%%%%%%%%%%%
\section{Stress \& strain rate (moment) invariants}\label{sec:stress_invariants}
\begin{flushright} {\tiny {\color{gray} stress\_sr\_invariants.tex}} \end{flushright}
%~~~~~~~~~~~~~~~~~~~~~~~~~~~~~~~~~~~~~~~~~~~~~~~~~~~~~~~~~~~~~~~~~~~~~~~~~~~~~~~~~~~~~~~~~~~~~~~~~~

The implementation of the plasticity criterions relies essentially 
on the (moment) invariants $\III_{1,2,3}$ of the (deviatoric) stress ${\bm \tau}$ 
and the (deviatoric) strainrate tensors $\dot{\bm \varepsilon}$:

\begin{eqnarray}
{\III}_1({\bm \sigma}) &=& \sigma_{xx}+\sigma_{yy}+\sigma_{zz}\\
{\III}_2({\bm \tau})   
&=&\frac{1}{2}(\tau_{xx}^2 + \tau_{yy}^2 + \tau_{zz}^2 ) + \tau_{xy}^2 + \tau_{xz}^2 + \tau_{yz}^2  \\
{\III}_3({\bm \tau}) 
&=& \frac{1}{3} \tau_{xx} (  \tau_{xx}^2 + 3 \tau_{xy}^2 + 3 \tau_{xz}^2  )     \nonumber\\
&+& \frac{1}{3} \tau_{yy} (3 \tau_{xy}^2 +   \tau_{yy}^2 + 3 \tau_{yz}^2  )     \nonumber\\
&+& \frac{1}{3} \tau_{zz} (3 \tau_{xz}^2 + 3 \tau_{yz}^2 +   \tau_{zz}^2)       \nonumber\\
&+& 2 \tau_{xy} \tau_{xz} \tau_{yz}  
\end{eqnarray}
and also the second invariant of the deviatoric strain rate is:
\begin{eqnarray}
{\III}_2(\dot{\bm{\varepsilon}}^d)
&=& \frac{1}{2} \left[ (\dot{\varepsilon}_{xx}^d)^2 + (\dot{\varepsilon}_{yy}^d)^2 + (\dot{\varepsilon}_{zz}^d)^2   \right] 
+ (\dot{\varepsilon}_{xy}^d)^2  
+ (\dot{\varepsilon}_{xz}^d)^2  
+ (\dot{\varepsilon}_{yz}^d)^2  \nonumber\\
&=& \frac{1}{6} \left[ (\dot{\varepsilon}_{xx}-\dot{\varepsilon}_{yy})^2 
+ (\dot{\varepsilon}_{yy}-\dot{\varepsilon}_{zz})^2 
+ (\dot{\varepsilon}_{xx}-\dot{\varepsilon}_{zz})^2 \right] 
+ \dot{\varepsilon}_{xy}^2 + \dot{\varepsilon}_{xz}^2 + \dot{\varepsilon}_{yz}^2 \label{eq:I2epsd} 
\end{eqnarray}
Proofs of these relationships are given in Appendix~\ref{app:invariants}.

We have 
\begin{eqnarray}
\tau_{xx}^2 + \tau_{yy}^2 + \tau_{zz}^2
&=& 
\left(\sigma_{xx}-\frac13 I_1\right)^2 + 
\left(\sigma_{yy}-\frac13 I_1\right)^2 + 
\left(\sigma_{zz}-\frac13 I_1\right)^2  \nonumber\\
&=&
\sigma_{xx}^2 + \sigma_{yy}^2 + \sigma_{zz}^2 
-\frac23 I_1 (\sigma_{xx} + \sigma_{yy} + \sigma_{zz}) 
+3\frac19 I_1^2 \nonumber\\
&=&
\sigma_{xx}^2 + \sigma_{yy}^2 + \sigma_{zz}^2 
-\frac23 I_1^2 +\frac13 I_1^2 \nonumber\\
&=&
\sigma_{xx}^2 + \sigma_{yy}^2 + \sigma_{zz}^2 
-\frac13 I_1^2  \nonumber\\
&=&
\sigma_{xx}^2 + \sigma_{yy}^2 + \sigma_{zz}^2 
-\frac13 (\sigma_{xx} + \sigma_{yy} + \sigma_{zz})^2 \nonumber\\
&=&
\sigma_{xx}^2 + \sigma_{yy}^2 + \sigma_{zz}^2 
-\frac13 (\sigma_{xx}^2 + \sigma_{yy}^2 + \sigma_{zz}^2
+2\sigma_{xx}\sigma_{yy}+2\sigma_{xx}\sigma_{zz}+2\sigma_{yy}\sigma_{zz} ) 
\nonumber\\
&=& \frac13 (
3\sigma_{xx}^2 + 3\sigma_{yy}^2 + 3\sigma_{zz}^2 
-\sigma_{xx}^2 - \sigma_{yy}^2 - \sigma_{zz}^2
-2\sigma_{xx}\sigma_{yy}-2\sigma_{xx}\sigma_{zz}-2\sigma_{yy}\sigma_{zz} ) 
\nonumber\\
&=& \frac13 (
2\sigma_{xx}^2 + 2\sigma_{yy}^2 + 2\sigma_{zz}^2 
-2 \sigma_{xx}\sigma_{yy}-2 \sigma_{xx}\sigma_{zz}-2 \sigma_{yy}\sigma_{zz} )\\
&=& \frac13 ((\sigma_{xx}-\sigma_{yy})^2 + (\sigma_{xx}-\sigma_{zz})^2
+ (\sigma_{yy}-\sigma_{zz})^2)
\end{eqnarray}
so that 
\[
{\III}_2({\bm \tau})   
=\frac{1}{6}\left[(\sigma_{xx}-\sigma_{yy})^2 + (\sigma_{yy}-\sigma_{zz})^2 + (\sigma_{xx}-\sigma_{zz})^2 \right]  
+ \sigma_{xy}^2 + \sigma_{xz}^2 + \sigma_{yz}^2 
\]

\begin{remark}
${\III}_2({\bm \tau})$ is often called $J_2$ or $J_2'$ so that one sometimes speaks of $J_2$-plasticity.
\end{remark}

These (second) invariants are almost always used under a square root so we define:
\begin{mdframed}[backgroundcolor=blue!5]
\begin{equation}
\tau_{e}=\sqrt{{\III}_2({\bm \tau})}
\quad\quad
\quad\quad
\dot{\varepsilon}_{e}=\sqrt{{\III}_2(\dot{\bm \varepsilon}^d)}
\label{eq:tauepse}
\end{equation}
\end{mdframed}
Note that these quantities have the same dimensions as their tensor counterparts, i.e. $\si{\pascal}$ 
for stresses and $\si{\per\second}$ for strain rates.

If the stress tensor is such that it is diagonal, i.e.
\[
{\bm \sigma}= \left( \begin{array}{ccc}
\sigma_1 & 0 & 0 \\
0 & \sigma_2 & 0 \\
0 & 0 & \sigma_3
\end{array}\right)
\qquad
{\rm and}
\qquad
{\bm \tau}= \left( \begin{array}{ccc}
\tau_1 & 0 & 0 \\
0 & \tau_2 & 0 \\
0 & 0 & \tau_3
\end{array}\right)
\]
then the invariants are 
\begin{eqnarray}
{\III}_1({\bm \sigma}) &=& \sigma_1 + \sigma_2+ \sigma_3 \nonumber\\
{\III}_2({\bm \tau}) &=& \frac{1}{6}\left[(\sigma_{1}-\sigma_{2})^2 + (\sigma_{2}-\sigma_{3})^2 
+ (\sigma_{1}-\sigma_{3})^2 \right] \label{eq:I2s123}\\ 
{\III}_3({\bm \tau}) 
&=& \tau_1\tau_2\tau_3 \nn\\
&=& \frac{1}{3} {\rm tr}[{\bm \tau}\cdot{\bm \tau}\cdot {\bm \tau}]  \nn\\
&=& \frac{1}{3} {\rm tr}
\left[
\left(
\begin{array}{ccc}
\tau_1 & 0 & 0 \\
0 & \tau_2 & 0 \\
0 & 0 & \tau_3 
\end{array}
\right)
\cdot
\left(
\begin{array}{ccc}
\tau_1 & 0 & 0 \\
0 & \tau_2 & 0 \\
0 & 0 & \tau_3 
\end{array}
\right)
\cdot
\left(
\begin{array}{ccc}
\tau_1 & 0 & 0 \\
0 & \tau_2 & 0 \\
0 & 0 & \tau_3 
\end{array}
\right)
\right] \nn\\
&=&  \frac{1}{3} {\rm tr}
\left(
\begin{array}{ccc}
\tau_1^3 & 0 & 0 \\
0 & \tau_2^3 & 0 \\
0 & 0 & \tau_3^3 
\end{array}
\right) \nn\\
&=& \frac{1}{3}(\tau_1^3+\tau_2^3+\tau_3^3) \nn\\
&=&  \frac{1}{3} [ 
(\sigma_1-{\III}_1({\bm \sigma})/3)^3+  
(\sigma_2-{\III}_1({\bm \sigma})/3)^3+
(\sigma_3-{\III}_1({\bm \sigma})/3)^3 ]   \nonumber\\ 
&=&  \frac{1}{3\cdot 27} [ 
(3\sigma_1-{\III}_1({\bm \sigma}))^3+  
(3\sigma_2-{\III}_1({\bm \sigma}))^3+
(3\sigma_3-{\III}_1({\bm \sigma}))^3 ]   \nonumber\\ 
&=& \frac{1}{81}
\left[
(2\sigma_1-\sigma_2-\sigma_3)^3+
(2\sigma_2-\sigma_1-\sigma_3)^3+
(2\sigma_3-\sigma_1-\sigma_2)^3
\right] 
\label{eq:3rdinvb} \label{eq:I3tau}
\end{eqnarray}
The formulation of the third invariant of ${\bm \tau}$  in Eq.~\ref{eq:I3tau} 
is used in Wojciechowski \cite{wojc18}.













%-----------------------------------------------------------
\index{general}{Plain Strain}
\section{Two-dimensional plane strain calculations \label{ss:plane_strain}} 
\begin{flushright} {\tiny {\color{gray} \tt plane\_strain.tex}} \end{flushright}
%~~~~~~~~~~~~~~~~~~~~~~~~~~~~~~~~~~~~~~~~~~~~~~~~~~~~~~~~~~~~~~~~~~~~~~~~~~~~~~~~~~~~~~~~~~~~~~~~~~

We start from the 3D strain rate tensor 
\[
\dot{\bm \varepsilon}(\vec\upnu) = 
\left(
\begin{array}{ccc}
\dot{\varepsilon}_{xx} & \dot{\varepsilon}_{xy} & \dot{\varepsilon}_{xz} \\
\dot{\varepsilon}_{yx} & \dot{\varepsilon}_{yy} & \dot{\varepsilon}_{yz} \\
\dot{\varepsilon}_{zx} & \dot{\varepsilon}_{zy} & \dot{\varepsilon}_{zz} 
\end{array}
\right)
\]

The plane strain assumption is such that the problem at hand is assumed to be 
infinite in a given direction. In the case of computational geodynamics, most 2D 
modelling is a vertical section of the crust-lithosphere-mantle
and the underlying implicit assumption is then that the orogen/rift/subduction/etc ... 
is infinite in the direction perpendicular to the screen/paper.  

Let us assume that the deformation takes place in the $x,y$-plane,
so that $w=0$ (velocity in the $z$ direction is zero) and $\partial_z \rightarrow 0$ 
(no change in the $z$ direction).
We then have $\dot{\varepsilon}_{zz}=0$ as well as $\dot{\varepsilon}_{xz}=0$ 
and $\dot{\varepsilon}_{yz}=0$, so that the strain rate tensor is 
\[
\dot{\bm \varepsilon}(\vec\upnu)=
\left( \begin{array}{ccc}
\dot{\varepsilon}_{xx} & \dot{\varepsilon}_{xy} & 0 \\
\dot{\varepsilon}_{yx} & \dot{\varepsilon}_{yy} & 0 \\
0 & 0 & 0
\end{array}\right)
\]

%------------------------------------
\subsubsection{Incompressible flow}

If the flow is incompressible then the deviatoric stress tensor is given by
\[
\bm\tau 
= 2 \eta \dot{\bm \varepsilon}^d(\vec\upnu)
= 2 \eta \left(\dot{\bm \varepsilon}(\vec\upnu) 
-\frac13 \underbrace{{\rm tr}[\dot{\bm \varepsilon}]}_{=0} 
{\bm 1}\right)
= 2 \eta \dot{\bm \varepsilon}(\vec\upnu) 
=
\left(\begin{array}{ccc}
\tau_{xx} & \tau_{xy} & 0 \\
\tau_{yx} & \tau_{yy} & 0 \\
0 & 0 & 0
\end{array}\right)
\]
One then discards the unnecessary line and column in the tensor, leaving a $2\times 2$ matrix.
Finding the principal stress components is then trivial since we have done it in 2D already.

It is important to keep in mind that the invariants we need to implement 
the rheologies are ${\III}_1({\bm \sigma})$,  ${\III}_2({\bm \tau})$ and ${\III}_3({\bm \tau})$.
By formulating our yield surfaces with pressure $p=-{\III}_1({\bm \sigma})/3$ we can then 
avoid confusion, and since the other two invariants are functions of ${\bm \tau}$ the pressure 
term does not pose any problem: simply set $\tau_{xz}$, $\tau_{yz}$ and $\tau_{zz}$ to zero in the 
equations of Section~\ref{sec:stress_invariants} and we obtain:
\begin{eqnarray}
{\III}_2({\bm \tau}) &=&\frac{1}{2}(\tau_{xx}^2 + \tau_{yy}^2 ) + \tau_{xy}^2 \\ 
{\III}_3({\bm \tau}) 
&=& \frac{1}{3} \tau_{xx} (  \tau_{xx}^2 + 3 \tau_{xy}^2 ) 
+ \frac{1}{3} \tau_{yy} (3 \tau_{xy}^2 +   \tau_{yy}^2 )   \nn\\
&=& \frac{1}{3}(  \tau_{xx}^3 + 3 \tau_{xx}\tau_{xy}^2  
+ 3 \tau_{yy} \tau_{xy}^2 +   \tau_{yy}^3 )   \nn\\
&=& \frac{1}{3}(  \tau_{xx}^3 + 3 (\tau_{xx}+\tau_{yy}) \tau_{xy}^2  +  \tau_{yy}^3 )   \nn\\
&=& \frac{1}{3}(  \tau_{xx}^3 +  \tau_{yy}^3 )  \qquad \text{since } \tau_{ii}=0 
\end{eqnarray}



The principal stresses of the deviatoric stress tensor $\bm\tau$ are given by
\begin{eqnarray}
\tau_1 &=& \frac{ \tau_{xx}+\tau_{yy}}{2} 
+ \sqrt{ \left(\frac{\tau_{xx}-\tau_{yy}}{2}\right)^2 +\tau_{xy}^2 } \nn\\
\tau_2 &=& \frac{ \tau_{xx}+\tau_{yy}}{2} 
- \sqrt{ \left(\frac{\tau_{xx}-\tau_{yy}}{2}\right)^2 +\tau_{xy}^2 } 
\end{eqnarray}
The full stress tensor is then
\[
\bm\sigma = -p \bm 1 + \bm\tau
= \left(\begin{array}{ccc}
-p+\tau_{xx} & \tau_{xy} & 0 \\
\tau_{yx} & -p+\tau_{yy} & 0 \\
0 & 0 & -p
\end{array}\right)
\]
so it remains a $3\times 3$ tensor!

However, looking at the conservation of momentum, 
\[
\vec\nabla \cdot \bm\sigma + \rho \vec g = \vec 0
\]
Given the conditions for plane-strain then $\vec g$ is likely to be in 
the $xy$-plane so that the $z$ component of the equation becomes:
\[
-\partial_z p = 0
\]
and since we have $\partial_z \rightarrow 0$ anyways this equation 
is automatically fulfilled. Then, we might as well proceed 
by considering that the stress tensor is in fact 2D as the third row/column
has no incidence. In that case the pressure is given by $p=-{\III}_1(\bm\sigma)/2$.
In the plasticity yield criterion or plastic potential we will 
need the full stress ${\bm \sigma}$ only via its first invariant (i.e. the pressure). 
The other two invariants are those of the deviatoric stress. 

Let us start from the deviatoric stress tensor:
\[
\bm\tau
=
\bm\sigma - \frac12 {\III}_1(\bm\sigma)
=
\left(\begin{array}{cc}
\sigma_{xx} & \sigma_{xy} \\ 
\sigma_{xy} & \sigma_{yy} 
\end{array}\right)
-\frac{\sigma_{xx}+\sigma_{yy}}{2} 
\left(\begin{array}{cc}
1 & 0 \\ 0 & 1
\end{array}\right)
=
\left(
\begin{array}{cc}
(\sigma_{xx}-\sigma_{yy})/2 & \sigma_{xy} \\
\sigma_{xy} & -(\sigma_{xx}-\sigma_{yy})/2
\end{array}
\right)
\]
The second invariant of the deviatoric stress tensor is then 
\[
{\III}_2(\bm\tau) = \frac12 \bm\tau:\bm\tau
= \frac12 \left( 2 (\sigma_{xx}-\sigma_{yy})^2/4 + 2 \sigma_{xy}^2 \right)
%= \left(\frac{\sigma_{xx}-\sigma_{yy}}{2} \right)^2 + \sigma_{xy}^2
\]
or
\begin{mdframed}[backgroundcolor=blue!5]
\[
{\III}_2(\bm\tau) 
= \left(\frac{\sigma_{xx}-\sigma_{yy}}{2} \right)^2 + \sigma_{xy}^2
\]
\end{mdframed}


and the effective deviatoric stress
\[
\tau_e = 
\sqrt{\left(\frac{\sigma_{xx}-\sigma_{yy}}{2} \right)^2 + \sigma_{xy}^2}
\]


Remark: Using the form of ${\III}_2(\bm\tau)$ above one arrives at  
\begin{eqnarray}
\frac{\partial {\III}_2(\bm\tau)}{\partial \sigma_{xx}} 
&=&  2 \frac{1}{2}  \frac{\sigma_{xx}-\sigma_{yy}}{2} = \tau_{xx}\nn \\
\frac{\partial {\III}_2(\bm\tau)}{\partial \sigma_{yy}} 
&=& -2 \frac{1}{2}  \frac{\sigma_{xx}-\sigma_{yy}}{2} = \tau_{yy} \nn\\
\frac{\partial {\III}_2(\bm\tau)}{\partial \sigma_{xy}} 
&=& 2 \sigma_{xy} =  2 \tau_{xy}\nn
\end{eqnarray}
which is ...wrong! One should first write the second invariant 
for the generic case of the deviatoric stress tensor (without 
assuming it is symmetric):
\[
{\III}_2(\bm\tau) = \frac12 \bm\tau:\bm\tau
= \frac12 \left( 2 (\sigma_{xx}-\sigma_{yy})^2/4 + \sigma_{xy}^2 + \sigma_{yx}^2 \right)
= \left(\frac{\sigma_{xx}-\sigma_{yy}}{2} \right)^2 + \frac12\sigma_{xy}^2 + \frac12\sigma_{yx}^2
\]
Then
\begin{eqnarray}
\frac{\partial {\III}_2(\bm\tau)}{\partial \sigma_{xx}} 
&=&  2 \frac{1}{2}  \frac{\sigma_{xx}-\sigma_{yy}}{2} = \tau_{xx} \nn\\
\frac{\partial {\III}_2(\bm\tau)}{\partial \sigma_{yy}} 
&=& -2 \frac{1}{2}  \frac{\sigma_{xx}-\sigma_{yy}}{2} =  \tau_{yy} \nn\\
\frac{\partial {\III}_2(\bm\tau)}{\partial \sigma_{xy}} 
&=&  \sigma_{xy} =  \tau_{xy} \nn\\
\frac{\partial {\III}_2(\bm\tau)}{\partial \sigma_{yx}} 
&=&  \sigma_{yx} =  \tau_{yx} \nn
\end{eqnarray}
which can be simply written as
\begin{mdframed}[backgroundcolor=blue!5]
\[
\frac{\partial {\III}_2(\bm\tau)}{\partial \bm\sigma} 
=\bm\tau
\]
\end{mdframed}



















%------------------------------------
\subsubsection{Compressible flow}
If the flow is not incompressible, then the deviatoric strain rate tensor is
\[
\dot{\bm \varepsilon}^d(\vec\upnu) 
= \dot{\bm \varepsilon}(\vec\upnu) -\frac{1}{3} {\rm tr}[\dot{\bm \varepsilon}]   {\bm 1} 
= \dot{\bm \varepsilon}(\vec\upnu) -\frac{1}{3} (\dot{\varepsilon}_{xx} +\dot{\varepsilon}_{yy}   )  {\bm 1} 
=
\left(
\begin{array}{ccc}
\frac{2}{3}\dot{\varepsilon}_{xx} -\frac{1}{3}\dot{\varepsilon}_{yy} & \dot{\varepsilon}_{xy} & 0 \\
\dot{\varepsilon}_{yx} & -\frac{1}{3}\dot{\varepsilon}_{xx} +\frac{2}{3} \dot{\varepsilon}_{yy} & 0 \\
0 & 0 & -\frac{1}{3} \dot{\varepsilon}_{xx} -\frac{1}{3}\dot{\varepsilon}_{yy}
\end{array}
\right)
\]
The deviatoric stress tensor now has the form
\[
\bm\tau=
\left(\begin{array}{ccc}
\tau_{xx} & \tau_{xy} & 0 \\
\tau_{yx} & \tau_{yy} & 0 \\
0 & 0 & \tau_{zz}
\end{array}\right)
\]

We are interested in the principal components
of the deviatoric stress tensor $\bm \tau$ so that we now have the following determinant to compute:
\[
\left|  
\begin{array}{ccc}
\tau_{xx}-\lambda & \tau_{xy} & 0 \\
\tau_{xy} & \tau_{yy}-\lambda & 0 \\
0 &0 & \tau_{zz} -\lambda
\end{array}
\right|
=0
\]
which yields the following characteristic equation:
\[
(\tau_{zz} -\lambda)(\lambda-\tau_1)(\lambda-\tau_2) =0
\]
where $\tau_{1,2}$ have previously been obtained in the 2D case:
\begin{eqnarray}
\tau_1 
&=& \frac{ \tau_{xx}+\tau_{yy}}{2} 
+ \sqrt{ \left(\frac{\tau_{xx}-\tau_{yy}}{2}\right)^2 +\tau_{xy}^2 } 
\nn\\
\tau_2 &=& \frac{ \tau_{xx}+\tau_{yy}}{2} 
- \sqrt{ \left(\frac{\tau_{xx}-\tau_{yy}}{2}\right)^2 +\tau_{xy}^2 } 
\end{eqnarray}
We have 
\begin{eqnarray}
\tau_{xx}+\tau_{yy} &=& 2\eta \frac13 (\dot{\varepsilon}_{xx} + \dot{\varepsilon}_{yy}) \nn\\
\tau_{xx}-\tau_{yy} &=& 2\eta (\dot{\varepsilon}_{xx} - \dot{\varepsilon}_{yy})
\end{eqnarray}
Then 
\begin{eqnarray}
\tau_1 
&=& \frac{ \tau_{xx}+\tau_{yy}}{2} 
+ \sqrt{ \left(\frac{\tau_{xx}-\tau_{yy}}{2}\right)^2 +\tau_{xy}^2 } 
\nn\\
&=& \eta \frac13 (\dot{\varepsilon}_{xx} + \dot{\varepsilon}_{yy})
+ \eta \sqrt{ (\dot{\varepsilon}_{xx} - \dot{\varepsilon}_{yy})^2 +4  \dot{\varepsilon}_{xy}^2  } 
\nn\\
\tau_2 
&=& \frac{ \tau_{xx}+\tau_{yy}}{2} 
- \sqrt{ \left(\frac{\tau_{xx}-\tau_{yy}}{2}\right)^2 +\tau_{xy}^2 } \nn\\
&=& \eta \frac13 (\dot{\varepsilon}_{xx} + \dot{\varepsilon}_{yy})
- \eta \sqrt{ (\dot{\varepsilon}_{xx} - \dot{\varepsilon}_{yy})^2 +4  \dot{\varepsilon}_{xy}^2  } 
\end{eqnarray}
It does not look like it is going to simplify down the road ... Also, 
the third eigenvalue/principal stress remains and it is not clear whether it is 
larger or smaller than the other two.
The 3D framework is then probably the most appropriate.











Let us now turn to the second invariant of the deviatoric strain rate 
(see Eq.~\eqref{eq:I2epsd}):
\begin{eqnarray}
{\III}_2(\dot{\bm{\varepsilon}}^d)
&=& \frac{1}{2} \dot{\bm{\varepsilon}}^d:\dot{\bm{\varepsilon}}^d \\
&=& \frac{1}{2} \left[ (\dot{\varepsilon}_{xx}^d)^2 + (\dot{\varepsilon}_{yy}^d)^2 + (\dot{\varepsilon}_{zz}^d)^2   \right] 
+ (\dot{\varepsilon}_{xy}^d)^2  
+ (\dot{\varepsilon}_{xz}^d)^2  
+ (\dot{\varepsilon}_{yz}^d)^2  
\end{eqnarray}
But there is also an expression for ${\III}_2(\dot{\bm{\varepsilon}}^d)$ directly as a function of the $\dot{\bm\varepsilon}_{ij}$ components 
(see Eq.~\eqref{eq:I2epsd}):
\begin{eqnarray}
{\III}_2(\dot{\bm{\varepsilon}}^d)
&=& \frac{1}{6} \left[ (\dot{\varepsilon}_{xx}-\dot{\varepsilon}_{yy})^2 
+ (\dot{\varepsilon}_{yy}-\dot{\varepsilon}_{zz})^2 
+ (\dot{\varepsilon}_{xx}-\dot{\varepsilon}_{zz})^2 \right] 
+ \dot{\varepsilon}_{xy}^2 + \dot{\varepsilon}_{xz}^2 + \dot{\varepsilon}_{yz}^2 \\
&=& 
\frac{1}{6} \left[ (\dot{\varepsilon}_{xx}-\dot{\varepsilon}_{yy})^2 
+ (\dot{\varepsilon}_{yy})^2 
+ (\dot{\varepsilon}_{xx})^2 \right] 
+ \dot{\varepsilon}_{xy}^2 \\
&=& \frac{1}{6} \left[ \dot{\varepsilon}_{xx}^2 
-2\dot{\varepsilon}_{xx}\dot{\varepsilon}_{yy}
+\dot{\varepsilon}_{yy}^2 
+ \dot{\varepsilon}_{yy}^2 
+ \dot{\varepsilon}_{xx}^2 \right] 
+ \dot{\varepsilon}_{xy}^2 \\
&=& \frac{1}{6} \left[ 
2\dot{\varepsilon}_{xx}^2 
-2\dot{\varepsilon}_{xx}\dot{\varepsilon}_{yy}
+2\dot{\varepsilon}_{yy}^2 
\right] 
+ \dot{\varepsilon}_{xy}^2 \\
&=& 
\frac{1}{3} \left[ 
\dot{\varepsilon}_{xx}^2 
-\dot{\varepsilon}_{xx}\dot{\varepsilon}_{yy}
+\dot{\varepsilon}_{yy}^2 
\right] 
+ \dot{\varepsilon}_{xy}^2 
\end{eqnarray}

{\color{darkgray}
If we now do things the old/wrong(?) way, one would start directly from the 2D strain rate tensor 
\[
\dot{\bm \varepsilon} = 
\left(
\begin{array}{cc}
\dot{\varepsilon}_{xx} & \dot{\varepsilon}_{xy}  \\
\dot{\varepsilon}_{yx} & \dot{\varepsilon}_{yy}  
\end{array}
\right)
\]
The deviatoric strain rate tensor is then logically defined as 
\[
\dot{\bm \varepsilon}^d 
= \dot{\bm \varepsilon} -\frac{1}{2} Tr[\dot{\bm \varepsilon}]   {\bm 1} 
= \dot{\bm \varepsilon} -\frac{1}{2} (\dot{\varepsilon}_{xx} +\dot{\varepsilon}_{yy}   )  {\bm 1} 
\]
or,
\[
\dot{\bm \varepsilon}^d = 
\left(
\begin{array}{cc}
\frac{1}{2}\dot{\varepsilon}_{xx} -\frac{1}{2}\dot{\varepsilon}_{yy} & \dot{\varepsilon}_{xy}  \\
\dot{\varepsilon}_{yx} & -\frac{1}{2}\dot{\varepsilon}_{xx} +\frac{1}{2} \dot{\varepsilon}_{yy}  \\
\end{array}
\right)
\]
Let us now turn to the second invariant of the deviatoric strain rate 
(see Section 3.21 in fieldstone)
\begin{eqnarray}
{\III}_2(\dot{\bm{\varepsilon}}^d)
&=& \frac{1}{2} \dot{\bm{\varepsilon}}^d:\dot{\bm{\varepsilon}}^d \nn\\
&=& \frac{1}{2}[ (\frac{1}{2}\dot{\varepsilon}_{xx} -\frac{1}{2}\dot{\varepsilon}_{yy})^2 + (-\frac{1}{2}\dot{\varepsilon}_{xx} +\frac{1}{2} \dot{\varepsilon}_{yy})^2  ] + \dot{\varepsilon}_{xy}^2 \nn\\
&=& \frac12 [ \frac14 (2\dot{\varepsilon}_{xx}^2  -4 \dot{\varepsilon}_{xx}\dot{\varepsilon}_{yy} +2\dot{\varepsilon}_{yy}^2 )  ] + \dot{\varepsilon}_{xy}^2 \nn\\
&=& \frac14 [ \dot{\varepsilon}_{xx}^2  -2 \dot{\varepsilon}_{xx}\dot{\varepsilon}_{yy} +\dot{\varepsilon}_{yy}^2   ] + \dot{\varepsilon}_{xy}^2 
\end{eqnarray}
which is not the same as the previous expression! 
}













%%%%%%%%%%%%%%%%%%%%%%%%%%%%%%%%%%%%%%%%%%%%%%%%%%%%%%%%%%%%%%%%%%%%%%%%%%%%%%%%%%%%%%%%%%%%%%%%%%%%
\section{Alternative principal stresses notations}\label{sec:altinv}
\begin{flushright} {\tiny {\color{gray} physics.tex}} \end{flushright}

The principal stresses of the stress tensor ${\bm \sigma}$ are $\sigma_1$, $\sigma_2$
and $\sigma_3$ with $\sigma_1 \geq \sigma_2 \geq \sigma_3$.
Following Wojciechowski \cite{wojc18}, we start by stating that the intermediate principal 
stress can always be represented as a linear combination of two other stresses:
\begin{equation}
\sigma_2 = (1-b)\sigma_1 + b \sigma_3
\qquad
{\rm where}
\qquad
b = \frac{\sigma_1-\sigma_2}{\sigma_1-\sigma_3}\in [0,1]
\end{equation}
The quantity $b$ is called the principal stress ratio. \index{general}{Principal Stress Ratio}
Let us now introduce the maximum shear plane stresses $\sigma_m$ and $\tau_m$ such that
\footnote{Although most of this section is inspired by Wojciechowski \cite{wojc18}, 
I have decided not to use his notations which are very confusing since he denotes $\sigma_m$ by $p$} 
\begin{equation}
\boxed{\sigma_m=\frac{\sigma_1+\sigma_3}{2}}
\qquad
\boxed{\tau_m=\frac{\sigma_1-\sigma_3}{2}}
\end{equation}
so that we have 
\begin{eqnarray}
\sigma_1 &=& \sigma_m+\tau_m \\
\sigma_2 &=& \sigma_m-a\tau_m \\ 
\sigma_3 &=& \sigma_m-\tau_m
\end{eqnarray}
The quantity $a\in[-1,1]$ is an equivalent measure of the principal stress ratio and 
is defined as 
\begin{equation}
a=2b-1 =2 \frac{\sigma_1-\sigma_2}{\sigma_1-\sigma_3}-1=\frac{\sigma_1-2\sigma_2+\sigma_3}{\sigma_1-\sigma_3}
\end{equation}
We can introduce $a$, $\sigma_m$ and $\tau_m$ in the invariants above:
\begin{eqnarray}
{\III}_1({\bm \sigma}) 
&=& \sigma_1 + \sigma_2 + \sigma_3 \nn\\
&=& (\sigma_m+\tau_m) + (\sigma_m-a\tau_m) + (\sigma_m-\tau_m) \nn\\
&=& 3\sigma_m -a\tau_m \\
{\III}_2({\bm \tau}) 
&=&\frac{1}{6}\left[(\sigma_{1}-\sigma_{2})^2 +(\sigma_{2}-\sigma_{3})^2 +(\sigma_{1}-\sigma_{3})^2\right]\nn\\ 
&=&\frac{1}{6}\left[(\sigma_m+\tau_m-\sigma_m+a\tau_m)^2 +(\sigma_m-a\tau_m-\sigma_m+\tau_m)^2 
+(\sigma_m+\tau_m-\sigma_m+\tau_m)^2\right]\nn\\ 
&=&\frac{1}{6}\left[(\tau_m+a\tau_m)^2 +(-a\tau_m+\tau_m)^2 +(\tau_m+\tau_m)^2\right]\nn\\ 
&=&\frac{\tau_m^2}{6}\left[(1+a)^2 +(-a+1)^2 + 4 \right]\nn\\ 
&=&\frac{\tau_m^2}{6}\left[ 1+2a+a^2 +1 - 2a+a^2 + 4 \right]\nn\\ 
&=&\frac{\tau_m^2}{3}\left( a^2 +3 \right)
\end{eqnarray}
Using the definition of the third invariant of Eq.~(\ref{eq:3rdinvb}):
\begin{eqnarray}
{\III}_3({\bm \tau}) 
&=& \frac{1}{81} \left[
(2\sigma_1-\sigma_2-\sigma_3)^3+
(2\sigma_2-\sigma_1-\sigma_3)^3+
(2\sigma_3-\sigma_1-\sigma_2)^3
\right] \nn\\
&=& \frac{1}{81} \left[
(2\sigma_m+2\tau_m-\sigma_m+a\tau_m-\sigma_m+\tau_m)^3+
(2\sigma_m-2a\tau_m-\sigma_m-\tau_m-\sigma_m+\tau_m)^3+
(2\sigma_m-2\tau_m-\sigma_m-\tau_m-\sigma_m+a\tau_m)^3
\right] \nn\\
&=& \frac{1}{81} \left[ (2\tau_m+a\tau_m+\tau_m)^3+ (-2a\tau_m-\tau_m+\tau_m)^3+ (-2\tau_m-\tau_m+a\tau_m)^3 \right] \nn\\
&=& \frac{\tau_m^3}{81} \left[ (3+a)^3+ (-2a)^3+ (-3+a)^3 \right] \nn\\
&=& \frac{\tau_m^3}{81} \left[ 27 +9a + 3a^2 + a^3  -8a^3 -27 +9a -3a^2 + a^3 \right] \nn\\
&=& \frac{\tau_m^3}{81} \left( 18a  -6 a^3  \right) \nn\\
&=& \frac{2a \tau_m^3}{27} \left( 3 - a^2  \right) 
\end{eqnarray}
which is different than Eq. (14) of  Wojciechowski \cite{wojc18}!!

To recap:
\begin{eqnarray}
\boxed{{\III}_1({\bm \sigma}) =  3\sigma_m -a\tau_m } 
\qquad
\boxed{{\III}_2({\bm \tau}) =\frac{\tau_m^2}{3}\left( a^2 +3 \right)}
\qquad
\boxed{{\III}_3({\bm \tau}) = \frac{2a \tau_m^3}{27} \left( 3 - a^2  \right) }
\end{eqnarray}

\begin{remark}
Wojciechowski \cite{wojc18} defines the Lode angle \index{general}{Lode Angle} 
as being the opposite of my definition in Eq.~\ref{eq:lodang}.
\end{remark}

Finally, we can show using Eqs.~(\ref{eq:sigma1},\ref{eq:sigma2},\ref{eq:sigma3}) that
\begin{eqnarray}
a 
&=&\frac{\sigma_1-2\sigma_2+\sigma_3}{\sigma_1-\sigma_3} \nn\\
&=& 
\frac{
\sqrt{{\III}_2({\bm \tau})} \left(-\frac{1}{\sqrt{3}}\sin \uptheta +\cos\uptheta \right) 
-2
\sqrt{{\III}_2({\bm \tau})} \left(\frac{2}{\sqrt{3}}\sin \uptheta   \right)   
+
\sqrt{{\III}_2({\bm \tau})} \left(-\frac{1}{\sqrt{3}}\sin \uptheta- \cos \uptheta \right)  
}{
\sqrt{{\III}_2({\bm \tau})} \left(-\frac{1}{\sqrt{3}}\sin \uptheta +\cos\uptheta \right)
- 
\sqrt{{\III}_2({\bm \tau})} \left(-\frac{1}{\sqrt{3}}\sin \uptheta- \cos \uptheta \right)  
}
\nn \\
&=& 
\frac{
\left(-\frac{1}{\sqrt{3}}\sin \uptheta +\cos\uptheta \right) 
-2
\left(\frac{2}{\sqrt{3}}\sin \uptheta   \right)   
+
\left(-\frac{1}{\sqrt{3}}\sin \uptheta- \cos \uptheta \right)  
}{
\left(-\frac{1}{\sqrt{3}}\sin \uptheta +\cos\uptheta \right)
- 
\left(-\frac{1}{\sqrt{3}}\sin \uptheta- \cos \uptheta \right)  
}
\nn \\
&=& 
\frac{
-\frac{6}{\sqrt{3}}\sin \uptheta  
}
{
2\cos\uptheta
}
\nn\\
&=& -\frac{3}{\sqrt{3}} \frac{\sin\uptheta}{\cos\uptheta} \nn\\
&=& -\sqrt{3} \tan\uptheta
\end{eqnarray}
Here again we arrive at the opposite of Eq. (16) of Wojciechowski \cite{wojc18}. 

\newpage
%====================================================================================
%====================================================================================

\section{Recap of notations and definitions of stress invariants \label{ss:recapInv}}
\begin{flushright} {\tiny {\color{gray} recap\_invariants.tex}} \end{flushright}
%~~~~~~~~~~~~~~~~~~~~~~~~~~~~~~~~~~~~~~~~~~~~~~~~~~~~~~~~~~~~~~~~~~~~~~~~~~~~~~~~~~~~~~~~~~~~~~~~~~

When it comes to stress invariants, I urge the reader to be extremely careful when considering 
their source(s). As we have seen these come in two main flavours (principal and moment invariants)
and they are often written for the full stress or deviatoric tensor. On top of it all, 
typos are common like in any source and the occasional minus sign or factor 2 or 3 can be missing.
This is the reason why I have spent substantial time re-deriving those in the past pages 
with a consistent set of notations:
\begin{center}
\begin{tabular}{ll}
\hline
${\bm \sigma}$ & (full) stress tensor \\
$\sigma_1$, $\sigma_2$, $\sigma_3$ & principal stresses \\ 
${\bm \tau}$   & deviatoric stress tensor \\
$\tau_1$, $\tau_2$, $\tau_3$ & principal deviatoric stresses \\ 
${\cal I}_1({\bm T})$ & first moment invariant of tensor ${\bm T}$ \\
${\cal I}_2({\bm T})$ & second moment invariant of tensor ${\bm T}$ \\
${\cal I}_3({\bm T})$ & third moment invariant of tensor ${\bm T}$ \\
${\tau}_{e}=\sqrt{{\cal I}_2({\bm \tau})}$ & effective deviatoric stress \\
$\dot{{\varepsilon}}_{e}=\sqrt{{\cal I}_2(\dot{\bm \varepsilon}^d)}$ & effective deviatoric strain rate \\
\hline
\end{tabular}
\end{center}
Proofs of all the following relationships are given in Appendix~\ref{app:invariants}.

\begin{itemize}
\item First invariant %-----------------------------------

\begin{eqnarray}
{\cal I}_1(\bm\sigma) &=& \sigma_{xx}+\sigma_{yy}+\sigma_{zz} \nn\\
{\cal I}_1(\bm\tau) &=& 0 \nn\\ 
\frac{\partial {\cal I}_1(\bm\sigma)}{\partial \bm\sigma} &=& {\bm 1}  \nn
\end{eqnarray}


\item Second invariant %-----------------------------------


\begin{eqnarray}
{\cal I}_2(\bm\tau) 
&=& \frac12 {\bm \tau}:{\bm \tau} \nn\\
&=& \frac12 {\rm tr} [{\bm \tau}\cdot {\bm \tau}] \nn\\
&=& \frac12 \sum_{ij} \tau_{ij}\tau_{ji}  \nn\\
&=& \frac12 ( \tau_{xx}^2 + \tau_{yy}^2 +\tau_{zz}^2 + 2\tau_{xy}^2+ 2\tau_{xz}^2+ 2\tau_{yz}^2) \nn\\
&=& \frac{1}{6}\left[(\sigma_{xx}-\sigma_{yy})^2 + (\sigma_{yy}-\sigma_{zz})^2 
+ (\sigma_{xx}-\sigma_{zz})^2 \right]  + \sigma_{xy}^2 + \sigma_{xz}^2 + \sigma_{yz}^2 \nn\\
&=&  -\frac16 {\cal I}_1(\bm\sigma)^2 + {\cal I}_2(\bm\sigma) \nn\\
\frac{\partial {\cal I}_2(\bm\tau)}{\partial \bm\sigma} &=& \bm\tau   \nn
\end{eqnarray}

\item Third invariant %-----------------------------------

\begin{eqnarray}
{\cal I}_3(\bm\tau) 
&=& \frac13 \sum_{ijk} \tau_{ij}\tau_{jk}\tau_{ki} \nn\\
&=& {\rm det}(\bm\tau) \nn\\
&=& \frac13 {\rm tr} [{\bm \tau}\cdot {\bm \tau} \cdot {\bm \tau}] \nn\\
&=& \frac{2}{27}  {\cal I}_1(\bm\sigma)^3 - \frac23{\cal I}_1(\bm\sigma) {\cal I}_2(\bm\sigma)
+{\cal I}_3(\bm\sigma) \nn\\
\frac{\partial {\cal I}_3(\bm\tau)}{\partial \bm\sigma} &=&
\end{eqnarray}

\begin{equation}
\theta_{\rm L}=\frac{1}{3} \sin^{-1} 
\left( -\frac{3\sqrt{3}}{2} \frac{{\cal I}_3({\bm \tau})}{{\cal I}_2({\bm \tau})^{3/2}} \right)
\end{equation}


\end{itemize}


\begin{eqnarray}
\frac{\partial {\cal I}_1(\bm\sigma)}{\partial \bm\sigma} &=& {\bm 1} \\
\frac{\partial {\cal I}_2(\bm\sigma)}{\partial \bm\sigma} &=& {\bm \sigma} \\
\frac{\partial {\cal I}_3(\bm\sigma)}{\partial \bm\sigma} &=& \bm\sigma \cdot \bm\sigma
\end{eqnarray}




















































