\begin{flushright} {\tiny {\color{gray} \tt mms\_annulus\_kinbc.tex}} \end{flushright}
%~~~~~~~~~~~~~~~~~~~~~~~~~~~~~~~~~~~~~~~~~~~~~~~~~~~~~~~~~~~~~~~~~~~~~~~~~~~~~~~~~~~~~~~~~~~~~~~~~~

Let us consider an annulus domain of inner radius $R_1$ and outer radius $R_2$.
Boundary conditions are $\vec\upnu(r=R_1)=\upnu_1 \vec{e}_\theta$ and 
$\vec\upnu(r=R_2)=\upnu_2 \vec{e}_\theta$. Density is assumed to be zero in the domain. 

As seen in Section~\ref{ss:ankbc}, the Laplacian of a vector $\vec{A}$ is given 
by\footnote{\url{https://en.wikipedia.org/wiki/Vector\_Laplacian}} 
\[
\nabla^2 \vec{A} = \nabla(\nabla\cdot\vec{A}) - \nabla\times(\nabla \times\vec{A})
=
\left(
\begin{array}{l}
\frac{\partial^2 A_r}{\partial r^2} + \frac{1}{r} \frac{\partial A_r}{\partial r} - 
\frac{1}{r^2} A_r  + \frac{1}{r^2} \frac{\partial^2 A_r}{\partial \theta^2}  - 
\frac{2}{r^2} \frac{\partial A_\theta}{\partial \theta} \\ \\
\frac{\partial^2 A_\theta}{\partial r^2} + \frac{1}{r} \frac{\partial A_\theta}{\partial r} - 
\frac{1}{r^2} A_\theta  + \frac{1}{r^2} \frac{\partial^2 A_\theta}{\partial \theta^2}  
+ \frac{2}{r^2} \frac{\partial A_r}{\partial \theta} 
\end{array}
\right)
\]
%=
%\left(
%\begin{array}{l}
%\Delta A_r \\ \\ \Delta A_\theta
%\end{array}
%\right)
Given the symmetry of the problem and the boundary conditions we know that the 
solution is as follows: 
\[
{\vec \upnu}(r,\theta) = \upnu_\theta(r) {\vec e}_\theta
\]

Using this velocity field, we can now obtain the pressure field 
by solving the Stokes equation
\[
-\vec\nabla p + \eta {\vec\nabla}^2 {\vec \upnu} = {\vec 0}
\]
since density is zero. Because of symmetry we also expect the pressure to
be a function of $r$ only, i.e. $p(r)$.
The Stokes equation in polar coordinates then writes:
\begin{eqnarray}
-\frac{\partial p}{\partial r} &=& 0 \\
-\frac{1}{r}\underbrace{\frac{\partial p}{\partial \theta}}_{=0} + 
\frac{\partial^2 \upnu_\theta}{\partial r^2} + \frac{1}{r} \frac{\partial \upnu_\theta}{\partial r} 
- \frac{1}{r^2} \upnu_\theta  &=&0 
\end{eqnarray}
so that the pressure is a constant which we arbitrarily set to zero. 
We assume $\upnu_\theta(r) = r^\alpha$, and the second equation above becomes: 
\[
\alpha(\alpha-1) r^{\alpha-2}  + \alpha r^{\alpha-2}  - r^{\alpha-2}  = 0 
\]
reducing to $\alpha^2-1=0$, i.e. $\alpha=\pm 1$, since the above equation must be valid
for any value of $r$.
The generic solution then can be written as
\[
\upnu_\theta (r) = A r + \frac{B}{r}
\]
Using the b.c. : 
\[
\upnu_\theta (R_1) = A R_1 + \frac{B}{R_1} = \upnu_1
\]
\[
\upnu_\theta (R_2) = A R_2 + \frac{B}{R_2} = \upnu_2
\]
or, 
\[
A + \frac{B}{R_1^2} = \frac{\upnu_1}{R_1}
\]
\[
A + \frac{B}{R_2^2} = \frac{\upnu_2}{R_2}
\]
so
\[
B=\frac{ \frac{\upnu_1}{R_1}-\frac{\upnu_2}{R_2}  }{\frac{1}{R_1^2} - \frac{1}{R_2^2}} = 
R_1R_2 \frac{ \upnu_1R_2-\upnu_2R_1    }{R_2^2-R_1^2}
\]
and 
\[
A=\frac{\upnu_2R_2-\upnu_1R_1}{R_2^2-R_1^2}
\]
%\includegraphics[width=10cm]{PROJECTS/annulus/velprof}

\begin{mdframed}[backgroundcolor=blue!5]
\[
\upnu_\theta (r) = \frac{\upnu_2R_2-\upnu_1R_1}{R_2^2-R_1^2}  r 
+ \frac{R_1R_2}{r} \frac{ \upnu_1R_2-\upnu_2R_1    }{R_2^2-R_1^2}
\]
\end{mdframed}

We can verify that the flow is indeed incompressible:
\[
\vec\nabla\cdot\vec\upnu = \frac{1}{r}\frac{\partial (r\upnu_r)}{\partial r} 
+ \frac{1}{r}\frac{\partial \upnu_\theta}{\partial \theta} = 0
\]
since $\upnu_r=0$ and $\upnu_\theta$ does not depend on $\theta$.

Note that we could have used a non-zero density: as long as it does not 
depend on $\theta$ and the gravity points towards the center,
it allows for a decoupling of the equations, thereby 
only contributing to a lithostatic pressure field.
