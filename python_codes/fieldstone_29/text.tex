
\index{stones}{$Q_1 \times P_0$ element}
\index{stones}{Mixed Formulation}
\index{stones}{Open Boundary Conditions}

\Literature \cite{chgv12}

In what follows we will investigate the use of the so-called open boundary conditions in 
the very simple context of a 2D Stokes sphere experiment. 


We start with a domain without the sphere. Essentially, it is what people would 
call an aquarium.
Free slip boundary conditions are prescribed on the sides and no-slip conditions at the bottom. 
The top surface is left free. The fluid has a density $\rho_0=1$ and a viscosity $\eta_0=1$.
In the absence of any density difference in the domain there is no active buoyancy force
so that we expect a zero velocity field and a lithostatic pressure field. 
This is indeed what we recover:  

\begin{center}
\includegraphics[width=5cm]{python_codes/fieldstone_29/results/aquarium/vel}
\includegraphics[width=5cm]{python_codes/fieldstone_29/results/aquarium/p}
\end{center}

If we now implement a sphere parametrised by its density $\rho_s=\rho_0+1$, its 
viscosity $\eta_s=10^3$ and its radius $R_s=0.123$ in the middle of the domain,  
we see clear velocity field which logically shows the sphere falling downward and 
a symmetric return flow of the fluid on each side:

\begin{center}
\includegraphics[width=5cm]{python_codes/fieldstone_29/results/sphere_free_slip/vel}
\includegraphics[width=5cm]{python_codes/fieldstone_29/results/sphere_free_slip/p}
\includegraphics[width=5cm]{python_codes/fieldstone_29/results/sphere_free_slip/density}
\end{center}

Unfortunately it has been widely documented that the presence of free-slip 
boundary conditions affects the evolution of subduction \cite{chgv12}, 
even when these are placed rather far from the subduction zone. 
A proposed solution to this problem is the use of 'open boundary conditions'
which are in fact stress boundary conditions. 
The main idea is to prescribe a stress on the lateral boundaries (instead of free slip)
so that it balances out exactly the existing lithostatic pressure inside the domain 
along the side walls. Only pressure deviations with respect to the 
lithostatic are responsible for flow and such boundary conditions allow flow  
across the boundaries.

We need the lithostatic pressure and compute it before hand (which is trivial in 
our case but can prove to be a bit more tedious in real life situations when for instance
density varies in the domain as a function of temperature and/or pressure).

\begin{lstlisting}
plith = np.zeros(nnp,dtype=np.float64)
for i in range(0,nnp):
    plith[i]=(Ly-y[i])*rho0*abs(gy)
\end{lstlisting}


Let us start with a somewhat pathological case: even in the absence of the 
sphere, what happens when no 
boundary conditions are prescribed on the sides? The answer is simple: 
think about an aquarium without side walls, or a broken dam. The velocity field 
indeed shows a complete collapse of the fluid left and right of the bottom.

\begin{center}
\includegraphics[width=5cm]{python_codes/fieldstone_29/results/nobc/vel}
\includegraphics[width=5cm]{python_codes/fieldstone_29/results/nobc/p}
\end{center}

Let us then continue (still with no sphere) but let us now switch on the open 
boundary conditions. Since the side boundary conditions match the lithostatic 
pressure we expect no flow at all in the absence of any density perturbation
in the system. This is indeed what is recovered:

\begin{center}
\includegraphics[width=5cm]{python_codes/fieldstone_29/results/no_sphere_openbc/vel}
\includegraphics[width=5cm]{python_codes/fieldstone_29/results/no_sphere_openbc/p}
\end{center}
 
Finally, let us reintroduce the sphere. This 
time flow is allowed through the left and right side boundaries:

\begin{center}
\includegraphics[width=5cm]{python_codes/fieldstone_29/results/sphere_openbc/vel}
\includegraphics[width=5cm]{python_codes/fieldstone_29/results/sphere_openbc/p}
\end{center}

Finally, although horizontal velocity Dirichlet boundary conditions and open 
boundary conditions are not compatible, the same is not true for the vertical component 
of the velocity: the open b.c. implementation acts on the horizontal velocity 
dofs only, so that one can fix the vertical component to zero, as is shown hereunder:

\begin{center}
\includegraphics[width=5cm]{python_codes/fieldstone_29/results/sphere_openbc_v0/vel}
\includegraphics[width=5cm]{python_codes/fieldstone_29/results/sphere_openbc_v0/p}
\end{center}

We indeed see that the in/outflow on the sides is perpendicular to the boundaries. 

Turning now to the actual implementation, we see that it is quite trivial, since 
all element edges are vertical, and all have the same vertical dimension $h_x$. 
Since we use a $Q_0$ approximation for the pressure we need to prescribe a single
pressure value in the middle of the element. Finally because of the sign of the 
normal vector projection onto the $x$-axis, we obtain:

\begin{lstlisting}
if open_bc_left and x[icon[0,iel]]<eps: # left side
   pmid=0.5*(plith[icon[0,iel]]+plith[icon[3,iel]])
   f_el[0]+=0.5*hy*pmid
   f_el[6]+=0.5*hy*pmid
if open_bc_right and x[icon[1,iel]]>Lx-eps: # right side
   pmid=0.5*(plith[icon[1,iel]]+plith[icon[2,iel]])
   f_el[2]-=0.5*hy*pmid
   f_el[4]-=0.5*hy*pmid
\end{lstlisting}

These few lines of code are added after the elemental matrices and rhs are built, 
and before the application of other Dirichlet boundary conditions, and assembly.





