\noindent
\includegraphics[height=1.25cm]{images/pictograms/under_construction}
\includegraphics[height=1.25cm]{images/pictograms/tools}
\includegraphics[height=1.25cm]{images/pictograms/pic}

%%%%%%%%%%%%%%%%%%%%%%%%%%%%%%%%%%%%%%%%%%%%%%%%%%%%%%%%%%%%%%%%%%%%%%%%%%%%%%%%%%%%%%%%%%%%%%%%%%%

\begin{flushright} {\tiny {\color{gray} python\_codes/fieldstone\_163/text.tex}} \end{flushright}

%\lstinputlisting[language=bash,basicstyle=\small]{python_codes/template_keywords.key}

\par\noindent\rule{\textwidth}{0.4pt}

\begin{center}
\inpython
{\small Code: \url{https://github.com/cedrict/fieldstone/tree/master/python_codes/fieldstone_163}}
\end{center}

\par\noindent\rule{\textwidth}{0.4pt}

{\sl This stone was developed in collaboration with Arie van den Berg}. \index{contributors}{A. van den Berg}

\par\noindent\rule{\textwidth}{0.4pt}

%%%%%%%%%%%%%%%%%%%%%%%%%%%%%%%%%%%%%%%%%%%%%%%%%%%%%%%%%%%%%%%%%%%%%%%%%%%%%%%%%%%%%%%%%%%%%%%%%%%

The first version of this code considers a mesh of $nelx \times nely$ elements spanning the unit square.
The coordinates of the nodes are stored in the \verb|x,y| arrays and the connectivity is \verb|icon|. 
At each node of the mesh the velocity $(u,v)=(-0.5+y,0.5-x)$ is prescribed:
\begin{verbatim}
do i=1,np
   u(i)=-(0.5-y(i))
   v(i)=(0.5-x(i))
end do
\end{verbatim}

Inside the domain \verb|swarm_n| markers are placed randomly:
\begin{verbatim}
do i=1,swarm_n
   call random_number(eta)
   call random_number(chi)
   swarm_x(i)=eta*Lx
   swarm_y(i)=chi*Ly
end do
\end{verbatim}

The markers are then localised, i.e. one must find for each the element it resides in.
The velocity is then computed on the marker by means of $Q_1$ basis functions
and it is advected with a first order Euler step for simplicity: 
\begin{verbatim}
do i=1,swarm_n

   ! find cell
   ielx=swarm_x(i)/Lx*nelx+1
   iely=swarm_y(i)/Ly*nely+1
   iel(i)=nelx*(iely-1)+ielx

   ! find local coordinates in element
   xmin=x(icon(1,iel(i)))
   xmax=x(icon(3,iel(i)))
   ymin=y(icon(1,iel(i)))
   ymax=y(icon(3,iel(i)))
   r=((swarm_x(i)-xmin)/(xmax-xmin)-0.5d0)*2.d0
   s=((swarm_y(i)-ymin)/(ymax-ymin)-0.5d0)*2.d0

   ! evaluate Q1 basis functions
   N(1)=0.25*(1-r)*(1-s)
   N(2)=0.25*(1+r)*(1-s)
   N(3)=0.25*(1+r)*(1+s)
   N(4)=0.25*(1-r)*(1+s)

   ! compute velocity
   swarm_u(i)=sum(N*u(icon(:,iel(i))))
   swarm_v(i)=sum(N*v(icon(:,iel(i))))

   ! advect marker
   swarm_x(i)=swarm_x(i)+swarm_u(i)*dt
   swarm_y(i)=swarm_y(i)+swarm_v(i)*dt

end do
\end{verbatim}

Looking at this code we find that the background resolution of the mesh 
plays virtually no role, and that this loop is executed for each marker
but localising and advecting one marker can be done completely 
independently from another marker. 
Assuming that a large number of markers is used, say ${\cal O}(10^7)$, 
one could be tempted to think of parallelisation to improve performance. 

There are usually 2 main paradigms when it comes to parallelization,
see for example \textcite{vacp22} (2022): openmp and mpi. 
We fill now focus on mpi and therefore initialise mpi in the code as follows:
 
\begin{verbatim}
program opla
use mpi
[...]
call mpi_init(ierr)
call mpi_comm_size (mpi_comm_world,nproc,ierr)
call mpi_comm_rank (mpi_comm_world,iproc,ierr)
\end{verbatim}

The code is then compiled with 
\begin{verbatim}
> mpif90 -O3 stone.f90
\end{verbatim}
and can be run in parallel as follows:
\begin{verbatim}
> mpirun -np 4 ./a.out 
\end{verbatim}

%---------------------------------------------------------------
\section*{First approach}

Although it is often a bad idea, one could wish to use the code above and
modify it as little as possible while inserting the required mpi commands. Let us 
adopt here this (doomed\footnote{spoiler alert}) approach for the sake of learning.

We then proceed to change 
\begin{verbatim}
do i=1,swarm_n
   [...]
end do
\end{verbatim}
into
\begin{verbatim}
do i=1+iproc,swarm_n,nproc
   print *,iproc,i
   [...]
end do
\end{verbatim}
The reason for this is as follows: let us set \verb|swarm_n|=15 and run the code with nproc=3.
The print statement returns the following lines\footnote{I have reordered the lines
for better readibility}:
\begin{verbatim}
 iproc=           0 takes care of marker           1
 iproc=           0 takes care of marker           4
 iproc=           0 takes care of marker           7
 iproc=           0 takes care of marker          10
 iproc=           0 takes care of marker          13
 iproc=           1 takes care of marker           2
 iproc=           1 takes care of marker           5
 iproc=           1 takes care of marker           8
 iproc=           1 takes care of marker          11
 iproc=           1 takes care of marker          14
 iproc=           2 takes care of marker           3
 iproc=           2 takes care of marker           6
 iproc=           2 takes care of marker           9
 iproc=           2 takes care of marker          12
 iproc=           2 takes care of marker          15
\end{verbatim}
We see that each thread only deals with a subset of the markers.
We then proceed to modify the last two lines of the big loop as follows:
\begin{verbatim}
   xtemp(i)=swarm_x(i)+swarm_u(i)*dt
   ytemp(i)=swarm_y(i)+swarm_v(i)*dt
\end{verbatim}
where \verb|xtemp| and \verb|ytemp| are \verb|swarm_n|-long arrays that have been set to zero beforehand.  
In this case the \verb|xtemp| array would look like this at the end on each thread (the empty spaces indicate
that nothing was written there):
\begin{verbatim}
         ---------------------------------------------------
iproc=0  | X |  |  | X |  |  | X |  |  | X |  |  | X |  |  |
         ---------------------------------------------------

         ---------------------------------------------------
iproc=1  |  | X |  |  | X |  |  | X |  |  | X |  |  | X |  | 
         ---------------------------------------------------

         ---------------------------------------------------
iproc=2  |  |  | X |  |  | X |  |  | X |  |  | X |  |  | X |
         ---------------------------------------------------
\end{verbatim}
All we have to do is to do a global summation of these arrays into \verb|swarm_x| across 
all threads (same for \verb|swarm_y|):
\begin{verbatim}
call mpi_allreduce(xtemp,swarm_x,swarm_n,mpi_double_precision,mpi_sum,mpi_comm_world,ierr)
call mpi_allreduce(ytemp,swarm_y,swarm_n,mpi_double_precision,mpi_sum,mpi_comm_world,ierr)
\end{verbatim}
In the end we have parallelized this loop. We finish by adding 
mpi commands that allow for the timing of sections of code:
\begin{verbatim}
start  = MPI_Wtime()
[...]
finish  = MPI_Wtime()
print *,'time=',finish-start,'s'
\end{verbatim}
In practice one would use a RK4 algorithm to advect the markers. 
In order to approximate the cost of RK4 without actually implementing it since 
we are not interested in the advection itself we can wrap the inside of the loop 
inside an additional loop that enforces that localisation and advection happen 
4 times for each marker (similar to RK4):
\begin{verbatim}
do i=1+iproc,swarm_n,nproc
   do k=1,4 ! to mimic cost of RK4

   end do
end do
\end{verbatim}

\subsection*{Timings}

We then proceed to run the code on 1,2,3,4 threads\footnote{This 
was run on my Thinkpad laptop with Intel Core i7-7820HQ CPUi at 2.90GHz.}:

\begin{verbatim}
advect:  0.19242236700000004             0
allreduce:   1.4708635000000081E-002     0

advect:  0.13948245200000001             1
advect:  0.14018181199999999             0
allreduce:   9.7357410000000005E-002     0
allreduce:   9.7526858999999994E-002     1

advect:  0.12392915000000000             1
advect:  0.12359175000000000             0
advect:  0.12411840600000000             2
allreduce:  0.19212144400000003          2
allreduce:  0.20365041300000000          0
allreduce:  0.20382717899999997          1

advect:  0.13753345000000000             1
advect:  0.13849821400000001             2
advect:  0.13913681200000000             3
advect:  0.13707424700000001             0
allreduce  0.20147564300000001           3
allreduce  0.20159588100000003           0
allreduce  0.20169826699999999           2
allreduce  0.20180693799999999           1
\end{verbatim}

We immediately see the problem: when using only 1 thread (i.e. no parallelisation 
at all) the entire loop over all 10 million markers only takes 0.2s! 
This raises the question whether we should even have tried to parallelise this 
bit of code! 
Second, as soon as multiple threads are used the cost of the allreduce becomes 
as large as the cost to carry out the big loop. After all 0.2s to for all threads 
to send each other 2x 10 million double precision numbers is not too surprising.

This approach is doomed, unless the tasks inside the loop over markers would be 
very costly.

This explains why the above approach was not used in \textcite{thie11} (2011):
\begin{displayquote}
{\color{darkgray} 
In the case of parallel computations, the (grid) domain decomposition leads 
to a partitioning of
the cloud across the processes. Each process only stores the part of
the cloud necessary to compute the part of the FE matrix it needs
to build. This allows for subsequent memory savings (per process):
a 128x128x32 grid with an average of 27 points per element
contains over 14 millions points, whose storage in double precision
(x, y, z and $\epsilon$ fields) occupies about 430Mb in memory. Using only
16 processes reduces the memory requirement to less than 27Mb
per process. Also, most of the algorithms dealing with the cloud
(advection, strain increment) are in ${\cal O}(N)$, so that the cloud decomposition 
also improves the overall efficiency.

[...]

Also, when a given
point is advected and enters an element which belongs to two processes, 
its position must be communicated to the other process. A
very similar procedure must take place when a point leaves such
an element: it must be ‘deleted’ from the list of the other process.
The intrinsic MPI routine \verb|mpi_alltoall| is extensively used to this
purpose.
}
\end{displayquote}






