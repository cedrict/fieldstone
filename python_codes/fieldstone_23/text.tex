\noindent
\includegraphics[height=1.25cm]{images/pictograms/benchmark}
\includegraphics[height=1.25cm]{images/pictograms/msc}
\includegraphics[height=1.25cm]{images/pictograms/FEM}
\includegraphics[height=1.25cm]{images/pictograms/clean}

%%%%%%%%%%%%%%%%%%%%%%%%%%%%%%%%%%%%%%%%%%%%%%%%%%%%%%%%%%%%%%%%%%%%%%%%%%%%%%%%%%%%%%%%%%%%%%%%%%%

\begin{center}
\inpython
{\small Code: \url{https://github.com/cedrict/fieldstone/tree/master/python_codes/fieldstone_23}}
\end{center}

\par\noindent\rule{\textwidth}{0.4pt}

This work is part of the MSc thesis of T. Weir (2018).

\par\noindent\rule{\textwidth}{0.4pt}

Last revision: Feb. 1st, 2026.

\par\noindent\rule{\textwidth}{0.4pt}

%%%%%%%%%%%%%%%%%%%%%%%%%%%%%%%%%%%%%%%%%%%%%%%%%%%%%%%%%%%%%%%%%%%%%%%%%%%%%%%%%%

We first start with an isothermal Stokes flow, so that we disregard the heat 
transport equation and the equations we wish to solve are simply:

\begin{align}
-\vec\nabla p +\vec\nabla \cdot \left[2 \eta \left(\dot\varepsilon(\vec \upnu)
            - \frac{1}{3}(\vec\nabla \cdot \vec\upnu) \mathbf 1\right)  \right] +   \rho \vec{g} &=\vec{0}
  &
  & \textrm{in $\Omega$},
  \\
  \vec\nabla \cdot (\rho \vec\upnu) &= 0
  &
  & \textrm{in $\Omega$}
\end{align}

The second equation can be rewritten 
$\vec\nabla \cdot (\rho \vec{\upnu}) =  \rho \vec\nabla \cdot \vec{\upnu} 
+ \vec{\upnu} \cdot \vec{\nabla}\rho=0$
or, 
\[
\vec\nabla \cdot \vec{\upnu} + \frac{1}{\rho} \vec{\upnu} \cdot {\vec\nabla}\rho = 0
\]
Note that this presupposes that the density is not zero anywhere in the domain.

We use a mixed formulation and therefore keep both velocity and pressure as unknowns. 
We end up having to solve the following system:
\[
\left(
\begin{array}{cc}
\K & \G \\ \G^T+\Z & 0 
\end{array}
\right)
\cdot
\left(
\begin{array}{c}
\vec{\cal V} \\ \vec{\cal P}
\end{array}
\right)
=
\left(
\begin{array}{c}
\vec{f} \\ \vec{h}
\end{array}
\right)
\quad\quad
{\rm or,}
\quad\quad
{\cal A} \cdot X = \vec{b}
\]
Where $\K$ is the stiffness matrix, $\G$ is the discrete gradient operator, 
$\G^T$ is the discrete divergence operator, $\vec{\cal V}$ the velocity vector, 
$\vec{\cal P}$ the pressure vector.
Note that the term $\Z\cdot \vec{\cal V}$ derives from term $\vec{\upnu} \cdot \vec{\nabla}\rho$ 
in the continuity equation. 

Each block $\K$, $\G$ , $\Z$ and vectors $f$ and $h$ are built separately 
in the code and assembled into the matrix ${\cal A}$ and vector $\vec{b}$ 
afterwards. ${\cal A}$ and $\vec{b}$ are then passed to the solver. 
We will see later that there are alternatives to solve this approach which do not require to 
build the full Stokes matrix $\A$. 

{\sl Remark}: the term $\Z\cdot \vec{\cal V}$ is often put in the rhs (i.e. added to $h$) so that 
the Stokes matrix ${\cal A}$ retains the same structure as in the incompressible case. This is indeed 
how it is implemented in ASPECT (check?). This however requires more work since the rhs depends 
on the solution and some iterations are then needed. 

In the case of a compressible flow the strain rate tensor and the deviatoric strain 
rate tensor are no more equal (since ${\vec\nabla}\cdot\vec{\upnu} \neq 0$).
The deviatoric strainrate tensor is given by\footnote{See the ASPECT manual for a 
justification of the 3 value in the denominator in 2d and 3d.} 
\[
\dot{\bm \varepsilon}^d(\vec{\upnu})=
\dot{\bm \varepsilon}(\vec{\upnu})-\frac{1}{3} \text{Tr}(\dot{\bm \varepsilon}) {\bm 1}
=\dot{\bm \varepsilon}(\vec{\upnu})-\frac{1}{3} ({\vec\nabla}\cdot \vec{\upnu}) {\bm 1}
\]
In that case:
\begin{eqnarray}
\dot{\varepsilon}_{xx}^d 
&=& \frac{\partial u}{\partial x}
-\frac{1}{3} \left( \frac{\partial u}{\partial x} + \frac{\partial v}{\partial y} \right) 
= \frac{2}{3}\frac{\partial u}{\partial x}
-\frac{1}{3} \frac{\partial v}{\partial y}
%=
%\frac{2}{3} \sum_{i=1}^4 \frac{\partial N_i}{\partial x}\;  u_i 
%-\frac{1}{3} \sum_{i=1}^4 \frac{\partial N_i}{\partial y}\;  v_i 
\\
\dot{\varepsilon}_{yy}^d 
&=& \frac{\partial v}{\partial y}
-\frac{1}{3} \left( \frac{\partial u}{\partial x} + \frac{\partial v}{\partial y} \right) 
=-\frac{1}{3} \frac{\partial u}{\partial x} 
+ \frac{2}{3} \frac{\partial v}{\partial y} 
%=-\frac{1}{3}  \sum_{i=1}^4 \frac{\partial N_i}{\partial x}\;  u_i
%+ \frac{2}{3} \sum_{i=1}^4 \frac{\partial N_i}{\partial y}\;  v_i
\\
\dot{\varepsilon}_{xy}^d 
&=& 
\frac12 \left( \frac{\partial u}{\partial y} +\frac{\partial v}{\partial x}  \right)
%= \sum_{i=1}^4 \frac{\partial N_i}{\partial y}\;  u_i
%+ \sum_{i=1}^4 \frac{\partial N_i}{\partial x}\;  v_i
\end{eqnarray}
and then 
\[
\dot{\bm \varepsilon}^d(\vec{\upnu})
=
\left(
\begin{array}{cc}
\frac{2}{3} \frac{\partial u}{\partial x} -\frac{1}{3} \frac{\partial v}{\partial y} &
\frac{1}{2}\frac{\partial u}{\partial y} + \frac{1}{2}\frac{\partial v}{\partial x}  \\ \\
\frac{1}{2}\frac{\partial u}{\partial y} + \frac{1}{2}\frac{\partial v}{\partial x}  &
-\frac{1}{3} \frac{\partial u}{\partial x} +\frac{2}{3} \frac{\partial v}{\partial y} 
\end{array}
\right)
\]

From $\vec{\tau} = 2\eta \vec{\epsilon}^d$ we arrive at:
\[
\left(
\begin{array}{c}
\tau_{xx}\\
\tau_{yy}\\
\tau_{xy}\\
\end{array}
\right)
=
2\eta
\left(
\begin{array}{c}
\dot{\epsilon}_{xx}^d \\
\dot{\epsilon}_{yy}^d \\
\dot{\epsilon}_{xy}^d 
\end{array}
\right)
=2 \eta
\left(
\begin{array}{ccc}
2/3 & -1/3& 0 \\
-1/3 & 2/3 & 0 \\
0 & 0 & 1/2 \\
\end{array}
\right)
\cdot 
\left(
\begin{array}{c}
\frac{\partial u}{\partial x} \\ 
\frac{\partial v}{\partial y} \\ 
\frac{\partial u}{\partial y}\! +\! \frac{\partial v}{\partial x} \\
\end{array}
\right)
=
\eta
\left(
\begin{array}{ccc}
4/3 & -2/3& 0 \\
-2/3 & 4/3 & 0 \\
0 & 0 & 1 \\
\end{array}
\right)
\cdot 
\left(
\begin{array}{c}
\frac{\partial u}{\partial x} \\ 
\frac{\partial v}{\partial y} \\ 
\frac{\partial u}{\partial y}\! +\! \frac{\partial v}{\partial x} \\
\end{array}
\right)
\]
or, 
\[
\vec{\tau} = {\bm C}_\eta\cdot {\bm B} \cdot \vec{\cal V}
\]


\paragraph{Implementation} 
All terms of the linear system are common many previous stones and described in 
the {\tt manual.pdf}, but not $\Z_e$.
In practice it is computed as follows:

\begin{lstlisting}
for iel in range(0, nel):
    Z_el=np.zeros((m_V*ndof_V,1),dtype=np.float64)
    for iq in [-1,1]:
        for jq in [-1,1]:
            [...]
            drhodxq=np.dot(dNdx_V,rho_nodal[icon_V[:,iel]])
            drhodyq=np.dot(dNdy_V,rho_nodal[icon_V[:,iel]])
            for i in range(0,m_V):
                Z_el[ndof_V*i  ,0]-=N_V[i]*drhodxq/rhoq *JxWq 
                Z_el[ndof_V*i+1,0]-=N_V[i]*drhodyq/rhoq *JxWq 
\end{lstlisting}

Note that the exact denisty gradient could have been computed from $\rho(x,y)$
and prescribed at the quadrature points. 
In any case the term we need to integrate is not a polynomial so that 
the employed quadrature rule is not adequate.


{\color{red} TODO: implement Q2Q1 and better matrix building !!!}


\newpage
In order to test our implementation we have created five manufactured solutions.
Note that contrarily to previous cases using odd number of elements in 
each direction still yields very strong checkerboard modes.


%--------------------------------------------------------------------
\subsection*{benchmark \#1 ({\tt ibench=1})}

Starting from a density field given by
$\rho(x,y) = xy$, we postulate the following velocity field:
\begin{eqnarray}
u(x,y) &=& \frac{C_x}{x} \nn\\
v(x,y) &=& \frac{C_y}{y} \nn
\end{eqnarray}
We then confirm that 
\[
\vec\nabla \cdot (\rho \vec\upnu) 
= \rho \vec\nabla \cdot \vec\upnu
+ \vec\upnu \cdot \vec\nabla \rho
= xy \left(\frac{\partial u}{\partial x}+\frac{\partial v}{\partial y} \right)
+ \frac{C_x}{x} \frac{\partial \rho}{\partial x}
+ \frac{C_y}{y} \frac{\partial \rho}{\partial y}
= xy \left( -\frac{C_x}{x^2} - \frac{C_y}{y^2}  \right)
+ \frac{C_x y}{x} + \frac{C_y x}{y}=0
\]
From the definition of the velocity we can compute the strain rate and deviatoric strain rate tensors:

\begin{eqnarray}
\dot{\bm \varepsilon} 
&=& 
\begin{pmatrix}
-C_x/x^ 2 & 0 \\
0 & -C_y/y^2
\end{pmatrix} \nn\\
\dot{\bm \varepsilon}^d 
&=& \dot{\bm \varepsilon} -\frac13  (\vec\nabla \cdot \vec\upnu) {\bm 1} \nn\\
&=& \dot{\bm \varepsilon} -\frac13  \left( -\frac{C_x}{x^2} - \frac{C_y}{y^2} \right) {\bm 1} \nn\\
&=&\dot{\bm \varepsilon} +\frac13  \left( \frac{C_x}{x^2} + \frac{C_y}{y^2}  \right) {\bm 1} \nn\\
&=&
\begin{pmatrix}
-\frac{2C_x}{3x^2} + \frac{C_y}{3y^2} & 0 \\
0 & \frac{C_x}{3x^2} - \frac{2C_y}{3y^2}
\end{pmatrix} \nn
\end{eqnarray}

We further postulate $g_x(x,y) = \frac{1}{x}$ and $g_y(x,y) = \frac{1}{y}$.
Assuming the viscosity to be constant, the momentum conservation equation is then
\begin{eqnarray}
\vec{0}&=& -\vec\nabla p + \vec\nabla \cdot (2 \eta \dot{\bm \varepsilon}^d )+ \rho \vec{g} \nn\\
&=& 
\begin{pmatrix} 
-\frac{\partial p}{\partial x} \\
-\frac{\partial p}{\partial y} 
\end{pmatrix} 
+  2 \eta \vec\nabla \cdot   
\begin{pmatrix}
-\frac{2C_x}{3x^2} + \frac{C_y}{3y^2} & 0 \\
0 & \frac{C_x}{3x^2} - \frac{2C_y}{3y^2}
\end{pmatrix}
+ 
\begin{pmatrix}
y \\x
\end{pmatrix} \nn\\
&=&
\begin{pmatrix} 
-\frac{\partial p}{\partial x} + 2\eta \frac{\partial}{\partial x} (-\frac{2C_x}{3x^2} +\frac{C_y}{3y^2})+ y \\
-\frac{\partial p}{\partial y} + 2\eta \frac{\partial}{\partial y} (\frac{C_x}{3x^2} -\frac{2C_y}{3y^2})+ x 
\end{pmatrix} \nn\\
&=& 
\begin{pmatrix} 
-\frac{\partial p}{\partial x} + 2\eta \frac{\partial}{\partial x} (-\frac{2C_x}{3x^2} )+ y \\
-\frac{\partial p}{\partial y} + 2\eta \frac{\partial}{\partial y} (-\frac{2C_y}{3y^2})+ x 
\end{pmatrix} 
\end{eqnarray}
The first line leads to write 
\[
p(x,y) = -\frac{4 \eta C_x}{3x^2}   + yx + f(y)
\]
Inserting this in the second line:
\[
-\frac{\partial }{\partial y}  \left( -\frac{4 \eta C_x}{3x^2}   + yx + f(y) \right) 
+ 2\eta \frac{\partial}{\partial y} \left(-\frac{2C_y}{3y^2} \right)+ x = 0
\]
\[
-x - f'(y) + 2\eta \frac{\partial}{\partial y} \left(-\frac{2C_y}{3y^2} \right)+ x = 0
\]
or, 
\[
f(y) =   -\frac{4 \eta C_y}{3y^2} + Const
\]
and finally
\[
p(x,y) =  -\frac{4 \eta C_x}{3x^2}  -\frac{4 \eta C_y}{3y^2}   + yx    + Const 
\]
Requiring $\iint_\square p \; dxdy=0$ where $\square=[1:2]\times [1:2]$ (because of the 
expressions for velocity and gravity the origin of the axis system cannot be included in the 
domain) leads to 
\begin{eqnarray}
\iint_\square  
\left( -\frac{4 \eta C_x}{3x^2}  -\frac{4 \eta C_y}{3y^2}   + yx    + Const \right) 
&=& 
\iint_\square   \left( -\frac{4 \eta C_x}{3x^2}  -\frac{4 \eta C_y}{3y^2}   + yx  \right) dxdy
+\iint_\square   Const \; dx dy \nn\\
&=&\iint_\square   \left( -\frac{4 \eta C_x}{3x^2}  -\frac{4 \eta C_y}{3y^2}   + yx  \right) dxdy
+ Const
\end{eqnarray}
so that 
\[
Const = -\int_1^2 \int_1^2  \left( -\frac{4 \eta C_x}{3x^2}  -\frac{4 \eta C_y}{3y^2}   + yx  \right) dxdy
\]
We then set $\eta=C_x=C_y=1$ for simplicity so that we find\footnote{Thank you WolframAlpha}:
\[
Const= -11/12
\]
and in the end
\[
p(x,y) = -\frac{4}{3x^2} -\frac{4}{3y^2} + yx -\frac{11}{12}
\]

\begin{lstlisting}
def gx(x,y,ibench):
    if ibench==1: val=1/y

def gy(x,y,ibench):
    if ibench==1: val=1/x

def density(x,y,ibench):
    if ibench==1: val=x*y

def u_th(x,y,ibench):
    if ibench==1: val=1/x

def v_th(x,y,ibench):
    if ibench==1: val=1/y

def p_th(x,y,ibench):
    if ibench==1:
       val = -4/(3*x**2) - 4/(3*y**2) + x*y  -11./12

def sr_xx_th(x,y,ibench):
    if ibench==1: val=-1/x**2

def sr_yy_th(x,y,ibench):
    if ibench==1: val=-1/y**2

def sr_xy_th(x,y,ibench):
    if ibench==1: val=0
\end{lstlisting}


\newpage
%------------------------------------------------ 
\subsection*{benchmark \#2 ({\tt ibench=2})}
Starting from the density field
$  \rho = \cos(x)\cos(y)$ we derive the velocity field:
\begin{equation}
u(x,y) = \frac{C_x}{\cos(x)} , 
v(x,y) = \frac{C_y}{\cos(y)}
\end{equation}
with $g_x = \frac{1}{\cos(y)}$ and $g_y = \frac{1}{\cos(x)}$ and 
this leads us to a pressure field given by: 
\begin{equation}
p =  \eta \Bigg(\frac{4C_x \sin(x)}{3\cos^2(x)} + \frac{4C_y \sin(y)}{3\cos^2(y)}\Bigg) 
+( \sin(x) + \sin(y) ) + C_0
\end{equation}
\[
\dot{\epsilon}_{xx} = C_x \frac{\sin(x)}{\cos^2(x)}
\quad
\quad
\quad
\dot{\epsilon}_{yy} = C_y \frac{\sin(y)}{\cos^2(y)}
\quad
\quad
\quad
\dot{\epsilon}_{xy} = 0 
\]
We choose $\eta=1$ and $C_x=C_y=1$. The domain is the unit square $[0:1]\times[0:1]$ and we obtain 
$C_0$ as before and obtain 
\[
C_0 = 2 - 2 \cos(1) + 8/3 (\frac{1}{\cos (1)} - 1)
\simeq 3.18823730
\]
(thank you WolframAlpha)


\begin{lstlisting}
def gx(x,y,ibench):
    if ibench==2: val = 1/np.cos(y)

def gy(x,y,ibench):
    if ibench==2: val=1/np.cos(x)

def density(x,y,ibench):
    if ibench==2: val = np.cos(x)*np.cos(y)

def u_th(x,y,ibench):
    if ibench==2: val=1/np.cos(x)

def v_th(x,y,ibench):
    if ibench==2: val=1/np.cos(y)

def p_th(x,y,ibench):
    if ibench==2:
       val = 4*np.sin(x)/(3*np.cos(x)**2)+4*np.sin(y)/(3*np.cos(y)**2) \
           + np.sin(x) + np.sin(y) - 2+2*np.cos(1)-8/3*(1/np.cos(1)-1) 

def sr_xx_th(x,y,ibench):
    if ibench==2: val=np.sin(x)/np.cos(x)**2

def sr_yy_th(x,y,ibench):
    if ibench==2: val=np.sin(y)/np.cos(y)**2

def sr_xy_th(x,y,ibench):
    if ibench==2: val=0
\end{lstlisting}









\newpage
%---------------------------------------------
\subsection*{benchmark \#3 ({\tt ibench=3}) - 1D Cartesian Linear}

The fields are as follows:
\begin{eqnarray}
g_x &=& -1   \nn\\
g_y &=& 0   \nn\\
\rho &=& x   \nn\\
u &=& 1/x   \nn\\
v &=& 0    \nn\\
p &=& -4/(3x^2) -x^2/2 + 11/6 \nn\\
\dot{\varepsilon}_{xx} &=& -1/x^2  \nn\\
\dot{\varepsilon}_{yy} &=& 0  \nn\\
\dot{\varepsilon}_{xy} &=& 0  \nn
\end{eqnarray}

\begin{lstlisting}
def gx(x,y,ibench):
    if ibench==3: val=-1

def gy(x,y,ibench):
    if ibench==3: val=0

def density(x,y,ibench):
    if ibench==3: val=x

def u_th(x,y,ibench):
    if ibench==3: val=1/x

def v_th(x,y,ibench):
    if ibench==3: val=0

def p_th(x,y,ibench):
    if ibench==3: val =  -4/(3*x**2) - x**2/2 + 11/6

def sr_xx_th(x,y,ibench):
    if ibench==3: val=-1/x**2

def sr_yy_th(x,y,ibench):
    if ibench==3: val=0 

def sr_xy_th(x,y,ibench):
    if ibench==3: val=0 
\end{lstlisting}


\newpage
%---------------------------------------------
\subsection*{benchmark \#4 ({\tt ibench=4}) - by Arie van den Berg}

The fields are as follows:
\begin{eqnarray}
g_x &=& 10   \nn\\
g_y &=& 0   \nn\\
\rho &=& 1/(1-x)   \nn\\
u &=& 1-x   \nn\\
v &=& 0    \nn\\
p &=& 10\ln (x-1) + 290.9  \nn\\
\dot{\varepsilon}_{xx} &=& -1  \nn\\
\dot{\varepsilon}_{yy} &=& 0  \nn\\
\dot{\varepsilon}_{xy} &=& 0  \nn
\end{eqnarray}


\begin{lstlisting}
def gx(x,y,ibench):
    if ibench==4: val=10

def gy(x,y,ibench):
    if ibench==4: val=0

def density(x,y,ibench):
    if ibench==4: val = 1/(1-x)

def u_th(x,y,ibench):
    if ibench==4: val=1-x

def v_th(x,y,ibench):
    if ibench==4: val=0

def p_th(x,y,ibench):
    if ibench==4: val = 10*log(x-1) +290.9 

def sr_xx_th(x,y,ibench):
    if ibench==4: val=0 

def sr_yy_th(x,y,ibench):
    if ibench==4: val=0 

def sr_xy_th(x,y,ibench):
    if ibench==4: val=0 
\end{lstlisting}











\newpage
%---------------------------------------------
\subsection*{benchmark \#5 ({\tt ibench=5}) - 1D Cartesian Sinusoidal}

The fields are as follows:
\begin{eqnarray}
g_x &=& -1   \nn\\
g_y &=& 0   \nn\\
\rho &=& \cos(x)   \nn\\
u &=& 1/\cos(x)   \nn\\
v &=& 0    \nn\\
p &=& 4\sin(x)/(3 \cos^2 x) - \sin(x) - 0.674723  \nn\\
\dot{\varepsilon}_{xx} &=&   \nn\\
\dot{\varepsilon}_{yy} &=& 0 \nn\\
\dot{\varepsilon}_{xy} &=& 0  \nn
\end{eqnarray}


\begin{lstlisting}
def gx(x,y,ibench):
    if ibench==5: val=-1

def gy(x,y,ibench):
    if ibench==5: val=0

def density(x,y,ibench):
    if ibench==5: val=np.cos(x)

def u_th(x,y,ibench):
    if ibench==5: val=1/np.cos(x)

def v_th(x,y,ibench):
    if ibench==5: val=0

def p_th(x,y,ibench):
    if ibench==5: val = 4*np.sin(x)/(3*np.cos(x)**2) -np.sin(x) -0.674723

def sr_xx_th(x,y,ibench):
    if ibench==5: val=0 

def sr_yy_th(x,y,ibench):
    if ibench==5: val=0 

def sr_xy_th(x,y,ibench):
    if ibench==5: val=0 
\end{lstlisting}


\newpage
%%%%%%%%%%%%%%%%%%%%%%%%%%%%%%%%%%%%%%%%%%%%%%
\section*{Results}

\begin{center}
\includegraphics[width=5.7cm]{python_codes/fieldstone_23/RESULTS/bench1/errors.pdf}
\includegraphics[width=5.7cm]{python_codes/fieldstone_23/RESULTS/bench2/errors.pdf}
\includegraphics[width=5.7cm]{python_codes/fieldstone_23/RESULTS/bench3/errors.pdf}\\
\includegraphics[width=5.7cm]{python_codes/fieldstone_23/RESULTS/bench4/errors.pdf}
\includegraphics[width=5.7cm]{python_codes/fieldstone_23/RESULTS/bench5/errors.pdf}\\
{\captionfont Discretisation errors for all five benchmarks.}
\end{center}

We see that all benchmarks are solutions of the Stokes equations 
(velocity field converges at the expected rate) but the pressure
checkerboard modes are so large that the pressure error is large 
and the pressure error does not converge at all.






