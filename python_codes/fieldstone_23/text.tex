\noindent
\includegraphics[height=1.25cm]{images/pictograms/benchmark}
\includegraphics[height=1.25cm]{images/pictograms/msc}
\includegraphics[height=1.25cm]{images/pictograms/FEM}
\includegraphics[height=1.25cm]{images/pictograms/clean}

%%%%%%%%%%%%%%%%%%%%%%%%%%%%%%%%%%%%%%%%%%%%%%%%%%%%%%%%%%%%%%%%%%%%%%%%%%%%%%%%%%%%%%%%%%%%%%%%%%%

\begin{center}
\inpython
{\small Code: \url{https://github.com/cedrict/fieldstone/tree/master/python_codes/fieldstone_23}}
\end{center}

\par\noindent\rule{\textwidth}{0.4pt}

Part of this work originates in the MSc thesis of T. Weir (2018).

\par\noindent\rule{\textwidth}{0.4pt}

Last revision: Feb. 6th, 2026.

\par\noindent\rule{\textwidth}{0.4pt}

%%%%%%%%%%%%%%%%%%%%%%%%%%%%%%%%%%%%%%%%%%%%%%%%%%%%%%%%%%%%%%%%%%%%%%%%%%%%%%%%%%

We first start with an isothermal Stokes flow, so that we disregard the heat 
transport equation and the equations we wish to solve are simply:

\begin{align}
-\vec\nabla p +\vec\nabla \cdot \left[2 \eta \left(\dot\varepsilon(\vec \upnu)
            - \frac{1}{3}(\vec\nabla \cdot \vec\upnu) \mathbf 1\right)  \right] +   \rho \vec{g} &=\vec{0}
  &
  & \textrm{in $\Omega$},
  \\
  \vec\nabla \cdot (\rho \vec\upnu) &= 0
  &
  & \textrm{in $\Omega$}
\end{align}

The second equation can be rewritten 
$\vec\nabla \cdot (\rho \vec{\upnu}) =  \rho \vec\nabla \cdot \vec{\upnu} 
+ \vec{\upnu} \cdot \vec{\nabla}\rho=0$
or, 
\[
\vec\nabla \cdot \vec{\upnu} + \frac{1}{\rho} \vec{\upnu} \cdot {\vec\nabla}\rho = 0
\]
Note that this equation presupposes that the density is not zero anywhere in the domain.
Implicitely, we also make the assumption that the density field is known 
when the equations above need to be solved.

We use a mixed formulation and therefore keep both velocity and pressure as unknowns. 
We end up having to solve the following system:
\[
\left(
\begin{array}{cc}
\K & \G \\ \G^T+\Z & 0 
\end{array}
\right)
\cdot
\left(
\begin{array}{c}
\vec{\cal V} \\ \vec{\cal P}
\end{array}
\right)
=
\left(
\begin{array}{c}
\vec{f} \\ \vec{h}
\end{array}
\right)
\quad\quad
{\rm or,}
\quad\quad
{\cal A} \cdot X = \vec{b}
\]
where $\K$ is the stiffness matrix, $\G$ is the discrete gradient operator, 
$\G^T$ is the discrete divergence operator, $\vec{\cal V}$ the velocity vector, 
$\vec{\cal P}$ the pressure vector.
Note that the term $\Z\cdot \vec{\cal V}$ derives from term $\rho^{-1} \vec{\upnu} \cdot \vec{\nabla}\rho$ 
in the continuity equation. 

The elemental matrix blocks $\K_e$, $\G_e$, $\Z_e$ as well as 
the elemental vectors $\vec{f}_e$ and $\vec{h}_e$ are built inside a 
loop of elements and assembled into the matrix ${\cal A}$ and vector $\vec{b}$. 
${\cal A}$ and $\vec{b}$ are then passed to the solver. 

{\sl Remark}: the term $\Z\cdot \vec{\cal V}$ is often put in the rhs (i.e. added to $h$) so that 
the Stokes matrix ${\cal A}$ retains the same structure as in the incompressible case. This is indeed 
how it is implemented in ASPECT (check?). This however requires more work since the rhs depends 
on the solution and some iterations are then needed. 

In the case of a compressible flow the strain rate tensor and the deviatoric strain 
rate tensor are no more equal (since ${\vec\nabla}\cdot\vec{\upnu} \neq 0$).
The deviatoric strainrate tensor is given by\footnote{See the ASPECT manual for a 
justification of the 3 value in the denominator in 2d and 3d.} 
\[
\dot{\bm \varepsilon}^d(\vec{\upnu})=
\dot{\bm \varepsilon}(\vec{\upnu})-\frac{1}{3} \text{Tr}(\dot{\bm \varepsilon}) {\bm 1}
=\dot{\bm \varepsilon}(\vec{\upnu})-\frac{1}{3} ({\vec\nabla}\cdot \vec{\upnu}) {\bm 1}
\]
In that case:
\begin{eqnarray}
\dot{\varepsilon}_{xx}^d 
&=& \frac{\partial u}{\partial x}
-\frac{1}{3} \left( \frac{\partial u}{\partial x} + \frac{\partial v}{\partial y} \right) 
= \frac{2}{3}\frac{\partial u}{\partial x}
-\frac{1}{3} \frac{\partial v}{\partial y}
%=
%\frac{2}{3} \sum_{i=1}^4 \frac{\partial N_i}{\partial x}\;  u_i 
%-\frac{1}{3} \sum_{i=1}^4 \frac{\partial N_i}{\partial y}\;  v_i 
\\
\dot{\varepsilon}_{yy}^d 
&=& \frac{\partial v}{\partial y}
-\frac{1}{3} \left( \frac{\partial u}{\partial x} + \frac{\partial v}{\partial y} \right) 
=-\frac{1}{3} \frac{\partial u}{\partial x} 
+ \frac{2}{3} \frac{\partial v}{\partial y} 
%=-\frac{1}{3}  \sum_{i=1}^4 \frac{\partial N_i}{\partial x}\;  u_i
%+ \frac{2}{3} \sum_{i=1}^4 \frac{\partial N_i}{\partial y}\;  v_i
\\
\dot{\varepsilon}_{xy}^d 
&=& 
\frac12 \left( \frac{\partial u}{\partial y} +\frac{\partial v}{\partial x}  \right)
%= \sum_{i=1}^4 \frac{\partial N_i}{\partial y}\;  u_i
%+ \sum_{i=1}^4 \frac{\partial N_i}{\partial x}\;  v_i
\end{eqnarray}
and then 
\[
\dot{\bm \varepsilon}^d(\vec{\upnu})
=
\left(
\begin{array}{cc}
\frac{2}{3} \frac{\partial u}{\partial x} -\frac{1}{3} \frac{\partial v}{\partial y} &
\frac{1}{2}\frac{\partial u}{\partial y} + \frac{1}{2}\frac{\partial v}{\partial x}  \\ \\
\frac{1}{2}\frac{\partial u}{\partial y} + \frac{1}{2}\frac{\partial v}{\partial x}  &
-\frac{1}{3} \frac{\partial u}{\partial x} +\frac{2}{3} \frac{\partial v}{\partial y} 
\end{array}
\right)
\]

From $\vec{\tau} = 2\eta \vec{\epsilon}^d$ we arrive at:
\[
\left(
\begin{array}{c}
\tau_{xx}\\
\tau_{yy}\\
\tau_{xy}\\
\end{array}
\right)
=
2\eta
\left(
\begin{array}{c}
\dot{\epsilon}_{xx}^d \\
\dot{\epsilon}_{yy}^d \\
\dot{\epsilon}_{xy}^d 
\end{array}
\right)
=2 \eta
\left(
\begin{array}{ccc}
2/3 & -1/3& 0 \\
-1/3 & 2/3 & 0 \\
0 & 0 & 1/2 \\
\end{array}
\right)
\cdot 
\left(
\begin{array}{c}
\frac{\partial u}{\partial x} \\ 
\frac{\partial v}{\partial y} \\ 
\frac{\partial u}{\partial y}\! +\! \frac{\partial v}{\partial x} \\
\end{array}
\right)
=
\eta
\left(
\begin{array}{ccc}
4/3 & -2/3& 0 \\
-2/3 & 4/3 & 0 \\
0 & 0 & 1 \\
\end{array}
\right)
\cdot 
\left(
\begin{array}{c}
\frac{\partial u}{\partial x} \\ 
\frac{\partial v}{\partial y} \\ 
\frac{\partial u}{\partial y}\! +\! \frac{\partial v}{\partial x} \\
\end{array}
\right)
\]
or, 
\[
\vec{\tau} = {\bm C}_\eta\cdot {\bm B} \cdot \vec{\cal V}
\]


\paragraph{Implementation} 
All terms of the linear system are common many previous stones and described in 
the {\tt manual.pdf}, but not $\Z_e$.
In practice it is computed as follows for the $Q_1\times P_0$:

\begin{lstlisting}
for iel in range(0, nel):
    Z_el=np.zeros((m_P,m_V*ndof_V),dtype=np.float64)
    for iq in range(0,nq_per_dim):
        for jq in range(0,nq_per_dim):
            rq=qcoords[iq]
            sq=qcoords[jq]
            weightq=qweights[iq]*qweights[jq]
            [...]
            drhodxq=np.dot(dNdx_V,rho_nodal[icon_V[:,iel]])
            drhodyq=np.dot(dNdy_V,rho_nodal[icon_V[:,iel]])
            for i in range(0,m_V):
                Z_el[0,ndof_V*i  ]-=N_V[i]*drhodxq/rhoq *JxWq 
                Z_el[0,ndof_V*i+1]-=N_V[i]*drhodyq/rhoq *JxWq 
\end{lstlisting}

Note that the exact density gradient could have been computed from $\rho(x,y)$
and prescribed at the quadrature points. 
In any case the term we need to integrate is not a polynomial so that 
the employed quadrature rule is not adequate.

In the case $Q_2\times Q_1$ elements are used, we need to be  
bit more careful. Let $q$ be a pressure test function.
The term under consideration is given by
\[
\int q \frac{1}{\rho}  \vec\upnu \cdot \vec\nabla \rho dV
= 
\int q \frac{1}{\rho}  \left( u \frac{\partial \rho}{\partial x} + v \frac{\partial \rho}{\partial y}\right) dV
\]
We have $q= \vec{\bN}_p^T \cdot \vec{\cal Q} =  \vec{\cal Q}^T \cdot \vec{\bN}_p$ so that
\[
\vec{\cal Q}^T \cdot \int \frac{1}{\rho} \vec{\bN}_p  
\left( u \frac{\partial \rho}{\partial x} + v \frac{\partial \rho}{\partial y}\right) dV
\]
Because $u = \sum_i \bN_i u_i$ and $v = \sum_i \bN_i v_i$ the term in parenthesis can be written
\[
\left( u \frac{\partial \rho}{\partial x} + v \frac{\partial \rho}{\partial y}\right) 
=
\left( 
\bN_1 \frac{\partial \rho}{\partial x} , \bN_1 \frac{\partial \rho}{\partial y},
\bN_2 \frac{\partial \rho}{\partial x} , \bN_2 \frac{\partial \rho}{\partial y},
...
\bN_{m_V} \frac{\partial \rho}{\partial x} , \bN_{m_V} \frac{\partial \rho}{\partial y} \right)
\cdot 
\vec{\cal V}
\]
where $\vec{\cal V}^T=(u_1,v_1,u_2,v_2...)$
Then 
\[
\vec{\cal Q}^T \cdot 
\left[\int \frac{1}{\rho} \vec{\bN}_p  
\left( 
\bN_1 \frac{\partial \rho}{\partial x} , \bN_1 \frac{\partial \rho}{\partial y},
\bN_2 \frac{\partial \rho}{\partial x} , \bN_2 \frac{\partial \rho}{\partial y},
...
\bN_{m_V} \frac{\partial \rho}{\partial x} , \bN_{m_V} \frac{\partial \rho}{\partial y} \right)
dV 
\right]
\cdot 
\vec{\cal V}
\]
which yields the matrix $\Z$ (of size $m_P \times ndof_V*m_V$ for each element):
\[
\Z = \int \frac{1}{\rho} \vec{\bN}_p  
\left( 
\bN_1 \frac{\partial \rho}{\partial x} , \bN_1 \frac{\partial \rho}{\partial y},
\bN_2 \frac{\partial \rho}{\partial x} , \bN_2 \frac{\partial \rho}{\partial y},
...
\bN_{m_V} \frac{\partial \rho}{\partial x} , \bN_{m_V} \frac{\partial \rho}{\partial y} \right)
dV 
\]
This is then implemented as follows:

\begin{lstlisting}
for iel in range(0, nel):
    Z_el=np.zeros((m_P,m_V*ndof_V),dtype=np.float64)
    for iq in range(0,nq_per_dim):
        for jq in range(0,nq_per_dim):
            rq=qcoords[iq]
            sq=qcoords[jq]
            weightq=qweights[iq]*qweights[jq]
            [...]
            drhodxq=np.dot(dNdx_V,rho_nodal[icon_V[:,iel]])
            drhodyq=np.dot(dNdy_V,rho_nodal[icon_V[:,iel]])
            for i in range(0,m_V):
                for j in range(0,m_P):
                    Z_el[j,ndof_V*i  ]-=N_P[j]*N_V[i]*drhodxq/rhoq *JxWq 
                    Z_el[j,ndof_V*i+1]-=N_P[j]*N_V[i]*drhodyq/rhoq *JxWq 
\end{lstlisting}
Of course this implementation also works for the $Q_1\times P_0$ element.

Finally note that if a Lagrange multiplier technique is not used 
in the case of the $Q_1\times P_0$ element the checkerboard modes are 
extremely large and also derail the velocity solution. 
The \lstinline{pnormalise} controls this. 
I found it was not needed in the $Q_2\times Q_1$ case so that it is 
not implemented and a postprocessing step insures that the 
average pressure is zero. 

As usual for discontinuous pressure elements I project the pressure $p$
field onto the V-nodes to yield the $q$ pressure field. 
In what follows the discretisation errors for both $p$ and $q$ is measured.

$2^2$ quadrature points are used for the $Q_1\times P_0$ element
while $3^2$ quadrature points are used for the $Q_2\times Q_1$ element (I have
tried $4^2$ and results are virtually unchanged).

\newpage
%%%%%%%%%%%%%%%%%%%%%%%%%%%%%%%%%%%%%%%%%%%%%%%%%%%%%%%%%%%%%%%%%%%%%%%%%%%%%%%%%%%%%%%%
\section*{Results}

%\subsection*{Incompressible case}

In this section we first check that the code works as expected with 
both $Q_1\times P_0$ and $Q_2\times Q_1$ elements.
We therefore run the (incompressible) Donea \& Huerta manufactured solution and indeed
recover the expected rates:

\begin{center}
\includegraphics[width=8cm]{python_codes/fieldstone_23/RESULTS/bench0/errors_q1p0.pdf}
\includegraphics[width=8cm]{python_codes/fieldstone_23/RESULTS/bench0/errors_q2q1.pdf}
\end{center}


\newpage
In order to test our implementation we have created five manufactured solutions.
Note that contrarily to previous cases using odd number of elements in 
each direction still yields very strong checkerboard modes.

Conclusions of the following 5 experiments:
\begin{itemize}
\item $Q_1\times P_0$ element: pressure field showcases a very large checkerboard mode and 
proves then unusable. The velocity error does however converge quadratically as expected.
There also seems to be a very big difference between even and odd numbers of elements 
in the amplitude of the checkerboard mode. 
\item $Q_2\times Q_1$ element: both velocity and pressure errors converge with the 
expected rates.
\end{itemize}


%--------------------------------------------------------------------
\subsection*{benchmark \#1 ({\tt ibench=1})}

Starting from a density field given by
$\rho(x,y) = xy$, we postulate the following velocity field:
\begin{eqnarray}
u(x,y) &=& \frac{C_x}{x} \nn\\
v(x,y) &=& \frac{C_y}{y} \nn
\end{eqnarray}
We then confirm that 
\[
\vec\nabla \cdot (\rho \vec\upnu) 
= \rho \vec\nabla \cdot \vec\upnu
+ \vec\upnu \cdot \vec\nabla \rho
= xy \left(\frac{\partial u}{\partial x}+\frac{\partial v}{\partial y} \right)
+ \frac{C_x}{x} \frac{\partial \rho}{\partial x}
+ \frac{C_y}{y} \frac{\partial \rho}{\partial y}
= xy \left( -\frac{C_x}{x^2} - \frac{C_y}{y^2}  \right)
+ \frac{C_x y}{x} + \frac{C_y x}{y}=0
\]
From the definition of the velocity we can compute the strain rate and deviatoric strain rate tensors:

\begin{eqnarray}
\dot{\bm \varepsilon} 
&=& 
\begin{pmatrix}
-C_x/x^ 2 & 0 \\
0 & -C_y/y^2
\end{pmatrix} \nn\\
\dot{\bm \varepsilon}^d 
&=& \dot{\bm \varepsilon} -\frac13  (\vec\nabla \cdot \vec\upnu) {\bm 1} \nn\\
&=& \dot{\bm \varepsilon} -\frac13  \left( -\frac{C_x}{x^2} - \frac{C_y}{y^2} \right) {\bm 1} \nn\\
&=&\dot{\bm \varepsilon} +\frac13  \left( \frac{C_x}{x^2} + \frac{C_y}{y^2}  \right) {\bm 1} \nn\\
&=&
\begin{pmatrix}
-\frac{2C_x}{3x^2} + \frac{C_y}{3y^2} & 0 \\
0 & \frac{C_x}{3x^2} - \frac{2C_y}{3y^2}
\end{pmatrix} \nn
\end{eqnarray}

We further postulate $g_x(x,y) = \frac{1}{x}$ and $g_y(x,y) = \frac{1}{y}$ 
so that 
\[
\rho \vec{g} = xy 
\begin{pmatrix}
1/x \\ 1/y
\end{pmatrix}
=
\begin{pmatrix}
 y\\ x
\end{pmatrix}
\]

Assuming the viscosity to be constant ($\eta=1$), the momentum conservation equation is then
\begin{eqnarray}
\vec{0}&=& -\vec\nabla p + \vec\nabla \cdot (2  \dot{\bm \varepsilon}^d )+ \rho \vec{g} \nn\\
&=& 
\begin{pmatrix} 
-\frac{\partial p}{\partial x} \\
-\frac{\partial p}{\partial y} 
\end{pmatrix} 
+    \vec\nabla \cdot  2 
\begin{pmatrix}
-\frac{2C_x}{3x^2} + \frac{C_y}{3y^2} & 0 \\
0 & \frac{C_x}{3x^2} - \frac{2C_y}{3y^2}
\end{pmatrix}
+ 
\begin{pmatrix}
 y \\ x
\end{pmatrix} \nn\\
&=&
\begin{pmatrix} 
-\frac{\partial p}{\partial x} +  \frac{\partial}{\partial x} (-\frac{4C_x}{3x^2} +\frac{2C_y}{3y^2})+ y \\
-\frac{\partial p}{\partial y} +  \frac{\partial}{\partial y} (\frac{2C_x}{3x^2} -\frac{4C_y}{3y^2})+ x 
\end{pmatrix} \nn\\
&=& 
\begin{pmatrix} 
-\frac{\partial p}{\partial x} +  \frac{\partial}{\partial x} (-\frac{4C_x}{3x^2})+ y \\
-\frac{\partial p}{\partial y} +  \frac{\partial}{\partial y} (-\frac{4C_y}{3y^2})+ x 
\end{pmatrix} 
\end{eqnarray}
The first line leads to write 
\[
p(x,y) = -\frac{4 C_x}{3x^2}   + yx + f(y)
\]
Inserting this in the second line:
\[
-\frac{\partial }{\partial y}  \left(  yx + f(y) \right) 
+  \frac{\partial}{\partial y} \left(-\frac{4C_y}{3y^2} \right)+ x = 0
\]
\[
-x - f'(y) +  \frac{\partial}{\partial y} \left(-\frac{4C_y}{3y^2} \right)+ x = 0
\]
or, 
\[
f(y) =   -\frac{4  C_y}{3y^2} + C_0
\]
and finally
\[
p(x,y) =  -\frac{4  C_x}{3x^2}  -\frac{4 C_y}{3y^2}   + yx    + C_0
\]
Requiring $\iint_\square p \; dxdy=0$ where $\square=[1:2]\times [1:2]$ (because of the 
expressions for velocity and gravity the origin of the axis system cannot be included in the 
domain) leads to 
\begin{eqnarray}
\iint_\square  
\left( -\frac{4  C_x}{3x^2}  -\frac{4  C_y}{3y^2}   + yx    + C_0 \right) 
&=& 
\iint_\square   \left( -\frac{4  C_x}{3x^2}  -\frac{4 C_y}{3y^2}   + yx  \right) dxdy
+\iint_\square   Const \; dx dy \nn\\
&=&\iint_\square   \left( -\frac{4  C_x}{3x^2}  -\frac{4  C_y}{3y^2}   + yx  \right) dxdy
+ Const
\end{eqnarray}
so that 
\[
Const = -\int_1^2 \int_1^2  \left( -\frac{4  C_x}{3x^2}  -\frac{4  C_y}{3y^2}   + yx  \right) dxdy
\]
We then set $C_x=C_y=1$ for simplicity so that we find\footnote{Thank you WolframAlpha}:
\[
Const= -11/12
\]
and in the end
\[
p(x,y) = -\frac{4}{3x^2} -\frac{4}{3y^2} + yx -\frac{11}{12}
\]

\begin{center}
\includegraphics[width=8cm]{python_codes/fieldstone_23/RESULTS/bench1/errors_q1p0.pdf}
\includegraphics[width=8cm]{python_codes/fieldstone_23/RESULTS/bench1/errors_q2q1.pdf}
\end{center}

\newpage
%------------------------------------------------ 
\subsection*{benchmark \#2 ({\tt ibench=2})}
Starting from the density field
$\rho = \cos(x)\cos(y)$ we postulate the velocity field:
\begin{eqnarray}
u(x,y) &=& \frac{C_x}{\cos(x)}  \nn\\
v(x,y) &=& \frac{C_y}{\cos(y)}
\end{eqnarray}
with $g_x = \frac{1}{\cos(y)}$ and $g_y = \frac{1}{\cos(x)}$.
We choose $\eta=1$ and $C_x=C_y=1$. 
We then confirm that 
\[
\vec\nabla \cdot (\rho \vec\upnu) 
=\vec\nabla \cdot (\cos(y), \cos(x))=0
\]
The strain rate tensor components are then given by
\begin{eqnarray}
\dot{\epsilon}_{xx} &=&  \frac{\sin(x)}{\cos^2(x)} \nn\\
\dot{\epsilon}_{yy} &=&  \frac{\sin(y)}{\cos^2(y)} \nn\\
\dot{\epsilon}_{xy} &=& 0 
\end{eqnarray}
and the deviatoric tensor follows:
\begin{eqnarray}
\dot{\bm \varepsilon}^d 
&=& \dot{\bm \varepsilon} -\frac13 (\vec\nabla\cdot \vec\upnu) {\bm 1} \nn\\
&=& 
\begin{pmatrix}
 \frac{\sin(x)}{\cos^2(x)} & 0 \\
0 & \frac{\sin(y)}{\cos^2(y)} 
\end{pmatrix}
-\frac13 \left(\frac{\sin(x)}{\cos^2(x)}+\frac{\sin(y)}{\cos^2(y)}\right){\bm 1} \nn\\
&=& 
\begin{pmatrix}
\frac23 \frac{\sin(x)}{\cos^2(x)} -\frac13 \frac{\sin(y)}{\cos^2(y)}   & 0 \\
0 &-\frac13 \frac{\sin(x)}{\cos^2(x)}+  \frac23 \frac{\sin(y)}{\cos^2(y)} 
\end{pmatrix}
\nn\\
\rho \vec{g} &=& \cos(x)\cos(y)
\begin{pmatrix}
\frac{1}{\cos(y)} \\
\frac{1}{\cos(x)}
\end{pmatrix} \nn\\
&=& 
\begin{pmatrix}
\cos(x) \\
\cos(y)
\end{pmatrix} \nn
\end{eqnarray}
Let us consider the $x$-component of the momentum equation:
\[
-\frac{\partial p}{\partial x} + 2 \eta \left( \partial_x \dot{\epsilon}_{xx}^d +
\partial_y \dot{\epsilon}_{xy}^d \right) + \cos(x) = 0
\]
\[
-\frac{\partial p}{\partial x} + 2 \frac{\partial}{\partial x} 
\left( \frac23 \frac{\sin(x)}{\cos^2(x)} -\frac13 \frac{\sin(y)}{\cos^2(y)} \right)
 + \cos(x) = 0
\]
\[
\frac{\partial}{\partial x} 
\left( -p +\frac43 \frac{\sin(x)}{\cos^2(x)} -\frac23 \frac{\sin(y)}{\cos^2(y)} \right)
 + \cos(x) = 0
\]
We then integrate with respect to $x$:
\[
-p +\frac43 \frac{\sin(x)}{\cos^2(x)} -\frac23 \frac{\sin(y)}{\cos^2(y)} 
+ \sin x + f(y) =0
\]
where $f(y)$ is a function of $y$ only, or:
\[
p(x,y) =
\frac43 \frac{\sin(x)}{\cos^2(x)} -\frac23 \frac{\sin(y)}{\cos^2(y)} 
+ \sin x + f(y) 
\]
We now turn to the $y$-component of the momentum equation:
\[
-\frac{\partial p}{\partial y} + 2 \eta \left( \partial_x \dot{\epsilon}_{xy}^d +
\partial_y \dot{\epsilon}_{yy}^d \right) + \cos(y) = 0
\]
\[
-\frac{\partial p}{\partial y} + 2\partial_y \dot{\epsilon}_{yy}^d + \cos(y) =0
\]
\[
-\frac{\partial p}{\partial y} +  \frac{\partial }{\partial y}
\left( -\frac23 \frac{\sin(x)}{\cos^2(x)}+  \frac43 \frac{\sin(y)}{\cos^2(y)} \right)
+ \cos(y) =0
\]
and now insert the obtained pressure:
\[
-\frac{\partial }{\partial y} 
\left( \frac43 \frac{\sin(x)}{\cos^2(x)} -\frac23 \frac{\sin(y)}{\cos^2(y)} + \sin x + f(y)   \right)
+ 
\frac{\partial }{\partial y}
 \left( -\frac23 \frac{\sin(x)}{\cos^2(x)}+  \frac43 \frac{\sin(y)}{\cos^2(y)} \right)
+ \cos(y) =0
\]
\[
\frac{\partial }{\partial y} 
\left( 
\frac23 \frac{\sin(y)}{\cos^2(y)} 
+ f(y) 
+ \frac43 \frac{\sin(y)}{\cos^2(y)} \right)
+ \cos(y) =0
\]
\[
\frac{\partial }{\partial y} \left(   f(y) + \frac63 \frac{\sin(y)}{\cos^2(y)}   \right) + \cos(y) = 0 
\]
or, 
\[
f(y)=\sin (y) + \frac63 \frac{\sin(y)}{\cos^2(y)}  +C_0
\]
This leads us to a pressure field given by: 
\begin{eqnarray}
p(x,y) 
&=&  \frac43 \frac{\sin(x)}{\cos^2(x)} -\frac23 \frac{\sin(y)}{\cos^2(y)} + \sin x + f(y) \nn\\
&=&  \frac43 \frac{\sin(x)}{\cos^2(x)} -\frac23 \frac{\sin(y)}{\cos^2(y)} + \sin x + 
\sin (y) + \frac63 \frac{\sin(y)}{\cos^2(y)}  +C_0 \nn\\
&=&  \frac43 \frac{\sin(x)}{\cos^2(x)} +\frac43 \frac{\sin(y)}{\cos^2(y)} + \sin x + \sin (y)   +C_0 \nn
\end{eqnarray}
%or, more generally:
%\begin{equation}
%p 
%=  \eta \Bigg(\frac{4C_x \sin(x)}{3\cos^2(x)} + \frac{4C_y \sin(y)}{3\cos^2(y)}\Bigg) 
%+( \sin(x) + \sin(y) ) + C_0
%\end{equation}

The domain is the unit square $[0:1]\times[0:1]$ and we obtain 
$C_0$ as before\footnote{\url{https://www.wolframalpha.com/}} and obtain
\[
C_0 = -\iint_\square 
\left(\frac43 \frac{\sin(x)}{\cos^2(x)} +\frac43 \frac{\sin(y)}{\cos^2(y)} + \sin x + \sin (y) \right) dxdy
= -(2 - 2 \cos(1) + 8/3 (\frac{1}{\cos (1)} - 1))
\simeq -3.18823730
\]



\begin{center}
\includegraphics[width=8cm]{python_codes/fieldstone_23/RESULTS/bench2/errors_q1p0.pdf}
\includegraphics[width=8cm]{python_codes/fieldstone_23/RESULTS/bench2/errors_q2q1.pdf}
\end{center}





\newpage
%---------------------------------------------
\subsection*{benchmark \#3 ({\tt ibench=3}) - 1D Cartesian Linear}

The fields are as follows:
\begin{eqnarray}
g_x &=& -1   \nn\\
g_y &=& 0   \nn\\
\rho &=& x   \nn\\
u &=& 1/x   \nn\\
v &=& 0    \nn\\
p &=& -4/(3x^2) -x^2/2 + 11/6 \nn\\
\dot{\varepsilon}_{xx} &=& -1/x^2  \nn\\
\dot{\varepsilon}_{yy} &=& 0  \nn\\
\dot{\varepsilon}_{xy} &=& 0  \nn
\end{eqnarray}
We then confirm that 
\[
\vec\nabla \cdot (\rho \vec\upnu) 
=\vec\nabla \cdot (1,0)=0
\]



\begin{eqnarray}
\dot{\bm \varepsilon}^d 
&=& \dot{\bm \varepsilon} -\frac13 (\vec\nabla\cdot \vec\upnu) {\bm 1} \nn\\
&=& 
\begin{pmatrix}
-1/x^2 & 0 \\
0 & 0
\end{pmatrix}
-\frac13 \left(  -1/x^2  \right){\bm 1} \nn\\
&=& 
\begin{pmatrix}
-\frac{2}{3x^2} & 0 \\
0 & \frac{1}{3x^2}
\end{pmatrix}
\end{eqnarray}

Then, setting $\eta=1$
\[
-\partial_x p + 2 \partial_x(-\frac{2}{3x^2} ) -x =0
\]
\[
-\partial_x p +  \partial_x(-\frac{4}{3x^2} ) -x =0
\]
leads to 
\[
p = -\frac{4}{3x^2} -x^2/2 + f(y)
\]
We turn to the $x$-component of the momentum equation:
\[
-\partial_y p + 2 \partial_y(\frac{1}{3x^2} )  =0
\]
\[
-\partial_y ( -\frac{4}{3x^2} -x^2/2 + f(y)  ) + 2 \partial_y(\frac{1}{3x^2} )  =0
\]
or $-\partial_y f =0$, i.e. $f(y)=C_0$ and 
\[
p = -\frac{4}{3x^2} -x^2/2 + C_0 
\]
The final constant is computed by requiring $\iint_\square p(x,y) dx dy=0$ on $[1:2]\times[1:2]$.
We find
\[
0=\iint_\square p(x,y) dx dy
=\int_0^1\int_0^1 (  -\frac{4}{3x^2} -x^2/2 + C_0 ) dx dy 
=\int_0^1\int_0^1 (  -\frac{4}{3x^2} -x^2/2  ) dx dy + C_0
\]
leading to $C_0= 11/6$.


\begin{center}
\includegraphics[width=8cm]{python_codes/fieldstone_23/RESULTS/bench3/errors_q1p0.pdf}
\includegraphics[width=8cm]{python_codes/fieldstone_23/RESULTS/bench3/errors_q2q1.pdf}
\end{center}

\newpage
%---------------------------------------------
\subsection*{benchmark \#4 ({\tt ibench=4}) - by Arie van den Berg}

The fields are as follows:
\begin{eqnarray}
g_x &=& -10   \nn\\
g_y &=& 0   \nn\\
\rho &=& 1/(1-x)   \nn\\
u &=& 1-x   \nn\\
v &=& 0    \nn\\
p &=& 10\ln (x-1) + 290.9  \nn\\
\dot{\varepsilon}_{xx} &=& -1  \nn\\
\dot{\varepsilon}_{yy} &=& 0  \nn\\
\dot{\varepsilon}_{xy} &=& 0  \nn
\end{eqnarray}
as defined on the $[20:21]\times [20:21]$ domain.
We then confirm that 
\[
\vec\nabla \cdot (\rho \vec\upnu) 
=\vec\nabla \cdot (1,0  )=0
\]


\begin{eqnarray}
\dot{\bm \varepsilon}^d 
&=& \dot{\bm \varepsilon} -\frac13 (\vec\nabla \cdot \vec\upnu) {\bm 1} \nn\\
&=& \dot{\bm \varepsilon} -\frac13 (-1) {\bm 1} \nn\\
&=& \begin{pmatrix}
-1 +1/3 & 0 \\
0 & 1/3
\end{pmatrix} \nn\\
&=& \begin{pmatrix}
-2/3 & 0 \\
0 & 1/3
\end{pmatrix} \nn
\end{eqnarray}


We further postulate $g_x(x,y) = 10$ and $g_y(x,y) = 0$.
Assuming the viscosity to be constant ($\eta=1$, the momentum conservation equation is then
\begin{eqnarray}
\vec{0}&=& -\vec\nabla p + \vec\nabla \cdot (2 \dot{\bm \varepsilon}^d )+ \rho \vec{g} \nn\\
&=& 
\begin{pmatrix} 
-\frac{\partial p}{\partial x} \\
-\frac{\partial p}{\partial y} 
\end{pmatrix} 
+  2 \eta \vec\nabla \cdot   
\begin{pmatrix}
-2/3 & 0 \\
0 & 1/3
\end{pmatrix}
+ 
\frac{1}{1-x}
\begin{pmatrix}
-10 \\ 0
\end{pmatrix} \nn\\
&=&
\begin{pmatrix} 
-\frac{\partial p}{\partial x} - \frac{10}{1-x} \\
-\frac{\partial p}{\partial y}  
\end{pmatrix} \nn
\end{eqnarray}
The second line tells us that the pressure is independent of $y$. We then integrate the  
first line:
\[
\frac{\partial p}{\partial x} = \frac{10}{x-1}
\]
which yields
\[
p(x) 
= \int \frac{10}{x-1} dx 
= 10 \log (x-1) + C_0
\]
Normalisation:
\[
0=\iint_\square p(x,y) dx dy 
= \int_{20}^{21}\int_{20}^{21} ( 10 \log (x-1) + C_0 ) dx dy
= \int_{20}^{21}\int_{20}^{21}  10 \log (x-1) dx dy + C_0
\simeq 29.7030486691745 +C_0
\]

10*log(x-1)-29.703


The velocity is a first-order polynomial so that we expect super-convergence 
of the velocity error. 


\begin{center}
\includegraphics[width=8cm]{python_codes/fieldstone_23/RESULTS/bench4/errors_q1p0.pdf}
\includegraphics[width=8cm]{python_codes/fieldstone_23/RESULTS/bench4/errors_q2q1.pdf}
\end{center}









\newpage
%---------------------------------------------
\subsection*{benchmark \#5 ({\tt ibench=5}) - 1D Cartesian Sinusoidal}

The fields are as follows:
\begin{eqnarray}
g_x &=& -1   \nn\\
g_y &=& 0   \nn\\
\rho &=& \cos(x)   \nn\\
u &=& 1/\cos(x)   \nn\\
v &=& 0    \nn\\
p &=& 4\sin(x)/(3 \cos^2 x) - \sin(x) - 0.674723  \nn\\
\dot{\varepsilon}_{xx} &=&   \nn\\
\dot{\varepsilon}_{yy} &=& 0 \nn\\
\dot{\varepsilon}_{xy} &=& 0  \nn
\end{eqnarray}
We then confirm that 
\[
\vec\nabla \cdot (\rho \vec\upnu) 
=\vec\nabla \cdot (1,0 )=0
\]




\begin{center}
\includegraphics[width=8cm]{python_codes/fieldstone_23/RESULTS/bench5/errors_q1p0.pdf}
\includegraphics[width=8cm]{python_codes/fieldstone_23/RESULTS/bench5/errors_q2q1.pdf}
\end{center}



