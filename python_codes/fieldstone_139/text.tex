\begin{flushright} {\tiny {\color{gray} python\_codes/fieldstone\_139/text.tex}} \end{flushright}

%\lstinputlisting[language=bash,basicstyle=\small]{python_codes/fieldstone_01/keywords}

\begin{center}

\fbox{\textbf{\huge \color{teal} P}}
Codes at \url{https://github.com/cedrict/fieldstone/tree/master/python_codes/fieldstone_139}
\end{center}

\par\noindent\rule{\textwidth}{0.4pt}

{\sl This stone was developed in collaboration with Rayane Meghezi}. \index{contributors}{R. Meghezi}

\par\noindent\rule{\textwidth}{0.4pt}
%%%%%%%%%%%%%%%%%%%%%%%%%%%%%%%%%%%%%%%%%%%%%%%%%%%%%%%%%%%%%%%%%%%%%%%%%%%%%%%%%%%%%%%%%%%%%%

This \stone originates in \textcite{simp17} book, chapter 9.
It investigates evolution of a one-dimensional (1D) river profile cutting into an uplifting
block of crust.
\[
\frac{\partial h}{\partial t} + \vec{\upnu}\cdot \vec{\nabla} h = -\vec\nabla \cdot (q_s \vec{n}) + w
\]
where $h$ is the elevation of the topography, $\vec\upnu=(u,v)$ is the rate of horizontal tectonic 
motion in the
$x$- and $y$-directions, $w$ is the vertical rate of uplift or subsidence, $q_s$ is the volumetric
sediment flux per unit width (with units \si{\square\meter\per\second}), and 
$\vec{n}$ is a unit vector directed down the slope of
the surface (i.e., $\vec{n} = -\vec\nabla h/S$ where $S$ is the local surface slope defined 
as $|\vec\nabla h|$).

Simpson further explains that ``The sediment flux
depends on numerous factors, most notably the local slope and the discharge of water flowing over
the surface. In situations where sediment transport is controlled by the supply of material rather
than detachment processes (e.g., as would be the case in an alluvial river), one can write the sediment
flux as
\[
q_s = c_0 S + b(q-q_c)^n S^m
\]
where $c_0$ , $b$, $n$, and $m$ are positive constants, $S$ is the local slope, 
$q$ is the local water discharge, and $q_c$ is
a critical discharge below which there is no sediment transport. This relation includes contributions
from two terms, a {\it hillslope} flux that depends linearly on the local slope (term 1) 
and a {\it fluvial} flux that depends nonlinearly on both the water discharge and the local slope (term 2).

The discharge of water flowing over the surface accumulates, for example, due to run-off from rainfall. Here,
rather than computing the discharge from the equations governing the conservation of mass and
momentum of the surface water \textcite{sica06}, we use a simple empirically 
based power law relationship between the water discharge and the upstream drainage area A,
that is
\[
q=kA^p
\]
where $k$ and $p$ are positive constants. A similar relation can be introduced for the critical discharge
\[
q_c=kA_c^p
\]
where $A_c$ is the critical drainage area below which there is no fluvial sediment transport. Substituting
equations 9.2, 9.3, and 9.4 into 9.1 leads to
\[
\frac{\partial h}{\partial t} + u \frac{\partial h}{\partial x}+ v \frac{\partial h}{\partial y}
= \vec\nabla \cdot \left[
(c_0 + c A_e^r S^{m-1}) \vec\nabla h
\right] + w
\]
where $c=bk^n$, $r=pn$ and $A_e$ is the effective drainage area defined as $A_e = A-A_c$ 
for $A>A_c$ and $A_e = 0$ for $A \le A_c$.




''
