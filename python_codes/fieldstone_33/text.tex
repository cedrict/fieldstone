This is based on the community benchmark for viscoplastic thermal convection
in a 2D square box \cite{tosn15} as already carried out in \ref{f_28}.

In this experiment the geometry is an annulus of inner radius 
$R_1=1.22$ and outer radius $R_2=2.22$. 
The rheology and buoyancy forces are identical to those of the box 
experiment. The initial temperature is now given by:
\[
T(r,\theta) = s+A\; s(1-s) \cos(N_0 \theta)
\quad\quad s=\frac{R_2-r}{R_2-R_1} \in [0,1]
\]
where $s$ in the normalised depth, $A$ is the amplitude of the perturbation and $N_0$ the 
number of lobes.

\begin{center}
\includegraphics[width=4cm]{python_codes/fieldstone_33/images/T_N03}
\includegraphics[width=4cm]{python_codes/fieldstone_33/images/T_N05}
\includegraphics[width=4cm]{python_codes/fieldstone_33/images/T_N11}\\
{\small Examples of initial temperature fields for $N_0=3,5,11$}
\end{center}

Boundary conditions can be either no-slip or free-slip on both inner and outer boundary. However, when free-slip 
is used on both a velocity null space exists and must be filtered out. In other words,
the solver may be able to come up with a solution to the Stokes operator, but 
that solution plus an arbitrary rotation is also an equally valid solution.
This additional velocity field can be problematic since it is used for advecting temperature (and/or compositions)
and it also essentially determines the time step value for a chosen mesh size (CFL condition).

For these reasons the nullspace must be removed from the obtained solution after every timestep.
There are two types of nullspace removal: removing net angular momentum, and removing net rotations.




We calculate the following output parameters: 
\begin{itemize}
\item the average temperature $<T>$
\begin{equation}
\langle T \rangle = \frac{\int_\Omega T  d\Omega }{\int_\Omega d \Omega }
=\frac{1}{V_\Omega}\int_\Omega T d\Omega
\end{equation}
\item the root mean square velocity $v_{rms}$
\begin{equation}
v_{rms} = \left ( \frac{\int_\Omega |{\bm v}|^2 \;  d \Omega}{\int_\Omega d\Omega }  \right )^{1/2}
=\left ( \frac{1}{V_\Omega} \int_\Omega |{\bm v}|^2 \;  d \Omega \right )^{1/2}
\end{equation}
\item the heat transfer through both boundaries $Q$ ({\color{red} TODO}) 
\begin{equation}
Q_{inner, outer} = R_{inner, outer} \int_\Gamma \boldsymbol{q} \cdot \boldsymbol{{n}} \; d\Gamma 
\end{equation}
\item the Nusselt number at both boundaries $Nu$  ({\color{red} TODO}) 
\begin{equation}
Nu_{inner, outer} = Q_{inner, outer} \bigg / \frac{-k \; 2 \pi}{\ln{\frac{R_{inner}}{R_{outer}}}}
\end{equation}
\item the power spectrum of the temperature field:
\begin{equation}
PS_n(T) = \left |\int_\Omega T(r, \theta) e^{in\theta} d\Omega \right |^2.
\end{equation}
\end{itemize}




\fbox{
\parbox{10cm}{{\bf features}
\begin{itemize}
\item $Q_1\times P_0$ element
\item incompressible flow
\item penalty formulation
\item Dirichlet boundary conditions
\item non-isothermal
\item non-isoviscous
\item annulus geometry
\end{itemize}
}}



TODO:

nonlinear iterations!!
