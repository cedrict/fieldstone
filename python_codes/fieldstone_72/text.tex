\lstinputlisting[language=bash,basicstyle=\small]{python_codes/fieldstone_72/keywords}

\begin{center}
Code at \url{https://github.com/cedrict/fieldstone/tree/master/python_codes/fieldstone_72}
\end{center}

\par\noindent\rule{\textwidth}{0.4pt}

%%%%%%%%%%%%%%%%%%%%%%%%%%%%%%%%%%%%%%%%%%%%%%%%%%%%%%%%%%%%%%%%%%%%%%%%%%%%%%%%%%%%%%%%%%%%%%%%%%%%

\subsection*{Manufactured solution \#1}

The analytical solution originates in Lamichhane (2017) \cite{lami17} and is 
presented in Section~\ref{ss:mms11}. 
The quadrilateral MINI element used here also originates in the same article 
and comes in two flavours with two different bubble functions $b_1$ and $b_2$
as explained in Section~\ref{ss:quadmini}.
The two bubble functions are:
\begin{eqnarray}
b_1(r,s) &=& (1-r^2)(1-s^2)(1-r)(1-s)\\
b_2(r,s) &=& (1-r^2)(1-s^2)(1+\beta(r+s))
\end{eqnarray}
The common term to both bubbles insures that the bubble is exactly zero on all four edges of the 
element. What differentiates them is the remaining term, which is bilinear ($b_1$) or linear ($b_2$). 
Both also satisfy $b_1(0,0)=b_2(0,0)=1$. The paper uses $\beta=1/4$.

The velocity and pressure fields for the benchmark are shown hereunder:
\begin{center}
\includegraphics[width=7cm]{python_codes/fieldstone_72/results/mms/vel}
\includegraphics[width=7cm]{python_codes/fieldstone_72/results/mms/p}
\end{center}

During the debugging process I ended up 
implementing various Gauss quadratures schemes, from $2^2$ to $6^2$ points but the results
show that $2^2$ quadrature points per element are sufficient since the more expensive ones
yield identical results. 
The results from the article are different than mine but I suspect that what the 
author measured could be different than what I measure (see Table 1,2 in \cite{lami17}). 
The trends are similar though, with $b_2$ performing better than $b_1$:

\begin{center}
\includegraphics[width=7cm]{python_codes/fieldstone_72/results/mms/errors_v}
\includegraphics[width=7cm]{python_codes/fieldstone_72/results/mms/errors_p}\\
{\captionfont Left: velocity error in $L_2$ norm; Right: pressure error in $L_2$ norm.\\
Resolutions from $8\times8$ until $80\times80$.}
\end{center}

The root mean square velocity is also measured for both bubble functions.
As above we see that $b_2$ performs better than $b_1$:
\begin{center}
\includegraphics[width=9cm]{python_codes/fieldstone_72/results/mms/vrms}
\end{center}

\vspace{.5cm}

\underline{Influence of mesh nodes position:} I have also repeated these 
experiments with a mesh whose internal nodes have been 
randomly moved by up to $\pm$20\% of $h_x$ or $h_y$ around the initial position. 

\begin{center}
\includegraphics[width=5cm]{python_codes/fieldstone_72/results/mms/area16}
\includegraphics[width=5cm]{python_codes/fieldstone_72/results/mms/area32}
\includegraphics[width=5cm]{python_codes/fieldstone_72/results/mms/area64}\\
{\captionfont area of elements for randomized meshes. Left to right: 16x16, 32x32, 64x64}
\end{center}


\begin{center}
\includegraphics[width=5cm]{python_codes/fieldstone_72/results/mms/errors_v_rand}
\includegraphics[width=5cm]{python_codes/fieldstone_72/results/mms/errors_p_rand}
\includegraphics[width=5cm]{python_codes/fieldstone_72/results/mms/vrms_rand}
\end{center}

We find that the error convergence rate is unchanged for velocities but is now less 
for pressure (higher than 1, lower than 1.5). 

\vspace{.5cm}
\underline{Another bubble?} 
To make a point, I have created a third bubble function which is simply $b_3(r,s)=(1-r^2)(1-s^2)$.
Technically it is zero on the sides and 1 in the middle so it fulfills the 
same requirements as the other 2 bubble functions. 
However, we see that this function is not sufficient to stabilise the element as the pressure 
showcases a typical error mode:
\begin{center}
\includegraphics[width=7cm]{python_codes/fieldstone_72/results/mms/p_b3}
\includegraphics[width=7cm]{python_codes/fieldstone_72/results/mms/p_error_b3}
\end{center}

\vspace{.5cm}
\underline{influence of $\beta$:} Finally, I use a $2\times 2$ quadrature and look at 
the errors and the vrms for various values of $\beta$:
\begin{center}
\includegraphics[width=5cm]{python_codes/fieldstone_72/results/mms/errors_v_beta}
\includegraphics[width=5cm]{python_codes/fieldstone_72/results/mms/errors_p_beta}
\includegraphics[width=5cm]{python_codes/fieldstone_72/results/mms/vrms_beta}
\end{center}
It looks like $\beta\in[0.0001,0.01]$ does better than all other higher values. Also, looking at the 
field in Paraview, no trace of the error modes as we just saw above.
The difference between 0.01 and 0.25 is somewhat small, but values of $\beta$ above 0.5 
clearly yield less accurate results. 

I can also plot these same results in a different way, i.e. placing $\beta$ on the $x-$axis:

\begin{center}
\includegraphics[width=5cm]{python_codes/fieldstone_72/results/mms/errors_v_beta2}
\includegraphics[width=5cm]{python_codes/fieldstone_72/results/mms/errors_p_beta2}
\includegraphics[width=5cm]{python_codes/fieldstone_72/results/mms/vrms_beta2}\\
{\captionfont The dotted line corresponds to $\beta=1/4$ as used in \cite{lami17}.}
\end{center}

one could conclude that $\beta=10^{-4}-10^{-2}$ 
would be a better choice but the question remains whether these
conclusions hold for other benchmarks.


\underline{Looking at the condition number of the $\K$ matrix}
I have computed the condition number of the $\K$ matrix for both bubbles and for various mesh resolutions. 
We see that the bubble 1 yields condition numbers substantially higher than bubble 2, both increasing 
quadratically with $h$. 

\begin{center}
\includegraphics[width=6cm]{python_codes/fieldstone_72/results/mms/eigenvalues/e}
\end{center}


%_____________________________________
\subsection*{Manufactured solution \#2}

This is the second mms mentioned in Lamichhane \cite{lami17}. 

\includegraphics[width=5cm]{python_codes/fieldstone_72/results/mms2/errors_v}
\includegraphics[width=5cm]{python_codes/fieldstone_72/results/mms2/errors_p}
\includegraphics[width=5cm]{python_codes/fieldstone_72/results/mms2/vrms}







\newpage
%_____________________________________
\subsection*{The SolCx benchmark}

Because of the viscosity jump at $x=L_x/2$, it has been widely reported that 
error convergence rates depend on wheter the number of element in the $x-$direction 
is even or odd.

\begin{center}
\includegraphics[width=5cm]{python_codes/fieldstone_72/results/solcx/errors_v_even}
\includegraphics[width=5cm]{python_codes/fieldstone_72/results/solcx/errors_p_even}
\includegraphics[width=5cm]{python_codes/fieldstone_72/results/solcx/vrms_even}\\
\includegraphics[width=5cm]{python_codes/fieldstone_72/results/solcx/errors_v_odd}
\includegraphics[width=5cm]{python_codes/fieldstone_72/results/solcx/errors_p_odd}
\includegraphics[width=5cm]{python_codes/fieldstone_72/results/solcx/vrms_odd}
\end{center}

Velocity error convergence rates are 2 for even numbers of elements and 1 for odd numbers, 
such as for the Q1Q1 stab. 
Pressure error convergence rate seems to be 0.5 for odd numbers of elements and
0.65 for even numbers ... 

\begin{center}
\includegraphics[width=5cm]{python_codes/fieldstone_72/results/solcx/vel}
\includegraphics[width=5cm]{python_codes/fieldstone_72/results/solcx/p}
\includegraphics[width=5cm]{python_codes/fieldstone_72/results/solcx/p_error}
\end{center}

We see that the pressure showcases an strong error (even at high resolution)
alobg the interface. 


%_____________________________________
\subsection*{The SolKz benchmark}

\begin{center}
\includegraphics[width=5cm]{python_codes/fieldstone_72/results/solkz/errors_v}
\includegraphics[width=5cm]{python_codes/fieldstone_72/results/solkz/errors_p}
\includegraphics[width=5cm]{python_codes/fieldstone_72/results/solkz/vrms}
\end{center}

%_____________________________________
\subsection*{The SolVi benchmark}

\begin{center}
\includegraphics[width=5cm]{python_codes/fieldstone_72/results/solvi/errors_v}
\includegraphics[width=5cm]{python_codes/fieldstone_72/results/solvi/errors_p}
\includegraphics[width=5cm]{python_codes/fieldstone_72/results/solvi/vrms}
\end{center}



%_____________________________________
\subsection*{The Stokes Sphere}

\begin{center}
\includegraphics[width=7cm]{python_codes/fieldstone_72/results/sphere/pstats}
\includegraphics[width=7cm]{python_codes/fieldstone_72/results/sphere/vrms}
\end{center}


\begin{center}
\includegraphics[width=7cm]{python_codes/fieldstone_72/results/sphere/vel_b1}
\includegraphics[width=7cm]{python_codes/fieldstone_72/results/sphere/vel_b2}\\
\includegraphics[width=7cm]{python_codes/fieldstone_72/results/sphere/p_b1}
\includegraphics[width=7cm]{python_codes/fieldstone_72/results/sphere/p_b2}\\
{\captionfont Left is b1, Right is b2.}
\end{center}

It looks like b2 yields more anomalous pressure modes inside the sphere. 



\newpage
%_____________________________________
\subsection*{Rayleight-Taylor instability}

\begin{center}
\includegraphics[width=7cm]{python_codes/fieldstone_72/results/RT/vy_b1}
\includegraphics[width=7cm]{python_codes/fieldstone_72/results/RT/vy_b2}
\end{center}

Bubble 2 does better than bubble 1. $\beta$ does not seem to play a role.





\newpage
%_____________________________________
\subsection*{Sinking block}
This is the very same experiment as in Stone 53. It consists of a negatively buoyant 
square object falling in a fluid in a square domain. 
This particular experiment proved to be the 'downfall' of the Q1Q1-stab element
since the stabilisation term is akin to a pressure diffusion and therefore
acts on the lithostatic pressure and also tends to smooth the effect of small 
density variations (Thieulot \& Bangerth, in prep.). 

Unless specified otherwise bubble 2 has $\beta=0.25$ by default.

%............................
\subsubsection*{Full density} 
Results indicate that the element performs adequately, especially the 
pressure field which looks smooth.  

\begin{center}
\includegraphics[width=7cm]{python_codes/fieldstone_72/results/block/full/density}
\includegraphics[width=7cm]{python_codes/fieldstone_72/results/block/full/viscosity}\\
\includegraphics[width=7cm]{python_codes/fieldstone_72/results/block/full/vel}
\includegraphics[width=7cm]{python_codes/fieldstone_72/results/block/full/p}\\
{\captionfont $64\times 64$. Viscosity ratio is 10, $\delta \rho=8$, bubble 1.}
\end{center}

We see that the element is capable of reprensenting a linear pressure profile
(the overpressure signal due to the block is negligible compared to the 
hydrostatic pressure).

\begin{center}
\includegraphics[width=10cm]{python_codes/fieldstone_72/results/block/full/plines}\\
{\captionfont Pressure profile measured at $x=L_x/2$ for various resolutions. Same parameters
as previous figure.}
\end{center}

As before we produce the characteristic figures of the velocity and pressure in the middle of the 
block as a function of the viscosity ratio and the density difference. The results obtained 
with the quadrilateral MINI element agree nicely with those obtained with the $Q_2\times Q_1$ element:
 
\begin{center}
\includegraphics[width=7cm]{python_codes/fieldstone_72/results/block/full/results_v}
\includegraphics[width=7cm]{python_codes/fieldstone_72/results/block/full/results_p}
\end{center}


\underline{Influence of $\beta$ parameter for bubble 2:} I set $\delta\rho=8$, resolution 64x64, $\eta^\star=10$,
and record the pressure profile 

\begin{center}
\includegraphics[width=10cm]{python_codes/fieldstone_72/results/block/full/betastudy/plines}
\end{center}

We can conclude that the bubble type and the value of $\beta$ do not seem to significantly affect 
the pressure profile in this case.

%............................
\subsubsection*{Reduced density} 
This is the same experiment as above but $\rho_1$ has been removed from the density
everywhere in the domain, so that the surrounding material has zero density 
and the block has a density $\delta \rho$.
The velocity and pressure field are then:

\begin{center}
\includegraphics[width=7cm]{python_codes/fieldstone_72/results/block/reduced/vel}
\includegraphics[width=7cm]{python_codes/fieldstone_72/results/block/reduced/p}\\
{\captionfont $64\times 64$. Viscosity ratio is 10, $\delta \rho=8$. bubble 1}
\end{center}

We then turn to the pressure along the vertical line $x=L_x/2$ as obtained with 
both bubble functions and find that both yield visually similar profiles:

\begin{center}
\includegraphics[width=7cm]{python_codes/fieldstone_72/results/block/reduced/plines_b1}
\includegraphics[width=7cm]{python_codes/fieldstone_72/results/block/reduced/plines_b2}\\
{\captionfont Pressure profile measured at $x=L_x/2$ for various resolutions and for both bubble functions.
Left is bubble 1, right is bubble 2.}
\end{center}

I hereafter plot the pressure profiles for both bubble functions at the highest resolution, i.e. $96\times 96$.
We see that differences are somewhat minimal, although bubble 2 yields a pressure above the block which 
showcases worrying oscillations. Also the results are remarquably similar to those obtained with the Taylor-Hood
$Q_2\times Q_1$ element:

\begin{center}
\includegraphics[width=7cm]{python_codes/fieldstone_72/results/block/reduced/plines_b12}
\includegraphics[width=7cm]{python_codes/fieldstone_72/results/block/reduced/plines_b12_zoom}
\end{center}

Finally the velocity and pressure inside the block unsurprisingly match nicely with those obtained with the Taylor-Hood
$Q_2\times Q_1$ element:

\begin{center}
\includegraphics[width=7cm]{python_codes/fieldstone_72/results/block/reduced/results_v}
\includegraphics[width=7cm]{python_codes/fieldstone_72/results/block/reduced/results_p}
\end{center}


\underline{Influence of $\beta$ parameter for bubble 2:} I set $\delta\rho=8$, resolution 64x64, $\eta^\star=10$,
and record the pressure profile 

\begin{center}
\includegraphics[width=7cm]{python_codes/fieldstone_72/results/block/reduced/betastudy/plines}
\includegraphics[width=7cm]{python_codes/fieldstone_72/results/block/reduced/betastudy/plines_zoom}
\end{center}

It is here very obvious that low values of $\beta$ yield problematic pressure oscillations. 
It looks like $\beta \geq 0.25$ actually yield near identical results to those obtained with bubble 1,
but $\beta=1$ or $\beta=2$ also yields unwanted oscillations.

