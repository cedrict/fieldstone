\lstinputlisting[language=bash,basicstyle=\small]{python_codes/fieldstone_72/keywords}

\begin{center}
Code at \url{https://github.com/cedrict/fieldstone/tree/master/python_codes/fieldstone_72}
\end{center}

\par\noindent\rule{\textwidth}{0.4pt}

%%%%%%%%%%%%%%%%%%%%%%%%%%%%%%%%%%%%%%%%%%%%%%%%%%%%%%%%%%%%%%%%%%%%%%%%%%%%%%%%%%%%%%%%%%%%%%%%%%%%

\subsubsection*{Manufactured solution}

The analytical solution originates in Lamichhane (2017) \cite{lami17} and is 
presented in Section~\ref{ss:mms11}. 
The quadrilateral MINI element used here also originates in the same article 
and comes in two flavours with two different bubble functions $b_1$ and $b_2$
as explained in Section~\ref{ss:quadmini}.
The two bubble functions are:
\begin{eqnarray}
b_1(r,s) &=& (1-r^2)(1-s^2)(1-r)(1-s)\\
b_2(r,s) &=& (1-r^2)(1-s^2)(1+\beta(r+s))
\end{eqnarray}
The common term to both bubbles insures that the bubble is exactly zero on all four edges of the 
element. What differentiates them is the remaining term, which is bilinear ($b_1$) or linear ($b_2$). 
Both also satisfy $b_1(0,0)=b_2(0,0)=1$. The paper uses $\beta=1/4$.

The velocity and pressure fields for the benchmark are shown hereunder:
\begin{center}
\includegraphics[width=7cm]{python_codes/fieldstone_72/results/mms/vel}
\includegraphics[width=7cm]{python_codes/fieldstone_72/results/mms/p}
\end{center}

During the debugging process I ended up 
implementing various Gauss quadratures schemes, from $2^2$ to $6^2$ points but the results
show that $2^2$ quadrature points per element are sufficient since the more expensive ones
yield identical results. 
The results from the article are different than mine but I suspect that what the 
author measured could be different than what I measure (see Table 1,2 in \cite{lami17}). 
The trends are similar though, with $b_2$ performing better than $b_1$:

\begin{center}
\includegraphics[width=7cm]{python_codes/fieldstone_72/results/mms/errors_v}
\includegraphics[width=7cm]{python_codes/fieldstone_72/results/mms/errors_p}\\
{\captionfont Left: velocity error in $L_2$ norm; Right: pressure error in $L_2$ norm.\\
Resolutions from $8\times8$ until $80\times80$.}
\end{center}

The root mean square velocity is also measured for both bubble functions.
As above we see that $b_2$ performs better than $b_1$:
\begin{center}
\includegraphics[width=9cm]{python_codes/fieldstone_72/results/mms/vrms}
\end{center}

\vspace{.5cm}

\underline{Influence of mesh nodes position:} I have also repeated these 
experiments with a mesh whose internal nodes have been 
randomly moved by up to $\pm$20\% of $h_x$ or $h_y$ around the initial position. 

\begin{center}
\includegraphics[width=5cm]{python_codes/fieldstone_72/results/mms/area16}
\includegraphics[width=5cm]{python_codes/fieldstone_72/results/mms/area32}
\includegraphics[width=5cm]{python_codes/fieldstone_72/results/mms/area64}\\
{\captionfont area of elements for randomized meshes. Left to right: 16x16, 32x32, 64x64}
\end{center}


\begin{center}
\includegraphics[width=5cm]{python_codes/fieldstone_72/results/mms/errors_v_rand}
\includegraphics[width=5cm]{python_codes/fieldstone_72/results/mms/errors_p_rand}
\includegraphics[width=5cm]{python_codes/fieldstone_72/results/mms/vrms_rand}
\end{center}

We find that the error convergence rate is unchanged for velocities but is now less 
for pressure (higher than 1, lower than 1.5). 

\vspace{.5cm}
\underline{Another bubble?} 
To make a point, I have created a third bubble function which is simply $b_3(r,s)=(1-r^2)(1-s^2)$.
Technically it is zero on the sides and 1 in the middle so it fulfills the 
same requirements as the other 2 bubble functions. 
However, we see that this function is not sufficient to stabilise the element as the pressure 
showcases a typical error mode:
\begin{center}
\includegraphics[width=7cm]{python_codes/fieldstone_72/results/mms/p_b3}
\includegraphics[width=7cm]{python_codes/fieldstone_72/results/mms/p_error_b3}
\end{center}

\vspace{.5cm}
\underline{influence of $\beta$:} Finally, I use a $2\times 2$ quadrature and look at 
the errors and the vrms for various values of $\beta$:
\begin{center}
\includegraphics[width=5cm]{python_codes/fieldstone_72/results/mms/errors_v_beta}
\includegraphics[width=5cm]{python_codes/fieldstone_72/results/mms/errors_p_beta}
\includegraphics[width=5cm]{python_codes/fieldstone_72/results/mms/vrms_beta}
\end{center}
It looks like $\beta\in[0.0001,0.01]$ does better than all other higher values. Also, looking at the 
field in Paraview, no trace of the error modes as we just saw above.
The difference between 0.01 and 0.25 is somewhat small, but values of $\beta$ above 0.5 
clearly yield less accurate results. 












\newpage
%_____________________________________
\subsubsection*{Sinking block}
This is the very same experiment as in Stone 53. It consists of a negatively buoyant 
square object falling in a fluid in a square domain. 

%............................
\underline{Full density} 
Results indicate that the element performs adequately, especially the 
pressure field which looks smooth.  

\begin{center}
\includegraphics[width=7cm]{python_codes/fieldstone_72/results/block/full/density}
\includegraphics[width=7cm]{python_codes/fieldstone_72/results/block/full/viscosity}\\
\includegraphics[width=7cm]{python_codes/fieldstone_72/results/block/full/vel}
\includegraphics[width=7cm]{python_codes/fieldstone_72/results/block/full/p}\\
{\captionfont $64\times 64$. Viscosity ratio is 10, $\delta \rho=8$, bubble 1.}
\end{center}

We see that the element is capable of reprensenting a linear pressure profile
(the overpressure signal due to the block is negligible compared to the 
hydrostatic pressure).

\begin{center}
\includegraphics[width=10cm]{python_codes/fieldstone_72/results/block/full/plines}\\
{\captionfont Pressure profile measured at $x=L_x/2$ for various resolutions. Same parameters
as previous figure.}
\end{center}

As before we produce the characteristic figures of the velocity and pressure in the middle of the 
block as a function of the viscosity ratio and the density difference. The results obtained 
with the quadrilateral MINI element agree nicely with those obtained with the $Q_2\times Q_1$ element:
 
\begin{center}
\includegraphics[width=7cm]{python_codes/fieldstone_72/results/block/full/results_v}
\includegraphics[width=7cm]{python_codes/fieldstone_72/results/block/full/results_p}
\end{center}


%............................
\underline{Reduced density} 
This is the same experiment as above but $\rho_1$ has been removed from the density
everywhere in the domain, so that the surrounding material has zero density 
and the block has a density $\delta \rho$.
The velocity and pressure field are then:

\begin{center}
\includegraphics[width=7cm]{python_codes/fieldstone_72/results/block/reduced/vel}
\includegraphics[width=7cm]{python_codes/fieldstone_72/results/block/reduced/p}\\
{\captionfont $64\times 64$. Viscosity ratio is 10, $\delta \rho=8$. bubble 1}
\end{center}

We then turn to the pressure along the vertical line $x=L_x/2$ as obtained with 
both bubble functions and find that both yield visually similar profiles:

\begin{center}
\includegraphics[width=7cm]{python_codes/fieldstone_72/results/block/reduced/plines_b1}
\includegraphics[width=7cm]{python_codes/fieldstone_72/results/block/reduced/plines_b2}\\
{\captionfont Pressure profile measured at $x=L_x/2$ for various resolutions and for both bubble functions.
Left is bubble 1, right is bubble 2.}
\end{center}

I hereafter plot the pressure profiles for both bubble functions at the highest resolution, i.e. $96\times 96$.
We see that differences are somewhat minimal, although bubble 2 yields a pressure above the block which 
showcases worrying oscillations. Also the results are remarquably similar to those obtained with the Taylor-Hood
$Q_2\times Q_1$ element:

\begin{center}
\includegraphics[width=7cm]{python_codes/fieldstone_72/results/block/reduced/plines_b12}
\includegraphics[width=7cm]{python_codes/fieldstone_72/results/block/reduced/plines_b12_zoom}
\end{center}

Finally the velocity and pressure inside the block unsurprisingly match nicely with those obtained with the Taylor-Hood
$Q_2\times Q_1$ element:

\begin{center}
\includegraphics[width=7cm]{python_codes/fieldstone_72/results/block/reduced/results_v}
\includegraphics[width=7cm]{python_codes/fieldstone_72/results/block/reduced/results_p}
\end{center}



