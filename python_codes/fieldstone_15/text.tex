\includegraphics[height=1.25cm]{images/pictograms/benchmark}
\includegraphics[height=1.25cm]{images/pictograms/FEM}
\includegraphics[height=1.25cm]{images/pictograms/clean}

%%%%%%%%%%%%%%%%%%%%%%%%%%%%%%%%%%%%%%%%%%%%%%%%%%%%%%%%%%%%%%%%%%%%%%%%%%%%%%%%%%%%%%%%%%%%%%%%%%%

\begin{center}
\inpython
{\small Code: \url{https://github.com/cedrict/fieldstone/tree/master/python_codes/fieldstone_15}}
\end{center}

\par\noindent\rule{\textwidth}{0.4pt}

Last revision: January 8th, 2026.

\par\noindent\rule{\textwidth}{0.4pt}
%%%%%%%%%%%%%%%%%%%%%%%%%%%%%%%%%%%%%%%%%%%%%%%%%%%%%%%%%%%%%%%%%%%%%%%%%%%%%%%%%%%%%%%%%%%%

The details of the numerical setup are presented in Section~\ref{MMM-mms1}.
The main difference resides in the Schur complement approach to solve the 
Stokes system, as presented in Section \ref{MMM-sec:solvers} (see {\bf solver\_cg}).
This iterative solver is very easy to implement once the blocks $\K$ and $\G$, 
as well as the rhs vectors $\vec{f}$ and $\vec{h}$ have been built. 

Note that the storage of all FE matrices is naive (I use full arrays) and
that sparse storage should be used for lower memory use and overall better performance.

\begin{center}
\includegraphics[width=15cm]{python_codes/fieldstone_15/RESULTS/solution.pdf}\\
{\captionfont Solution plot automatically generated with matplotlib.}
\end{center}

Rather interestingly the pressure checkerboard modes are not nearly as present as 
in \stone~\ref{f01} which uses a full matrix approach. 

Looking at the discretisation errors for velocity and pressure, we 
of course recover the same rates and values as in the full matrix case.

\begin{center}
\includegraphics[width=12cm]{python_codes/fieldstone_15/RESULTS/errors.pdf}
\end{center}

Finally, for each experiment the normalised residual (see {\bf solver\_cg}) was recorded. We see that 
all things equal the resolution has a strong influence on the number of iterations the solver must
perform to reach the required tolerance. This is one of the manifestations of the fact that the 
$Q_1 \times P_0$ element is not a stable element: the condition number of the matrix increases with 
resolution. We will see that this is not the case of stable elements such as $Q_2\times Q_1$ (
see for example \stone~147).

\begin{center}
\includegraphics[width=14cm]{python_codes/fieldstone_15/RESULTS/residual.pdf}
\end{center}

