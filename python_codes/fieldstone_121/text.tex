\begin{flushright} {\tiny {\color{gray} python\_codes/fieldstone\_121/text.tex}} \end{flushright}

%\lstinputlisting[language=bash,basicstyle=\small]{python_codes/fieldstone_01/keywords}

\begin{center}

\fbox{\textbf{\huge \color{teal} P}}
Codes at \url{https://github.com/cedrict/fieldstone/tree/master/python_codes/fieldstone_121}
\end{center}

\par\noindent\rule{\textwidth}{0.4pt}

%%%%%%%%%%%%%%%%%%%%%%%%%%%%%%%%%%%%%%%%%%%%%%%%%%%%%%%%%%%%%%%%%%%%%%%%%%%%%%%%%%%%%%%%%%%%%%%%

The domain is a 2D Cartesian box of size $L_x \times L_y$ with 
$L_x=200~\si{\km}$ and $L_y=100~\si{\km}$.
The isothermal incompressible Stokes equations are solved on a mesh 
of $nelx\times nely$ $Q_2\times Q_1$ elements (same as \aspect). 

It contains a single fluid characterised by the dislocation creep 
effective viscosity of \textcite{gatt20} (2020) and \textcite{gath21} (2021):
\[
\eta_{disl}(\dot{\varepsilon}_e,T)  = \frac{A_0}{2\dot{\varepsilon}_e}\left[ 
1 + \tanh\left( A_1 ( \log_{10}   \dot{\varepsilon}_e - A_2 )  \right)
\right]
\]
with 
\begin{align}
A_0(T) &= a_0 + b_0 T \nn\\ 
A_1(T) &= a_1 + b_1 T \nn\\ 
A_2(T) &= a_2 + b_2 T +c_2*T^2 \nn
\end{align}
and
\begin{align}
a_0 &= 4.4\cdot 10^8 \nn\\
b_0 &= -5.26\cdot 10^4 \nn\\
a_1 &= 2.11\cdot 10^{-2} \nn\\
b_1 &= 1.74\cdot 10^{-4} \nn\\
a_2 &= -41.8 \nn\\
b_2 &= 4.21\cdot 10^{-2} \nn\\
c_2 &= -1.14\cdot 10^{-5} \nn
\end{align}

The setup is similar to the one in \textcite{gupm14} (2014) although the formulation is purely 
Eulerian and will rely on periodic boundary conditions when large shear values are used.
Boundary conditions are $\vec{\upnu}=(+u_0,0)$ on the top, $\vec{\upnu}=(-u_0,0)$ on the 
bottom, and $v=0$ on the left and right boundaries. We thereby obtain a flow 
that is parallel to the $x$-axis. $u_0$ is set to $1~\si{\cm\per\year}$.

The rheology is nonlinear so Picard nonlinear iterations are implemented. These stop
when the difference between two consecutively obtained velocity fields falls below
a given tolerance $tol=10^{-3}$. 

The temperature field is set to a constant value $T_0$ throughout the domain.

If no thermal or mechanical inhomogeneity is implemented, this setup results in 
a velocity field $\vec{\upnu}=(u,v)$ that is such that $v(x,y)=0$, and 
$u(x,y)=2u_0 (y-L_y/2)/L_y$, with $\dot{\varepsilon}_{xx}=\dot{\varepsilon}_{yy}=0$ 
and $\dot{\varepsilon}_{xy}=2u_0/L_y$. 

\begin{center}
\includegraphics[width=7cm]{python_codes/fieldstone_121/results/vel}
\end{center}

We therefore need a form of weak seed/zone to localise the deformation and 
initiate the strain weakening process.





