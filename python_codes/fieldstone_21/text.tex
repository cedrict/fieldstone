\noindent
\includegraphics[height=1.25cm]{images/pictograms/aspect_logo}
\includegraphics[height=1.25cm]{images/pictograms/benchmark}
\includegraphics[height=1.25cm]{images/pictograms/FEM}
\includegraphics[height=1.25cm]{images/pictograms/paraview}
\includegraphics[height=1.25cm]{images/pictograms/publication}

%%%%%%%%%%%%%%%%%%%%%%%%%%%%%%%%%%%%%%%%%%%%%%%%%%%%%%%%%%%%%%%%%%%%%%%%%%%%%%%%%%%%%%%%%%%%%%%%%%%

\begin{flushright} {\tiny {\color{gray} python\_codes/fieldstone\_21/text.tex}} \end{flushright}

\par\noindent\rule{\textwidth}{0.4pt}

\begin{center}
\inpython
{\small Code: \url{https://github.com/cedrict/fieldstone/tree/master/python_codes/fieldstone_21}}
\end{center}

\par\noindent\rule{\textwidth}{0.4pt}

{\sl This benchmark was developed in collaboration with Prof. E.G.P. Puckett}. 
\index{contributors}{E.G.P. Puckett}

\par\noindent\rule{\textwidth}{0.4pt}

Last revision: September 7th, 2025.

\par\noindent\rule{\textwidth}{0.4pt}

%%%%%%%%%%%%%%%%%%%%%%%%%%%%%%%%%%%%%%%%%%%%%%%%%%%%%%%%%%%%%%%%%%%%%%%%%%%%%%%%%%%%%%%%%%%%%%%%%%%

The benchmark is described fully in Section~\ref{MMM-ss:anconv}.
It also found its way (albeit with some modifications) in \textcite{gadb24} (2024). 
The following results have been obtained with $k=4$.
Taylor-Hood $Q_2\times Q_1$ elements are used with an isoparametric mapping. 

For simplicity the layout of the points is borrowed from Stone~\ref{f09} which 
showcases $Q_1 \times P_0$ elements. However the points are then 'grouped' 
to form $Q_2$ elements and $Q_1$ elements. 
The used parameters $nelr$ and $nelt$ are read in and then halved. Concretely, 
when the script is ran with $nelr=10$ the real number of $Q_2$ elements 
in the radial direction is then 5. In this stone the number of elements
in the tangential direction is fixed to $nelt=12*nelr$.

In this current version of the code, I have opted not to remove the pressure nullspace
off the matrix and I simply carry out a simple post-processing which consists of 
removing the average pressure as measured on the inner boundary (or outer boundary -- it 
yields the same result) from the pressure fields $p$ and $q$ (the projection of the 
$Q_1$ pressure onto the $Q_2$ mesh). 

With regards to the mesh using $Q_2$ elements allow us to use curved edges.
All nodes are placed on a circle. An isoparametric mapping is used.
We should probably do a super-parametric mapping like in ASPECT. 

Since the viscosity is 1, we need not scale the $\G$ block.

Finally, the pressure normalisation is rather crude: I simply make sure that 
the average pressure on the inner boundary is zero.

%-----------------------------------------------
\subsection*{Results}

As expected we recover a cubic convergence for the velocity error and quadratic for the 
pressure. We also recover nearly identical results as those obtained with ASPECT (note that 
in this case $h=(R_2-R_1)/nelr=1/nelr$): 

\begin{center}
\includegraphics[width=10cm]{python_codes/fieldstone_21/RESULTS/errors.pdf}
\end{center}

Just for a change I first compute the velocity gradient tensor 
first ${\bm L}(\vec\upnu)=\vec\nabla\vec\upnu$ by means of 
the three different methods presented in Section~\ref{MMM-ss:nodderiv}.
From ${\bm L}(\vec\upnu)$ I then compute $\dot{\bm \varepsilon}(\vec\upnu)$. 
I also compute the error convergence of the nodal strain rate components 
and we recover a quadratic convergence for methods 2 and 3 while method 1 
shows an exponent of 1.5 (and is way less accurate than the other two!):

\begin{center}
\includegraphics[width=5.7cm]{python_codes/fieldstone_21/RESULTS/errors_sr1.pdf}
\includegraphics[width=5.7cm]{python_codes/fieldstone_21/RESULTS/errors_sr2.pdf}
\includegraphics[width=5.7cm]{python_codes/fieldstone_21/RESULTS/errors_sr3.pdf}\\
{\captionfont Error rates for strain rate components 
as obtained with method 1,2,3 from left to right.}
\end{center}

Also the root mean square velocity is logically found to converge to its 
expected analytical value.

\begin{center}
\includegraphics[width=8cm]{python_codes/fieldstone_21/RESULTS/vrms.pdf}
\end{center}

The pressure $p$ and its projection onto the $Q_2$ grid at 
$r=R_1$ and $r=R_2$ are plotted here under:

\begin{center}
\includegraphics[width=8cm]{python_codes/fieldstone_21/RESULTS/pressure_R1.pdf}
\includegraphics[width=8cm]{python_codes/fieldstone_21/RESULTS/pressure_R2.pdf}\\
{\captionfont Left: pressure $p$ and $q$ at $r=R_1$; Right: 
pressure $p$ and $q$ at $r=R_2$.}
\end{center}

\newpage
\begin{center}
a)\includegraphics[width=7.3cm]{python_codes/fieldstone_21/RESULTS/vel}
b)\includegraphics[width=7.3cm]{python_codes/fieldstone_21/RESULTS/q}\\
c)\includegraphics[width=7.3cm]{python_codes/fieldstone_21/RESULTS/Lxx}
d)\includegraphics[width=7.3cm]{python_codes/fieldstone_21/RESULTS/Lxy}\\
e)\includegraphics[width=7.3cm]{python_codes/fieldstone_21/RESULTS/Lyx}
f)\includegraphics[width=7.3cm]{python_codes/fieldstone_21/RESULTS/Lyy}\\
{\captionfont a) velocity field; b) pressure; 
c) $L_{xx}$; 
d) $L_{xy}$; 
e) $L_{yx}$; 
f) $L_{yy}$.}
\end{center}
