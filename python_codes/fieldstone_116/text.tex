\begin{flushright} {\tiny {\color{gray} python\_codes/fieldstone\_116/text.tex}} \end{flushright}

\lstinputlisting[language=bash,basicstyle=\small]{python_codes/fieldstone_116/keywords.ascii}

\begin{center}
\fbox{\textbf{\huge \color{teal} P}}
Codes at \url{https://github.com/cedrict/fieldstone/tree/master/python_codes/fieldstone_116}
\end{center}

\par\noindent\rule{\textwidth}{0.4pt}

%%%%%%%%%%%%%%%%%%%%%%%%%%%%%%%%%%%%%%%%%%%%%%%%%%%%%%%%%%%%%%%%%%%%%%%%%%%%%%%%%%%%%%%%%%%%%%

As we have seen in Section~\ref{XXXX} the formulation of the $\C$ matrix is substantially 
less straightforward when viscosity variations/contrasts are present in the domain. 

This stone builds on the previous \stone~115 and explores the various approaches for 
mutiple typical manufactured solutions (SolCx, SolKz, SolVi, ...).





In the case of the macro element stabilisation, an effective viscosity is computed based on the 
elemental viscosity inside the 4 elements. 


%======================================================
\subsection*{SolKz}

we start with SolKz, which showcases 6 orders of magnitude viscosity variation in the 
domain albeit in a very smooth way:
\[
\eta(x,y)=\exp (2 B y) \qquad \text{with} \qquad B=13.8155
\]




