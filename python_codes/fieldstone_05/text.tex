\lstinputlisting[language=bash,basicstyle=\small]{python_codes/fieldstone_05/keywords}

%Taken from aspect manual. 
The SolCx benchmark is intended to test the accuracy of the solution to a problem that has a large jump in the viscosity along a line through the domain. Such situations are common in geophysics: for example, the viscosity in a cold, subducting slab is much larger than in the surrounding, relatively hot mantle material.

The SolCx benchmark computes the Stokes flow field of a fluid driven by spatial density variations, subject to a spatially variable viscosity. Specifically, the domain is $\Omega = [0,1]^2$, gravity is ${\bm g} = (0,-1)^T$ and the density is given by 
\begin{equation}
\rho(x,y) = \sin(\pi y) \cos(\pi x)
\end{equation}
Boundary conditions are free slip on all of the sides of the domain and the temperature plays no role in this benchmark. 
The viscosity is prescribed as follows:
\begin{equation}
\mu(x,y) = 
\left\{
\begin{array}{lll}
1 & \text{for} & x<0.5 \\
10^6 & \text{for} & x>0.5 \\
\end{array}
\right.
\end{equation}
Note the strongly discontinuous viscosity field yields a stagnant flow 
in the right half of the domain and thereby yields a pressure discontinuity along the interface. 

The SolCx benchmark was previously showcased in Duretz et al (2011) \cite{dumg11} (references to earlier 
uses of the benchmark are available there) and its analytic solution is given in \cite{zhon96}. 
It has been carried out in Kronbichler et al (2012) \cite{krhb12} and Gerya et al (2013) \cite{gemd13}. 
Note that the source code which evaluates the velocity and pressure fields for both SolCx and SolKz is 
distributed as part of the open source package Underworld (Moresi et al, 2007 \cite{moql07}, http://underworldproject.org).

In this particular example, the viscosity is computed analytically at the quadrature points (i.e. tracers are 
not used to attribute a viscosity to the element). 
If the number of elements is even in any direction, all elements (and their associated quadrature points)
have a constant viscosity ($1$ or  $10^6$). If it is odd, then the elements situated 
at the viscosity jump have half their integration points with $\mu=1$ and half with $\mu=10^6$ 
(which is a pathological case since the used quadrature rule inside elements cannot represent 
accurately such a jump).  

\begin{center}
\includegraphics[width=7cm]{python_codes/fieldstone_05/results/errors_even.pdf}
\includegraphics[width=7cm]{python_codes/fieldstone_05/results/errors_odd.pdf}\\
{\captionfont Velocity and pressure error convergence as a function of mesh size and for various values
of the penalty parameter. Left: even number of elements in each direction; Right: odd numbers.
}
\end{center}

Because of the high viscosity in the right part of the domain, the penalty parameter should 
be high enough to insure an incompressible flow and thereby recover the expected convergence rate
(at least for the even case). Note that values higher than $\lambda=10^{10}$ yield erroneous solutions 
due to round-off errors. 

\begin{center}
\includegraphics[width=16cm]{python_codes/fieldstone_05/results/solution.pdf}
{\captionfont Various fields for 100x100 mesh}
\end{center}

Note that a fortran version of this code is available in the same folder.


