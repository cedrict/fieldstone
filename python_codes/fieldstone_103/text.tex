\includegraphics[height=1.5cm]{images/pictograms/tools}

\lstinputlisting[language=bash,basicstyle=\small]{python_codes/fieldstone_103/keywords.ascii}

\begin{center}
Code at \url{https://github.com/cedrict/fieldstone/tree/master/python_codes/fieldstone_103}
\end{center}

\par\noindent\rule{\textwidth}{0.4pt}

%%%%%%%%%%%%%%%%%%%%%%%%%%%%%%%%%%%%%%%%%%%%%%%%%%%%%%%%%%%%%%%%%%%%%%%%%%%%%%%%%%%%%%%%%%%%%%%%%%%%

\begin{center}
\includegraphics[width=1cm]{images/fortran/fortran} 
\end{center}

The topic of Conformal Mesh Refinement is introduced in Section~\ref{MMM-ss:cmr}.
This code is in Fortran for the same reason \stone~102 is. 
However, it is not nearly as complete as \stone~102: It can only refine 
above a certain height, as the different types of subdivisions are not implemented.
Furthermore, there is no FE solver in it either. It is just an example of 
simple 3D conformal refinement meshing.

The {\filenamefont prog.f90} code showcases the decomposition of one element and generates
the {\filenamefont refined.vtu} file:
\begin{center}
\includegraphics[width=4.5cm]{python_codes/fieldstone_103/visu1}
\includegraphics[width=4.5cm]{python_codes/fieldstone_103/visu2}
\includegraphics[width=4.5cm]{python_codes/fieldstone_103/visu3}
\end{center}
The internal numbering goes as follows:
\begin{center}
\includegraphics[width=7cm]{python_codes/fieldstone_103/images/numbering}\\
{\captionfont Taken from ? Numbers added by me. There are additional handwritten notes
in the images folder.}
\end{center}

In order to finish the {\filenamefont stone.f90} code, one would need to 
implement all four (+rotations) of these subdivisions:
\begin{center}
\includegraphics[width=8cm]{python_codes/fieldstone_103/images/scde}\\
{\captionfont Taken from ?}
\end{center}

Note that there is another way to subdivide an element:
\begin{center}
\includegraphics[width=5cm]{python_codes/fieldstone_103/images/habo04}\\
{\captionfont Taken from \cite{habo04}.}
\end{center}

Finally, we can run the code on a $20\times16\times12$ original mesh:
\begin{center}
\includegraphics[width=7cm]{python_codes/fieldstone_103/before}
\includegraphics[width=7cm]{python_codes/fieldstone_103/after}
\end{center}

