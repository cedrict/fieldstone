\includegraphics[height=1.25cm]{images/pictograms/replication}
\includegraphics[height=1.25cm]{images/pictograms/aspect_logo}
\includegraphics[height=1.25cm]{images/pictograms/benchmark}
\includegraphics[height=1.25cm]{images/pictograms/FEM}
\includegraphics[height=1.25cm]{images/pictograms/temperature}

%%%%%%%%%%%%%%%%%%%%%%%%%%%%%%%%%%%%%%%%%%%%%%%%%%%%%%%%%%%%%%%%%%%%%%%%%%%%%%%%%%%%%%%%%%%%%%%%%%%

%\lstinputlisting[language=bash,basicstyle=\small]{python_codes/fieldstone_38/keywords.ascii}

\begin{center}
\inpython
{\small Code: \url{https://github.com/cedrict/fieldstone/tree/master/python_codes/fieldstone_38}}
\end{center}

\par\noindent\rule{\textwidth}{0.4pt}

%%%%%%%%%%%%%%%%%%%%%%%%%%%%%%%%%%%%%%%%%%%%%%%%%%%%%%%%%%%%%%%%%%%%%%%%%%%%%%%%%%%%%%%%%%%%%%%%%%%%

This \stone is based on \textcite{mcrw74} (1974) published in J. Fluid Mech.
As for every \stone aiming at reproducing results off a publication I here include de abstract
of the article:

\begin{center}
\begin{minipage}{13cm}
{\small 
Plate tectonics provides a remarkably accurate kinematic description of the
motion of the earth's crust but a fully dynamical theory requires a n understanding 
of convection in the mantle. Thus the properties of plates and of the
mantle must be related to a systematic study of convection. This paper reviews
both the geophysical information and the fluid dynamics of convection in a
Boussinesq fluid of infinite Prandtl number. Numerical experiments have been
carried out on several simple two-dimensional models, in which convection is
driven by imposed horizontal temperature gradients or else by heating either
internally or from below. The results are presented and analysed in terms of
simple physical models. Although the computations are highly idealized and
omit variation of viscosity and other major features of mantle convection, they
can be related to geophysical measurements. I n particular, the external gravity
field depends on changes in surface elevation; this suggests an observational
means of investigating convection in the upper mantle.}
\end{minipage}
\end{center}

The domain is a 2d square of size $L$. The fluid is incompressible, and the Boussinesq 
approximation is used. Boundary conditions are free slip on all sides. $T=0$ is prescribed 
at the bottom and $T(x)=T_0 \cos (\pi x/L)$ at the top. The two sides are insulated.

Following Sato \& Thompson (1976) \cite{sath76}, 
parameters are $\kappa=1.5\cdot 10^{-6}~\si{\square\m\per\second}$, 
$\rho_0=3700~\si{\kg\per\cubic\metre}$, $\alpha=2\cdot 10^{-5}~\si{\kelvin}^{-1}$, 
$\nu=\eta/\rho_0=2\cdot10^{17}~\si{\square\metre\second}$, $g=10~\si{\metre\per\square\second}$, 
$C_p=1200~\si{\joule\per\kg\per\kelvin}$, $T_0=0.1,1,10~\si{\kelvin}$.  

We then deduce that $k=\kappa \rho_0 C_p = 6.66$. The Rayleigh number is given by
\[
\Ranb 
= \frac{ \rho_0 g \alpha T_0 L^3  }{\kappa \eta}
= \frac{  g \alpha T_0 L^3  }{\kappa \nu}
\]
In the paper three values $\Ranb=45.8,158,4580$ are used  corresponding to $T_0=0.1,1,10$. 
However there is a problem since then 
\[
L = \left( \frac{\Ranb \kappa \eta}{\rho_0 g \alpha } T_0   \right)^{1/3}
= \left( \frac{458\cdot 1.5e-6\cdot 7.4e20}{3700\cdot 10\cdot 2e-5 } 1   \right)^{1/3}
= \left( 6.87e17 \right)^{1/3}
\simeq 882.361 \si{\kilo\metre}
\]
and the vertical dimension of the box is given in \cite{mcrw74} as 700\si{\km}. 
We then find that there is effectively a factor 2 missing in the Rayleigh number in order
to reconcile the above parameters with the desired Rayleigh numbers... 
Given how very plausible $\kappa$, $C_p$, $\alpha$ and $\rho_0$ are, I choose to 
decrease the viscosity $\nu$ by a factor two.
We then have 
\[
\{ 45.733, 457.33, 4573.3 \} = \frac{ \rho_0 g \alpha  L^3  }{\kappa \eta} \{ 0.1, 1, 10\}
\]
Still not sure how/why in the paper (see caption of Fig.~4 of \cite{mcrw74}) 
they report Rayleigh numbers 45.8, 458, 4580...?

\begin{center}
\includegraphics[width=7cm]{python_codes/fieldstone_38/images/sath76_a}
\includegraphics[width=9cm]{python_codes/fieldstone_38/images/sath76_b}\\
{\captionfont Left: (a) Boundary conditions; (b) Mesh layout - the authors
use $P_2\times P_1$ elements;\\
Right: Isotherms and velocity field: (a) Steady-state conduction;
velocity field; (c) Steady-state isotherms R = 4580.}
\end{center}

The code is based on \stone~03, and because velocities are very small for low Rayleigh numbers
reduced densities are used, i.e. the buoyancy force in the Stokes equation is given by $-\alpha(T-T_0)\vec{g}$.
Steady state is reached when the velocity and temperature fields do not change by more than $tol=10^{-6}$.

Note that what is called $\Nunb$ in the code is in fact $Q=\int_0^{L_x} q_y(x) dx$.

I have also run the very same experiments with \aspect{} at meshes with uniform level 5 (32x32 
elements) and found that level 6 did not change anything to the results. 
The prm file is hereby joined to this repository.

As explained in \textcite{mcrw74}, for $\Ranb <<1$ advection of heat is negligible and $T$ is 
therefore harmonic:
\[
T(x,y) = \Delta T \cos (\pi x/L) \frac{\sinh \pi z/L}{\sinh \pi L_y/L_x}
\]
When $T_0=0.01$, only heat diffusion is present and we recover the analytical solution presented 
in McKenzie \etal:

\begin{center}
\includegraphics[width=8cm]{python_codes/fieldstone_38/RESULTS/T0_0p001_32x32/T}
\includegraphics[width=8cm]{python_codes/fieldstone_38/RESULTS/T0_0p001_32x32/T_anal}\\
{\captionfont Left: computed temperature, $48\times48$ resolution; Right: analytical solution.}
\end{center}

%%%%%%%%%%%%%%%%%%%%%%%%%%%%%%%%%%%%%%%%%%%%%%%%%%%%%%%%%%%%%%%%%%%%%%%%%%%%%%%%%%%%%%
\subsubsection*{Steady-state study}

\begin{center}
\includegraphics[width=15cm]{python_codes/fieldstone_38/RESULTS/Q.pdf}\\
\includegraphics[width=15cm]{python_codes/fieldstone_38/RESULTS/vrms.pdf}\\
{\captionfont Heat flux on top boundary (top) and root mean square velocity (bottom)
as measured with \stone~38 and \aspect on $64\times 64$ mesh.}
\end{center}

%\newpage

%\begin{center}
%\includegraphics[height=4cm]{python_codes/fieldstone_38/RESULTS/T0_0p01_64x64/vel}
%\includegraphics[height=4cm]{python_codes/fieldstone_38/RESULTS/T0_0p01_64x64/T}
%\includegraphics[height=4cm]{python_codes/fieldstone_38/RESULTS/T0_0p01_64x64/qy}\\
%\includegraphics[height=4cm]{python_codes/fieldstone_38/RESULTS/T0_0p1_64x64/vel}
%\includegraphics[height=4cm]{python_codes/fieldstone_38/RESULTS/T0_0p1_64x64/T}
%\includegraphics[height=4cm]{python_codes/fieldstone_38/RESULTS/T0_0p1_64x64/qy}\\
%...missing T0=1 row...\\
%\includegraphics[height=4cm]{python_codes/fieldstone_38/RESULTS/T0_10_64x64/vel}
%\includegraphics[height=4cm]{python_codes/fieldstone_38/RESULTS/T0_10_64x64/T}
%\includegraphics[height=4cm]{python_codes/fieldstone_38/RESULTS/T0_10_64x64/qy}\\
%\includegraphics[height=4cm]{python_codes/fieldstone_38/RESULTS/T0_100_64x64/vel}
%\includegraphics[height=4cm]{python_codes/fieldstone_38/RESULTS/T0_100_64x64/T}
%\includegraphics[height=4cm]{python_codes/fieldstone_38/RESULTS/T0_100_64x64/qy}\\
%{\captionfont Obtained with \stone~38 on $64\times64$ mesh. 
%Top to bottom: $T_0=0.01,0.1,1,10,100$}
%\end{center}


