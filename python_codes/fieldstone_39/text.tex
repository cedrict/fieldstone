The Drucker-Prager yield function is given by the function $f$:
\[
f = p \sin\phi + c\cos \phi - \tau
\]
where $\tau$ is the square root of the second invariant of the deviatoric stress.
We have
\[
p=\frac{1}{2}(\sigma_1+\sigma_3) 
\]
and 
\[
\tau = \frac{1}{2}(\sigma_1-\sigma_3)
\]
Inserting these into $f$ yields:
\[
f= \frac{1}{2}(\sigma_1+\sigma_3) \sin\phi + c \cos\phi - \frac{1}{2}(\sigma_1-\sigma_3)
\]
The yield condition $f=0$ can be reworked as follows:
\[
\sigma_1 - \frac{1 + \sin\phi}{1-\sin\phi} \sigma_3 - 2 \frac{\cos \phi}{1-\sin\phi} c = 0
\]
The third term can further be modified as follows:
\[
\frac{\cos \phi}{1-\sin\phi}
=\frac{\sqrt{1-\sin^2 \phi}}{\sqrt{(1-\sin\phi)^2}}
=\frac{\sqrt{(1-\sin \phi)(1+\sin\phi)}}{\sqrt{(1-\sin\phi)^2}}
=\sqrt{
\frac{1+\sin\phi}{1-\sin\phi}
}
\]
Finally, we define $N_\phi$ as follows 
\[
N_\phi=\frac{1+\sin \phi}{1-\sin\phi}
\]
so that the yield condition becomes:
\[
\sigma_1 - N_\phi \sigma_3 - 2 \sqrt{N_\phi} c = 0
\]
which is Eq.~3 of the article by Choi \& Petersen \cite{chpe15}.

This paper offers a solution to the problem of the angle of shear bands in 
geodynamic models. The underlying idea is based on simple modifications 
brought to existing incompressible flow codes. Note that the codes
featured in that paper also implemented elastic behaviour but this can 
be easily switched off by setting $Z=1$ in their equations.

Their plasticity implementation starts with a modification of the 
continuity equation:
\[
\vec\nabla\cdot\vec\upnu = R = 2 \sin\psi \, \dot{\varepsilon}_p
\]
where $R$ is the dilation rate, $\Psi$ is the dilation angle 
and $\dot{\varepsilon}_p$ is the square root of 
the second invariant of the plastic strain rate.

Under this assumption, the deviatoric strain rate tensor is given by
\[
\dot{\bm \varepsilon}^d(\upnu)
= \dot{\bm \varepsilon}(\upnu)- \frac{1}{2} Tr[\dot{\bm \varepsilon}(\upnu)] {\bm 1}
= \dot{\bm \varepsilon}(\upnu)- \frac{1}{2} \vec\nabla\cdot\vec\upnu \; {\bm 1}
= \dot{\bm \varepsilon}(\upnu)- \frac{1}{2} R \; {\bm 1}
\]
Turning now to the momentum conservation equation:
\begin{eqnarray}
-\vec\nabla p + \vec\nabla \cdot {\bm \tau} 
&=& -\vec\nabla p + \vec\nabla \cdot (2 \eta \dot{\bm \varepsilon}^d(\vec\upnu))  \\
&=& -\vec\nabla p + \vec\nabla \cdot \left[ 2 \eta \left(\dot{\bm \varepsilon}(\upnu)- \frac{1}{2} R \; {\bm 1}\right) \right] \\
&=& -\vec\nabla p 
+ \vec\nabla \cdot \left( 2 \eta \dot{\bm \varepsilon}(\upnu)\right) - \vec\nabla(\eta R) 
\end{eqnarray}
The last term is then an addition to the right hand side of the momentum equation 
and its weak form is as follows:
\[
\vec f' 
= \int_\Omega N_v \vec\nabla(\eta R) dV
= 2 \sin \Psi \int_\Omega N_v \vec\nabla(\eta \dot{\varepsilon}_p) dV
\]
This formulation proves to be problematic since in order to compute the gradient, we would
need the viscosity and the plastic strain rate on the mesh nodes and both these quantities
are effectively computed on the quadrature points. One option could be to project those quadrature
values onto the nodes, which may introduce interpolation errors/artefacts and/or smoothing. 
Another option is to resort to integration by parts:
\[
\int_\Omega N_v \vec\nabla(\eta \dot{\varepsilon}_p) dV
= \left[ N_v \eta \dot{\varepsilon}_p \right]_\Gamma 
-
\int_\Omega \vec\nabla N_v (\eta \dot{\varepsilon}_p) dV
\]
The last term is now trivial to compute since the shape function derivatives, the viscosity
and the plastic strain rate are known at the quadrature points. Remains the surface term. 
We will neglect it for now to simplify our implementation and note that a) it will not directly 
affect what happens inside the domain, b) it could be somewhat important when shear bands
intersect with the free surface. 

Finally, we need to define what the plastic strain rate tensor is. When using a rigid plastic 
rheology, the only deformation mechanism {\it is} plasticity so that the plastic strain rate {\it is}
the strain rate. When using a visco-plastic rheology, the plastic strain rate is the strain rate 
of the zones above/at yield (the shear bands, where the vrm is active).
 













