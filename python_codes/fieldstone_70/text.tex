\lstinputlisting[language=bash,basicstyle=\small]{python_codes/fieldstone_70/keywords}

\begin{center}
Code at \url{https://github.com/cedrict/fieldstone/tree/master/python_codes/fieldstone_70}
\end{center}

\par\noindent\rule{\textwidth}{0.4pt}

%%%%%%%%%%%%%%%%%%%%%%%%%%%%%%%%%%%%%%%%%%%%%%%%%%%%%%%%%%%%%%%%%%%%%%%%%%%%%%%%%%%%%%%%%%%%%%%%%%%%

The setup is borrowed from Duretz et al (2020) \cite{dudy20}. One major difference is 1) no elasticity,
2) no Newton solver.


The model consists of a $100\times 30$km slice of Westerly Granite (Hansen \& Carter, 1983), 
which comprises an imperfection at the center of the domain.
This weak inclusion with a 2 km radius serves to initiate
localized deformations. The model configuration is thus perfectly symmetric. 
The model includes a vertical temperature gradient ($-15\degree$C/km), 
a constant density (2700 kg/m$^3$), and the acceleration of gravity ($g_y=-10$m/s$^2$).
The inclusion is characterrised by $\eta=10^{20}\text{Pa}\cdot \text{s}$.
The domain is subjected to kinematic boundary conditions, which cause a pure shear stress
state:

\begin{center}
\includegraphics[width=12cm]{python_codes/fieldstone_70/images/fig1}\\
{\captionfont Figure taken from \cite{dudy20}}
\end{center}

All boundaries are free slip. The material can behave in a ductile manner in the lower, hot
part of the domain (displaying temperature-dependent power law creep), and in a viscoplastic 
manner in the upper, cold part of the domain. 
The parameters are 
$A=3.1623\cdot 10^{-26}Pa^{-n}s^{-1}$, $Q=186.5\cdot 10^3 J/mol$, $n=3.3$, $c=50MPa$, 
$\phi=arctan(0.6)\simeq 31\degree$ and $\eta^{vp}=10^{21}$. 

The background strainrate is set such that $\dot{\varepsilon}_{bc}=10^{-15}s^{-1}$, i.e. 
$v_{bc}=\pm \dot{\varepsilon}_{bc} L_x/2 $ on the sides and $v=\pm \dot{\varepsilon}_{bc} L_y/2 $
on the top and bottom boundaries.
Pressure is normalised such that it is on average zero on the top boundary. 
Picard iterarions are used. 

The effective viscosity at every quadrature point is computed as follows:
\[
\eta_{eff} = \left( \frac{1}{\eta_{eff,v}}  + \frac{1}{\eta_{eff,vp}}  \right)^{-1}
\]
with 
\[
\eta_{eff,v} = A^{-1/n} \dot{\varepsilon}^{-1+1/n} \exp \frac{Q}{nRT}
\]
and 
\[
\eta_{eff,vp} = \frac{p \sin \phi + c \cos \phi}{2 \dot{\varepsilon}}  + \eta_{vp}
\]
For the first nonlinear iteration there is no pre-existing velocity or pressure field so 
I then set $\dot{\varepsilon}=10^{-25}$, just so that the rheological model does not explode. 
The viscosity is also maintained between $\eta_{min}=10^{19}$ and $\eta_{max}=10^{25}$.

TODO: a) compute nonlinear residual b) remove p nullspace from matrix directly c) plot visc/plast regimes


\begin{center}
\includegraphics[width=7cm]{python_codes/fieldstone_70/u}
\includegraphics[width=7cm]{python_codes/fieldstone_70/v}\\
\includegraphics[width=7cm]{python_codes/fieldstone_70/vel}
\includegraphics[width=7cm]{python_codes/fieldstone_70/q}\\
\includegraphics[width=7cm]{python_codes/fieldstone_70/e}
\includegraphics[width=7cm]{python_codes/fieldstone_70/etaeff}\\
extension, resolution 120x40, 25 nl iterations only.
\end{center}




