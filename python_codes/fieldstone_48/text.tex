The numbering of the nodes is somewhat inconsistent: the nodes in the $Q_1$
element are numbered in a counterclockwise manner while they are numberd
left-to-right, bottom-to-top for the $Q_2$ and $Q_3$.

\begin{verbatim}
              Q1/P0          Q2/Q1           Q3/Q2
          
          2===========3   6=====7=====8  12==13==14==15
          |           |   |     |     |   |   |   |   |
          |           |   |     |     |   8===9==10==11
          |           |   3=====4=====5   |   |   |   |
          |           |   |     |     |   4===5===6===7
          |           |   |     |     |   |   |   |   |
          0===========1   0=====1=====2   0===1===2===3

          .===========.   2===========3   6=====7=====8
          |           |   |           |   |     |     |
          |           |   |           |   |     |     |
          |     0     |   |           |   3=====4=====5
          |           |   |           |   |     |     |
          |           |   |           |   |     |     |
          .===========.   0===========1   0=====1=====2       

          mV=4, mP=1      mV=9, mP=4      mV=16, mP=9  

\end{verbatim}

In the code the 'order' parameter can take values 1,2 and 3 which 
correspond to the polynomial order of the velocity approximation ($Q_1$, $Q_2$ and $Q_3$).

When both nelx and nely values have been chosen, the total number of element 
for a regular 2D grid is simply:
\begin{lstlisting}
nel=nelx*nely
\end{lstlisting}

The number of nodes in each direction is then easily computed:
\begin{lstlisting}
nnx=order*nelx+1 
nny=order*nely+1 
\end{lstlisting}
and so is then the total number of velocity nodes:
\begin{lstlisting}
NV=nnx*nny
\end{lstlisting}

The total number of pressure nodes is as follows:
\begin{lstlisting}
if order==1:
   NP=nelx*nely
if order==2:
   NP=(nelx+1)*(nely+1)
if order==3:
   NP=(2*nelx+1)*(2*nely+1)
\end{lstlisting}

Each velocity node has 2 dofs (ndofV=2) while pressure nodes have one dof (ndofP=1) so that 
the size of the blocks and the assembled FE matrix are given by:

\begin{lstlisting}
NfemV=NV*ndofV      
NfemP=NP*ndofP    
Nfem=NfemV+NfemP
\end{lstlisting}

For the linear element, 2 quadrature points per dimension are enough (nqperdim=2), 
while 3 are necessary for the quadratic element (nqperdim=3) and 4 are 
necessary for the cubic element (nqperdim=4), which can be conveniently implemented as follows:
\begin{lstlisting}
nqperdim=order+1
\end{lstlisting}

Because we wish to use a regular grid, the layout of the points for all three elements 
can be implemented easily:

\begin{lstlisting}
counter=0    
for j in range(0,nny):
    for i in range(0,nnx):
        xV[counter]=i*hx/order
        yV[counter]=j*hy/order
        counter+=1
\end{lstlisting}




