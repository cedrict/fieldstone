%\lstinputlisting[language=bash,basicstyle=\small]{python_codes/fieldstone_120/keywords}

\begin{center}
\inpython~Code at \url{https://github.com/cedrict/fieldstone/tree/master/python_codes/fieldstone_120}
\end{center}

\par\noindent\rule{\textwidth}{0.4pt}

%%%%%%%%%%%%%%%%%%%%%%%%%%%%%%%%%%%%%%%%%%%%%%%%%%%%%%%%%%%%%%%%%%%%%%%%%%%%%%%%%%%%%%%%%%%%%%%%%%%%

The idea here is to create a library of basis functions and quadrature rules, as well as 
other FE-related tools so as to be able to write much more compact FE codes. 
The boundary conditions of this exercise are:
\begin{itemize}
\item two-dimensional space
\item Cartesian domain
\item Continuous Galerkin method
\item most of commonly used element pairs for Stokes equations 
\item Dirichlet boundary conditions 
\item isoparametric mapping
\end{itemize}

There are three files which contain all the required tools to build most of a FE code:

\begin{itemize}

\item {\pythonfile FEbasis2D.py} contains 

\begin{itemize}
\item \lstinline{NNN(r,s,space)}: returns the basis functions $\bN_i (i=1,...m)$ at position $r,s$.
\item \lstinline{dNNNdr(r,s,space)}: returns basis function derivative $\partial_r\bN_i (i=1,...m)$ at position $r,s$.
\item \lstinline{dNNNds(r,s,space)}: returns basis function derivative $\partial_s\bN_i (i=1,...m)$ at position $r,s$.
\item \lstinline{NNN_r(space)}: returns the $r_i (i=1,...m)$ coordinates of the support nodes.
\item \lstinline{NNN_s(space)}: returns the $s_i (i=1,...m)$ coordinates of the support nodes.
\item \lstinline{NNN_m(space)}: returns the number of support nodes.
\item \lstinline{visualise_nodes(space)}: generates a png file with the support nodes and the shape of the reference element.
\end{itemize}


\item {\pythonfile FEquadrature.py} contains
\begin{itemize}
\item \lstinline{quadrature(space,nqpts)}: it returns the number of quadrature points in the element, 
their coordinates and associated weights. If the element is a quadrilateral then it contains 
\lstinline{nqpts}$\times$\lstinline{nqpts} quadrature points. 
If it is a triangle then it contains \lstinline{nqpts} quadrature points in total. 
In that case only 3,6,7,12,13,16 are authorized.   
\item \lstinline{qcoords_1D(nqpts)}: returns the coordinates of the \lstinline{nqpts} quadrature points between -1 and 1.
\item \lstinline{qweights_1D(nqpts)}:returns the weights of the \lstinline{nqpts} quadrature points.
\item \lstinline{visualise_quadrature_points(space,nqpts)}: generates a png file.
\end{itemize}


\item {\pythonfile FEtools.py} contains

\begin{itemize}
\item \lstinline{cartesian_mesh(Lx,Ly,nelx,nely,space)}:
\item \lstinline{export_swarm_to_ascii(x,y,filename)}:
\item \lstinline{export_swarm_scalar_to_ascii(x,y,f,filename)}:
\item \lstinline{export_swarm_vector_to_ascii(x,y,u,v,filename)}:
\item \lstinline{export_connectivity_array_to_ascii(x,y,icon,filename)}:
\item \lstinline{export_elements_to_vtu(x,y,icon,space,filename)}:
\item \lstinline{export_swarm_to_vtu(x,y,filename)}:
\item \lstinline{export_swarm_vector_to_vtu(x,y,vx,vy,filename)}:
\item \lstinline{export_swarm_scalar_to_vtu(x,y,scalar,filename)}:
\item \lstinline{bc_setup(x,y,Lx,Ly,ndof,left,right,bottom,top)}:
\item \lstinline{J(m,dNdr,dNds,x,y)}:
\item \lstinline{assemble_K(K_el,A_sparse,iconV,mV,ndofV,iel)}:
\item \lstinline{assemble_G(G_el,A_sparse,iconV,iconP,NfemV,mV,mP,ndofV,ndofP,iel)}:
\item \lstinline{assemble_f(f_el,rhs,iconV,mV,ndofV,iel)}:
\item \lstinline{apply_bc(K_el,G_el,f_el,h_el,bc_val,bc_fix,iconV,mV,ndofV,iel)}:
\end{itemize}


\end{itemize}

\newpage
%--------------------------------------------------------------------
\section*{Supported element spaces}

\begin{center}
\includegraphics[width=5cm]{python_codes/fieldstone_120/spaces/Q1_nodes}
\includegraphics[width=5cm]{python_codes/fieldstone_120/spaces/Q1+_nodes}
\includegraphics[width=5cm]{python_codes/fieldstone_120/spaces/Q2_nodes}\\
\includegraphics[width=5cm]{python_codes/fieldstone_120/spaces/Q2s_nodes}
\includegraphics[width=5cm]{python_codes/fieldstone_120/spaces/Q3_nodes}
\includegraphics[width=5cm]{python_codes/fieldstone_120/spaces/Q4_nodes}\\
\includegraphics[width=3.53cm]{python_codes/fieldstone_120/spaces/RT1_nodes}
\includegraphics[width=3.53cm]{python_codes/fieldstone_120/spaces/RT2_nodes}
\includegraphics[width=3.53cm]{python_codes/fieldstone_120/spaces/DSSY1_nodes}
\includegraphics[width=3.53cm]{python_codes/fieldstone_120/spaces/DSSY2_nodes}\\
\includegraphics[width=5cm]{python_codes/fieldstone_120/spaces/P1_nodes}
\includegraphics[width=5cm]{python_codes/fieldstone_120/spaces/P2_nodes}
\includegraphics[width=5cm]{python_codes/fieldstone_120/spaces/P3_nodes}\\
\includegraphics[width=5cm]{python_codes/fieldstone_120/spaces/P1+_nodes}
\includegraphics[width=5cm]{python_codes/fieldstone_120/spaces/P2+_nodes}
\end{center}

%-----------------------------------------------------------------------------
\section*{Finite element pairs for the Stokes equation}

\begin{itemize}
\item $Q_1\times Q_0$: see Section~\ref{ss:pairq1p0}
\item $Q_2\times Q_1$: see Section~\ref{ss:pairq2q1}
\item $Q_2\times P_{-1}$: see Section~\ref{ss:pairq2pm1}
\item $Q_3\times Q_2$: see Section~\ref{XXX}
\item $Q_1^+\times Q_1$: Quadrilateral mini, see Section~\ref{ss:quadmini}
\item $Q_2^{(8)}\times Q_1$: Serendipity, see Section~\ref{sec:serendipity2D}
\item $Q_1\times Q_0$-DSSY1: see Section~\ref{ss:dssy_2D}
\item $Q_1\times Q_0$-DSSY2: see Section~\ref{ss:dssy_2D}
\item $Q_1\times Q_0$-RT1: see Section~\ref{ss:RTq1p0}
\item $Q_1\times Q_0$-RT2: see Section~\ref{ss:RTq1p0}
\item $P_1\times P_0$: see Section~\ref{ss:}
\item $P_2\times P_1$: see Section~\ref{ss:p2p1}
\item $P_3\times P_2$: see Section~\ref{ss:}
\item $P_1^{+}\times P_{1}$: MINI, see Section~\ref{pair:mini}
\item $P_2^+\times P_{-1}$: Crouzeix-Raviart, see Section~\ref{sec:crouzeix-raviart}
\end{itemize}

ADD stabilised elements?
chinese Q2Q1 stone 52 ? only for deformed
P2/P1+P0 element ?!!?!
Q2/Q1+Q0

%-----------------------------------------------------------------------------
\section*{Breakdown of the code}

One starts by loading the required finite element functions 
for the basis functions, the numerical quadrature and various tools:
\begin{lstlisting}
import FEbasis2D as FE
import FEquadrature as Q
import FEtools as Tools 
\end{lstlisting}

The domain is a unit square:
\begin{lstlisting}
Lx=1
Ly=1
\end{lstlisting}

It is discretised by means of a $nelx\times nely$ cells. If quadrilateral 
elements are to be used then there are $nelx\times nely$ elements. If 
triangular elements are to be used then the cells are cut into two 
triangles and there are then $2\times nelx\times nely$ elements.

\begin{lstlisting}
nelx=16
nely=16
\end{lstlisting}

There are four boundaries to the domain (left, right, bottom and top). For the 
benchmark under consideration we need to impose no slip boundary conditions 
on all sides of the domain:
\begin{lstlisting}
left_bc  ='no_slip'
right_bc ='no_slip'
bottom_bc='no_slip'
top_bc   ='no_slip'
\end{lstlisting}

In two dimensions there are two velocity degrees of freedom per 
velocity node but only one pressure degree of freedom per pressure node:
\begin{lstlisting}
ndofV=2
ndofP=1
\end{lstlisting}

A finite element space must be assigned to both velocity and pressure. 
\begin{lstlisting}
Vspace='Q2'
Pspace='Q1'
\end{lstlisting}

A quadrature order must be assigned. If quadrilateral elements are used
this parameter is the number of quadrature points per dimension. 
If triangular elements are used it is the total number of quadrature points 
inside the element.  
\begin{lstlisting}
nqpts=6
\end{lstlisting}

For the chosen velocity and pressure spaces we retrieve the number of nodes 
('support points') inside an element.
\begin{lstlisting}
mV=FE.NNN_m(Vspace)
mP=FE.NNN_m(Pspace)
\end{lstlisting}

We then setup the quadrature rule for an element. This function 
returns the number of quadrature points inside the element, 
their coordinates in the $r,s$ space and their weights: 
\begin{lstlisting}
nqel,qcoords_r,qcoords_s,qweights=Q.quadrature(Vspace,nqpts)
\end{lstlisting}

The mesh is then created, or rather the meshes: one for the 
velocity nodes, one for the pressure nodes. They both count the 
same number of elements. There are \lstinline{NV} velocity nodes and their
coordinates are stored in the \lstinline{xV} and \lstinline{yV} arrays.
Likewise there are \lstinline{NP} pressure nodes and their
coordinates are stored in the \lstinline{xP} and \lstinline{yP} arrays. 

\begin{lstlisting}
NV,nel,xV,yV,iconV=Tools.cartesian_mesh(Lx,Ly,nelx,nely,Vspace)
NP,nel,xP,yP,iconP=Tools.cartesian_mesh(Lx,Ly,nelx,nely,Pspace)
\end{lstlisting}

Now that we know the number of elements and nodes we can compute the 
total number of quadrature points \lstinline{nq}, 
the total number of velocity dofs \lstinline{NfemV}, 
the total number of pressure dofs \lstinline{NfemP}, 
and the total number of dofs:

\begin{lstlisting}
nq=nqel*nel
NfemV=NV*ndofV
NfemP=NP*ndofP
Nfem=NfemV+NfemP
\end{lstlisting}

We will later need two arrays of size \lstinline{NfemV} (we are only imposing
velocity boundary conditions). \lstinline{bc_fix} is a boolean array.
We set \lstinline{bc_fix[i]=True} if the value of a given velocity dof \lstinline{i} is set. 
and the value of \lstinline{bc_val[i]} is the value of the desired boundary condition.

\begin{lstlisting}
bc_fix,bc_val=Tools.bc_setup(xV,yV,Lx,Ly,ndofV,left_bc,right_bc,bottom_bc,top_bc)
\end{lstlisting}

Then we proceed to compute the volume of each element, i.e. 
\[
V_e = \int_{\Omega_e} dV = \sum_{iq=1}^{n_q} \omega_{i_q} |J_{i_q}|
\]
which translates as follows: 
\begin{lstlisting}
for iel in range(0,nel):
  for iq in range(0,nqel):
    rq=qcoords_r[iq]
    sq=qcoords_s[iq]
    weightq=qweights[iq]
    NNNV=FE.NNN(rq,sq,Vspace)
    dNNNVdr=FE.dNNNdr(rq,sq,Vspace)
    dNNNVds=FE.dNNNds(rq,sq,Vspace)
    jcob,jcbi,dNNNVdx,dNNNVdy=Tools.J(mV,dNNNVdr,dNNNVds,xV[iconV[0:mV,iel]],yV[iconV[0:mV,iel]])
    area[iel]+=jcob*weightq
\end{lstlisting}



\newpage
%%%%%%%%%%%%%%%%%%%%%%%%%%%%%%%%%%%%%%%%%%%%%%%%%%%%%%%%%%%%%%%%%%%%%%%%%%%%%%%
\section*{Results - VJ3 mms}

%----------------------------
\subsection*{$Q_1\times Q_0$}

\begin{center}
\includegraphics[width=8cm]{python_codes/fieldstone_120/results/Q1Q0-velocity-h.pdf}
\includegraphics[width=8cm]{python_codes/fieldstone_120/results/Q1Q0-pressure-h.pdf}\\
\includegraphics[width=8cm]{python_codes/fieldstone_120/results/Q1Q0-velocity-Nfem.pdf}
\includegraphics[width=8cm]{python_codes/fieldstone_120/results/Q1Q0-pressure-Nfem.pdf}
\end{center}

%----------------------------
\subsection*{$Q_2\times Q_1$}

\begin{center}
\includegraphics[width=8cm]{python_codes/fieldstone_120/results/Q2Q1-velocity-h.pdf}
\includegraphics[width=8cm]{python_codes/fieldstone_120/results/Q2Q1-pressure-h.pdf}\\
\includegraphics[width=8cm]{python_codes/fieldstone_120/results/Q2Q1-velocity-Nfem.pdf}
\includegraphics[width=8cm]{python_codes/fieldstone_120/results/Q2Q1-pressure-Nfem.pdf}
\end{center}

higher convergence when less quadrature points are used ?!

\newpage
%----------------------------
\subsection*{$Q_3\times Q_2$}

\begin{center}
\includegraphics[width=8cm]{python_codes/fieldstone_120/results/Q3Q2-velocity-h.pdf}
\includegraphics[width=8cm]{python_codes/fieldstone_120/results/Q3Q2-pressure-h.pdf}\\
\includegraphics[width=8cm]{python_codes/fieldstone_120/results/Q3Q2-velocity-Nfem.pdf}
\includegraphics[width=8cm]{python_codes/fieldstone_120/results/Q3Q2-pressure-Nfem.pdf}
\end{center}

%----------------------------
\subsection*{$Q_1^+\times Q_1$}

\begin{center}
\includegraphics[width=8cm]{python_codes/fieldstone_120/results/Q1+Q1-velocity-h.pdf}
\includegraphics[width=8cm]{python_codes/fieldstone_120/results/Q1+Q1-pressure-h.pdf}
\includegraphics[width=8cm]{python_codes/fieldstone_120/results/Q1+Q1-velocity-Nfem.pdf}
\includegraphics[width=8cm]{python_codes/fieldstone_120/results/Q1+Q1-pressure-Nfem.pdf}
\end{center}

\newpage
%----------------------------
\subsection*{$P_1^+\times P_1$}
\begin{center}
\includegraphics[width=8cm]{python_codes/fieldstone_120/results/P1+P1-velocity-h.pdf}
\includegraphics[width=8cm]{python_codes/fieldstone_120/results/P1+P1-pressure-h.pdf}
\includegraphics[width=8cm]{python_codes/fieldstone_120/results/P1+P1-velocity-Nfem.pdf}
\includegraphics[width=8cm]{python_codes/fieldstone_120/results/P1+P1-pressure-Nfem.pdf}
\end{center}

%----------------------------
\subsection*{$P_2\times P_1$}
\begin{center}
\includegraphics[width=8cm]{python_codes/fieldstone_120/results/P2P1-velocity-h.pdf}
\includegraphics[width=8cm]{python_codes/fieldstone_120/results/P2P1-pressure-h.pdf}
\includegraphics[width=8cm]{python_codes/fieldstone_120/results/P2P1-velocity-Nfem.pdf}
\includegraphics[width=8cm]{python_codes/fieldstone_120/results/P2P1-pressure-Nfem.pdf}
\end{center}

\newpage
%----------------------------
\subsection*{$P_2^+\times P_{-1}$}
\begin{center}
\includegraphics[width=8cm]{python_codes/fieldstone_120/results/P2+P-1-velocity-h.pdf}
\includegraphics[width=8cm]{python_codes/fieldstone_120/results/P2+P-1-pressure-h.pdf}
\includegraphics[width=8cm]{python_codes/fieldstone_120/results/P2+P-1-velocity-Nfem.pdf}
\includegraphics[width=8cm]{python_codes/fieldstone_120/results/P2+P-1-pressure-Nfem.pdf}
\end{center}

%----------------------------
\subsection*{$P_3\times P_2$}
\begin{center}
\includegraphics[width=8cm]{python_codes/fieldstone_120/results/P3P2-velocity-h.pdf}
\includegraphics[width=8cm]{python_codes/fieldstone_120/results/P3P2-pressure-h.pdf}
\includegraphics[width=8cm]{python_codes/fieldstone_120/results/P3P2-velocity-Nfem.pdf}
\includegraphics[width=8cm]{python_codes/fieldstone_120/results/P3P2-pressure-Nfem.pdf}
\end{center}

\newpage
%----------------------------
\subsection*{$Q_1\times Q_0$-RT1}
\begin{center}
\includegraphics[width=8cm]{python_codes/fieldstone_120/results/RT1Q0-velocity-h.pdf}
\includegraphics[width=8cm]{python_codes/fieldstone_120/results/RT1Q0-pressure-h.pdf}
\includegraphics[width=8cm]{python_codes/fieldstone_120/results/RT1Q0-velocity-Nfem.pdf}
\includegraphics[width=8cm]{python_codes/fieldstone_120/results/RT1Q0-pressure-Nfem.pdf}
\end{center}

%----------------------------
\subsection*{$Q_1\times Q_0$-RT2}
\begin{center}
\includegraphics[width=8cm]{python_codes/fieldstone_120/results/RT2Q0-velocity-h.pdf}
\includegraphics[width=8cm]{python_codes/fieldstone_120/results/RT2Q0-pressure-h.pdf}
\includegraphics[width=8cm]{python_codes/fieldstone_120/results/RT2Q0-velocity-Nfem.pdf}
\includegraphics[width=8cm]{python_codes/fieldstone_120/results/RT2Q0-pressure-Nfem.pdf}
\end{center}

\newpage
%----------------------------
\subsection*{$Q_1\times Q_0$-DSSY1}
\begin{center}
\includegraphics[width=8cm]{python_codes/fieldstone_120/results/DSSY1Q0-velocity-h.pdf}
\includegraphics[width=8cm]{python_codes/fieldstone_120/results/DSSY1Q0-pressure-h.pdf}
\includegraphics[width=8cm]{python_codes/fieldstone_120/results/DSSY1Q0-velocity-Nfem.pdf}
\includegraphics[width=8cm]{python_codes/fieldstone_120/results/DSSY1Q0-pressure-Nfem.pdf}
\end{center}

%----------------------------
\subsection*{$Q_1\times Q_0$-DSSY2}
\begin{center}
\includegraphics[width=8cm]{python_codes/fieldstone_120/results/DSSY2Q0-velocity-h.pdf}
\includegraphics[width=8cm]{python_codes/fieldstone_120/results/DSSY2Q0-pressure-h.pdf}
\includegraphics[width=8cm]{python_codes/fieldstone_120/results/DSSY2Q0-velocity-Nfem.pdf}
\includegraphics[width=8cm]{python_codes/fieldstone_120/results/DSSY2Q0-pressure-Nfem.pdf}
\end{center}

\newpage
%------------------------------
\subsection*{Altogether}


\begin{center}
\includegraphics[width=13.5cm]{python_codes/fieldstone_120/results/errors-velocity-all}\\
\includegraphics[width=13.5cm]{python_codes/fieldstone_120/results/errors-pressure-all}\\
\includegraphics[width=13.5cm]{python_codes/fieldstone_120/results/vrms}
\end{center}


\newpage
\begin{tabular}{l|cc|cc|}
\hline
                     & d\&h &   & vj3 & \\
                     & v    & p & v & p \\
\hline
\hline
$P_1\times P_0$       &      &   &  &   \\
$P_2\times P_1$       &      &   & 3 & 2  \\
$P_3\times P_2$       &      &   & 4 & 3  \\
$P_1^+\times P_{1}$   &      &   & 2 & {\bf 1.5}  \\
$P_2^+\times P_{-1}$  &      &   & 3 & 2  \\
$Q_1\times Q_0$       & 2    & X & 2 & X  \\
$Q_2\times Q_1$       &      &   & 3 & 2  \\
$Q_3\times Q_2$       &      &   & 4 & 3  \\
$Q_1^+\times Q_1$     &      &   & 2 & {\bf 1.5}  \\
$Q_1\times Q_0$-RT1   &      &   & 2 & 1(?)  \\
$Q_1\times Q_0$-RT2   &      &   & 2 & 1  \\
$Q_1\times Q_0$-DSSY1 &      &   & 2 & 1  \\
$Q_1\times Q_0$-DSSY2 &      &   & 2 & 1  \\
\hline
\end{tabular} 

\vspace{1cm}

Conclusions:
\begin{itemize}
\item ...
\end{itemize}

\newpage
%%%%%%%%%%%%%%%%%%%%%%%%%%%%%%%%%%%%%%%%%%%%%%%%%%%%%%%%%%%%%%%%%%%%%%%%%%%%%%%
\section*{Results - square sinker}

Unit square. No-slip boundary conditions. domain filled with fluid with $\eta_f=1$ and $\rho_f=1$.
Square sinker in the middle of the domain of size $0.25\times 0.25$ with $\eta_s=100$ and
$\rho_s=1.001$. Gravity is $\vec{g}=-\vec{e}_y$. Resolutions nelx=8, 16, 32, 64, 128 
are chosen so that element boundaries align with sinker boundaries 
(material averaging is irrelevant). 

There is no analytical solution so by setting $\vec{\upnu}^{th}=\vec{0}$ and 
$p^{th}=0$ the computed errors are in fact the vrms and prms shown hereunder.

\begin{center}
\includegraphics[width=6cm]{python_codes/fieldstone_120/results_sinker/velocity}
\includegraphics[width=6cm]{python_codes/fieldstone_120/results_sinker/pressure}\\
{\captionfont Velocity and pressure field on $64\times 64$ mesh of $Q_2\times Q_1$ elements.}
\end{center}


\begin{center}
\includegraphics[width=8.5cm]{python_codes/fieldstone_120/results_sinker/errors-velocity-all}
\includegraphics[width=8.5cm]{python_codes/fieldstone_120/results_sinker/errors-velocity-subset}\\
\includegraphics[width=8.5cm]{python_codes/fieldstone_120/results_sinker/errors-pressure-all}
\includegraphics[width=8.5cm]{python_codes/fieldstone_120/results_sinker/errors-pressure-subset}\\
{\captionfont We find that the DSSY and RT elements yield very abnormal results at low resolution
so they are removed from the plots in the right column. The prms of the regular $Q_1\times Q_0$ element 
is also removed.}
\end{center}







