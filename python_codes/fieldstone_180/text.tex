\noindent
\includegraphics[height=1.25cm]{images/pictograms/benchmark}
\includegraphics[height=1.25cm]{images/pictograms/FDM}
\includegraphics[height=1.25cm]{images/pictograms/temperature}

%%%%%%%%%%%%%%%%%%%%%%%%%%%%%%%%%%%%%%%%%%%%%%%%%%%%%%%%%%%%%%%%%%%%%%%%%%%%%%%%%%%%%%%%%%%%%%%%%%%

\begin{flushright} {\tiny {\color{gray} python\_codes/fieldstone\_180/text.tex}} \end{flushright}

\par\noindent\rule{\textwidth}{0.4pt}

\begin{center}
\inpython
{\small Code: \url{https://github.com/cedrict/fieldstone/tree/master/python_codes/fieldstone_180}}
\end{center}

\par\noindent\rule{\textwidth}{0.4pt}

{\sl This stone was developed with input from H. Brett}. \index{contributors}{H. Brett}

\par\noindent\rule{\textwidth}{0.4pt}

Last revision: September 5th, 2025.

\par\noindent\rule{\textwidth}{0.4pt}

%%%%%%%%%%%%%%%%%%%%%%%%%%%%%%%%%%%%%%%%%%%%%%%%%%%%%%%%%%%%%%%%%%%%%%%%%%%%%%%%%%%%%%%%%%%%%%%%%%%

We wish to solve the steady state diffusion equation $\vec\nabla \cdot (-k \vec\nabla T)=0$
in the case where the heat conductivity is not necessarily constant in space.
The theory behind the special stencil used here is presented in Section~\ref{MMM-}.


%%%%%%%%%%%%%%%%%%%%%%%%%%%%%%%%%%%%%%%%%%%%%%%%%%%%%%%%%%%%%%
\section*{A simple example}

Let us consider the case of 5 nodes inside the segment $[0,1]$:
\begin{lstlisting}
nnx=5                         
x=np.array([0,0.3,0.4,0.9,1]) 
\end{lstlisting}
We set $k=1$ for all nodes:
\begin{lstlisting}
hcond=np.ones(nnx)  
\end{lstlisting}
The cell sizes are then computed as follows\footnote{There is surely a more python
way of doing this}:
\begin{lstlisting}
h=np.zeros(nnx-1,dtype=np.float64)
for i in range(0,nnx-1):
    h[i]=x[i+1]-x[i]
\end{lstlisting}
The matrix and rhs of the linear system are declared:
\begin{lstlisting}
b=np.zeros(nnx,dtype=np.float64)
A=lil_matrix((nnx,nnx),dtype=np.float64)
\end{lstlisting}
Boundary conditions are then applied, in this case $T_0=0$ and $T_4=1$, 
which translates as follows:
\begin{lstlisting}
A[0,0]=1         ; b[0]=0     # left boundary condition
A[nnx-1,nnx-1]=1 ; b[nnx-1]=1 # right boundary condition
\end{lstlisting}
For each internal node the stencil is applied:
\begin{lstlisting}
for i in range(1,nnx-1):
    A[i,i-1]=-(hcond[i-1]+hcond[i])/h[i-1]/(h[i-1]+h[i])
    A[i,i]  = (hcond[i+1]+hcond[i])/h[i]  /(h[i-1]+h[i]) \
            + (hcond[i-1]+hcond[i])/h[i-1]/(h[i-1]+h[i]) 
    A[i,i+1]=-(hcond[i+1]+hcond[i])/h[i]  /(h[i-1]+h[i]) 
\end{lstlisting}
Finally the linear system is solved and the temperature 
field is obtained:
\begin{lstlisting}
T=sps.linalg.spsolve(sps.csr_matrix(A),b)
\end{lstlisting}
When printing the temperature solution we find 
\begin{lstlisting}
T= [0.  0.3 0.4 0.9 1. ] 
\end{lstlisting}
which is a straight line between 0 and 1, as expected.





