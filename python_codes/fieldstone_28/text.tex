\includegraphics[height=1.25cm]{images/pictograms/replication}
\includegraphics[height=1.25cm]{images/pictograms/benchmark}
\includegraphics[height=1.25cm]{images/pictograms/msc}
\includegraphics[height=1.25cm]{images/pictograms/FEM}
\includegraphics[height=1.25cm]{images/pictograms/temperature}
\includegraphics[height=1.25cm]{images/pictograms/nonlinear}

%%%%%%%%%%%%%%%%%%%%%%%%%%%%%%%%%%%%%%%%%%%%%%%%%%%%%%%%%%%%%%%%%%%%%%%%%%%%%%%%%%%%%%%%%%%%%%

\lstinputlisting[language=bash,basicstyle=\small]{python_codes/fieldstone_28/keywords.ascii}

\begin{center}
\inpython
{\small Code: \url{https://github.com/cedrict/fieldstone/tree/master/python_codes/fieldstone_28}}
\end{center}

\par\noindent\rule{\textwidth}{0.4pt}

{\sl This fieldstone was developed in collaboration with Rens Elbertsen}. 
\index{contributors}{Rens Elbertsen}

\par\noindent\rule{\textwidth}{0.4pt}
%%%%%%%%%%%%%%%%%%%%%%%%%%%%%%%%%%%%%%%%%%%%%%%%%%%%%%%%%%%%%%%%%%%%%%%%%%%%%%%%%%%%%%%%%%%%%%

This \stone is based on \textcite{tosn15} (2015).
As for every \stone aiming at reproducing results off a publication I here include de abstract
of the article:

\begin{center}
\begin{minipage}{13cm}
{\small
Numerical simulations of thermal convection in the Earth’s mantle often employ a pseudoplastic 
rheology in order to mimic the plate-like behavior of the lithosphere. Yet the benchmark tests available
in the literature are largely based on simple linear rheologies in which the viscosity is either assumed to be
constant or weakly dependent on temperature. Here we present a suite of simple tests based on nonlinear
rheologies featuring temperature, pressure, and strain rate-dependent viscosity. Eleven different codes
based on the finite volume, finite element, or spectral methods have been used to run five benchmark cases
leading to stagnant lid, mobile lid, and periodic convection in a 2-D square box. For two of these cases, we
also show resolution tests from all contributing codes. In addition, we present a bifurcation analysis, describing 
the transition from a mobile lid regime to a periodic regime, and from a periodic regime to a stagnant
lid regime, as a function of the yield stress. At a resolution of around 100 cells or elements in both vertical
and horizontal directions, all codes reproduce the required diagnostic quantities with a discrepancy of at
most 3\% in the presence of both linear and nonlinear rheologies. Furthermore, they consistently predict
the critical value of the yield stress at which the transition between different regimes occurs. As the most
recent mantle convection codes can handle a number of different geometries within a single solution
framework, this benchmark will also prove useful when validating viscoplastic thermal convection 
simulations in such geometries.}
\end{minipage}
\end{center}

The viscosity field $\eta$ is calculated as the harmonic average between a linear part $\eta_{lin}$ 
that depends on temperature only or on temperature and depth $d$ , and a non-linear,
plastic part $\eta_{plast}$ dependent on the strain rate:

\begin{equation}
\eta(T,z,\dot{\boldsymbol{\epsilon}}) = 
2 \left(\frac{1}{\eta_\text{lin}(T,z)} + \frac{1}{\eta_\text{plast}(\dot{\boldsymbol{\epsilon}})} \right)^{-1}. 
\label{eq:eta}
\end{equation}

The linear part is given by the linearized Arrhenius law (the so-called Frank-Kamenetskii approximation \cite{fran69}):

\begin{equation}
\eta_\text{lin} (T,z) = \exp(-\gamma_T T + \gamma_{z} z), \label{eq:eta_Ty}
\end{equation}

where $\gamma_T = \ln ( \Delta\eta_T)$ and $\gamma_{z}=\ln(\Delta\eta_{z})$ are parameters controlling the total viscosity 
contrast due to temperature ($\Delta\eta_T$) and pressure ($\Delta\eta_{z}$). The non-linear part is given by \cite{trha98,trha98b}: 

\begin{equation}
\eta_\text{plast} (\dot{\boldsymbol{\epsilon}}) 
= \eta^{*} + \frac{\sigma_Y}{\sqrt{\dot{\boldsymbol{\epsilon}}:\dot{\boldsymbol{\epsilon}}}}, \label{eq:eta_sigma}
\end{equation}
where $\eta^*$ is a constant representing the effective viscosity at high stresses \cite{stlh14} and $\sigma_Y$ is the yield stress, also assumed to be constant. In 2-D, the denominator in the second term of 
equation (\ref{eq:eta_sigma}) is given explicitly by

\begin{equation}
\sqrt{\dot{\boldsymbol{\epsilon}}:\dot{\boldsymbol{\epsilon}}} 
= \sqrt{\dot{\epsilon}_{ij} \dot{\epsilon}_{ij} } 
= \sqrt{\left( \frac{\partial u_x}{\partial x} \right)^2 + \frac{1}{2} \left( \frac{\partial u_x}{\partial y} 
+ \frac{\partial u_y}{\partial x} \right)^2 + \left( \frac{\partial u_y}{\partial y} \right)^2  }.
\end{equation}

The viscoplastic flow law (equation \ref{eq:eta}) leads to linear viscous 
deformation at low stresses (equation (\ref{eq:eta_Ty})) 
and to plastic deformation for stresses that exceed $\sigma_Y$ (equation (\ref{eq:eta_sigma})), 
with the decrease in viscosity limited by the choice of $\eta^{*}$ \cite{stlh14}.

In all cases that we present, the domain is a two-dimensional square box. The mechanical 
boundary conditions are for all boundaries free-slip 
with no flux across, i.e. $\tau_{xy}=\tau_{yx}=0$ and $\boldsymbol{u}\cdot \boldsymbol{n}=0$, 
where $\boldsymbol{n}$ denotes the outward normal to 
the boundary. Concerning the energy equation, the bottom and top boundaries are isothermal, 
with the temperature $T$ set to 1 and 0, respectively, 
while side-walls are assumed to be insulating, i.e. $\partial T/\partial x = 0$. 
The initial distribution of the temperature field is prescribed as follows:

\begin{equation}
T(x,y) = (1-y) + A \cos(\pi x)\sin(\pi y), \label{eq:initemp}
\end{equation}
where $A=0.01$ is the amplitude of the initial perturbation.


In the following Table, we list the benchmark cases according to the parameters used. 
\begin{center}
\begin{tabular}{c c c c c c c} 
\hline
Case & $Ra$ & $\Delta\eta_T$ & $\Delta\eta_y$ & $\eta^*$ & $\sigma_Y$ & Convective regime \\
\hline
1   & $10^2$ & $10^5$    & 1  & -- & --             & Stagnant lid    \\
2   & $10^2$ & $10^5$    & 1  & $10^{-3}$ & 1       & Mobile lid \\
3   & $10^2$ & $10^5$    & 10 & --  & --            & Stagnant lid \\
4   & $10^2$ & $10^5$    & 10 & $10^{-3}$ & 1       & Mobile lid  \\
5a  & $10^2$ & $10^5$    & 10 & $10^{-3}$ & 4       & Periodic  \\
5b  & $10^2$ & $10^5$    & 10 & $10^{-3}$ & 3 -- 5  & Mobile lid -- Periodic -- Stagnant lid \\
\hline
\end{tabular}\\
{\small Benchmark cases and corresponding parameters.} 
\end{center}

In Cases 1 and 3 the viscosity is directly calculated from equation (\ref{eq:eta_Ty}), 
while for Cases 2, 4, 5a, and 5b, we used equation (\ref{eq:eta}). For a given mesh resolution, 
Case 5b requires running simulations with yield stress varying between 3 and 5


In all tests, the reference Rayleigh number is set at the surface ($y=1$) to $10^2$, and the viscosity contrast due to temperature $\Delta\eta_T$ is $10^5$, implying therefore a maximum effective Rayleigh number of $10^7$ for $T=1$. Cases 3, 4, 5a, and 5b employ in addition a depth-dependent rheology with viscosity contrast  $\Delta\eta_z=10$. Cases 1 and 3 assume a linear viscous rheology that leads to a stagnant lid regime. Cases 2 and 4 assume a viscoplastic rheology that leads instead to a mobile lid regime. Case 5a also assumes a viscoplastic rheology but a higher yield stress, which ultimately causes the emergence of a strictly periodic regime. The setup of Case 5b is identical to that of Case 5a but the test consists in running several simulations using different yield stresses. Specifically, we varied $\sigma_Y$ between 3 and 5 in increments of 0.1 in order to identify the values of the yield stress corresponding to the transition from mobile to periodic and from periodic to stagnant lid regime. 

NOTE: no Crank-Nicolson scheme for temperature !!

%---------------------------------------------------------------------------------
\subsubsection*{Case 0: Newtonian case, a la Blankenbach et al., 1989}

\includegraphics[width=5cm]{python_codes/fieldstone_28/results_case0/vrms.pdf}
\includegraphics[width=5cm]{python_codes/fieldstone_28/results_case0/Nu.pdf}
\includegraphics[width=5cm]{python_codes/fieldstone_28/results_case0/vrms_Nu.pdf}

\includegraphics[width=5cm]{python_codes/fieldstone_28/results_case0/temp}
\includegraphics[width=5cm]{python_codes/fieldstone_28/results_case0/vel}



\newpage %-------------------------------------------------------
\subsubsection*{Case 1}

In this case $\eta^\star=0$ and $\sigma_Y=0$ so that $\eta_{plast}$ can be discarded.
The CFL number is set to 0.5 and the viscosity is given by 
$\eta(T,z,\dot{\boldsymbol{\epsilon}}) =   \eta_\text{lin}(T,z) $.
And since $\Delta \eta_z=1$ then $\gamma_z=0$ so that
$\eta_\text{lin} (T,z) = \exp(-\gamma_T T )$

\begin{center}
\includegraphics[width=16cm]{python_codes/fieldstone_28/results_case1/tosn15b}
\end{center}

\begin{center}
\includegraphics[width=7.8cm]{python_codes/fieldstone_28/results_case1/vrms.pdf}
\includegraphics[width=7.8cm]{python_codes/fieldstone_28/results_case1/Nu.pdf}\\
\includegraphics[width=7.8cm]{python_codes/fieldstone_28/results_case1/vrms_Nu.pdf}
\includegraphics[width=7.8cm]{python_codes/fieldstone_28/results_case1/Tavrg.pdf}
\end{center}

\begin{center}
\includegraphics[width=5cm]{python_codes/fieldstone_28/results_case1/T_profile.pdf}
\includegraphics[width=5cm]{python_codes/fieldstone_28/results_case1/eta_profile.pdf}
\includegraphics[width=5cm]{python_codes/fieldstone_28/results_case1/V_profile.pdf}
\end{center}
\newpage
\begin{center}
\includegraphics[width=7.cm]{python_codes/fieldstone_28/results_case1/u}
\includegraphics[width=7.cm]{python_codes/fieldstone_28/results_case1/v}\\
\includegraphics[width=7.cm]{python_codes/fieldstone_28/results_case1/vel}
\includegraphics[width=7.cm]{python_codes/fieldstone_28/results_case1/T}\\
\includegraphics[width=7.cm]{python_codes/fieldstone_28/results_case1/rho}
\includegraphics[width=7.cm]{python_codes/fieldstone_28/results_case1/mueff}
\end{center}









\newpage %-------------------------------------------------------
\subsubsection*{Case 2}

\includegraphics[width=16cm]{python_codes/fieldstone_28/results_case4/tosn15}

\begin{center}
\includegraphics[width=7.8cm]{python_codes/fieldstone_28/results_case2/vrms.pdf}
\includegraphics[width=7.8cm]{python_codes/fieldstone_28/results_case2/Nu.pdf}\\
\includegraphics[width=7.8cm]{python_codes/fieldstone_28/results_case2/vrms_Nu.pdf}
\includegraphics[width=7.8cm]{python_codes/fieldstone_28/results_case2/Tavrg.pdf}
\end{center}

\begin{center}
\includegraphics[width=5cm]{python_codes/fieldstone_28/results_case2/T_profile.pdf}
\includegraphics[width=5cm]{python_codes/fieldstone_28/results_case2/eta_profile.pdf}
\includegraphics[width=5cm]{python_codes/fieldstone_28/results_case2/V_profile.pdf}
\end{center}

\newpage
\begin{center}
\includegraphics[width=7.cm]{python_codes/fieldstone_28/results_case2/u}
\includegraphics[width=7.cm]{python_codes/fieldstone_28/results_case2/v}\\
\includegraphics[width=7.cm]{python_codes/fieldstone_28/results_case2/vel}
\includegraphics[width=7.cm]{python_codes/fieldstone_28/results_case2/T}\\
\includegraphics[width=7.cm]{python_codes/fieldstone_28/results_case2/rho}
\includegraphics[width=7.cm]{python_codes/fieldstone_28/results_case2/mueff}
\end{center}




\newpage %-------------------------------------------------------
\subsubsection*{Case 3}

\includegraphics[width=16cm]{python_codes/fieldstone_28/results_case3/tosn15}

\begin{center}
\includegraphics[width=7.8cm]{python_codes/fieldstone_28/results_case3/vrms.pdf}
\includegraphics[width=7.8cm]{python_codes/fieldstone_28/results_case3/Nu.pdf}\\
\includegraphics[width=7.8cm]{python_codes/fieldstone_28/results_case3/vrms_Nu.pdf}
\includegraphics[width=7.8cm]{python_codes/fieldstone_28/results_case3/Tavrg.pdf}
\end{center}

\begin{center}
\includegraphics[width=5cm]{python_codes/fieldstone_28/results_case3/T_profile.pdf}
\includegraphics[width=5cm]{python_codes/fieldstone_28/results_case3/eta_profile.pdf}
\includegraphics[width=5cm]{python_codes/fieldstone_28/results_case3/V_profile.pdf}
\end{center}

\newpage
\begin{center}
\includegraphics[width=7.cm]{python_codes/fieldstone_28/results_case3/u}
\includegraphics[width=7.cm]{python_codes/fieldstone_28/results_case3/v}\\
\includegraphics[width=7.cm]{python_codes/fieldstone_28/results_case3/vel}
\includegraphics[width=7.cm]{python_codes/fieldstone_28/results_case3/T}\\
\includegraphics[width=7.cm]{python_codes/fieldstone_28/results_case3/rho}
\includegraphics[width=7.cm]{python_codes/fieldstone_28/results_case3/mueff}
\end{center}




\newpage %-------------------------------------------------------
\subsubsection*{Case 4}

\includegraphics[width=16cm]{python_codes/fieldstone_28/results_case4/tosn15}

\begin{center}
\includegraphics[width=7.8cm]{python_codes/fieldstone_28/results_case4/vrms.pdf}
\includegraphics[width=7.8cm]{python_codes/fieldstone_28/results_case4/Nu.pdf}\\
\includegraphics[width=7.8cm]{python_codes/fieldstone_28/results_case4/vrms_Nu.pdf}
\includegraphics[width=7.8cm]{python_codes/fieldstone_28/results_case4/Tavrg.pdf}
\end{center}

\begin{center}
\includegraphics[width=5cm]{python_codes/fieldstone_28/results_case4/T_profile.pdf}
\includegraphics[width=5cm]{python_codes/fieldstone_28/results_case4/eta_profile.pdf}
\includegraphics[width=5cm]{python_codes/fieldstone_28/results_case4/V_profile.pdf}
\end{center}

\newpage
\begin{center}
\includegraphics[width=7.cm]{python_codes/fieldstone_28/results_case4/u}
\includegraphics[width=7.cm]{python_codes/fieldstone_28/results_case4/v}\\
\includegraphics[width=7.cm]{python_codes/fieldstone_28/results_case4/vel}
\includegraphics[width=7.cm]{python_codes/fieldstone_28/results_case4/T}\\
\includegraphics[width=7.cm]{python_codes/fieldstone_28/results_case4/rho}
\includegraphics[width=7.cm]{python_codes/fieldstone_28/results_case4/mueff}
\end{center}

\newpage %-------------------------------------------------------
\subsubsection*{Case 5a - Periodic Solutions}

\begin{center}
\includegraphics[width=7.8cm]{python_codes/fieldstone_28/results_case5/vrms.pdf}
\includegraphics[width=7.8cm]{python_codes/fieldstone_28/results_case5/Nu.pdf}\\
\includegraphics[width=7.8cm]{python_codes/fieldstone_28/results_case5/vrms_Nu.pdf}
\includegraphics[width=7.8cm]{python_codes/fieldstone_28/results_case5/Tavrg.pdf}\\
\includegraphics[width=7.8cm]{python_codes/fieldstone_28/results_case5/u.pdf}
\includegraphics[width=7.8cm]{python_codes/fieldstone_28/results_case5/v.pdf}
\end{center}


\newpage
\noindent
\begin{center}
\includegraphics[width=3.74cm]{python_codes/fieldstone_28/results_case5/T_0000}
\includegraphics[width=3.74cm]{python_codes/fieldstone_28/results_case5/T_0010}
\includegraphics[width=3.74cm]{python_codes/fieldstone_28/results_case5/T_0020}
\includegraphics[width=3.74cm]{python_codes/fieldstone_28/results_case5/T_0030}\\
\includegraphics[width=3.74cm]{python_codes/fieldstone_28/results_case5/T_0040}
\includegraphics[width=3.74cm]{python_codes/fieldstone_28/results_case5/T_0050}
\includegraphics[width=3.74cm]{python_codes/fieldstone_28/results_case5/T_0060}
\includegraphics[width=3.74cm]{python_codes/fieldstone_28/results_case5/T_0070}\\
\includegraphics[width=3.74cm]{python_codes/fieldstone_28/results_case5/T_0080}
\includegraphics[width=3.74cm]{python_codes/fieldstone_28/results_case5/T_0090}
\includegraphics[width=3.74cm]{python_codes/fieldstone_28/results_case5/T_0100}
\includegraphics[width=3.74cm]{python_codes/fieldstone_28/results_case5/T_0110}\\
\includegraphics[width=3.74cm]{python_codes/fieldstone_28/results_case5/T_0120}
\includegraphics[width=3.74cm]{python_codes/fieldstone_28/results_case5/T_0130}
\includegraphics[width=3.74cm]{python_codes/fieldstone_28/results_case5/T_0140}
\includegraphics[width=3.74cm]{python_codes/fieldstone_28/results_case5/T_0150}\\
{\captionfont Temperature field evolution}
\end{center}

%-----------------------------------------------
\subsubsection*{Case 5b - Bifurcation Analysis}

Case 5b is an extension of Case 5a. Here we varied the yield stress from $\sigma_Y=$ 3 to 5 with the goal of identifying 
the critical values at which the system transitions from a steady, mobile lid regime (for low values of $\sigma_Y$ ),
to a periodic regime first (for intermediate values of $\sigma_Y$ ), and to a steady, stagnant lid regime afterward (for
high values of $\sigma_Y$).

\todo[inline]{Carry out all these runs and plot against paper data}

