\includegraphics[height=1.25cm]{images/pictograms/benchmark}
\includegraphics[height=1.25cm]{images/pictograms/FEM}
\includegraphics[height=1.25cm]{images/pictograms/temperature}

%%%%%%%%%%%%%%%%%%%%%%%%%%%%%%%%%%%%%%%%%%%%%%%%%%%%%%%%%%%%%%%%%%%%%%%%%%%%%%%%%%%%%%%%%%%%%%%%%%%

\begin{flushright} {\tiny {\color{gray} python\_codes/fieldstone\_141/text.tex}} \end{flushright}

\lstinputlisting[language=bash,basicstyle=\small]{python_codes/fieldstone_141/keywords.key}

\begin{center}
\fbox{\textbf{\huge \color{teal} P}}
Code: \url{https://github.com/cedrict/fieldstone/tree/master/python_codes/fieldstone_141}
\end{center}

\par\noindent\rule{\textwidth}{0.4pt}
%%%%%%%%%%%%%%%%%%%%%%%%%%%%%%%%%%%%%%%%%%%%%%%%%%%%%%%%%%%%%%%%%%%%%%%%%%%%%%%%%%%%%%%%%%%%%%

The title of this \stone is voluntarily meant to be funny. 
Nobody calls what follows the 'Beaumont trick' but it has been 
prominent in many of his (and his collaborators') publications, all based on the \sopale code.
Let us look at the following figures from a few papers and focus on the 
initial temperature field.

\begin{center}
\fbox{\includegraphics[width=8cm]{python_codes/fieldstone_141/images/albe15}}
\fbox{\includegraphics[width=8cm]{python_codes/fieldstone_141/images/cubh07}}\\
\fbox{\includegraphics[width=8cm]{python_codes/fieldstone_141/images/bubj14}}
\fbox{\includegraphics[width=8cm]{python_codes/fieldstone_141/images/hube11}}\\
\fbox{\includegraphics[width=8cm]{python_codes/fieldstone_141/images/hube14}}
\fbox{\includegraphics[width=8cm]{python_codes/fieldstone_141/images/wabj08}}\\
\fbox{\includegraphics[width=8cm]{python_codes/fieldstone_141/images/bejb09}}
\fbox{\includegraphics[width=8cm]{python_codes/fieldstone_141/images/bube17}}\\
{\captionfont 
A: Taken from \textcite{albe15} (2015),
B: Taken from \textcite{bube17} (2017),
C: Taken from \textcite{bubj14} (2014).
D: Taken from \textcite{hube11} (2011),
E: Taken from \textcite{hube14} (2014),
F: Taken from \textcite{wabj08} (2008),
G: Taken from \textcite{bejb09} (2009),
H: Taken from \textcite{cubh07} (2007).}
\end{center}

What is common to all is (aside from margins, cratons, sediment layers, etc ...) the presence 
of a crust, a lithosphere and a mantle. 
The temperature at the top of the model is $0~\si{\celsius}$, $550~\si{\celsius}$ at the base of the 
crust, $1330-1380~\si{\celsius}$ at the base of the lithosphere (it depends on its thickness), 
and on some of the figures we see that the temperature is $1500~\si{\celsius}$ at the bottom of the 
domain (which is sometimes 600 or 660~\si{\km} thick).

In between these depths the geotherm is linear (maybe parabolic in the crust where heat production
is present, but not too relevant here).

At this stage it is worth noting that at the base of the model a heat flux value is prescribed
(i.e. a Neumann boundary condition) of about $21~\si{\milli\watt\per\square\meter}$.

Another point worth noticing is the very large value of the heat conductivity 
$k=52~\si{\watt\per\meter\kelvin}$
of the mantle in \textcite{wabj08} (2008) (also in \textcite{wabj08b,wabj08c} - not shown here) and 
in \textcite{bejb09} (2009) and other publications (see figure above).
Indeed, we can look at the table of properties of all materials in the 2008 paper and confirm 
this is not a mistake in the figure:

\begin{center}
\fbox{\includegraphics[width=16cm]{python_codes/fieldstone_141/images/wabj08-2}}\\
{\captionfont Taken from \textcite{wabj08} (2008).}
\end{center}

The purpose of this \stone is to answer these two questions: why did the authors set the 
conductivity of the mantle to such high values and why did they use heat flux boundary conditions?

Let us focus for a minute on \textcite{bejb09} (figure 'G'): at \SI{120}{\km} depth the temperature is \SI{1336}{\celsius}
while it is set to \SI{1552}{\celsius} at the base of the model. 
The temperature gradient is then 
\[
\frac{\Delta T}{\Delta z} = \frac{1552-1336}{660-120} = 0.4~\si{\celsius\per\km}
\]
The heat flux is then $|q_y|=k \Delta T/\Delta y = 52*0.4=
20.8~\si{\milli\watt\per\square\meter}$, which is exactly the value prescribed 
at the bottom.
Turning now to \textcite{hube14} (figure 'E') or \textcite{hube11} (figure 'D'): 
The temperature difference is 
$\Delta T=1520-1330=190~\si{\celsius}$ 
over $\Delta y=600-125=475~\si{\km}$, so that once again the 
temperature gradient is 0.4~\si{\celsius\per\km}.

These values for the temperature gradient in the upper mantle do agree well 
with adiabatic temperature profiles in the (upper) mantle, see for example 
\textcite{kayy10}. 

In what follows we'll focus for simplicity 
on the setup of \textcite{hube11} (2011): a 35km crust, a 90km lithosphere, 
a 600km deep domain, with no variation in the horizontal direction. 


%what does Taras ?1


%------------------------------------------------------------------------------
\subsubsection*{Remarks about the code}

The first obvious thing to be mentioned is that this is inherently a 
1D problem in the $y$-direction.
However, for practical reasons I have reused an existing 2D code. As such, 
we need to specify a lateral extent $L_x$ to the domain which can be only a 
few kilometers. 

The code is a FEM code relying on bi-linear elements ($Q_1$). Both 
Dirichlet and Neumann boundary conditions are implemented. 
Before we use it to investigate the problem highlighted above, we need to 
make sure it works as intended, i.e. we need to carry out analytical benchmarks.

Each element is assigned a heat conductivity $k$, 
heat capacity $C_p$ and density $\rho$.
The timestep value is computed by means of a CFL condition for diffusive 
processes, i.e. $dt=0.75*\min(h_x,h_y)^2/\max(\kappa)$ where
$h_x,h_y$ is the size of an element, and $\kappa$ is the heat diffusivity
given by $\kappa=k/\rho C_p$. The computed timestep is capped to $10^5~\si{\year}$.

The heat flux vector is computed in each corner of each element using the elemental 
heat conductivity and then averaged out on the nodes. 

All results are exported to ascii and vtu files. Note that the latter have been stretched
by a factor 20 in the horizontal direction so as to facilitate visualisation in paraview. 

Note that if temperature is prescribed at the bottom of the domain then 
the heat flux is unconstrained. On the other hand, if the heat flux is prescribed on 
the boundary then it is the temperature that is unconstrained.

%------------------------------------------------------------------------------
\subsubsection*{benchmark \#1 ({\tt test=1})}

The domain is $5\times 100~\si{\km}$. 
Temperature boundary conditions are imposed at the top and bottom with 
$T_{top}=0~\si{\celsius}$
and $T_{bottom}=100~\si{\celsius}$.
The initial temperature $T(x,y,t=0)$ is set to $T_{top}$.

A time-dependent analytical solution exists for this problem and is given ...

{\color{red} to do !}

%------------------------------------------------------------------------------
\subsubsection*{benchmark \#2 ({\tt test=2})}

The domain is $5\times 100~\si{\km}$. 
Temperature boundary conditions are imposed at the top with 
$T_{top}=0\si{\celsius}$
and a heat flux is prescribed at the bottom 
$q_{bottom}=0.03~\si{\watt\per\square\meter}$.
The initial temperature $T(x,y,t=0)$ is set to $T_{top}$.

The expected steady state solution is such that the temperature is 
$T(x,y=L_y,t)=T_{top}$ and $q_y(x,y,t)=q_{bottom}$. 
Since then $|q_y|=k \Delta T/L_y$, then we expect a temperature at the
bottom of $60~\si{\celsius}$.

\begin{center}
\includegraphics[width=8cm]{python_codes/fieldstone_141/results/test2/temperature}
\includegraphics[width=8cm]{python_codes/fieldstone_141/results/test2/heat_flux}
\end{center}

\begin{center}
\includegraphics[width=5cm]{python_codes/fieldstone_141/results/test2/T0000}
\includegraphics[width=5cm]{python_codes/fieldstone_141/results/test2/T0500}
\includegraphics[width=5cm]{python_codes/fieldstone_141/results/test2/T1000}\\
\includegraphics[width=5cm]{python_codes/fieldstone_141/results/test2/T2000}
\includegraphics[width=5cm]{python_codes/fieldstone_141/results/test2/T3000}
\includegraphics[width=5cm]{python_codes/fieldstone_141/results/test2/T4000}\\
{\captionfont Time evolution of the temperature field.}
\end{center}

%------------------------------------------------------------------------------
\subsubsection*{Real application ({\tt test=0})}

We want to track the depth of the 1330~\si{\celsius} isotherm.
We then run three models:
\begin{itemize}
\item $k=5$ in the mantle, Neumann b.c. at the bottom with $q_{bottom}=0.0208$  
\item $k=5$ in the mantle, Dirichlet b.c. at the bottom with $T_{bottom}=1520$  
\item $k=52$ in the mantle, Neumann b.c. at the bottom with $q_{bottom}=0.0208$  
\end{itemize}
Note that in the case $k=52$ the time step dt is essentially 10x smaller than in the 
$k=5$ case so that the code needs to run longer.

\begin{center}
\includegraphics[width=8cm]{python_codes/fieldstone_141/results/test0/depth}
\includegraphics[width=8cm]{python_codes/fieldstone_141/results/test0/heat_flux}\\
\includegraphics[width=8cm]{python_codes/fieldstone_141/results/test0/temperature}
\end{center}

We see that the depth of the 1330 isotherm increases with time if $k=5$ values are used.
Using a higher $k$ value prevents that.

\begin{center}
\includegraphics[width=6cm]{python_codes/fieldstone_141/results/test0/temps_0Myr}
\includegraphics[width=6cm]{python_codes/fieldstone_141/results/test0/temps_5Myr}\\
{\captionfont Left: temperatures at t=0; Right: temperatures at t=5Myr. 
from left to right: k05D, k05N, k52N.}
\end{center}






