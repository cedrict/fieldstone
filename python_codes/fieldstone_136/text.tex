\begin{flushright} {\tiny {\color{gray} python\_codes/fieldstone\_136/text.tex}} \end{flushright}

%\lstinputlisting[language=bash,basicstyle=\small]{python_codes/fieldstone_01/keywords}

\begin{center}
\inpython Codes at \url{https://github.com/cedrict/fieldstone/tree/master/python_codes/fieldstone_136}
\end{center}

\par\noindent\rule{\textwidth}{0.4pt}
%---------------------------------------------------------------------------------------------

\textcite{gusa98} built a regionalized upper mantle (RUM) reference model.
The form of the RUM model is a set of velocity
profiles as functions of depth through the upper mantle for each of the different
tectonic provinces of Earth. Together the profiles constitute a three-dimensional
model which incorporates considerable structural detail.
The RUM model includes subducting slabs as sharp fast features in the
upper mantle. The RUM model is designed
to represent as much of upper mantle heterogeneity as seen by body wave travel
times as possible with a simple model. It can be useful as a reference model for
individual tectonic regions. 
A companion paper \parencite{sagu98} about the methodology is also available.

The RUM project has (an outdated) webpage: \url{http://rses.anu.edu.au/seismology/projects/RUM/}.
Unfortunately the download links are all broken. 
After contacting Malcolm Sambridge, he was kind enough to send me all the RUM files. 

From these files I use the slab contours in {\tt slab/Contours/}. 
The files *.slb contain slab contours of individual coherent slab bodies.
The first line contains a name which is the name of the file as they are
named here.  Subsequently come individual contours.  First the number of
samples on the contour.  Then longitude (0-360), latitude, and depth.
The contours are drawn such that they follow approximately the top of
the seismogenic region.

The code reads all files and generates a vtu file for each as well
as a single vtu file containing all slabs. 

\begin{center}
\includegraphics[width=7cm]{python_codes/fieldstone_136/images/view1}
\includegraphics[width=7cm]{python_codes/fieldstone_136/images/view2}\\
\includegraphics[width=7cm]{python_codes/fieldstone_136/images/view3}
\includegraphics[width=7cm]{python_codes/fieldstone_136/images/view4}
\end{center}
