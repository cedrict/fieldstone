\noindent
\includegraphics[height=1.25cm]{images/pictograms/replication}
\includegraphics[height=1.25cm]{images/pictograms/benchmark}
\includegraphics[height=1.25cm]{images/pictograms/FEM}
\includegraphics[height=1.25cm]{images/pictograms/paraview}

%%%%%%%%%%%%%%%%%%%%%%%%%%%%%%%%%%%%%%%%%%%%%%%%%%%%%%%%%%%%%%%%%%%%%%%%%%%%%%%%%%%%%%%%%%%%%%%%%%%

\begin{flushright} {\tiny {\color{gray} python\_codes/fieldstone\_25/text.tex}} \end{flushright}

\par\noindent\rule{\textwidth}{0.4pt}

\begin{center}
\inpython
{\small Code: \url{https://github.com/cedrict/fieldstone/tree/master/python_codes/fieldstone_25}}
\end{center}

\par\noindent\rule{\textwidth}{0.4pt}

Last revision: March 11th, 2025.

\par\noindent\rule{\textwidth}{0.4pt}

%%%%%%%%%%%%%%%%%%%%%%%%%%%%%%%%%%%%%%%%%%%%%%%%%%%%%%%%%%%%%%%%%%%%%%%%%%%%%%%%%%%%%%%%%%%%%%%%%%%

This numerical experiment was first presented in \textcite{vaks97} (1997). It is detailed 
in Section~\ref{MMM-ss:vaks97}. It consists of an isothermal Rayleigh-Taylor instability 
in a two-dimensional box of size $L_x=0.9142$ and $L_y=1$. Two Newtonian fluids are present 
in the box. The buoyant fluid layer is placed at the bottom of the box and the interface 
between both fluids is given by $y(x)=0.2+0.02\cos \left( \frac{\pi x}{L_x} \right)$.
The bottom fluid is parametrised by its mass density $\rho_1=1000$ and its viscosity $\eta_1=1,10,100$, 
while the layer above is parametrised by $\rho_2=1010$ and $\eta_2=100$ (the entire
experiment is built with dimensionless quantities).

No-slip boundary conditions are applied at the bottom and at the top of the box 
while free-slip boundary conditions are applied on the sides. 

In the original benchmark the system is run over 2000 units of dimensionless time and the 
timing and position of various upwellings/downwellings is monitored. 
In this \stone only the root mean square velocity and various velocity and pressure
statistics are measured at $t=0$ as the code is indeed not yet foreseen of any 
algorithm capable of tracking deformation.
To be clear, this is also not the goal here: we here wish to investigate the 
differences between the $Q_2\times Q_1$ (see Section~\ref{MMM-ss:pairq2q1})
and the mapped and unmapped $Q_2\times P_{-1}$ (see Section~\ref{MMM-ss:pairq2pm1})
finite element pairs (isoparametric mapping is used for both).

With regards to the FE mesh, another approach than the ones presented in the extensive 
literature which showcases results of this benchmark is taken: in this \stone the mesh 
is initially fitted to the fluids interface and the resolution is progressively increased. 
This results in the following typical mesh:

\begin{center}
\includegraphics[width=6cm]{python_codes/fieldstone_25/newresults/mats}
\end{center}
Note that in this case each cell is assigned a density and a viscosity value that 
is constant over the entire cell. 
Also, this approach requires to compute how many cells below the interface
for a given resolution. We see on the images below that for meshes $21^2-24^2$ there 
are 4 elements below the interface, for meshes $25^2-29^2$ there are 6 cells, and that 
we get to 7 cells for $30^2$ (these steps will be reflected in the measurents presented 
afterwards). The algorithm is as follows:
\begin{lstlisting}
on_interface=np.zeros(NV,dtype=bool) 
jtarget=2*int(nely/5)+1 -1 
counter = 0
for j in range(0,nny):
    for i in range(0,nnx):
        yinterface=0.2+0.02*np.cos(np.pi*xV[counter]/Lx)
        if j==jtarget:
           yV[counter]=yinterface
           on_interface[counter]=True
        if j<jtarget:
           yV[counter]=yinterface*(j+1-1.)/(jtarget+1-1.)
        if j>jtarget:
           dy=(Ly-yinterface)/(nny-jtarget-1)
           yV[counter]=yinterface+dy*(j-jtarget)
        if j==nny-1:
           yV[counter]=1.
        if j==0:
           yV[counter]=0.
        counter += 1
\end{lstlisting}

Various meshes are shown in the following figure:

\begin{center}
\includegraphics[width=3.42cm]{python_codes/fieldstone_25/images/mesh.0000.jpg}
\includegraphics[width=3.42cm]{python_codes/fieldstone_25/images/mesh.0001.jpg}
\includegraphics[width=3.42cm]{python_codes/fieldstone_25/images/mesh.0002.jpg}
\includegraphics[width=3.42cm]{python_codes/fieldstone_25/images/mesh.0003.jpg}
\includegraphics[width=3.42cm]{python_codes/fieldstone_25/images/mesh.0004.jpg}\\
\includegraphics[width=3.42cm]{python_codes/fieldstone_25/images/mesh.0005.jpg}
\includegraphics[width=3.42cm]{python_codes/fieldstone_25/images/mesh.0006.jpg}
\includegraphics[width=3.42cm]{python_codes/fieldstone_25/images/mesh.0007.jpg}
\includegraphics[width=3.42cm]{python_codes/fieldstone_25/images/mesh.0008.jpg}
\includegraphics[width=3.42cm]{python_codes/fieldstone_25/images/mesh.0009.jpg}\\
{\captionfont Meshes generated for various resolutions: $21^2$, $22^2$, $23^2$, 
$24^2$, $25^2$, $26^2$, $27^2$, $28^2$, $29^2$ and $30^2$.}
\end{center}

Although it is somewhat a trivial affair to deform the mesh (only the vertical 
position of the nodes is changed), this begs the question as to where the middle node
of the cell should be placed for best accuracy?
At this stage I urge the reader to read \stone~76 and come back here afterwards. 
Here node 8 is the middle point of the cell corners.
\begin{lstlisting}
for iel in range(0,nel):
    xV[iconV[8,iel]]=(xV[iconV[0,iel]]+xV[iconV[1,iel]]+xV[iconV[2,iel]]+xV[iconV[3,iel]])/4
    yV[iconV[8,iel]]=(yV[iconV[0,iel]]+yV[iconV[1,iel]]+yV[iconV[2,iel]]+yV[iconV[3,iel]])/4
\end{lstlisting}

After the mesh has been stretched cell edges can be straightened (all cells are then 
trapezes since lateral edges are all vertical) as parameterized by \lstinline{curved}.  
\begin{lstlisting}
if not curved: 
   for iel in range(0,nel):
       xV[iconV[4,iel]]= (xV[iconV[0,iel]]+xV[iconV[1,iel]])/2
       yV[iconV[4,iel]]= (yV[iconV[0,iel]]+yV[iconV[1,iel]])/2
       xV[iconV[6,iel]]= (xV[iconV[2,iel]]+xV[iconV[3,iel]])/2
       yV[iconV[6,iel]]= (yV[iconV[2,iel]]+yV[iconV[3,iel]])/2
       xV[iconV[5,iel]]= (xV[iconV[1,iel]]+xV[iconV[2,iel]])/2
       yV[iconV[5,iel]]= (yV[iconV[1,iel]]+yV[iconV[2,iel]])/2
       xV[iconV[7,iel]]= (xV[iconV[0,iel]]+xV[iconV[3,iel]])/2
       yV[iconV[7,iel]]= (yV[iconV[0,iel]]+yV[iconV[3,iel]])/2
\end{lstlisting}

The coordinates of the cell center are stored in arrays \lstinline{xc} and \lstinline{yc}
computed as follows: the middle of each cell is given by the forward mapping of the 
point $(r,s)=(0,0)$.
\begin{lstlisting}
for iel in range(0,nel):
    NNNV[0:9]=NNV(0,0)
    xc[iel]=NNNV.dot(xV[iconV[:,iel]])
    yc[iel]=NNNV.dot(yV[iconV[:,iel]])
\end{lstlisting}

Pressure is normalized so that $\int_\Omega p dV= 0$.
Pressure is also computed on all 9 velocity nodes of each cell and stored in 
the \lstinline{qq} array of size $m_V \times nel$. Its min/max is then computed and 
exported alongside min/max of $u$ and $v$.

The \lstinline{debug} parameter switches on the export of many fields to ascii files 
for debugging purposes.

Note that there is also a $Q_1\times P_0$ version of the code in the folder.

The code has been benchmarked against the Donea \& Huerta manufactured solution 
described in Section~\ref{MMM-mms1} and results are presented on the next page.

\newpage
%..............................................................................
\subsection*{Manufactured solution D\&H - $Q_2\times Q_1$ and $Q_2\times P_{-1}$}

Results are obtained with {\tt script\_errors\_q2} bash script.

\begin{center}
\includegraphics[width=8cm]{python_codes/fieldstone_25/results/doneahuerta/errv.pdf}
\includegraphics[width=8cm]{python_codes/fieldstone_25/results/doneahuerta/errp.pdf}\\
\includegraphics[width=8cm]{python_codes/fieldstone_25/results/doneahuerta/vrms.pdf}
\includegraphics[width=8cm]{python_codes/fieldstone_25/results/doneahuerta/max_vel.pdf}\\
\includegraphics[width=8cm]{python_codes/fieldstone_25/results/doneahuerta/min_u.pdf}
\includegraphics[width=8cm]{python_codes/fieldstone_25/results/doneahuerta/max_u.pdf}\\
\includegraphics[width=8cm]{python_codes/fieldstone_25/results/doneahuerta/min_v.pdf}
\includegraphics[width=8cm]{python_codes/fieldstone_25/results/doneahuerta/max_v.pdf}\\
\includegraphics[width=8cm]{python_codes/fieldstone_25/results/doneahuerta/min_p.pdf}
\includegraphics[width=8cm]{python_codes/fieldstone_25/results/doneahuerta/max_p.pdf}\\
\end{center}

\newpage
Velocity measured at the 'interface'
\begin{center}
\includegraphics[width=8cm]{python_codes/fieldstone_25/results/doneahuerta/interface_u_16.pdf}
\includegraphics[width=8cm]{python_codes/fieldstone_25/results/doneahuerta/interface_v_16.pdf}\\
\includegraphics[width=8cm]{python_codes/fieldstone_25/results/doneahuerta/interface_u_32.pdf}
\includegraphics[width=8cm]{python_codes/fieldstone_25/results/doneahuerta/interface_v_32.pdf}\\
\includegraphics[width=8cm]{python_codes/fieldstone_25/results/doneahuerta/interface_u_64.pdf}
\includegraphics[width=8cm]{python_codes/fieldstone_25/results/doneahuerta/interface_v_64.pdf}\\
\includegraphics[width=8cm]{python_codes/fieldstone_25/results/doneahuerta/interface_u_128.pdf}
\includegraphics[width=8cm]{python_codes/fieldstone_25/results/doneahuerta/interface_v_128.pdf}\\
\includegraphics[width=8cm]{python_codes/fieldstone_25/results/doneahuerta/interface_u_160.pdf}
\includegraphics[width=8cm]{python_codes/fieldstone_25/results/doneahuerta/interface_v_160.pdf}\\
\end{center}

\newpage
Pressure measured at the bottom:
\begin{center}
\includegraphics[width=8cm]{python_codes/fieldstone_25/results/doneahuerta/pbottom16.pdf}
\includegraphics[width=8cm]{python_codes/fieldstone_25/results/doneahuerta/pbottom32.pdf}\\
\includegraphics[width=8cm]{python_codes/fieldstone_25/results/doneahuerta/pbottom64.pdf}
\includegraphics[width=8cm]{python_codes/fieldstone_25/results/doneahuerta/pbottom128.pdf}\\
\includegraphics[width=8cm]{python_codes/fieldstone_25/results/doneahuerta/pbottom160.pdf}
\end{center}

The error convergence for velocity and pressure are third and second order 
respectively, as expected. All other measurements are following the analytical solution.
All finite elements pairs yield nearly identical solutions.
Based on these promising results we now turn to the Rayleigh-Taylor experiment.

\newpage
%..............................................................................
\subsection*{Rayleigh-Taylor instability, Isoviscous case - $Q_2\times Q_1$ and $Q_2\times P_{-1}$}

Results are obtained with {\tt script\_errors\_q2} bash script.

\begin{center}
\includegraphics[width=8cm]{python_codes/fieldstone_25/results/isoviscous/vrms.pdf}
\includegraphics[width=8cm]{python_codes/fieldstone_25/results/isoviscous/max_vel.pdf}\\
\includegraphics[width=8cm]{python_codes/fieldstone_25/results/isoviscous/min_u.pdf}
\includegraphics[width=8cm]{python_codes/fieldstone_25/results/isoviscous/max_u.pdf}\\
\includegraphics[width=8cm]{python_codes/fieldstone_25/results/isoviscous/min_v.pdf}
\includegraphics[width=8cm]{python_codes/fieldstone_25/results/isoviscous/max_v.pdf}\\
\includegraphics[width=8cm]{python_codes/fieldstone_25/results/isoviscous/min_p.pdf}
\includegraphics[width=8cm]{python_codes/fieldstone_25/results/isoviscous/max_p.pdf}\\
\end{center}



\newpage
\begin{center}
\includegraphics[width=8cm]{python_codes/fieldstone_25/results/isoviscous/interface_u_32.pdf}
\includegraphics[width=8cm]{python_codes/fieldstone_25/results/isoviscous/interface_v_32.pdf}\\
\includegraphics[width=8cm]{python_codes/fieldstone_25/results/isoviscous/interface_u_64.pdf}
\includegraphics[width=8cm]{python_codes/fieldstone_25/results/isoviscous/interface_v_64.pdf}\\
\includegraphics[width=8cm]{python_codes/fieldstone_25/results/isoviscous/interface_u_128.pdf}
\includegraphics[width=8cm]{python_codes/fieldstone_25/results/isoviscous/interface_v_128.pdf}\\
\includegraphics[width=8cm]{python_codes/fieldstone_25/results/isoviscous/interface_u_160.pdf}
\includegraphics[width=8cm]{python_codes/fieldstone_25/results/isoviscous/interface_v_160.pdf}\\
\end{center}

\begin{center}
\includegraphics[width=14cm]{python_codes/fieldstone_25/results/isoviscous/pbottom16.pdf}\\
\includegraphics[width=14cm]{python_codes/fieldstone_25/results/isoviscous/pbottom32.pdf}\\
\includegraphics[width=14cm]{python_codes/fieldstone_25/results/isoviscous/pbottom64.pdf}\\
\includegraphics[width=14cm]{python_codes/fieldstone_25/results/isoviscous/pbottom128.pdf}\\
\includegraphics[width=14cm]{python_codes/fieldstone_25/results/isoviscous/pbottom160.pdf}
\end{center}

Looking at low resolution results we see that the unmapped $Q_2\times P_{-1}$ with curved boundaries
yields a pressure field that seems to showcase some form of sawtooth pattern at the bottom. 
Looking at the amplitude we see that it is actually rather small (about $0.1\%$) and the amplitude 
does decrease with increasing resolution.


\vspace{1cm}

{\color{red} TO DO: rerun all cases, fill following table and carry out extrapolation. }

\begin{tabular}{lllll}
\hline
nelx & hx & $\upnu_{rms}(\times 10^{-6})$ & $\upnu^\star_{rms}(\times 10^{-6})$ (extrap.)  & rate \\
\hline\hline
25   & 0.036568 & 185.30019494838673 & 185.2949797 & 4.009763002 \\
50   & 0.018284 & 185.29530341925154 & 185.2949798 & 4.0348654   \\
100  & 0.009142 & 185.29499975286718 & X & X \\
200  & 0.004571 & 185.29498097047734 & X & X \\
\hline
8    & 0.114275      & 185.9669437569909  &  185.2954174 & 4.404188247 \\
16   & 0.0571375     & 185.32713277686372 &  185.2949839 & 4.057065851 \\
32   & 0.02856875    & 185.29691524980284 &  185.2949784 & 3.995213562 \\
64   & 0.014284375   & 185.29509989938787 &  185.2949832 & 6.637356434 \\
128  & 0.0071421875  & 185.29498605775253 &  X           & X           \\
256  & 0.00357109375 & 185.29498319303758 &  X           & X           \\
\hline
\end{tabular}







\newpage
%..............................................................................
\paragraph{Viscosity ratio = 10 - $Q_2\times Q_1$}....

{\color{red} these are old results- I leave them here but it should all be re-run}

\begin{center}
\includegraphics[width=5.cm]{python_codes/fieldstone_25/results/010_100/vel}
\includegraphics[width=5.cm]{python_codes/fieldstone_25/results/010_100/u}
\includegraphics[width=5.cm]{python_codes/fieldstone_25/results/010_100/v}\\
\includegraphics[width=6cm]{python_codes/fieldstone_25/results/vrms_010.pdf}
\includegraphics[width=6cm]{python_codes/fieldstone_25/results/max_vel_010.pdf}\\
\includegraphics[width=6cm]{python_codes/fieldstone_25/results/min_u_010.pdf}
\includegraphics[width=6cm]{python_codes/fieldstone_25/results/max_u_010.pdf}\\
\includegraphics[width=6cm]{python_codes/fieldstone_25/results/min_v_010.pdf}
\includegraphics[width=6cm]{python_codes/fieldstone_25/results/max_v_010.pdf}\\
\includegraphics[width=6cm]{python_codes/fieldstone_25/results/min_p_010.pdf}
\includegraphics[width=6cm]{python_codes/fieldstone_25/results/max_p_010.pdf}\\
{\captionfont Results obtained with $Q_2\times Q_1$ elements} 
\end{center}




\begin{tabular}{lllll}
\hline
nelx & hx & $\upnu_{rms}(\times 10^{-6})$ & $\upnu^\star_{rms}(\times 10^{-6})$ (extrap.)  & rate \\
\hline\hline
8    & 0.114275      & 0.0006757876577772155 & 0.0006608591144 & 0.09867640232 \\
16   & 0.0571375     & 0.0006748007227229576 & 0.0006729590166 & 1.001309003   \\ 
32   & 0.02856875    & 0.000673879034543635  & 0.0006729654426 & 1.011510945   \\
64   & 0.014284375   & 0.0006734186084028846 & X& X\\
128  & 0.0071421875  & 0.0006731902248433064 & X& X\\
256  & 0.00357109375 & 0.000673075657106445  & X& X\\
\hline
\end{tabular}












\newpage
%..............................................................................
\paragraph{Viscosity ratio = 100 - $Q_2\times Q_1$}...

{\color{red} these are old results- I leave them here but it should all be re-run}

\begin{center}
\includegraphics[width=5.cm]{python_codes/fieldstone_25/results/001_100/vel}
\includegraphics[width=5.cm]{python_codes/fieldstone_25/results/001_100/u}
\includegraphics[width=5.cm]{python_codes/fieldstone_25/results/001_100/v}\\
\includegraphics[width=6cm]{python_codes/fieldstone_25/results/vrms_001.pdf}
\includegraphics[width=6cm]{python_codes/fieldstone_25/results/max_vel_001.pdf}\\
\includegraphics[width=6cm]{python_codes/fieldstone_25/results/min_u_001.pdf}
\includegraphics[width=6cm]{python_codes/fieldstone_25/results/max_u_001.pdf}\\
\includegraphics[width=6cm]{python_codes/fieldstone_25/results/min_v_001.pdf}
\includegraphics[width=6cm]{python_codes/fieldstone_25/results/max_v_001.pdf}\\
\includegraphics[width=6cm]{python_codes/fieldstone_25/results/min_p_001.pdf}
\includegraphics[width=6cm]{python_codes/fieldstone_25/results/max_p_001.pdf}\\
{\captionfont Results obtained with $Q_2\times Q_1$ elements} 
\end{center}

\begin{tabular}{lllll}
\hline
$nelx$ & $h_x$ & $\upnu_{rms}(\times 10^{-6})$ & $\upnu^\star_{rms}(\times 10^{-6})$ (extrap.)  & rate \\
\hline\hline
8    & 0.114275      & 0.0015073487507568553 &  0.001441934413 & 7.364696091 \\
16   & 0.0571375     & 0.0014423313101926754 &  0.001441886911 & 3.154474001 \\
32   & 0.02856875    & 0.0014419368207029728 &  0.0014418534   & 1.09262734 \\
64   & 0.014284375   & 0.0014418925165763435 &  X              & X   \\
128  & 0.0071421875  & 0.0014418717420808874 &  X              & X   \\
256  & 0.00357109375 & 0.0014418612856352821 &  X              & X   \\
\hline
\end{tabular}



