\noindent
\includegraphics[height=1.25cm]{images/pictograms/benchmark}
\includegraphics[height=1.25cm]{images/pictograms/under_construction}
\includegraphics[height=1.25cm]{images/pictograms/FDM}
\includegraphics[height=1.25cm]{images/pictograms/paraview}


%%%%%%%%%%%%%%%%%%%%%%%%%%%%%%%%%%%%%%%%%%%%%%%%%%%%%%%%%%%%%%%%%%%%%%%%%%%%%%%%%%%%%%%%%%%%%%%%%%%

\begin{flushright} {\tiny {\color{gray} python\_codes/fieldstone\_153/text.tex}} \end{flushright}

%\lstinputlisting[language=bash,basicstyle=\small]{python_codes/template_keywords.key}

\par\noindent\rule{\textwidth}{0.4pt}

\begin{center}
\inpython
{\small Code: \url{https://github.com/cedrict/fieldstone/tree/master/python_codes/fieldstone_153}}
\end{center}

\par\noindent\rule{\textwidth}{0.4pt}

%{\sl This stone was developed in collaboration with Donald Duck}. \index{contributors}{D. Duck}
%\par\noindent\rule{\textwidth}{0.4pt}

{\bf \color{teal} Purpose}: implement and document bla bla 

\par\noindent\rule{\textwidth}{0.4pt}

%%%%%%%%%%%%%%%%%%%%%%%%%%%%%%%%%%%%%%%%%%%%%%%%%%%%%%%%%%%%%%%%%%%%%%%%%%%%%%%%%%%%%%%%%%%%%%%%%%%

This \stone is inspired by Exercise 5.2 of \textcite{gery19book}.

We start from
\begin{equation}
\vec\nabla^2 \omega = -\frac{1}{\eta_0} \left( -\frac{\partial (\rho g_x)}{\partial y}
+\frac{\partial (\rho g_y)}{\partial x} \right)
\end{equation}
In what follows we assume that the domain is a Cartesian box aligned 
with the $x,y$ axis. The gravity is constant and vertical so that $g_x=0$ and
then 
\begin{equation}
\vec\nabla^2 \omega 
= -\frac{g_y}{\eta_0} 
\frac{\partial \rho}{\partial x} 
\end{equation}


As shown in Section~\ref{MMM-ss:fdm_diff2D}, the Laplacian of $\omega$
can be discretised with a second order central difference stencil:
\[
\vec\nabla^2 \omega |_{i,j} 
= \frac{\omega_{i+1,j} -2\omega_{i,j} + 2\omega_{i-1,j}}{h_x^2}
+ \frac{\omega_{i,j+1} -2\omega_{i,j} + 2\omega_{i,j-1}}{h_y^2}
+{\cal O}(h_x^2,h_y^2)
\]
while the density gradient is approximated by a second order
central difference:
\[
\frac{\partial \rho}{\partial x}|_{i,j} 
= \frac{\rho_{i+1,j}-\rho_{i-1,j}}{2 h_x} + {\cal O}(h_x^2)
\]
In the end, 
\[
\frac{\omega_{i+1,j} -2\omega_{i,j} + 2\omega_{i-1,j}}{h_x^2}
+ \frac{\omega_{i,j+1} -2\omega_{i,j} + 2\omega_{i,j-1}}{h_y^2}
= -\frac{g_y}{\eta_0} 
\frac{\rho_{i+1,j}-\rho_{i-1,j}}{2 h_x} 
\]
We then define the global indices
\begin{eqnarray}
k  &=& j \cdot nnx + i \nn\\
k_E &=& j \cdot nnx + i+1 \nn\\
k_W &=& j \cdot nnx + i-1 \nn\\
k_N &=& (j+1) \cdot nnx + i \nn\\
k_S &=& (j-1) \cdot nnx + i \nn
\end{eqnarray}
so that 
\[
\frac{\omega_{k_E} -2\omega_{k} + 2\omega_{k_W}}{h_x^2}
+ 
\frac{\omega_{k_N} -2\omega_{k} + 2\omega_{k_S}}{h_y^2}
= -\frac{g_y}{\eta_0} 
\frac{\rho_{k_E}-\rho_{k_W}}{2 h_x} 
\]
or, 
\[
\frac{\omega_{k_E}}{h_x^2}+
\frac{\omega_{k_W}}{h_x^2}+
\frac{\omega_{k_N}}{h_y^2}+
\frac{\omega_{k_S}}{h_y^2}-
\left(\frac{2}{h_x^2} + \frac{2}{h_y^2} \right) \omega_k
= -\frac{g_y}{\eta_0} 
\frac{\rho_{k_E}-\rho_{k_W}}{2 h_x} 
\]
This stencil is used for all interior nodes. All nodes on the boundaries 
have $\omega=0$.

Having obtained $\omega$ we can solve for $\Psi$ by solving
$\vec\nabla^2 \Psi = -\omega$

Finally we can recover the velocity field with $u=\partial_y \Psi$ and $v=-\partial_x \Psi$.
Once again using a second order centered approach:
\[
u_{i,j} = \frac{\Psi_{i,j+i}-\Psi_{i,j-i}}{2 h_y}
\]
\[
v_{i,j} = - \frac{\Psi_{i+1,j}-\Psi_{i-1,j}}{2 h_x}
\]

%%%%%%%%%%%%%%%%%%%%%%%%%%%%%%%%%%%%%%%%%%%%%%%%%%%%%%%%%%%%%%%%%%%%%%%%%%%%%%%%%%%%%%%%%%%%%%%%%%%
\par\noindent\rule{\textwidth}{0.4pt}

\vspace{.5cm}

\begin{center}
\fbox{\begin{minipage}{0.9\textwidth}
{\color{teal}To Do, open questions, future work?}
\begin{itemize}
\item do smthg
\end{itemize}
\end{minipage}}
\end{center}

%%%%%%%%%%%%%%%%%%%%%%%%%%%%%%%%%%%%%%%%%%%%%%%%%%%%%%%%%%%%%%%%%%%%%%%%%%%%%%%%%%%%%%%%%%%%%%%%%%%
\vspace{.5cm}

\Literature:\\
\fullcite{xxxxYY}


