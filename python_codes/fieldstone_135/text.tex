\includegraphics[height=1.25cm]{images/pictograms/visualisation}
\includegraphics[height=1.25cm]{images/pictograms/3d}

\begin{flushright} {\tiny {\color{gray} python\_codes/fieldstone\_135/text.tex}} \end{flushright}

%\lstinputlisting[language=bash,basicstyle=\small]{python_codes/fieldstone_01/keywords}

\begin{center}
\inpython ~
Code: \url{https://github.com/cedrict/fieldstone/tree/master/python_codes/fieldstone_135}
\end{center}

\par\noindent\rule{\textwidth}{0.4pt}

{\sl This stone was developed in collaboration with Neil Ribe}. \index{contributors}{N. Ribe}

\par\noindent\rule{\textwidth}{0.4pt}
%%%%%%%%%%%%%%%%%%%%%%%%%%%%%%%%%%%%%%%%%%%%%%%%%%%%%%%%%%%%%%%%%%%%%%%%%%%%%%%%%%%%%%%%%%%%%%

The data is obtained from \url{https://github.com/usgs/slab2}. More precisely,
the csv files used here are in {\tt slab2code/Input/09\_21}. The data is several hundreds of Mb
and it is not mine so it is not uploaded in the fieldstone repository. Go get 
it on the slab2 github and place it in the slab2 folder.

The 29 files are read in, the depth, latitude and longitude are extracted and exported
to vtu. 

\begin{center}
\includegraphics[width=8cm]{python_codes/fieldstone_135/images/view1.png}
\includegraphics[width=8cm]{python_codes/fieldstone_135/images/view2.png}\\
\includegraphics[width=8cm]{python_codes/fieldstone_135/images/view3.png}
\includegraphics[width=8cm]{python_codes/fieldstone_135/images/view4.png}\\
\includegraphics[width=8cm]{python_codes/fieldstone_135/images/id1.png}
\includegraphics[width=8cm]{python_codes/fieldstone_135/images/id2.png}\\
\includegraphics[width=8cm]{python_codes/fieldstone_135/images/id3.png}
\includegraphics[width=8cm]{python_codes/fieldstone_135/images/id4.png}\\
{\captionfont shell+topo obtained from \stone~69, scaled 95\%.}
\end{center}
