\lstinputlisting[language=bash,basicstyle=\small]{python_codes/fieldstone_73/keywords.ascii}

\begin{center}
Code at \url{https://github.com/cedrict/fieldstone/tree/master/python_codes/fieldstone_73}
\end{center}

\par\noindent\rule{\textwidth}{0.4pt}

{\sl This stone was developed in collaboration with Bob Myhill}. \index{contributors}{B. Myhill}

\par\noindent\rule{\textwidth}{0.4pt}
%%%%%%%%%%%%%%%%%%%%%%%%%%%%%%%%%%%%%%%%%%%%%%%%%%%%%%%%%%%%%%%%%%%%%%%%%%%%%%%%%%%%%%%%%%%%%%


%.................................
\subsubsection*{Building the mesh}

The mesh counts $nnp=nnx \times nny$ points. 
In fieldstone 1 we have seen that we can build the node coordinates as follows:
\begin{lstlisting}
x = np.empty(nnp,dtype=np.float64)  # x coordinates
y = np.empty(nnp,dtype=np.float64)  # y coordinates
counter = 0
for j in range(0, nny):
    for i in range(0, nnx):
        x[counter]=i*Lx/float(nelx)
        y[counter]=j*Ly/float(nely)
        counter += 1
\end{lstlisting}

The new approach taken here is
\begin{lstlisting}
xs = np.linspace(0., 1., nnx, dtype=np.float64)
ys = np.linspace(0., 1., nny, dtype=np.float64)
xv, yv = np.meshgrid(xs, ys)
x = xv.flatten()
y = yv.flatten()
xy = np.vstack((x, y)).T
\end{lstlisting}

np.linspace returns evenly spaced numbers over a specified interval, 
in this case nnx points between 0 and 1. 

Example :
\begin{verbatim}
>>> nnx, nny = (3, 2)
>>> x = np.linspace(0, 1, nnx)
>>> y = np.linspace(0, 1, nny)
>>> xv, yv = np.meshgrid(x, y)
>>> xv
array([[0. , 0.5, 1. ],
       [0. , 0.5, 1. ]])
>>> yv
array([[0.,  0.,  0.],
       [1.,  1.,  1.]])
x = xv.flatten()
y = yv.flatten()
>>> x
[0.  0.5 1.  0.  0.5 1. ]
>>> y
[0. 0. 0. 1. 1. 1.]
xy = np.vstack((x, y)).T
[[0.  0.5 1. ]
 [0.  0.5 1. ]]
[[0. 0. 0.]
 [1. 1. 1.]]
[0.  0.5 1.  0.  0.5 1. ]
[0. 0. 0. 1. 1. 1.]
[[0.  0. ]
 [0.5 0. ]
 [1.  0. ]
 [0.  1. ]
 [0.5 1. ]
 [1.  1. ]]
>>> xy.shape
(6, 2)
\end{verbatim}


%...............................................
\subsubsection*{Building the connectivity array}

The connectivity array is of size $m \times nel$ 
where $m$ is the number of nodes per element, and $nel$ is the number 
of element in the mesh: 

\begin{lstlisting}
icon = np.zeros((m, nel), dtype=np.int32)
\end{lstlisting}

The original version is a simple double for loop
which for each element $iel$ finds its 4 corners and 
stores them in $icon[iel,0:3]$:

\begin{lstlisting}
counter = 0
for j in range(0, nely):
    for i in range(0, nelx):
        icon[0, counter] = i + j * (nelx + 1)
        icon[1, counter] = i + 1 + j * (nelx + 1)
        icon[2, counter] = i + 1 + (j + 1) * (nelx + 1)
        icon[3, counter] = i + (j + 1) * (nelx + 1)
        counter += 1
\end{lstlisting}

In the new version the same is achieved
using linspace and meshgrids again\todo{Bob: say more?}:

\begin{lstlisting}
xis = np.linspace(0., nelx-1, nelx, dtype=np.int32)
yis = np.linspace(0., nely-1, nely, dtype=np.int32)
xiv, yiv = np.meshgrid(xis, yis)
icon = np.array([xiv + yiv * nnx,
                 (xiv + 1) + yiv * nnx,
                 (xiv + 1) + (yiv + 1) * nnx,
                 xiv + (yiv + 1) * nnx]).reshape((m, nel))
\end{lstlisting}


%.................................................
\subsubsection*{Building the boundary conditions} 


\begin{lstlisting}
bc_fix = np.zeros(Nfem, dtype=np.bool)  # boundary condition, yes/no
bc_val = np.zeros(Nfem, dtype=np.float64)  # boundary condition, value
for i in range(0, nnp):
    if x[i]<eps:
       bc_fix[i*ndof]   = True ; bc_val[i*ndof]   = 0.
       bc_fix[i*ndof+1] = True ; bc_val[i*ndof+1] = 0.
    if x[i]>(Lx-eps):
       bc_fix[i*ndof]   = True ; bc_val[i*ndof]   = 0.
       bc_fix[i*ndof+1] = True ; bc_val[i*ndof+1] = 0.
    if y[i]<eps:
       bc_fix[i*ndof]   = True ; bc_val[i*ndof]   = 0.
       bc_fix[i*ndof+1] = True ; bc_val[i*ndof+1] = 0.
    if y[i]>(Ly-eps):
       bc_fix[i*ndof]   = True ; bc_val[i*ndof]   = 0.
       bc_fix[i*ndof+1] = True ; bc_val[i*ndof+1] = 0.
\end{lstlisting}



bc\_fix has been replaced by bc\_inds
bc\_val has been replaced by bc\_vals\todo{Bob: say more?}:

\begin{lstlisting}
raw_b_inds = np.where(np.logical_or.reduce((x < eps, x > Lx-eps,
                                            y < eps, y > Ly-eps)))[0]
# the [0] index above is necessary because numpy.where returns a tuple
# with len(number of dimensions), which is here equal to one.

bc_inds = np.sort(np.hstack((raw_b_inds*ndof, raw_b_inds*ndof+1)))
bc_vals = np.array([0. for idx in bc_inds])
\end{lstlisting}

%.................................................
\subsubsection*{Building the FE matrix}

The following arrays are unchanged between both versions:

\begin{lstlisting}
a_mat = np.zeros((Nfem,Nfem),dtype=np.float64)  # matrix of Ax=b
b_mat = np.zeros((3,ndof*m),dtype=np.float64)   # gradient matrix B 
rhs   = np.zeros(Nfem,dtype=np.float64)         # right hand side of Ax=b
N     = np.zeros(m,dtype=np.float64)            # shape functions
u     = np.zeros(nnp,dtype=np.float64)          # x-component velocity
v     = np.zeros(nnp,dtype=np.float64)          # y-component velocity
k_mat = np.array([[1,1,0],[1,1,0],[0,0,0]],dtype=np.float64) 
c_mat = np.array([[2,0,0],[0,2,0],[0,0,1]],dtype=np.float64) 
\end{lstlisting}

Only the arrays containing the shape function derivarives in r,s and x,y
coordinates have been changed from 

\begin{lstlisting}
dNdx  = np.zeros(m,dtype=np.float64)   
dNdy  = np.zeros(m,dtype=np.float64)    
dNdr  = np.zeros(m,dtype=np.float64)     
dNds  = np.zeros(m,dtype=np.float64)       
\end{lstlisting}
 
to this:

\begin{lstlisting}
dNdxy = np.zeros((2, m), dtype=np.float64)  
dNdrs = np.zeros((2, m), dtype=np.float64)  
\end{lstlisting}

i.e,
\[
{\tt dNdrs} = 
\left(
\begin{array}{cccc}
\partial_r N_1 & \partial_r N_2 & \dots & \partial_r N_m \\ 
\partial_s N_1 & \partial_s N_2 & \dots & \partial_s N_m 
\end{array}
\right)
\qquad
\qquad
{\tt dNdxy} = 
\left(
\begin{array}{cccc}
\partial_x N_1 & \partial_x N_2 & \dots & \partial_x N_m \\ 
\partial_y N_1 & \partial_y N_2 & \dots & \partial_y N_m 
\end{array}
\right)
\]

We see that the loop over elements and quadrature points have disappeared in the new version. 
What follows is specific to the pythonic version. 

\begin{lstlisting}
ijq = np.array([[iq, jq] for iq in [-1, 1] for jq in [-1, 1]]).T
rsq = ijq/sqrt3
\end{lstlisting}
Concretely ijq is a $2\times 4$ array:
\begin{verbatim}
[[-1 -1  1  1]
 [-1  1 -1  1]]
\end{verbatim}
and so is rsq:
\begin{verbatim}
[[-0.57735027 -0.57735027  0.57735027  0.57735027]
 [-0.57735027  0.57735027 -0.57735027  0.57735027]]
\end{verbatim}
This in fact corresponds of the explicit double for loop of the original version:
\begin{lstlisting}
for iq in [-1, 1]:
    for jq in [-1, 1]:
        rq=iq/sqrt3
        sq=jq/sqrt3
\end{lstlisting}


Two arrays are then declared which will contain the values of the shape function $N$
and its derivatives $\partial_r N,\partial_s N$ at the nqel quadrature points of an element:
\begin{lstlisting}
Nq = np.zeros((m, nqel), dtype=np.float64)        
dNdrsq = np.zeros((2, m, nqel), dtype=np.float64)

Nq[0, :] = 0.25*(1.-rsq[0])*(1.-rsq[1])
Nq[1, :] = 0.25*(1.+rsq[0])*(1.-rsq[1])
Nq[2, :] = 0.25*(1.+rsq[0])*(1.+rsq[1])
Nq[3, :] = 0.25*(1.-rsq[0])*(1.+rsq[1])

dNdrsq[:, 0, :] = [-0.25*(1.-rsq[1]), -0.25*(1.-rsq[0])]
dNdrsq[:, 1, :] = [+0.25*(1.-rsq[1]), -0.25*(1.+rsq[0])]
dNdrsq[:, 2, :] = [+0.25*(1.+rsq[1]), +0.25*(1.+rsq[0])]
dNdrsq[:, 3, :] = [-0.25*(1.+rsq[1]), +0.25*(1.-rsq[0])]
\end{lstlisting}


The Jacobian matrix of the transformation is then built,
as well as its inverse and its determinant. 
The Jacobian is built with the einsum function which
evaluates the Einstein summation convention on the operands\todo{Bob: I 
need your help here. how did u arrive to kej->eqij?}:
\begin{lstlisting}
jcb = np.einsum('ikq, kej->eqij', dNdrsq, xy[icon])
jcob = np.linalg.det(jcb)
jcbi = np.linalg.inv(jcb)
\end{lstlisting}


Then the coordinates xq,yq and the derivatives of the shape functions 
are computed at all quadrature points 

\begin{lstlisting}
xyeq = np.einsum('kq, kej->jeq', Nq, xy[icon])
dNdxyeq = np.einsum('eqij, jkq->ikeq', jcbi, dNdrsq)
\end{lstlisting}
We find that xyeq has shape $2\times nel \times 4$
while dNdxyeq has shape $2\times 4\times nel\times 4$.
\todo{At that stage this is magic to me. I can't picture 4D arrays, nor 
do I know whether the first four corresponds to nqel or m.}

The ${\bm B}$ matrix is then built, and stored for every element, 
hence its shape $3 \times ndof*m \times nel$.

\begin{lstlisting}
meq_null = np.zeros((m, nel, nqel), dtype=np.float64)
b_mat = np.array([[dNdxyeq[0], meq_null],
                  [meq_null,   dNdxyeq[1]],
                  [dNdxyeq[1], dNdxyeq[0]]]).reshape(3, ndof*m, nel, nqel,order='F')
\end{lstlisting}
In the original version it reads:
\begin{lstlisting}
for i in range(0, m):
    b_mat[0:3, 2*i:2*i+2] = [[dNdx[i],0.     ],
                             [0.     ,dNdy[i]],
                             [dNdy[i],dNdx[i]]]
\end{lstlisting}
so we see that the array $meq\_null$ stands for a zero (which explains its name).

The elemental matrix and rhs for all elements are then computed with einsum again:
\todo{happy it works but 'jieq, jk, kleq, eq->eil' is magic once again, and 
definitely not readable.}
\begin{lstlisting}
a_el = (np.einsum('jieq, jk, kleq, eq->eil', b_mat, c_mat, b_mat, jcob)
        * viscosity * weightqq)
b_el = np.einsum('iq, jeq, eq->eji', Nq, body_force(xyeq), jcob)*weightqq
\end{lstlisting}
We find that $a\_el$ has shape $nel\times8 \times 8$ while
$b\_el$ has shape $nel\times 2\times4$.

The same process as above is repeated for the one-point integration of the 
penalty term. 

Finally all the elemental matrices and vectors need to be assembled into 
the global FE matrix. \todo{Bob: help !!}

\begin{lstlisting}
m_indices = ((ndof*icon).T[:, np.newaxis, :]
             + np.indices((ndof,))[0, np.newaxis, :, np.newaxis]) # iel, k1, i1

mkk_indices = m_indices.reshape(nel, ndof*m, order='F') # iel, 1kk
mm_indices = (mkk_indices[:, :, np.newaxis] +
              0*mkk_indices[:, np.newaxis, :]) # iel, 1kk, 2kk
mm_indices = (mm_indices, np.einsum('ijm -> imj', mm_indices))

np.add.at(rhs, m_indices, b_el)
np.add.at(a_mat, mm_indices, a_el)
\end{lstlisting}

















\newpage



\begin{tabular}{lll}
\hline
 & Pros & Cons \\
\hline
\hline
old & readable &  slow\\
    & easy to debug         &       \\
new & faster & memory usage \\
    &        & not so readable \\
    &        & loops not visible \\
\hline
\end{tabular}


