\lstinputlisting[language=bash,basicstyle=\small]{python_codes/fieldstone_143/keywords.key}

\begin{center}
Code at \url{https://github.com/cedrict/fieldstone/tree/master/python_codes/fieldstone_143}
\end{center}

\par\noindent\rule{\textwidth}{0.4pt}

%%%%%%%%%%%%%%%%%%%%%%%%%%%%%%%%%%%%%%%%%%%%%%%%%%%%%%%%%%%%%%%%%%%%%%%%%%%%%%%%%%%%%%%%%%%%%%%%%%%%

This \stone is based on \textcite{yahb13} (2013). 

In the paper the authors state that their simple setup
``also ensures the reproducibility of our results that can be easily tested by any code-
solving Stokes equations with free-slip boundary conditions.''
This is {\it exactly} what we are going to do!

The authors state: ``The Cartesian box is $6000~\si{\km}$ wide and $3000~\si{\km}$
deep. The grid resolution used is of 601$\times$301 nodes which
corresponds to a spatial resolution of $10~\si{\km}$. [...] 
All the mechanical boundary conditions are free slip and all the viscosities 
in our model are linear viscous.'' This makes indeed for a simple model.

\begin{center}
\includegraphics[width=15cm]{python_codes/fieldstone_143/images/yahb13_a}\\
{\captionfont 
Taken from \cite{yahb13}. Initial configuration of the model. 
(a) Model setup. $L$: length of mantle space between the left-hand 
side of the box and the continental lithosphere; $l_c$ : length of continental lithosphere; 
$l_o$: length of oceanic lithosphere; 
$l_r=500~\si{\km}$: length corresponding to the thinning of the lithosphere due to the ridge; 
$F_d$: downwelling force; 
$F_u$: upwelling force. These buoyancy forces are implemented by setting variable
densities in the red and blue columns. The area of the columns is $A_d$ and $A_u$ for the downwelling and the
upwelling force, respectively. (b) Viscosity structure. $\eta_c$, $\eta_o$, and $\eta_{um}$ correspond
to the viscosity of the continental lithosphere, oceanic lithosphere, and upper mantle, respectively. The
viscosity of the lower mantle corresponds to a simplified profile (black line). The viscosity models from
\cite{civs12} (2012) are also provided (Family-A model in red and Family-B model in blue) as references. }
\end{center}

Please read the original paper for a justification of the chosen values for most 
of the parameters.

``The oceanic lithosphere is not attached
to the right-hand side of the box, for one upper mantle cell
separates it from the model box edge. This ensures that the
lithosphere is free to move laterally and is not affected by
the free-slip boundary condition imposed at the right side [...].''

``We [...] fixed typical thicknesses to
$200~\si{\km}$ and $100~\si{\km}$ [...] for the continental and the oceanic lithosphere, 
respectively. The width of the continental lithosphere is similarly
set in all our models to $2000~\si{\km}$.''

The density is set to $\rho_{ref}=3250~\si{\kg\per\cubic\meter}$ everywhere in the model box. Density
differences are only assigned to the upwelling and the
downwelling zones that drive, by buoyancy, the convection
cell. The density value in the upwelling ($\rho_u$) and
in the downwelling ($\rho_d$) areas depends on the buoyancy force
applied and are expressed as follows:

\begin{eqnarray}
\rho_u &=& \rho_{ref} - \frac{F_u}{\rho_{ref} g A_u} \\
\rho_d &=& \rho_{ref} + \frac{F_d}{\rho_{ref} g A_d} 
\end{eqnarray}
where $F_u$ and $F_d$ correspond to the absolute values of the 
upwelling and downwelling forces, respectively, and g is the gravitational 
acceleration set to $9.81~\si{\meter\per\square\second}$.
Note that the values for $A_u$ and $A_d$ is never specified!

The viscosity in the lower mantle corresponds
to a very simplified profile: From $660~\si{\km}$ to $2650~\si{\km}$ depth,
the viscosity is set to $\eta_{um}=5\cdot 10^{22}~\si{\pascal\second}$. 
Below, the viscosity is set to $10^{21} Pa s$ (D'' layer). 

Above the lower mantle, the viscosities
are not well resolved and therefore need to be tested.
The authors ``considered the viscosity of the continental 
lithosphere, of the oceanic lithosphere, and of the
upper mantle as linear and constant.''

\begin{center}
\includegraphics[width=10cm]{python_codes/fieldstone_143/images/yahb13_b}\\
{\captionfont 
Taken from \cite{yahb13}. 
Sketch of the mantle convective cell flow in (a) collision mode and (b) drift mode. In the
collision mode, the anchored slab does not retreat and prevents the trench-ward motion of the upper plate
and is thus blocked. In drift mode, the subducting slab is retreating and accommodates the oceanic litho-
sphere convergence that can freely move. Thick black and blue arrows show mantle flow and slab motion,
respectively. Thin black and magenta arrows correspond to the lithosphere and trench displacements,
respectively. These thin arrows are replaced by crosses when the motion is zero with respect to the upwelling.
Large grey and magenta arrows correspond to compressional and extensional areas at active and passive plate
margins, respectively.
}
\end{center}

%%%%%%%%%%%%%%%%%%%%%%%%%%%%%%%%%%%%%%%%%%%%%%%%%%%%%%%%%%%%%%%%%%%%%%%%%%%%%%%%%%%%%%%
\section*{About the code}

the basis functions are in a separate file that originates in stone 96.

Both elements are available



%%%%%%%%%%%%%%%%%%%%%%%%%%%%%%%%%%%%%%%%%%%%%%%%%%%%%%%%%%%%%%%%%%%%%%%%%%%%%%%%%%%%%%%
\section*{Collision mode}

In the collision mode, $L = 0$ and the lithosphere cannot
move freely above the convection cell. The net driving force involved in this
example amounts to $2\cdot 10^{13}~\si{\newton\per\meter}$. This total force is evenly
distributed ($F_u = F_d = 10^{13}~\si{\newton\per\meter}$). 




\begin{center}
\includegraphics[width=12cm]{python_codes/fieldstone_143/images/yahb13_c}\\
{\captionfont Taken from \cite{yahb13}.
(a) Strain rate at the surface across the $3000~\si{\km}$ long continental and oceanic lithospheres
($L=0~\si{\km}$ and $l_ c=2000~\si{\km}$). 
(b) Streamlines obtained for the downwelling driven model
($F_u=0$, $F_d=2 \cdot 10^{13} N m 1$ ), 
the upwelling driven model (F u = 2 · 10 13 N m 1 , F d = 0 N m 1 ), and
the reference model ($F_u=10^{13}~\si{\newton\per\meter}$, F d = 10 13 N m -1 ).
}
\end{center}












































