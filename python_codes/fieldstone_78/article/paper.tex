\documentclass[a4paper]{article}
\usepackage[cm]{fullpage}
\usepackage[maxnames=6]{biblatex}
\addbibresource{../../../biblio_geosciences.bib}
\usepackage{tikz}
\usepackage{bm}
\usepackage{siunitx}
\usepackage{amsfonts}

%%%%%%%%%%%%%%%%%%%%%%%%%%%%%%%%%%%%%%%%%%%%%%%%%%%%%%%%%%%%%%%%

\title{
On the use of $Q_1\times P_0$ macro-elements for the Stokes equation with an application to 
geodynamical buoyancy-driven flows
}

\author{C. Thieulot*}


%\keywords{Finite Elements, Stokes equations, macro-elements, buoyancy-driven flow}


\begin{document}
\maketitle

\begin{abstract}
for the first time 7 macro-elements are systematically compared in 2D on a set of benchmarks
\end{abstract}

\tableofcontents

%%%%%%%%%%%%%%%%%%%%%%%%%%%%%%%%%%%%
\section{Introduction} \label{sec1}


The Finite Element Method (FEM) is one of most popular methods in Computational Fluid Dynamics (CFD). 
When the method was first implemented 
in the early days of computers, memory limitations were severe and CFD practitioners logically 
focused on 'cheap' elements with a small memory footprint, such as the smallest velocity-pressure pair to 
solve the Stokes equations on quadrilateral: the bi/tri-linear velocity-constant pressure element, 
also denoted $Q_1 \times P_0$ (or sometimes $Q_1\times Q_0$, see \cite{grsa}). 

Unfortunately it was soon demonstrated \cite{XXX} that this element pair is not LBB-stable and this instability
manifests itself via the presence of the spurious chequerboard mode on regular meshes, despite a perfectly usable and 
reasonably accurate velocity field. 
This lead some part of the community to use stable elements instead, such as the Taylor-Hood pair $Q_2\times Q_1$
(or its simplicial counterpart the $P_2\times P_1$ element), 
while the rest investigated various options to stabilise this element:
One can cite approaches which include a matrix block in the lower right part of 
the otherwise empty part of the Stokes matrix\cite{kesi88,sike90,vibo92,nosi98},
the $Q_1^+\times P_0$ element of Fortin\cite{fort81}, or the $\tilde{Q}_1\times P_0$ element which adds 
a $Q_2$ bubble function to the $Q_1$ space \cite[p265]{brfo}.
Finally various stabilisations were also proposed for its 
$Q_1\times Q_1$ cousin \cite{dobo04,bodg06} but it was recently shown\cite{thba22} 
that this stabilised element is not suited to model buoyancy-driven flows, 
while retaining its usability in engineering applications.  

A third option, seldom explored, is the use of so-called macro-elements composed of $Q_1\times P_0$ elements. 
Two were shown to be stable in the mid eighties\cite{leta81,sten84}, followed by three more twenty years
later\cite{qizh07} but to the author's knowledge these macro-elements were never 
implemented and compared with each other, nor put through a series of benchmarks showcasing complex flow in the 
presence of large viscosity variations/contrasts. The present article aims at 
filling this gap and drawing conlusions as to their usability also in the case of buoyancy-driven flow
since the field of computational geodynamics has long relied on the $Q_1\times P_0$ element for the 
solution of the buoyancy-driven incompressible Stokes flow in the presence of 
dominating lithostatic pressure gradient due to the Earth's gravity field.  



%%%%%%%%%%%%%%%%%%%%%%%%%%%%%%%%%%%%%%%%%%%%%%%%%%%%%%%%%%%%%
\section{The $Q_1\times P_0$ element}\label{sec2}

The $Q_1\times P_0$ (also sometimes called 
$Q_1 \times Q_0$ \cite{grsa}) element has been the object of a lot of attention in the 80's and 90's
mostly because of its computational attractiveness and is nearly always discussed in 
textbooks dealing with the Finite Element modelling of flows \cite{grsa,dohu03,bobf08,bobf13}.
It does however not satisfy the LBB inf-sup condition and it is therefore unstable. 
This instability usually manifests itself through spurious pressure modes.

The problems of the $Q_1\times P_0$ element have long been recognised 
and have led computational geodynamicists to rely on 
higher-order inherently stable elements, such as the $Q_2\times P_{-1}$ \cite{mabl15} or the 
$Q_2\times Q_1$ \cite{krhb12} or on stabilised equal-order elements such as the $Q_1\times Q_1$ \cite{busa13}.

Arguments in favour of the $Q_1\times P_0$ element are as follows:
a) It insures mass balance at the element level \cite[p459]{grsa};
b) The pressure modes have been thoroughly analysed \cite{sagl81a,sagl81b,grsi94}.
One pressure mode is the hydrostatic mode, which is  
physical. The other one is the so-called chequerboard mode which is spurious;
c) When filtered out \cite{chpc95} the recovered pressure is usable;
d) Combined with a penalty formulation, the elimination of the pressure degrees of freedom 
yield a very compact SPD stiffness matrix \cite{zigo75,hulb79,zina82,redd82,odks82};
e) Like all other unstable elements, an irregular mesh yields a 
better numerical behaviour of the element \cite{qizh07} and pressure modes substantially subdue.


%Brezzi \& Fortin p244: " the known experimental fact that on a general
%distorted mesh pressure modes disappear and the inf-sup constant is independent of h. 
%This last fact is still resisting analysis."





%%%%%%%%%%%%%%%%%%%%%%%%%%%%%%%%%%%%%%%%%%%%%%%%%%%%%%%%%%%%%
\section{Macro-elements}\label{sec3}

Following Stenberg\cite{sten90}, we define by a macroelement $M$ "a connected set of elements of which the 
intersection of any two is either empty, a vertex, or one edge or face in $\mathbb{R}^2$ 
and $\mathbb{R}^3$, respectively".

In this work I focus on quadrilateral meshes and consider 7 
macro-elements, 5 from from the published literature and 2 new ones. These are shown in Figure~\ref{fig:mes}.
Note that macro-element S is often cited in the literature as example of a stable macro-element, 
e.g. \cite{chba93}+books, but rarely the LT one which in fact predates it by three years. 
Also QZ3 was mentioned in 1995 by Idehlsohn et al \cite{idsn95} but not used in calculations. 

\begin{figure}[t]
\input{tikz_all_macros}
\caption{
a) velocity and pressure nodes for a $Q_1\times P_0$ element;
b) Stenberg 'S' macro-element \cite{sten84}; 
c) Le Tallec macro-element \cite{leta81,leru86}; 
d,e,f) QZ1, QZ2, and QZ3 macro-elements \cite{qizh07}; g,h) macro-elements A and B (this study).
Velocity nodes are represented by filled-in circles. The purple ones indicate internal nodes which belong to 3 elements.
For each macro-element a distance $\epsilon$ is indicated which controls the positioning of the internal nodes.
\label{fig:mes}}
\end{figure}

Looking  at Fig~\ref{fig:mes}, we can also make the observation that some macro-elements 
have a two-fold symmetry (along the axis or the diagonals) while others have a four-fold one 
(axis and diagonals). May be more important is the fact that S, QZ3 and A are 'anisotropic', i.e., in 
the case of QZ3 for example, there is an edge joining two opposite corners but none joining the 
other two opposite corners.
In Table~\ref{tab1} I summarize the various properties which characterise each macro-element.


\begin{table}[h!]
\centering
\begin{tabular}{cccccc}
\hline
{name} & {axes symmetry}  & {diagonals symmetry} & {\# add. nodes inside} 
& {\# elements per m-e} & {mid edge points}  \\
\hline
\hline
S   &  Yes &   No & 2  & 5 & Yes \\
LT  &  Yes &  Yes & 9 & 12 & Yes \\
QZ1 &  Yes &  Yes & 9 & 12 & Yes \\
QZ2 &  Yes &  Yes & 5 & 8  & Yes \\
QZ3 &  No  &  Yes & 3 & 6  & Yes \\
A   &  No  &  Yes & 4 & 7  & Yes \\
B   &  Yes &  Yes & 4 & 5  & No  \\
\hline
\end{tabular}
\caption{Characteristics of all 7 macro-elements as shown in Fig.~\ref{fig:mes}.\label{tab1}}%
\end{table}


The stability of a velocity-pressure pair or a macro-element is often difficult to prove 
and various techniques have been proposed \cite{bobf13}. 
All 5 published macro-elements were shown to be LBB-stable and all showcase at least 
an internal node shared by an odd number of elements which makes a chequerboard pattern impossible. 
However, things are not that simple since macro-element B showcases 
4 such points and as we will see it is not stable and chequerboard patterns still occur 
albeit at the macro-element scale (I have used the macro-element technique of Stenberg 
also in Lamichhane to show that a patch of 2x2 m-e of B type have indeed a chequerboard mode).
Looking at a macro-element is therefore not sufficient to determine whether it is LBB-stable or not. 

Qi et al \cite{qizh07} state "[...] from the points of implementation and the approximation property, we 
would list five types of macro-elements from the best to worst as S, LT, QZ1, QZ2, QZ3." 
The authors unfortunately do not provide any numerical comparison between these 5. 
%Understand what their ranking is based on. My pragmatic approach - numerical testing!

Despite the existing literature, many questions remain with regards to these macro-elements:
\begin{itemize}
\item Based on a handful of benchmarks is it possible to establish a ranking, or at the very least can we pinpoint the objectively worse one(s)?

\item Is there an optimal position for the additional internal nodes? (i.e. can we determine an optimal value of the 
$\epsilon$ parameter for each macro-element?)

\item Can these macro-elements be safely used for geodynamical 
buoyancy-driven flows in the presence of a strong hydrostatic pressure gradient?
\end{itemize}


%- how can we generate a nodal pressure field? Pressure recovery , from disc to continuous ?

%- can we extend all/some to 3D? stability proof ?

%compute indicator (eigenvalue, ... ?) for meshes and rank them ? see chappelle bathe

%I need to understand how to prove LBB stability and apply it to 3D elts ?

%Paper content/new/inovation:

%run stokes sphere too.

%can I compute Gel for macro elt ? or a 2x2 block ?

%I disregard randomised regular meshes bc how much random is not clear.

%\begin{figure*}
%\centerline{\includegraphics[width=342pt,height=9pc,draft]{empty}}
%\caption{This is the sample figure caption.\label{fig2}}
%\end{figure*}

%Additional Literature ? \cite{bowr08}






%%%%%%%%%%%%%%%%%%%%%%%%%%%%%%%%%%%%%%%%%%%%%%%%%%%%%%%%%%%%%%%%%%%%%%%%%5
\section{Implementation}


For the purpose of this paper, we are concerned with the numerical solution of 
the incompressible and isothermal Stokes equations:
\begin{eqnarray}
-\nabla \cdot \left[ 2\eta \dot\varepsilon({\bm u}) \right] + \nabla p &=& \rho \bm g \qquad  \textrm{in $\Omega$},
\label{eq:conv_momentum}  \\  
-\nabla \cdot {\bm u} &=& 0    \qquad    \textrm{in $\Omega$},   \label{eq:conv_mass} 
\end{eqnarray}
where $\eta$ is the viscosity, $\rho$ the density, ${\bm g}$ the gravity vector, $\dot\varepsilon(\cdot)$
denotes the symmetric gradient operator defined by $\dot\varepsilon({\bm u})
=\frac 12 (\nabla {\bm u} + \nabla {\bm u}^{T})$, 
and $\Omega\subset{\mathbb R}^d, d=2\text{ or }3$ is the domain
of interest. Both the viscosity $\eta$
and the density $\rho$ will, in general, be spatially variable.

In this paper the Stokes equations are discretised using the finite element method.
The unknowns are the velocity vector ${\bm u}$ and the pressure $p$ so that 
one speaks of a Mixed Finite Element method. 
A straightforward application of the Galerkin method yields the finite-dimensional 
variational problem: 
\textit{Find ${\bm u}_h\in {\cal Q}_1,p_h\in {\cal Q}_0$
so that
\begin{eqnarray}
\label{eq:discrete-formulation}
\left(\varepsilon(\bm v_h), 2\eta \varepsilon(\bm u_h)\right)  - ( \nabla \cdot \bm v_h, p_h) &=&   ({\bm v}_h,\rho \bm g),\\
-(q_h,\nabla \cdot \bm u_h) &=& 0,
\end{eqnarray}
for all test functions ${\bm v}_h\in {\cal Q}_1, q_h\in {\cal Q}_0$.}
This procedure is rather standard and well documented in many 
textbooks \cite{grsa,dohu03,bobf13} and the reader is referred to these sources 
for more detail.


All (macro-)elements have been implemented in a dedicated Python code (see Supplementary material). 
An isoparametric mapping is used. 
The pressure nullspace due to the Dirichlet boundary conditions on all sides is removed 
by means of a Lagrange multiplier technique so as to insure that the pressure field 
satisfies $\int_\Omega p dV= 0$.
The fully assembled Stokes matrix is passed to a Python solver (WHICH exactly). 
Better solving strategies have of course 
been designed to solve the resulting saddle point \cite{begl05} but are not 
implemented since optimal performance is not of interest here.

The projection of the discontinuous pressure $p$ onto the velocity nodes is denoted by $q$. 
For simplicity I have adopted the simple procedure explained in REF: each 
elemental pressure is added to the nodes making this element and later averaged. 

When an analytical solution exists convergence rates are computed and 
we expect for smooth viscosity fields and/or boundary conditions that the convergence rates 
are optimal, i.e., that the errors satisfy the relationships
\begin{eqnarray}
\| {\bm u} - {\bm u}_h \|_{L_2} &=&  {\cal O}(h^{2}),     \\  
\| p - p_h \|_{L_2}   &=& {\cal O}(h^{1}),
  \label{eq:error-rates}
\end{eqnarray}
where $h$ is the maximal diameter over all cells in the mesh.
I will also measure and report on the convergence rate index for the nodal pressure $q$.

Looking at Fig.~\ref{fig:mes} we see that the elements inside the macro-elements 
have various sizes and shapes so 
the default version of each macro-element is the one for which all internal elements have the same area, and 
I then define the average element size
$\langle h \rangle$ as the square root of the average elemental area.










\section{Conclusions}\label{sec5}

Conclusions about mms:



Conlcusions about buoyancy-driven flows:

macro-e have two advantages: no checkerboard modes AND stable. So CG applied to Schur should converge in a number of iterations independent of h. test with ELEFANT-2 ?

use in geodynamics possible IF reduced density or rather high resolution (but then why not Q2Q1). S is probably best overal.

awkward with particle-in-cell , inner structure of macro-element could influence shear band angle. 

also, if used at high res, full density, what about mesh deformation?

extension to 3D ?




\printbibliography


\end{document}
