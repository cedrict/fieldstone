The details of the numerical setup are presented in Section \ref{f1}.

Each element has $m_V=16$ vertices so in total $ndof_V\times m_V=32$ 
velocity dofs and 
$ndof_P*m_P=9$ pressure dofs. The total number of 
velocity dofs is therefore $NfemV=nnp \times ndofV$ while the total number of
pressure dofs is $NfemP=nel$. The total number of dofs is then $Nfem=NfemV+NfemP$.

As a consequence, matrix $\K$ has size $NfemV,NfemV$ and matrix $\G$ has size $NfemV,NfemP$.
Vector $f$ is of size $NfemV$ and vector $h$ is of size $NfemP$.  

\begin{verbatim}

60===61===62===63===64===65===66===67===68===70
||             ||             ||             ||
50   51   52   53   54   55   56   57   58   59
||             ||             ||             ||
40   41   42   43   44   45   46   47   48   49
||             ||             ||             ||
30===31===32===33===34===35===36===37===38===39
||             ||             ||             ||
20   21   22   23   24   25   26   27   28   29
||             ||             ||             ||
10   11   12   13   14   15   16   17   18   19
||             ||             ||             ||
00===01===02===03===04===05===06===07===08===09

Example of 3x2 mesh. nnx=10, nny=7, nnp=70, nelx=3, nely=2, nel=6
\end{verbatim}


\begin{verbatim}
12===13===14===15           06=====07=====08
||   ||   ||   ||           ||     ||     ||
08===09===10===11           ||     ||     ||
||   ||   ||   ||           03=====04=====05
04===05===06===07           ||     ||     ||
||   ||   ||   ||           ||     ||     ||
00===01===02===03           00=====01=====02

Velocity (Q3)               Pressure (Q2)

(r,s)_{00}=(-1,-1)          (r,s)_{00}=(-1,-1) 
(r,s)_{01}=(-1/3,-1)        (r,s)_{01}=(0,-1) 
(r,s)_{02}=(+1/3,-1)        (r,s)_{02}=(+1,-1) 
(r,s)_{03}=(+1,-1)          (r,s)_{03}=(-1,0) 
(r,s)_{04}=(-1,-1/3)        (r,s)_{04}=(0,0) 
(r,s)_{05}=(-1/3,-1/3)      (r,s)_{05}=(+1,0) 
(r,s)_{06}=(+1/3,-1/3)      (r,s)_{06}=(-1,+1) 
(r,s)_{07}=(+1,-1/3)        (r,s)_{07}=(0,+1) 
(r,s)_{08}=(-1,+1/3)        (r,s)_{08}=(+1,+1) 
(r,s)_{09}=(-1/3,+1/3)
(r,s)_{10}=(+1/3,+1/3)
(r,s)_{11}=(+1,+1/3)
(r,s)_{12}=(-1,+1)
(r,s)_{13}=(-1/3,+1)
(r,s)_{14}=(+1/3,+1)
(r,s)_{15}=(+1,+1)

\end{verbatim}







{\color{red} Write about 4 point quadrature}.








\fbox{
\parbox{10cm}{{\bf features}
\begin{itemize}
\item $Q_3\times Q_2$ element \index{$Q_3 \times Q_2$}
\item incompressible flow \index{incompressible flow}
\item mixed formulation \index{mixed formulation}
\item isothermal \index{isothermal}
\item isoviscous \index{isoviscous}
\item analytical solution \index{analytical solution}
\end{itemize}
}}

\begin{center}
\includegraphics[width=10cm]{python_codes/fieldstone_19/errors}
\end{center}

velocity error rate is cubic, pressure superconvergent since the pressure field
is quadratic and therefore lies into the Q2 space.
