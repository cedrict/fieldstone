

Facts about planet Mars\footnote{\url{https://en.wikipedia.org/wiki/Mars}}:
\begin{itemize}
\item average radius $R=3389.5 \pm 0.2 \si{\km}$
\item equatorial radius $3396.2 \pm 0.1 \si{\km}$
\item polar radius $3376.2 \pm 0.1 \si{\km}$
\item volume $1.6318 \cdot 10^{20} \si{\cubic\metre}$
\item mass $6.4171 \cdot 10^{23}\si{\kilo\gram}$
\item mean density $\langle\rho\rangle= 3934\si{\kilo\gram\per\cubic\meter}$
\item moment of inertia $I=0.3644 \pm 0.0005$
\item surface gravity $g=3.72076 \si{\metre\per\square\second}$
\end{itemize}

The surface gravity can be obtained with 
\[
g_{surf}=\frac{{\cal G} M}{R^2} 
=\frac{6.67430 \cdot 10^{-11} \; 6.4171 \cdot 10^{23} }{(3.3895\cdot 10^6)^2}
\simeq 3.727977
\]

The internal structure of the planet is not settled although 
it is now widely accepted that the planet has a core. 

B. Steinberger was kind enough to communicate to us the density and viscosity 
profiles used in Steinverger \etal (2010) \cite{stwt10} \footnote{Files sent to us 
we slightly altered for the purpose of this work. The density profile was missing 
the 50km near the center of the planet so padding was used. The viscosity profile 
starts below the moho at 50\si{\km} and stopped at the cmb. }:

\begin{center}
\includegraphics[width=7cm]{python_codes/fieldstone_96/data/rho1}
\includegraphics[width=7cm]{python_codes/fieldstone_96/data/eta1}\\
\includegraphics[width=7cm]{python_codes/fieldstone_96/data/rho2}
\includegraphics[width=7cm]{python_codes/fieldstone_96/data/eta2}\\
{\captionfont Data curtesy of B. Steinberger, from \cite{stwt10}} \\
{\tiny {\color{gray} in python\_codes/fieldstone\_96/data/}}
\end{center}


\begin{center}
\includegraphics[width=5.7cm]{python_codes/fieldstone_96/images/stwt10_b}
\includegraphics[width=5.7cm]{python_codes/fieldstone_96/images/stwt10_c}
\includegraphics[width=5.7cm]{python_codes/fieldstone_96/images/stwt10_d}\\
{\captionfont Taken from Steinverger \etal (2010) \cite{stwt10}}
\end{center}

In table 1 of Steinberger \etal \cite{stwt10}: crust thickness is set to 50km. The core radius is set 
to 1389.5km. However in the data set we were sent it seems that the cmb is at 1422\si{\km}
radius.
To simplify things we take $R=3389\si{\km}$ and $R_{cmb}=1422\si{\km}$ in the code.

There are three python files in this \stone:
\begin{itemize}
\item {\pythonfile parameters.py}: the main physical and geometrical parameters are defined in it;
\item {\pythonfile generate\_nodes.py}: this code generates the two files  {\sl mypoints} and {\sl mysegments}
which will be first concatenated into {\sl mesh.poly} and then passed to the Triangle mesher. 
The mesher then returns {\sl mesh.1.node} and {\sl mesh.1.ele}.  
\item {\pythonfile stone.py} This is the 'real' code: the above mentioned files are read in and are used to 
build the mesh. Density and viscosity profile datasets are read in so that viscosity, density and 
gravity acceleration can then be assigned to elements/nodes/quadrature points. Boundary conditions 
are set up, the FEM is built, and the system solved. The pressure at the surface of the planet is 
normalised to zero average. Results are then exported to ascii and vtu file(s).
\end{itemize}
In order to run the code, simply make use of the provided script {\shellscriptfile run\_script}. 
In it the call to the Triangle mesher is carried out:
\begin{verbatim}
../../../../triangle/triangle   -q -j -O -a4000000000 -o2 -pc  mesh.poly
\end{verbatim}
The number following the {\tt -a} option is the maximum size of an element. This controls the 
average size of the generated elements inside the domain. Decreasing this number automatically 
generates more elements, i.e. a higher resolution. 

\begin{center}
\includegraphics[width=7cm]{python_codes/fieldstone_96/images/notes1}\\
\includegraphics[width=7cm]{python_codes/fieldstone_96/images/notes2}
\end{center}




 
