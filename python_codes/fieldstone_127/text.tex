\begin{flushright} {\tiny {\color{gray} python\_codes/fieldstone\_01/text.tex}} \end{flushright}

\lstinputlisting[language=bash,basicstyle=\small]{python_codes/fieldstone_01/keywords}

\begin{center}

\fbox{\textbf{\huge \color{teal} P}}
Codes at \url{https://github.com/cedrict/fieldstone/tree/master/python_codes/fieldstone_127}
\end{center}

\par\noindent\rule{\textwidth}{0.4pt}

{\sl This stone was developed in collaboration with Fanny garel}. \index{contributors}{F. Garel}

%--------------------------------------------------------------------------------------------------

The domain is $L_x \times L_y=200\times 50~\si{\km}$. Pure shear boundary conditions are imposed by 
applying a velocity $\vec\upnu=(+u_{bc},0)$ at the top and $\vec\upnu=(-u_{bc},0)$ at the bottom so 
that $\dot\varepsilon_{0}=\frac12 (2u_{bc}/L_y=u_{bc}/L_y)$.
The temperature is constant and set to $T_0$ in the domain.
Two materials are present: the background matrix (material 1) and a circular inclusion (material 2)
centered in the domain and of $10~\si{\km}$ radius with a constant viscosity $\eta_i=\SI{1e25}{\pascal\second}$.
Buoyancy forces are neglected so that density values are irrelevant.

Meshing is based on \stone~93. Crouzeix-Raviart elements are used. Pressure is normalised to be
on average zero over the domain.

Say smthg about NL iterations


\begin{center}
\includegraphics[width=10cm]{python_codes/fieldstone_127/results/mesh1}
\end{center}

The rheology of the matrix can take several forms:
\begin{itemize}

%-------------------------
\item {\python rheology=0}: Constant viscosity $\eta=\SI{1e25}{\pascal\second}$. This renders
the physics linear (and independent of temperature). Changing the velocity boundary conditions
only changes the magnitude of the computed fields.

%-------------------------
\item {\python rheology=1}: Diffusion creep only, with $Q_{\rm df}=\SI{410e3}{\joule\per\mole}$, 
$A_{\rm df}=1e-7$, no activation volume.
\[
\eta_{\rm df}=A_{\rm df}^{-1}\exp \frac{Q_{\rm df}}{RT}
\]

%-------------------------
\item {\python rheology=2}: Diffusion creep of rheology 1 + tanh formulation from \textcite{gatt20} 

\[
\sigma = (a_0+b_0T) \left( 1+ tanh( (a_1+b_1T)(\log_{10}(\dot\varepsilon_e)- (a_2+b_2T+c_2T^2)) )  \right)
\]

%-------------------------
\item {\python rheology=3}: Diffusion creep of rheology 1 + erf formulation from \textcite{gatt20} 


%-------------------------
\item {\python rheology=4}: Diffusion creep of rheology 1 + Dislocation creep from \textcite{gocg19} (2019):
\[
\eta_{\rm ds} = A_{\rm ds}^{-1/n} \dot\varepsilon_e^{-1+1/n_{\rm ds}}\exp \frac{Q_{\rm df}}{n_{\rm ds}RT}
\]
with $A_{\rm ds}=5.27e-29$, $n_{\rm ds}=4.5$, $Q_{\rm ds}=443e3$.

\end{itemize}



\newpage
%------------------------------------------------------------------------------
\subsection*{Results - rheology=0}

\begin{center}
\includegraphics[width=8cm]{python_codes/fieldstone_127/results/rheo0/u}
\includegraphics[width=8cm]{python_codes/fieldstone_127/results/rheo0/v}\\
\includegraphics[width=8cm]{python_codes/fieldstone_127/results/rheo0/press}
\includegraphics[width=8cm]{python_codes/fieldstone_127/results/rheo0/eta}\\
\includegraphics[width=8cm]{python_codes/fieldstone_127/results/rheo0/sr}
\end{center}

\newpage
%------------------------------------------------------------------------------
\subsection*{Results - rheology=1}

\begin{center}
\includegraphics[width=8cm]{python_codes/fieldstone_127/results/rheo1/u}
\includegraphics[width=8cm]{python_codes/fieldstone_127/results/rheo1/v}\\
\includegraphics[width=8cm]{python_codes/fieldstone_127/results/rheo1/press}
\includegraphics[width=8cm]{python_codes/fieldstone_127/results/rheo1/eta}\\
\includegraphics[width=8cm]{python_codes/fieldstone_127/results/rheo1/sr}
{\captionfont sr -14, T=1400}
\end{center}


