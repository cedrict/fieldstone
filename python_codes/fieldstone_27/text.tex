In what follows we will be re-doing the numerical experiments presented in 
Zhong et al. \cite{zhgh93}.

The first benchmark showcases a unit square domain with free slip 
boundary conditions prescribed on all sides.
The resolution is fixed to $64\times64$ $Q_1 \times P_0$ elements. 
The flow is isoviscous and the buoyancy force ${\bm f}$ is given by 
\begin{eqnarray}
f_x &=& 0 \nonumber\\
f_y &=& \rho_0 \alpha T(x,y) \nonumber
\end{eqnarray}
with the temperature field given by 
\[
T(x,y) = \cos(kx) \delta(y-y_0)
\]
where $k=2\pi/\lambda$ and $\lambda$ is a wavelength, 
and $y_0$ represents the location of the buoyancy strip.
We set $g_y=-1$ and prescribe $\rho(x,y)=\rho_0 \alpha \cos(kx) \delta(y-y_0)$ on the nodes
of the mesh.

One can prove (\cite{zhgh93} and refs. therein) that 
there is an analytic solution for the surface stress $\sigma_{zz}$
\footnote{Note that in the paper the authors use $\rho \alpha g$ which does not have the 
dimensions of a stress}
\[
\frac{\sigma_{yy}}{\rho \alpha g h} =
\frac{\cos (kx)}{\sinh^2(k)}
\left[
k(1-y_0)\sinh(k) \cosh(ky_0)-k \sinh(k(1-y_0))
+\sinh(k) \sinh(ky_0)
\right]
\]

We choose $\rho_0 \alpha = 64$, $\eta=1$ (note that in this case the 
normalising coefficient of the stress is exactly 1 (since $h=L_x/nelx=1/64)$ so it is not implemented in the code).
$\lambda=1$ is set to 1 and we explore $y_0 = \frac{63}{64},\frac{62}{64},\frac{59}{64}$ and $y_0=32/64$.
Under these assumptions the density field for $y_0=59/64$ is:
\begin{center}
\includegraphics[width=7cm]{python_codes/fieldstone_27/rho}
\end{center}

We can recover the stress at the boundary by computing 
the $yy$ component of the stress tensor in the top row of elements: 
\[
\sigma_{yy} = -p + 2 \eta \dot{\epsilon}_{yy}
\]
Note that pressure is by definition elemental, and that strain rate
components are then also computed in the middle of each element.

These elemental quantities can be projected onto the nodes (see section \ref{f_XX})
by means of the C$\rightarrow$N algorithm or a least square algorithm (LS). 

\begin{center}
\includegraphics[width=7cm]{python_codes/fieldstone_27/results/32_64/sigmazz.pdf}
\includegraphics[width=7cm]{python_codes/fieldstone_27/results/32_64/sigmazz_error.pdf}\\
\includegraphics[width=7cm]{python_codes/fieldstone_27/results/59_64/sigmazz.pdf}
\includegraphics[width=7cm]{python_codes/fieldstone_27/results/59_64/sigmazz_error.pdf}\\
\includegraphics[width=7cm]{python_codes/fieldstone_27/results/62_64/sigmazz.pdf}
\includegraphics[width=7cm]{python_codes/fieldstone_27/results/62_64/sigmazz_error.pdf}\\
\includegraphics[width=7cm]{python_codes/fieldstone_27/results/63_64/sigmazz.pdf}
\includegraphics[width=7cm]{python_codes/fieldstone_27/results/63_64/sigmazz_error.pdf}
\end{center}

The consistent boundary flux (CBF) method allows us to compute traction vectors ${\bm t}={\bm \sigma}\cdot{\bm n}$
on the boundary of the domain. On the top boundary, ${\bm n}=(0,1)$ so that ${\bm t}=(\sigma_{xy}, \sigma_{yy})^T$ and 
$t_y$ is the quantity we need to consider and compare to other results.

In the following table are shown the results presented in \cite{zhgh93} alongside the results obtained with Fieldstone:
\begin{center}
\begin{tabular}{l||llll}
\hline
Method             & $y_0=63/64$ & $y_0=62/64$ &  $y_0=59/64$
\footnote{The paper says 60/64 in the last column but it is in fact 59/64}
 & $y_0=32/64$\\ 
\hline
\hline
Analytic solution                   & 0.995476 & 0.983053  &  0.912506 & 0.178136 \\
Pressure smoothing \cite{zhgh93}    & 1.15974  & 1.06498   &  0.911109 & n.a. \\
CBF                \cite{zhgh93}    & 0.994236 & 0.982116  &  0.912157 & n.a. \\
\hline
\hline
fieldstone: elemental               & 0.824554 (-17.17 \%) & 0.978744 (-0.44\%) & 0.909574 (-0.32 \%) & 0.177771 (-0.20 \%)\\
fieldstone: nodal (C$\rightarrow$N) & 0.824554 (-17.17 \%) & 0.978744 (-0.44\%) & 0.909574 (-0.32 \%) & 0.177771 (-0.20 \%)\\
fieldstone: LS                      & 1.165321 ( 17.06 \%) & 1.070105 ( 8.86\%) & 0.915496 ( 0.33 \%) & 0.178182 ( 0.03 \%)\\
fieldstone: CBF                     & 0.994236 ( -0.13 \%) & 0.982116 (-0.10\%) & 0.912157 (-0.04 \%) & 0.177998 (-0.08 \%)\\
\hline
\end{tabular}
\end{center}
We see that we recover the published results with the same exact accuracy, thereby validating our implmentation.

On the following figures are shown the velocity, pressure and traction fields for two cases $y_0=32/64$ and $y_0=63/64$.
\begin{center}
\includegraphics[width=5cm]{python_codes/fieldstone_27/results/32_64/vel}
\includegraphics[width=5cm]{python_codes/fieldstone_27/results/32_64/p}
\includegraphics[width=5cm]{python_codes/fieldstone_27/results/32_64/tractions}\\
\includegraphics[width=5cm]{python_codes/fieldstone_27/results/59_64/vel}
\includegraphics[width=5cm]{python_codes/fieldstone_27/results/59_64/p}
\includegraphics[width=5cm]{python_codes/fieldstone_27/results/59_64/tractions}\\
\includegraphics[width=5cm]{python_codes/fieldstone_27/results/62_64/vel}
\includegraphics[width=5cm]{python_codes/fieldstone_27/results/62_64/p}
\includegraphics[width=5cm]{python_codes/fieldstone_27/results/62_64/tractions}\\
\includegraphics[width=5cm]{python_codes/fieldstone_27/results/63_64/vel}
\includegraphics[width=5cm]{python_codes/fieldstone_27/results/63_64/p}
\includegraphics[width=5cm]{python_codes/fieldstone_27/results/63_64/tractions}
\end{center}

Here lies the superiority of our approach over the one presented in the original article: 
our code computes all traction vectors on all boundaries at once.

\todo[inline]{explain how Medge is arrived at!}

\todo[inline]{compare with ASPECT ??!}

\todo[inline]{gauss-lobatto integration?}

\todo[inline]{pressure average on surface instead of volume ?}



\fbox{
\parbox{10cm}{{\bf features}
\begin{itemize}
\item $Q_1\times P_0$ element \index{$Q_1 \times P_0$}
\item incompressible flow \index{incompressible flow}
\item mixed formulation \index{mixed formulation}
\item isothermal \index{isothermal}
\item isoviscous \index{isoviscous}
\item analytical solution \index{analytical solution}
\item pressure smoothing \index{pressure smoothing} 
\item consistent boundary flux \index{CBF}
\end{itemize}
}}

