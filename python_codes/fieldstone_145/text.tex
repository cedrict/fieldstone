\begin{flushright} {\tiny {\color{gray} python\_codes/fieldstone\_145/text.tex}} \end{flushright}

\lstinputlisting[language=bash,basicstyle=\small]{python_codes/fieldstone_145/keywords.key}

\begin{center}

\fbox{\textbf{\huge \color{teal} P}}
Code at \url{https://github.com/cedrict/fieldstone/tree/master/python_codes/fieldstone_145}
\end{center}

\par\noindent\rule{\textwidth}{0.4pt}

{\sl This stone was developed in collaboration with Frederic Gueydan}. \index{contributors}{F. Gueydan}

\par\noindent\rule{\textwidth}{0.4pt}
%%%%%%%%%%%%%%%%%%%%%%%%%%%%%%%%%%%%%%%%%%%%%%%%%%%%%%%%%%%%%%%%%%%%%%%%%%%%%%%%%%%%%%%%%%%%%%

This \stone is a simple example of how one can easily read in (p)vtu files, extract the raw data for the fields and 
then for example post-process them, or plot them in a different way.
I cannot predict your exact requirements when it comes to colorscale, aspect ratio, legends, types of data, etc ...
so you will need to figure a few things out on your own. 
