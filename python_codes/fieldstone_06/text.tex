
\includegraphics[height=1.5cm]{images/pictograms/benchmark}

\lstinputlisting[language=bash,basicstyle=\small]{python_codes/fieldstone_06/keywords.ascii}

\begin{center}
Code at \url{https://github.com/cedrict/fieldstone/tree/master/python_codes/fieldstone_06}
\end{center}

\par\noindent\rule{\textwidth}{0.4pt}
%%%%%%%%%%%%%%%%%%%%%%%%%%%%%%%%%%%%%%%%%%%%%%%%%%%%%%%%%%%%%%%%%%%%%%%%%%%%%%%%%%%%%%%%%

The SolKz benchmark \cite{repa87} is similar to the SolCx benchmark.
but the viscosity is now a function of the space coordinates: 
\begin{equation}
\eta(y)=\exp(By) \quad {\rm with} \quad B=13.8155
\end{equation}
It is however not a discontinuous function but grows exponentially with the vertical coordinate so that its overall variation is again $10^6$. 
The forcing is again chosen by imposing a spatially variable density variation as follows:
\begin{equation}
\rho(x,y)=\sin(2y) \cos(3\pi x)
\end{equation}
Free slip boundary conditions are imposed on all sides of the domain.
This benchmark is presented in Zhong (1996) \cite{zhon96} as well and is studied 
in Duretz \etal (2011) \cite{dumg11} and Gerya \etal (2013) \cite{gemd13}.

\begin{center}
\includegraphics[width=14cm]{python_codes/fieldstone_06/results/solution.pdf}
\end{center}

\begin{center}
\includegraphics[width=10cm]{python_codes/fieldstone_06/results/errors.pdf}\\
{\captionfont Velocity and pressure error as a function of mesh size.}
\end{center}
