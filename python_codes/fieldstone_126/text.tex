\begin{flushright} {\tiny {\color{gray} python\_codes/fieldstone\_126/text.tex}} \end{flushright}

%\lstinputlisting[language=bash,basicstyle=\small]{python_codes/fieldstone_125/keywords}

\begin{center}

\fbox{\textbf{\huge \color{teal} P}}
Code at \url{https://github.com/cedrict/fieldstone/tree/master/python_codes/fieldstone_126}
\end{center}

\par\noindent\rule{\textwidth}{0.4pt}

%%%%%%%%%%%%%%%%%%%%%%%%%%%%%%%%%%%%%%%%%%%%%%%%%%%%%%%%%%%%%%%%%%%%%%%%%%%%%%%%%%%%%%%%%%%%%


What follows is borrowed from \textcite{grfr03} (2003).
Two main relations are used, expressing fluid mass conservation
\[
\frac{\partial (\rho \upphi)}{\partial t} + \vec\nabla\cdot (\rho \vec{\upnu}) = f
\]
and Darcy's law
\[
\vec\upnu = - {\bm K} \cdot (\vec{\nabla} P + \rho \vec{g})
\]
where $\rho$ is the density of the fluid, $\upphi$ is the
connected mean porosity of the rock, $\vec{\upnu}$ is the Darcy flow
rate (fluid particle velocity times porosity)\footnote{Whaaaat?}, ${\bm K}$ is the
permeability tensor, $P$ is the total pore pressure and $\vec{g}$ is
the gravity.

First, during the evolution, we consider that the density 
$\rho = \rho' + \rho_m$ will evolve in time of an amount $\rho'$ 
around its local mean value $\rho_m$ , and finally, we rewrite the fluid mass
conservation by neglecting the terms $\rho'$ compared to $\rho_m$.

Then,
\[
\frac{\partial ((\rho'+\rho_m) \upphi)}{\partial t} + \vec\nabla\cdot ((\rho'+\rho_m) \vec{\upnu}) = f
\]
\[
(\rho'+\rho_m) \frac{\partial \upphi}{\partial t} 
+\upphi \frac{\partial (\rho'+\rho_m)}{\partial t} 
+ \vec\nabla\cdot ((\rho'+\rho_m) \vec{\upnu}) = f
\]
Since $\rho_m$ is constant then $\partial_t \rho_m =0$ so we obtain
\[
\rho_m \frac{\partial \upphi}{\partial t} 
+\upphi \frac{\partial \rho' }{\partial t} 
+ \vec\nabla\cdot (\rho_m \vec{\upnu}) = f
\]


Second, we consider that the total pore pressure $P$ can
be decomposed as $P = p + P_{hydro}$ such that $P_{hydro}$ is the
hydrostatic pressure verifying $\vec\nabla P_{hydro} + \rho_m \vec{g} = \vec{0}$. 
Then 
\begin{align}
\vec\upnu &= {\bm K} \cdot (\vec{\nabla} (p + P_{hydro}) + (\rho'+\rho_m) \vec{g}) \nn\\
\Rightarrow \vec\upnu &= {\bm K} \cdot (\vec{\nabla} p + \rho' \vec{g}) \nn
\end{align}



Furthermore, JUSTIFY!!\footnote{Why does grad rho' g disappear ?!? is it because 
it is negligible in our context compared to pressure gradients?}
\[
\upphi \frac{\partial \rho'}{\partial t}
=
\rho_m C_f  \upphi \frac{\partial p}{\partial t}
\]
so in the end 
\begin{equation}
\rho_m C_f  \upphi \frac{\partial p}{\partial t}
+ \rho_m \frac{\partial \upphi}{\partial t} 
- \vec\nabla \cdot (\rho_m {\bm K} \cdot \vec{\nabla} p ) ) = f
\end{equation}

We assume an isotropic permeability so that the tensor now simply writes:
\[
K = K_0 \upphi^3
\]
and then 
\begin{equation}
\rho_m C_f  \upphi \frac{\partial p}{\partial t}
+ \rho_m \frac{\partial \upphi}{\partial t} 
- \vec\nabla \cdot (\rho_m K \cdot \vec{\nabla} p ) ) = f
\end{equation}
The discretised weak form takes the form
\[
\underbrace{
\left( \int_\Omega \vec{\bN}^T \rho_m C_f \upphi \vec{\bN} dV  \right)
}_{\bm M}
 \cdot \frac{\partial \vec{\cal P}}{\partial t}
+
\underbrace{
\left( \int_\Omega {\bm B}^T \cdot \rho_m {\bm K} \cdot {\bm B} dV \right)
}_{{\bm K}_d}
 \cdot \vec{\cal P}
=
\underbrace{
\int_\Omega \vec{\bN}^T f dV 
-
\int_\Omega \vec{\bN}^T   \rho_m \frac{\partial \upphi}{\partial t} dV
}_{\vec{b}}
\]


The porosity $\upphi$ evolution equation is given by:
\[
\frac{\partial \upphi}{\partial t} = \alpha \dot{\varepsilon}_e
\qquad
\Rightarrow
\qquad
\upphi^{t+\delta t} = \upphi^t + \alpha \dot{\varepsilon}_e \delta t
\]
while the source term $f$ equation is:
\[
\frac{\partial f}{\partial t} = \varphi f \dot{\varepsilon}_e
\qquad
\Rightarrow
\qquad
f^{t+\delta t} = f^t (1 + \varphi  \dot{\varepsilon}_e \delta t)
\]
Likewise the time derivative is discretised by means of a simple explict 
scheme:
\[
\frac{\partial \vec{\cal P}}{\partial t} \simeq
\frac{ \vec{\cal P}^{t+\delta t} - \vec{\cal P}^t  }{\delta t}
\]

We arrive at 
\[
( {\bm M} + {\bm K}_d) \cdot \vec{\cal P}^{t+\delta t} = {\bm M} \cdot \vec{\cal P}^t  + \vec{b}
\]
The matrix is symmetric positive definite which is good news.

The PDE above contains two parameters $\phi$ (and therefore also $K$) and $f$
which depend on the strainrate. 
The strainrate field will ultimately come from the other code which 
solves the Stokes equations in the shear band. The timestep of that code 
is of the order of at least a few thousands of years. However the characteristic time step
for Darcy is about a year. This means that in the span of 1 tectonic timestep
the Darcy equations will be solved with $\upphi$ and $f$ constant. 
This simplifies things considerably since this renders the Darcy equation a linear 
diffusion equation with a heat source. 

In the end, when we incorporate the Darcy solver in the Stokes code
we may even choose to solve the steady-state Darcy equation once per 
tectonic time step! 

Given the nature of the equation describing the evolution of the porosity
it can and will ultimately exceed physically meaningful values so we 
implement a cutoff value $\upphi_{max}$.


characteristic time
\[
t_c 
= \frac{\upphi C_f h^2}{K} 
= \frac{\upphi^{-2} C_f h^2}{K_0} 
\]


TODO:
- $f$ constant, derive analytical soltuion, compare with ss results


dilatancy a la chpe15 ?
