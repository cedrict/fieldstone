
For now what follows only deals with viscous behavior.


\subsubsection{Linear viscous aka Newtonian} \index{Newtonian fluid}

Simply put, a Newtonian fluid is a fluid in which the viscous stresses at every point are linearly proportional 
to the local strain rate.
Mathematically speaking, this means that the fourth-order tensor ${\bm C}$ relating the viscous stress 
tensor to the strain rate tensor does not depend on the stress state and velocity of the flow.
\[
{\bm s}={\bm C} \cdot \dot{\bm \varepsilon}
\]

One very often make sthe assumption that the fluid is isotropic, i.e. its mechanical properties are the 
same along any direction. As a consequence the fourth order viscosity tensor 
${\bm C}$ is symmetric and will have only two independent real parameters: 
a bulk viscosity coefficient, that defines the resistance of the medium to gradual uniform compression; 
and a dynamic viscosity coefficient $\eta$ that expresses its resistance to gradual shearing, 
(we here neglect the so-called rotational viscosity coefficient which results from a coupling between the fluid flow and the rotation of the individual particles). %wiki



Rather logically we denote by non-Newtonian fluids with are not Newtonian, i.e. their viscosity (tensor)
depends on stress. Such fluids are part of our daily life, e.g. honey, toothpaste, paint, blood, and shampoo. 
 


%------------------------------
\subsubsection{Power-law model}
\index{Power-law model}

One of the simplest non-Newtonian viscosity model is the power-law model:
\begin{equation}
\eta = K \dot{\varepsilon}_{II}^{(n-1)/2}
\end{equation}
where $\dot{\varepsilon}_{II}$ is the second invariant of the strain rate tensor as defined in 
Section~\ref{sec:invariants}, and $n$ and $K$ are parameters. $n$ is called the power-law index.

Note that a Newtonian viscosity is recovered when $n=1$. Also $n$ and $K$ may depend on temperature
\cite[p339]{reddybook2}.


%------------------------------
\subsubsection{Carreau model}



%------------------------------
\subsubsection{Bingham model}

%------------------------------
\subsubsection{Dislocation creep}

%------------------------------
\subsubsection{Diffusion creep}

%------------------------------
\subsubsection{Peierls creep}



