
For now what follows only deals with viscous behavior.


\subsubsection{Linear viscous aka Newtonian} \index{Newtonian fluid}

Simply put, a Newtonian fluid is a fluid in which the viscous stresses at every point are linearly proportional 
to the local strain rate.
Mathematically speaking, this means that the fourth-order tensor ${\bm C}$ relating the viscous stress 
tensor to the strain rate tensor does not depend on the stress state and velocity of the flow.
\[
{\bm s}={\bm C} \cdot \dot{\bm \varepsilon}
\]

One very often make sthe assumption that the fluid is isotropic, i.e. its mechanical properties are the 
same along any direction. As a consequence the fourth order viscosity tensor 
${\bm C}$ is symmetric and will have only two independent real parameters: 
a bulk viscosity coefficient, that defines the resistance of the medium to gradual uniform compression; 
and a dynamic viscosity coefficient $\eta$ that expresses its resistance to gradual shearing, 
(we here neglect the so-called rotational viscosity coefficient which results from a coupling between the fluid flow and the rotation of the individual particles). %wiki



Rather logically we denote by non-Newtonian fluids with are not Newtonian, i.e. their viscosity (tensor)
depends on stress. Such fluids are part of our daily life, e.g. honey, toothpaste, paint, blood, and shampoo. 
 


%------------------------------
\subsubsection{Power-law model}
\index{Power-law model}

One of the simplest non-Newtonian viscosity model is the power-law model:
\begin{equation}
\eta = K \dot{\varepsilon}_{II}^{(n-1)/2}
\end{equation}
where $\dot{\varepsilon}_{II}$ is the second invariant of the strain rate tensor as defined in 
Section~\ref{sec:invariants}, and $n$ and $K$ are parameters. $n$ is called the power-law index.

Note that a Newtonian viscosity is recovered when $n=1$. Also $n$ and $K$ may depend on temperature
\cite[p339]{reddybook2}.


%------------------------------
\subsubsection{Carreau model}



%------------------------------
\subsubsection{Bingham model}


%------------------------------
\subsubsection{Herschel-Bulkley visco-plastic model}

The Herschel-Bulkley model is effectively a combination of the power-law model and 
a simple plastic model:
\begin{eqnarray}
{\bm s} &=& 2 \left(  K \dot{\varepsilon}^{n-1} + \frac{\tau_0}{\dot{\varepsilon}}\right)\dot{\bm \varepsilon} \qquad \text{ if } {s}_{II}>\tau_0 \\
\dot{\bm \varepsilon} &=& {\bm 0} \qquad {s}_{II} \leq \tau_0 \\
\end{eqnarray}
in which $\dot{\varepsilon}=\sqrt{\dot{\varepsilon}_{II}}$, 
$\tau_0$ is the yield stress, $K$ the consistency, and $n$ is the flow index \cite{bemj04}.
The flow index measures the degree to which the fluid is shear-thinning ($n<1$) or shear-thickening ($n>1$).
If $n=1$ and $\tau_0=0$ the model reduces to the Newtonian model. 

The term between parenthesis above is the nonlinear effective viscosity. Concretely, the implementation goes as 
follows\footnote{\url{https://en.wikipedia.org/wiki/Herschel-Bulkley_fluid}}:
\[
\eta_{eff} = 
\left\{
\begin{array}{cc}
\eta_0 & \dot{\varepsilon}\leq \dot{\varepsilon}_0 \\ 
K \dot{\varepsilon}^{n-1} + \frac{\tau_0}{\dot{\varepsilon}} & \dot{\varepsilon} \geq \dot{\varepsilon}_0
\end{array}
\right.
\]
The limiting viscosity $\eta_0$ is chosen such that 
$\eta_0 =  K \dot{\varepsilon}_0^{n-1} + \frac{\tau_0}{\dot{\varepsilon}_0}$

A large limiting viscosity means that the fluid will only flow in response to a large applied force. 
This feature captures the Bingham-type behaviour of the fluid. 
Note that when strain rates are large, the power-law behavior dominates. 

As we have seen for Bingham fluids, the equations above are not easily amenable to implementation so that 
one usually resorts to regularisation, which is a modification of the 
equations by introducing a new material parameter which controls the exponential 
growth of stress. This way the equation is valid for both yielded and unyielded areas
\cite{blmi97,papa87}:
\[
\eta_{eff} = K \dot{\varepsilon}^{n-1} + \frac{\tau_0}{\dot{\varepsilon}} [1 - \exp(-m \dot{\varepsilon})] 
\]
When the strain rate becomes (very) small a Taylor expansion of the regularisation 
term yields $1- \exp(-m \dot{\varepsilon}) \sim m \dot{\varepsilon} $ so that 
$\eta_{eff} \rightarrow m \tau_0$.

\todo[inline]{Channel flow of wikipedia with analytical solution!}


%------------------------------
\subsubsection{Dislocation creep}

%------------------------------
\subsubsection{Diffusion creep}

%------------------------------
\subsubsection{Peierls creep}



