\begin{flushright} {\tiny {\color{gray} \tt mms\_phdo19.tex}} \end{flushright}
%~~~~~~~~~~~~~~~~~~~~~~~~~~~~~~~~~~~~~~~~~~~~~~~~~~~~~~~~~~~~~~~~~~~~~~~~~~~~~~~~~~~~~~~~~~~~~~~~~~

This benchmark is presented in Phillips \etal \cite{phdo19}. 
Inner radius is 1, outer radius is 3.
The right hand side term of the Poisson equation is given by:
\begin{eqnarray}
f(r,\theta,\phi)&=&
\frac{\sin^2\theta }{R^2} 
\left[
(\cos\phi-\sin\phi)(20\sin^2\theta -15) -\sin 2\phi (10\sin^2 \theta -6)
\right] \nn\\
&\times & \left[ \left(\frac{r}{R_{inner}}\right)^2 -1  \right]
\left[ \left(\frac{r}{R_{outer}}\right)^2 -1  \right] \nn\\
&+&\sin^4 \theta
\left[ \cos\phi -\sin\phi -\frac12 \sin 2\phi \right]
\left[
\frac{20r^2}{R_{inner}^2R_{outer}^2} -6 \left( \frac{1}{R_{inner}^2}+\frac{1}{R_{outer}^2}  \right)
\right]
\end{eqnarray}
Note: what is $R$ here??

The solution is then 
\[
T(r,\theta,\phi) = \sin^4 \theta \left[ \cos\phi -\sin\phi -\frac12 \sin 2\phi \right]
\left[ \left(\frac{r}{R_{inner}}\right)^2 -1  \right]
\left[ \left(\frac{r}{R_{outer}}\right)^2 -1  \right]
\]
This expression is used to generate the Dirichlet boundary conditions on the inner and outer surfaces.




