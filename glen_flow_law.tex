
Ice and rocks share similarities in terms of (viscous) rheology.
Glen's law is the most commonly used flow law for ice in glaciers and ice sheets \cite{glen55}
and it is actually a power-law type rheology:
\[
\dot{\bm \varepsilon} = A {\bm \tau}^n 
\]
with $n\sim 3$ and $A\sim 2.4\cdot 10^{-24} \text{Pa}^{-3}\cdot \text{s}^{-1}$ at $0\degree$C.
The effective viscosity is then given by
\[
\eta = \frac{1}{2 A \tau_e^{n-1}} 
\]
\begin{center}
\includegraphics[height=5cm]{images/rheology/glen}
\includegraphics[height=5cm]{images/rheology/goko01}\\
{\captionfont Left: Taken from Glen \cite{glen55}; Right: taken from \cite{goko01}.}
\end{center}
Most of these studies suggest values of the power-law exponent $n\sim 2-4$, and there seems to be 
a general indication that the exponent is lower at lower stresses.

The $A$ coefficient above has been found to depend on temperature and is reasonably described 
with an Arrhenius law:
\[
A(T)=A_0 \exp\left( -\frac{Q}{RT} \right)
\]
A standard formulation is the Paterson-Budd law with a fixed Glen exponent $n=3$ and 
a split Arrhenius term \cite{pabu82}:
\[
A=3.615 \cdot 10^{-13} \text{Pa}^{-3}\cdot \text{s}^{-1}, 
\qquad Q=60 \; \text{kJ}/\text{mol}, \qquad if\quad T<263\text{K} 
\]
\[
A=1.733\cdot 10^{3} \text{Pa}^{-3}\cdot \text{s}^{-1}, 
\qquad  Q=139 \; \text{kJ}/\text{mol}, \qquad if\quad T>263\text{K}
\]
Be careful that in these two equations the temperature $T$ is the pressure-adjusted 
temperature \cite{pabu82}.
Note that $A$ is also affected by the water content and the presence of impurities. 

Finally, Glen's law is the standard rheology used for ice-sheet modelling 
but it does not account for the complex evolution of fabric and resulting anisotropy.
Indeed the grain size evolution (growth \& reduction)
plays a large role in the rheology \textcite{begh21} (2021).
See \stone~59. 


\Literature 
\textcite{alle92} (1992),
\textcite{grev97} (1997),
\textcite{grbl09} (2009),
\textcite{issg15} (2015),
\textcite{krab16} (2016),
\textcite{jidb17} (2017),
\textcite{heah18} (2018),
Very mathematics heavy papers: \cite{jora11,chgp13}.


