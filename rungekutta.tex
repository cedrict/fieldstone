These methods were developed around 1900 by the German mathematicians Carl Runge and Martin Kutta.
The RK methods are methods for the numerical integration of ODEs. These methods are well 
documented in any numerical analysis textbook and the reader is referred to \cite{gery10,tack10}.






Any Runge-Kutta method is uniquely identified by its Butcher tableau.


The following method is called the Runge-Kutta-Fehlberg method and is 
commonly abbreviated RKF45. Its Butcher tableau is as follows: 

\begin{tabular}{c|cccccc}
0 & \\
1/4 	&1/4\\ 
3/8 	&3/32 		&9/32 \\
12/13 	&1932/2197 	&-7200/2197 &	7296/2197\\
1 	&439/216 	&-8 	&3680/513 &	-845/4104\\
1/2 	&-8/27 		&2 	&-3544/2565& 	1859/4104 &	-11/40 	\\
\hline
&16/135 	&0 		&6656/12825 	&28561/56430 	&-9/50& 	2/55\\
&25/216 	&0 	&1408/2565 	&2197/4104 	&-1/5 	&0 
\end{tabular}


The first row of coefficients at the bottom of the table gives the fifth-order accurate method, and the second row gives the fourth-order accurate method. 
