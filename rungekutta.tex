These methods were developed around 1900 by the German mathematicians Carl Runge and Martin Kutta.
The RK methods are methods for the numerical integration of 
ODEs\footnote{\url{https://en.wikipedia.org/wiki/Runge-Kutta_methods}}. These methods are well 
documented in any numerical analysis textbook and the reader is referred to \cite{gery10,tack10}.
Any Runge-Kutta method is uniquely identified by its Butcher tableau (REF?) which contains 
all necessary coefficients to build the algorithm.

The simplest Runge–Kutta method is the (forward) Euler method. Its tableau is:

\begin{tabular}{c|c}
0 & \\
\hline
 & 1
\end{tabular}

\index{midpoint method} \index{RK2}
The standard second-order RK method method (also called midpoint method) is:

\begin{tabular}{c|cccccc}
0 & \\
1/2 & 1/2 \\
\hline
 & 0 & 1 
\end{tabular}

\index{Heun's emthod}
Another second-order RK method, called Heun's 
method\footnote{\url{https://en.wikipedia.org/wiki/Heun's_method}} is follows:

\begin{tabular}{c|cccccc}
0 & \\
1 & 1 \\
\hline
 & 1/2 & 1/2 
\end{tabular}

\index{RK3}
A third-order RK method is as follows:

\begin{tabular}{c|ccccc}
0 & \\
1/2 & 1/2 \\
1 & -1 & 2 \\ 
\hline
 & 1/6 & 4/6  & 1/6
\end{tabular}


\index{RK4}
The RK4 method falls in this framework. Its tableau is:

\begin{tabular}{c|cccccc}
0 & \\
1/2 & 1/2 \\
1/2 & 0 & 1/2 \\
1 & 0 & 0 & 1 \\
\hline
 & 1/6 & 1/6 & 1/3 & 1/6 
\end{tabular}

A slight variation of the standard RK4 method is also due to Kutta in 1901 and is called the 3/8-rule. 
Almost all of the error coefficients are smaller than in the standard method but it requires slightly more FLOPs 
per time step. Its Butcher tableau is

\begin{tabular}{c|cccccc}
0 & \\
1/3 & 1/3 \\
2/3 & -1/3 & 1 \\
1 & 1 & -1 & 1 \\
\hline
 & 1/8 & 3/8 & 3/8 & 1/8 
\end{tabular}


\index{RK45} \index{Runge-Kutta-Fehlberg method}
The following method is called the Runge-Kutta-Fehlberg method and is 
commonly abbreviated 
RKF45\footnote{\url{https://en.wikipedia.org/wiki/Runge-Kutta-Fehlberg_method}}
. Its Butcher tableau is as follows: 

\begin{tabular}{c|cccccc}
0 & \\
1/4 	&1/4\\ 
3/8 	&3/32 		&9/32 \\
12/13 	&1932/2197 	&-7200/2197 &	7296/2197\\
1 	&439/216 	&-8 	&3680/513 &	-845/4104\\
1/2 	&-8/27 		&2 	&-3544/2565& 	1859/4104 &	-11/40 	\\
\hline
&16/135 	&0 		&6656/12825 	&28561/56430 	&-9/50& 	2/55\\
&25/216 	&0 	&1408/2565 	&2197/4104 	&-1/5 	&0 
\end{tabular}


The first row of coefficients at the bottom of the table gives the fifth-order accurate method, and the second row gives the fourth-order accurate method. 

%....................................................................................
\subsubsection{Using RK methods to advect particles/markers \label{sec:rkparticles}}

In the context of geodynamical modelling, one is usually confronted to the following problem:
now that I have a velocity field on my FE mesh, how can I use it to advect the Lagrangian 
markers?

Runge-Kutta methods are used to this effect but only their spatial component is used:
the velocity solution is not recomputed at the intermediate fractional timesteps, i.e. 
only the coefficients of the right hand side of the tableaus is used.

The RK1 method is simple. Carry out a loop over markers and 
\begin{enumerate}
\item interpolate velocity $\vec\upnu_{m}$ onto each marker $m$
\item compute new position as follows: $\vec r_m(t+\delta t)=\vec r_m(t) + \vec\upnu_m \delta t$
\end{enumerate}

The RK2 method is also simple but requires a bit more work. Carry out a loop over markers and 
\begin{enumerate}
\item interpolate velocity $\vec\upnu_{m}$ onto each marker $m$ at position $\vec r_m$
\item compute new intermediate position as follows: $\vec r_m^{(1)}(t+\delta t)=\vec r_m(t) + \vec\upnu_m \delta t/2$
\item compute velocity $\vec\upnu_{m}^{(1)}$ at position $\vec r_m^{(1)}$
\item compute new position: $\vec r_m(t+\delta t)=\vec r_m(t) + \vec\upnu_m^{(1)} \delta t$ 
\end{enumerate}
Note that the intermediate positions could be in a different element of the mesh so extra care must be taken when 
computing intermediate velocities. 

The RK3 method introduces two intermediate steps. Carry out a loop over markers and 
\begin{enumerate}
\item interpolate velocity $\vec\upnu_{m}$ onto each marker $m$ at position $\vec r_m$
\item compute new intermediate position as follows: $\vec r_m^{(1)}(t+\delta t)=\vec r_m(t) + \vec\upnu_m \delta t/2$
\item compute velocity $\vec\upnu_{m}^{(1)}$ at position $\vec r_m^{(1)}$
\item compute new intermediate position as follows: 
$\vec r_m^{(2)}(t+\delta t)=\vec r_m(t) + (2\vec\upnu_m^{(1)}-\vec\upnu_m) \delta t/2$
\item compute velocity $\vec\upnu_{m}^{(2)}$ at position $\vec r_m^{(2)}$
\item compute new position: 
$\vec r_m(t+\delta t)=\vec r_m(t) + (\vec\upnu_m +4 \vec\upnu_m^{(1)} + \vec\upnu_m^{(2)}    )\delta t/6$ 
\end{enumerate}

The following example is borrowed from \cite{maie12}, itself borrowed from Fullsack \cite[Section 5.4]{full95}.
It is a whirl flow \cite{otti89}, a flow with rotational symmetry in which concentric layers of material
rotate around  a centre with an angular velocity:
\[
\omega(r)= \omega_0 \frac{r}{r_0} \exp\left(-\frac{r}{r_0}  \right)
\]  
The box is $[-0.5,0.5]\times[-0.5,0.5]$, $r_0=0.25$, $\omega_0=0.3$ and $\delta t=1$. 
$60\times 60$ particles are regularly positioned inside the $[-0.3,0.3]\times[-0.3,0.3]$ square.
Maierova \cite{maie12} has carried out this experiment for the above Runge-Kutta methods.

\begin{center}
\includegraphics[height=4cm]{images/rk/maie12a}\\
{\small Model domain with particles colored at three
different time-steps: (A) t = 0 (initial position of particles), (B) t = 50, and (C) t = 200.
The advection is computed using the fourth-order Runge-Kutta scheme. Taken from \cite{maie12}}
\end{center}

\begin{center}
\includegraphics[height=4cm]{images/rk/maie12b}
\includegraphics[height=4cm]{images/rk/maie12c}\\
{\small The same plot as above, but for different advection schemes at t = 100.
Advection was computed using (A) the fourth-order Runge-Kutta scheme, (B) the mid-
point method, (C) Heun’s method and (D) the explicit Euler method. Taken from \cite{maie12}}
\end{center}






