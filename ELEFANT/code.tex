 \subsubsection{assign\_values\_to\_qpoints.f90}

 \subsubsection{basis\_functions\_V.f90}
 This file contains 3 functions: 
 \subsubsection{compute\_dNdx\_dNdy.f90}
 This subroutine computes $\partial{\bN}/\partial x$ and $\partial{\bN}/\partial y$
 at a location $r,s$ passed as argument.
 \subsubsection{compute\_elemental\_matrices.f90}

 \subsubsection{interpolate\_onto\_nodes.f90}

 \subsubsection{make\_matrix.f90}
 This subroutine loops over all elements, build their elemental matrices and rhs, 
 apply the boundary conditions ont these elemental matrices, and then 
 assembles them in the global matrix, either in CSR or in MUMPS format.
 \subsubsection{markers\_setup.f90}
 REDO and rebase on element, local coords and basis fcts
 \subsubsection{output\_mesh.f90}
 This subroutine produces the {\filenamefont meshV.vtu} file which only 
 contains the corner nodes.
 \subsubsection{output\_qpoints}

 \subsubsection{output\_swarm.f90}
 This subroutine produces the {\filenamefont markers.vtu} file which contains the 
 swarm of particles with all their properties.
 \subsubsection{postprocessors.f90}

 \subsubsection{prescribe\_stokes\_solution.f90}

 \subsubsection{quadrature\_setup.f90}

 \subsubsection{setup\_cartesian2D.f90}
 
 \subsubsection{setup\_cartesian3D.f90}
 
 \subsubsection{solve\_stokes}

 \subsubsection{template}

