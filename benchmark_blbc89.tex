\begin{flushright} {\tiny {\color{gray} benchmark\_blbc89.tex}} \end{flushright}
%~~~~~~~~~~~~~~~~~~~~~~~~~~~~~~~~~~~~~~~~~~~~~~~~~~~~~~~~~~~~~~~~~~~~~~~~~~~~~~~~~~~~~~~~~~~~~~~~~~

The abstract of the original publication by Blankenbach \etal (1989) \cite{blbc89} reads:
\begin{center}
{\it 
We have carried out a comparison study for a set of benchmark problems 
which are relevant for convection in the Earth's mantle. The cases comprise 
steady isoviscous convection, variable viscosity convection and time-dependent 
convection with internal heating. We compare Nusselt numbers, velocity, 
temperature, heat-flow , topography and geoid data. Among the applied codes 
are finite-difference, finite-element and spectral methods. In a synthesis 
we give best estimates of the `true' solutions and ranges of uncertainty. We
recommend these data for the validation of convection codes in the future.
}
\end{center}

The temperature is fixed to zero on top and to $\Delta T$ at the bottom, 
with reflecting symmetry at the sidewalls (i.e. $\partial_x T=0$) 
and there are no internal heat sources. 
Free-slip conditions are implemented on all boundaries. 

The Rayleigh number is given by
\[
\Ranb = \frac{\alpha g_y \Delta T h^3 }{\kappa \nu}
=\frac{\alpha g_y \Delta T h^3 \rho^2 c_p}{k \mu}
\]
The initial temperature field is given by 
\[
T(x,y)=(1-y) - 0.01\cos(\pi x) \sin(\pi x)
\]
The perturbation in the initial temperature fields leads to 
a perturbation of the density field and sets the fluid in motion. 

Depending on the initial Rayleigh number, the system ultimately reaches a 
steady state after some time. 

\begin{center}
a)\includegraphics[width=4.5cm]{images/benchmark_blbc89/temp1a}
b)\includegraphics[width=4.5cm]{images/benchmark_blbc89/temp1b}
c)\includegraphics[width=4.5cm]{images/benchmark_blbc89/temp1c}\\
{\captionfont Temperature fields at steady-state for 
$\Ranb=10^4$ (a), $\Ranb=10^5$ (b), $\Ranb=10^6$ (c).
Obtained with \elefant code \cite{thie14}.}
\end{center}


\begin{center}
a)\includegraphics[width=13cm]{images/benchmark_blbc89/krhb12a}
b)\includegraphics[width=13cm]{images/benchmark_blbc89/krhb12b}\\
{\captionfont 
a) Results for the 2-D benchmark problem with uniform mesh refinement. 
\# DoFs indicates the number of degrees of freedom.
Reference results from Blankenbach \etal (1989).
b) Results with adaptive mesh refinement. The number of degrees of freedom (\# DoFs) for
each finest mesh size $h$ varies between time steps; 
the indicated numbers provide a typical range.
Note that these are obtained for $\Ranb=216,000$.}
\end{center}


%\begin{center}
%\begin{tabular}{llcccc}
%\hline
%          &           & Blankenbach \etal (1989) &  Tackley (1994)  & King (2009) &   \\
%          &           & \cite{blbc89} &  \cite{tack94}  & \cite{king09}  & \elefant  \\
%\hline
%\hline
%$\Ranb=10^4$ & $\upnu_{rms}$ &  $42.864947  \pm 0.000020$ & 42.775 (0.2\%) & 42.867  (0.005\%) & 42.867 (0.01\%) \\ 
%             & $\Nunb$       &  $4.884409   \pm 0.000010$ & 4.878  (0.1\%) & 4.885   (0.02\%)  & 4.882  (0.05\%)\\
%$\Ranb=10^5$ & $\upnu_{rms}$ &  $193.21454  \pm 0.00010 $ & 193.11 (0.05\%)& 193.248 (0.02\%)  & 193.255 (0.02\%)\\
%             & $\Nunb$       &  $10.534095  \pm 0.000010$ & 10.531 (0.03\%)& 10.536  (0.02\%)  & 10.507 (0.26\%) \\
%$\Ranb=10^6$ & $\upnu_{rms}$ &  $833.98977  \pm 0.00020 $ & 833.55 (0.05\%)& 834.353 (0.04\%)  & 834.712   (0.08\%)\\
%             & $\Nunb$       &  $21.972465  \pm 0.000020$ & 21.998 (0.1\%)& 21.981  (0.04\%)  & 21.695  (1.2\%)\\
%\hline
%\end{tabular} \\
%{\captionfont Steady state Nusselt number and Vrms measurements as reported in the literature and 
%obtained with \elefant on a $200\times 200$ grid.}
%\end{center} 


\begin{center}
\begin{tabular}{llll}
\hline
                                       &              & $\upnu_{rms}$             & $\Nunb$                   \\
\hline
Blankenbach \etal (1989) \cite{blbc89} & $\Ranb=10^4$ & $42.864947  \pm 0.000020$ & $4.884409   \pm 0.000010$ \\
                                       & $\Ranb=10^5$ & $193.21454  \pm 0.00010 $ & $10.534095  \pm 0.000010$ \\
                                       & $\Ranb=10^6$ & $833.98977  \pm 0.00020 $ & $21.972465  \pm 0.000020$ \\
\hline
Tackley (1994) \cite{tack94}           & $\Ranb=10^4$ & 42.775                    & 4.878  \\
                                       & $\Ranb=10^5$ & 193.11                    & 10.531 \\
                                       & $\Ranb=10^6$ & 833.55                    & 21.998 \\
\hline
King (2009) \cite{king09}              & $\Ranb=10^4$ & 42.867                    & 4.885   \\
                                       & $\Ranb=10^5$ & 193.248                   & 10.536  \\
                                       & $\Ranb=10^6$ & 834.353                   & 21.981  \\
\hline
Thieulot (2014) \cite{thie14}          & $\Ranb=10^4$ & 42.867                    & 4.882   \\
                                       & $\Ranb=10^5$ & 193.255                   & 10.507  \\
                                       & $\Ranb=10^6$ & 834.712                   & 21.695  \\
\hline
\aspect  \cite{aspectmanual}           & $\Ranb=10^4$ &       &       \\
                                       & $\Ranb=10^5$ &       &       \\
                                       & $\Ranb=10^6$ &       &       \\
\hline
\end{tabular}\\
{\captionfont Steady state Nusselt number and Vrms measurements as reported in the literature and 
obtained with \elefant on a $200\times 200$ grid. King (2009) results on 200x200 grid with ConMan.}
\end{center} 


\Literature: 
\textcite{trab90} (1990),
\textcite{ogaw93} (1993),
\textcite{trha98} (1998),
\textcite{auha99} (1999),
\textcite{chgs02} (2002),
\textcite{chhl08} (2008),
\textcite{kaks05} (2005),
\textcite{king09} (2009),
\textcite{bepo10} (2010),
\textcite{lezh11} (2011),
\textcite{dawk11} (2011),
\textcite{vyrc13} (2013),
\textcite{trbs21} (2021),
\textcite{dakg22} (2002),
\textcite{siwi20} (2020),
\stone 3.



