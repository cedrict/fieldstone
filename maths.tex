\begin{flushright} {\tiny {\color{gray} maths.tex}} \end{flushright}
%~~~~~~~~~~~~~~~~~~~~~~~~~~~~~~~~~~~~~~~~~~~~~~~~~~~~~~~~~~~~~~~~~~~~~~~~~~~~~~~~~~~~~~~~~~~~~~~~~~

%----------------------------------------------------
\subsection{About vectors}

\begin{remark}
In this document I have chosen to (when possible) use the notation $\vec{a}$
to denote a vector and ${\bm a}$ to denote a tensor/matrix. More often than not 
the same notation ${\bm a}$ is used for both in the literature.
\end{remark}

In mathematics, physics and engineering, a Euclidean vector or simply a vector 
is a geometric object that has magnitude (or length) and direction. 
Many algebraic operations on real numbers such as addition, subtraction, multiplication, 
and negation have close analogues for vectors.

Let $\vec{v}$ be a vector in 3D space. 
Its Euclidean norm (or magnitude) is given in a coordinate-free way by 
\[
|\vec{v}|:=\sqrt{\vec{v}\cdot\vec{v}}
\]
This definition makes use of the dot product, see next section.
The Euclidean norm is also called the $L_2-$norm, or $2-$norm. It is also 
sometimes noted $||\cdot ||_2$. 

In Cartesian coordinates the vector $\vec{v}$ is given by
\[
\vec{v}=
\left(
\begin{array}{c}
v_x \\ v_y \\ v_z
\end{array}
\right)
=
v_x \vec{e}_x + 
v_y \vec{e}_y + 
v_z \vec{e}_z 
\qquad
\text{with}
\qquad
\vec{e}_x=
\left(
\begin{array}{c}
1 \\ 0 \\ 0
\end{array}
\right)
\quad
\vec{e}_y=
\left(
\begin{array}{c}
0 \\ 1 \\ 0
\end{array}
\right)
\quad
\vec{e}_z=
\left(
\begin{array}{c}
0 \\ 0 \\ 1
\end{array}
\right)
\]
Its norm then simply writes
\[
|\vec{v}| = \sqrt{v_x^2 + v_y^2 + v_z^2}
\]

A unit vector is any vector with a length of one. 
A vector of arbitrary length can be divided by its length to create a unit vector.
If $\vec{a}$ is a vector, the corresponding unit vector is often denoted
\[
\vec{e}_a = \frac{\vec{a}}{|\vec{a}|}
\]


%---------------------------------------------------------------
\subsection{dot products, cross products and dyadic products}

The {\bf dot product} (or sometimes called inner product, or even scalar product) of two vectors is denoted by 
$\vec{a}\cdot \vec{b}$ and is defined as:
\[
\vec{a}\cdot \vec{b} = |\vec{a}| \; |\vec{b}| \; \cos\theta
\]
where $\theta$  is the measure of the angle between $\vec{a}$ and ${b}$.

\todo[inline]{FIGURE}

In Cartesian coordinates the dot product can also be defined as the sum 
of the products of the components of each vector as
\[
\vec{a}\cdot\vec{b} = a_xb_x + a_yb_y + a_zb_z  
\]
The dot product can also be interpreted as an answer to the question ``how similar are vectors $\vec{a}$
and $\vec{b}$ in magnitude and direction?'' Indeed, if $\vec{a}=\vec{b}$ then $\theta=0$ and $\cos\theta=1$, while if 
$\vec{a}$ is perpendicular to $\vec{b}$, then $\theta=\pi/2$, $\cos\theta=0$ and $\vec{a}\cdot \vec{b}=0$. 

In Cartesian coordinates, we find that 
\[
\vec{v} \cdot \vec{e}_x 
= (v_x \vec{e}_x + v_y \vec{e}_y + v_z \vec{e}_z ) \cdot \vec{e}_x
= v_x \underbrace{\vec{e}_x \cdot \vec{e}_x}_{=1}
+ v_y \underbrace{\vec{e}_y \cdot \vec{e}_x}_{=0}
+ v_z \underbrace{\vec{e}_z \cdot \vec{e}_x}_{=0} 
=v_x
\]
In this case the interpretation of $\vec{v} \cdot \vec{e}_x$ could be ``how much of $\vec{v}$
is in the direction $\vec{e}_x$''.

The {\bf cross product} (also  called the vector product or outer product) of two vectors is also a vector.
It is denoted $\vec{a} \times \vec{b}$ and defined as 
\[
\vec{c} = \vec{a} \times \vec{b} = |\vec{a}| \; |\vec{b}|\; \sin\theta \; \vec{n}
\]
where $\theta$  is the measure of the angle between $\vec{a}$ and ${b}$ and
and $\vec{n}$ is a unit vector perpendicular to both $\vec{a}$ and $\vec{b}$ 
which completes a right-handed system.

\todo[inline]{FIGURE}

The norm of the cross product, say $|\vec{c}|=|\vec{a} \times \vec{b}|$, is actually the 
area of the parallelogram having $\vec{a}$ and $\vec{b}$ as sides.

Also note that $\vec{a} \times \vec{b} = - \vec{b} \times \vec{a}$ (think about the direction of the 
normal vector in each case). In Cartesian coordinates the cross product can be written as
\[
\vec{a} \times \vec{b} = (a_yb_z-a_zb_y) \vec{e}_x + (a_zb_x-a_xb_z) \vec{e}_y + (a_xb_y-a_yb_x) \vec{e}_z  
\]

Finally, let us look at the {\bf dyadic product} of two vectors $\vec{a}$ and $\vec{b}$ which denoted by
$\vec{a}\; \vec{b}^T$ (juxtaposed; no symbols, multiplication signs, crosses, dots, etc...). The 
result is a tensor:
\[
\vec{a}=
\left(
\begin{array}{c}
a_x \\ a_y \\ a_z
\end{array}
\right),
\qquad
\vec{b}=
\left(
\begin{array}{c}
b_x \\ b_y \\ b_z
\end{array}
\right),
\qquad\qquad
\vec{a}\vec{b}^T 
=
\left(
\begin{array}{c}
a_x \\ a_y \\ a_z
\end{array}
\right)
(b_x \; b_y \; b_z)
=
\left(
\begin{array}{ccc}
a_x b_x & a_xb_y & a_xb_z \\
a_y b_x & a_yb_y & a_yb_z \\
a_z b_x & a_zb_y & a_zb_z 
\end{array}
\right)
\]

In conclusion the dot product yields a scalar, the cross product yields a vector and the dyadic 
product yields a tensor. 






%---------------------------------------------------------------
\subsection{Rotation matrix}

After much confusion, \url{https://mathworld.wolfram.com/RotationMatrix.html}
is a source of clarity: one must be careful when speaking of 'rotation matrix'.
Indeed, there are two possible conventions: rotation of the axes, and rotation 
of the object relative to fixed axes.

We consider in $\R^2$ the matrix ${\bm R}$ that rotates a given vector $\vec{v}$
by a counterclockwise angle $\theta$ in a fixed coordinate system.
It writes
\[
{\bm R}=
\left(
\begin{array}{cc}
\cos\theta & -\sin \theta \\
\sin\theta & \cos\theta
\end{array}
\right)
\]
with $\vec{v}'={\bm R}\cdot \vec{v}$.

On the other hand, consider the matrix that rotates the coordinate system through 
a counterclockwise angle $\theta$. The coordinates of the fixed vector $\vec{v}$ in the rotated 
coordinate system are now given by a rotation matrix which is the transpose of 
the fixed-axis matrix and, as can be seen in the above diagram, is equivalent to rotating 
the vector by a counterclockwise angle of $\theta$ relative to a fixed set of axes, giving 
\[
{\bm R}=
\left(
\begin{array}{cc}
\cos\theta & \sin \theta \\
-\sin\theta & \cos\theta
\end{array}
\right)
\]
In the following example we start from $\vec{v}=(2,1)$. If we rotate the vector by 90\si{\degree}, 
the rotation matrix is given by 
\[
{\bm R}=
\left(
\begin{array}{cc}
0 & -1 \\ 1 & 0 
\end{array}
\right)
\]
so that $\vec{v}'=(-1,2)$. 
If we rotate the axis by 90\si{\degree}, the 
rotation matrix is given by 
\[
{\bm R}=
\left(
\begin{array}{cc}
0 & 1 \\ -1 & 0 
\end{array}
\right)
\]
and the coordinates of the resulting vector are $\vec{v}'=(1,-2)$.

\begin{flushright} {\tiny {\color{gray} (rotation\_matrix.tex)}} \end{flushright}
%~~~~~~~~~~~~~~~~~~~~~~~~~~~~~~~~~~~~~~~~~~~~~~~~~~~~~~~~~~~~~~~~~~~~~~~~~~~~~~~~~~~~~~~~~~~~~~~~~~

\begin{center}
\begin{tikzpicture}
%\draw[step=1cm,gray,very thin] (0,0) grid (16,7); %background grid

\draw[->,thick] (1,3)--(3,3);
\draw[->,thick] (1,3)--(1,5);
\node[] at (3,2.75) {\small $x$};
\node[] at (0.75,5) {\small $y$};
\draw[->,thick,color=blue] (1,3)--(3,4);
\node[] at (3.75,4) {\small $\vec{v}=(2,1)$};

\draw[->,thick] (5.5,4)--(7.5,5.5);
\draw[->,thick] (5.5,4)--(7.5,2.5);

\node[] at (7,4) {\small $\theta=90^o$};

\node[rotate=37] at (6.5,5.1) {\small rotate vector};
\node[rotate=-37] at (6.5,2.9) {\small rotate axis system};

\draw[->,thick] (10,4.5)--(12,4.5);
\draw[->,thick] (10,4.5)--(10,6.5);
\node[] at (12,4.25) {\small $x$};
\node[] at (9.75,6.5) {\small $y$};
\draw[->,thick,color=blue] (10,4.5)--(9,6.5);
\node[] at (8.8,6.8) {\small $\vec{v}'=(-1,2)$};

\draw[->,thick] (12,1)--(12,3);
\draw[->,thick] (12,1)--(10,1);
\node[] at (12,3.25) {\small $x$};
\node[] at (9.75,1) {\small $y$};
\draw[->,thick,color=blue] (12,1)--(14,2);
\node[] at (14,2.25) {\small $\vec{v}'=(1,-2)$};

\end{tikzpicture}
\end{center}




