\begin{flushright} {\tiny {\color{gray} maths.tex}} \end{flushright}
%~~~~~~~~~~~~~~~~~~~~~~~~~~~~~~~~~~~~~~~~~~~~~~~~~~~~~~~~~~~~~~~~~~~~~~~~~~~~~~~~~~~~~~~~~~~~~~~~~~

%----------------------------------------------------
\subsection{About vectors}

\begin{remark}
In this document I have chosen to (when possible) use the notation $\vec{a}$
to denote a vector and ${\bm a}$ to denote a tensor/matrix. More often than not 
the same notation ${\bm a}$ is used for both in the literature.
\end{remark}

In mathematics, physics and engineering, a Euclidean vector or simply a vector 
is a geometric object that has magnitude (or length) and direction. 
Many algebraic operations on real numbers such as addition, subtraction, multiplication, 
and negation have close analogues for vectors.

Let $\vec{v}$ be a vector in 3D space. 
Its Euclidean norm (or magnitude) is given in a coordinate-free way by 
\[
|\vec{v}|:=\sqrt{\vec{v}\cdot\vec{v}}
\]
This definition makes use of the dot product, see next section.
The Euclidean norm is also called the $L_2-$norm, or $2-$norm. It is also 
sometimes noted $||\cdot ||_2$. 

In Cartesian coordinates the vector $\vec{v}$ is given by
\[
\vec{v}=
\left(
\begin{array}{c}
v_x \\ v_y \\ v_z
\end{array}
\right)
=
v_x \vec{e}_x + 
v_y \vec{e}_y + 
v_z \vec{e}_z 
\qquad
\text{with}
\qquad
\vec{e}_x=
\left(
\begin{array}{c}
1 \\ 0 \\ 0
\end{array}
\right)
\quad
\vec{e}_y=
\left(
\begin{array}{c}
0 \\ 1 \\ 0
\end{array}
\right)
\quad
\vec{e}_z=
\left(
\begin{array}{c}
0 \\ 0 \\ 1
\end{array}
\right)
\]
Its norm then simply writes
\[
|\vec{v}| = \sqrt{v_x^2 + v_y^2 + v_z^2}
\]

A unit vector is any vector with a length of one. 
A vector of arbitrary length can be divided by its length to create a unit vector.
If $\vec{a}$ is a vector, the corresponding unit vector is often denoted
\[
\vec{e}_a = \frac{\vec{a}}{|\vec{a}|}
\]


%---------------------------------------------------------------
\subsection{dot products, cross products and dyadic products}

The {\bf dot product} (or sometimes called inner product, or even scalar product) of two vectors is denoted by 
$\vec{a}\cdot \vec{b}$ and is defined as:
\[
\vec{a}\cdot \vec{b} = |\vec{a}| \; |\vec{b}| \; \cos\theta
\]
where $\theta$  is the measure of the angle between $\vec{a}$ and ${b}$.

\todo[inline]{FIGURE}

In Cartesian coordinates the dot product can also be defined as the sum 
of the products of the components of each vector as
\[
\vec{a}\cdot\vec{b} = a_xb_x + a_yb_y + a_zb_z  
\]
The dot product can also be interpreted as an answer to the question ``how similar are vectors $\vec{a}$
and $\vec{b}$ in magnitude and direction?'' Indeed, if $\vec{a}=\vec{b}$ then $\theta=0$ and $\cos\theta=1$, while if 
$\vec{a}$ is perpendicular to $\vec{b}$, then $\theta=\pi/2$, $\cos\theta=0$ and $\vec{a}\cdot \vec{b}=0$. 

In Cartesian coordinates, we find that 
\[
\vec{v} \cdot \vec{e}_x 
= (v_x \vec{e}_x + v_y \vec{e}_y + v_z \vec{e}_z ) \cdot \vec{e}_x
= v_x \underbrace{\vec{e}_x \cdot \vec{e}_x}_{=1}
+ v_y \underbrace{\vec{e}_y \cdot \vec{e}_x}_{=0}
+ v_z \underbrace{\vec{e}_z \cdot \vec{e}_x}_{=0} 
=v_x
\]
In this case the interpretation of $\vec{v} \cdot \vec{e}_x$ could be ``how much of $\vec{v}$
is in the direction $\vec{e}_x$''.

The {\bf cross product} (also  called the vector product or outer product) of two vectors is also a vector.
It is denoted $\vec{a} \times \vec{b}$ and defined as 
\[
\vec{c} = \vec{a} \times \vec{b} = |\vec{a}| \; |\vec{b}|\; \sin\theta \; \vec{n}
\]
where $\theta$  is the measure of the angle between $\vec{a}$ and ${b}$ and
and $\vec{n}$ is a unit vector perpendicular to both $\vec{a}$ and $\vec{b}$ 
which completes a right-handed system.

\todo[inline]{FIGURE}

The norm of the cross product, say $|\vec{c}|=|\vec{a} \times \vec{b}|$, is actually the 
area of the parallelogram having $\vec{a}$ and $\vec{b}$ as sides.

Also note that $\vec{a} \times \vec{b} = - \vec{b} \times \vec{a}$ (think about the direction of the 
normal vector in each case). In Cartesian coordinates the cross product can be written as
\[
\vec{a} \times \vec{b} = (a_yb_z-a_zb_y) \vec{e}_x + (a_zb_x-a_xb_z) \vec{e}_y + (a_xb_y-a_yb_x) \vec{e}_z  
\]

Finally, let us look at the {\bf dyadic product} of two vectors $\vec{a}$ and $\vec{b}$ which denoted by
$\vec{a}\; \vec{b}^T$ (juxtaposed; no symbols, multiplication signs, crosses, dots, etc...). The 
result is a tensor:
\[
\vec{a}=
\left(
\begin{array}{c}
a_x \\ a_y \\ a_z
\end{array}
\right),
\qquad
\vec{b}=
\left(
\begin{array}{c}
b_x \\ b_y \\ b_z
\end{array}
\right),
\qquad\qquad
\vec{a}\vec{b}^T 
=
\left(
\begin{array}{c}
a_x \\ a_y \\ a_z
\end{array}
\right)
(b_x \; b_y \; b_z)
=
\left(
\begin{array}{ccc}
a_x b_x & a_xb_y & a_xb_z \\
a_y b_x & a_yb_y & a_yb_z \\
a_z b_x & a_zb_y & a_zb_z 
\end{array}
\right)
\]

In conclusion the dot product yields a scalar, the cross product yields a vector and the dyadic 
product yields a tensor. 






%---------------------------------------------------------------
\subsection{Rotation matrix}

After much confusion, \url{https://mathworld.wolfram.com/RotationMatrix.html}
is a source of clarity: one must be careful when speaking of 'rotation matrix'.
Indeed, there are two possible conventions: rotation of the axes, and rotation 
of the object relative to fixed axes.

We consider in $\R^2$ the matrix ${\bm R}$ that rotates a given vector $\vec{v}$
by a counterclockwise angle $\theta$ in a fixed coordinate system.
It writes
\[
{\bm R}=
\left(
\begin{array}{cc}
\cos\theta & -\sin \theta \\
\sin\theta & \cos\theta
\end{array}
\right)
\]
with $\vec{v}'={\bm R}\cdot \vec{v}$.

On the other hand, consider the matrix that rotates the coordinate system through 
a counterclockwise angle $\theta$. The coordinates of the fixed vector $\vec{v}$ in the rotated 
coordinate system are now given by a rotation matrix which is the transpose of 
the fixed-axis matrix and, as can be seen in the above diagram, is equivalent to rotating 
the vector by a counterclockwise angle of $\theta$ relative to a fixed set of axes, giving 
\[
{\bm R}=
\left(
\begin{array}{cc}
\cos\theta & \sin \theta \\
-\sin\theta & \cos\theta
\end{array}
\right)
\]
In the following example we start from $\vec{v}=(2,1)$. If we rotate the vector by 90\si{\degree}, 
the rotation matrix is given by 
\[
{\bm R}=
\left(
\begin{array}{cc}
0 & -1 \\ 1 & 0 
\end{array}
\right)
\]
so that $\vec{v}'=(-1,2)$. 
If we rotate the axis by 90\si{\degree}, the 
rotation matrix is given by 
\[
{\bm R}=
\left(
\begin{array}{cc}
0 & 1 \\ -1 & 0 
\end{array}
\right)
\]
and the coordinates of the resulting vector are $\vec{v}'=(1,-2)$.

\begin{flushright} {\tiny {\color{gray} (rotation\_matrix.tex)}} \end{flushright}
%~~~~~~~~~~~~~~~~~~~~~~~~~~~~~~~~~~~~~~~~~~~~~~~~~~~~~~~~~~~~~~~~~~~~~~~~~~~~~~~~~~~~~~~~~~~~~~~~~~

\begin{center}
\begin{tikzpicture}
%\draw[step=1cm,gray,very thin] (0,0) grid (16,7); %background grid

\draw[->,thick] (1,3)--(3,3);
\draw[->,thick] (1,3)--(1,5);
\node[] at (3,2.75) {\small $x$};
\node[] at (0.75,5) {\small $y$};
\draw[->,thick,color=blue] (1,3)--(3,4);
\node[] at (3.75,4) {\small $\vec{v}=(2,1)$};

\draw[->,thick] (5.5,4)--(7.5,5.5);
\draw[->,thick] (5.5,4)--(7.5,2.5);

\node[] at (7,4) {\small $\theta=90^o$};

\node[rotate=37] at (6.5,5.1) {\small rotate vector};
\node[rotate=-37] at (6.5,2.9) {\small rotate axis system};

\draw[->,thick] (10,4.5)--(12,4.5);
\draw[->,thick] (10,4.5)--(10,6.5);
\node[] at (12,4.25) {\small $x$};
\node[] at (9.75,6.5) {\small $y$};
\draw[->,thick,color=blue] (10,4.5)--(9,6.5);
\node[] at (8.8,6.8) {\small $\vec{v}'=(-1,2)$};

\draw[->,thick] (12,1)--(12,3);
\draw[->,thick] (12,1)--(10,1);
\node[] at (12,3.25) {\small $x$};
\node[] at (9.75,1) {\small $y$};
\draw[->,thick,color=blue] (12,1)--(14,2);
\node[] at (14,2.25) {\small $\vec{v}'=(1,-2)$};

\end{tikzpicture}
\end{center}





%-------------------------------------------------
\subsection{Inverse of a 2x2 matrix \label{sec:inv2x2}}

Let us assume we wish to solve the 
system $\bm A \cdot \vec X = \vec b$, with $\vec X=(x,y)$. Then the solution is given by
The solution is given by
\[
x=\frac{1}{det(\bm A)}
\left|
\begin{array}{cc}
b_1 & a_{21} \\
b_2 & a_{22}
\end{array}
\right|
\qquad
y=\frac{1}{det(\bm A)}
\left|
\begin{array}{cc}
a_{11} & b_1\\
a_{21} & b_2
\end{array}
\right|
\]



%------------------------------------------------------
\subsection{Inverse of a 3x3 matrix \label{sec:inv3x3}}

Let us consider the $3 \times 3$ matrix ${\bm M}$
\[
{\bm M}=
\left(
\begin{array}{ccc}
M_{xx} & M_{xy} & M_{xz} \\
M_{yx} & M_{yy} & M_{yz} \\
M_{zx} & M_{zy} & M_{zz} 
\end{array}
\right)
\]

\begin{enumerate}
\item
Find $det({\bm M})$, the determinant of the Matrix ${\bm M}$.
The determinant will usually show up in the denominator of the inverse. 
If the determinant is zero, the matrix won't have an inverse.

\item  Find ${\bm M}^T$ , the transpose of the matrix. Transposing means reflecting 
the matrix about the main diagonal.

\[
{\bm M}^T=
\left(
\begin{array}{ccc}
M_{xx} & M_{yx} & M_{zx} \\
M_{xy} & M_{yy} & M_{zy} \\
M_{xz} & M_{yz} & M_{zz} 
\end{array}
\right)
\]

\item  Find the determinant of each of the $2\times2$ 
minor matrices. For instance $\tilde{M}_{xx}=M_{yy}M_{zz}-M_{yz}M_{zy}$,
or $\tilde{M}_{xz}=M_{xy}M_{yz}- M_{xz}M_{yy}$.

\item assemble the $\tilde{\bm M}$ matrix:

\[
\tilde{\bm M}=
\left(
\begin{array}{ccc}
+\tilde{M}_{xx} & -\tilde{M}_{xy} & +\tilde{M}_{xz} \\
-\tilde{M}_{yx} & +\tilde{M}_{yy} & -\tilde{M}_{yz} \\
+\tilde{M}_{zx} & -\tilde{M}_{zy} & +\tilde{M}_{zz} 
\end{array}
\right)
\]

\item the inverse of ${\bm M}$ is then given by
\[
{\bm M}^{1} = \frac{1}{det({\bm M})} \tilde{\bm M}
\]

\end{enumerate}

Another approach which of course is equivalent to the above is Cramer's rule. 
Let us assume we wish to solve the 
system $\bm A \cdot \vec X = \vec b$, with $\vec X=(x,y,z)$. Then the solution is given by
\[
x=
\frac{1}{det(\bm M)}
\left| 
\begin{array}{ccc}
b_1 & a_{12} & a_{13} \\
b_2 & a_{22} & a_{23} \\
b_3 & a_{32} & a_{33}
\end{array}
\right|
\qquad
y=
\frac{1}{det(\bm M)}
\left| 
\begin{array}{ccc}
a_{11} & b_1 & a_{13} \\
a_{21} & b_2 & a_{23} \\
a_{31} & b_3 & a_{33} 
\end{array}
\right|
\qquad
z=
\frac{1}{det(\bm M)}
\left| 
\begin{array}{ccc}
a_{11} & a_{12} & b_1\\
a_{21} & a_{22} & b_2\\
a_{31} & a_{32} & b_3
\end{array}
\right|
\]

%-----------------------------------------------------
\subsection{Symmetric matrices}

Any {\sl symmetric} matrix 
has only real eigenvalues,
is always diagonalizable,
and has orthogonal eigenvectors.
A symmetric $N\times N$ real matrix ${\bm M}$ is said to be
\begin{itemize}
\item {\bf positive definite} if $\vec x \cdot {\bm M} \cdot \vec x >0$ for every non-zero vector $\vec x$ of n real numbers. All the eigenvalues of a  Symmetric Positive Definite (SPD) matrix are positive.
 If A and B are positive definite, then so is A+B.
The matrix inverse of a positive definite matrix is also positive definite.
An SPD matrix has a unique Cholesky decomposition. In other words the matrix ${\bm M}$ 
is positive definite if and only if there exists a unique
lower triangular matrix ${\bm L}$, with real and strictly positive diagonal elements, 
such that ${\bm M} = {\bm L}{\bm L}^T$
(the product of a lower triangular matrix and its conjugate transpose).
This factorization is called the Cholesky decomposition of ${\bm M}$.
\index{general}{Symmetric Positive Definite}

\item {\bf positive semi-definite} if $\vec x \cdot {\bm M} \cdot  \vec x \geq 0$
\item {\bf negative definite} if $\vec x \cdot {\bm M} \cdot \vec x < 0$
\item {\bf negative semi-definite} if $\vec x \cdot {\bm M} \cdot \vec x \leq 0$
\end{itemize}

The Stokes linear system
\[
\left( \begin{array}{cc}
\mathbb{K} & \mathbb{G}  \\ \mathbb{G}^T &  0
\end{array} \right) \cdot
\left( \begin{array}{c}  {\bm v} \\ {\bm p}  \end{array} \right) = 
\left( \begin{array}{c}  {\bm f} \\ {\bm g}  \end{array} \right) 
\]
is {\bf indefinite} (i.e. it has positive as well as negative eigenvalues).

A square matrix that is not invertible is called {\bf singular} or degenerate. A square matrix is singular if and only if its determinant is 0. Singular matrices are rare in the sense that if you pick a random square matrix, it will almost surely not be singular.

%--------------------------------
\subsection{Schur complement}

From wiki.
In linear algebra and the theory of matrices, the Schur complement of a matrix block (i.e., a submatrix within a larger matrix) is defined as follows.
Suppose ${\bm A}$, ${\bm B}$, ${\bm C}$, ${\bm D}$
are respectively $p\times p$, $p \times q$, $q \times p$ and $q \times q$ matrices, and $\mathbb{D}$ is invertible. Let
\[
{\bm M}=
\left( \begin{array}{cc}
{\bm A} & {\bm B}  \\ 
{\bm C} & {\bm D}
\end{array} \right) 
\]
so that ${\bm M}$ is a $(p+q)\times(p+q)$ matrix.
Then the Schur complement of the block ${\bm D}$ of the matrix ${\bm M}$ is the $p \times p$ matrix
\[
{\bm S}={\bm A}-{\bm B}\cdot {\bm D}^{-1}\cdot {\bm C}
\]
Application to solving linear equations: The Schur complement arises naturally 
in solving a system of linear equations such as
\begin{eqnarray}
{\bm A}\cdot\vec x+{\bm B}\cdot \vec y &=& \vec f \nonumber\\
{\bm C}\cdot\vec x+{\bm D}\cdot \vec y &=& \vec g \nonumber
\end{eqnarray}
where $\vec x$, $\vec f$ are $p$-dimensional vectors, $\vec y$, $\vec g$
are $q$-dimensional vectors,
and ${\bm A}$, ${\bm B}$, ${\bm C}$, ${\bm D}$ are as above.
Multiplying the bottom equation by ${\bm B}\cdot {\bm D}^{-1}$ 
and then subtracting from the top equation one obtains
\[
({\bm A}-{\bm B}\cdot {\bm D}^{-1}\cdot {\bm C})\cdot \vec x = \vec f - {\bm B}\cdot {\bm D}^{-1}\cdot \vec g
\]
Thus if one can invert ${\bm D}$ as well as the Schur complement of ${\bm D}$, one can solve for $\vec x$,
and then by using the equation $\bm C\cdot \vec x + \bm D \cdot \vec y = \vec g$ one can solve for $y$.
This reduces the problem of inverting a $(p+q) \times (p+q)$ matrix to that of inverting
a $p \times p$ matrix and a $q \times q$ matrix.
In practice one needs ${\bm D}$ to be well-conditioned in order for 
this algorithm to be numerically accurate.

Considering now the Stokes system: 
\[
\left( \begin{array}{cc}
\K & \G  \\ \G^T &  -\C
\end{array} \right) \cdot
\left( \begin{array}{c}  {\vec v} \\ {\vec p}  \end{array} \right) = 
\left( \begin{array}{c}  {\vec f} \\ {\vec g}  \end{array} \right) 
\]
Factorising for ${\vec p}$ we end up with a {\bf velocity-Schur complement}.
Solving for ${\vec p}$ in the second equation and inserting the expression
for ${\vec p}$ into the first equation we have
\[
\mathbb{S}_v \cdot {\vec v}  = {\vec f} 
\quad\quad
\text{with}
\quad\quad
\mathbb{S}_v=\K+ \G \cdot \C^{-1} \cdot \G^T
\]
Factorising for $\vec v$ we get a {\bf pressure-Schur complement}.
\[
\mathbb{S}_p \cdot {\vec p}  = \G^T \cdot \K^{-1}\cdot {\vec f}
\quad\quad
\text{with}
\quad\quad
\mathbb{S}_p = \G^T \cdot \K^{-1}\cdot \G + \C 
\]





