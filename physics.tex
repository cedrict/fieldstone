\subsection{Strain rate and spin tensor} \label{ss:srst}
\index{general}{Velocity Gradient}
\index{general}{Strain Rate}
\index{general}{Spin Tensor}

The velocity gradient ${\bm L}$ is given in Cartesian coordinates by:
\begin{equation}
{\bm L}(\vec\upnu)=
\vec\nabla\vec\upnu = 
\left(
\begin{array}{ccc}
\frac{\partial u}{\partial x} & \frac{\partial v}{\partial x} & \frac{\partial w}{\partial x} \\\\
\frac{\partial u}{\partial y} & \frac{\partial v}{\partial y} & \frac{\partial w}{\partial y} \\\\
\frac{\partial u}{\partial z} & \frac{\partial v}{\partial z} & \frac{\partial w}{\partial z} 
\end{array}
\right)
\end{equation}
It can be decomposed into its symmetric and skew-symmetric parts according to:
\begin{equation}
\vec\nabla\vec\upnu = (\vec\nabla\vec\upnu)^s + (\vec\nabla\vec\upnu)^w 
= \dot{\bm \varepsilon}(\vec \upnu) +  \dot{\bm \omega}(\vec \upnu)
\end{equation}
The symmetric part is called the strain rate (or rate of deformation)\footnote{Note that often the dot is omitted and for example the \aspect manual uses the ${\bm \varepsilon}$ notation.}:
\begin{equation}
\dot{\bm \varepsilon}(\vec \upnu) = \frac{1}{2}\left( \vec\nabla\vec\upnu + (\vec\nabla\vec\upnu)^T \right)
\end{equation}
The skew-symmetric tensor is called spin tensor (or vorticity tensor):
\begin{equation}
\dot{\bm \omega}(\vec \upnu) = \frac{1}{2}\left( \vec\nabla\vec\upnu - (\vec\nabla\vec\upnu)^T \right)
\end{equation}

\begin{remark}
In the mathematical literature a different notation for the strain rate tensor is often used, i.e. 
${\bm D}(\vec \upnu)$ - or simply ${\bm D}$, such as for instance in Fullsack (1995) \cite{full95}.
\end{remark}

%.............................................
\section{Viscous Newtonian rheology}
\begin{flushright} {\tiny {\color{gray} physics.tex}} \end{flushright}

The relationship between velocity-related stresses and
velocity derivatives is such that the total stress tensor has the form \cite{berc09}
\begin{equation}
{\bm \sigma} = -p {\bm 1} + {\bm A}:\dot{\bm \varepsilon}(\vec\upnu)
\end{equation}
where $p$ is the thermodynamic pressure which is a function of the density $\rho$ and the temperature $T$ (an equation of state is then needed)
and ${\bm A}$ is the fourth-rank stiffness tensor.

Since both the stress and the strain tensors are symmetric and for isotropic 
fluids we have (see Malvern \cite{malvern})
\begin{equation}
{\bm A}:\dot{\bm \varepsilon}(\vec\upnu) 
= \lambda (\vec\nabla\cdot\vec\upnu) {\bm 1} + 2\eta \dot{\bm \varepsilon}(\vec\upnu)
\end{equation}
where $\lambda$ is the bulk viscosity and $\eta$ is the dynamic viscosity\footnote{also sometimes called shear viscosity}. 
The stress tensor is then 
\begin{equation}
{\bm \sigma} = (-p  + \lambda (\vec\nabla\cdot\vec\upnu)) {\bm 1} + 2\eta \dot{\bm \varepsilon}(\vec\upnu)
\end{equation}
By writing 
\[
\dot{\bm \varepsilon}(\vec\upnu) 
= \frac{1}{3}{\rm tr}(\dot{\bm \varepsilon}(\vec\upnu)) {\bm 1} + \dot{\bm \varepsilon}^d(\vec\upnu) =
 \frac{1}{3}(\vec\nabla\cdot\vec\upnu) {\bm 1} + \dot{\bm \varepsilon}^d (\vec\upnu)
\]
where $\dot{\bm \varepsilon}^d(\vec\upnu)$ is the deviatoric strain rate tensor and 
(in Cartesian coordinates)
\begin{equation}
\vec\nabla\cdot\vec\upnu = 
\text{div} (\vec\upnu ) =
{\rm tr}(\dot{\bm \varepsilon}(\vec\upnu)) =
\frac{\partial u}{\partial x}+ 
\frac{\partial v}{\partial y}+ 
\frac{\partial w}{\partial z} 
\end{equation} 
where ${\rm tr}$ is the trace operator, we arrive at
\begin{eqnarray}
{\bm \sigma} 
&=& (-p+\lambda(\vec\nabla\cdot\vec\upnu)) {\bm 1} + 2\eta \left[ \frac{1}{3}(\vec\nabla\cdot\vec\upnu) {\bm 1} + \dot{\bm \varepsilon}^d \right] \\
&=& \left[ -p+\left(\lambda+\frac{2}{3}\eta\right)(\vec\nabla\cdot\vec\upnu)\right] {\bm 1} + 2\eta  \dot{\bm \varepsilon}^d  
\end{eqnarray}
Introducing the second viscosity $\zeta=\lambda+\frac{2}{3}\eta$:
\begin{eqnarray}
{\bm \sigma} 
&=& \left[ -p+ \zeta (\vec\nabla\cdot\vec\upnu)\right] {\bm 1} + 2\eta  \dot{\bm \varepsilon}^d \\ 
&=&  -\overline{p} {\bm 1} + 2\eta  \dot{\bm \varepsilon}^d  
\end{eqnarray}
The effect of the volume viscosity $\zeta$ is that the mechanical pressure $\overline{p}$
is not equivalent to the thermodynamic pressure $p$ 
\begin{equation}
\overline{p}=p - \zeta (\vec\nabla\cdot\vec\upnu)
\end{equation}
In other words: the isotropic average of the total stress is {\sl not} the pressure term!
This difference is usually neglected (and it is safe to do so, see \cite[section 7.02.3.2.2]{berc09}) 
by explicitly assuming $\zeta=0$ (also called the Stokes assumption \cite[p256]{scto01}), 
so that one can then refer to pressure as a single well-defined value.
Note that in the case of an incompressible Newtonian Fluid, 
the strain rate tensor is deviatoric $({\text tr}(\dot{\bm \varepsilon}(\vec\upnu)) 
= \text{div}(\vec\upnu) =0)$ and the above considerations vanish.

Finally, for both compressible and incompressible flow, the stress tensor becomes simply
\begin{mdframed}[backgroundcolor=blue!5]
\begin{equation}
{\bm \sigma}=-p {\bm 1} + 2\eta \dot{\bm \varepsilon}^d(\vec\upnu) = -p {\bm 1} + {\bm \tau}
\end{equation}
\end{mdframed}
where ${\bm \tau} = 2\eta \dot{\bm \varepsilon}^d(\vec\upnu)$ is the deviatoric stress tensor.

\begin{remark}
On page 256 of Schubert, Turcotte and Olson \cite{scto01}, 
equation 6.5.3, the authors write $\tau_{ii}/3=k_B e_{ii}$ while stating that $\tau$ is deviatoric in equation 6.4.2. 
This is an obvious conflict of notations. 
\end{remark}

%------------------------------------------------------------------------
\section{The heat transport equation - energy conservation equation \label{ss:hte}}
\begin{flushright} {\tiny {\color{gray} physics.tex}} \end{flushright}

%The heat contained within a volume $dV$ is $\rho C_p T dV$ where 
%$C_p$ is the specific heat. The total heat ${\cal H}$ contained
%by $V$ is the sum ( or integral) of all the elements within $V$:
%\[
%{\cal H}=\int_V \rho C_p T dV
%\]
%Changes in ${\cal H}$ can only occur if heat flows across the surface $S$. 
%If $Q$ is the rate at which heat flows outward, then the rate of change of ${\cal H}$ 
%must equal $Q$, or
%\[
%-\frac{\partial {\cal H}}{\partial t} = Q
%\]
%The negative sign is required because the volume cools if $Q$ is positive. 
%The heat flux depends on the temperature gradient $\vec\nabla T$ , 
%and heat always flows down the temperature gradient. Hence the heat
%flux across the surface element $dS$, $dQ$, is
%\[
%dQ= - k \vec\nabla T \cdot \vec{n} \; dS
%\]
%where $k$ is the thermal conductivity. The dot product between $\vec\nabla T$ and $\vec{n}dS$ 
%takes the direction of the temperature gradient into account.
%We can then write:
%\[
%-\frac{\partial }{\partial t} \int_V \rho C_p T dV = - \int_S  k \vec\nabla T \cdot \vec{n} \; dS
%\]
%Using Gauss' theorem the right hand side becomes:
%\[
%\int_S  k \vec\nabla T \cdot \vec{n} \; dS
%= 
%\int_V \vec\nabla \cdot ( k \vec\nabla T ) dV
%\]
%so that:
%\[
%\int_V \left[ \frac{\partial }{\partial t} (\rho C_p T) - \vec\nabla \cdot( k \vec\nabla T ) \right] dV 
%\]
%Since $V$ can be any volume enclosed by an arbitrary surface, this 
%equation can only be true if the term in square brackets is zero everywhere, or:
%\[
%\rho C_p \frac{\partial T}{\partial t}  =  \vec\nabla \cdot( k \vec\nabla T ) 
%\]
%assuming density and specific heat to be constant in time.
%In a fluid which moves, the heat transport must include a term
%\[
%\int_S \rho C_p T \vec\upnu\cdot \vec{n} \; dS
%\]
%to take account of the heat carried by the fluid moving with velocity $\vec\upnu$. 
%This term is called the advection term in the equations, and we then have:
%\[
%\rho C_p \left( 
%\frac{\partial T}{\partial t} + (\vec\upnu \cdot \vec\nabla) T \right) =  \vec\nabla \cdot( k \vec\nabla T ) 
%\]
















%------------------------------------------------------------------------
\section{The momentum conservation equations} 
\begin{flushright} {\tiny {\color{gray} physics.tex}} \end{flushright}

As explained in Section~\ref{ss:nondim}, in Earth science applications the Navier-Stokes 
equations reduce to the Stokes equation:
\begin{equation}
{\vec \nabla}\cdot {\bm \sigma} + \rho {\vec g} = \vec{0}
\label{eq:forcebal}
\end{equation}
Since 
\begin{equation}
{\bm \sigma} = -p {\bm 1} + {\bm \tau}
\end{equation}
it also writes
\begin{equation}
-{\vec \nabla}p + {\vec \nabla}\cdot {\bm \tau} + \rho {\vec g} = \vec{0}
\end{equation}
Using the relationship ${\bm \tau} = 2 \eta \dot{\bm \varepsilon}^d(\vec\upnu)$ we arrive at 
\begin{mdframed}[backgroundcolor=blue!5]
\begin{equation}
-{\vec \nabla}p + {\vec \nabla}\cdot (2 \eta \dot{\bm \varepsilon}^d(\vec\upnu) ) + \rho {\vec g} = \vec{0}
\end{equation}
\end{mdframed}

The divergence of a tensor field in cylindrical coordinates ($r,\theta,z$)
has been obtained in Section~\ref{ss:cylcoord}.
The equations of motion \eqref{eq:forcebal} becomes\footnote{\url{https://en.wikipedia.org/wiki/Linear_elasticity}}

 \begin{align}
\frac{\partial \sigma_{rr}}{\partial r} + \cfrac{1}{r}\frac{\partial \sigma_{r\theta}}{\partial \theta} + \frac{\partial \sigma_{rz}}{\partial z} + \cfrac{1}{r}(\sigma_{rr}-\sigma_{\theta\theta}) + \rho g_r &= 0 \label{eq:momelr}\\
\frac{\partial \sigma_{r\theta}}{\partial r} + \cfrac{1}{r}\frac{\partial \sigma_{\theta\theta}}{\partial \theta} + \frac{\partial \sigma_{\theta z}}{\partial z} + \cfrac{2}{r}\sigma_{r\theta} + \rho g_\theta &=  0 \label{eq:momeltheta}\\
\frac{\partial \sigma_{rz}}{\partial r} + \cfrac{1}{r}\frac{\partial \sigma_{\theta z}}{\partial \theta} + \frac{\partial \sigma_{zz}}{\partial z} + \cfrac{1}{r}\sigma_{rz} + \rho g_z &= 0
\end{align}


%------------------------------------------------------------------------
\section{The mass conservation equations} 
\begin{flushright} {\tiny {\color{gray} physics.tex}} \end{flushright}
\index{general}{Solenoidal Field} 
\index{general}{Divergence-free}
\index{general}{Continuity Equation}
\index{general}{Mass Conservation Equation}

The mass conservation equation (often called continuity equation) is given by
\[
\frac{D\rho}{Dt} + \rho {\vec \nabla}\cdot{\vec \upnu} = 0
\]
or, since 
\[
\frac{D\rho}{Dt} = \frac{\partial \rho}{\partial t} + {\vec \upnu}\cdot {\vec \nabla}\rho
\]
then 
\[
\frac{D\rho}{Dt} + \rho {\vec \nabla}\cdot{\vec \upnu} = 
\frac{\partial \rho}{\partial t} + {\vec \upnu}\cdot {\vec \nabla}\rho
 + \rho {\vec \nabla}\cdot{\vec \upnu} = 0 
\]
and finally:
\begin{mdframed}[backgroundcolor=blue!5]
\begin{equation}
\frac{\partial \rho}{\partial t} + {\vec \nabla}\cdot(\rho {\vec \upnu}) = 0
\label{eq:massconvgen}
\end{equation}
\end{mdframed}
In the case of an incompressible flow, then $\partial \rho/\partial t=0$ and 
${\vec \nabla}\rho=0$, i.e. $D\rho/Dt=0$ and the remaining equation is simply:
\[
{\vec \nabla}\cdot{\vec \upnu} = 0
\]
A vector field that is divergence-free is also called 
solenoidal\footnote{\url{https://en.wikipedia.org/wiki/Solenoidal_vector_field}}.


In cylindrical coordinates $(r,\theta,\phi)$ 
the continuity equation for an incompressible fluid is :

\begin{mdframed}[backgroundcolor=blue!5]
\[
\frac{1}{r} \frac{\partial}{\partial r} (r \upnu_r) 
+
\frac{1}{r} \frac{\partial \upnu_\theta}{\partial \theta}
+
\frac{\partial \upnu_z}{\partial z}=0
\]
\end{mdframed}
\index{general}{Mass Conservation Equation (Cylindrical Coordinates)}


In spherical coordinates $(r,\theta,\phi)$ 
the continuity equation for an incompressible fluid is :

\begin{mdframed}[backgroundcolor=blue!5]
\begin{equation}
\frac{1}{r^2} \frac{\partial}{\partial r} (r^2 \upnu_r) 
+
\frac{1}{r \sin\theta} \frac{\partial}{\partial \theta} (\upnu_\theta \sin\theta)
+
\frac{1}{r \sin\theta} \frac{\partial \upnu_\phi}{\partial \phi}=0
\label{eq:divscc}
\end{equation}
\end{mdframed}
\index{general}{Mass Conservation Equation (Spherical Coordinates)}



%------------------------------------------------------------------------------
\section{The equations in \aspect manual}
\begin{flushright} {\tiny {\color{gray} physics.tex}} \end{flushright}

The following is lifted off the \aspect manual.
We focus on the system of equations in a $d=2$- or $d=3$-dimensional
domain $\Omega$ that describes the motion of a highly viscous fluid driven
by differences in the gravitational force due to a density that depends on
the temperature. In the following, we largely follow the exposition of this
material in Schubert, Turcotte and Olson \cite{scto01}.

Specifically, we consider the following set of equations for velocity $\vec\upnu$, pressure $p$ and temperature $T$:
\begin{align}
  \label{eq:stokes-1}
  -\vec\nabla \cdot \left[2\eta \left(\dot{\bm \varepsilon}(\vec \upnu)
                                  - \frac{1}{3}(\vec\nabla \cdot \vec \upnu)\mathbf 1\right)
                \right] + \vec\nabla p &=
  \rho \vec g
  &
  & \textrm{in $\Omega$},
  \\
  \label{eq:stokes-2}
  \vec\nabla \cdot (\rho \vec \upnu) &= 0
  &
  & \textrm{in $\Omega$},
  \\
  \label{eq:temperature}
  \rho C_p \left(\frac{\partial T}{\partial t} + \vec \upnu\cdot\vec\nabla T\right)
  - \vec\nabla\cdot k\vec\nabla T
  &=
  \rho H
  \notag
  \\
  &\quad
  +
  2\eta
  \left(\dot\varepsilon(\vec\upnu) - \frac{1}{3}(\vec\nabla \cdot \vec \upnu)\mathbf 1\right)
  :
  \left(\dot\varepsilon(\vec\upnu) - \frac{1}{3}(\vec\nabla \cdot \vec \upnu)\mathbf 1\right)
  \\
  &\quad
  +\alpha T \left( \vec \upnu \cdot \vec\nabla p \right)
  && \textrm{in $\Omega$},
  \notag
\end{align}
where $\dot{\bm \varepsilon}(\vec\upnu) = \frac{1}{2}(\vec\nabla \vec\upnu + \vec\nabla \vec\upnu^T)$ 
is the symmetric gradient of the velocity (often called the
\textit{strain rate} tensor).

In this set of equations, \eqref{eq:stokes-1} and \eqref{eq:stokes-2}
represent the compressible Stokes equations in which $\vec\upnu =\vec\upnu (\mathbf x,t)$ 
is the velocity field and $p=p(\mathbf x,t)$ the pressure
field. Both fields depend on space $\mathbf x$ and time $t$. Fluid flow is
driven by the gravity force that acts on the fluid and that is proportional to
both the density of the fluid and the strength of the gravitational pull.

Coupled to this Stokes system is equation \eqref{eq:temperature} for the
temperature field $T=T(\mathbf x,t)$ that contains heat conduction terms as
well as advection with the flow velocity $\mathbf v$. The right hand side
terms of this equation correspond to
\begin{itemize}
\item internal heat production for example due to radioactive decay;
\item friction (shear) heating;
\item adiabatic compression of material;
\end{itemize}

In order to arrive at the set of equations in the \aspect manual
we need to 
\begin{itemize}
\item neglect the $\partial p/\partial t$ \todo{wrong rephrase} 
\item neglect the $\partial \rho / \partial t$  in \eqref{eq:massconvgen}.
\end{itemize}
from equations above. A partial answer is given in the next section. 

----------------------------------------

Also, their definition of the shear heating term $\Phi$ is:
\[
\Phi = k_B ({\vec \nabla}\cdot{\vec \upnu})^2 + 2\eta \dot{\bm \varepsilon}^d:\dot{\bm \varepsilon}^d
\]
For many fluids the bulk viscosity $k_B$ is very small and is often taken to be zero, an assumption known
as the Stokes assumption: $k_B=\lambda+2\eta/3=0$. \index{general}{Bulk Viscosity}
Note that $\eta$ is the dynamic viscosity and $\lambda$ the second viscosity. 
\index{general}{Dynamic Viscosity}
\index{general}{Second Viscosity}
Also, 
\[
{\bm \tau}=2\eta \dot{\bm \varepsilon} + \lambda ({\bm \nabla}\cdot{\vec \upnu}) {\bm 1}
\]
but since $k_B=\lambda+2\eta/3=0$, then $\lambda=-2\eta/3$ so 
\[
{\bm \tau}=2\eta \dot{\bm \varepsilon} -\frac{2}{3}\eta ({\bm \nabla}\cdot{\vec \upnu}) {\bm 1} = 2\eta \dot{\bm \varepsilon}^d
\]

\newpage
%---------------------------------------------------------------------------------------------------
\section{Equations for thermal convection in an anelastic, compressible, self-gravitating spherical mantle }
\begin{flushright} {\tiny {\color{gray} physics.tex}} \end{flushright}
%------------------------------------------------------------------------------

What follows is borrowed from Section 2.1 of Gli{\v{s}}ovi{\'c} \etal (2012) \cite{glfm12}.
We start from the conservation mass, momentum and energy equations (the full Navier-Stokes equations):
\begin{eqnarray}
\frac{\partial \rho}{\partial t} + \vec\nabla \cdot (\rho \vec\upnu) = \vec{0} \\
\rho \frac{D\vec\upnu}{Dt} = \vec\nabla\cdot {\bm \sigma} + \rho \vec{g} \\
\rho C_p \frac{D T}{Dt} = \vec\nabla \cdot k \vec\nabla T + \alpha T \frac{Dp}{Dt} + \Phi + Q
\end{eqnarray}
In solving for the mantle flow field that satisfies the equation of momentum conservation, 
we incorporate all effects arising from
self-gravitation and we must therefore explicitly consider the 3-D variation of 
gravity throughout Earth's interior. The
gravitational acceleration is written as
\[
\vec{g} = \vec\nabla \phi
\]
where $\phi$ is Earth's gravitational potential which satisfies Poisson's equation
\[
\Delta \phi = - 4 \pi {\cal G} \rho
\]
The gravitational potential is expressed as
\[
\phi = \phi_0(r) + \phi_1(r,\theta,\phi)
\]
where the subscript $0$ denotes a hydrostatic reference state, 
in which the structure of the mantle (density, gravity, pressure, temperature) varies
with radius alone and the subscript $1$ denotes all 3D perturbations arising from the 
thermal convection process. This decomposition makes sense in the context of a perfect sphere.

The total perturbed density and pressure fields in the mantle may similarly be expressed as
\[
\rho = \rho_0(r) + \rho_1(r,\theta,\phi)
\]
\[
p = p_0(r) + p_1(r,\theta,\phi)
\]
The equation of state relates the density perturbations to the temperature and pressure perturbations 
as follows
\[
\rho_1 = \rho_0[1-\alpha(T-T_0(r))+K_T^{-1} (p-p_0(r))] 
\]
where $K_T$ is the bulk modulus and the term $T_0(r)$ represents the horizontally averaged temperature (i.e.
the geotherm) which varies with radius only. 
The effects of compressibility on the density are found to be at least two orders of magnitude
smaller than the effects of temperature variations. Therefore, the last term 
of this equation is often neglected.
Note that this expression is a first order expansion of any Equation of State. \index{general}{Equation of State}

Also, this equation can be misleading if one forgets that the parameters $\alpha$ and $K_T$
cannot be constant but must be related through Maxwell relations (for example,
their definitions 
\[
\alpha = \frac{1}{V} \left( \frac{\partial V}{\partial T} \right)_P 
= -\frac{1}{\rho} \left( \frac{\partial \rho}{\partial T} \right)_P
\]
and 
\[
K_T 
= - V \left( \frac{\partial P}{\partial V} \right)_T
= \rho \left( \frac{\partial P}{\partial \rho} \right)_T
\]
imply that 
\[
\frac{\partial (\alpha\rho)}{\partial P}
= 
-\frac{\partial (\rho/K_T)}{\partial T}
\]
Some models can be found in the geophysical literature in which assumptions made inconsistently
about thermodynamic parameters (either constant or depth-dependent) violate the Maxwell rules.

\index{general}{ALA}
\index{general}{Anelastic Liquid Approximation}
Important simplifications are made assuming the anelastic-liquid approximation 
(e.g. Jarvis \& McKenzie (1980) \cite{jamc80}, Solheim \& Peltier (1990) \cite{sope90}).
This approximation is justified because the velocities associated with mantle convection 
are very slow compared to the local sound speed and
hence acoustic waves cannot be generated by the slow changes in the mantle pressure field. 
We therefore neglect the time derivative of density, thereby eliminating sound waves:
\[
\frac{\partial \rho}{\partial t} \simeq 0
\] 
For the same reason, the pressure distribution may be considered (to first-order accuracy) 
as the pressure of a fluid in hydrostatic equilibrium which yields
\[
\frac{Dp}{Dt} = \frac{\partial p}{\partial t} + \vec\upnu\cdot\vec\nabla p 
\simeq
- u_r \rho_0(r) g_0(r)
\]
The equations are then rewritten in terms of dimensionless variables according to the relations:
\begin{eqnarray}
r'&=&\frac{r}{d} \\
\upnu' &=& \frac{\upnu}{U} \\
t' &=& \frac{U}{d/t} \\
T' &=& \frac{T}{\Delta T} \\
\rho' &=& \frac{\rho}{\rho_{0s}} \\
g' &=& \frac{g}{g_{0s}} \\
\phi' &=& \frac{\phi}{g_{0s}d} \\
\alpha' &=& \frac{\alpha}{\alpha_s} \\
p' &=& \frac{p}{\alpha_s \Delta T \rho_{0s} g_{0s} d} \\
\tau_{ij} &=& \frac{\tau_{ij}}{\alpha_s \Delta T \rho_{0s} g_{0s} d} \\
\eta' &=& \frac{\eta}{\eta_s} \\
k' &=& \frac{k}{k_s} \\
Q' &=& \frac{Q d^2}{k_s \Delta T} \\
U &=& \frac{\rho_{0s}g_{0s}\alpha_s \Delta T d^2}{\eta_s}
\end{eqnarray}

in which the primes represent the dimensionless variables, the subscript $s$ means 
that we consider the surface value of the variable to which it
is applied. The length scale $d$ and temperature scale $\Delta T$ are respectively 
the depth of the mantle and the difference of temperature between
the bottom and the top of the mantle. 

Often one deals with dimensionless variables and the primes
are dropped for notational convenience (this is the case in what follows).

It is a tedious but trivial exercise to show that the dimensionless equation of 
conservation of momentum is then written as follows:
\[
\rho \frac{\Ranb_s}{Pr_s} \frac{D\vec\upnu}{Dt} =
\frac{\rho}{\alpha_s \Delta T} \vec\nabla \phi - \vec\nabla p + \vec\nabla \cdot {\bm \tau}
\]
in which we introduce the surface Rayleigh $\Ranb_s$ and Prandtl $\Prnb_s$ numbers defined, 
respectively, by
\[
\Ranb_s=\frac{\rho_{0s}^2 C_p g_{0s} \alpha_s \Delta T d^3}{k_s \eta_s}
\qquad
\qquad
Pr_s= \frac{\eta_s C_p}{k_s}
\]
Because of the very high viscosity of mantle rocks, the left-hand term 
is smaller than the other terms by several orders of magnitude
and may therefore be neglected. This important simplification is called the 
infinite Prandtl number approximation. \index{general}{Prandtl number}

The equation of energy conservation may also be rewritten in terms of the surface Rayleigh number, as follows
\[
\frac{D T}{D t} = \frac{1}{\rho \Ranb_s} \left( \vec\nabla\cdot k\vec\nabla T + Q \right) 
+\frac{Di}{\rho} \left( -\alpha T \frac{Dp}{Dt} + \Phi  \right)
\]
where $Di$ is the dissipation number (see Peltier (1972) \cite{pelt72}) 
which measures the importance of compression work and frictional heating, and it is defined
as \index{general}{Dissipation Number}
\[
Di=\frac{\alpha_s g_{0s} d}{C_p}
\]
$Di$ also measures the ratio of the depth of mantle convection ($d$) to 
the adiabatic scale height ($C_p/\alpha g_0$ ) and for whole-mantle convection is
close to order 1 (see Jarvis \& McKenzie (1980) \cite{jamc80}).

After simplifications, the dimensionless system of governing equations is written as

\begin{eqnarray}
\vec\nabla\cdot(\rho_0 \vec\upnu) =0 \\
\frac{\rho}{\alpha_s \Delta T} \vec\nabla \phi - \vec\nabla p + \vec\nabla \cdot {\bm \tau} = \vec{0} \\
\frac{\partial T}{\partial t} + \vec\upnu\cdot\vec\nabla T =  
\frac{1}{\rho_0 \Ranb_s} \left( \vec\nabla\cdot k\vec\nabla T + Q \right) +
\frac{Di}{\rho_0} (-\alpha T \rho_0 g_0 u_r + \Phi) 
\end{eqnarray}
with 
\[
\Delta \phi = -4\pi {\cal G} \rho
\]
\[
\rho_1=\rho_0(1-\alpha(T-T_0(r)))
\]
\todo[inline]{VERIFY all this !}

%------------------------------------------------------------------------------
\section{Non-dimensionalisation of the Navier-Stokes equations}\label{ss:nondim}
\begin{flushright} {\tiny {\color{gray} physics.tex}} \end{flushright}

%______________________________________________
\subsection{Approach \# 1 - isothermal flow}
%We start from the following form of the momentum conservation equation: 
%\[
%\frac{\partial \vec\upnu}{\partial t} + (\vec\upnu \cdot \vec\nabla)\vec\upnu 
%=
%-\frac{1}{\rho}\vec\nabla p + \nu \Delta \vec \upnu + \vec g
%\]
%where $\nu$ is the kinematic viscosity. \index{general}{Kinematic Viscosity}

We define (see for instance Massimi \etal (2006) \cite{maqs06}) four reference 
quantities which are relevant for geodynamics\footnote{Note that in the paper 
the authors conflate $\rho$ and $\tilde{\rho}$ which prevents them from non-dimensionalising
all terms as we do here.}:
\begin{itemize}
\item a reference viscosity value $\underline{\eta}=10^{20}~\si{\pascal\second}$
\item a reference mass density $\underline{\rho}=1000~\si{\kg\per\cubic\metre}$
\item a reference time $\underline{t}=1\text{Myr}\simeq 3.15\cdot 10^{13}~\si{\second}$
\item a reference length $\underline{l}=1000~\si{\metre}$
\item a reference gravity $\underline{g}=9.81~\si{\metre\per\square\second}$
\end{itemize}
It follows that a reference pressure can be obtained:
\[
\underline{p}=\underline{\rho} \underline{g} \underline{l} = 9.81\cdot 10^6~\si{\pascal}
\]
Note that there is unfortunately no natural selection for the pressure scale. 
We could also have used $\underline{p}=\underline{\rho}\underline{\upnu}^2$ 
where dynamic effects are dominant i.e. high velocity flows,
or $\underline{p}=\underline{\eta}\underline{\upnu}/\underline{l}$ 
where viscous effects are dominant i.e. creeping flows (which 
is the case in geodynamics).
The definition of a reference velocity is more straightforward:
\[
\underline{\upnu} = \frac{\underline{l}}{\underline{t}} = 1~\si{\mm\per\year}
\]
We define dimensionless variables through:
\[
{\color{teal}x} = \frac{x}{\underline{l}}
\qquad
{\color{teal}y} = \frac{y}{\underline{l}}
\qquad
{\color{teal}z} = \frac{z}{\underline{l}}
\qquad
{\color{teal}\vec\upnu'} = \frac{\vec\upnu}{\underline{\upnu}}
\qquad
{\color{teal}t} = \frac{t}{\underline{t}}
\qquad
{\color{teal} \eta} = \frac{\eta}{\underline{\eta}}
\qquad
{\color{teal} g} =\frac{g}{\underline{g}}
\]
where the {\color{teal}teal} color indicates dimensionless values.

Consequently, time and space derivatives will be rescaled as follows:
\[
{\color{teal} \vec\nabla} = \underline{l}\; \vec\nabla
\qquad
{\color{teal} \partial_t} = \underline{t}\; \partial_t 
\]
Using this scaling relations the Navier-Stokes equation become:
\[
\frac{\rho \underline{l}}{\underline{t}^2} {\color{teal} \frac{\partial \vec\upnu}{\partial t}}
+
\frac{\rho \underline{l}}{\underline{t}^2} {\color{teal} (\vec\upnu \cdot\vec\nabla) \vec\upnu} 
=
- \underline{\rho} \underline{g} {\color{teal} \vec\nabla p}
+
\frac{\underline{\eta}}{\underline{l}\underline{t}} 
{\color{teal} \vec\nabla \cdot \eta (\vec\nabla \vec \upnu + \vec\nabla \vec \upnu ^T)}
+ \rho \vec{g}
\]
I make ${\color{teal}\rho}= \rho/\underline{\rho}$ appear in the left hand side:
\[
\frac{\underline{\rho} \underline{l}}{\underline{t}^2}
{\color{teal}\rho}
 {\color{teal} \frac{\partial \vec\upnu}{\partial t}}
+
\frac{\underline{\rho} \underline{l}}{\underline{t}^2} 
{\color{teal}\rho}
{\color{teal} (\vec\upnu \cdot\vec\nabla) \vec\upnu} 
=
- \underline{\rho} g {\color{teal} \vec\nabla p}
+
\frac{\underline{\eta}}{\underline{l}\underline{t}} 
{\color{teal} \vec\nabla \cdot \eta (\vec\nabla \vec \upnu + \vec\nabla \vec \upnu ^T)}
+ \rho \vec{g}
\]
which we can divide by $\underline{\rho} \underline{l}/\underline{t}^2$ to obtain:
\[
 {\color{teal}\rho} \left(
{\color{teal} \frac{\partial \vec\upnu}{\partial t}}
+
{\color{teal} (\vec\upnu \cdot\vec\nabla) \vec\upnu} 
\right)
=
- \frac{ \underline{g} \underline{t}^2 }{\underline{l}} {\color{teal} \vec\nabla p}
+
\frac{\underline{\eta} \underline{t}}{\underline{\rho} \underline{l}^2 } 
{\color{teal} \vec\nabla \cdot \eta (\vec\nabla \vec \upnu + \vec\nabla \vec \upnu ^T)}
+ \frac{\underline{t}^2}{ \underline{l}}  {\color{teal} \rho} \vec{g}
\]
One can recognise in this equation the Reynolds and Froude 
non-dimensional numbers (the ratio between the inertial and viscous forces, and the 
ratio between buoyancy and inertial forces respectively). 
\[
Re = \frac{\underline{\rho} \underline{l}^2}{\underline{\eta} \underline{t} }
\qquad
Fr= \frac{\underline{l}}{\underline{g}\underline{t}^2 }
\]
From this we conclude that inertial forces in the Earth's mantle
are small compared to viscous forces.
We can then write:
\[
\boxed{
{\color{teal}\rho} \left(
{\color{teal} \frac{\partial \vec\upnu}{\partial t}}
+
{\color{teal} (\vec\upnu \cdot\vec\nabla) \vec\upnu}  \right)
=
- \frac{1}{Fr} {\color{teal} \vec\nabla p}
+
\frac{1}{Re}
{\color{teal} \vec\nabla \cdot \eta (\vec\nabla \vec \upnu + \vec\nabla \vec \upnu ^T)}
+ \frac{1}{Fr} {\color{teal} \vec{g}}
}
\]
In our case, given the definitions taken above, we have:
\[
Re \simeq 3.174 \cdot 10^{-24}
\qquad
Fr \simeq 1.027 \cdot 10^{-25}
\]
so that the inertial terms can be dropped from the momentum equation (thereby yielding the 
dimensionless Stokes equations):
\[
{\color{teal} \vec\nabla \cdot \eta (\vec\nabla \vec \upnu + \vec\nabla \vec \upnu ^T)}
- \frac{Re}{Fr} {\color{teal} \vec\nabla p}
+ \frac{Re}{Fr}  {\color{teal} \vec{g} }
=0
\]
Note that in our case $Re/Fr\simeq 30.5$. 

%_________________________________________________________________________
\subsection{Approach \# 2 - Temperature dependent \label{ss:dimeqs2}}

\begin{flushright} {\tiny {\color{gray} dimensionless\_equations2.tex.tex}} \end{flushright}
%~~~~~~~~~~~~~~~~~~~~~~~~~~~~~~~~~~~~~~~~~~~~~~~~~~~~~~~~~~~~~~~~~~~~~~~~~~~~~~~~~~~~~~~~~~~~~~~~~~

Let us now consider a box heated from below and cooled from above. 
We define 4 fundamental reference quantities:
\begin{itemize}
\item a length $L_{ref}$ (\si{\metre}), ({\color{violet} $L$})
\item a temperature $T_{ref}$ (\si{\kelvin}), ({\color{violet} $\uptheta$})
\item a viscosity $\eta_{ref}$ (\si{\pascal\second}), ({\color{violet} $ML^{-1}T^{-1}$})
\item a thermal diffusion coefficient $\kappa_{ref}$ (\si{\square\metre\per\second}), ({\color{violet} $L^2T^{-1}$})
\end{itemize}
From these reference quantities one can form secondary ones, such as
\begin{itemize}
\item a time $t_{ref} = L_{ref}^2 / \kappa_{ref}$ (aka the diffusion time)
\item a velocity $\upnu_{ref} = L_{ref} / t_{ref} = \kappa_{ref}/L_{ref}$
\item an acceleration $g_{ref} = \upnu_{ref} / t_{ref} = \kappa_{ref}^2/L_{ref}^3$
\item a strain rate $\dot{\varepsilon}_{ref} = t_{ref}^{-1} = \kappa_{ref} / L_{ref}^2$
\item a pressure $p_{ref} = \eta_{ref} \dot{\varepsilon}_{ref} = \eta_{ref} t_{ref}^{-1}$
\item a reference density $\rho_{ref} = \eta_{ref} L_{ref} t_{ref}/L_{ref}^3 = \eta_{ref} L_{ref}^{-2} t_{ref}$
\item a reference mass $M_{ref} = \eta_{ref} L_{ref} t_{ref}$
\item a reference energy $E_{ref} = \eta_{ref} L_{ref} t_{ref} \frac{L_{ref}^2}{t_{ref}^2} 
= \eta_{ref} \frac{L_{ref}^3}{t_{ref}}$
\item a reference heat conductivity\footnote{Units: W/m/K} $k_{ref}= E_{ref}/t_{ref}/L_{ref}/T_{ref}
= \eta_{ref} L_{ref}^2/t_{ref}^2/T_{ref}$
\item a reference heat capacity\footnote{Units: J/kg/K} 
$C_{ref}=E_{ref}/M_{ref}/T_{ref} = L_{ref}^2 /t_{ref}^2/T_{ref} $
\item a reference heat production coefficient\footnote{Units: W/kg}
$H_{ref}= E_{ref}/t_{ref}/M_{ref} = \frac{L_{ref}^2}{t_{ref}^3}$
\item a reference heat flux\footnote{Units: \si{\watt\per\square\meter}, or \si{\kg\per\cubic\second}} 
$q_{ref}= \eta_{ref} L_{ref} t_{ref}^{-2}$
\end{itemize}
We define {\color{teal}dimensionless} quantities as follows:
\begin{equation}
{\color{teal}x} = \frac{x}{L_{ref}}
\quad
{\color{teal}\vec\upnu} = \frac{\vec\upnu}{\upnu_{ref}}
\quad
{\color{teal}t} = \frac{t}{t_{ref}}
\quad
{\color{teal} \eta} = \frac{\eta}{\eta_{ref}}
\quad
{\color{teal} g} =\frac{g}{g_{ref}}
\quad
{\color{teal} k} = \frac{k}{k_{ref}}
\quad
{\color{teal} C_p} = \frac{C_p}{C_{ref}}
\quad
{\color{teal} \rho} =\frac{\rho}{\rho_{\text{ref}}}
\end{equation}

\begin{equation}
{\color{teal} H} = \frac{H}{H_{ref}}
\quad
{\color{teal} \vec\nabla} = L_{ref}\; \vec\nabla
\quad
{\color{teal} \partial_t} = t_{ref}\; \partial_t 
\quad
{\color{teal} T} = \frac{T}{T_{ref}}
\quad
{\color{teal} \dot{\varepsilon}} = \dot{\varepsilon} \; t_{ref}
\label{eq:physics_adimrels}
\end{equation}
We start from the standard Navier-Stokes equation\footnote{\url{https://en.wikipedia.org/wiki/Navier-Stokes_equations}}
\[
\rho \frac{D \vec\upnu}{D t}
=
-\vec\nabla p + \vec\nabla \cdot (2 \eta \dot{\bm\varepsilon})
+ \rho \vec{g} 
\]
and assume that the density is temperature-dependent (Boussinesq approximation) so that
\[
\rho \frac{D \vec\upnu}{D t}
=
-\vec\nabla p + \vec\nabla \cdot (2 \eta \dot{\bm\varepsilon})
+ \rho_0(1-\alpha T) \vec{g} 
\]
and remove the hydrostatic pressure (although we keep using $p$ for simplicity, $p$ is now the dynamic pressure):
\[
\rho \frac{D \vec\upnu}{D t}
=
-\vec\nabla p + \vec\nabla \cdot (2 \eta \dot{\bm\varepsilon})
- \rho_0\alpha T \vec{g} 
\]
We divide this equation by $p_{ref}=\eta_{ref}\dot{\varepsilon}_{ref}$:
\[
\frac{1}{\eta_{ref}\dot{\varepsilon}_{ref}} \rho
\frac{D \vec\upnu}{D t} 
=
-\vec\nabla {\color{teal}p} + \vec\nabla \cdot 2 \frac{\eta}{\eta_{ref}} \frac{\dot{\bm\varepsilon}}{\dot{\varepsilon}_{ref}}
- \frac{\rho_0\alpha T \vec{g}}{\eta_{ref} \dot{\varepsilon}_{ref}} 
\]
Let us call $\vec{e}$ the positive vertical vector ($\vec{e}_z$ in Cartesian coordinates, $\vec{e}_r$ in spherical coordinates), then 
$\vec{g} = -g_0 \vec{e}$ and we can write  (using $\dot{\varepsilon}_{ref}=\kappa_{ref}/L^2_{ref}$)
\[
\frac{1 }{\eta_{ref}\dot{\varepsilon}_{ref}}
(\rho_{ref}{\color{teal}\rho})\frac{D(\upnu_{ref} {\color{teal}\vec\upnu})}{D t}
=
-\vec\nabla {\color{teal}p} + \vec\nabla \cdot 2 {\color{teal} \eta} {\color{teal} \dot{\bm\varepsilon}}
+ \frac{\rho_0\alpha T g_0 }{\eta_{ref} (\kappa_{ref}/ L_{ref}^2)} \vec{e}
\]
Finally, dividing by $L_{ref}^{-1}$ (i.e. multiplying by $L_{ref}$) yields
\[
\frac{\upnu_{ref} \rho_{ref} L_{ref}}{\eta_{ref}\dot{\varepsilon}_{ref}  }
{\color{teal}\rho} \frac{{\color{teal} D \vec\upnu}}{ t_{ref} {\color{teal} D t}}
=
-{\color{teal} \vec\nabla} {\color{teal}p} + {\color{teal} \vec\nabla} \cdot 2 {\color{teal} \eta} {\color{teal} \dot{\bm\varepsilon}}
+ \frac{\rho_0\alpha ({\color{teal}T}T_{ref}) g_0 L_{ref}^3}{\eta_{ref} \kappa_{ref}} \vec{e}
\]
and finally (using $\upnu_{ref}=L_{ref}/t_{ref}$) 
\[
\frac{\rho_{ref} \kappa_{ref}}{\eta_{ref}}
{\color{teal}\rho} \frac{{\color{teal} D \vec\upnu}}{\color{teal} D t}
=
-{\color{teal} \vec\nabla} {\color{teal}p} + {\color{teal} \vec\nabla} \cdot 2 {\color{teal} \eta} {\color{teal} \dot{\bm\varepsilon}}
+ \frac{\rho_0\alpha T_{ref} g_0 L_{ref}^3}{\eta_{ref} \kappa_{ref}} {\color{teal} T} \vec{e}
\]
In the context of a system with a temperature difference $\Delta T$ 
between the bottom and top boundaries separated by a distance $H$, one would then take $T_{ref} = \Delta T$ 
and $L_{ref}=H$ so that the equation becomes:
\[
\underbrace{\frac{\rho_{ref} \kappa_{ref}}{\eta_{ref}}}_{\Prnb^{-1}}
{\color{teal}\rho} \frac{{\color{teal} D \vec\upnu}}{\color{teal} D t}
=
-{\color{teal} \vec\nabla} {\color{teal}p} + {\color{teal} \vec\nabla} \cdot 2 {\color{teal} \eta} {\color{teal} \dot{\bm\varepsilon}}
+ \underbrace{\frac{\rho_0\alpha \Delta T g_0 H^3}{\eta_{ref} \kappa_{ref}} }_{\Ranb} {\color{teal} T}\vec{e}
\]
and we obviously recover the classical definition of the Rayleigh number.

\begin{mdframed}[backgroundcolor=blue!5]
\[
\frac{1}{\Prnb}
{\color{teal}\rho} \frac{{\color{teal} D \vec\upnu}}{\color{teal} D t}
=
-{\color{teal} \vec\nabla} {\color{teal}p} + {\color{teal} \vec\nabla} \cdot 2 {\color{teal} \eta} {\color{teal} \dot{\bm\varepsilon}}
+ \Ranb {\color{teal} T}\vec{e}
\]
\end{mdframed}


On the left side of the equation we recognize the (inverse of the) Prandlt number $\Prnb=\frac{\eta}{\rho \kappa}$. 
We can estimate the dimensionless number before the inertial term for Earth
geodynamics:
\[
\Prnb \simeq \frac{10^{20-23}}{3000 \cdot 10^{-6}} >> 10^{23}
\]
Its inverse is then extremely small and this is why we neglect the inertial terms
in mantle modelling.


Note that if the fluid is isoviscous, one can then set $\eta_{ref}=\eta=\eta_0$ and then ${\color{teal}\eta}=1$ 
%and then 
%\[
%-{\color{teal} \vec\nabla} {\color{teal}p} + {\color{teal} \vec\nabla} \cdot 2  {\color{teal} \dot{\bm\varepsilon}}
%+ \Ranb {\color{teal} T} \vec{e}= \vec{0}
%\]

Turning now to the continuity equation $\vec\nabla \cdot\vec\upnu = 0$,
it is trivial to show that  ${\color{teal} \vec\nabla } \cdot  {\color{teal}\vec\upnu} = 0$.
Finally, starting from the simple heat transport equation:
\[
\frac{\partial T}{\partial t} + \vec\upnu\cdot\vec\nabla T = \kappa \Delta T
\]
We divide each side by $T_{ref}$ so that 
\[
\frac{\partial {\color{teal}T}}{\partial t} + \vec\upnu\cdot\vec\nabla {\color{teal}T} = \kappa \Delta {\color{teal}T}
\]
We now divide each side by the reference velocity $\upnu_{ref}$ 
and we obtain
\[
\frac{L_{ref}}{\kappa_{ref}} \frac{\partial {\color{teal}T}}{\partial t} 
+ {\color{teal} \vec\upnu} \cdot\vec\nabla {\color{teal}T} 
=  \frac{L_{ref}}{\kappa_{ref}}  \kappa \Delta {\color{teal}T}
\]
We multiply each side by $L_{ref}$ and we finally get
\[
\frac{L_{ref}^2}{\kappa_{ref}} 
\frac{\partial {\color{teal}T}}{\partial t}
+ {\color{teal} \vec\upnu} \cdot  {\color{teal}\vec\nabla} {\color{teal}T} =  {\color{teal} \kappa} {\color{teal}\Delta} {\color{teal}T}
\]
and finally
\[
\frac{ {\color{teal} \partial T}}{ {\color{teal} \partial t}} 
+ {\color{teal} \vec\upnu} \cdot  {\color{teal}\vec\nabla} {\color{teal}T} =  {\color{teal} \kappa} {\color{teal}\Delta} {\color{teal}T}
\]
The set of dimensionless equations is then:

\begin{mdframed}[backgroundcolor=blue!5]
\begin{eqnarray}
-{\color{teal} \vec\nabla} {\color{teal}p} + {\color{teal} \vec\nabla} \cdot 2  {\color{teal}\eta \dot{\bm\varepsilon}}
+ \Ranb  {\color{teal} T} \vec{e} &=& \vec{0} \label{eq:adimmm1}\\
{\color{teal} \vec\nabla } \cdot  {\color{teal}\vec\upnu} &=& 0 \label{eq:adimmm2}\\
\frac{ {\color{teal} \partial T}}{ {\color{teal} \partial t}} 
+ {\color{teal} \vec\upnu} \cdot  {\color{teal}\vec\nabla} {\color{teal}T} 
&=&  {\color{teal} \kappa} {\color{teal}\Delta} {\color{teal}T} \label{eq:adimmm3}
\end{eqnarray}
\end{mdframed}

\index{general}{Extended Boussinesq Approximation}
\index{general}{EBA}
Looking now at the Extended Boussinesq Approximation (EBA), we have to conside two additional terms in the 
energy equation:
\begin{itemize}
\item the shear heating $\Phi$ (See Eq.\eqref{eq:physicsshearheating}) 
$\Phi = 2 \eta  \dot{\bm \varepsilon}^d : \dot{\bm \varepsilon}^d$ 
\item the adiabatic heating $\alpha T \vec\upnu\cdot\vec\nabla p$
\end{itemize}
We start this time from 
\[
\rho C_p \left(\frac{\partial T}{\partial t} + \vec \upnu\cdot\vec\nabla T\right)
- \vec\nabla\cdot k\vec\nabla T 
= \rho H + 2\eta\dot{\bm \varepsilon}^d : \dot{\bm \varepsilon}^d 
+\alpha T  \vec\upnu \cdot \vec\nabla p 
\]

\[
\rho C_p \left(\frac{\partial T}{\partial t} + \vec \upnu\cdot\vec\nabla T\right)
- \vec\nabla\cdot k\vec\nabla T 
= \rho H + 2\eta\dot{\bm \varepsilon}^d : \dot{\bm \varepsilon}^d 
+\alpha T  \vec\upnu \cdot \vec\nabla p 
\]

\[
\rho C_p \frac{T_{ref}}{t_{ref}} \left(\frac{\partial {\color{teal}T}}{\partial {\color{teal}t}} 
+ {\color{teal} \vec \upnu}\cdot {\color{teal}\vec\nabla} {\color{teal}T}\right)
- \frac{T_{ref}}{L_{ref}^2} {\color{teal}\vec\nabla}\cdot k
{\color{teal}\vec\nabla} {\color{teal}T} 
= \rho_{ref} {\color{teal}\rho} H + \frac{\eta_{ref}}{t_{ref}^2} 2 {\color{teal}\eta} 
{\color{teal}\dot{\bm \varepsilon}^d} : 
{\color{teal}\dot{\bm \varepsilon}^d} 
+ \frac{p_{ref}}{t_{ref}} {\color{teal} \alpha T}  
{\color{teal}\vec\upnu} \cdot {\color{teal} \vec\nabla}
{\color{teal} p} 
\]
we then use $p_{ref} = \eta_{ref} t_{ref}^{-1}$ and
$\rho_{ref} = \eta_{ref} L_{ref}^{-2} t_{ref}$
\[
\rho_{ref} C_{ref} \frac{T_{ref}}{t_{ref}} {\color{teal} \rho} {\color{teal}C_p} 
\left(\frac{ {\color{teal} \partial T}}{ {\color{teal} \partial t}} 
+ {\color{teal} \vec \upnu}\cdot {\color{teal}\vec\nabla} {\color{teal}T}\right)
- \frac{T_{ref} k_{ref}}{L_{ref}^2} 
{\color{teal}\vec\nabla}\cdot {\color{teal} k}
{\color{teal}\vec\nabla} {\color{teal}T} 
= \frac{\eta_{ref}}{L_{ref}^2} t_{ref}  \frac{L_{ref}^2}{t_{ref}^3} 
{\color{teal}\rho} {\color{teal} H} 
+ \frac{\eta_{ref}}{t_{ref}^2} 2 {\color{teal}\eta} 
{\color{teal}\dot{\bm \varepsilon}^d} : 
{\color{teal}\dot{\bm \varepsilon}^d} 
+ \frac{\eta_{ref}}{t_{ref}^2} {\color{teal} \alpha T}  
{\color{teal}\vec\upnu} \cdot {\color{teal} \vec\nabla}
{\color{teal} p} 
\]
or, multiplying all by $t^2_{ref}/\eta_{ref}$:
\[
\frac{t_{ref}^2}{\eta_{ref}}
\rho_{ref} C_{ref} \frac{T_{ref}}{t_{ref}} {\color{teal} \rho} {\color{teal}C_p} 
\left(\frac{ {\color{teal} \partial T}}{ {\color{teal} \partial t}} 
+ {\color{teal} \vec \upnu}\cdot {\color{teal}\vec\nabla} {\color{teal}T}\right)
- \frac{t_{ref}^2}{\eta_{ref}} \frac{T_{ref} k_{ref}}{L_{ref}^2} 
{\color{teal}\vec\nabla}\cdot {\color{teal} k}
{\color{teal}\vec\nabla} {\color{teal}T} 
=  
{\color{teal}\rho} {\color{teal} H} 
+  2 {\color{teal}\eta} 
{\color{teal}\dot{\bm \varepsilon}^d} : 
{\color{teal}\dot{\bm \varepsilon}^d} 
+  {\color{teal} \alpha T}  
{\color{teal}\vec\upnu} \cdot {\color{teal} \vec\nabla}
{\color{teal} p} 
\]
We then make use of $C_{ref}=L_{ref}^2 /t_{ref}^2/T_{ref}$
and $k_{ref}= \eta_{ref} L_{ref}^2/t_{ref}^2/T_{ref}$
to arrive at
\[
{\color{teal} \rho} {\color{teal}C_p} 
\left(\frac{ {\color{teal} \partial T}}{ {\color{teal} \partial t}} 
+ {\color{teal} \vec \upnu}\cdot {\color{teal}\vec\nabla} {\color{teal}T}\right)
- 
{\color{teal}\vec\nabla}\cdot {\color{teal} k}
{\color{teal}\vec\nabla} {\color{teal}T} 
=  
{\color{teal}\rho} {\color{teal} H} 
+  2 {\color{teal}\eta} 
{\color{teal}\dot{\bm \varepsilon}^d} : 
{\color{teal}\dot{\bm \varepsilon}^d} 
+  {\color{teal} \alpha T}  
{\color{teal}\vec\upnu} \cdot {\color{teal} \vec\nabla}
{\color{teal} p} 
\]






%_________________________________________________________________________
\section{The Navier-Stokes equations in cylindrical coordinates}
\begin{flushright} {\tiny {\color{gray} physics.tex}} \end{flushright}

In cylindrical coordinates, $(r,\theta,z)$, the continuity equation for an incompressible fluid is 
\begin{mdframed}[backgroundcolor=blue!5]
\begin{equation}
\frac{1}{r} \frac{\partial}{\partial r} (r \upnu_r) + 
\frac{1}{r} \frac{\partial}{\partial \theta} (\upnu_\theta) + 
\frac{\partial \upnu_z}{\partial z} =0
\end{equation}
\end{mdframed}
The Navier-Stokes equations of motion for an incompressible fluid with uniform viscosity are:
\begin{eqnarray}
\rho \left(  \frac{D\upnu_r}{Dt} -\frac{\upnu_\theta^2}{r} \right) 
&=& -\frac{\partial p}{\partial r} + f_r + \eta
\left( \Delta \upnu_r - \frac{\upnu_r}{r^2} - \frac{2}{r^2} \frac{\partial \upnu_\theta}{\partial \theta}
\right)
\nn\\
\rho \left(  \frac{D\upnu_\theta}{Dt} +\frac{\upnu_\theta \upnu_r}{r} \right) 
&=&
-\frac{1}{r} \frac{\partial p}{\partial \theta} + f_\theta + \eta
\left(
\Delta \upnu_\theta - \frac{\upnu_\theta}{r^2} + \frac{2}{r^2} \frac{\partial \upnu_r}{\partial\theta}
\right)
\nn\\
\rho \frac{D\upnu_z}{Dt} 
&=& 
-\frac{\partial p}{\partial z} + f_z + \eta \Delta \upnu_z
\end{eqnarray}
where the Lagrangian or material derivative is
\[
\frac{D}{Dt} = \frac{\partial}{\partial t} 
+ \upnu_r \frac{\partial}{\partial r}  
+ \frac{\upnu_\theta}{r} \frac{\partial}{\partial \theta}
+ \upnu_z \frac{\partial}{\partial z}  
\]
and the Laplacian operator is \index{general}{Laplace Operator} \index{general}{Laplacian} 
\begin{equation}
\Delta 
= \frac{\partial^2 }{\partial r^2}  +\frac{1}{r} \frac{\partial }{\partial r}
+ \frac{1}{r^2}  \frac{\partial^2}{\partial \theta^2}
+ \frac{\partial^2 }{\partial z^2}
\end{equation}


%------------------------------------------------------------------------------
\section{The Stokes equations in spherical coordinates}
\begin{flushright} {\tiny {\color{gray} physics.tex}} \end{flushright}

In spherical coordinates, $(r,\theta,\phi)$, the continuity equation for an incompressible fluid is 
\begin{mdframed}[backgroundcolor=blue!5]
\begin{equation}
\frac{1}{r^2} \frac{\partial}{\partial r} (r^2 \upnu_r) + 
\frac{1}{r \sin \theta} \frac{\partial}{\partial \theta} (\upnu_\theta \sin \theta)+
\frac{1}{r \sin \theta} \frac{\partial \upnu_\phi}{\partial \phi} = 0
\end{equation}
\end{mdframed}
Concerning the momentum equation, we start from 
\begin{eqnarray}
{\vec \nabla}\cdot {\bm \sigma} + {\vec f} &=& \vec{0} 
\end{eqnarray}
The buoyancy force $\vec{f}$ is nearly always given by 
$\vec{f}=\rho \vec{g} = - \rho g \; \vec{e}_r$ (with $g>0$), i.e. $f_\phi=f_\theta=0$, and then
\[
- {\vec \nabla}p + {\vec \nabla}\cdot {\bm \tau} - \rho g \vec{e}_r = \vec{0}
\]
or,
\begin{eqnarray}
- ({\vec \nabla}p)_r      + ({\vec \nabla}\cdot {\bm \tau})_r     &=& \rho {g} \nonumber\\
- ({\vec \nabla}p)_\theta + ({\vec \nabla}\cdot {\bm \tau})_\theta&=&0  \nonumber\\
- ({\vec \nabla}p)_\phi   + ({\vec \nabla}\cdot {\bm \tau})_\phi  &=&0  \nonumber
\end{eqnarray}
The pressure gradient is simply given by:
\begin{eqnarray}
({\vec \nabla}p)_r &=& \frac{\partial p}{\partial r}  \nonumber\\
({\vec \nabla}p)_\theta &=& \frac{1}{r}\frac{\partial p}{\partial \theta}  \nonumber\\
({\vec \nabla}p)_\phi &=& \frac{1}{r\sin\theta}\frac{\partial p}{\partial \phi}  \nonumber
\end{eqnarray}
We now turn to the remaining three components of the divergence of deviatoric stress in 
spherical coordinates $r,\theta,\phi$, which are given by\footnote{Would be nice to 
have a ref here} %Eq.(\ref{eq_divtensor}):

\begin{eqnarray}
({\vec \nabla}\cdot {\bm \tau})_r 
&=& 
\frac{\partial \tau_{rr}}{\partial r} 
+ \frac{1}{r} \frac{\partial \tau_{\theta r}}{\partial \theta} 
+ \frac{1}{r \sin\theta} \frac{\partial \tau_{\phi r}}{\partial \phi} 
+ \frac{2 \tau_{rr} - \tau_{\theta\theta} -\tau_{\phi\phi}}{r} 
+ \frac{\tau_{\theta r} \cot\theta}{r} 
\nonumber\\
({\vec \nabla}\cdot {\bm \tau})_\theta
&=& 
\frac{\partial \tau_{r\theta}}{\partial r} 
+ \frac{1}{r} \frac{\partial \tau_{\theta \theta}}{\partial \theta} 
+ \frac{1}{r \sin\theta} \frac{\partial \tau_{\phi \theta}}{\partial \phi} 
+ \frac{3 \tau_{\theta r} + (\tau_{\theta\theta}- \tau_{\phi\phi}) \cot\theta}{r} 
 \nonumber\\
({\vec \nabla}\cdot {\bm \tau})_\phi
&=& 
\frac{\partial \tau_{r\phi}}{\partial r} 
+ \frac{1}{r} \frac{\partial \tau_{\theta \phi}}{\partial \theta} 
+ \frac{1}{r \sin\theta} \frac{\partial \tau_{\phi\phi}}{\partial \phi} 
+\frac{3 \tau_{r \phi }+2 \tau_{\phi \theta} \cot\theta}{r} 
\end{eqnarray}
And finally the momentum equation writes: 
\begin{eqnarray}
-\frac{\partial p}{\partial r} 
+\frac{\partial \tau_{rr}}{\partial r} 
+ \frac{1}{r} \frac{\partial \tau_{\theta r}}{\partial \theta} 
+ \frac{1}{r \sin\theta} \frac{\partial \tau_{\phi r}}{\partial \phi} 
+ \frac{2 \tau_{rr} - \tau_{\theta\theta} -\tau_{\phi\phi}}{r} 
+ \frac{\tau_{\theta r} \cot\theta}{r} 
 &=&  \rho g
\nonumber\\
\nonumber\\
- \frac{1}{r}\frac{\partial p}{\partial \theta}  
+\frac{\partial \tau_{r\theta}}{\partial r} 
+ \frac{1}{r} \frac{\partial \tau_{\theta \theta}}{\partial \theta} 
+ \frac{1}{r \sin\theta} \frac{\partial \tau_{\phi \theta}}{\partial \phi} 
+ \frac{3 \tau_{\theta r} + (\tau_{\theta\theta}- \tau_{\phi\phi}) \cot\theta}{r} 
  &=& 0 
\nonumber\\
\nonumber\\
- \frac{1}{r\sin\theta}\frac{\partial p}{\partial \phi} 
+\frac{\partial \tau_{r\phi}}{\partial r} 
+ \frac{1}{r} \frac{\partial \tau_{\theta \phi}}{\partial \theta} 
+ \frac{1}{r \sin\theta} \frac{\partial \tau_{\phi\phi}}{\partial \phi} 
+\frac{3 \tau_{r \phi }+2 \tau_{\phi \theta} \cot\theta}{r} 
&=&0
\label{eq_st1}
\end{eqnarray}
The deviatoric stress tensor components are
\begin{eqnarray}
\tau_{rr} 
&=& 2 \eta 
\left[ \dot{\varepsilon}_{rr} - 
\frac{1}{3} (\dot{\varepsilon}_{rr} + \dot{\varepsilon}_{\theta\theta} + \dot{\varepsilon}_{\phi\phi} ) \right] \nonumber\\
\tau_{\theta\theta} &=& 2 \eta 
\left[ \dot{\varepsilon}_{\theta\theta} - 
\frac{1}{3} (\dot{\varepsilon}_{rr} + \dot{\varepsilon}_{\theta\theta} + \dot{\varepsilon}_{\phi\phi} ) \right] \nonumber\\
\tau_{\phi\phi} &=& 2 \eta 
\left[ \dot{\varepsilon}_{\phi\phi} - 
\frac{1}{3} (\dot{\varepsilon}_{rr} + \dot{\varepsilon}_{\theta\theta} + \dot{\varepsilon}_{\phi\phi} ) \right] \nonumber\\
\tau_{r\theta} &=&  2 \eta  \dot{\varepsilon}_{r\theta}\\
\tau_{r\phi} &=& 2 \eta  \dot{\varepsilon}_{r\phi} \\
\tau_{\theta\phi} &=&  2 \eta  \dot{\varepsilon}_{\theta\phi}
\end{eqnarray}
with
\begin{eqnarray}
\dot\varepsilon_{rr} 
&=& \frac{\partial \upnu_r}{\partial r} \nonumber\\
\dot\varepsilon_{\theta\theta} 
&=& \frac{\upnu_r}{r} + \frac{1}{r} \frac{\partial \upnu_\theta}{\partial \theta}  \nonumber\\
\dot\varepsilon_{\phi\phi} 
&=& \frac{1}{r \sin\theta} \frac{\partial \upnu_\phi}{\partial \phi} +
\frac{\upnu_r}{r} +\frac{\upnu_\theta \cot \theta}{r} \nonumber\\
\dot\varepsilon_{\theta r} = \dot\varepsilon_{r\theta}   
&=& \frac{1}{2} \left( r \frac{\partial}{\partial r} (\frac{\upnu_\theta}{r} ) 
+\frac{1}{r} \frac{\partial \upnu_r}{\partial \theta} \right) \nonumber\\
\dot\varepsilon_{\phi r} = \dot\varepsilon_{r\phi}      
&=&  \frac{1}{2} \left(  \frac{1}{r \sin\theta} \frac{\partial \upnu_r}{\partial \phi} 
+ r \frac{\partial }{\partial r} (\frac{\upnu_\phi}{r}) \right)  \nonumber\\
\dot\varepsilon_{\phi \theta} = \dot\varepsilon_{\theta\phi} 
&=& \frac{1}{2} \left( \frac{\sin \theta}{r} \frac{\partial }{\partial \theta} (\frac{\upnu_\phi}{\sin\theta}) + \frac{1}{r \sin\theta} \frac{\partial \upnu_\theta}{\partial \phi}    \right) \nonumber
\end{eqnarray}
We go further by assuming the fluid to be incompressible
(i.e. $\dot{\varepsilon}_{rr} + \dot{\varepsilon}_{\theta\theta} + \dot{\varepsilon}_{\phi\phi} =0$) and then:
\begin{eqnarray}
\tau_{rr} = 2 \eta \dot{\varepsilon}_{rr}  
&=&
2 \eta  \frac{\partial \upnu_r}{\partial r} \nonumber\\
\tau_{\theta\theta} = 2 \eta \dot{\varepsilon}_{\theta\theta} 
&=& 2\eta \left( \frac{\upnu_r}{r} + \frac{1}{r} \frac{\partial \upnu_\theta}{\partial \theta} \right)
\nonumber\\
\tau_{\phi\phi} = 2 \eta \dot{\varepsilon}_{\phi\phi}  
&=& 2\eta \left( 
 \frac{1}{r \sin\theta} \frac{\partial \upnu_\phi}{\partial \phi} +
\frac{\upnu_r}{r} +\frac{\upnu_\theta \cot \theta}{r}  \right)
\nonumber\\
\tau_{r\theta} =  2 \eta  \dot{\varepsilon}_{r\theta}
&=&
\eta \left( r \frac{\partial}{\partial r} (\frac{\upnu_\theta}{r} ) 
+\frac{1}{r} \frac{\partial \upnu_r}{\partial \theta} \right)
\nonumber\\
\tau_{r\phi} = 2 \eta  \dot{\varepsilon}_{r\phi}
&=&
\eta \left(  \frac{1}{r \sin\theta} \frac{\partial \upnu_r}{\partial \phi} 
+ r \frac{\partial }{\partial r} (\frac{\upnu_\phi}{r}) \right)
\nonumber\\
\tau_{\theta\phi} =  2 \eta  \dot{\varepsilon}_{\theta\phi} 
&=&
\eta \left( \frac{\sin \theta}{r} \frac{\partial }{\partial \theta} (\frac{\upnu_\phi}{\sin\theta}) + \frac{1}{r \sin\theta} \frac{\partial \upnu_\theta}{\partial \phi}    \right)
\nonumber
\end{eqnarray}
Inserting these expressions in Eq.~\eqref{eq_st1} is a cumbersome affair...
Under the assumption that the fluid is also isoviscous, 
we get\footnote{I have not thoroughly checked these equations yet} 
\begin{eqnarray}
\rho g
&=&
-\frac{\partial p}{\partial r} + \eta \left( \Delta \upnu_r - \frac{2\upnu_r}{r^2} 
-\frac{2}{r^2} \frac{\partial \upnu_\theta}{\partial \theta} - \frac{2 \upnu_\theta \cot \theta}{r^2}
-\frac{2}{r^2 \sin\theta} \frac{\partial \upnu_\phi}{\partial \phi}
\right) 
\nonumber\\
0
&=& 
-\frac{1}{r}\frac{\partial p}{\partial \theta} 
+\eta \left(
\Delta \upnu_\theta + \frac{2}{r^2} \frac{\partial \upnu_r}{\partial \theta}
-\frac{\upnu_\theta}{r^2 \sin^2\theta } -\frac{2 \cot \theta}{r^2 \sin\theta}
\frac{\partial \upnu_\phi}{\partial \phi}  
\right) 
\nonumber\\
0
&=& 
- \frac{1}{r \sin\theta} \frac{\partial p}{\partial \phi}  + \eta
\left(
\Delta \upnu_\phi + \frac{2}{r^2 \sin\theta} \frac{\partial \upnu_r}{\partial \phi}
-\frac{\upnu_\phi}{r^2 \sin^2 \theta} + \frac{2 \cot \theta}{r^2 \sin\theta}
\frac{\partial \upnu_\theta}{\partial \phi}
\right) \nn\\
\end{eqnarray}
and the Laplacian operator is \index{general}{Laplace Operator} \index{general}{Laplacian} 
\[
\Delta = \frac{1}{r^2} \frac{\partial }{\partial r}\left( r^2 \frac{\partial }{\partial r}\right)
+\frac{1}{r^2 \sin\theta} \frac{\partial }{\partial \theta}
\left(
\sin\theta \frac{\partial }{\partial\theta}
\right)
+ \frac{1}{r^2 \sin^2\theta} \frac{\partial^2 }{\partial\phi^2}
\]

\todo[inline]{
Before being used these equations should be checked against multiple sources.  
}

%------------------------------------------------------------------------------
\section{The equations for axisymmetric geometries \label{ss:axicyleqs}}
\begin{flushright} {\tiny {\color{gray} physics.tex}} \end{flushright}
\begin{flushright} {\tiny {\color{gray} axisymmetric\_eqs.tex}} \end{flushright}
%~~~~~~~~~~~~~~~~~~~~~~~~~~~~~~~~~~~~~~~~~~~~~~~~~~~~~~~~~~~~~~~~~~~~~~~~~~~~~~~~~~~~~~~~~~~~~~~~~~

In what follows we are concerned with incompressible flow.
In some cases the assumption can be made that the object we wish to sudy has an 
axisymmetric geometry, for example a plume:

\begin{center}
a)\includegraphics[width=5cm]{images/axisymmetry/keki97}
b)\includegraphics[width=5cm]{images/axisymmetry/lesy96a}
c)\includegraphics[width=5cm]{images/axisymmetry/lesy96b}\\
{\captionfont a)Taken from \textcite{keki97} (1997); 
b,c) Taken from \textcite{lesy96} (1996).}
\end{center}

Looking at the figure above we see that there are in fact two cases: axisymmetry in 
cylindrical coordinates (b) and axisymmetry in spherical coordinates (c).

As mentioned in \textcite{keki97} (1997): "By imposing axisymmetry,
we restrict the problem to two degrees of freedom, reducing the computational
effort significantly over 3D calculations."
However, \cite{reki04} (2004) also mention:
"An important caveat of axisymmetric calculations is that there are no variations 
in the $\phi$ direction (i.e., there are no $\phi$ derivatives in
the governing equations). Thus, as we get further from
the pole, the results become increasingly less physical.
Downwelling drips off the pole are actually downwelling
doughnuts that follow the entire small circle. In a fully
3D calculation, this doughnut feature would in reality be a drip."


See Section~\ref{ss:cyl_axi} for the FE formulation of these equations.

%---------------------------------------
\subsubsection{In cylindrical coordinates}

The velocity vector is $\vec{\upnu}=(\upnu_r,\upnu_\theta,\upnu_z)$. 
Due to the symmetry we have $\upnu_\theta=0$, $\partial_\theta \rightarrow 0$ 
and the Stokes equations 
then become \footnote{\url{https://en.wikipedia.org/wiki/Navier-Stokes_equations}}

\begin{eqnarray}
-\frac{\partial p}{\partial r} + \eta
\left(
\frac1r \frac{\partial}{\partial r} ( r  \frac{\partial \upnu_r}{\partial r}   ) 
+  \frac{\partial^2 \upnu_r}{\partial z^2} - \frac{\upnu_r}{r^2}
\right) +\rho g_r&=& 0 
\\
-\frac{\partial p}{\partial z} + \eta
\left(
\frac1r \frac{\partial}{\partial r} ( r  \frac{\partial \upnu_z}{\partial r}   ) 
+  \frac{\partial^2 \upnu_z}{\partial z^2} 
\right) +\rho g_z&=& 0 \\
\frac1r \frac{\partial}{\partial r} (r \upnu_r) + \frac{\partial \upnu_z}{\partial z} &=& 0
\end{eqnarray}


The strain rate tensor in cylindrical coordinates is given by 

\begin{eqnarray}
\dot\varepsilon_{rr} 
&=& \frac{\partial \upnu_r}{\partial r} 
\\
\dot\varepsilon_{\theta\theta} 
&=& \frac{\upnu_r}{r} + \frac{1}{r} \frac{\partial \upnu_\theta}{\partial \theta}  
\\
\dot\varepsilon_{\theta r} = \dot\varepsilon_{r\theta} 
&=& \frac{1}{2} \left(   \frac{\partial \upnu_\theta}{\partial r} - \frac{\upnu_\theta}{r} 
+\frac{1}{r} \frac{\partial \upnu_r}{\partial \theta}  \right)
\\
\dot\varepsilon_{zz} 
&=& \frac{\partial \upnu_z}{\partial z} 
\\
\dot{\varepsilon}_{rz} = \dot{\varepsilon}_{zr} 
&=& \frac{1}{2}\left( \frac{\partial \upnu_r}{\partial z} + \frac{\partial \upnu_z}{\partial r}  \right) 
\\
\dot{\varepsilon}_{\theta z} = \dot{\varepsilon}_{z \theta} &=& \frac{1}{2}\left( 
\frac{1}{r} \frac{\partial \upnu_z}{\partial \theta} + \frac{\partial \upnu_\theta}{\partial z}  \right) 
\end{eqnarray}

In the axisymmetric case, we have $\upnu_\theta=0$ and $\partial_\theta \rightarrow 0$ so that 
\begin{eqnarray}
\dot\varepsilon_{rr} &=& \frac{\partial \upnu_r}{\partial r}  \\
\dot\varepsilon_{\theta\theta} &=& \frac{\upnu_r}{r} \\
\dot\varepsilon_{r\theta} = \dot\varepsilon_{\theta r} &=& 0\\
\dot\varepsilon_{zz} &=& \frac{\partial \upnu_z}{\partial z} \\
\dot{\varepsilon}_{rz} = \dot{\varepsilon}_{zr} 
&=& \frac{1}{2}\left( \frac{\partial \upnu_r}{\partial z} + \frac{\partial \upnu_z}{\partial r}  \right) \\
\dot{\varepsilon}_{\theta z} = \dot{\varepsilon}_{z \theta} &=& 0
\end{eqnarray}
or, 
\[
\dot{\bm\varepsilon}
=
\left(
\begin{array}{ccc}
\dot\varepsilon_{rr} & 0 & \dot{\varepsilon}_{rz} \\
0 & \dot{\varepsilon}_{\theta\theta}  & 0 \\
\dot{\varepsilon}_{zr} & 0 & \dot\varepsilon_{zz}
\end{array}
\right)
\]

\Literature: Daly \& Raefsky (1985) \cite{dara85}, Kiefer \& Hager (1992) \cite{kiha92}.

This is implemented in \stone~36,63,90,91,92,96,106.

%---------------------------------------
\subsubsection{In spherical coordinates}

Assuming the flow velocity does not depend on $\phi$ ($\partial_\phi =0$) and therefore also that $\upnu_\phi=0$
\[
0=-\frac{\partial p}{\partial r} + f_r + \eta \left(\Delta v_r - \frac{2v_r}{r^2} -\frac{2}{r^2} \frac{\partial v_\theta}{\partial \theta} - \frac{2 v_\theta \cot \theta }{r^2} \right)
\]
\[
0 = -\frac{1}{r} \frac{\partial p}{\partial \theta} + \eta \left(\Delta v_\theta + \frac{2}{r^2} \frac{\partial v_r}{\partial \theta}  - \frac{v_\theta}{r^2 \sin^2 \theta} \right)
\]
with
\[
\Delta = \frac{1}{r^2} \frac{\partial }{\partial r}\left( r^2 \frac{\partial }{\partial r}\right)
+\frac{1}{r^2 \sin\theta} \frac{\partial }{\partial \theta}
\left(
\sin\theta \frac{\partial }{\partial\theta}
\right)
\]


\[
\Delta = \frac{1}{r^2} \frac{\partial }{\partial r}\left( r^2 \frac{\partial }{\partial r}\right)
+\frac{1}{r^2 \sin\theta} \frac{\partial }{\partial \theta}
\left(
\sin\theta \frac{\partial }{\partial\theta}
\right)
+ \frac{1}{r^2 \sin^2\theta} \frac{\partial^2 }{\partial\phi^2}
\]

THESE EQUATIONS SHOULD BE CHECKED and RE-CHECKED !!



From \cite{zebi93}:
\begin{equation}
\frac{1}{r^2} \frac{\partial}{\partial r} (r^2 \upnu_r) + 
\frac{1}{r \sin \theta} \frac{\partial}{\partial \theta} (\upnu_\theta \sin \theta)+
\frac{1}{r \sin \theta} \frac{\partial \upnu_\phi}{\partial \phi} = 0
\end{equation}
Pb with 1/r2 ??

\begin{eqnarray}
0 &=& -\frac{\partial p}{\partial r} + (1-\zeta) Ra \; r \; T + 
\frac{1}{r^2}\frac{\partial}{\partial r} \left( 2 \eta r^2 \frac{\partial \upnu_r}{\partial r} \right)
+ \frac{1}{r^2 \sin\theta} \frac{\partial}{\partial\theta} 
\left( \eta \sin\theta \frac{\partial \upnu_r}{\partial\theta} \right)
+\frac{\partial}{\partial \theta} \left(\eta \frac{\partial}{\partial r} \frac{\upnu_\theta}{r} \right)
\end{eqnarray}
where $\zeta=R_i/R_o$


The dimensional form of the energy equation in a spherical axisymmetric geometry is given by
(assuming the conductivity $k$ to be constant):
\[
\rho C_p \left( \frac{\partial T}{\partial t}  + 
\upnu_r \frac{\partial T}{\partial r} + \frac{\upnu_\theta}{r} \frac{\partial T}{\partial \theta}
\right)
=
k \frac{1}{r^2} \frac{\partial}{\partial r} \left( r^2 \frac{\partial T}{\partial r} \right)
+
k \frac{1}{r^2 \sin\theta} 
\frac{\partial}{\partial \theta} \left( \sin\theta \frac{\partial T}{\partial \theta}  \right) 
...
\]

THESE EQUATIONS SHOULD BE CHECKED and RE-CHECKED !!






\newpage
%---------------------------------
\section{The Boussinesq approximation}
\index{general}{Boussinesq Approximation}
\begin{flushright} {\tiny {\color{gray} physics.tex}} \end{flushright}

As nicely explained in Spiegel \& Veronis \cite{spve60}: "In the study of problems of thermal convection it is a frequent practice to simplify the basic equations by introducing certain approximations which are attributed to
Boussinesq (1903). The Boussinesq approximations can best be summarized by two
statements: 
\begin{enumerate}
\item The fluctuations in density which appear with the advent of motion
result principally from thermal (as opposed to pressure) effects. 
\item In the equations
for the rate of change of momentum and mass, density variations may be neglected except
when they are coupled to the gravitational acceleration in the buoyancy force."
\end{enumerate}
Note that their paper examines the Boussinesq approximation for compressible fluids.  

[from \aspect{} manual]
The Boussinesq approximation assumes that the density can be
considered constant in all occurrences in the equations with the exception of
the buoyancy term on the right hand side of \eqref{eq:stokes-1}. The primary
result of this assumption is that the continuity equation \eqref{eq:stokes-2}
will now read ${\vec \nabla}\cdot{\vec \upnu} = 0$.
This implies that the strain rate tensor is deviatoric.
Under the Boussinesq approximation, the equations are much simplified:

\begin{align}
  \label{eq:stokes-1a}
  -\vec\nabla \cdot \left[2\eta \dot{\bm \varepsilon}(\vec\upnu)
                \right] + \vec\nabla p &=
  \rho \vec{g}
  &
  & \textrm{in $\Omega$},
  \\
  \label{eq:stokes-2a}
  \vec\nabla \cdot (\rho \vec\upnu) &= 0
  &
  & \textrm{in $\Omega$},
  \\
  \label{eq:temperaturee}
  \rho_0 C_p \left(\frac{\partial T}{\partial t} + \vec\upnu \cdot\vec\nabla T\right)
  - \vec\nabla\cdot k\vec\nabla T
  &=
  \rho H
  &
  & \textrm{in $\Omega$}
\end{align}
Note that all terms on the rhs of the temperature equations have disappeared, with the exception 
of the source term.


%---------------------------------
\section{The Extended Boussinesq approximation}
\index{general}{Extended Boussinesq Approximation}
\index{general}{EBA}
\begin{flushright} {\tiny {\color{gray} physics.tex}} \end{flushright}

Yuen \etal (2007) \cite{yumc07} state that the background of the extended Boussinesq 
equations can be found described in 
Christensen and Yuen (1985) \cite{chyu85} and more completely in Matyska and Yuen (2007) \cite{mayu07}.

\Literature \cite{hayk91,hayk93}

\newpage
%%%%%%%%%%%%%%%%%%%%%%%%%%%%%%%%%%%%%%%%%%%%%%%%%%%%%%%%%%%%%%%%%%%%%%%%%%%%%%%%%%%%%%%%%%55
\section{Stokes equation for elastic medium}

\begin{flushright} {\tiny {\color{gray} elastic\_equations.tex}} \end{flushright}
%~~~~~~~~~~~~~~~~~~~~~~~~~~~~~~~~~~~~~~~~~~~~~~~~~~~~~~~~~~~~~~~~~~~~~~~~~~~~~~~~~~~~~~~~~~~~~~~~~~

{\large \color{orange} This will be moved to Section \ref{chapt:elasticity}}

What follows is mostly borrowed from Becker \& Kaus lecture notes \cite{beka}.

The strong form of the PDE that governs force balance in a medium is given by
\[
\vec{\nabla}\cdot{\bm \sigma}  + \vec{f} = \vec{0}
\]
where ${\bm \sigma}$ is the stress tensor and $\vec{f}$ is a body force.

The stress tensor is related to the strain tensor through the generalised 
Hooke's law\footnote{\url{https://en.wikipedia.org/wiki/Hooke's_law}}:
\begin{equation}
\sigma_{ij}=\sum_{kl}C_{ijkl}\varepsilon_{kl} 
\qquad
\text{or}
\qquad
{\bm \sigma} = {\bm C} : {\bm \varepsilon}
\label{eq:oone}
\end{equation}
where ${\bm C}$ is the fourth-order elastic tensor.

Due to the inherent symmetries of ${\bm \sigma}$, ${\bm \varepsilon}$, and ${\bm C}$, 
only 21 elastic coefficients of the latter are independent. 
For isotropic linear media (which have the same physical properties in any direction), ${\bm C}$ 
can be reduced to only two independent numbers (for example the bulk modulus $K$ and the shear modulus $G$ 
that quantify the material's resistance to changes in volume and to shearing deformations, respectively).
Thus
\[
C_{ijkl} = \lambda \delta_{ij}\delta_{kl} + \mu (\delta_{ik}\delta_{jl}+\delta_{il}\delta_{jk})
\]
so that Eq.~\eqref{eq:oone} becomes:
\[
\sigma_{ij}=\lambda \varepsilon_{kk} \delta_{ij} + 2\mu \varepsilon_{ij}
\]
or
\begin{mdframed}[backgroundcolor=blue!5]
\begin{equation}
{\bm \sigma}=\lambda (\vec{\nabla}\cdot\vec{u}) {\bm 1} +2\mu {\bm \varepsilon}(\vec{u}) \label{eq:twoELAST}
\end{equation}
\end{mdframed}
where $\lambda$ is the Lam\'e parameter and $\mu$ is the shear 
modulus\footnote{It is also sometimes written $G$}.
The term $\vec{\nabla}\cdot\vec{u}$ is the isotropic dilation.

\index{general}{Lam\'e Parameter} 
\index{general}{Shear Modulus}

This can be re-written in the 6-dimensional stress/strain space as
\[
\underbrace{
\left(
\begin{array}{c}
\sigma_{xx} \\
\sigma_{yy} \\
\sigma_{zz} \\
\sigma_{xy} \\
\sigma_{xz} \\
\sigma_{yz} 
\end{array}
\right)}
_{\vec{\sigma}}
=
\underbrace{
\left(
\begin{array}{cccccc}
\lambda+2\mu & \lambda & \lambda & 0 & 0 & 0 \\ 
\lambda & \lambda+2\mu & \lambda & 0 & 0 & 0 \\ 
\lambda & \lambda & \lambda+2\mu & 0 & 0 & 0 \\
0 & 0 & 0 & \mu & 0 & 0 \\ 
0 & 0 & 0 & 0 & \mu & 0 \\ 
0 & 0 & 0 & 0 & 0 & \mu  
\end{array}
\right)}
_{{\bm C}}
\cdot
\underbrace{
\left(
\begin{array}{c}
\varepsilon_{xx} \\
\varepsilon_{yy} \\
\varepsilon_{zz} \\
\varepsilon_{xy} \\
\varepsilon_{xz} \\
\varepsilon_{yz} 
\end{array}
\right)}
_{\vec{\varepsilon}}
\]
or, in terms of the compliance matrix ${\bm C}^{-1}$,
\index{general}{Compliance Matrix}
\[
\vec{\varepsilon} 
= {\bm C}^{-1} \cdot \vec{\sigma}
\]
with
\[
{\bm C}^{-1}
=
\frac{1}{\mu(3\lambda+2\mu)}
\left(
\begin{array}{cccccc}
\lambda+\mu & -\lambda/2 & -\lambda/2 & 0 & 0 & 0 \\
-\lambda/2 & \lambda+\mu & -\lambda/2 & 0 & 0 & 0 \\
-\lambda/2 & -\lambda/2 & \lambda+\mu & 0 & 0 & 0 \\
0 & 0 & 0 & 3\lambda+2\mu & 0 & 0 \\ 
0 & 0 & 0 & 0 & 3\lambda+2\mu & 0 \\ 
0 & 0 & 0 & 0 & 0 & 3\lambda+2\mu  
\end{array}
\right)
\]
If we define the Young's modulus as $E=\mu(3\lambda+2\mu)/(\lambda+\mu)$ 
and the Poisson's ratio as $\nu=\lambda(\lambda+\mu)/2$, then
\[
{\bm C}^{-1}
=
\frac{1}{E}
\left(
\begin{array}{cccccc}
1 & -\nu & -\nu & 0 & 0 & 0 \\
-\nu & 1 & -\nu & 0 & 0 & 0 \\
-\nu & -\nu & 1 & 0 & 0 & 0 \\
0 & 0 & 0 & 2(1+\nu) & 0 & 0 \\ 
0 & 0 & 0 & 0 & 2(1+\nu) & 0 \\ 
0 & 0 & 0 & 0 & 0 & 2(1+\nu) 
\end{array}
\right)
\]
Note that the determinant  of ${\bm C}^{-1}$ is $8(1+\nu)^5(1-2\nu)E^{-6}$,
so that when $\nu\rightarrow 1/2$ (incompressible material), the compliance
matrix is singular and the stress cannot be given as a function of strain \cite{lubliner}.


The strain tensor is related to the displacement as follows: \index{general}{Strain Tensor}
\[
{\bm \varepsilon}(\vec{u}) 
= \frac{1}{2}(\vec{\nabla}\vec{u} + (\vec{\nabla}\vec{u})^T)
\]
The incompressibility (or bulk modulus) $K$ is defined as $p=-K \vec{\nabla}\cdot\vec{u}$ 
where $p$ is the pressure with \index{general}{Bulk Modulus}
\begin{eqnarray}
p&=&-\frac{1}{3} \text{tr}({\bm \sigma}) \nonumber\\
 &=& -\frac{1}{3} [ \lambda (\vec{\nabla}\cdot\vec{u}) \text{tr}[{\bm 1}] + 2 \mu {\rm tr}[{\bm \varepsilon}(\vec{u})]] \nonumber\\
 &=& -\frac{1}{3} [ \lambda (\vec{\nabla}\cdot\vec{u})  3  + 2 \mu  (\vec{\nabla}\cdot\vec{u}) ] \nonumber\\
 &=& -\left[ \lambda + \frac{2}{3} \mu \right] (\vec{\nabla}\cdot\vec{u})  
\end{eqnarray}
so that 
\begin{mdframed}[backgroundcolor=blue!5]
\[
p=-K \vec{\nabla}\cdot\vec{u} 
\qquad
\text{with}
\qquad
K=\lambda+\frac{2}{3}\mu
\]
\end{mdframed}

\begin{remark}
Eq. (\ref{eq:oone}) and (\ref{eq:twoELAST}) are analogous to the ones that one has to solve
in the context of viscous flow using the penalty method. In this case $\lambda$ is the penalty coefficient, 
$\vec{u}$ is the velocity, and $\mu$ is then the dynamic viscosity.
\end{remark}

The Lam\'e parameter and the shear modulus are also linked to $\nu$ the poisson ratio, 
and $E$, Young's modulus: \index{general}{Poisson Ratio} \index{general}{Young's Modulus}
\[
\lambda=\mu\frac{2\nu}{1-2\nu}
=\frac{\nu E}{(1+\nu)(1-2\nu)}
\quad\quad
{\rm with}
\quad\quad
E=2\mu(1+\nu)
\]
The shear modulus, expressed often in GPa, describes the material's response to shear stress.
The poisson ratio describes the response in the direction orthogonal to uniaxial stress.
The Young modulus, expressed in GPa, describes the material's strain response to uniaxial stress in the 
direction of this stress.


\Literature: solvers for 3D Stokes and elasticity problems with
heterogeneous coefficients \cite{samb20}





















%%%%%%%%%%%%%%%%%%%%%%%%%%%%%%%%%%%%%%%%%%%%%%%%%%%%%%%%%%%%%%%%%%55
\newpage
\section{The strain rate tensor in all coordinate systems}
\begin{flushright} {\tiny {\color{gray} strainrate\_tensor.tex}} \end{flushright}


The strain rate tensor $\dot{\bm\varepsilon}(\vec\upnu)$ is given by
\begin{equation}
\dot{\bm \varepsilon}({\vec \upnu}) = \frac{1}{2}( {\vec \nabla}{\vec \upnu}+ ({\vec \nabla}{\vec \upnu})^T) 
\end{equation}

%.....................................
\subsection{Cartesian coordinates}
\begin{eqnarray}
\dot\varepsilon_{xx} &=& \frac{\partial u}{\partial x} \\
\dot\varepsilon_{yy} &=& \frac{\partial v}{\partial y} \\
\dot\varepsilon_{zz} &=& \frac{\partial w}{\partial z} \\
\dot\varepsilon_{yx} =
\dot\varepsilon_{xy} &=& \frac{1}{2} \left( \frac{\partial u}{\partial y} + \frac{\partial v}{\partial x}  \right)\\
\dot\varepsilon_{zx} =
\dot\varepsilon_{xz} &=& \frac{1}{2} \left( \frac{\partial u}{\partial z} + \frac{\partial w}{\partial x}  \right)\\
\dot\varepsilon_{zy} =
\dot\varepsilon_{yz} &=& \frac{1}{2} \left( \frac{\partial v}{\partial z} + \frac{\partial w}{\partial y}  \right)
\end{eqnarray}

In the \aspect manual there is an interesting discussion about the strain rate tensor in the case of 
2D models: "The notion we adopt here is to think of two-dimensional models in the following way: 
We assume that the domain we want to solve on is a two-dimensional
cross section (parameterized by x and y coordinates) that extends infinitely far in both negative and positive
z direction. Further, we assume that the velocity is zero in z direction and that all variables have no
variation in z direction. As a consequence, we ought to really think of these two-dimensional models as
three-dimensional ones in which the z component of the velocity is zero and so are all z derivatives."

This of course makes sense but it means that when the deviatoric strain rate tensor needs to be 
computed, then it is given by
\[
\dot{\bm \varepsilon}^d = \dot{\bm \varepsilon}^d - \frac{1}{\bm 3} (\vec\nabla\cdot\vec\upnu) {\bm 1}
=
\left(
\begin{array}{ccc}
\dot{\varepsilon}_{xx} & \dot{\varepsilon}_{xy} &  0 \\
\dot{\varepsilon}_{xy} & \dot{\varepsilon}_{yy} &  0 \\
0 &0 & 0
\end{array}
\right)
- \frac{1}{\bm 3} (\dot{\varepsilon}_{xx}+\dot{\varepsilon}_{yy}) {\bm 1}
=
\frac{1}{3}
\left(
\begin{array}{ccc}
2 \dot{\varepsilon}_{xx} - \dot{\varepsilon}_{yy} & 3\dot{\varepsilon}_{xy} &  0 \\
3\dot{\varepsilon}_{xy} & -\dot{\varepsilon}_{xx}+2 \dot{\varepsilon}_{yy} &  0 \\
0 &0 & -\dot{\varepsilon}_{xx}-\dot{\varepsilon}_{yy}
\end{array}
\right)
\]
As a consequence the shear heating term $\Phi$ is given by 
\begin{eqnarray}
\Phi = 2 \eta \dot{\bm \varepsilon}^d :\dot{\bm \varepsilon}^d 
&=& 2 \eta \frac19
\left[
(2\dot{\varepsilon}_{xx}-\dot{\varepsilon}_{yy})^2 +
(-\dot{\varepsilon}_{xx}+2\dot{\varepsilon}_{yy})^2 +
2\cdot 9\dot{\varepsilon}_{xy}^2
+( -\dot{\varepsilon}_{xx}-\dot{\varepsilon}_{yy})^2 \right] \nn\\
&=& 2\eta \frac19
\left[
4\dot{\varepsilon}_{xx}^2 - 4 \dot{\varepsilon}_{xx}\dot{\varepsilon}_{yy}
+\dot{\varepsilon}_{yy}^2
+ \dot{\varepsilon}_{xx}^2 - 4 \dot{\varepsilon}_{xx}\dot{\varepsilon}_{yy}
+ 4\dot{\varepsilon}_{yy}^2
+ 18 \dot{\varepsilon}_{xy}^2
+ \dot{\varepsilon}_{xx}^2 +2\dot{\varepsilon}_{xx}\dot{\varepsilon}_{yy}
+ \dot{\varepsilon}_{yy}^2 \right] \nn\\
&=& 2\eta \frac19
\left[ 6 \dot{\varepsilon}_{xx}^2 
+ 6 \dot{\varepsilon}_{yy}^2 
-6 \dot{\varepsilon}_{xx}\dot{\varepsilon}_{yy}
+ 18 \dot{\varepsilon}_{xy}^2 \right] \nn\\
&=& 2\eta \left[ \frac{2}{3} \dot{\varepsilon}_{xx}^2 
+ \frac23 \dot{\varepsilon}_{yy}^2 
-\frac23 \dot{\varepsilon}_{xx}\dot{\varepsilon}_{yy}
+ 2 \dot{\varepsilon}_{xy}^2 \right]
\end{eqnarray}


%.............................................
\subsection{Polar coordinates \label{ss:srpc}}

\begin{eqnarray}
\dot\varepsilon_{rr} 
&=& \frac{\partial \upnu_r}{\partial r} \\
\dot\varepsilon_{\theta\theta} 
&=& \frac{\upnu_r}{r} + \frac{1}{r} \frac{\partial \upnu_\theta}{\partial \theta}  \\
\dot\varepsilon_{\theta r} = \dot\varepsilon_{r\theta} 
&=& \frac{1}{2} \left(   \frac{\partial \upnu_\theta}{\partial r} - \frac{\upnu_\theta}{r} 
+\frac{1}{r} \frac{\partial \upnu_r}{\partial \theta}  \right) 
\end{eqnarray}



%........................................................
\subsection{Cylindrical coordinates \label{ss:srcc}}

\begin{eqnarray}
\dot\varepsilon_{rr} 
&=& \frac{\partial \upnu_r}{\partial r} 
\\
\dot\varepsilon_{\theta\theta} 
&=& \frac{\upnu_r}{r} + \frac{1}{r} \frac{\partial \upnu_\theta}{\partial \theta}  
\\
\dot\varepsilon_{\theta r} = \dot\varepsilon_{r\theta} 
&=& \frac{1}{2} \left(   \frac{\partial \upnu_\theta}{\partial r} - \frac{\upnu_\theta}{r} 
+\frac{1}{r} \frac{\partial \upnu_r}{\partial \theta}  \right)
\\
\dot\varepsilon_{zz} 
&=& \frac{\partial \upnu_z}{\partial z} 
\\
\dot{\varepsilon}_{rz} = \dot{\varepsilon}_{zr} 
&=& \frac{1}{2}\left( \frac{\partial \upnu_r}{\partial z} + \frac{\partial \upnu_z}{\partial r}  \right) 
\\
\dot{\varepsilon}_{\theta z} = \dot{\varepsilon}_{z \theta} &=& \frac{1}{2}\left( 
\frac{1}{r} \frac{\partial \upnu_z}{\partial \theta} + \frac{\partial \upnu_\theta}{\partial z}  \right) 
\end{eqnarray}


The velocity divergence is given by
\begin{eqnarray}
\vec{\nabla}\cdot\vec\upnu 
&=& \dot{\varepsilon}_{rr} + \dot{\varepsilon}_{\theta\theta} + \dot{\varepsilon}_{zz} 
 = \cfrac{\partial \upnu_r}{\partial r} + \cfrac{1}{r}\left(\cfrac{\partial \upnu_\theta}{\partial \theta} 
+ \upnu_r \right)  + \cfrac{\partial \upnu_z}{\partial z}
\end{eqnarray} 


%......................................................
\subsection{Spherical coordinates \label{ss:srsc}}

\begin{eqnarray}
\dot\varepsilon_{rr} 
&=& \frac{\partial \upnu_r}{\partial r} \\
\dot\varepsilon_{\theta\theta} 
&=& \frac{\upnu_r}{r} + \frac{1}{r} \frac{\partial \upnu_\theta}{\partial \theta}  \\
\dot\varepsilon_{\phi\phi} 
&=& \frac{1}{r \sin\theta} \frac{\partial \upnu_\phi}{\partial \phi} +
\frac{\upnu_r}{r} +\frac{\upnu_\theta \cot \theta}{r} \\
\dot\varepsilon_{\theta r} = \dot\varepsilon_{r\theta}   
&=& \frac{1}{2} \left( r \frac{\partial}{\partial r} (\frac{\upnu_\theta}{r} ) 
+\frac{1}{r} \frac{\partial \upnu_r}{\partial \theta} \right) \\
\dot\varepsilon_{\phi r} = \dot\varepsilon_{r\phi}      
&=&  \frac{1}{2} \left(  \frac{1}{r \sin\theta} \frac{\partial \upnu_r}{\partial \phi} 
+ r \frac{\partial }{\partial r} (\frac{\upnu_\phi}{r}) \right)  \\
\dot\varepsilon_{\phi \theta} = \dot\varepsilon_{\theta\phi} 
&=& \frac{1}{2} \left( \frac{\sin \theta}{r} \frac{\partial }{\partial \theta} (\frac{\upnu_\phi}{\sin\theta}) + \frac{1}{r \sin\theta} \frac{\partial \upnu_\theta}{\partial \phi}    \right) 
\end{eqnarray}



%...............................................................................
\subsection{Relationship between Cartesian and polar coordinates expressions}

We can go from Cartesian to polar coordinates  via the $2\times 2$ transformation matrix:
\begin{equation}
{\cal P}=
\left(
\begin{array}{ccc}
\cos\theta & \sin\theta \\
-\sin\theta & \cos\theta
\end{array}
\right)
\end{equation}
The rows correspond to the components of $\vec{e}_r$ and $\vec{e}_\theta$ in the Cartesian basis.
A vector $\vec{\upnu}$ transforms from one orthonormal basis to another by multiplying it by 
the matrix ${\cal P}$. As we have seen before, this yields
\begin{eqnarray}
\upnu_r &=& u \cos\theta + v \sin\theta \\
\upnu_\theta &=& -u \sin\theta + v \cos\theta
\end{eqnarray}
A second-order tensor ${\bm a}$ is Cartesian coordinates transforms into ${\bm a}^\star$
in polar coordinates by 
\[
{\bm a}^\star = {\cal P} \cdot {\bm a} \cdot {\cal P}^T
\]
and obviously 
\[
{\bm a} = {\cal P}^T \cdot {\bm a}^\star \cdot {\cal P}
\]
We obtain for the strain rate tensor (or the stress tensor):
\begin{eqnarray}
\dot{\varepsilon}_{rr} 
&=& \dot{\varepsilon}_{xx} \cos^2\theta + \dot{\varepsilon}_{yy} \sin^2\theta 
+ 2 \dot{\varepsilon}_{xy} \sin\theta\cos\theta \nn\\
\dot{\varepsilon}_{\theta\theta}
&=& \dot{\varepsilon}_{xx} \sin^2\theta + \dot{\varepsilon}_{yy} \cos^2\theta 
- 2 \dot{\varepsilon}_{xy} \sin\theta\cos\theta \nn\\
\dot{\varepsilon}_{r\theta} 
&=& \dot{\varepsilon}_{xy} (\cos^2\theta-\sin^2\theta) + 
(\dot{\varepsilon}_{yy} - \dot{\varepsilon}_{xx})\sin\theta \cos\theta \nn
\end{eqnarray}
Using the trigonometric identities $\sin 2\theta = 2 \sin\theta\cos\theta$
and $\cos^2\theta-\sin^2\theta = \cos 2\theta$
, then 
we obtain 
\begin{eqnarray}
\dot{\varepsilon}_{rr} 
&=& \dot{\varepsilon}_{xx} \cos^2\theta + \dot{\varepsilon}_{yy} \sin^2\theta 
+  \dot{\varepsilon}_{xy} \sin 2\theta \nn\\
\dot{\varepsilon}_{\theta\theta}
&=& \dot{\varepsilon}_{xx} \sin^2\theta + \dot{\varepsilon}_{yy} \cos^2\theta 
-  \dot{\varepsilon}_{xy} \sin 2\theta \nn\\
\dot{\varepsilon}_{r\theta} 
&=& \dot{\varepsilon}_{xy} \cos 2\theta + 
\frac12(\dot{\varepsilon}_{yy} - \dot{\varepsilon}_{xx}) \sin 2\theta  \nn
\end{eqnarray}



and likewise:
\begin{eqnarray}
\dot{\varepsilon}_{xx} 
&=& \dot{\varepsilon}_{rr} \cos^2\theta + \dot{\varepsilon}_{\theta\theta} \sin^2\theta - 2 \dot{\varepsilon}_{r\theta} \sin\theta\cos\theta \\
\dot{\varepsilon}_{yy}
&=& \dot{\varepsilon}_{rr} \sin^2\theta + \dot{\varepsilon}_{\theta\theta} \cos^2\theta + 2 \dot{\varepsilon}_{r\theta} \sin\theta\cos\theta \\
\dot{\varepsilon}_{xy} 
&=& \dot{\varepsilon}_{r\theta} (\cos^2\theta-\sin^2\theta) + 
(\dot{\varepsilon}_{rr} - \dot{\varepsilon}_{\theta\theta})\sin\theta \cos\theta \label{ss:srboth}
\end{eqnarray}



















\newpage
%-------------------------------
\section{Boundary conditions}
\begin{flushright} {\tiny {\color{gray} physics.tex}} \end{flushright}

%wiki
In mathematics, the Dirichlet (or first-type) 
boundary condition is a type of boundary condition, named after Peter Gustav Lejeune Dirichlet.
When imposed on an ODE or PDE, it specifies the values that a solution needs 
to take along the boundary of the domain.
Note that a Dirichlet boundary condition may also be referred to as a fixed boundary condition. 

The Neumann (or second-type) boundary condition is a type of boundary condition, 
named after Carl Neumann. When imposed on an ordinary or a partial differential equation, 
the condition specifies the values in which the derivative of a solution is 
applied within the boundary of the domain.

It is possible to describe the problem using other boundary conditions: 
a Dirichlet boundary condition specifies the values of the solution itself 
(as opposed to its derivative) on the boundary, whereas the Cauchy boundary condition, 
mixed boundary condition and Robin boundary condition are all different types of combinations 
of the Neumann and Dirichlet boundary conditions.

\index{general}{Dirichlet Boundary Condition}
\index{general}{Neumann Boundary Condition}

%....................................
\subsection{The Stokes equations}

You may find the following terms in the computational geodynamics literature:

\begin{itemize}
\item { free surface}: this means that no force is acting on the surface, i.e. ${\bm \sigma}\cdot {\vec n}={\vec 0}$. It is usually used on the top boundary of the domain and allows for topography evolution.
\item { free slip}: ${\vec \upnu}\cdot \vec n = 0$ and $({\bm \sigma}\cdot{\vec n})\times {\vec n}={\vec 0}$. This condition ensures a frictionless flow parallel to the boundary where it is prescribed.
\item { no slip}: this means that the velocity (or displacement) is exactly zero on the boundary, i.e. ${\vec \upnu}={\vec 0}$.
\item { prescribed velocity}: ${\vec \upnu}={\vec \upnu}_{bc}$
\item stress b.c.: 
\item open .b.c.: see \stone 29. 
\end{itemize}

%....................................
\subsection{The heat transport equation}

There are two types of boundary conditions for this equation: temperature boundary conditions (Dirichlet boundary conditions) and heat flux boundary conditions (Neumann boundary conditions). 

\newpage
%------------------------------------------
\section{Meaningful physical quantities}
\begin{flushright} {\tiny {\color{gray} physics.tex}} \end{flushright}

\begin{itemize}
\item {\color{violet} Velocity} $\vec \upnu (\text{m/s})$: This is a vector quantity and both magnitude and direction are needed to define it. It is the rate of change of position with respect to a frame of reference.
\item {\color{violet} Root mean square velocity} $\upnu_{rms} (\text{m/s})$: 
\begin{equation}
\upnu_{rms} = \left ( \frac{\int_\Omega |{\vec \upnu}|^2 \;  dV}{\int_\Omega dV }  \right )^{1/2}
=\left ( \frac{1}{V_\Omega} \int_\Omega |{\vec \upnu}|^2 \;  dV \right )^{1/2} \label{eqVrms}
\end{equation}
\begin{remark}
$V_\Omega$ is usually computed numerically at the same time that $\upnu_{vrms}$ is computed.
\end{remark}
In Cartesian coordinates, for a cuboid domain of size $Lx\times L_y \times Lz$, 
the $\upnu_{rms}$ is simply given by:
\begin{equation}
\upnu_{rms}  = \left ( \frac{1}{L_xL_yL_z} \int_0^{L_x}\int_0^{L_y}\int_0^{L_z} 
(u^2 + v^2 + w^2) dxdydz  \right )^{1/2}
\end{equation}
In the case of an annulus domain, although calculations are carried out 
in Cartesian coordinates, it makes sense
to look at the radial velocity component $\upnu_r$ and the tangential velocity 
component $\upnu_\theta$, and their respective
root mean square averages:
\begin{equation}
\upnu_r|_{rms}  =\left ( \frac{1}{V_\Omega} \int_\Omega v_r^2 \;  d \Omega \right )^{1/2} \label{eqVrVrms}
\end{equation}
\begin{equation}
\upnu_\theta|_{rms}  = \left ( \frac{1}{V_\Omega} \int_\Omega v_\theta^2 \;  d \Omega \right )^{1/2} \label{eqThetaVrms}
\end{equation}


\item {\color{violet} Pressure} $p$ (\si{\pascal}):
\item {\color{violet} Stress tensor} ${\bm \sigma}$ (\si{\pascal}): \index{general}{Stress Tensor}
\item {\color{violet} Strain tensor} ${\bm \varepsilon}$ (dimensionless): \index{general}{Strain Tensor}
\item {\color{violet} Strain rate tensor} $ \dot{\bm \varepsilon}$ (\si{\per\second}): 
\index{general}{Strain Rate Tensor}

%--------------------------------------------------------------------------------------------------
\item {\color{violet} Argand Number}: 
Non-dimensional number (Ar) representing the ratio of the stress arising 
from crustal thickness
contrasts (vertical stress) to the stress required to deform
the material at ambient strain rates (horizontal stress)
It is commonly used in mountain building
dynamics as a measure of the tendency of an orogen to
collapse under its own gravitational potential energy.
See England \& McKenzie \cite{enmc82}, Houseman \& England \cite{hoen86a}.

%--------------------------------------------------------------------------------------------------
\item {\color{violet} (Thermal) Rayleigh number} $\Ranb$ (or $\Ranb_T$) (X): \index{general}{Rayleigh Number}
It is a dimensionless number that expresses the	ratio of the driving forces to the opposing forces.
The buoyancy force comes from the volumetric thermal expansion while the viscous forces and 
the heat diffusivity oppose convection (the latter one smoothes out thermal gradients). 

The Rayleigh number for convection driven by a constant temperature hot base and a cold surface
in a domain of thickness $D$ is:
\[
\Ranb 
= \frac{\rho_0 g \alpha D^3 }{\eta \kappa}  \cdot  \Delta T
= \frac{\rho_0^2 C_p g \alpha D^3 \Delta T}{\eta k}
\]
The Rayleigh number for convection driven by a hot base (constant basal heat flow $q_b$)
and a colder surface is:
\[
\Ranb = \frac{\rho_0 g \alpha D^3}{\eta \kappa } \cdot  \frac{q_b D}{k}
\]  
The Rayleigh number for convection driven by internal heating $H$ (production per cubic meter) is:
\[
\Ranb = \frac{\rho_0 g \alpha D^3}{\eta \kappa} \cdot  \frac{H D^2}{k }
\]
The Rayleigh number for convection driven by both basal heat flow and internal heating is:	
\[
\Ranb = \frac{\rho_0 g \alpha D^3}{\eta \kappa} \cdot  \frac{q_b D + H D^2}{k }
\]
For convection to occur, the Rayleigh number must be larger than the so-called critical 
Rayleigh number, which ranges from 600 to 3000 (it depends on the boundary conditions and the 
geometry).
\index{general}{Critical Rayleigh Number}

%--------------------------------------------------------------------------------------------------
\item {\color{violet}Compositional Rayleigh Number} $\Ranb_C$: \index{general}{Compositional Rayleigh Number} 
\[
\Ranb_C= \frac{\Delta \rho_C  g  D^3}{\kappa \eta_0}
\]
where $\Delta \rho_C$ is the difference in density between the distinct material compositions
(when compared at identical temperatures). See for instance Trim \etal (2020) \cite{trlb20}.

%--------------------------------------------------------------------------------------------------
\item {\color{violet} Prandtl number} $\Prnb$ (X): \index{general}{Prandtl Number} 
It is named after the German physicist 
Ludwig Prandtl\footnote{\url{https://en.wikipedia.org/wiki/Ludwig_Prandtl}} 
and is defined as the ratio of momentum diffusivity to thermal diffusivity. 
It is given as: 
\[
\Prnb = \frac{\text{momentum diffusivity}}{\text{thermal diffusivity}} = \frac{\eta/\rho}{k/(\rho C_p)}= \frac{\eta C_p}{k}
\]
For Earth materials, we have $Pr \sim (10^{21} 1000)/3 >> 1$, 
which means that momentum diffusivity dominates.

%..........................................
\item {\color{violet} Nusselt number} $\Nunb$ (X): \index{general}{Nusselt Number}  
the Nusselt number ($\Nunb$) 
is the ratio of convective to conductive heat transfer across (normal to) the boundary. 
The conductive component is measured under the same conditions as the heat convection 
but with a (hypothetically) stagnant (or motionless) fluid.

In practice the Nusselt number $\Nunb$ of a layer (typically the mantle of a planet) is defined as follows:
\begin{equation}
\Nunb = \frac{q}{q_c}
\end{equation} 
where $q$ is the heat transferred by convection while $q_c=k \Delta T /D$ 
is the amount of heat that would be conducted through a layer of
thickness $D$ with a temperature difference $\Delta T$ across it with 
$k$ being the thermal conductivity.

For 2D Cartesian systems of size ($L_x$,$L_y$) the $\Nunb$ is computed \cite{blbc89}
\[
\Nunb = 
\frac{\frac{1}{L_x}\int_{0}^{L_x} k \frac{\partial T}{\partial y}(x,y=L_y) dx }
{-\frac{1}{L_x}\int_0^{L_x} k T(x,y=0) /L_y dx}
=-L_y \frac{\int_{0}^{L_x} \frac{\partial T}{\partial y}(x,y=L_y) dx }{\int_0^{L_x} T(x,y=0) dx}
\]
i.e. it is the mean surface temperature gradient
over the mean bottom temperature.

\todo[inline]{finish. not happy with definition. Look at literature}

Note that in the case when no convection takes place then the measured heat flux at the top is 
the one obtained from a purely conductive profile which yields $\Nunb$=1.

Note that a relationship  $\Ranb \propto \Nunb^\alpha $ exists between the Rayleigh 
number $\Ranb$ and the Nusselt number $\Nunb$ in convective systems, see \cite{wodd09} and references therein. 

Turning now to cylindrical geometries with inner radius $R_1$ and outer radius $R_2$, 
we define $f=R_1/R_2$. A small value of $f$ corresponds to a high
degree of curvature. We assume now that $R_2-R_1=1$, so that $R_2=1/(1-f)$ and $R_1=f/(1-f)$. 
Following \cite{jarv93}, the Nusselt number at the inner and outer boundaries are:
\begin{equation}
\boxed{
\Nunb_{inner} 
=\frac{f \ln f}{1-f} \frac{1}{2\pi} \int_0^{2\pi} \left( \frac{\partial T}{\partial r}\right)_{r=R_1}d\theta
}
\label{eqNuAnnIn}
\end{equation}
\begin{equation}
\boxed{
\Nunb_{outer} 
= \frac{\ln f}{1-f} \frac{1}{2\pi} \int_0^{2\pi} \left( \frac{\partial T}{\partial r} \right)_{r=R_2} d\theta
}
\label{eqNuAnnOut}
\end{equation}

Note that a conductive geotherm in such an annulus between temperatures $T_1$ and $T_2$ is given by 
\[
T_c(r)=\frac{\ln (r/R_2)}{\ln(R_1/R_2)} = \frac{\ln(r(1-f))}{\ln f}
\]
so that 
\[
\frac{\partial T_c}{\partial r} = \frac{1}{r}\frac{1}{\ln f} 
\]
We then find:
\begin{eqnarray}
\Nunb_{inner} 
&=& \frac{f \ln f}{1-f} \frac{1}{2\pi} \int_0^{2\pi} \left( \frac{\partial T_c}{\partial r} \right)_{r=R_1} d\theta
= \frac{f \ln f}{1-f} \frac{1}{R_1}\frac{1}{\ln f} 
= 1 \\
\Nunb_{outer} 
&=& \frac{\ln f}{1-f} \frac{1}{2\pi} \int_0^{2\pi} \left( \frac{\partial T_c}{\partial r} \right)_{r=R_2} d\theta 
= \frac{\ln f}{1-f} \frac{1}{R_2}\frac{1}{\ln f} = 1 
\end{eqnarray}
As expected, the recovered Nusselt number at both boundaries is exactly 1 when the temperature field is
given by a steady state conductive geotherm.

\todo[inline]{derive formula for Earth size R1 and R2}

\Literature \cite{hohr87}
 
%..........................................
\item {\color{violet} Temperature} (\si{\kelvin}):

%--------------------------------------------------------------------------------------------------
\item {\color{violet} (Dynamic) Viscosity} (\si{\pascal\second}): 
For air it is roughly $10^{-5}\si{\pascal\second}$, 
about $10^{-3}~\si{\pascal\second}$ for water, 
about $10^{10}~\si{\pascal\second}$ for ice and 
about $10^{17}~\si{\pascal\second}$ for salt. 

%--------------------------------------------------------------------------------------------------
\item {\color{violet} Entropy} $S$ (\si{\joule\per\kelvin})

%--------------------------------------------------------------------------------------------------
\item {\color{violet} (mass) Density} $\rho$ (\si{\kg\per\cubic\metre}):

%--------------------------------------------------------------------------------------------------
\item {\color{violet} Heat capacity} $C_p$ (\si{\joule\per\kelvin}): 
It is the measure of the heat/energy required to increase the 
temperature of a unit quantity of a substance by unit degree. Note that the {\it specific} 
heat capacity $c_P$ of a substance is the heat capacity of a sample of the substance 
divided by the mass of the sample, with units \si{\joule\per\kelvin\per\kg}.

``Different substances respond to heat in different ways. If a metal chair sits in the bright sun 
on a hot day, it may become quite hot to the touch. An equal mass of water under the same sun exposure will 
not become nearly as hot. This means that water has a high heat capacity (the amount of heat required to 
raise the temperature of an object by $1~\si{\celsius}$). Water is very resistant to changes in temperature, 
while metals generally are not.''
\footnote{\url{https://chem.libretexts.org/Bookshelves/Introductory_Chemistry/Introductory_Chemistry_(CK-12)/17\%3A_Thermochemistry/17.04\%3A_Heat_Capacity_and_Specific_Heat}}


%--------------------------------------------------------------------------------------------------
\item {\color{violet} Heat conductivity}, or thermal conductivity $k$ (\si{\watt\per\metre\per\kelvin}). 
It is the property of a material that indicates its ability to conduct heat. It appears primarily 
in Fourier's Law for heat conduction.
Note that it is a function of temperature, especially in mantle convection settings,
see Bonneville \& Capolsini (1999) \cite{boca99} and refs therein, 
Miyauchi \& Kameyama (2013) \cite{mika13}, Hofmeister \& Yuen (2007) \cite{hoyu07}. 
Note also that it can be a tensorial quantity in anisotropic context.
The heat conductivity of many rocks was determined in \cite{ando13}.

%--------------------------------------------------------------------------------------------------
\item {\color{violet} Heat diffusivity}: $\kappa=k/(\rho C_p)$ ($\si{\square\meter\per\second}$). 
Substances with high thermal diffusivity rapidly adjust their temperature to that of their surroundings, because they 
conduct heat quickly in comparison to their volumetric heat capacity or 'thermal bulk'.

%--------------------------------------------------------------------------------------------------
\item {\color{violet} thermal expansion} $\alpha$ (\si{\per\kelvin}): 
it is the tendency of a matter to change in volume in response to a change in temperature. 
Note that it is a function of temperature, especially in mantle convection settings \cite{mika13}.
\[
\alpha = \frac{1}{V} \left(\frac{\partial V}{\partial T}\right)_P
\]


%--------------------------------------------------------------------------------------------------
\item {\color{violet} Urey Ratio}: mantle heat production divided by heat loss. It is a key constraint 
for thermal history models. Recent Urey ratio estimates are in the range of 0.21-0.49. \cite{lecm11}

%--------------------------------------------------------------------------------------------------
\item {\color{violet} Shear modulus}: modulus of rigidity, usually expressed in \si{\giga\pascal}. 
It describes the material response to shear stress.

%--------------------------------------------------------------------------------------------------
\item {\color{violet} Poisson ratio}: response in the direction orthogonal to uniaxial stress.

%--------------------------------------------------------------------------------------------------
\item {\color{violet} Young's modulus}: describes the material strain response to uniaxial stress
in the direction of this stress, usually expressed in \si{\giga\pascal}.

%--------------------------------------------------------------------------------------------------
\item {\color{violet} Average viscosity}: following Christensen (1983) \cite{chri83b}, 
one can compute the averaged viscosity in a domain as follows:
\begin{equation}
\langle \eta \rangle = \frac{\int_V \eta \dot{\varepsilon}_e^2 dV}{\int_V  \dot{\varepsilon}_e^2 dV }
\label{eq:avrgeta}
\end{equation}


\end{itemize}


\todo[inline]{check aspect manual The 2D cylindrical shell benchmarks by Davies \etal 5.4.12}


\newpage
%------------------------------------------------------
\section{Principal stress and principal invariants} \label{sec:princ_stress}
\begin{flushright} {\tiny {\color{gray} physics.tex}} \end{flushright}

\index{general}{Maximum Shear Stress} 
\index{general}{Principal Stress}

As seen before (see Section~\ref{sec:stresstensor}) 
the stress tensor is a symmetric $3\times3$ real matrix, and linear algebra tells us that it 
therefore has three mutually orthogonal unit-length eigenvectors $\vec{n}_{1}$, $\vec{n}_{2}$, 
$\vec{n}_{3}$ and three real eigenvalues $\lambda _{1},\lambda _{2},\lambda _{3}$ 
such that ${\bm \sigma}\!\cdot\! \vec{n}_i=\lambda_{i} \vec{n}_{i}$.

%from wiki stress 
As a consequence, in a coordinate system with axes $\vec{n}_{1},\vec{n}_{2},\vec{n}_{3}$, 
the stress tensor is a diagonal matrix, and has only the three normal components $\lambda _{1},\lambda _{2},\lambda _{3}$
i.e. the principal stresses. If the three eigenvalues are equal, the stress is an isotropic compression or tension, always perpendicular to any surface, there is no shear stress, and the tensor is a diagonal matrix in any coordinate frame.

\subsection{In two dimensions}

We are looking for the stress tensor eigenvector vector $\vec{n}=(n_x,n_y)$ associated to the
eigenvalue $\lambda$ such that 
\[
\left(
\begin{array}{cc}
\sigma_{xx} & \sigma_{xy} \\
\sigma_{xy} & \sigma_{yy} 
\end{array}
\right)
\cdot
\left(
\begin{array}{c}
n_x \\ n_y
\end{array}
\right)
=
\lambda
\left(
\begin{array}{c}
n_x \\ n_y
\end{array}
\right)
\]
or,
\[
\left(
\begin{array}{cc}
\sigma_{xx} & \sigma_{xy} \\
\sigma_{xy} & \sigma_{yy} 
\end{array}
\right)
\cdot
\left(
\begin{array}{c}
n_x \\ n_y
\end{array}
\right)
-
\left(
\begin{array}{cc}
\lambda & 0 \\ 
0 & \lambda 
\end{array}
\right)
\cdot
\left(
\begin{array}{c}
n_x \\ n_y
\end{array}
\right)
= \vec{0}
\]
i.e.,
\[
\left(
\begin{array}{cc}
\sigma_{xx}-\lambda  & \sigma_{xy} \\
\sigma_{xy} & \sigma_{yy} -\lambda 
\end{array}
\right)
\cdot
\left(
\begin{array}{c}
n_x \\ n_y
\end{array}
\right)
= \vec{0}
\]
which yields
\[
(\sigma_{xx}-\lambda)(\sigma_{yy}-\lambda)-\sigma_{xy}^2 =0
\]
or, 
\[
\lambda^2 - (\sigma_{xx}+\sigma_{yy}) \lambda   + (\sigma_{xx}\sigma_{yy}-\sigma_{xy}^2) =0
\]
The discriminant $\Delta$ is 
\begin{eqnarray}
\Delta 
&=& (\sigma_{xx}+\sigma_{yy})^2-4(\sigma_{xx}\sigma_{yy}-\sigma_{xy}^2)  \nn\\
&=& (\sigma_{xx}-\sigma_{yy})^2 +4\sigma_{xy}^2  \nn
\end{eqnarray}
The roots are given by:
\begin{eqnarray}
\lambda_\pm 
&=& \frac{ (\sigma_{xx}+\sigma_{yy}) \pm \sqrt{ (\sigma_{xx}-\sigma_{yy})^2 +4\sigma_{xy}^2 } }{2} \nn\\
&=& \frac{ \sigma_{xx}+\sigma_{yy}}{2} \pm \sqrt{ \left(\frac{\sigma_{xx}-\sigma_{yy}}{2}\right)^2 +\sigma_{xy}^2 } \nn
\end{eqnarray}
The two principal stresses are then:
\begin{mdframed}[backgroundcolor=blue!5]
\begin{eqnarray}
\sigma_1 &=& \frac{ \sigma_{xx}+\sigma_{yy}}{2} 
+ \sqrt{ \left(\frac{\sigma_{xx}-\sigma_{yy}}{2}\right)^2 +\sigma_{xy}^2 } \nn\\
\sigma_2 &=& \frac{ \sigma_{xx}+\sigma_{yy}}{2} 
- \sqrt{ \left(\frac{\sigma_{xx}-\sigma_{yy}}{2}\right)^2 +\sigma_{xy}^2 } \label{eq:princ_stress_2D} 
\end{eqnarray}
\end{mdframed}
with the convention $\sigma_1>\sigma_2$.
The maximum shear stress is defined as one-half the difference between the two principal 
stresses 
\begin{mdframed}[backgroundcolor=blue!5]
\begin{equation}
\tau_{\text max}=
\frac{\sigma_1-\sigma_2}{2}
=\sqrt{ \left(\frac{\sigma_{xx}-\sigma_{yy}}{2}\right)^2 +\sigma_{xy}^2 }
\label{eq:max_shear_stress_2D} 
\end{equation}
\end{mdframed}
The eigenvector $\vec{n}_1$ corresponding to $\sigma_1$ is obtained by solving 
\[
{\bm \sigma}\!\cdot\! \vec{n}_1 = \sigma_1 \vec{n}_1
\]
and same for the other eigenvalue/vector:
\[
{\bm \sigma} \!\cdot\! \vec{n}_2 = \sigma_2 \vec{n}_2
\]
Each is a system of two equations with two unknowns. These are not difficult to solve, 
but can prove cumbersome. Note that linear algebra tells us that $\vec{n}_1\cdot\vec{n}_2=0$, 
i.e. the eigenvectors form a basis of $\mathbb{R}^2$.

This is the reason why often people go another route. One can ask the question: what is the 
value of the angle $\theta_p$ which, if used to perform a rotation of the axis system, yields 
a stress tensor that is diagonal, with the principal stresses on the diagonal?
 
\begin{center}
\includegraphics[width=8cm]{images/princ_stress/PrincipalStress}\\
{\scriptsize Taken from \url{https://www.efunda.com/formulae/solid_mechanics/mat_mechanics/plane_stress_principal.cfm}}
\end{center}
The rotation matrix is 
\[
{\bm R}=
\left(
\begin{array}{cc}
\cos\theta_p & -\sin\theta_p \\
\sin\theta_p & \cos\theta_p
\end{array}
\right)
\]
and the image of ${\bm \sigma}$ by means of the axis rotation is 
${\bm \sigma}'= {\bm R}\cdot {\bm \sigma}\cdot {\bm R}^{-1}$, i.e.
\begin{eqnarray}
{\bm \sigma}' 
&=&
\left(
\begin{array}{cc}
\cos\theta_p & -\sin\theta_p \\
\sin\theta_p & \cos\theta_p
\end{array}
\right)
\cdot
\left(
\begin{array}{cc}
\sigma_{xx} & \sigma_{xy} \\
\sigma_{xy} & \sigma_{yy} 
\end{array}
\right)
\cdot
\left(
\begin{array}{cc}
\cos\theta_p & \sin\theta_p \\
-\sin\theta_p & \cos\theta_p
\end{array}
\right) \nn\\
&=&\left(
\begin{array}{cc}
\cos\theta_p & -\sin\theta_p \\
\sin\theta_p & \cos\theta_p
\end{array}
\right)
\cdot
\left(
\begin{array}{cc}
\sigma_{xx} \cos\theta_p - \sigma_{xy} \sin\theta_p  &
\sigma_{xx} \sin\theta_p + \sigma_{xy} \cos\theta_p  \\
\sigma_{xy} \cos\theta_p - \sigma_{yy} \sin\theta_p & 
\sigma_{xy} \sin\theta_p + \sigma_{yy} \cos\theta_p 
\end{array}
\right) \nn\\
&=&
\left(
\begin{array}{cc}
\dots & 
\cos\theta_p(\sigma_{xx} \sin\theta_p + \sigma_{xy} \cos\theta_p )-
\sin\theta_p(\sigma_{xy} \sin\theta_p + \sigma_{yy} \cos\theta_p ) \\
\dots & \dots 
\end{array}
\right) \nn 
\end{eqnarray}
In the matrix above I have only computed the off diagonal term since 
we are actually looking for $\theta_p$ such that $\sigma_{xy}'=0$, or
\begin{eqnarray}
\cos\theta_p(\sigma_{xx} \sin\theta_p + \sigma_{xy} \cos\theta_p )-
\sin\theta_p(\sigma_{xy} \sin\theta_p + \sigma_{yy} \cos\theta_p ) &=& 0 \nn\\
\sin\theta_p \cos\theta_p (\sigma_{xx}-\sigma_{yy}) + ( \cos^2\theta_p -\sin^2\theta_p )\sigma_{xy} &=& 0 \nn 
\end{eqnarray}
and then 
\[
\frac{ \sin\theta_p \cos\theta_p}{ \cos^2\theta_p -\sin^2\theta_p }
= \frac{\sigma_{xy}}{ \sigma_{xx}-\sigma_{yy} }
\]
The left hand term is actually a trigonometric 
identity\footnote{\url{https://en.wikipedia.org/wiki/List_of_trigonometric_identities}}:
\[
\frac{ \sin\theta_p \cos\theta_p}{ \cos^2\theta_p -\sin^2\theta_p } 
= \frac{\frac12 \sin 2\theta_p}{\cos 2\theta_p}
= \frac{1}{2} \tan 2\theta_p
\]
and finally:
\[
\tan 2\theta_p = \frac{ 2\sigma_{xy}}{ \sigma_{xx}-\sigma_{yy} }
\qquad
\text{or}
\qquad
\boxed{
\theta_p = \frac{1}{2} \tan^{-1} \frac{ 2\sigma_{xy}}{ \sigma_{xx}-\sigma_{yy} }
}
\]
Once $\theta_p$ has been found the other direction is given by $\theta_p +\pi/2$.

\vspace{.5cm}

\noindent \underline{Example}: Let us assume a diagonal stress tensor of the form 
\[
{\bm \sigma} = 
\left(
\begin{array}{cc}
a & 0 \\
0 & b
\end{array}
\right)
\]
then $\tan 2\theta_p = 0$, and then $\theta_p=0$. The principal directions are the horizontal and 
vertical directions, i.e. the Cartesian axis system, which is consistent.


\todo[inline]{add a remark that 2D does not exist and that plane strain incompressible actually is
what is going on above}


%...................................... 
\subsection{In three dimensions}

We are looking for the stress tensor eigenvector vector $\vec{n}=(n_x,n_y,n_z)$ associated to the
eigenvalue $\lambda$ such that 
\[
\left(\begin{array}{ccc}
\sigma_{xx} & \sigma_{xy} & \sigma_{xz} \\
\sigma_{xy} & \sigma_{yy} & \sigma_{yz} \\
\sigma_{xz} & \sigma_{yz} & \sigma_{zz}
\end{array}\right)
\cdot
\left(\begin{array}{c}
n_x \\ n_y \\ n_z
\end{array}\right)
=
\lambda
\left(\begin{array}{c}
n_x \\ n_y \\ n_z
\end{array}\right)
\]
or,
\[
\left(\begin{array}{ccc}
\sigma_{xx} & \sigma_{xy} & \sigma_{xz} \\
\sigma_{xy} & \sigma_{yy} & \sigma_{yz} \\
\sigma_{xz} & \sigma_{yz} & \sigma_{zz}
\end{array}\right)
\cdot
\left(\begin{array}{c}
n_x \\ n_y \\ n_z
\end{array}\right)
-
\left(\begin{array}{ccc}
\lambda & 0 & 0\\ 
0 & \lambda  & 0 \\
0 & 0 & \lambda 
\end{array}\right)
\cdot
\left(\begin{array}{c}
n_x \\ n_y \\ n_z
\end{array}\right)
= \vec{0}
\]

\[
\left(\begin{array}{ccc}
\sigma_{xx}-\lambda & \sigma_{xy} & \sigma_{xz} \\
\sigma_{xy} & \sigma_{yy}-\lambda & \sigma_{yz} \\
\sigma_{xz} & \sigma_{yz} & \sigma_{zz} -\lambda
\end{array}\right)
\cdot
\left(\begin{array}{c}
n_x \\ n_y \\ n_z
\end{array}\right)
= \vec{0}
\]
Non-trivial solutions of this equation require 
\[
\left|  
\begin{array}{ccc}
\sigma_{xx}-\lambda & \sigma_{xy} & \sigma_{xz} \\
\sigma_{xy} & \sigma_{yy}-\lambda & \sigma_{yz} \\
\sigma_{xz} & \sigma_{yz} & \sigma_{zz} -\lambda
\end{array}
\right|
=0
\]
Expanding the determinant results in the following cubic equation:
\begin{eqnarray}
0 
&=&
(\sigma_{xx}-\lambda) [ ( \sigma_{yy}-\lambda )( \sigma_{zz} -\lambda)- \sigma_{yz}^2]
- \sigma_{xy} [ \sigma_{xy} ( \sigma_{zz} -\lambda) - \sigma_{yz} \sigma_{xz} ] 
+ \sigma_{xz} [ \sigma_{xy}  \sigma_{yz} - (\sigma_{yy}-\lambda )  \sigma_{xz} ]  \nn\\
&=& (\sigma_{xx}-\lambda) [ \sigma_{yy}\sigma_{zz} -\lambda (\sigma_{yy}+ \sigma_{zz})+ \lambda ^2- \sigma_{yz}^2]
- \sigma_{xy} [ \sigma_{xy} ( \sigma_{zz} -\lambda) - \sigma_{yz} \sigma_{xz} ] 
+ \sigma_{xz} [ \sigma_{xy}  \sigma_{yz} - (\sigma_{yy}-\lambda )  \sigma_{xz} ] \nn\\
&=& -\lambda^3
+ (\sigma_{xx} + \sigma_{yy} + \sigma_{zz} )\lambda^2
+ (-\sigma_{yy}\sigma_{zz} -\sigma_{xx}\sigma_{yy} -\sigma_{xx}\sigma_{zz} 
  +\sigma_{yz}^2 +\sigma_{xy}^2 + \sigma_{xz}^2 )\lambda
+ det({\bm \sigma}) \nn
\end{eqnarray}
or, after multiplying the last line by -1,
\begin{equation}
\lambda^3 - {\cal K}_1({\bm \sigma}) \lambda^2 + {\cal K}_2({\bm \sigma}) \lambda -{\cal K}_3({\bm \sigma})=0
\label{eq:prinv:KKK}
\end{equation}
with\footnote{Note that in the equation \eqref{eq:prinv:KKK} there is often a plus sign in front of ${\cal K}_2$ 
but not always. Be careful when reading literature!}:
\begin{eqnarray}
{\cal K}_1({\bm \sigma}) &=& \sigma_{xx}+\sigma_{yy}+\sigma_{zz}\nn\\
{\cal K}_2({\bm \sigma}) &=& \sigma_{xx}\sigma_{yy}+\sigma_{yy}\sigma_{zz}+\sigma_{xx}\sigma_{zz}
-\sigma_{xy}^2 -\sigma_{xz}^2 -\sigma_{yz}^2 \nn\\
{\cal K}_3({\bm \sigma}) 
&=& \det ({\bm \sigma}) \nn\\
&=& \sigma_{xx}\sigma_{yy}\sigma_{zz}-\sigma_{xx}\sigma_{yz}^2
-\sigma_{xy}^2\sigma_{zz}+\sigma_{xy}\sigma_{yz}\sigma_{xz}
+\sigma_{xz}\sigma_{xy}\sigma_{yz}-\sigma_{xz}^2\sigma_{yy} \nn\\
&=& \sigma_{xx}\sigma_{yy}\sigma_{zz}
+2\sigma_{xy}\sigma_{yz}\sigma_{xz}
-( \sigma_{xx}\sigma_{yz}^2 +  \sigma_{zz} \sigma_{xy}^2 + \sigma_{yy}\sigma_{xz}^2 )
\end{eqnarray}
\index{general}{Principal Invariants}

\noindent ${\cal K}_1$, ${\cal K}_2$ and ${\cal K}_3$ are called {\bf principal}
invariants\footnote{\url{https://en.wikipedia.org/wiki/Invariants_of_tensors}} 
(see also Appendix A.1 of Zienkiewicz \& Taylor \cite{zita2} or  Eq.~(6.4) of Freudenthal \& Geiringer \cite{frge58}). 
These invariants can be written in a coordinate-free manner\footnote{Proofs are in 
Appendix~\ref{app:invariants}}:
\begin{mdframed}[backgroundcolor=blue!5]
\begin{eqnarray}
{\cal K}_1({\bm \sigma}) &=& {\rm tr}({\bm \sigma})  \nn\\
{\cal K}_2({\bm \sigma}) &=& \frac{1}{2}(  {\rm tr}({\bm \sigma}) ^2 - {\rm tr}({\bm \sigma}^2)  ) \nn\\
{\cal K}_3({\bm \sigma}) &=& \det ({\bm \sigma}) \nn
\end{eqnarray}
\end{mdframed}
and if the stress tensor is diagonal, we have
\begin{eqnarray}
{\cal K}_1({\bm \sigma}) &=& \sigma_{1}+\sigma_{2}+\sigma_{3}\nn\\
{\cal K}_2({\bm \sigma}) &=& \sigma_{1}\sigma_{2}+\sigma_{2}\sigma_{3}+\sigma_{1}\sigma_{3} \nn\\
{\cal K}_3({\bm \sigma}) &=& \sigma_1\sigma_2\sigma_3 \nn
\end{eqnarray}
The principal invariants ${\cal K}_{\{1,2,3\}}$ are related to the {\bf moment} 
invariants ${\cal I}_{\{1,2,3\}}$ 
(see Section~\ref{sec:invariants}) as follows (Appendix A.2 of Zienkiewicz \& Taylor \cite{zita2}):
\begin{eqnarray}
{\cal I}_1({\bm \sigma})&=& {\cal K}_1({\bm \sigma}) \\ 
{\cal I}_2({\bm \sigma})&=& \frac{1}{2}{\cal K}_1({\bm \sigma})^2 -{\cal K}_2({\bm \sigma}) \label{eq:IK2}\\
{\cal I}_3({\bm \sigma})&=& \frac{1}{3}{\cal K}_1({\bm \sigma})^3 -{\cal K}_1({\bm \sigma}) 
{\cal K}_2({\bm \sigma}) + {\cal K}_3({\bm \sigma}) \label{eq:IK3}
\end{eqnarray}
\todo[inline]{write proofs in appendix}
Very often we will find ourselves interested in the principal components 
of the deviatoric stress tensor $\bm \tau$ so that we now have the following determinant to compute:
\[
\left|  
\begin{array}{ccc}
\tau_{xx}-\lambda & \tau_{xy} & \tau_{xz} \\
\tau_{xy} & \tau_{yy}-\lambda & \tau_{yz} \\
\tau_{xz} & \tau_{yz} & \tau_{zz} -\lambda
\end{array}
\right|
=0
\]
and therefore obtain the following cubic equation
\begin{equation}
\lambda^3 - {\cal K}_1({\bm \tau}) \lambda^2 + {\cal K}_2({\bm \tau}) \lambda -{\cal K}_3({\bm \tau})=0
\end{equation}
By definition of a deviatoric tensor we have ${\cal K}_1({\bm \tau})=0$ and 
then Eqs.~\eqref{eq:IK2} and \eqref{eq:IK3} become
\begin{eqnarray}
{\cal I}_2({\bm \tau}) &=&  - {\cal K}_2({\bm \tau}) \\
{\cal I}_3({\bm \tau}) &=&  {\cal K}_3({\bm \tau}) 
\end{eqnarray}
so that the cubic equation becomes
\begin{equation} 
\lambda^3 -  {\cal I}_2({\bm \tau}) \lambda -  {\cal I}_3({\bm \tau}) =0 \label{opopop}
\end{equation}
Noting the trigonometric identity\footnote{see section 7.4 of Owen \& Hinton \cite{owhi}}
\begin{equation}
\sin 3\theta = 3 \sin \theta - 4 \sin^3 \theta
\qquad
{\rm or,}
\qquad
\sin^3 \theta - \frac{3}{4}\sin \theta + \frac{1}{4} \sin 3\theta = 0\label{pc_eq2}
\end{equation}
and substituting $\lambda=r\sin \theta$ into (\ref{opopop}) we have\footnote{Note that $r$ and $\theta$ have nothing 
to do with polar, cylindrical or spherical coordinates.}
\begin{equation}
\sin^3 \theta -\frac{ {\cal I}_2({\bm \tau})   }{r^2} \sin \theta -\frac{ {\cal I}_3({\bm \tau})  }{r^3}=0\label{pc_eq3}
\end{equation}
Comparing (\ref{pc_eq2}) and (\ref{pc_eq3}) gives
\begin{eqnarray}
r&=&\frac{2}{\sqrt{3}}\sqrt{ {\cal I}_2({\bm \tau})  }\label{pc_eq4bis} \\
\sin 3 \theta &=& -\frac{4 {\cal I}_3({\bm \tau})  }{r^3}=-\frac{3\sqrt{3}}{2}\frac{ {\cal I}_3({\bm \tau}) }{ {\cal I}_2({\bm \tau}) ^{3/2}} \label{pc_eq4}
\end{eqnarray}

The so-called Lode angle  \cite{zico74} is then given by \index{general}{Lode Angle}
\begin{mdframed}[backgroundcolor=blue!5]
\begin{equation}
\theta_{\rm L}=\frac{1}{3} \sin^{-1} 
\left( -\frac{3\sqrt{3}}{2} \frac{{\cal I}_3({\bm \tau})}{{\cal I}_2({\bm \tau})^{3/2}} \right)
\label{eq:lodang}
\end{equation}
\end{mdframed}
with $-\pi/6 <\theta_{\rm L} <\pi/6 $. The very same equation is 
also found in Willett (1992) \cite{will92} for instance.

The first root of (\ref{pc_eq4}) with $\theta_{\rm L}$ determined for $3\theta_{\rm L}$ in the 
range $\pm \pi/2$ is a convenient alternative to the third invariant, ${\cal I}_3({\bm \tau})$. 
By noting the cyclic nature of $\sin (3\theta_{\rm L}+2n \pi)$ we have immediatly the three 
(and only three) possible values of $\sin \theta_{\rm L} $ which define the three principal stresses. 
The deviatoric principal stresses are given by $\lambda=r \sin \theta_{\rm L}$ on substitution 
of the three values of $\sin \theta_{\rm L}$ in turn. 

We then obtain 
\begin{equation}
\left\{
\begin{array}{c}
\tau_1 \\ \\
\tau_2 \\ \\
\tau_3
\end{array}
\right\}
= \frac{2  }{\sqrt{3}}\sqrt{ {\cal I}_2({\bm \tau})  }
\left\{
\begin{array}{c}
\sin (\theta_{\rm L} + 2\pi/3)  \\ \\
\sin \theta_{\rm L}   \\ \\
\sin (\theta_{\rm L} + 4\pi/3  )
\end{array}
\right\}
\end{equation}
with $\tau_1>\tau_2>\tau_3$ and $-\pi/6 \leq \theta_{\rm L} \leq \pi/6$. It is indeed easy to verify that 
for $-\pi/6 \leq \theta_{\rm L} \leq \pi/6$ we have  
$\sin (\theta_{\rm L} + 2\pi/3) > \sin \theta_{\rm L} > \sin (\theta_{\rm L} + 4\pi/3)$.

Finally, we wish to compute the principal stresses of the full stress tensor ${\bm \sigma}$.
In the right coordinate system both stress and deviatoric stress tensors are diagonal and 
${\bm \sigma}=-p {\bm 1} + {\bm \tau}$ writes:
\[
\left(
\begin{array}{ccc}
\sigma_1 &0 &0 \\
0& \sigma_2 &0 \\
0&0 & \sigma_3  
\end{array}
\right)
=
\left(
\begin{array}{ccc}
-p&0&0\\
0&-p&0\\
0&0&-p
\end{array}
\right)
+
\left(
\begin{array}{ccc}
\tau_1 & 0&0 \\
0& \tau_2 & 0\\
0&0 & \tau_3  
\end{array}
\right)
\]
so that (since $p=-\frac{1}{3}tr({\bm \sigma})=-\frac{1}{3}{\cal I}_1({\bm \sigma})$) 
\begin{eqnarray}
\sigma_1 &=& \tau_1 - p = \tau_1 + \frac{1}{3}{\cal I}_1({\bm \sigma})\\ 
\sigma_2 &=& \tau_2 - p = \tau_2 + \frac{1}{3}{\cal I}_1({\bm \sigma})\\ 
\sigma_3 &=& \tau_3 - p = \tau_3 + \frac{1}{3}{\cal I}_1({\bm \sigma}) 
\end{eqnarray}
and finally the total principal stresses are
\begin{equation}
\left\{
\begin{array}{c}
\sigma_1 \\ \\
\sigma_2 \\ \\
\sigma_3
\end{array}
\right\}
= \frac{2  }{\sqrt{3}}\sqrt{ {\cal I}_2({\bm \tau})  }
\left\{
\begin{array}{c}
\sin (\theta_{\rm L} + 2\pi/3)  \\ \\
\sin \theta_{\rm L}   \\ \\
\sin (\theta_{\rm L} + 4\pi/3  )
\end{array}
\right\}
+
\frac{{\cal I}_1({\bm \sigma})}{3}
\left\{
\begin{array}{c}
1 \\ \\
1 \\ \\
1
\end{array}
\right\}
\end{equation}
with $\sigma_1>\sigma_2>\sigma_3$ and $-\pi/6 \leq \theta_{\rm L} \leq \pi/6$. 
We have
\begin{eqnarray}
\sin (\theta_{\rm L} + 2\pi/3)  
&=& \sin \theta_{\rm L} \cos 2\pi/3 + \cos \theta_{\rm L} \sin 2\pi/3 \nn\\
&=& -\frac{1}{2}\sin \theta_{\rm L}  + \cos \theta_{\rm L} \frac{\sqrt{3}}{2} \\
\sin (\theta_{\rm L} + 4\pi/3)  
&=& \sin \theta_{\rm L} \cos 4\pi/3 + \cos \theta_{\rm L} \sin 4\pi/3  \nn\\
&=& -\frac{1}{2} \sin \theta_{\rm L} - \cos \theta_{\rm L} \frac{\sqrt{3}}{2} 
\end{eqnarray}
so that 
\begin{eqnarray}
\left\{
\begin{array}{c}
\sigma_1 \\ \\
\sigma_2 \\ \\
\sigma_3
\end{array}
\right\}
&=& \frac{2  }{\sqrt{3}}\sqrt{ {\cal I}_2({\bm \tau}) }
\left\{
\begin{array}{c}
-\frac{1}{2}\sin \theta_{\rm L}  + \cos \theta_{\rm L} \frac{\sqrt{3}}{2} \\ \\
\sin \theta_{\rm L}   \\ \\
-\frac{1}{2} \sin \theta_{\rm L} - \cos \theta_{\rm L} \frac{\sqrt{3}}{2} 
\end{array}
\right\}
+
\frac{{\cal I}_1({\bm \sigma})}{3}
\left\{
\begin{array}{c}
1 \\ \\
1 \\ \\
1
\end{array}
\right\} \\
&=& \sqrt{ {\cal I}_2({\bm \tau}) }
\left\{
\begin{array}{c}
-\frac{1}{\sqrt{3}}\sin \theta_{\rm L}  + \cos \theta_{\rm L} \\ \\
\frac{2}{\sqrt{3}} \sin \theta_{\rm L} \\ \\
-\frac{1}{\sqrt{3}}\sin \theta_{\rm L}  - \cos \theta_{\rm L} 
\end{array}
\right\}
+
\frac{{\cal I}_1({\bm \sigma})}{3}
\left\{
\begin{array}{c}
1 \\ \\
1 \\ \\
1
\end{array}
\right\}
\label{eq:sig123}
\end{eqnarray}





\begin{remark} The Lode angle is one of the Lode 
coordinates\footnote{\url{https://en.wikipedia.org/wiki/Lode_coordinates}},
or Haigh-Westergaard coordinates. 
\index{general}{Haigh-Westergaard Coordinates}
\index{general}{Lode Coordinates}
\end{remark}

\begin{remark} The Lode angle $\theta_{\rm L}$ is essentially similar to the 
Lode parameter \index{general}{Lode Parameter} defined by $-\sqrt{3}\tan\theta$ \cite{owhi}.
\end{remark}

\begin{remark}
There are 3 different Lode angles, as 
explained online\footnote{\url{https://en.wikipedia.org/wiki/Lode_coordinates}}:
\[
\sin 3\theta_s = -\sin 3 \bar{\theta}_s = \cos 3\theta_c = \frac{3\sqrt{3}}{2}\frac{{\cal I}_3({\bm \tau})}{({\cal I}_2({\bm \tau}))^{3/2}}
\]
and they are related by $\theta_s = \frac{\pi}{6}-\theta_c$ and $\theta_s = -\bar{\theta}_s$. 
The one used in this document is in fact the $\bar{\theta}_s$ above.
\label{rq:signs}
\end{remark}

%\newpage
To recap:
\begin{mdframed}[backgroundcolor=blue!5]
\begin{eqnarray}
\sigma_1 &=& \frac{{\cal I}_1({\bm \sigma})}{3} + \sqrt{{\cal I}_2({\bm \tau})} \left(-\frac{1}{\sqrt{3}}\sin \theta_{\rm L}  +\cos\theta_{\rm L} \right) \label{eq:sigma1} \\ 
\sigma_2 &=& \frac{{\cal I}_1({\bm \sigma})}{3} + \sqrt{{\cal I}_2({\bm \tau})} \left(\frac{2}{\sqrt{3}}\sin 
\theta_{\rm L}   \right)    \label{eq:sigma2} \\
\sigma_3 &=& \frac{{\cal I}_1({\bm \sigma})}{3} + \sqrt{{\cal I}_2({\bm \tau})} \left(-\frac{1}{\sqrt{3}}\sin \theta_{\rm L}  - \cos \theta_{\rm L} \right)    \label{eq:sigma3}
\end{eqnarray}
\end{mdframed}


We will later need $\sigma_1-\sigma_3$ and $\sigma_1+\sigma_3$ so we compute these
quantities hereafter:

\begin{eqnarray}
\sigma_1 -\sigma_3
&=&  \sqrt{{\cal I}_2({\bm \tau})} \left( 
-\frac{1}{\sqrt{3}}\sin \theta_{\rm L}  + \cos \theta_{\rm L} 
+\frac{1}{\sqrt{3}}\sin \theta_{\rm L}  + \cos \theta_{\rm L} \right) \nn\\
&=& 2 \cos \theta_{\rm L} \sqrt{{\cal I}_2({\bm \tau})} \\ 
\sigma_1 + \sigma_3 
&=&   
\frac{{\cal I}_1({\bm \sigma})}{3} + \sqrt{{\cal I}_2({\bm \tau})} \left(-\frac{1}{\sqrt{3}}\sin 
\theta_{\rm L}  + \cos \theta_{\rm L} \right)   
+\frac{{\cal I}_1({\bm \sigma})}{3} + \sqrt{{\cal I}_2({\bm \tau})} \left(-\frac{1}{\sqrt{3}}\sin 
\theta_{\rm L}  - \cos \theta_{\rm L} \right)   
 \nn\\
&=& 
\frac{2}{3} {\cal I}_1({\bm \sigma}) -\sqrt{ {\cal I}_2({\bm \tau})} \frac{2}{\sqrt{3}}\sin \theta_{\rm L} 
\end{eqnarray}
or, 
\begin{mdframed}[backgroundcolor=blue!5]
\begin{eqnarray}
\frac{\sigma_1 -\sigma_3}{2} &=&  \cos \theta_{\rm L} \sqrt{{\cal I}_2({\bm \tau})}  \label{eq:sig13a} \\
\frac{\sigma_1 + \sigma_3}{2} &=& \frac{1}{3} {\cal I}_1({\bm \sigma}) -\sqrt{{\cal I}_2({\bm \tau})} \frac{1}{\sqrt{3}}\sin \theta_{\rm L} \label{eq:sig13b}
\end{eqnarray}
\end{mdframed}



\begin{remark}
The expression for the Lode angle is different in \cite[p101]{book_zitf} than in \cite{zico74} or \cite[p62]{zita2}. They all look suspiciously wrong too.
\end{remark}


%.........................................................................
\subsection{About the 2nd principal invariant of the deviatoric stress}

\begin{eqnarray}
{\cal K}_2({\bm \tau}) 
&=& \frac{1}{2}[{\rm Tr}({\bm \tau}) ^2 - {\rm Tr}({\bm \tau}^2)] \nonumber\\
&=& \frac{1}{2}[ (\tau_{xx}+\tau_{yy})^2 - (\tau_{xx}^2+2\tau_{xy}^2+\tau_{yy}^2)] \nonumber\\
&=& \frac{1}{2}[ \tau_{xx}^2+2\tau_{xx}\tau_{yy} +\tau_{yy}^2 - \tau_{xx}^2-2\tau_{xy}^2-\tau_{yy}^2] \nonumber\\
&=& \frac{1}{2}[ 2\tau_{xx}\tau_{yy} -2\tau_{xy}^2] \nonumber\\
&=& \tau_{xx}\tau_{yy} -\tau_{xy}^2 \nonumber\\
&=& \left(\sigma_{xx}-\frac{\sigma_{xx}+\sigma_{yy}}{2}\right)
\left(\sigma_{yy}-\frac{\sigma_{xx}+\sigma_{yy}}{2}\right)
-\tau_{xy}^2 \nonumber\\
&=& \left(\sigma_{xx}-\frac{\sigma_{xx}+\sigma_{yy}}{2}\right)
\left(\sigma_{yy}-\frac{\sigma_{xx}+\sigma_{yy}}{2}\right)
-\tau_{xy}^2 \nonumber\\
&=& \left(\frac{\sigma_{xx}-\sigma_{yy}}{2}\right)
\left(\frac{-\sigma_{xx}+\sigma_{yy}}{2}\right)
-\tau_{xy}^2 \nonumber\\
&=& -\left(\frac{\sigma_{xx}-\sigma_{yy}}{2}\right)^2
-\tau_{xy}^2 \nonumber
\end{eqnarray}
Looking at Eq.~\eqref{eq:max_shear_stress_2D}, we can then write
\begin{mdframed}[backgroundcolor=blue!5]
\[
\tau_{max}= \sqrt{-{\cal K}_2(\bm\tau)}
=\sqrt{ \left(\frac{\sigma_{xx}-\sigma_{yy}}{2}\right)^2 +\sigma_{xy}^2 }
\]
\end{mdframed}


%------------------------------------------------------
\section{Tensor (moment) invariants}\label{sec:invariants}
\begin{flushright} {\tiny {\color{gray} physics.tex}} \end{flushright}

\index{general}{Tensor Invariant}
\index{general}{Moment Invariant}

There are many different notations used in the literature for invariants 
and these can prove to be 
confusing\footnote{No kidding, true story.}. Note that we only consider symmetric tensors in what follows.
Given a tensor $\bm{T}$,  one can compute its (moment) invariants as follows 
(see \cite[p.339]{reddybook2}, or Appendix A.2 of \cite{zita2})

\begin{eqnarray}
{\cal I}_1({\bm T}) 
&=& {\rm tr}[\bm{T}] \\
&=& T_{xx} + T_{yy} + T_{zz} \\ 
{\cal I}_2({\bm T}) 
&=& \frac{1}{2} {\rm tr}[{\bm T}\cdot{\bm T}] \\
&=& \frac{1}{2} \sum_{ij} T_{ij} T_{ji} \\
&=& \frac{1}{2} (T_{xx}^2 + T_{yy}^2 + T_{zz}^2) + T_{xy}^2 + T_{xz}^2 + T_{yz}^2 \\
{\cal I}_3({\bm T}) 
&=& \frac{1}{3} {\rm tr}[{\bm T}\cdot{\bm T}\cdot {\bm T}]   \\
&=& \frac{1}{3}\sum_i\sum_j \sum_k T_{ij} T_{jk} T_{ki}  
%&=& \frac{1}{3} (T_{xx} ( T_{xx}T_{xx} + T_{xy}T_{xy} + T_{xz}T_{xz} )) \qquad (i=j=x,k=x,y,z)\nn\\ 
%&&+ \frac{1}{3} (T_{yy} ( T_{yx}T_{yx} + T_{yy}T_{yy} + T_{yz}T_{yz} )) \qquad (i=j=y,k=x,y,z)\nn\\ 
%&&+ \frac{1}{3} (T_{zz} ( T_{zx}T_{zx} + T_{zy}T_{zy} + T_{zz}T_{zz} )) \qquad (i=j=z,k=x,y,z)\nn\\ 
%&&+\frac{2}{3} (T_{xy} ( T_{xx}T_{yx} + T_{xy}T_{yy} + T_{xz}T_{yz} )) \qquad (i=x,j=y,k=x,y,z)\nn\\ 
%&&+ \frac{2}{3} (T_{xz} ( T_{xx}T_{zx} + T_{xy}T_{zy} + T_{xz}T_{zz} )) \qquad (i=x,j=z,k=x,y,z)\nn\\ 
%&&+ \frac{2}{3} (T_{yz} ( T_{yx}T_{zx} + T_{yy}T_{zy} + T_{yz}T_{zz} )) \qquad (i=y,j=z,k=x,y,z)\nn\\ 
\end{eqnarray}


\[
\begin{array}{ccccc}
i & j & k & T_{ij}T_{jk}T_{ki} & symm \\
\hline
x&x&x&  T_{xx}T_{xx}T_{xx}  & T_{xx}^3  \\
y&x&x&  T_{yx}T_{xx}T_{xy}  & T_{xx}T_{xy}^2 \\
z&x&x&  T_{zx}T_{xx}T_{xz}  & T_{xx}T_{xz}^2 \\
x&y&x&  T_{xy}T_{yx}T_{xx}  & T_{xx}T_{xy}^2 \\
y&y&x&  T_{yy}T_{yx}T_{xy}  & T_{yy}T_{xy}^2 \\
z&y&x&  T_{zy}T_{yx}T_{xz}  & T_{xy}T_{xz}T_{yz} \\
x&z&x&  T_{xz}T_{zx}T_{xx}  & T_{xx}T_{xz}^2 \\
y&z&x&  T_{yz}T_{zx}T_{xy}  & T_{xy}T_{xz}T_{yz} \\
z&z&x&  T_{zz}T_{zx}T_{xz}  & T_{zz}T_{xz}^2 \\
\hline
x&x&y&  T_{xx}T_{xy}T_{yx}  & T_{xx}T_{xy}^2 \\
y&x&y&  T_{yx}T_{xy}T_{yy}  & T_{yy}T_{xy}^2 \\
z&x&y&  T_{zx}T_{xy}T_{yz}  & T_{xy}T_{xz}T_{yz} \\
x&y&y&  T_{xy}T_{yy}T_{yx}  & T_{yy}T_{xy}^2 \\
y&y&y&  T_{yy}T_{yy}T_{yy}  & T_{yy}^3  \\
z&y&y&  T_{zy}T_{yy}T_{yz}  & T_{yy}T_{yz}^2 \\
x&z&y&  T_{xz}T_{zy}T_{yx}  & T_{xy}T_{xz}T_{yz} \\
y&z&y&  T_{yz}T_{zy}T_{yy}  & T_{yy}T_{yz}^2 \\
z&z&y&  T_{zz}T_{zy}T_{yz}  & T_{zz}T_{yz}^2 \\
\hline
x&x&z&  T_{xx}T_{xz}T_{zx}  & T_{xx}T_{xz}^2 \\
y&x&z&  T_{yx}T_{xz}T_{zy}  & T_{xy}T_{xz}T_{yz}\\
z&x&z&  T_{zx}T_{xz}T_{zz}  & T_{zz}T_{xz}^2\\
x&y&z&  T_{xy}T_{yz}T_{zx}  & T_{xy}T_{yz}T_{yz}\\
y&y&z&  T_{yy}T_{yz}T_{zy}  & T_{yy}T_{yz}^2\\
z&y&z&  T_{zy}T_{yz}T_{zz}  & T_{zz}T_{yz}^2\\
x&z&z&  T_{xz}T_{zz}T_{zx}  & T_{zz}T_{xz}^2\\
y&z&z&  T_{yz}T_{zz}T_{zy}  & T_{zz}T_{yz}^2\\
z&z&z&  T_{zz}T_{zz}T_{zz}  & T_{zz}^3  \\
\hline
\end{array}
\]
In the end 
\[
\sum_{i=x,y,z} \sum_{j=x,y,z} \sum_{k=x,y,z}
T_{ij}T_{jk}T_{ki}
= T_{xx}( T_{xx}^2 + 3T_{xy}^2 + 3T_{xz}^2)
+ T_{yy}(3T_{xy}^2 +  T_{yy}^2 + 3T_{yz}^2   )
+ T_{zz}(3T_{xz}^2 + 3T_{yz}^2 + T_{zz}^2  )
+6T_{xy}T_{yz}T_{yz}
\]
and then the third moment invariant of the symmetric tensor ${\bm T}$
is given by:
\begin{eqnarray}
{\cal I}_3({\bm T}) 
&=& \frac{1}{3} T_{xx} (  T_{xx}^2 + 3 T_{xy}^2 + 3 T_{xz}^2  )     \nonumber\\
&+& \frac{1}{3} T_{yy} (3 T_{xy}^2 +   T_{yy}^2 + 3 T_{yz}^2  )     \nonumber\\
&+& \frac{1}{3} T_{zz} (3 T_{xz}^2 + 3 T_{yz}^2 +   T_{zz}^2)       \nonumber\\
&+& 2 T_{xy} T_{xz} T_{yz} \\
&=& \frac{1}{3} (T_{xx}^3+T_{yy}^3+T_{zz}^3) 
+T_{xx} ( T_{xy}^2 +  T_{xz}^2  ) 
+T_{yy} ( T_{xy}^2 +  T_{yz}^2  ) 
+T_{zz} ( T_{xz}^2 +  T_{yz}^2  ) + 2 T_{xy} T_{xz} T_{yz} 
\end{eqnarray}




%----------------------------------------------------------------
\section{Stress \& strain rate invariants}\label{sec:stress_invariants}

\begin{flushright} {\tiny {\color{gray} stress\_sr\_invariants.tex}} \end{flushright}
%~~~~~~~~~~~~~~~~~~~~~~~~~~~~~~~~~~~~~~~~~~~~~~~~~~~~~~~~~~~~~~~~~~~~~~~~~~~~~~~~~~~~~~~~~~~~~~~~~~

The implementation of the plasticity criterions relies essentially 
on the (moment) invariants $\III_{1,2,3}$ of the (deviatoric) stress ${\bm \tau}$ 
and the (deviatoric) strainrate tensors $\dot{\bm \varepsilon}$:

\begin{eqnarray}
{\III}_1({\bm \sigma}) &=& \sigma_{xx}+\sigma_{yy}+\sigma_{zz}\\
{\III}_2({\bm \tau})   
&=&\frac{1}{2}(\tau_{xx}^2 + \tau_{yy}^2 + \tau_{zz}^2 ) + \tau_{xy}^2 + \tau_{xz}^2 + \tau_{yz}^2  \\
{\III}_3({\bm \tau}) 
&=& \frac{1}{3} \tau_{xx} (  \tau_{xx}^2 + 3 \tau_{xy}^2 + 3 \tau_{xz}^2  )     \nonumber\\
&+& \frac{1}{3} \tau_{yy} (3 \tau_{xy}^2 +   \tau_{yy}^2 + 3 \tau_{yz}^2  )     \nonumber\\
&+& \frac{1}{3} \tau_{zz} (3 \tau_{xz}^2 + 3 \tau_{yz}^2 +   \tau_{zz}^2)       \nonumber\\
&+& 2 \tau_{xy} \tau_{xz} \tau_{yz}  
\end{eqnarray}
and also the second invariant of the deviatoric strain rate is:
\begin{eqnarray}
{\III}_2(\dot{\bm{\varepsilon}}^d)
&=& \frac{1}{2} \left[ (\dot{\varepsilon}_{xx}^d)^2 + (\dot{\varepsilon}_{yy}^d)^2 + (\dot{\varepsilon}_{zz}^d)^2   \right] 
+ (\dot{\varepsilon}_{xy}^d)^2  
+ (\dot{\varepsilon}_{xz}^d)^2  
+ (\dot{\varepsilon}_{yz}^d)^2  \nonumber\\
&=& \frac{1}{6} \left[ (\dot{\varepsilon}_{xx}-\dot{\varepsilon}_{yy})^2 
+ (\dot{\varepsilon}_{yy}-\dot{\varepsilon}_{zz})^2 
+ (\dot{\varepsilon}_{xx}-\dot{\varepsilon}_{zz})^2 \right] 
+ \dot{\varepsilon}_{xy}^2 + \dot{\varepsilon}_{xz}^2 + \dot{\varepsilon}_{yz}^2 \label{eq:I2epsd} 
\end{eqnarray}
Proofs of these relationships are given in Appendix~\ref{app:invariants}.

We have 
\begin{eqnarray}
\tau_{xx}^2 + \tau_{yy}^2 + \tau_{zz}^2
&=& 
\left(\sigma_{xx}-\frac13 I_1\right)^2 + 
\left(\sigma_{yy}-\frac13 I_1\right)^2 + 
\left(\sigma_{zz}-\frac13 I_1\right)^2  \nonumber\\
&=&
\sigma_{xx}^2 + \sigma_{yy}^2 + \sigma_{zz}^2 
-\frac23 I_1 (\sigma_{xx} + \sigma_{yy} + \sigma_{zz}) 
+3\frac19 I_1^2 \nonumber\\
&=&
\sigma_{xx}^2 + \sigma_{yy}^2 + \sigma_{zz}^2 
-\frac23 I_1^2 +\frac13 I_1^2 \nonumber\\
&=&
\sigma_{xx}^2 + \sigma_{yy}^2 + \sigma_{zz}^2 
-\frac13 I_1^2  \nonumber\\
&=&
\sigma_{xx}^2 + \sigma_{yy}^2 + \sigma_{zz}^2 
-\frac13 (\sigma_{xx} + \sigma_{yy} + \sigma_{zz})^2 \nonumber\\
&=&
\sigma_{xx}^2 + \sigma_{yy}^2 + \sigma_{zz}^2 
-\frac13 (\sigma_{xx}^2 + \sigma_{yy}^2 + \sigma_{zz}^2
+2\sigma_{xx}\sigma_{yy}+2\sigma_{xx}\sigma_{zz}+2\sigma_{yy}\sigma_{zz} ) 
\nonumber\\
&=& \frac13 (
3\sigma_{xx}^2 + 3\sigma_{yy}^2 + 3\sigma_{zz}^2 
-\sigma_{xx}^2 - \sigma_{yy}^2 - \sigma_{zz}^2
-2\sigma_{xx}\sigma_{yy}-2\sigma_{xx}\sigma_{zz}-2\sigma_{yy}\sigma_{zz} ) 
\nonumber\\
&=& \frac13 (
2\sigma_{xx}^2 + 2\sigma_{yy}^2 + 2\sigma_{zz}^2 
-2 \sigma_{xx}\sigma_{yy}-2 \sigma_{xx}\sigma_{zz}-2 \sigma_{yy}\sigma_{zz} )\\
&=& \frac13 ((\sigma_{xx}-\sigma_{yy})^2 + (\sigma_{xx}-\sigma_{zz})^2
+ (\sigma_{yy}-\sigma_{zz})^2)
\end{eqnarray}
so that 
\[
{\III}_2({\bm \tau})   
=\frac{1}{6}\left[(\sigma_{xx}-\sigma_{yy})^2 + (\sigma_{yy}-\sigma_{zz})^2 + (\sigma_{xx}-\sigma_{zz})^2 \right]  
+ \sigma_{xy}^2 + \sigma_{xz}^2 + \sigma_{yz}^2 
\]

\begin{remark}
${\III}_2({\bm \tau})$ is often called $J_2$ or $J_2'$ so that one sometimes speaks of $J_2$-plasticity.
\end{remark}

These (second) invariants are almost always used under a square root so we define:
\begin{mdframed}[backgroundcolor=blue!5]
\begin{equation}
\tau_{e}=\sqrt{{\III}_2({\bm \tau})}
\quad\quad
\quad\quad
\dot{\varepsilon}_{e}=\sqrt{{\III}_2(\dot{\bm \varepsilon}^d)}
\label{eq:tauepse}
\end{equation}
\end{mdframed}
Note that these quantities have the same dimensions as their tensor counterparts, i.e. $\si{\pascal}$ 
for stresses and $\si{\per\second}$ for strain rates.

If the stress tensor is such that it is diagonal, i.e.
\[
{\bm \sigma}= \left( \begin{array}{ccc}
\sigma_1 & 0 & 0 \\
0 & \sigma_2 & 0 \\
0 & 0 & \sigma_3
\end{array}\right)
\qquad
{\rm and}
\qquad
{\bm \tau}= \left( \begin{array}{ccc}
\tau_1 & 0 & 0 \\
0 & \tau_2 & 0 \\
0 & 0 & \tau_3
\end{array}\right)
\]
then the invariants are 
\begin{eqnarray}
{\III}_1({\bm \sigma}) &=& \sigma_1 + \sigma_2+ \sigma_3 \nonumber\\
{\III}_2({\bm \tau}) &=& \frac{1}{6}\left[(\sigma_{1}-\sigma_{2})^2 + (\sigma_{2}-\sigma_{3})^2 
+ (\sigma_{1}-\sigma_{3})^2 \right] \label{eq:I2s123}\\ 
{\III}_3({\bm \tau}) 
&=& \tau_1\tau_2\tau_3 \nn\\
&=& \frac{1}{3} {\rm tr}[{\bm \tau}\cdot{\bm \tau}\cdot {\bm \tau}]  \nn\\
&=& \frac{1}{3} {\rm tr}
\left[
\left(
\begin{array}{ccc}
\tau_1 & 0 & 0 \\
0 & \tau_2 & 0 \\
0 & 0 & \tau_3 
\end{array}
\right)
\cdot
\left(
\begin{array}{ccc}
\tau_1 & 0 & 0 \\
0 & \tau_2 & 0 \\
0 & 0 & \tau_3 
\end{array}
\right)
\cdot
\left(
\begin{array}{ccc}
\tau_1 & 0 & 0 \\
0 & \tau_2 & 0 \\
0 & 0 & \tau_3 
\end{array}
\right)
\right] \nn\\
&=&  \frac{1}{3} {\rm tr}
\left(
\begin{array}{ccc}
\tau_1^3 & 0 & 0 \\
0 & \tau_2^3 & 0 \\
0 & 0 & \tau_3^3 
\end{array}
\right) \nn\\
&=& \frac{1}{3}(\tau_1^3+\tau_2^3+\tau_3^3) \nn\\
&=&  \frac{1}{3} [ 
(\sigma_1-{\III}_1({\bm \sigma})/3)^3+  
(\sigma_2-{\III}_1({\bm \sigma})/3)^3+
(\sigma_3-{\III}_1({\bm \sigma})/3)^3 ]   \nonumber\\ 
&=&  \frac{1}{3\cdot 27} [ 
(3\sigma_1-{\III}_1({\bm \sigma}))^3+  
(3\sigma_2-{\III}_1({\bm \sigma}))^3+
(3\sigma_3-{\III}_1({\bm \sigma}))^3 ]   \nonumber\\ 
&=& \frac{1}{81}
\left[
(2\sigma_1-\sigma_2-\sigma_3)^3+
(2\sigma_2-\sigma_1-\sigma_3)^3+
(2\sigma_3-\sigma_1-\sigma_2)^3
\right] 
\label{eq:3rdinvb} \label{eq:I3tau}
\end{eqnarray}
The formulation of the third invariant of ${\bm \tau}$  in Eq.~\ref{eq:I3tau} 
is used in Wojciechowski \cite{wojc18}.













%-----------------------------------------------------------
\index{general}{Plain Strain}
\section{Two-dimensional plane strain calculations \label{ss:plane_strain}} 
\begin{flushright} {\tiny {\color{gray} plane\_strain.tex}} \end{flushright}
%~~~~~~~~~~~~~~~~~~~~~~~~~~~~~~~~~~~~~~~~~~~~~~~~~~~~~~~~~~~~~~~~~~~~~~~~~~~~~~~~~~~~~~~~~~~~~~~~~~

We start from the 3D strain rate tensor 
\[
\dot{\bm \varepsilon}(\vec\upnu) = 
\left(
\begin{array}{ccc}
\dot{\varepsilon}_{xx} & \dot{\varepsilon}_{xy} & \dot{\varepsilon}_{xz} \\
\dot{\varepsilon}_{yx} & \dot{\varepsilon}_{yy} & \dot{\varepsilon}_{yz} \\
\dot{\varepsilon}_{zx} & \dot{\varepsilon}_{zy} & \dot{\varepsilon}_{zz} 
\end{array}
\right)
\]

The plane strain assumption is such that the problem at hand is assumed to be 
infinite in a given direction. In the case of computational geodynamics, most 2D 
modelling is a vertical section of the crust-lithosphere-mantle
and the underlying implicit assumption is then that the orogen/rift/subduction/etc ... 
is infinite in the direction perpendicular to the screen/paper.  

Let us assume that the deformation takes place in the $x,y$-plane,
so that $w=0$ (velocity in the $z$ direction is zero) and $\partial_z \rightarrow 0$ 
(no change in the $z$ direction).
We then have $\dot{\varepsilon}_{zz}=0$ as well as $\dot{\varepsilon}_{xz}=0$ 
and $\dot{\varepsilon}_{yz}=0$, so that the strain rate tensor is 
\[
\dot{\bm \varepsilon}(\vec\upnu)=
\left( \begin{array}{ccc}
\dot{\varepsilon}_{xx} & \dot{\varepsilon}_{xy} & 0 \\
\dot{\varepsilon}_{yx} & \dot{\varepsilon}_{yy} & 0 \\
0 & 0 & 0
\end{array}\right)
\]

%------------------------------------
\subsubsection{Incompressible flow}

If the flow is incompressible then the deviatoric stress tensor is given by
\[
\bm\tau 
= 2 \eta \dot{\bm \varepsilon}^d(\vec\upnu)
= 2 \eta \left(\dot{\bm \varepsilon}(\vec\upnu) 
-\frac13 \underbrace{{\rm tr}[\dot{\bm \varepsilon}]}_{=0} 
{\bm 1}\right)
= 2 \eta \dot{\bm \varepsilon}(\vec\upnu) 
=
\left(\begin{array}{ccc}
\tau_{xx} & \tau_{xy} & 0 \\
\tau_{yx} & \tau_{yy} & 0 \\
0 & 0 & 0
\end{array}\right)
\]
One then discards the unnecessary line and column in the tensor, leaving a $2\times 2$ matrix.
Finding the principal stress components is then trivial since we have done it in 2D already.

It is important to keep in mind that the invariants we need to implement 
the rheologies are ${\III}_1({\bm \sigma})$,  ${\III}_2({\bm \tau})$ and ${\III}_3({\bm \tau})$.
By formulating our yield surfaces with pressure $p=-{\III}_1({\bm \sigma})/3$ we can then 
avoid confusion, and since the other two invariants are functions of ${\bm \tau}$ the pressure 
term does not pose any problem: simply set $\tau_{xz}$, $\tau_{yz}$ and $\tau_{zz}$ to zero in the 
equations of Section~\ref{sec:stress_invariants} and we obtain:
\begin{eqnarray}
{\III}_2({\bm \tau}) &=&\frac{1}{2}(\tau_{xx}^2 + \tau_{yy}^2 ) + \tau_{xy}^2 \\ 
{\III}_3({\bm \tau}) 
&=& \frac{1}{3} \tau_{xx} (  \tau_{xx}^2 + 3 \tau_{xy}^2 ) 
+ \frac{1}{3} \tau_{yy} (3 \tau_{xy}^2 +   \tau_{yy}^2 )   \nn\\
&=& \frac{1}{3}(  \tau_{xx}^3 + 3 \tau_{xx}\tau_{xy}^2  
+ 3 \tau_{yy} \tau_{xy}^2 +   \tau_{yy}^3 )   \nn\\
&=& \frac{1}{3}(  \tau_{xx}^3 + 3 (\tau_{xx}+\tau_{yy}) \tau_{xy}^2  +  \tau_{yy}^3 )   \nn\\
&=& \frac{1}{3}(  \tau_{xx}^3 +  \tau_{yy}^3 )  \qquad \text{since } \tau_{ii}=0 
\end{eqnarray}



The principal stresses of the deviatoric stress tensor $\bm\tau$ are given by
\begin{eqnarray}
\tau_1 &=& \frac{ \tau_{xx}+\tau_{yy}}{2} 
+ \sqrt{ \left(\frac{\tau_{xx}-\tau_{yy}}{2}\right)^2 +\tau_{xy}^2 } \nn\\
\tau_2 &=& \frac{ \tau_{xx}+\tau_{yy}}{2} 
- \sqrt{ \left(\frac{\tau_{xx}-\tau_{yy}}{2}\right)^2 +\tau_{xy}^2 } 
\end{eqnarray}
The full stress tensor is then
\[
\bm\sigma = -p \bm 1 + \bm\tau
= \left(\begin{array}{ccc}
-p+\tau_{xx} & \tau_{xy} & 0 \\
\tau_{yx} & -p+\tau_{yy} & 0 \\
0 & 0 & -p
\end{array}\right)
\]
so it remains a $3\times 3$ tensor!

However, looking at the conservation of momentum, 
\[
\vec\nabla \cdot \bm\sigma + \rho \vec g = \vec 0
\]
Given the conditions for plane-strain then $\vec g$ is likely to be in 
the $xy$-plane so that the $z$ component of the equation becomes:
\[
-\partial_z p = 0
\]
and since we have $\partial_z \rightarrow 0$ anyways this equation 
is automatically fulfilled. Then, we might as well proceed 
by considering that the stress tensor is in fact 2D as the third row/column
has no incidence. In that case the pressure is given by $p=-{\III}_1(\bm\sigma)/2$.
In the plasticity yield criterion or plastic potential we will 
need the full stress ${\bm \sigma}$ only via its first invariant (i.e. the pressure). 
The other two invariants are those of the deviatoric stress. 

Let us start from the deviatoric stress tensor:
\[
\bm\tau
=
\bm\sigma - \frac12 {\III}_1(\bm\sigma)
=
\left(\begin{array}{cc}
\sigma_{xx} & \sigma_{xy} \\ 
\sigma_{xy} & \sigma_{yy} 
\end{array}\right)
-\frac{\sigma_{xx}+\sigma_{yy}}{2} 
\left(\begin{array}{cc}
1 & 0 \\ 0 & 1
\end{array}\right)
=
\left(
\begin{array}{cc}
(\sigma_{xx}-\sigma_{yy})/2 & \sigma_{xy} \\
\sigma_{xy} & -(\sigma_{xx}-\sigma_{yy})/2
\end{array}
\right)
\]
The second invariant of the deviatoric stress tensor is then 
\[
{\III}_2(\bm\tau) = \frac12 \bm\tau:\bm\tau
= \frac12 \left( 2 (\sigma_{xx}-\sigma_{yy})^2/4 + 2 \sigma_{xy}^2 \right)
%= \left(\frac{\sigma_{xx}-\sigma_{yy}}{2} \right)^2 + \sigma_{xy}^2
\]
or
\begin{mdframed}[backgroundcolor=blue!5]
\[
{\III}_2(\bm\tau) 
= \left(\frac{\sigma_{xx}-\sigma_{yy}}{2} \right)^2 + \sigma_{xy}^2
\]
\end{mdframed}


and the effective deviatoric stress
\[
\tau_e = 
\sqrt{\left(\frac{\sigma_{xx}-\sigma_{yy}}{2} \right)^2 + \sigma_{xy}^2}
\]


Remark: Using the form of ${\III}_2(\bm\tau)$ above one arrives at  
\begin{eqnarray}
\frac{\partial {\III}_2(\bm\tau)}{\partial \sigma_{xx}} 
&=&  2 \frac{1}{2}  \frac{\sigma_{xx}-\sigma_{yy}}{2} = \tau_{xx}\nn \\
\frac{\partial {\III}_2(\bm\tau)}{\partial \sigma_{yy}} 
&=& -2 \frac{1}{2}  \frac{\sigma_{xx}-\sigma_{yy}}{2} = \tau_{yy} \nn\\
\frac{\partial {\III}_2(\bm\tau)}{\partial \sigma_{xy}} 
&=& 2 \sigma_{xy} =  2 \tau_{xy}\nn
\end{eqnarray}
which is ...wrong! One should first write the second invariant 
for the generic case of the deviatoric stress tensor (without 
assuming it is symmetric):
\[
{\III}_2(\bm\tau) = \frac12 \bm\tau:\bm\tau
= \frac12 \left( 2 (\sigma_{xx}-\sigma_{yy})^2/4 + \sigma_{xy}^2 + \sigma_{yx}^2 \right)
= \left(\frac{\sigma_{xx}-\sigma_{yy}}{2} \right)^2 + \frac12\sigma_{xy}^2 + \frac12\sigma_{yx}^2
\]
Then
\begin{eqnarray}
\frac{\partial {\III}_2(\bm\tau)}{\partial \sigma_{xx}} 
&=&  2 \frac{1}{2}  \frac{\sigma_{xx}-\sigma_{yy}}{2} = \tau_{xx} \nn\\
\frac{\partial {\III}_2(\bm\tau)}{\partial \sigma_{yy}} 
&=& -2 \frac{1}{2}  \frac{\sigma_{xx}-\sigma_{yy}}{2} =  \tau_{yy} \nn\\
\frac{\partial {\III}_2(\bm\tau)}{\partial \sigma_{xy}} 
&=&  \sigma_{xy} =  \tau_{xy} \nn\\
\frac{\partial {\III}_2(\bm\tau)}{\partial \sigma_{yx}} 
&=&  \sigma_{yx} =  \tau_{yx} \nn
\end{eqnarray}
which can be simply written as
\begin{mdframed}[backgroundcolor=blue!5]
\[
\frac{\partial {\III}_2(\bm\tau)}{\partial \bm\sigma} 
=\bm\tau
\]
\end{mdframed}



















%------------------------------------
\subsubsection{Compressible flow}
If the flow is not incompressible, then the deviatoric strain rate tensor is
\[
\dot{\bm \varepsilon}^d(\vec\upnu) 
= \dot{\bm \varepsilon}(\vec\upnu) -\frac{1}{3} {\rm tr}[\dot{\bm \varepsilon}]   {\bm 1} 
= \dot{\bm \varepsilon}(\vec\upnu) -\frac{1}{3} (\dot{\varepsilon}_{xx} +\dot{\varepsilon}_{yy}   )  {\bm 1} 
=
\left(
\begin{array}{ccc}
\frac{2}{3}\dot{\varepsilon}_{xx} -\frac{1}{3}\dot{\varepsilon}_{yy} & \dot{\varepsilon}_{xy} & 0 \\
\dot{\varepsilon}_{yx} & -\frac{1}{3}\dot{\varepsilon}_{xx} +\frac{2}{3} \dot{\varepsilon}_{yy} & 0 \\
0 & 0 & -\frac{1}{3} \dot{\varepsilon}_{xx} -\frac{1}{3}\dot{\varepsilon}_{yy}
\end{array}
\right)
\]
The deviatoric stress tensor now has the form
\[
\bm\tau=
\left(\begin{array}{ccc}
\tau_{xx} & \tau_{xy} & 0 \\
\tau_{yx} & \tau_{yy} & 0 \\
0 & 0 & \tau_{zz}
\end{array}\right)
\]

We are interested in the principal components
of the deviatoric stress tensor $\bm \tau$ so that we now have the following determinant to compute:
\[
\left|  
\begin{array}{ccc}
\tau_{xx}-\lambda & \tau_{xy} & 0 \\
\tau_{xy} & \tau_{yy}-\lambda & 0 \\
0 &0 & \tau_{zz} -\lambda
\end{array}
\right|
=0
\]
which yields the following characteristic equation:
\[
(\tau_{zz} -\lambda)(\lambda-\tau_1)(\lambda-\tau_2) =0
\]
where $\tau_{1,2}$ have previously been obtained in the 2D case:
\begin{eqnarray}
\tau_1 
&=& \frac{ \tau_{xx}+\tau_{yy}}{2} 
+ \sqrt{ \left(\frac{\tau_{xx}-\tau_{yy}}{2}\right)^2 +\tau_{xy}^2 } 
\nn\\
\tau_2 &=& \frac{ \tau_{xx}+\tau_{yy}}{2} 
- \sqrt{ \left(\frac{\tau_{xx}-\tau_{yy}}{2}\right)^2 +\tau_{xy}^2 } 
\end{eqnarray}
We have 
\begin{eqnarray}
\tau_{xx}+\tau_{yy} &=& 2\eta \frac13 (\dot{\varepsilon}_{xx} + \dot{\varepsilon}_{yy}) \nn\\
\tau_{xx}-\tau_{yy} &=& 2\eta (\dot{\varepsilon}_{xx} - \dot{\varepsilon}_{yy})
\end{eqnarray}
Then 
\begin{eqnarray}
\tau_1 
&=& \frac{ \tau_{xx}+\tau_{yy}}{2} 
+ \sqrt{ \left(\frac{\tau_{xx}-\tau_{yy}}{2}\right)^2 +\tau_{xy}^2 } 
\nn\\
&=& \eta \frac13 (\dot{\varepsilon}_{xx} + \dot{\varepsilon}_{yy})
+ \eta \sqrt{ (\dot{\varepsilon}_{xx} - \dot{\varepsilon}_{yy})^2 +4  \dot{\varepsilon}_{xy}^2  } 
\nn\\
\tau_2 
&=& \frac{ \tau_{xx}+\tau_{yy}}{2} 
- \sqrt{ \left(\frac{\tau_{xx}-\tau_{yy}}{2}\right)^2 +\tau_{xy}^2 } \nn\\
&=& \eta \frac13 (\dot{\varepsilon}_{xx} + \dot{\varepsilon}_{yy})
- \eta \sqrt{ (\dot{\varepsilon}_{xx} - \dot{\varepsilon}_{yy})^2 +4  \dot{\varepsilon}_{xy}^2  } 
\end{eqnarray}
It does not look like it is going to simplify down the road ... Also, 
the third eigenvalue/principal stress remains and it is not clear whether it is 
larger or smaller than the other two.
The 3D framework is then probably the most appropriate.











Let us now turn to the second invariant of the deviatoric strain rate 
(see Eq.~\eqref{eq:I2epsd}):
\begin{eqnarray}
{\III}_2(\dot{\bm{\varepsilon}}^d)
&=& \frac{1}{2} \dot{\bm{\varepsilon}}^d:\dot{\bm{\varepsilon}}^d \\
&=& \frac{1}{2} \left[ (\dot{\varepsilon}_{xx}^d)^2 + (\dot{\varepsilon}_{yy}^d)^2 + (\dot{\varepsilon}_{zz}^d)^2   \right] 
+ (\dot{\varepsilon}_{xy}^d)^2  
+ (\dot{\varepsilon}_{xz}^d)^2  
+ (\dot{\varepsilon}_{yz}^d)^2  
\end{eqnarray}
But there is also an expression for ${\III}_2(\dot{\bm{\varepsilon}}^d)$ directly as a function of the $\dot{\bm\varepsilon}_{ij}$ components 
(see Eq.~\eqref{eq:I2epsd}):
\begin{eqnarray}
{\III}_2(\dot{\bm{\varepsilon}}^d)
&=& \frac{1}{6} \left[ (\dot{\varepsilon}_{xx}-\dot{\varepsilon}_{yy})^2 
+ (\dot{\varepsilon}_{yy}-\dot{\varepsilon}_{zz})^2 
+ (\dot{\varepsilon}_{xx}-\dot{\varepsilon}_{zz})^2 \right] 
+ \dot{\varepsilon}_{xy}^2 + \dot{\varepsilon}_{xz}^2 + \dot{\varepsilon}_{yz}^2 \\
&=& 
\frac{1}{6} \left[ (\dot{\varepsilon}_{xx}-\dot{\varepsilon}_{yy})^2 
+ (\dot{\varepsilon}_{yy})^2 
+ (\dot{\varepsilon}_{xx})^2 \right] 
+ \dot{\varepsilon}_{xy}^2 \\
&=& \frac{1}{6} \left[ \dot{\varepsilon}_{xx}^2 
-2\dot{\varepsilon}_{xx}\dot{\varepsilon}_{yy}
+\dot{\varepsilon}_{yy}^2 
+ \dot{\varepsilon}_{yy}^2 
+ \dot{\varepsilon}_{xx}^2 \right] 
+ \dot{\varepsilon}_{xy}^2 \\
&=& \frac{1}{6} \left[ 
2\dot{\varepsilon}_{xx}^2 
-2\dot{\varepsilon}_{xx}\dot{\varepsilon}_{yy}
+2\dot{\varepsilon}_{yy}^2 
\right] 
+ \dot{\varepsilon}_{xy}^2 \\
&=& 
\frac{1}{3} \left[ 
\dot{\varepsilon}_{xx}^2 
-\dot{\varepsilon}_{xx}\dot{\varepsilon}_{yy}
+\dot{\varepsilon}_{yy}^2 
\right] 
+ \dot{\varepsilon}_{xy}^2 
\end{eqnarray}

{\color{darkgray}
If we now do things the old/wrong(?) way, one would start directly from the 2D strain rate tensor 
\[
\dot{\bm \varepsilon} = 
\left(
\begin{array}{cc}
\dot{\varepsilon}_{xx} & \dot{\varepsilon}_{xy}  \\
\dot{\varepsilon}_{yx} & \dot{\varepsilon}_{yy}  
\end{array}
\right)
\]
The deviatoric strain rate tensor is then logically defined as 
\[
\dot{\bm \varepsilon}^d 
= \dot{\bm \varepsilon} -\frac{1}{2} Tr[\dot{\bm \varepsilon}]   {\bm 1} 
= \dot{\bm \varepsilon} -\frac{1}{2} (\dot{\varepsilon}_{xx} +\dot{\varepsilon}_{yy}   )  {\bm 1} 
\]
or,
\[
\dot{\bm \varepsilon}^d = 
\left(
\begin{array}{cc}
\frac{1}{2}\dot{\varepsilon}_{xx} -\frac{1}{2}\dot{\varepsilon}_{yy} & \dot{\varepsilon}_{xy}  \\
\dot{\varepsilon}_{yx} & -\frac{1}{2}\dot{\varepsilon}_{xx} +\frac{1}{2} \dot{\varepsilon}_{yy}  \\
\end{array}
\right)
\]
Let us now turn to the second invariant of the deviatoric strain rate 
(see Section 3.21 in fieldstone)
\begin{eqnarray}
{\III}_2(\dot{\bm{\varepsilon}}^d)
&=& \frac{1}{2} \dot{\bm{\varepsilon}}^d:\dot{\bm{\varepsilon}}^d \nn\\
&=& \frac{1}{2}[ (\frac{1}{2}\dot{\varepsilon}_{xx} -\frac{1}{2}\dot{\varepsilon}_{yy})^2 + (-\frac{1}{2}\dot{\varepsilon}_{xx} +\frac{1}{2} \dot{\varepsilon}_{yy})^2  ] + \dot{\varepsilon}_{xy}^2 \nn\\
&=& \frac12 [ \frac14 (2\dot{\varepsilon}_{xx}^2  -4 \dot{\varepsilon}_{xx}\dot{\varepsilon}_{yy} +2\dot{\varepsilon}_{yy}^2 )  ] + \dot{\varepsilon}_{xy}^2 \nn\\
&=& \frac14 [ \dot{\varepsilon}_{xx}^2  -2 \dot{\varepsilon}_{xx}\dot{\varepsilon}_{yy} +\dot{\varepsilon}_{yy}^2   ] + \dot{\varepsilon}_{xy}^2 
\end{eqnarray}
which is not the same as the previous expression! 
}













%%%%%%%%%%%%%%%%%%%%%%%%%%%%%%%%%%%%%%%%%%%%%%%%%%%%%%%%%%%%%%%%%%%%%%%%%%%%%%%%%%%%%%%%%%%%%%%%%%%%
\section{Alternative principal stresses notations}\label{sec:altinv}
\begin{flushright} {\tiny {\color{gray} physics.tex}} \end{flushright}

The principal stresses of the stress tensor ${\bm \sigma}$ are $\sigma_1$, $\sigma_2$
and $\sigma_3$ with $\sigma_1 \geq \sigma_2 \geq \sigma_3$.
Following Wojciechowski \cite{wojc18}, we start by stating that the intermediate principal 
stress can always be represented as a linear combination of two other stresses:
\begin{equation}
\sigma_2 = (1-b)\sigma_1 + b \sigma_3
\qquad
{\rm where}
\qquad
b = \frac{\sigma_1-\sigma_2}{\sigma_1-\sigma_3}\in [0,1]
\end{equation}
The quantity $b$ is called the principal stress ratio. \index{general}{Principal Stress Ratio}
Let us now introduce the maximum shear plane stresses $\sigma_m$ and $\tau_m$ such that
\footnote{Although most of this section is inspired by Wojciechowski \cite{wojc18}, 
I have decided not to use his notations which are very confusing since he denotes $\sigma_m$ by $p$} 
\begin{equation}
\boxed{\sigma_m=\frac{\sigma_1+\sigma_3}{2}}
\qquad
\boxed{\tau_m=\frac{\sigma_1-\sigma_3}{2}}
\end{equation}
so that we have 
\begin{eqnarray}
\sigma_1 &=& \sigma_m+\tau_m \\
\sigma_2 &=& \sigma_m-a\tau_m \\ 
\sigma_3 &=& \sigma_m-\tau_m
\end{eqnarray}
The quantity $a\in[-1,1]$ is an equivalent measure of the principal stress ratio and 
is defined as 
\begin{equation}
a=2b-1 =2 \frac{\sigma_1-\sigma_2}{\sigma_1-\sigma_3}-1=\frac{\sigma_1-2\sigma_2+\sigma_3}{\sigma_1-\sigma_3}
\end{equation}
We can introduce $a$, $\sigma_m$ and $\tau_m$ in the invariants above:
\begin{eqnarray}
{\cal I}_1({\bm \sigma}) 
&=& \sigma_1 + \sigma_2 + \sigma_3 \nn\\
&=& (\sigma_m+\tau_m) + (\sigma_m-a\tau_m) + (\sigma_m-\tau_m) \nn\\
&=& 3\sigma_m -a\tau_m \\
{\cal I}_2({\bm \tau}) 
&=&\frac{1}{6}\left[(\sigma_{1}-\sigma_{2})^2 +(\sigma_{2}-\sigma_{3})^2 +(\sigma_{1}-\sigma_{3})^2\right]\nn\\ 
&=&\frac{1}{6}\left[(\sigma_m+\tau_m-\sigma_m+a\tau_m)^2 +(\sigma_m-a\tau_m-\sigma_m+\tau_m)^2 
+(\sigma_m+\tau_m-\sigma_m+\tau_m)^2\right]\nn\\ 
&=&\frac{1}{6}\left[(\tau_m+a\tau_m)^2 +(-a\tau_m+\tau_m)^2 +(\tau_m+\tau_m)^2\right]\nn\\ 
&=&\frac{\tau_m^2}{6}\left[(1+a)^2 +(-a+1)^2 + 4 \right]\nn\\ 
&=&\frac{\tau_m^2}{6}\left[ 1+2a+a^2 +1 - 2a+a^2 + 4 \right]\nn\\ 
&=&\frac{\tau_m^2}{3}\left( a^2 +3 \right)
\end{eqnarray}
Using the definition of the third invariant of Eq.~(\ref{eq:3rdinvb}):
\begin{eqnarray}
{\cal I}_3({\bm \tau}) 
&=& \frac{1}{81} \left[
(2\sigma_1-\sigma_2-\sigma_3)^3+
(2\sigma_2-\sigma_1-\sigma_3)^3+
(2\sigma_3-\sigma_1-\sigma_2)^3
\right] \nn\\
&=& \frac{1}{81} \left[
(2\sigma_m+2\tau_m-\sigma_m+a\tau_m-\sigma_m+\tau_m)^3+
(2\sigma_m-2a\tau_m-\sigma_m-\tau_m-\sigma_m+\tau_m)^3+
(2\sigma_m-2\tau_m-\sigma_m-\tau_m-\sigma_m+a\tau_m)^3
\right] \nn\\
&=& \frac{1}{81} \left[ (2\tau_m+a\tau_m+\tau_m)^3+ (-2a\tau_m-\tau_m+\tau_m)^3+ (-2\tau_m-\tau_m+a\tau_m)^3 \right] \nn\\
&=& \frac{\tau_m^3}{81} \left[ (3+a)^3+ (-2a)^3+ (-3+a)^3 \right] \nn\\
&=& \frac{\tau_m^3}{81} \left[ 27 +9a + 3a^2 + a^3  -8a^3 -27 +9a -3a^2 + a^3 \right] \nn\\
&=& \frac{\tau_m^3}{81} \left( 18a  -6 a^3  \right) \nn\\
&=& \frac{2a \tau_m^3}{27} \left( 3 - a^2  \right) 
\end{eqnarray}
which is different than Eq. (14) of  Wojciechowski \cite{wojc18}!!

To recap:
\begin{eqnarray}
\boxed{{\cal I}_1({\bm \sigma}) =  3\sigma_m -a\tau_m } 
\qquad
\boxed{{\cal I}_2({\bm \tau}) =\frac{\tau_m^2}{3}\left( a^2 +3 \right)}
\qquad
\boxed{{\cal I}_3({\bm \tau}) = \frac{2a \tau_m^3}{27} \left( 3 - a^2  \right) }
\end{eqnarray}

\begin{remark}
Wojciechowski \cite{wojc18} defines the Lode angle \index{general}{Lode Angle} 
as being the opposite of my definition in Eq.~\ref{eq:lodang}.
\end{remark}

Finally, we can show using Eqs.~(\ref{eq:sigma1},\ref{eq:sigma2},\ref{eq:sigma3}) that
\begin{eqnarray}
a 
&=&\frac{\sigma_1-2\sigma_2+\sigma_3}{\sigma_1-\sigma_3} \nn\\
&=& 
\frac{
\sqrt{{\cal I}_2({\bm \tau})} \left(-\frac{1}{\sqrt{3}}\sin \theta +\cos\theta \right) 
-2
\sqrt{{\cal I}_2({\bm \tau})} \left(\frac{2}{\sqrt{3}}\sin \theta   \right)   
+
\sqrt{{\cal I}_2({\bm \tau})} \left(-\frac{1}{\sqrt{3}}\sin \theta- \cos \theta \right)  
}{
\sqrt{{\cal I}_2({\bm \tau})} \left(-\frac{1}{\sqrt{3}}\sin \theta +\cos\theta \right)
- 
\sqrt{{\cal I}_2({\bm \tau})} \left(-\frac{1}{\sqrt{3}}\sin \theta- \cos \theta \right)  
}
\nn \\
&=& 
\frac{
\left(-\frac{1}{\sqrt{3}}\sin \theta +\cos\theta \right) 
-2
\left(\frac{2}{\sqrt{3}}\sin \theta   \right)   
+
\left(-\frac{1}{\sqrt{3}}\sin \theta- \cos \theta \right)  
}{
\left(-\frac{1}{\sqrt{3}}\sin \theta +\cos\theta \right)
- 
\left(-\frac{1}{\sqrt{3}}\sin \theta- \cos \theta \right)  
}
\nn \\
&=& 
\frac{
-\frac{6}{\sqrt{3}}\sin \theta  
}
{
2\cos\theta
}
\nn\\
&=& -\frac{3}{\sqrt{3}} \frac{\sin\theta}{\cos\theta} \nn\\
&=& -\sqrt{3} \tan\theta
\end{eqnarray}
Here again we arrive at the opposite of Eq. (16) of Wojciechowski \cite{wojc18}. 

\newpage
%====================================================================================
%====================================================================================

\section{Recap of notations and definitions of stress invariants \label{ss:recapInv}}
\begin{flushright} {\tiny {\color{gray} recap\_invariants.tex}} \end{flushright}
%~~~~~~~~~~~~~~~~~~~~~~~~~~~~~~~~~~~~~~~~~~~~~~~~~~~~~~~~~~~~~~~~~~~~~~~~~~~~~~~~~~~~~~~~~~~~~~~~~~

When it comes to stress invariants, I urge the reader to be extremely careful when considering 
their source(s). As we have seen these come in two many flavours (e.g. principal and moment invariants)
and they are often written for the full stress or deviatoric tensor. On top of it all, 
typos are common like in any source and the occasional minus sign or factor 2 or 3 can be missing.
This is the reason why I have spent substantial time re-deriving those in the past pages 
with a consistent set of notations:
\begin{center}
\begin{tabular}{ll}
\hline
${\bm \sigma}$ & (full) stress tensor \\
$\sigma_1$, $\sigma_2$, $\sigma_3$ & principal stresses \\ 
${\bm \tau}$   & deviatoric stress tensor \\
$\tau_1$, $\tau_2$, $\tau_3$ & principal deviatoric stresses \\ 
${\cal I}_1({\bm T})$ & first moment invariant of tensor ${\bm T}$ \\
${\cal I}_2({\bm T})$ & second moment invariant of tensor ${\bm T}$ \\
${\cal I}_3({\bm T})$ & third moment invariant of tensor ${\bm T}$ \\
${\tau}_{e}=\sqrt{{\cal I}_2({\bm \tau})}$ & effective deviatoric stress \\
$\dot{{\varepsilon}}_{e}=\sqrt{{\cal I}_2(\dot{\bm \varepsilon}^d)}$ & effective deviatoric strain rate \\
\hline
\end{tabular}
\end{center}
Proofs of all the following relationships are given in Appendix~\ref{app:invariants}.

\begin{itemize}
\item First invariant %-----------------------------------

\begin{eqnarray}
{\cal I}_1(\bm\sigma) &=& \sigma_{xx}+\sigma_{yy}+\sigma_{zz} \nn\\
{\cal I}_1(\bm\tau) &=& 0 \nn\\ 
\frac{\partial {\cal I}_1(\bm\sigma)}{\partial \bm\sigma} &=& {\bm 1}  \nn
\end{eqnarray}


\item Second invariant %-----------------------------------


\begin{eqnarray}
{\cal I}_2(\bm\tau) 
&=& \frac12 {\bm \tau}:{\bm \tau} \nn\\
&=& \frac12 {\rm tr} [{\bm \tau}\cdot {\bm \tau}] \nn\\
&=& \frac12 \sum_{ij} \tau_{ij}\tau_{ji}  \nn\\
&=& \frac12 ( \tau_{xx}^2 + \tau_{yy}^2 +\tau_{zz}^2 + 2\tau_{xy}^2+ 2\tau_{xz}^2+ 2\tau_{yz}^2) \nn\\
&=& \frac{1}{6}\left[(\sigma_{xx}-\sigma_{yy})^2 + (\sigma_{yy}-\sigma_{zz})^2 
+ (\sigma_{xx}-\sigma_{zz})^2 \right]  + \sigma_{xy}^2 + \sigma_{xz}^2 + \sigma_{yz}^2 \nn\\
&=&  -\frac16 {\cal I}_1(\bm\sigma)^2 + {\cal I}_2(\bm\sigma) \nn\\
\frac{\partial {\cal I}_2(\bm\tau)}{\partial \bm\sigma} &=& \bm\tau   \nn
\end{eqnarray}

\item Third invariant %-----------------------------------

\begin{eqnarray}
{\cal I}_3(\bm\tau) 
&=& \frac13 \sum_{ijk} \tau_{ij}\tau_{jk}\tau_{ki} \nn\\
&=& {\rm det}(\bm\tau) \nn\\
&=& \frac13 {\rm tr} [{\bm \tau}\cdot {\bm \tau} \cdot {\bm \tau}] \nn\\
&=& \frac{2}{27}  {\cal I}_1(\bm\sigma)^3 - \frac23{\cal I}_1(\bm\sigma) {\cal I}_2(\bm\sigma)
+{\cal I}_3(\bm\sigma) \nn\\
\frac{\partial {\cal I}_3(\bm\tau)}{\partial \bm\sigma} &=&
\end{eqnarray}

\begin{equation}
\theta_{\rm L}=\frac{1}{3} \sin^{-1} 
\left( -\frac{3\sqrt{3}}{2} \frac{{\cal I}_3({\bm \tau})}{{\cal I}_2({\bm \tau})^{3/2}} \right)
\end{equation}


\end{itemize}


\begin{eqnarray}
\frac{\partial {\cal I}_1(\bm\sigma)}{\partial \bm\sigma} &=& {\bm 1} \\
\frac{\partial {\cal I}_2(\bm\sigma)}{\partial \bm\sigma} &=& {\bm \sigma} \\
\frac{\partial {\cal I}_3(\bm\sigma)}{\partial \bm\sigma} &=& \bm\sigma \cdot \bm\sigma
\end{eqnarray}


















































