\subsection{Strain rate and spin tensor} \label{ss:srst}
\index{general}{Velocity Gradient}
\index{general}{Strain Rate}
\index{general}{Spin Tensor}

The velocity gradient ${\bm L}$ is given in Cartesian coordinates by:
\begin{equation}
{\bm L}(\vec\upnu)=
\vec\nabla\vec\upnu = 
\left(
\begin{array}{ccc}
\frac{\partial u}{\partial x} & \frac{\partial v}{\partial x} & \frac{\partial w}{\partial x} \\\\
\frac{\partial u}{\partial y} & \frac{\partial v}{\partial y} & \frac{\partial w}{\partial y} \\\\
\frac{\partial u}{\partial z} & \frac{\partial v}{\partial z} & \frac{\partial w}{\partial z} 
\end{array}
\right)
\end{equation}
It can be decomposed into its symmetric and skew-symmetric parts according to:
\begin{equation}
\vec\nabla\vec\upnu = (\vec\nabla\vec\upnu)^s + (\vec\nabla\vec\upnu)^w 
= \dot{\bm \varepsilon}(\vec \upnu) +  \dot{\bm \omega}(\vec \upnu)
\end{equation}
The symmetric part is called the strain rate (or rate of deformation)\footnote{Note that often the dot is omitted and for example the \aspect manual uses the ${\bm \varepsilon}$ notation.}:
\begin{equation}
\dot{\bm \varepsilon}(\vec \upnu) = \frac{1}{2}\left( \vec\nabla\vec\upnu + (\vec\nabla\vec\upnu)^T \right)
\end{equation}
The skew-symmetric tensor is called spin tensor (or vorticity tensor):
\begin{equation}
\dot{\bm \omega}(\vec \upnu) = \frac{1}{2}\left( \vec\nabla\vec\upnu - (\vec\nabla\vec\upnu)^T \right)
\end{equation}

\begin{remark}
In the mathematical literature a different notation for the strain rate tensor is often used, i.e. 
${\bm D}(\vec \upnu)$ - or simply ${\bm D}$, such as for instance in Fullsack (1995) \cite{full95}.
\end{remark}

%.............................................
\section{Viscous Newtonian rheology}
\begin{flushright} {\tiny {\color{gray} physics.tex}} \end{flushright}

The relationship between velocity-related stresses and
velocity derivatives is such that the total stress tensor has the form \cite{berc09}
\begin{equation}
{\bm \sigma} = -p {\bm 1} + {\bm A}:\dot{\bm \varepsilon}(\vec\upnu)
\end{equation}
where $p$ is the thermodynamic pressure which is a function of the density $\rho$ and the temperature $T$ (an equation of state is then needed)
and ${\bm A}$ is the fourth-rank stiffness tensor.

Since both the stress and the strain tensors are symmetric and for isotropic 
fluids we have (see Malvern \cite{malvern})
\begin{equation}
{\bm A}:\dot{\bm \varepsilon}(\vec\upnu) 
= \lambda (\vec\nabla\cdot\vec\upnu) {\bm 1} + 2\eta \dot{\bm \varepsilon}(\vec\upnu)
\end{equation}
where $\lambda$ is the bulk viscosity and $\eta$ is the dynamic viscosity\footnote{also sometimes called shear viscosity}. 
The stress tensor is then 
\begin{equation}
{\bm \sigma} = (-p  + \lambda (\vec\nabla\cdot\vec\upnu)) {\bm 1} + 2\eta \dot{\bm \varepsilon}(\vec\upnu)
\end{equation}
By writing 
\[
\dot{\bm \varepsilon}(\vec\upnu) 
= \frac{1}{3}{\rm tr}(\dot{\bm \varepsilon}(\vec\upnu)) {\bm 1} + \dot{\bm \varepsilon}^d(\vec\upnu) =
 \frac{1}{3}(\vec\nabla\cdot\vec\upnu) {\bm 1} + \dot{\bm \varepsilon}^d (\vec\upnu)
\]
where $\dot{\bm \varepsilon}^d(\vec\upnu)$ is the deviatoric strain rate tensor and 
(in Cartesian coordinates)
\begin{equation}
\vec\nabla\cdot\vec\upnu = 
\text{div} (\vec\upnu ) =
{\rm tr}(\dot{\bm \varepsilon}(\vec\upnu)) =
\frac{\partial u}{\partial x}+ 
\frac{\partial v}{\partial y}+ 
\frac{\partial w}{\partial z} 
\end{equation} 
where ${\rm tr}$ is the trace operator, we arrive at
\begin{eqnarray}
{\bm \sigma} 
&=& (-p+\lambda(\vec\nabla\cdot\vec\upnu)) {\bm 1} + 2\eta \left[ \frac{1}{3}(\vec\nabla\cdot\vec\upnu) {\bm 1} + \dot{\bm \varepsilon}^d \right] \\
&=& \left[ -p+\left(\lambda+\frac{2}{3}\eta\right)(\vec\nabla\cdot\vec\upnu)\right] {\bm 1} + 2\eta  \dot{\bm \varepsilon}^d  
\end{eqnarray}
Introducing the second viscosity $\zeta=\lambda+\frac{2}{3}\eta$:
\begin{eqnarray}
{\bm \sigma} 
&=& \left[ -p+ \zeta (\vec\nabla\cdot\vec\upnu)\right] {\bm 1} + 2\eta  \dot{\bm \varepsilon}^d \\ 
&=&  -\overline{p} {\bm 1} + 2\eta  \dot{\bm \varepsilon}^d  
\end{eqnarray}
The effect of the volume viscosity $\zeta$ is that the mechanical pressure $\overline{p}$
is not equivalent to the thermodynamic pressure $p$ 
\begin{equation}
\overline{p}=p - \zeta (\vec\nabla\cdot\vec\upnu)
\end{equation}
In other words: the isotropic average of the total stress is {\sl not} the pressure term!
This difference is usually neglected (and it is safe to do so, see \cite[section 7.02.3.2.2]{berc09}) 
by explicitly assuming $\zeta=0$ (also called the Stokes assumption \cite[p256]{scto01}), 
so that one can then refer to pressure as a single well-defined value.
Note that in the case of an incompressible Newtonian Fluid, 
the strain rate tensor is deviatoric $({\text tr}(\dot{\bm \varepsilon}(\vec\upnu)) 
= \text{div}(\vec\upnu) =0)$ and the above considerations vanish.

Finally, for both compressible and incompressible flow, the stress tensor becomes simply
\begin{mdframed}[backgroundcolor=blue!5]
\begin{equation}
{\bm \sigma}=-p {\bm 1} + 2\eta \dot{\bm \varepsilon}^d(\vec\upnu) = -p {\bm 1} + {\bm \tau}
\end{equation}
\end{mdframed}
where ${\bm \tau} = 2\eta \dot{\bm \varepsilon}^d(\vec\upnu)$ is the deviatoric stress tensor.

\begin{remark}
On page 256 of Schubert, Turcotte and Olson \cite{scto01}, 
equation 6.5.3, the authors write $\tau_{ii}/3=k_B e_{ii}$ while stating that $\tau$ is deviatoric in equation 6.4.2. 
This is an obvious conflict of notations. 
\end{remark}

%------------------------------------------------------------------------
\section{The heat transport equation - energy conservation equation \label{ss:hte}}
\begin{flushright} {\tiny {\color{gray} physics.tex}} \end{flushright}

%The heat contained within a volume $dV$ is $\rho C_p T dV$ where 
%$C_p$ is the specific heat. The total heat ${\cal H}$ contained
%by $V$ is the sum ( or integral) of all the elements within $V$:
%\[
%{\cal H}=\int_V \rho C_p T dV
%\]
%Changes in ${\cal H}$ can only occur if heat flows across the surface $S$. 
%If $Q$ is the rate at which heat flows outward, then the rate of change of ${\cal H}$ 
%must equal $Q$, or
%\[
%-\frac{\partial {\cal H}}{\partial t} = Q
%\]
%The negative sign is required because the volume cools if $Q$ is positive. 
%The heat flux depends on the temperature gradient $\vec\nabla T$ , 
%and heat always flows down the temperature gradient. Hence the heat
%flux across the surface element $dS$, $dQ$, is
%\[
%dQ= - k \vec\nabla T \cdot \vec{n} \; dS
%\]
%where $k$ is the thermal conductivity. The dot product between $\vec\nabla T$ and $\vec{n}dS$ 
%takes the direction of the temperature gradient into account.
%We can then write:
%\[
%-\frac{\partial }{\partial t} \int_V \rho C_p T dV = - \int_S  k \vec\nabla T \cdot \vec{n} \; dS
%\]
%Using Gauss' theorem the right hand side becomes:
%\[
%\int_S  k \vec\nabla T \cdot \vec{n} \; dS
%= 
%\int_V \vec\nabla \cdot ( k \vec\nabla T ) dV
%\]
%so that:
%\[
%\int_V \left[ \frac{\partial }{\partial t} (\rho C_p T) - \vec\nabla \cdot( k \vec\nabla T ) \right] dV 
%\]
%Since $V$ can be any volume enclosed by an arbitrary surface, this 
%equation can only be true if the term in square brackets is zero everywhere, or:
%\[
%\rho C_p \frac{\partial T}{\partial t}  =  \vec\nabla \cdot( k \vec\nabla T ) 
%\]
%assuming density and specific heat to be constant in time.
%In a fluid which moves, the heat transport must include a term
%\[
%\int_S \rho C_p T \vec\upnu\cdot \vec{n} \; dS
%\]
%to take account of the heat carried by the fluid moving with velocity $\vec\upnu$. 
%This term is called the advection term in the equations, and we then have:
%\[
%\rho C_p \left( 
%\frac{\partial T}{\partial t} + (\vec\upnu \cdot \vec\nabla) T \right) =  \vec\nabla \cdot( k \vec\nabla T ) 
%\]
















%------------------------------------------------------------------------
\section{The momentum conservation equations} 
\begin{flushright} {\tiny {\color{gray} physics.tex}} \end{flushright}

As explained in Section~\ref{ss:nondim}, in Earth science applications the Navier-Stokes 
equations reduce to the Stokes equation:
\begin{equation}
{\vec \nabla}\cdot {\bm \sigma} + \rho {\vec g} = \vec{0}
\label{eq:forcebal}
\end{equation}
Since 
\begin{equation}
{\bm \sigma} = -p {\bm 1} + {\bm \tau}
\end{equation}
it also writes
\begin{equation}
-{\vec \nabla}p + {\vec \nabla}\cdot {\bm \tau} + \rho {\vec g} = \vec{0}
\end{equation}
Using the relationship ${\bm \tau} = 2 \eta \dot{\bm \varepsilon}^d(\vec\upnu)$ we arrive at 
\begin{mdframed}[backgroundcolor=blue!5]
\begin{equation}
-{\vec \nabla}p + {\vec \nabla}\cdot (2 \eta \dot{\bm \varepsilon}^d(\vec\upnu) ) + \rho {\vec g} = \vec{0}
\end{equation}
\end{mdframed}

The divergence of a tensor field in cylindrical coordinates ($r,\theta,z$)
has been obtained in Section~\ref{ss:cylcoord}.
The equations of motion \eqref{eq:forcebal} becomes\footnote{\url{https://en.wikipedia.org/wiki/Linear_elasticity}}

 \begin{align}
\frac{\partial \sigma_{rr}}{\partial r} + \cfrac{1}{r}\frac{\partial \sigma_{r\theta}}{\partial \theta} + \frac{\partial \sigma_{rz}}{\partial z} + \cfrac{1}{r}(\sigma_{rr}-\sigma_{\theta\theta}) + \rho g_r &= 0 \label{eq:momelr}\\
\frac{\partial \sigma_{r\theta}}{\partial r} + \cfrac{1}{r}\frac{\partial \sigma_{\theta\theta}}{\partial \theta} + \frac{\partial \sigma_{\theta z}}{\partial z} + \cfrac{2}{r}\sigma_{r\theta} + \rho g_\theta &=  0 \label{eq:momeltheta}\\
\frac{\partial \sigma_{rz}}{\partial r} + \cfrac{1}{r}\frac{\partial \sigma_{\theta z}}{\partial \theta} + \frac{\partial \sigma_{zz}}{\partial z} + \cfrac{1}{r}\sigma_{rz} + \rho g_z &= 0
\end{align}


%------------------------------------------------------------------------
\section{The mass conservation equations} 
\begin{flushright} {\tiny {\color{gray} physics.tex}} \end{flushright}
\index{general}{Solenoidal Field} 
\index{general}{Divergence-free}
\index{general}{Continuity Equation}
\index{general}{Mass Conservation Equation}

The mass conservation equation (often called continuity equation) is given by
\[
\frac{D\rho}{Dt} + \rho {\vec \nabla}\cdot{\vec \upnu} = 0
\]
or, since 
\[
\frac{D\rho}{Dt} = \frac{\partial \rho}{\partial t} + {\vec \upnu}\cdot {\vec \nabla}\rho
\]
then 
\[
\frac{D\rho}{Dt} + \rho {\vec \nabla}\cdot{\vec \upnu} = 
\frac{\partial \rho}{\partial t} + {\vec \upnu}\cdot {\vec \nabla}\rho
 + \rho {\vec \nabla}\cdot{\vec \upnu} = 0 
\]
and finally:
\begin{mdframed}[backgroundcolor=blue!5]
\begin{equation}
\frac{\partial \rho}{\partial t} + {\vec \nabla}\cdot(\rho {\vec \upnu}) = 0
\label{eq:massconvgen}
\end{equation}
\end{mdframed}
In the case of an incompressible flow, then $\partial \rho/\partial t=0$ and 
${\vec \nabla}\rho=0$, i.e. $D\rho/Dt=0$ and the remaining equation is simply:
\[
{\vec \nabla}\cdot{\vec \upnu} = 0
\]
A vector field that is divergence-free is also called 
solenoidal\footnote{\url{https://en.wikipedia.org/wiki/Solenoidal_vector_field}}.


In cylindrical coordinates $(r,\theta,\phi)$ 
the continuity equation for an incompressible fluid is :

\begin{mdframed}[backgroundcolor=blue!5]
\[
\frac{1}{r} \frac{\partial}{\partial r} (r \upnu_r) 
+
\frac{1}{r} \frac{\partial \upnu_\theta}{\partial \theta}
+
\frac{\partial \upnu_z}{\partial z}=0
\]
\end{mdframed}
\index{general}{Mass Conservation Equation (Cylindrical Coordinates)}


In spherical coordinates $(r,\theta,\phi)$ 
the continuity equation for an incompressible fluid is :

\begin{mdframed}[backgroundcolor=blue!5]
\begin{equation}
\frac{1}{r^2} \frac{\partial}{\partial r} (r^2 \upnu_r) 
+
\frac{1}{r \sin\theta} \frac{\partial}{\partial \theta} (\upnu_\theta \sin\theta)
+
\frac{1}{r \sin\theta} \frac{\partial \upnu_\phi}{\partial \phi}=0
\label{eq:divscc}
\end{equation}
\end{mdframed}
\index{general}{Mass Conservation Equation (Spherical Coordinates)}



%------------------------------------------------------------------------------
\section{The equations in \aspect manual}
\begin{flushright} {\tiny {\color{gray} physics.tex}} \end{flushright}

The following is lifted off the \aspect manual.
We focus on the system of equations in a $d=2$- or $d=3$-dimensional
domain $\Omega$ that describes the motion of a highly viscous fluid driven
by differences in the gravitational force due to a density that depends on
the temperature. In the following, we largely follow the exposition of this
material in Schubert, Turcotte and Olson \cite{scto01}.

Specifically, we consider the following set of equations for velocity $\vec\upnu$, pressure $p$ and temperature $T$:
\begin{align}
  \label{eq:stokes-1}
  -\vec\nabla \cdot \left[2\eta \left(\dot{\bm \varepsilon}(\vec \upnu)
                                  - \frac{1}{3}(\vec\nabla \cdot \vec \upnu)\mathbf 1\right)
                \right] + \vec\nabla p &=
  \rho \vec g
  &
  & \textrm{in $\Omega$},
  \\
  \label{eq:stokes-2}
  \vec\nabla \cdot (\rho \vec \upnu) &= 0
  &
  & \textrm{in $\Omega$},
  \\
  \label{eq:temperature}
  \rho C_p \left(\frac{\partial T}{\partial t} + \vec \upnu\cdot\vec\nabla T\right)
  - \vec\nabla\cdot k\vec\nabla T
  &=
  \rho H
  \notag
  \\
  &\quad
  +
  2\eta
  \left(\dot\varepsilon(\vec\upnu) - \frac{1}{3}(\vec\nabla \cdot \vec \upnu)\mathbf 1\right)
  :
  \left(\dot\varepsilon(\vec\upnu) - \frac{1}{3}(\vec\nabla \cdot \vec \upnu)\mathbf 1\right)
  \\
  &\quad
  +\alpha T \left( \vec \upnu \cdot \vec\nabla p \right)
  && \textrm{in $\Omega$},
  \notag
\end{align}
where $\dot{\bm \varepsilon}(\vec\upnu) = \frac{1}{2}(\vec\nabla \vec\upnu + \vec\nabla \vec\upnu^T)$ 
is the symmetric gradient of the velocity (often called the
\textit{strain rate} tensor).

In this set of equations, \eqref{eq:stokes-1} and \eqref{eq:stokes-2}
represent the compressible Stokes equations in which $\vec\upnu =\vec\upnu (\mathbf x,t)$ 
is the velocity field and $p=p(\mathbf x,t)$ the pressure
field. Both fields depend on space $\mathbf x$ and time $t$. Fluid flow is
driven by the gravity force that acts on the fluid and that is proportional to
both the density of the fluid and the strength of the gravitational pull.

Coupled to this Stokes system is equation \eqref{eq:temperature} for the
temperature field $T=T(\mathbf x,t)$ that contains heat conduction terms as
well as advection with the flow velocity $\vec\upnu$. The right hand side
terms of this equation correspond to
\begin{itemize}
\item internal heat production for example due to radioactive decay;
\item friction (shear) heating;
\item adiabatic compression of material;
\end{itemize}

In order to arrive at the set of equations in the \aspect manual
we need to 
\begin{itemize}
\item neglect the $\partial p/\partial t$ \todo{wrong rephrase} 
\item neglect the $\partial \rho / \partial t$  in \eqref{eq:massconvgen}.
\end{itemize}
from equations above. A partial answer is given in the next section. 

----------------------------------------

Also, their definition of the shear heating term $\Phi$ is:
\[
\Phi = k_B ({\vec \nabla}\cdot{\vec \upnu})^2 + 2\eta \dot{\bm \varepsilon}^d:\dot{\bm \varepsilon}^d
\]
For many fluids the bulk viscosity $k_B$ is very small and is often taken to be zero, an assumption known
as the Stokes assumption: $k_B=\lambda+2\eta/3=0$. \index{general}{Bulk Viscosity}
Note that $\eta$ is the dynamic viscosity and $\lambda$ the second viscosity. 
\index{general}{Dynamic Viscosity}
\index{general}{Second Viscosity}
Also, 
\[
{\bm \tau}=2\eta \dot{\bm \varepsilon} + \lambda ({\bm \nabla}\cdot{\vec \upnu}) {\bm 1}
\]
but since $k_B=\lambda+2\eta/3=0$, then $\lambda=-2\eta/3$ so 
\[
{\bm \tau}=2\eta \dot{\bm \varepsilon} -\frac{2}{3}\eta ({\bm \nabla}\cdot{\vec \upnu}) {\bm 1} = 2\eta \dot{\bm \varepsilon}^d
\]

\newpage
%---------------------------------------------------------------------------------------------------
\section{Equations for thermal convection in an anelastic, compressible, self-gravitating spherical mantle }
\begin{flushright} {\tiny {\color{gray} physics.tex}} \end{flushright}
%------------------------------------------------------------------------------

What follows is borrowed from Section 2.1 of Gli{\v{s}}ovi{\'c} \etal (2012) \cite{glfm12}.
We start from the conservation mass, momentum and energy equations (the full Navier-Stokes equations):
\begin{eqnarray}
\frac{\partial \rho}{\partial t} + \vec\nabla \cdot (\rho \vec\upnu) = \vec{0} \\
\rho \frac{D\vec\upnu}{Dt} = \vec\nabla\cdot {\bm \sigma} + \rho \vec{g} \\
\rho C_p \frac{D T}{Dt} = \vec\nabla \cdot k \vec\nabla T + \alpha T \frac{Dp}{Dt} + \Phi + Q
\end{eqnarray}
In solving for the mantle flow field that satisfies the equation of momentum conservation, 
we incorporate all effects arising from
self-gravitation and we must therefore explicitly consider the 3-D variation of 
gravity throughout Earth's interior. The
gravitational acceleration is written as
\[
\vec{g} = \vec\nabla \phi
\]
where $\phi$ is Earth's gravitational potential which satisfies Poisson's equation
\[
\Delta \phi = - 4 \pi {\cal G} \rho
\]
The gravitational potential is expressed as
\[
\phi = \phi_0(r) + \phi_1(r,\theta,\phi)
\]
where the subscript $0$ denotes a hydrostatic reference state, 
in which the structure of the mantle (density, gravity, pressure, temperature) varies
with radius alone and the subscript $1$ denotes all 3D perturbations arising from the 
thermal convection process. This decomposition makes sense in the context of a perfect sphere.

The total perturbed density and pressure fields in the mantle may similarly be expressed as
\[
\rho = \rho_0(r) + \rho_1(r,\theta,\phi)
\]
\[
p = p_0(r) + p_1(r,\theta,\phi)
\]
The equation of state relates the density perturbations to the temperature and pressure perturbations 
as follows
\[
\rho_1 = \rho_0[1-\alpha(T-T_0(r))+K_T^{-1} (p-p_0(r))] 
\]
where $K_T$ is the bulk modulus and the term $T_0(r)$ represents the horizontally averaged temperature (i.e.
the geotherm) which varies with radius only. 
The effects of compressibility on the density are found to be at least two orders of magnitude
smaller than the effects of temperature variations. Therefore, the last term 
of this equation is often neglected.
Note that this expression is a first order expansion of any Equation of State. \index{general}{Equation of State}

Also, this equation can be misleading if one forgets that the parameters $\alpha$ and $K_T$
cannot be constant but must be related through Maxwell relations (for example,
their definitions 
\[
\alpha = \frac{1}{V} \left( \frac{\partial V}{\partial T} \right)_P 
= -\frac{1}{\rho} \left( \frac{\partial \rho}{\partial T} \right)_P
\]
and 
\[
K_T 
= - V \left( \frac{\partial P}{\partial V} \right)_T
= \rho \left( \frac{\partial P}{\partial \rho} \right)_T
\]
imply that 
\[
\frac{\partial (\alpha\rho)}{\partial P}
= 
-\frac{\partial (\rho/K_T)}{\partial T}
\]
Some models can be found in the geophysical literature in which assumptions made inconsistently
about thermodynamic parameters (either constant or depth-dependent) violate the Maxwell rules.

\index{general}{ALA}
\index{general}{Anelastic Liquid Approximation}
Important simplifications are made assuming the anelastic-liquid approximation 
(e.g. Jarvis \& McKenzie (1980) \cite{jamc80}, Solheim \& Peltier (1990) \cite{sope90}).
This approximation is justified because the velocities associated with mantle convection 
are very slow compared to the local sound speed and
hence acoustic waves cannot be generated by the slow changes in the mantle pressure field. 
We therefore neglect the time derivative of density, thereby eliminating sound waves:
\[
\frac{\partial \rho}{\partial t} \simeq 0
\] 
For the same reason, the pressure distribution may be considered (to first-order accuracy) 
as the pressure of a fluid in hydrostatic equilibrium which yields
\[
\frac{Dp}{Dt} = \frac{\partial p}{\partial t} + \vec\upnu\cdot\vec\nabla p 
\simeq
- u_r \rho_0(r) g_0(r)
\]
The equations are then rewritten in terms of dimensionless variables according to the relations:
\begin{eqnarray}
r'&=&\frac{r}{d} \\
\upnu' &=& \frac{\upnu}{U} \\
t' &=& \frac{U}{d/t} \\
T' &=& \frac{T}{\Delta T} \\
\rho' &=& \frac{\rho}{\rho_{0s}} \\
g' &=& \frac{g}{g_{0s}} \\
\phi' &=& \frac{\phi}{g_{0s}d} \\
\alpha' &=& \frac{\alpha}{\alpha_s} \\
p' &=& \frac{p}{\alpha_s \Delta T \rho_{0s} g_{0s} d} \\
\tau_{ij} &=& \frac{\tau_{ij}}{\alpha_s \Delta T \rho_{0s} g_{0s} d} \\
\eta' &=& \frac{\eta}{\eta_s} \\
k' &=& \frac{k}{k_s} \\
Q' &=& \frac{Q d^2}{k_s \Delta T} \\
U &=& \frac{\rho_{0s}g_{0s}\alpha_s \Delta T d^2}{\eta_s}
\end{eqnarray}

in which the primes represent the dimensionless variables, the subscript $s$ means 
that we consider the surface value of the variable to which it
is applied. The length scale $d$ and temperature scale $\Delta T$ are respectively 
the depth of the mantle and the difference of temperature between
the bottom and the top of the mantle. 

Often one deals with dimensionless variables and the primes
are dropped for notational convenience (this is the case in what follows).

It is a tedious but trivial exercise to show that the dimensionless equation of 
conservation of momentum is then written as follows:
\[
\rho \frac{\Ranb_s}{Pr_s} \frac{D\vec\upnu}{Dt} =
\frac{\rho}{\alpha_s \Delta T} \vec\nabla \phi - \vec\nabla p + \vec\nabla \cdot {\bm \tau}
\]
in which we introduce the surface Rayleigh $\Ranb_s$ and Prandtl $\Prnb_s$ numbers defined, 
respectively, by
\[
\Ranb_s=\frac{\rho_{0s}^2 C_p g_{0s} \alpha_s \Delta T d^3}{k_s \eta_s}
\qquad
\qquad
Pr_s= \frac{\eta_s C_p}{k_s}
\]
Because of the very high viscosity of mantle rocks, the left-hand term 
is smaller than the other terms by several orders of magnitude
and may therefore be neglected. This important simplification is called the 
infinite Prandtl number approximation. \index{general}{Prandtl number}

The equation of energy conservation may also be rewritten in terms of the surface Rayleigh number, as follows
\[
\frac{D T}{D t} = \frac{1}{\rho \Ranb_s} \left( \vec\nabla\cdot k\vec\nabla T + Q \right) 
+\frac{Di}{\rho} \left( -\alpha T \frac{Dp}{Dt} + \Phi  \right)
\]
where $Di$ is the dissipation number (see Peltier (1972) \cite{pelt72}) 
which measures the importance of compression work and frictional heating, and it is defined
as \index{general}{Dissipation Number}
\[
Di=\frac{\alpha_s g_{0s} d}{C_p}
\]
$Di$ also measures the ratio of the depth of mantle convection ($d$) to 
the adiabatic scale height ($C_p/\alpha g_0$ ) and for whole-mantle convection is
close to order 1 (see Jarvis \& McKenzie (1980) \cite{jamc80}).

After simplifications, the dimensionless system of governing equations is written as

\begin{eqnarray}
\vec\nabla\cdot(\rho_0 \vec\upnu) =0 \\
\frac{\rho}{\alpha_s \Delta T} \vec\nabla \phi - \vec\nabla p + \vec\nabla \cdot {\bm \tau} = \vec{0} \\
\frac{\partial T}{\partial t} + \vec\upnu\cdot\vec\nabla T =  
\frac{1}{\rho_0 \Ranb_s} \left( \vec\nabla\cdot k\vec\nabla T + Q \right) +
\frac{Di}{\rho_0} (-\alpha T \rho_0 g_0 u_r + \Phi) 
\end{eqnarray}
with 
\[
\Delta \phi = -4\pi {\cal G} \rho
\]
\[
\rho_1=\rho_0(1-\alpha(T-T_0(r)))
\]
\todo[inline]{VERIFY all this !}

%------------------------------------------------------------------------------
\section{Non-dimensionalisation of the Navier-Stokes equations}\label{ss:nondim}
\begin{flushright} {\tiny {\color{gray} physics.tex}} \end{flushright}

%______________________________________________
\subsection{Approach \# 1 - isothermal flow}
%We start from the following form of the momentum conservation equation: 
%\[
%\frac{\partial \vec\upnu}{\partial t} + (\vec\upnu \cdot \vec\nabla)\vec\upnu 
%=
%-\frac{1}{\rho}\vec\nabla p + \nu \Delta \vec \upnu + \vec g
%\]
%where $\nu$ is the kinematic viscosity. \index{general}{Kinematic Viscosity}

We define (see for instance Massimi \etal (2006) \cite{maqs06}) four reference 
quantities which are relevant for geodynamics\footnote{Note that in the paper 
the authors conflate $\rho$ and $\tilde{\rho}$ which prevents them from non-dimensionalising
all terms as we do here.}:
\begin{itemize}
\item a reference viscosity value $\underline{\eta}=10^{20}~\si{\pascal\second}$
\item a reference mass density $\underline{\rho}=1000~\si{\kg\per\cubic\metre}$
\item a reference time $\underline{t}=1\text{Myr}\simeq 3.15\cdot 10^{13}~\si{\second}$
\item a reference length $\underline{l}=1000~\si{\metre}$
\item a reference gravity $\underline{g}=9.81~\si{\metre\per\square\second}$
\end{itemize}
It follows that a reference pressure can be obtained:
\[
\underline{p}=\underline{\rho} \underline{g} \underline{l} = 9.81\cdot 10^6~\si{\pascal}
\]
Note that there is unfortunately no natural selection for the pressure scale. 
We could also have used $\underline{p}=\underline{\rho}\underline{\upnu}^2$ 
where dynamic effects are dominant i.e. high velocity flows,
or $\underline{p}=\underline{\eta}\underline{\upnu}/\underline{l}$ 
where viscous effects are dominant i.e. creeping flows (which 
is the case in geodynamics).
The definition of a reference velocity is more straightforward:
\[
\underline{\upnu} = \frac{\underline{l}}{\underline{t}} = 1~\si{\mm\per\year}
\]
We define dimensionless variables through:
\[
{\color{teal}x} = \frac{x}{\underline{l}}
\qquad
{\color{teal}y} = \frac{y}{\underline{l}}
\qquad
{\color{teal}z} = \frac{z}{\underline{l}}
\qquad
{\color{teal}\vec\upnu'} = \frac{\vec\upnu}{\underline{\upnu}}
\qquad
{\color{teal}t} = \frac{t}{\underline{t}}
\qquad
{\color{teal} \eta} = \frac{\eta}{\underline{\eta}}
\qquad
{\color{teal} g} =\frac{g}{\underline{g}}
\]
where the {\color{teal}teal} color indicates dimensionless values.

Consequently, time and space derivatives will be rescaled as follows:
\[
{\color{teal} \vec\nabla} = \underline{l}\; \vec\nabla
\qquad
{\color{teal} \partial_t} = \underline{t}\; \partial_t 
\]
Using this scaling relations the Navier-Stokes equation become:
\[
\frac{\rho \underline{l}}{\underline{t}^2} {\color{teal} \frac{\partial \vec\upnu}{\partial t}}
+
\frac{\rho \underline{l}}{\underline{t}^2} {\color{teal} (\vec\upnu \cdot\vec\nabla) \vec\upnu} 
=
- \underline{\rho} \underline{g} {\color{teal} \vec\nabla p}
+
\frac{\underline{\eta}}{\underline{l}\underline{t}} 
{\color{teal} \vec\nabla \cdot \eta (\vec\nabla \vec \upnu + \vec\nabla \vec \upnu ^T)}
+ \rho \vec{g}
\]
I make ${\color{teal}\rho}= \rho/\underline{\rho}$ appear in the left hand side:
\[
\frac{\underline{\rho} \underline{l}}{\underline{t}^2}
{\color{teal}\rho}
 {\color{teal} \frac{\partial \vec\upnu}{\partial t}}
+
\frac{\underline{\rho} \underline{l}}{\underline{t}^2} 
{\color{teal}\rho}
{\color{teal} (\vec\upnu \cdot\vec\nabla) \vec\upnu} 
=
- \underline{\rho} g {\color{teal} \vec\nabla p}
+
\frac{\underline{\eta}}{\underline{l}\underline{t}} 
{\color{teal} \vec\nabla \cdot \eta (\vec\nabla \vec \upnu + \vec\nabla \vec \upnu ^T)}
+ \rho \vec{g}
\]
which we can divide by $\underline{\rho} \underline{l}/\underline{t}^2$ to obtain:
\[
 {\color{teal}\rho} \left(
{\color{teal} \frac{\partial \vec\upnu}{\partial t}}
+
{\color{teal} (\vec\upnu \cdot\vec\nabla) \vec\upnu} 
\right)
=
- \frac{ \underline{g} \underline{t}^2 }{\underline{l}} {\color{teal} \vec\nabla p}
+
\frac{\underline{\eta} \underline{t}}{\underline{\rho} \underline{l}^2 } 
{\color{teal} \vec\nabla \cdot \eta (\vec\nabla \vec \upnu + \vec\nabla \vec \upnu ^T)}
+ \frac{\underline{t}^2}{ \underline{l}}  {\color{teal} \rho} \vec{g}
\]
One can recognise in this equation the Reynolds and Froude 
non-dimensional numbers (the ratio between the inertial and viscous forces, and the 
ratio between buoyancy and inertial forces respectively). 
\[
Re = \frac{\underline{\rho} \underline{l}^2}{\underline{\eta} \underline{t} }
\qquad
Fr= \frac{\underline{l}}{\underline{g}\underline{t}^2 }
\]
From this we conclude that inertial forces in the Earth's mantle
are small compared to viscous forces.
We can then write:
\[
\boxed{
{\color{teal}\rho} \left(
{\color{teal} \frac{\partial \vec\upnu}{\partial t}}
+
{\color{teal} (\vec\upnu \cdot\vec\nabla) \vec\upnu}  \right)
=
- \frac{1}{Fr} {\color{teal} \vec\nabla p}
+
\frac{1}{Re}
{\color{teal} \vec\nabla \cdot \eta (\vec\nabla \vec \upnu + \vec\nabla \vec \upnu ^T)}
+ \frac{1}{Fr} {\color{teal} \vec{g}}
}
\]
In our case, given the definitions taken above, we have:
\[
Re \simeq 3.174 \cdot 10^{-24}
\qquad
Fr \simeq 1.027 \cdot 10^{-25}
\]
so that the inertial terms can be dropped from the momentum equation (thereby yielding the 
dimensionless Stokes equations):
\[
{\color{teal} \vec\nabla \cdot \eta (\vec\nabla \vec \upnu + \vec\nabla \vec \upnu ^T)}
- \frac{Re}{Fr} {\color{teal} \vec\nabla p}
+ \frac{Re}{Fr}  {\color{teal} \vec{g} }
=0
\]
Note that in our case $Re/Fr\simeq 30.5$. 

%_________________________________________________________________________
\subsection{Approach \# 2 - Temperature dependent \label{ss:dimeqs2}}

\begin{flushright} {\tiny {\color{gray} dimensionless\_equations2.tex.tex}} \end{flushright}
%~~~~~~~~~~~~~~~~~~~~~~~~~~~~~~~~~~~~~~~~~~~~~~~~~~~~~~~~~~~~~~~~~~~~~~~~~~~~~~~~~~~~~~~~~~~~~~~~~~

Let us now consider a box heated from below and cooled from above. 
We define 4 fundamental reference quantities:
\begin{itemize}
\item a length $L_{ref}$ (\si{\metre}), ({\color{violet} $L$})
\item a temperature $T_{ref}$ (\si{\kelvin}), ({\color{violet} $\uptheta$})
\item a viscosity $\eta_{ref}$ (\si{\pascal\second}), ({\color{violet} $ML^{-1}T^{-1}$})
\item a thermal diffusion coefficient $\kappa_{ref}$ (\si{\square\metre\per\second}), ({\color{violet} $L^2T^{-1}$})
\end{itemize}
From these reference quantities one can form secondary ones, such as
\begin{itemize}
\item a time $t_{ref} = L_{ref}^2 / \kappa_{ref}$ (aka the diffusion time)
\item a velocity $\upnu_{ref} = L_{ref} / t_{ref} = \kappa_{ref}/L_{ref}$
\item an acceleration $g_{ref} = \upnu_{ref} / t_{ref} = \kappa_{ref}^2/L_{ref}^3$
\item a strain rate $\dot{\varepsilon}_{ref} = t_{ref}^{-1} = \kappa_{ref} / L_{ref}^2$
\item a pressure $p_{ref} = \eta_{ref} \dot{\varepsilon}_{ref} = \eta_{ref} t_{ref}^{-1}$
\item a reference density $\rho_{ref} = \eta_{ref} L_{ref} t_{ref}/L_{ref}^3 = \eta_{ref} L_{ref}^{-2} t_{ref}$
\item a reference mass $M_{ref} = \eta_{ref} L_{ref} t_{ref}$
\item a reference energy $E_{ref} = \eta_{ref} L_{ref} t_{ref} \frac{L_{ref}^2}{t_{ref}^2} 
= \eta_{ref} \frac{L_{ref}^3}{t_{ref}}$
\item a reference heat conductivity\footnote{Units: W/m/K} $k_{ref}= E_{ref}/t_{ref}/L_{ref}/T_{ref}
= \eta_{ref} L_{ref}^2/t_{ref}^2/T_{ref}$
\item a reference heat capacity\footnote{Units: J/kg/K} 
$C_{ref}=E_{ref}/M_{ref}/T_{ref} = L_{ref}^2 /t_{ref}^2/T_{ref} $
\item a reference heat production coefficient\footnote{Units: W/kg}
$H_{ref}= E_{ref}/t_{ref}/M_{ref} = \frac{L_{ref}^2}{t_{ref}^3}$
\item a reference heat flux\footnote{Units: \si{\watt\per\square\meter}, or \si{\kg\per\cubic\second}} 
$q_{ref}= \eta_{ref} L_{ref} t_{ref}^{-2}$
\end{itemize}
We define {\color{teal}dimensionless} quantities as follows:
\begin{equation}
{\color{teal}x} = \frac{x}{L_{ref}}
\quad
{\color{teal}\vec\upnu} = \frac{\vec\upnu}{\upnu_{ref}}
\quad
{\color{teal}t} = \frac{t}{t_{ref}}
\quad
{\color{teal} \eta} = \frac{\eta}{\eta_{ref}}
\quad
{\color{teal} g} =\frac{g}{g_{ref}}
\quad
{\color{teal} k} = \frac{k}{k_{ref}}
\quad
{\color{teal} C_p} = \frac{C_p}{C_{ref}}
\quad
{\color{teal} \rho} =\frac{\rho}{\rho_{\text{ref}}}
\end{equation}

\begin{equation}
{\color{teal} H} = \frac{H}{H_{ref}}
\quad
{\color{teal} \vec\nabla} = L_{ref}\; \vec\nabla
\quad
{\color{teal} \partial_t} = t_{ref}\; \partial_t 
\quad
{\color{teal} T} = \frac{T}{T_{ref}}
\quad
{\color{teal} \dot{\varepsilon}} = \dot{\varepsilon} \; t_{ref}
\label{eq:physics_adimrels}
\end{equation}
We start from the standard Navier-Stokes equation\footnote{\url{https://en.wikipedia.org/wiki/Navier-Stokes_equations}}
\[
\rho \frac{D \vec\upnu}{D t}
=
-\vec\nabla p + \vec\nabla \cdot (2 \eta \dot{\bm\varepsilon})
+ \rho \vec{g} 
\]
and assume that the density is temperature-dependent (Boussinesq approximation) so that
\[
\rho \frac{D \vec\upnu}{D t}
=
-\vec\nabla p + \vec\nabla \cdot (2 \eta \dot{\bm\varepsilon})
+ \rho_0(1-\alpha T) \vec{g} 
\]
and remove the hydrostatic pressure (although we keep using $p$ for simplicity, $p$ is now the dynamic pressure):
\[
\rho \frac{D \vec\upnu}{D t}
=
-\vec\nabla p + \vec\nabla \cdot (2 \eta \dot{\bm\varepsilon})
- \rho_0\alpha T \vec{g} 
\]
We divide this equation by $p_{ref}=\eta_{ref}\dot{\varepsilon}_{ref}$:
\[
\frac{1}{\eta_{ref}\dot{\varepsilon}_{ref}} \rho
\frac{D \vec\upnu}{D t} 
=
-\vec\nabla {\color{teal}p} + \vec\nabla \cdot 2 \frac{\eta}{\eta_{ref}} \frac{\dot{\bm\varepsilon}}{\dot{\varepsilon}_{ref}}
- \frac{\rho_0\alpha T \vec{g}}{\eta_{ref} \dot{\varepsilon}_{ref}} 
\]
Let us call $\vec{e}$ the positive vertical vector ($\vec{e}_z$ in Cartesian coordinates, $\vec{e}_r$ in spherical coordinates), then 
$\vec{g} = -g_0 \vec{e}$ and we can write  (using $\dot{\varepsilon}_{ref}=\kappa_{ref}/L^2_{ref}$)
\[
\frac{1 }{\eta_{ref}\dot{\varepsilon}_{ref}}
(\rho_{ref}{\color{teal}\rho})\frac{D(\upnu_{ref} {\color{teal}\vec\upnu})}{D t}
=
-\vec\nabla {\color{teal}p} + \vec\nabla \cdot 2 {\color{teal} \eta} {\color{teal} \dot{\bm\varepsilon}}
+ \frac{\rho_0\alpha T g_0 }{\eta_{ref} (\kappa_{ref}/ L_{ref}^2)} \vec{e}
\]
Finally, dividing by $L_{ref}^{-1}$ (i.e. multiplying by $L_{ref}$) yields
\[
\frac{\upnu_{ref} \rho_{ref} L_{ref}}{\eta_{ref}\dot{\varepsilon}_{ref}  }
{\color{teal}\rho} \frac{{\color{teal} D \vec\upnu}}{ t_{ref} {\color{teal} D t}}
=
-{\color{teal} \vec\nabla} {\color{teal}p} + {\color{teal} \vec\nabla} \cdot 2 {\color{teal} \eta} {\color{teal} \dot{\bm\varepsilon}}
+ \frac{\rho_0\alpha ({\color{teal}T}T_{ref}) g_0 L_{ref}^3}{\eta_{ref} \kappa_{ref}} \vec{e}
\]
and finally (using $\upnu_{ref}=L_{ref}/t_{ref}$) 
\[
\frac{\rho_{ref} \kappa_{ref}}{\eta_{ref}}
{\color{teal}\rho} \frac{{\color{teal} D \vec\upnu}}{\color{teal} D t}
=
-{\color{teal} \vec\nabla} {\color{teal}p} + {\color{teal} \vec\nabla} \cdot 2 {\color{teal} \eta} {\color{teal} \dot{\bm\varepsilon}}
+ \frac{\rho_0\alpha T_{ref} g_0 L_{ref}^3}{\eta_{ref} \kappa_{ref}} {\color{teal} T} \vec{e}
\]
In the context of a system with a temperature difference $\Delta T$ 
between the bottom and top boundaries separated by a distance $H$, one would then take $T_{ref} = \Delta T$ 
and $L_{ref}=H$ so that the equation becomes:
\[
\underbrace{\frac{\rho_{ref} \kappa_{ref}}{\eta_{ref}}}_{\Prnb^{-1}}
{\color{teal}\rho} \frac{{\color{teal} D \vec\upnu}}{\color{teal} D t}
=
-{\color{teal} \vec\nabla} {\color{teal}p} + {\color{teal} \vec\nabla} \cdot 2 {\color{teal} \eta} {\color{teal} \dot{\bm\varepsilon}}
+ \underbrace{\frac{\rho_0\alpha \Delta T g_0 H^3}{\eta_{ref} \kappa_{ref}} }_{\Ranb} {\color{teal} T}\vec{e}
\]
and we obviously recover the classical definition of the Rayleigh number.

\begin{mdframed}[backgroundcolor=blue!5]
\[
\frac{1}{\Prnb}
{\color{teal}\rho} \frac{{\color{teal} D \vec\upnu}}{\color{teal} D t}
=
-{\color{teal} \vec\nabla} {\color{teal}p} + {\color{teal} \vec\nabla} \cdot 2 {\color{teal} \eta} {\color{teal} \dot{\bm\varepsilon}}
+ \Ranb {\color{teal} T}\vec{e}
\]
\end{mdframed}


On the left side of the equation we recognize the (inverse of the) Prandlt number $\Prnb=\frac{\eta}{\rho \kappa}$. 
We can estimate the dimensionless number before the inertial term for Earth
geodynamics:
\[
\Prnb \simeq \frac{10^{20-23}}{3000 \cdot 10^{-6}} >> 10^{23}
\]
Its inverse is then extremely small and this is why we neglect the inertial terms
in mantle modelling.


Note that if the fluid is isoviscous, one can then set $\eta_{ref}=\eta=\eta_0$ and then ${\color{teal}\eta}=1$ 
%and then 
%\[
%-{\color{teal} \vec\nabla} {\color{teal}p} + {\color{teal} \vec\nabla} \cdot 2  {\color{teal} \dot{\bm\varepsilon}}
%+ \Ranb {\color{teal} T} \vec{e}= \vec{0}
%\]

Turning now to the continuity equation $\vec\nabla \cdot\vec\upnu = 0$,
it is trivial to show that  ${\color{teal} \vec\nabla } \cdot  {\color{teal}\vec\upnu} = 0$.
Finally, starting from the simple heat transport equation:
\[
\frac{\partial T}{\partial t} + \vec\upnu\cdot\vec\nabla T = \kappa \Delta T
\]
We divide each side by $T_{ref}$ so that 
\[
\frac{\partial {\color{teal}T}}{\partial t} + \vec\upnu\cdot\vec\nabla {\color{teal}T} = \kappa \Delta {\color{teal}T}
\]
We now divide each side by the reference velocity $\upnu_{ref}$ 
and we obtain
\[
\frac{L_{ref}}{\kappa_{ref}} \frac{\partial {\color{teal}T}}{\partial t} 
+ {\color{teal} \vec\upnu} \cdot\vec\nabla {\color{teal}T} 
=  \frac{L_{ref}}{\kappa_{ref}}  \kappa \Delta {\color{teal}T}
\]
We multiply each side by $L_{ref}$ and we finally get
\[
\frac{L_{ref}^2}{\kappa_{ref}} 
\frac{\partial {\color{teal}T}}{\partial t}
+ {\color{teal} \vec\upnu} \cdot  {\color{teal}\vec\nabla} {\color{teal}T} =  {\color{teal} \kappa} {\color{teal}\Delta} {\color{teal}T}
\]
and finally
\[
\frac{ {\color{teal} \partial T}}{ {\color{teal} \partial t}} 
+ {\color{teal} \vec\upnu} \cdot  {\color{teal}\vec\nabla} {\color{teal}T} =  {\color{teal} \kappa} {\color{teal}\Delta} {\color{teal}T}
\]
The set of dimensionless equations is then:

\begin{mdframed}[backgroundcolor=blue!5]
\begin{eqnarray}
-{\color{teal} \vec\nabla} {\color{teal}p} + {\color{teal} \vec\nabla} \cdot 2  {\color{teal}\eta \dot{\bm\varepsilon}}
+ \Ranb  {\color{teal} T} \vec{e} &=& \vec{0} \label{eq:adimmm1}\\
{\color{teal} \vec\nabla } \cdot  {\color{teal}\vec\upnu} &=& 0 \label{eq:adimmm2}\\
\frac{ {\color{teal} \partial T}}{ {\color{teal} \partial t}} 
+ {\color{teal} \vec\upnu} \cdot  {\color{teal}\vec\nabla} {\color{teal}T} 
&=&  {\color{teal} \kappa} {\color{teal}\Delta} {\color{teal}T} \label{eq:adimmm3}
\end{eqnarray}
\end{mdframed}

\index{general}{Extended Boussinesq Approximation}
\index{general}{EBA}
Looking now at the Extended Boussinesq Approximation (EBA), we have to conside two additional terms in the 
energy equation:
\begin{itemize}
\item the shear heating $\Phi$ (See Eq.\eqref{eq:physicsshearheating}) 
$\Phi = 2 \eta  \dot{\bm \varepsilon}^d : \dot{\bm \varepsilon}^d$ 
\item the adiabatic heating $\alpha T \vec\upnu\cdot\vec\nabla p$
\end{itemize}
We start this time from 
\[
\rho C_p \left(\frac{\partial T}{\partial t} + \vec \upnu\cdot\vec\nabla T\right)
- \vec\nabla\cdot k\vec\nabla T 
= \rho H + 2\eta\dot{\bm \varepsilon}^d : \dot{\bm \varepsilon}^d 
+\alpha T  \vec\upnu \cdot \vec\nabla p 
\]

\[
\rho C_p \left(\frac{\partial T}{\partial t} + \vec \upnu\cdot\vec\nabla T\right)
- \vec\nabla\cdot k\vec\nabla T 
= \rho H + 2\eta\dot{\bm \varepsilon}^d : \dot{\bm \varepsilon}^d 
+\alpha T  \vec\upnu \cdot \vec\nabla p 
\]

\[
\rho C_p \frac{T_{ref}}{t_{ref}} \left(\frac{\partial {\color{teal}T}}{\partial {\color{teal}t}} 
+ {\color{teal} \vec \upnu}\cdot {\color{teal}\vec\nabla} {\color{teal}T}\right)
- \frac{T_{ref}}{L_{ref}^2} {\color{teal}\vec\nabla}\cdot k
{\color{teal}\vec\nabla} {\color{teal}T} 
= \rho_{ref} {\color{teal}\rho} H + \frac{\eta_{ref}}{t_{ref}^2} 2 {\color{teal}\eta} 
{\color{teal}\dot{\bm \varepsilon}^d} : 
{\color{teal}\dot{\bm \varepsilon}^d} 
+ \frac{p_{ref}}{t_{ref}} {\color{teal} \alpha T}  
{\color{teal}\vec\upnu} \cdot {\color{teal} \vec\nabla}
{\color{teal} p} 
\]
we then use $p_{ref} = \eta_{ref} t_{ref}^{-1}$ and
$\rho_{ref} = \eta_{ref} L_{ref}^{-2} t_{ref}$
\[
\rho_{ref} C_{ref} \frac{T_{ref}}{t_{ref}} {\color{teal} \rho} {\color{teal}C_p} 
\left(\frac{ {\color{teal} \partial T}}{ {\color{teal} \partial t}} 
+ {\color{teal} \vec \upnu}\cdot {\color{teal}\vec\nabla} {\color{teal}T}\right)
- \frac{T_{ref} k_{ref}}{L_{ref}^2} 
{\color{teal}\vec\nabla}\cdot {\color{teal} k}
{\color{teal}\vec\nabla} {\color{teal}T} 
= \frac{\eta_{ref}}{L_{ref}^2} t_{ref}  \frac{L_{ref}^2}{t_{ref}^3} 
{\color{teal}\rho} {\color{teal} H} 
+ \frac{\eta_{ref}}{t_{ref}^2} 2 {\color{teal}\eta} 
{\color{teal}\dot{\bm \varepsilon}^d} : 
{\color{teal}\dot{\bm \varepsilon}^d} 
+ \frac{\eta_{ref}}{t_{ref}^2} {\color{teal} \alpha T}  
{\color{teal}\vec\upnu} \cdot {\color{teal} \vec\nabla}
{\color{teal} p} 
\]
or, multiplying all by $t^2_{ref}/\eta_{ref}$:
\[
\frac{t_{ref}^2}{\eta_{ref}}
\rho_{ref} C_{ref} \frac{T_{ref}}{t_{ref}} {\color{teal} \rho} {\color{teal}C_p} 
\left(\frac{ {\color{teal} \partial T}}{ {\color{teal} \partial t}} 
+ {\color{teal} \vec \upnu}\cdot {\color{teal}\vec\nabla} {\color{teal}T}\right)
- \frac{t_{ref}^2}{\eta_{ref}} \frac{T_{ref} k_{ref}}{L_{ref}^2} 
{\color{teal}\vec\nabla}\cdot {\color{teal} k}
{\color{teal}\vec\nabla} {\color{teal}T} 
=  
{\color{teal}\rho} {\color{teal} H} 
+  2 {\color{teal}\eta} 
{\color{teal}\dot{\bm \varepsilon}^d} : 
{\color{teal}\dot{\bm \varepsilon}^d} 
+  {\color{teal} \alpha T}  
{\color{teal}\vec\upnu} \cdot {\color{teal} \vec\nabla}
{\color{teal} p} 
\]
We then make use of $C_{ref}=L_{ref}^2 /t_{ref}^2/T_{ref}$
and $k_{ref}= \eta_{ref} L_{ref}^2/t_{ref}^2/T_{ref}$
to arrive at
\[
{\color{teal} \rho} {\color{teal}C_p} 
\left(\frac{ {\color{teal} \partial T}}{ {\color{teal} \partial t}} 
+ {\color{teal} \vec \upnu}\cdot {\color{teal}\vec\nabla} {\color{teal}T}\right)
- 
{\color{teal}\vec\nabla}\cdot {\color{teal} k}
{\color{teal}\vec\nabla} {\color{teal}T} 
=  
{\color{teal}\rho} {\color{teal} H} 
+  2 {\color{teal}\eta} 
{\color{teal}\dot{\bm \varepsilon}^d} : 
{\color{teal}\dot{\bm \varepsilon}^d} 
+  {\color{teal} \alpha T}  
{\color{teal}\vec\upnu} \cdot {\color{teal} \vec\nabla}
{\color{teal} p} 
\]





\newpage
%_________________________________________________________________________
\section{The Navier-Stokes equations in cylindrical coordinates}
\begin{flushright} {\tiny {\color{gray} physics.tex}} \end{flushright}

In cylindrical coordinates, $(r,\theta,z)$, the continuity equation for an incompressible fluid is 
\begin{mdframed}[backgroundcolor=blue!5]
\begin{equation}
\frac{1}{r} \frac{\partial}{\partial r} (r \upnu_r) + 
\frac{1}{r} \frac{\partial}{\partial \theta} (\upnu_\theta) + 
\frac{\partial \upnu_z}{\partial z} =0
\end{equation}
\end{mdframed}
or
\begin{equation}
\frac{\partial \upnu_r }{\partial r}  + \frac{\upnu_r}{r} +
\frac{1}{r} \frac{\partial}{\partial \theta} (\upnu_\theta) + 
\frac{\partial \upnu_z}{\partial z} =0
\end{equation}
The Navier-Stokes equations of motion for an incompressible fluid with uniform viscosity are:
\begin{eqnarray}
\rho \left(  \frac{D\upnu_r}{Dt} -\frac{\upnu_\theta^2}{r} \right) 
&=& -\frac{\partial p}{\partial r} + f_r + \eta
\left( \Delta \upnu_r - \frac{\upnu_r}{r^2} - \frac{2}{r^2} \frac{\partial \upnu_\theta}{\partial \theta}
\right)
\nn\\
\rho \left(  \frac{D\upnu_\theta}{Dt} +\frac{\upnu_\theta \upnu_r}{r} \right) 
&=&
-\frac{1}{r} \frac{\partial p}{\partial \theta} + f_\theta + \eta
\left(
\Delta \upnu_\theta - \frac{\upnu_\theta}{r^2} + \frac{2}{r^2} \frac{\partial \upnu_r}{\partial\theta}
\right)
\nn\\
\rho \frac{D\upnu_z}{Dt} 
&=& 
-\frac{\partial p}{\partial z} + f_z + \eta \Delta \upnu_z
\end{eqnarray}
where the Lagrangian or material derivative is
\[
\frac{D}{Dt} = \frac{\partial}{\partial t} 
+ \upnu_r \frac{\partial}{\partial r}  
+ \frac{\upnu_\theta}{r} \frac{\partial}{\partial \theta}
+ \upnu_z \frac{\partial}{\partial z}  
\]
and the Laplacian operator is \index{general}{Laplace Operator} \index{general}{Laplacian} 
\begin{equation}
\Delta 
= \frac{\partial^2 }{\partial r^2}  +\frac{1}{r} \frac{\partial }{\partial r}
+ \frac{1}{r^2}  \frac{\partial^2}{\partial \theta^2}
+ \frac{\partial^2 }{\partial z^2}
\end{equation}
and for an incompressible, Newtonian fluid

\begin{eqnarray}
\sigma_{rr}&=& -p + 2 \eta \frac{\partial \upnu_r}{\partial r} \\
\sigma_{\theta\theta}&=& -p +2\eta \left( \frac{1}{r} \frac{\partial \upnu_\theta}{\partial \theta} +\frac{u}{r}\right)\\
\sigma_{zz}&=& -p + 2 \eta \frac{\partial \upnu_z}{\partial z} \\
\sigma_{rz}&=& \eta \left( \frac{\partial \upnu_r}{\partial z}+\frac{\partial \upnu_z}{\partial r}\right)\\
\sigma_{r\theta}&=&\eta\left( \frac{1}{r} \frac{\partial \upnu_r}{\partial \theta}
+\frac{\partial \upnu_\theta}{\partial r} - \frac{\upnu_\theta}{r}   \right) \\
\sigma_{\theta z} &=& \eta \left( \frac{1}{r} \frac{\partial \upnu_z}{\partial\theta} 
+ \frac{\partial \upnu_\theta}{\partial z} \right)
\end{eqnarray}


\newpage
%------------------------------------------------------------------------------
\section{The Stokes equations in spherical coordinates}
\begin{flushright} {\tiny {\color{gray} physics.tex}} \end{flushright}

In spherical coordinates, $(r,\theta,\phi)$, the continuity equation for an incompressible fluid is 
\begin{mdframed}[backgroundcolor=blue!5]
\begin{equation}
\frac{1}{r^2} \frac{\partial}{\partial r} (r^2 \upnu_r) + 
\frac{1}{r \sin \theta} \frac{\partial}{\partial \theta} (\upnu_\theta \sin \theta)+
\frac{1}{r \sin \theta} \frac{\partial \upnu_\phi}{\partial \phi} = 0
\end{equation}
\end{mdframed}
Concerning the momentum equation, we start from 
\begin{eqnarray}
{\vec \nabla}\cdot {\bm \sigma} + {\vec f} &=& \vec{0} 
\end{eqnarray}
The buoyancy force $\vec{f}$ is nearly always given by 
$\vec{f}=\rho \vec{g} = - \rho g \; \vec{e}_r$ (with $g>0$), i.e. $f_\phi=f_\theta=0$, and then
\[
- {\vec \nabla}p + {\vec \nabla}\cdot {\bm \tau} - \rho g \vec{e}_r = \vec{0}
\]
or,
\begin{eqnarray}
- ({\vec \nabla}p)_r      + ({\vec \nabla}\cdot {\bm \tau})_r     &=& \rho {g} \nonumber\\
- ({\vec \nabla}p)_\theta + ({\vec \nabla}\cdot {\bm \tau})_\theta&=&0  \nonumber\\
- ({\vec \nabla}p)_\phi   + ({\vec \nabla}\cdot {\bm \tau})_\phi  &=&0  \nonumber
\end{eqnarray}
The pressure gradient is simply given by:
\begin{eqnarray}
({\vec \nabla}p)_r &=& \frac{\partial p}{\partial r}  \nonumber\\
({\vec \nabla}p)_\theta &=& \frac{1}{r}\frac{\partial p}{\partial \theta}  \nonumber\\
({\vec \nabla}p)_\phi &=& \frac{1}{r\sin\theta}\frac{\partial p}{\partial \phi}  \nonumber
\end{eqnarray}
We now turn to the remaining three components of the divergence of deviatoric stress in 
spherical coordinates $r,\theta,\phi$, which are given by\footnote{Would be nice to 
have a ref here} %Eq.(\ref{eq_divtensor}):

\begin{eqnarray}
({\vec \nabla}\cdot {\bm \tau})_r 
&=& 
\frac{\partial \tau_{rr}}{\partial r} 
+ \frac{1}{r} \frac{\partial \tau_{\theta r}}{\partial \theta} 
+ \frac{1}{r \sin\theta} \frac{\partial \tau_{\phi r}}{\partial \phi} 
+ \frac{2 \tau_{rr} - \tau_{\theta\theta} -\tau_{\phi\phi}}{r} 
+ \frac{\tau_{\theta r} \cot\theta}{r} 
\nonumber\\
({\vec \nabla}\cdot {\bm \tau})_\theta
&=& 
\frac{\partial \tau_{r\theta}}{\partial r} 
+ \frac{1}{r} \frac{\partial \tau_{\theta \theta}}{\partial \theta} 
+ \frac{1}{r \sin\theta} \frac{\partial \tau_{\phi \theta}}{\partial \phi} 
+ \frac{3 \tau_{\theta r} + (\tau_{\theta\theta}- \tau_{\phi\phi}) \cot\theta}{r} 
 \nonumber\\
({\vec \nabla}\cdot {\bm \tau})_\phi
&=& 
\frac{\partial \tau_{r\phi}}{\partial r} 
+ \frac{1}{r} \frac{\partial \tau_{\theta \phi}}{\partial \theta} 
+ \frac{1}{r \sin\theta} \frac{\partial \tau_{\phi\phi}}{\partial \phi} 
+\frac{3 \tau_{r \phi }+2 \tau_{\phi \theta} \cot\theta}{r} 
\end{eqnarray}
And finally the momentum equation writes: 
\begin{eqnarray}
-\frac{\partial p}{\partial r} 
+\frac{\partial \tau_{rr}}{\partial r} 
+ \frac{1}{r} \frac{\partial \tau_{\theta r}}{\partial \theta} 
+ \frac{1}{r \sin\theta} \frac{\partial \tau_{\phi r}}{\partial \phi} 
+ \frac{2 \tau_{rr} - \tau_{\theta\theta} -\tau_{\phi\phi}}{r} 
+ \frac{\tau_{\theta r} \cot\theta}{r} 
 &=&  \rho g
\nonumber\\
\nonumber\\
- \frac{1}{r}\frac{\partial p}{\partial \theta}  
+\frac{\partial \tau_{r\theta}}{\partial r} 
+ \frac{1}{r} \frac{\partial \tau_{\theta \theta}}{\partial \theta} 
+ \frac{1}{r \sin\theta} \frac{\partial \tau_{\phi \theta}}{\partial \phi} 
+ \frac{3 \tau_{\theta r} + (\tau_{\theta\theta}- \tau_{\phi\phi}) \cot\theta}{r} 
  &=& 0 
\nonumber\\
\nonumber\\
- \frac{1}{r\sin\theta}\frac{\partial p}{\partial \phi} 
+\frac{\partial \tau_{r\phi}}{\partial r} 
+ \frac{1}{r} \frac{\partial \tau_{\theta \phi}}{\partial \theta} 
+ \frac{1}{r \sin\theta} \frac{\partial \tau_{\phi\phi}}{\partial \phi} 
+\frac{3 \tau_{r \phi }+2 \tau_{\phi \theta} \cot\theta}{r} 
&=&0
\label{eq_st1}
\end{eqnarray}
The deviatoric stress tensor components are
\begin{eqnarray}
\tau_{rr} 
&=& 2 \eta 
\left[ \dot{\varepsilon}_{rr} - 
\frac{1}{3} (\dot{\varepsilon}_{rr} + \dot{\varepsilon}_{\theta\theta} + \dot{\varepsilon}_{\phi\phi} ) \right] \nonumber\\
\tau_{\theta\theta} &=& 2 \eta 
\left[ \dot{\varepsilon}_{\theta\theta} - 
\frac{1}{3} (\dot{\varepsilon}_{rr} + \dot{\varepsilon}_{\theta\theta} + \dot{\varepsilon}_{\phi\phi} ) \right] \nonumber\\
\tau_{\phi\phi} &=& 2 \eta 
\left[ \dot{\varepsilon}_{\phi\phi} - 
\frac{1}{3} (\dot{\varepsilon}_{rr} + \dot{\varepsilon}_{\theta\theta} + \dot{\varepsilon}_{\phi\phi} ) \right] \nonumber\\
\tau_{r\theta} &=&  2 \eta  \dot{\varepsilon}_{r\theta}\\
\tau_{r\phi} &=& 2 \eta  \dot{\varepsilon}_{r\phi} \\
\tau_{\theta\phi} &=&  2 \eta  \dot{\varepsilon}_{\theta\phi}
\end{eqnarray}
with
\begin{eqnarray}
\dot\varepsilon_{rr} 
&=& \frac{\partial \upnu_r}{\partial r} \nonumber\\
\dot\varepsilon_{\theta\theta} 
&=& \frac{\upnu_r}{r} + \frac{1}{r} \frac{\partial \upnu_\theta}{\partial \theta}  \nonumber\\
\dot\varepsilon_{\phi\phi} 
&=& \frac{1}{r \sin\theta} \frac{\partial \upnu_\phi}{\partial \phi} +
\frac{\upnu_r}{r} +\frac{\upnu_\theta \cot \theta}{r} \nonumber\\
\dot\varepsilon_{\theta r} = \dot\varepsilon_{r\theta}   
&=& \frac{1}{2} \left( r \frac{\partial}{\partial r} (\frac{\upnu_\theta}{r} ) 
+\frac{1}{r} \frac{\partial \upnu_r}{\partial \theta} \right) \nonumber\\
\dot\varepsilon_{\phi r} = \dot\varepsilon_{r\phi}      
&=&  \frac{1}{2} \left(  \frac{1}{r \sin\theta} \frac{\partial \upnu_r}{\partial \phi} 
+ r \frac{\partial }{\partial r} (\frac{\upnu_\phi}{r}) \right)  \nonumber\\
\dot\varepsilon_{\phi \theta} = \dot\varepsilon_{\theta\phi} 
&=& \frac{1}{2} \left( \frac{\sin \theta}{r} \frac{\partial }{\partial \theta} (\frac{\upnu_\phi}{\sin\theta}) + \frac{1}{r \sin\theta} \frac{\partial \upnu_\theta}{\partial \phi}    \right) \nonumber
\end{eqnarray}
We go further by assuming the fluid to be incompressible
(i.e. $\dot{\varepsilon}_{rr} + \dot{\varepsilon}_{\theta\theta} + \dot{\varepsilon}_{\phi\phi} =0$) and then:
\begin{eqnarray}
\tau_{rr} = 2 \eta \dot{\varepsilon}_{rr}  
&=&
2 \eta  \frac{\partial \upnu_r}{\partial r} \nonumber\\
\tau_{\theta\theta} = 2 \eta \dot{\varepsilon}_{\theta\theta} 
&=& 2\eta \left( \frac{\upnu_r}{r} + \frac{1}{r} \frac{\partial \upnu_\theta}{\partial \theta} \right)
\nonumber\\
\tau_{\phi\phi} = 2 \eta \dot{\varepsilon}_{\phi\phi}  
&=& 2\eta \left( 
 \frac{1}{r \sin\theta} \frac{\partial \upnu_\phi}{\partial \phi} +
\frac{\upnu_r}{r} +\frac{\upnu_\theta \cot \theta}{r}  \right)
\nonumber\\
\tau_{r\theta} =  2 \eta  \dot{\varepsilon}_{r\theta}
&=&
\eta \left( r \frac{\partial}{\partial r} (\frac{\upnu_\theta}{r} ) 
+\frac{1}{r} \frac{\partial \upnu_r}{\partial \theta} \right)
\nonumber\\
\tau_{r\phi} = 2 \eta  \dot{\varepsilon}_{r\phi}
&=&
\eta \left(  \frac{1}{r \sin\theta} \frac{\partial \upnu_r}{\partial \phi} 
+ r \frac{\partial }{\partial r} (\frac{\upnu_\phi}{r}) \right)
\nonumber\\
\tau_{\theta\phi} =  2 \eta  \dot{\varepsilon}_{\theta\phi} 
&=&
\eta \left( \frac{\sin \theta}{r} \frac{\partial }{\partial \theta} (\frac{\upnu_\phi}{\sin\theta}) + \frac{1}{r \sin\theta} \frac{\partial \upnu_\theta}{\partial \phi}    \right)
\nonumber
\end{eqnarray}
Inserting these expressions in Eq.~\eqref{eq_st1} is a cumbersome affair...
Under the assumption that the fluid is also isoviscous, 
we get\footnote{I have not thoroughly checked these equations yet} 
\begin{eqnarray}
\rho g
&=&
-\frac{\partial p}{\partial r} + \eta \left( \Delta \upnu_r - \frac{2\upnu_r}{r^2} 
-\frac{2}{r^2} \frac{\partial \upnu_\theta}{\partial \theta} - \frac{2 \upnu_\theta \cot \theta}{r^2}
-\frac{2}{r^2 \sin\theta} \frac{\partial \upnu_\phi}{\partial \phi}
\right) 
\nonumber\\
0
&=& 
-\frac{1}{r}\frac{\partial p}{\partial \theta} 
+\eta \left(
\Delta \upnu_\theta + \frac{2}{r^2} \frac{\partial \upnu_r}{\partial \theta}
-\frac{\upnu_\theta}{r^2 \sin^2\theta } -\frac{2 \cot \theta}{r^2 \sin\theta}
\frac{\partial \upnu_\phi}{\partial \phi}  
\right) 
\nonumber\\
0
&=& 
- \frac{1}{r \sin\theta} \frac{\partial p}{\partial \phi}  + \eta
\left(
\Delta \upnu_\phi + \frac{2}{r^2 \sin\theta} \frac{\partial \upnu_r}{\partial \phi}
-\frac{\upnu_\phi}{r^2 \sin^2 \theta} + \frac{2 \cot \theta}{r^2 \sin\theta}
\frac{\partial \upnu_\theta}{\partial \phi}
\right) \nn\\
\end{eqnarray}
and the Laplacian operator is \index{general}{Laplace Operator} \index{general}{Laplacian} 
\[
\Delta = \frac{1}{r^2} \frac{\partial }{\partial r}\left( r^2 \frac{\partial }{\partial r}\right)
+\frac{1}{r^2 \sin\theta} \frac{\partial }{\partial \theta}
\left(
\sin\theta \frac{\partial }{\partial\theta}
\right)
+ \frac{1}{r^2 \sin^2\theta} \frac{\partial^2 }{\partial\phi^2}
\]

\todo[inline]{
Before being used these equations should be checked against multiple sources.  
}

%------------------------------------------------------------------------------
\section{The equations for axisymmetric geometries \label{ss:axicyleqs}}
\begin{flushright} {\tiny {\color{gray} axisymmetric\_eqs.tex}} \end{flushright}
%~~~~~~~~~~~~~~~~~~~~~~~~~~~~~~~~~~~~~~~~~~~~~~~~~~~~~~~~~~~~~~~~~~~~~~~~~~~~~~~~~~~~~~~~~~~~~~~~~~

In what follows we are concerned with incompressible flow.
In some cases the assumption can be made that the object we wish to sudy has an 
axisymmetric geometry, for example a plume:

\begin{center}
a)\includegraphics[width=5cm]{images/axisymmetry/keki97}
b)\includegraphics[width=5cm]{images/axisymmetry/lesy96a}
c)\includegraphics[width=5cm]{images/axisymmetry/lesy96b}\\
{\captionfont a)Taken from \textcite{keki97} (1997); 
b,c) Taken from \textcite{lesy96} (1996).}
\end{center}

Looking at the figure above we see that there are in fact two cases: axisymmetry in 
cylindrical coordinates (b) and axisymmetry in spherical coordinates (c).

As mentioned in \textcite{keki97} (1997): "By imposing axisymmetry,
we restrict the problem to two degrees of freedom, reducing the computational
effort significantly over 3D calculations."
However, \cite{reki04} (2004) also mention:
"An important caveat of axisymmetric calculations is that there are no variations 
in the $\phi$ direction (i.e., there are no $\phi$ derivatives in
the governing equations). Thus, as we get further from
the pole, the results become increasingly less physical.
Downwelling drips off the pole are actually downwelling
doughnuts that follow the entire small circle. In a fully
3D calculation, this doughnut feature would in reality be a drip."


See Section~\ref{ss:cyl_axi} for the FE formulation of these equations.

%---------------------------------------
\subsubsection{In cylindrical coordinates}

The velocity vector is $\vec{\upnu}=(\upnu_r,\upnu_\theta,\upnu_z)$. 
Due to the symmetry we have $\upnu_\theta=0$, $\partial_\theta \rightarrow 0$ 
and the Stokes equations 
then become \footnote{\url{https://en.wikipedia.org/wiki/Navier-Stokes_equations}}

\begin{eqnarray}
-\frac{\partial p}{\partial r} + \eta
\left(
\frac1r \frac{\partial}{\partial r} ( r  \frac{\partial \upnu_r}{\partial r}   ) 
+  \frac{\partial^2 \upnu_r}{\partial z^2} - \frac{\upnu_r}{r^2}
\right) +\rho g_r&=& 0 
\\
-\frac{\partial p}{\partial z} + \eta
\left(
\frac1r \frac{\partial}{\partial r} ( r  \frac{\partial \upnu_z}{\partial r}   ) 
+  \frac{\partial^2 \upnu_z}{\partial z^2} 
\right) +\rho g_z&=& 0 \\
\frac1r \frac{\partial}{\partial r} (r \upnu_r) + \frac{\partial \upnu_z}{\partial z} &=& 0
\end{eqnarray}


The strain rate tensor in cylindrical coordinates is given by 

\begin{eqnarray}
\dot\varepsilon_{rr} 
&=& \frac{\partial \upnu_r}{\partial r} 
\\
\dot\varepsilon_{\theta\theta} 
&=& \frac{\upnu_r}{r} + \frac{1}{r} \frac{\partial \upnu_\theta}{\partial \theta}  
\\
\dot\varepsilon_{\theta r} = \dot\varepsilon_{r\theta} 
&=& \frac{1}{2} \left(   \frac{\partial \upnu_\theta}{\partial r} - \frac{\upnu_\theta}{r} 
+\frac{1}{r} \frac{\partial \upnu_r}{\partial \theta}  \right)
\\
\dot\varepsilon_{zz} 
&=& \frac{\partial \upnu_z}{\partial z} 
\\
\dot{\varepsilon}_{rz} = \dot{\varepsilon}_{zr} 
&=& \frac{1}{2}\left( \frac{\partial \upnu_r}{\partial z} + \frac{\partial \upnu_z}{\partial r}  \right) 
\\
\dot{\varepsilon}_{\theta z} = \dot{\varepsilon}_{z \theta} &=& \frac{1}{2}\left( 
\frac{1}{r} \frac{\partial \upnu_z}{\partial \theta} + \frac{\partial \upnu_\theta}{\partial z}  \right) 
\end{eqnarray}

In the axisymmetric case, we have $\upnu_\theta=0$ and $\partial_\theta \rightarrow 0$ so that 
\begin{eqnarray}
\dot\varepsilon_{rr} &=& \frac{\partial \upnu_r}{\partial r}  \\
\dot\varepsilon_{\theta\theta} &=& \frac{\upnu_r}{r} \\
\dot\varepsilon_{r\theta} = \dot\varepsilon_{\theta r} &=& 0\\
\dot\varepsilon_{zz} &=& \frac{\partial \upnu_z}{\partial z} \\
\dot{\varepsilon}_{rz} = \dot{\varepsilon}_{zr} 
&=& \frac{1}{2}\left( \frac{\partial \upnu_r}{\partial z} + \frac{\partial \upnu_z}{\partial r}  \right) \\
\dot{\varepsilon}_{\theta z} = \dot{\varepsilon}_{z \theta} &=& 0
\end{eqnarray}
or, 
\[
\dot{\bm\varepsilon}
=
\left(
\begin{array}{ccc}
\dot\varepsilon_{rr} & 0 & \dot{\varepsilon}_{rz} \\
0 & \dot{\varepsilon}_{\theta\theta}  & 0 \\
\dot{\varepsilon}_{zr} & 0 & \dot\varepsilon_{zz}
\end{array}
\right)
\]

\Literature: Daly \& Raefsky (1985) \cite{dara85}, Kiefer \& Hager (1992) \cite{kiha92}.

This is implemented in \stone~36,63,90,91,92,96,106.

%---------------------------------------
\subsubsection{In spherical coordinates}

Assuming the flow velocity does not depend on $\phi$ ($\partial_\phi =0$) and therefore also that $\upnu_\phi=0$
\[
0=-\frac{\partial p}{\partial r} + f_r + \eta \left(\Delta v_r - \frac{2v_r}{r^2} -\frac{2}{r^2} \frac{\partial v_\theta}{\partial \theta} - \frac{2 v_\theta \cot \theta }{r^2} \right)
\]
\[
0 = -\frac{1}{r} \frac{\partial p}{\partial \theta} + \eta \left(\Delta v_\theta + \frac{2}{r^2} \frac{\partial v_r}{\partial \theta}  - \frac{v_\theta}{r^2 \sin^2 \theta} \right)
\]
with
\[
\Delta = \frac{1}{r^2} \frac{\partial }{\partial r}\left( r^2 \frac{\partial }{\partial r}\right)
+\frac{1}{r^2 \sin\theta} \frac{\partial }{\partial \theta}
\left(
\sin\theta \frac{\partial }{\partial\theta}
\right)
\]


\[
\Delta = \frac{1}{r^2} \frac{\partial }{\partial r}\left( r^2 \frac{\partial }{\partial r}\right)
+\frac{1}{r^2 \sin\theta} \frac{\partial }{\partial \theta}
\left(
\sin\theta \frac{\partial }{\partial\theta}
\right)
+ \frac{1}{r^2 \sin^2\theta} \frac{\partial^2 }{\partial\phi^2}
\]

THESE EQUATIONS SHOULD BE CHECKED and RE-CHECKED !!



From \cite{zebi93}:
\begin{equation}
\frac{1}{r^2} \frac{\partial}{\partial r} (r^2 \upnu_r) + 
\frac{1}{r \sin \theta} \frac{\partial}{\partial \theta} (\upnu_\theta \sin \theta)+
\frac{1}{r \sin \theta} \frac{\partial \upnu_\phi}{\partial \phi} = 0
\end{equation}
Pb with 1/r2 ??

\begin{eqnarray}
0 &=& -\frac{\partial p}{\partial r} + (1-\zeta) Ra \; r \; T + 
\frac{1}{r^2}\frac{\partial}{\partial r} \left( 2 \eta r^2 \frac{\partial \upnu_r}{\partial r} \right)
+ \frac{1}{r^2 \sin\theta} \frac{\partial}{\partial\theta} 
\left( \eta \sin\theta \frac{\partial \upnu_r}{\partial\theta} \right)
+\frac{\partial}{\partial \theta} \left(\eta \frac{\partial}{\partial r} \frac{\upnu_\theta}{r} \right)
\end{eqnarray}
where $\zeta=R_i/R_o$


The dimensional form of the energy equation in a spherical axisymmetric geometry is given by
(assuming the conductivity $k$ to be constant):
\[
\rho C_p \left( \frac{\partial T}{\partial t}  + 
\upnu_r \frac{\partial T}{\partial r} + \frac{\upnu_\theta}{r} \frac{\partial T}{\partial \theta}
\right)
=
k \frac{1}{r^2} \frac{\partial}{\partial r} \left( r^2 \frac{\partial T}{\partial r} \right)
+
k \frac{1}{r^2 \sin\theta} 
\frac{\partial}{\partial \theta} \left( \sin\theta \frac{\partial T}{\partial \theta}  \right) 
...
\]

THESE EQUATIONS SHOULD BE CHECKED and RE-CHECKED !!






\newpage
%---------------------------------
\section{The Boussinesq approximation}
\index{general}{Boussinesq Approximation}
\begin{flushright} {\tiny {\color{gray} physics.tex}} \end{flushright}

As nicely explained in \textcite{spve60} (1960): 
\begin{displayquote}
{\color{darkgray}
In the study of problems of thermal convection it is a frequent practice to simplify the basic 
equations by introducing certain approximations which are attributed to
Boussinesq (1903). The Boussinesq approximations can best be summarized by two
statements: 
\begin{enumerate}
\item The fluctuations in density which appear with the advent of motion
result principally from thermal (as opposed to pressure) effects. 
\item In the equations
for the rate of change of momentum and mass, density variations may be neglected except
when they are coupled to the gravitational acceleration in the buoyancy force."
\end{enumerate}
}
\end{displayquote}
Note that their paper examines the Boussinesq approximation for compressible fluids.  

[from \aspect{} manual]
The Boussinesq approximation assumes that the density can be
considered constant in all occurrences in the equations with the exception of
the buoyancy term on the right hand side of \eqref{eq:stokes-1}. The primary
result of this assumption is that the continuity equation \eqref{eq:stokes-2}
will now read ${\vec \nabla}\cdot{\vec \upnu} = 0$.
This implies that the strain rate tensor is deviatoric.
Under the Boussinesq approximation, the equations are much simplified:

\begin{align}
  \label{eq:stokes-1a}
  -\vec\nabla \cdot \left[2\eta \dot{\bm \varepsilon}(\vec\upnu)
                \right] + \vec\nabla p &=
  \rho \vec{g}
  &
  & \textrm{in $\Omega$},
  \\
  \label{eq:stokes-2a}
  \vec\nabla \cdot (\rho \vec\upnu) &= 0
  &
  & \textrm{in $\Omega$},
  \\
  \label{eq:temperaturee}
  \rho_0 C_p \left(\frac{\partial T}{\partial t} + \vec\upnu \cdot\vec\nabla T\right)
  - \vec\nabla\cdot k\vec\nabla T
  &=
  \rho H
  &
  & \textrm{in $\Omega$}
\end{align}
Note that all terms on the rhs of the temperature equations have disappeared, with the exception 
of the source term.


In \textcite{vacp22} we read: 
\begin{displayquote}
The Boussinesq approximation (\textcite{ober79}; Boussinesq, 1903; Rayleigh, 1916) 
assumes that density
variations are so small that they can be neglected everywhere
except in the buoyancy term in the momentum equation, 
which is equivalent to using a constant reference density profile. 
This implies incompressibility [...]. In addition, adiabatic heating
and shear heating are not considered in the energy equation. 
This approximation is valid as long as density variations are small 
and the modelled processes would cause no substantial 
shear or adiabatic heating.
The Boussinesq approximation is often used in lithosphere-
scale models. Due to its simplicity, the approximation of in-
compressibility is sometimes also adopted for whole-mantle
convection models, wherein it is only approximately valid,
and it has been shown that compressibility can have a large
effect on the pattern of convective flow \cite{tack96}.
\end{displayquote}


%---------------------------------
\section{The Extended Boussinesq approximation}
\index{general}{Extended Boussinesq Approximation}
\index{general}{EBA}
\begin{flushright} {\tiny {\color{gray} physics.tex}} \end{flushright}

In \textcite{vacp22} we read: 
\begin{displayquote}
The extended Boussinesq approximation (\cite{chyu85,oxtu78}) is based
on the same assumptions as the BA but does consider adiabatic 
and shear heating. Since it includes adiabatic heating,
but not the associated volume and density changes, it can
lead to artificial changes of energy in the model, i.e. material 
is being heated or cooled based on the assumption that
it is compressed or it expands, but the mechanical work that
causes compression or expansion is not done. Consequently,
the extended Boussinesq approximation should only be used
in models without substantial adiabatic temperature changes.

For a comparison between some of these approximations
using benchmark models, see e.g. \textcite{sthe89},
\textcite{lezh08}, \textcite{kilv10}, \textcite{gadb20}. 
In addition, the choice of approximation may
also be limited by the numerical methods being employed
(for example, the accuracy of the solution for the variables
that affect the density). Also note that, technically, these 
approximations are all internally inconsistent to varying 
degrees, since they do not fulfil the definitions of
thermodynamic variables but use linearised versions instead,
and they use different density formulations in the different 
equations. Nevertheless, many of them are generally accepted 
and widely used in geodynamic modelling studies, as
they allow for simpler equations and more easily obtained solutions.
\end{displayquote}

\textcite{yumc07} (2007) state that the background of the extended Boussinesq 
equations can be found described in \textcite{chyu85} and 
more completely in \textcite{mayu07}.

\Literature \cite{hayk91,hayk93}



\newpage
%%%%%%%%%%%%%%%%%%%%%%%%%%%%%%%%%%%%%%%%%%%%%%%%%%%%%%%%%%%%%%%%%%%%%%%%%%%%%%%%%%%%%%%%%%55
\section{Stokes equation for elastic medium}

\begin{flushright} {\tiny {\color{gray} elastic\_equations.tex}} \end{flushright}
%~~~~~~~~~~~~~~~~~~~~~~~~~~~~~~~~~~~~~~~~~~~~~~~~~~~~~~~~~~~~~~~~~~~~~~~~~~~~~~~~~~~~~~~~~~~~~~~~~~

{\large \color{orange} This will be moved to Section \ref{chapt:elasticity}}

What follows is mostly borrowed from Becker \& Kaus lecture notes \cite{beka}.

The strong form of the PDE that governs force balance in a medium is given by
\[
\vec{\nabla}\cdot{\bm \sigma}  + \vec{f} = \vec{0}
\]
where ${\bm \sigma}$ is the stress tensor and $\vec{f}$ is a body force.

The stress tensor is related to the strain tensor through the generalised 
Hooke's law\footnote{\url{https://en.wikipedia.org/wiki/Hooke's_law}}:
\begin{equation}
\sigma_{ij}=\sum_{kl}C_{ijkl}\varepsilon_{kl} 
\qquad
\text{or}
\qquad
{\bm \sigma} = {\bm C} : {\bm \varepsilon}
\label{eq:oone}
\end{equation}
where ${\bm C}$ is the fourth-order elastic tensor.

Due to the inherent symmetries of ${\bm \sigma}$, ${\bm \varepsilon}$, and ${\bm C}$, 
only 21 elastic coefficients of the latter are independent. 
For isotropic linear media (which have the same physical properties in any direction), ${\bm C}$ 
can be reduced to only two independent numbers (for example the bulk modulus $K$ and the shear modulus $G$ 
that quantify the material's resistance to changes in volume and to shearing deformations, respectively).
Thus
\[
C_{ijkl} = \lambda \delta_{ij}\delta_{kl} + \mu (\delta_{ik}\delta_{jl}+\delta_{il}\delta_{jk})
\]
so that Eq.~\eqref{eq:oone} becomes:
\[
\sigma_{ij}=\lambda \varepsilon_{kk} \delta_{ij} + 2\mu \varepsilon_{ij}
\]
or
\begin{mdframed}[backgroundcolor=blue!5]
\begin{equation}
{\bm \sigma}=\lambda (\vec{\nabla}\cdot\vec{u}) {\bm 1} +2\mu {\bm \varepsilon}(\vec{u}) \label{eq:twoELAST}
\end{equation}
\end{mdframed}
where $\lambda$ is the Lam\'e parameter and $\mu$ is the shear 
modulus\footnote{It is also sometimes written $G$}.
The term $\vec{\nabla}\cdot\vec{u}$ is the isotropic dilation.

\index{general}{Lam\'e Parameter} 
\index{general}{Shear Modulus}

This can be re-written in the 6-dimensional stress/strain space as
\[
\underbrace{
\left(
\begin{array}{c}
\sigma_{xx} \\
\sigma_{yy} \\
\sigma_{zz} \\
\sigma_{xy} \\
\sigma_{xz} \\
\sigma_{yz} 
\end{array}
\right)}
_{\vec{\sigma}}
=
\underbrace{
\left(
\begin{array}{cccccc}
\lambda+2\mu & \lambda & \lambda & 0 & 0 & 0 \\ 
\lambda & \lambda+2\mu & \lambda & 0 & 0 & 0 \\ 
\lambda & \lambda & \lambda+2\mu & 0 & 0 & 0 \\
0 & 0 & 0 & \mu & 0 & 0 \\ 
0 & 0 & 0 & 0 & \mu & 0 \\ 
0 & 0 & 0 & 0 & 0 & \mu  
\end{array}
\right)}
_{{\bm C}}
\cdot
\underbrace{
\left(
\begin{array}{c}
\varepsilon_{xx} \\
\varepsilon_{yy} \\
\varepsilon_{zz} \\
\varepsilon_{xy} \\
\varepsilon_{xz} \\
\varepsilon_{yz} 
\end{array}
\right)}
_{\vec{\varepsilon}}
\]
or, in terms of the compliance matrix ${\bm C}^{-1}$,
\index{general}{Compliance Matrix}
\[
\vec{\varepsilon} 
= {\bm C}^{-1} \cdot \vec{\sigma}
\]
with
\[
{\bm C}^{-1}
=
\frac{1}{\mu(3\lambda+2\mu)}
\left(
\begin{array}{cccccc}
\lambda+\mu & -\lambda/2 & -\lambda/2 & 0 & 0 & 0 \\
-\lambda/2 & \lambda+\mu & -\lambda/2 & 0 & 0 & 0 \\
-\lambda/2 & -\lambda/2 & \lambda+\mu & 0 & 0 & 0 \\
0 & 0 & 0 & 3\lambda+2\mu & 0 & 0 \\ 
0 & 0 & 0 & 0 & 3\lambda+2\mu & 0 \\ 
0 & 0 & 0 & 0 & 0 & 3\lambda+2\mu  
\end{array}
\right)
\]
If we define the Young's modulus as $E=\mu(3\lambda+2\mu)/(\lambda+\mu)$ 
and the Poisson's ratio as $\nu=\lambda(\lambda+\mu)/2$, then
\[
{\bm C}^{-1}
=
\frac{1}{E}
\left(
\begin{array}{cccccc}
1 & -\nu & -\nu & 0 & 0 & 0 \\
-\nu & 1 & -\nu & 0 & 0 & 0 \\
-\nu & -\nu & 1 & 0 & 0 & 0 \\
0 & 0 & 0 & 2(1+\nu) & 0 & 0 \\ 
0 & 0 & 0 & 0 & 2(1+\nu) & 0 \\ 
0 & 0 & 0 & 0 & 0 & 2(1+\nu) 
\end{array}
\right)
\]
Note that the determinant  of ${\bm C}^{-1}$ is $8(1+\nu)^5(1-2\nu)E^{-6}$,
so that when $\nu\rightarrow 1/2$ (incompressible material), the compliance
matrix is singular and the stress cannot be given as a function of strain \cite{lubliner}.


The strain tensor is related to the displacement as follows: \index{general}{Strain Tensor}
\[
{\bm \varepsilon}(\vec{u}) 
= \frac{1}{2}(\vec{\nabla}\vec{u} + (\vec{\nabla}\vec{u})^T)
\]
The incompressibility (or bulk modulus) $K$ is defined as $p=-K \vec{\nabla}\cdot\vec{u}$ 
where $p$ is the pressure with \index{general}{Bulk Modulus}
\begin{eqnarray}
p&=&-\frac{1}{3} \text{tr}({\bm \sigma}) \nonumber\\
 &=& -\frac{1}{3} [ \lambda (\vec{\nabla}\cdot\vec{u}) \text{tr}[{\bm 1}] + 2 \mu {\rm tr}[{\bm \varepsilon}(\vec{u})]] \nonumber\\
 &=& -\frac{1}{3} [ \lambda (\vec{\nabla}\cdot\vec{u})  3  + 2 \mu  (\vec{\nabla}\cdot\vec{u}) ] \nonumber\\
 &=& -\left[ \lambda + \frac{2}{3} \mu \right] (\vec{\nabla}\cdot\vec{u})  
\end{eqnarray}
so that 
\begin{mdframed}[backgroundcolor=blue!5]
\[
p=-K \vec{\nabla}\cdot\vec{u} 
\qquad
\text{with}
\qquad
K=\lambda+\frac{2}{3}\mu
\]
\end{mdframed}

\begin{remark}
Eq. (\ref{eq:oone}) and (\ref{eq:twoELAST}) are analogous to the ones that one has to solve
in the context of viscous flow using the penalty method. In this case $\lambda$ is the penalty coefficient, 
$\vec{u}$ is the velocity, and $\mu$ is then the dynamic viscosity.
\end{remark}

The Lam\'e parameter and the shear modulus are also linked to $\nu$ the poisson ratio, 
and $E$, Young's modulus: \index{general}{Poisson Ratio} \index{general}{Young's Modulus}
\[
\lambda=\mu\frac{2\nu}{1-2\nu}
=\frac{\nu E}{(1+\nu)(1-2\nu)}
\quad\quad
{\rm with}
\quad\quad
E=2\mu(1+\nu)
\]
The shear modulus, expressed often in GPa, describes the material's response to shear stress.
The poisson ratio describes the response in the direction orthogonal to uniaxial stress.
The Young modulus, expressed in GPa, describes the material's strain response to uniaxial stress in the 
direction of this stress.


\Literature: solvers for 3D Stokes and elasticity problems with
heterogeneous coefficients \cite{samb20}























\newpage
%-------------------------------
\section{Boundary conditions}
\begin{flushright} {\tiny {\color{gray} physics.tex}} \end{flushright}

%wiki
In mathematics, the Dirichlet (or first-type) 
boundary condition is a type of boundary condition, named after Peter Gustav Lejeune Dirichlet.
When imposed on an ODE or PDE, it specifies the values that a solution needs 
to take along the boundary of the domain.
Note that a Dirichlet boundary condition may also be referred to as a fixed boundary condition. 

The Neumann (or second-type) boundary condition is a type of boundary condition, 
named after Carl Neumann. When imposed on an ordinary or a partial differential equation, 
the condition specifies the values in which the derivative of a solution is 
applied within the boundary of the domain.

It is possible to describe the problem using other boundary conditions: 
a Dirichlet boundary condition specifies the values of the solution itself 
(as opposed to its derivative) on the boundary, whereas the Cauchy boundary condition, 
mixed boundary condition and Robin boundary condition are all different types of combinations 
of the Neumann and Dirichlet boundary conditions.

\index{general}{Dirichlet Boundary Condition}
\index{general}{Neumann Boundary Condition}

%....................................
\subsection{The Stokes equations}

You may find the following terms in the computational geodynamics literature:

\begin{itemize}
\item { free surface}: this means that no force is acting on the surface, i.e. ${\bm \sigma}\cdot {\vec n}={\vec 0}$. It is usually used on the top boundary of the domain and allows for topography evolution.
\item { free slip}: ${\vec \upnu}\cdot \vec n = 0$ and $({\bm \sigma}\cdot{\vec n})\times {\vec n}={\vec 0}$. This condition ensures a frictionless flow parallel to the boundary where it is prescribed.
\item { no slip}: this means that the velocity (or displacement) is exactly zero on the boundary, i.e. ${\vec \upnu}={\vec 0}$.
\item { prescribed velocity}: ${\vec \upnu}={\vec \upnu}_{bc}$
\item stress b.c.: 
\item open .b.c.: see \stone 29. 
\end{itemize}

%....................................
\subsection{The heat transport equation}

There are two types of boundary conditions for this equation: temperature boundary conditions (Dirichlet boundary conditions) and heat flux boundary conditions (Neumann boundary conditions). 

\newpage
%------------------------------------------
\section{Meaningful physical quantities}
\begin{flushright} {\tiny {\color{gray} physics.tex}} \end{flushright}

\begin{itemize}
\item {\color{violet} Velocity} $\vec \upnu (\text{m/s})$: This is a vector quantity and both magnitude and direction are needed to define it. It is the rate of change of position with respect to a frame of reference.
\item {\color{violet} Root mean square velocity} $\upnu_{rms} (\text{m/s})$: 
\begin{equation}
\upnu_{rms} = \left ( \frac{\int_\Omega |{\vec \upnu}|^2 \;  dV}{\int_\Omega dV }  \right )^{1/2}
=\left ( \frac{1}{V_\Omega} \int_\Omega |{\vec \upnu}|^2 \;  dV \right )^{1/2} \label{eqVrms}
\end{equation}
\begin{remark}
$V_\Omega$ is usually computed numerically at the same time that $\upnu_{vrms}$ is computed.
\end{remark}
In Cartesian coordinates, for a cuboid domain of size $Lx\times L_y \times Lz$, 
the $\upnu_{rms}$ is simply given by:
\begin{equation}
\upnu_{rms}  = \left ( \frac{1}{L_xL_yL_z} \int_0^{L_x}\int_0^{L_y}\int_0^{L_z} 
(u^2 + v^2 + w^2) dxdydz  \right )^{1/2}
\end{equation}
In the case of an annulus domain, although calculations are carried out 
in Cartesian coordinates, it makes sense
to look at the radial velocity component $\upnu_r$ and the tangential velocity 
component $\upnu_\theta$, and their respective
root mean square averages:
\begin{equation}
\upnu_r|_{rms}  =\left ( \frac{1}{V_\Omega} \int_\Omega v_r^2 \;  d \Omega \right )^{1/2} \label{eqVrVrms}
\end{equation}
\begin{equation}
\upnu_\theta|_{rms}  = \left ( \frac{1}{V_\Omega} \int_\Omega v_\theta^2 \;  d \Omega \right )^{1/2} \label{eqThetaVrms}
\end{equation}


\item {\color{violet} Pressure} $p$ (\si{\pascal}):
\item {\color{violet} Stress tensor} ${\bm \sigma}$ (\si{\pascal}): \index{general}{Stress Tensor}
\item {\color{violet} Strain tensor} ${\bm \varepsilon}$ (dimensionless): \index{general}{Strain Tensor}
\item {\color{violet} Strain rate tensor} $ \dot{\bm \varepsilon}$ (\si{\per\second}): 
\index{general}{Strain Rate Tensor}

%--------------------------------------------------------------------------------------------------
\item {\color{violet} Argand Number}: 
Non-dimensional number (Ar) representing the ratio of the stress arising 
from crustal thickness
contrasts (vertical stress) to the stress required to deform
the material at ambient strain rates (horizontal stress)
It is commonly used in mountain building
dynamics as a measure of the tendency of an orogen to
collapse under its own gravitational potential energy.
See England \& McKenzie \cite{enmc82}, Houseman \& England \cite{hoen86a}.

%--------------------------------------------------------------------------------------------------
\item {\color{violet} (Thermal) Rayleigh number} $\Ranb$ (or $\Ranb_T$) (X): \index{general}{Rayleigh Number}
It is a dimensionless number that expresses the	ratio of the driving forces to the opposing forces.
The buoyancy force comes from the volumetric thermal expansion while the viscous forces and 
the heat diffusivity oppose convection (the latter one smoothes out thermal gradients). 

The Rayleigh number for convection driven by a constant temperature hot base and a cold surface
in a domain of thickness $D$ is:
\[
\Ranb 
= \frac{\rho_0 g \alpha D^3 }{\eta \kappa}  \cdot  \Delta T
= \frac{\rho_0^2 C_p g \alpha D^3 \Delta T}{\eta k}
\]
The Rayleigh number for convection driven by a hot base (constant basal heat flow $q_b$)
and a colder surface is:
\[
\Ranb = \frac{\rho_0 g \alpha D^3}{\eta \kappa } \cdot  \frac{q_b D}{k}
\]  
The Rayleigh number for convection driven by internal heating $H$ (production per cubic meter) is:
\[
\Ranb = \frac{\rho_0 g \alpha D^3}{\eta \kappa} \cdot  \frac{H D^2}{k }
\]
The Rayleigh number for convection driven by both basal heat flow and internal heating is:	
\[
\Ranb = \frac{\rho_0 g \alpha D^3}{\eta \kappa} \cdot  \frac{q_b D + H D^2}{k }
\]
For convection to occur, the Rayleigh number must be larger than the so-called critical 
Rayleigh number, which ranges from 600 to 3000 (it depends on the boundary conditions and the 
geometry).
\index{general}{Critical Rayleigh Number}

%--------------------------------------------------------------------------------------------------
\item {\color{violet}Compositional Rayleigh Number} $\Ranb_C$: \index{general}{Compositional Rayleigh Number} 
\[
\Ranb_C= \frac{\Delta \rho_C  g  D^3}{\kappa \eta_0}
\]
where $\Delta \rho_C$ is the difference in density between the distinct material compositions
(when compared at identical temperatures). See for instance Trim \etal (2020) \cite{trlb20}.

%--------------------------------------------------------------------------------------------------
\item {\color{violet} Prandtl number} $\Prnb$ (X): \index{general}{Prandtl Number} 
It is named after the German physicist 
Ludwig Prandtl\footnote{\url{https://en.wikipedia.org/wiki/Ludwig_Prandtl}} 
and is defined as the ratio of momentum diffusivity to thermal diffusivity. 
It is given as: 
\[
\Prnb = \frac{\text{momentum diffusivity}}{\text{thermal diffusivity}} = \frac{\eta/\rho}{k/(\rho C_p)}= \frac{\eta C_p}{k}
\]
For Earth materials, we have $Pr \sim (10^{21} 1000)/3 >> 1$, 
which means that momentum diffusivity dominates.

%..........................................
\item {\color{violet} Nusselt number} $\Nunb$ (X): \index{general}{Nusselt Number}  
the Nusselt number ($\Nunb$) 
is the ratio of convective to conductive heat transfer across (normal to) the boundary. 
The conductive component is measured under the same conditions as the heat convection 
but with a (hypothetically) stagnant (or motionless) fluid.

In practice the Nusselt number $\Nunb$ of a layer (typically the mantle of a planet) is defined as follows:
\begin{equation}
\Nunb = \frac{q}{q_c}
\end{equation} 
where $q$ is the heat transferred by convection while $q_c=k \Delta T /D$ 
is the amount of heat that would be conducted through a layer of
thickness $D$ with a temperature difference $\Delta T$ across it with 
$k$ being the thermal conductivity.

For 2D Cartesian systems of size ($L_x$,$L_y$) the $\Nunb$ is computed \cite{blbc89}
\[
\Nunb = 
\frac{\frac{1}{L_x}\int_{0}^{L_x} k \frac{\partial T}{\partial y}(x,y=L_y) dx }
{-\frac{1}{L_x}\int_0^{L_x} k T(x,y=0) /L_y dx}
=-L_y \frac{\int_{0}^{L_x} \frac{\partial T}{\partial y}(x,y=L_y) dx }{\int_0^{L_x} T(x,y=0) dx}
\]
i.e. it is the mean surface temperature gradient
over the mean bottom temperature.

\todo[inline]{finish. not happy with definition. Look at literature}

Note that in the case when no convection takes place then the measured heat flux at the top is 
the one obtained from a purely conductive profile which yields $\Nunb$=1.

Note that a relationship  $\Ranb \propto \Nunb^\alpha $ exists between the Rayleigh 
number $\Ranb$ and the Nusselt number $\Nunb$ in convective systems, see \cite{wodd09} and references therein. 

Turning now to cylindrical geometries with inner radius $R_1$ and outer radius $R_2$, 
we define $f=R_1/R_2$. A small value of $f$ corresponds to a high
degree of curvature. We assume now that $R_2-R_1=1$, so that $R_2=1/(1-f)$ and $R_1=f/(1-f)$. 
Following \cite{jarv93}, the Nusselt number at the inner and outer boundaries are:
\begin{equation}
\boxed{
\Nunb_{inner} 
=\frac{f \ln f}{1-f} \frac{1}{2\pi} \int_0^{2\pi} \left( \frac{\partial T}{\partial r}\right)_{r=R_1}d\theta
}
\label{eqNuAnnIn}
\end{equation}
\begin{equation}
\boxed{
\Nunb_{outer} 
= \frac{\ln f}{1-f} \frac{1}{2\pi} \int_0^{2\pi} \left( \frac{\partial T}{\partial r} \right)_{r=R_2} d\theta
}
\label{eqNuAnnOut}
\end{equation}

Note that a conductive geotherm in such an annulus between temperatures $T_1$ and $T_2$ is given by 
\[
T_c(r)=\frac{\ln (r/R_2)}{\ln(R_1/R_2)} = \frac{\ln(r(1-f))}{\ln f}
\]
so that 
\[
\frac{\partial T_c}{\partial r} = \frac{1}{r}\frac{1}{\ln f} 
\]
We then find:
\begin{eqnarray}
\Nunb_{inner} 
&=& \frac{f \ln f}{1-f} \frac{1}{2\pi} \int_0^{2\pi} \left( \frac{\partial T_c}{\partial r} \right)_{r=R_1} d\theta
= \frac{f \ln f}{1-f} \frac{1}{R_1}\frac{1}{\ln f} 
= 1 \\
\Nunb_{outer} 
&=& \frac{\ln f}{1-f} \frac{1}{2\pi} \int_0^{2\pi} \left( \frac{\partial T_c}{\partial r} \right)_{r=R_2} d\theta 
= \frac{\ln f}{1-f} \frac{1}{R_2}\frac{1}{\ln f} = 1 
\end{eqnarray}
As expected, the recovered Nusselt number at both boundaries is exactly 1 when the temperature field is
given by a steady state conductive geotherm.

\todo[inline]{derive formula for Earth size R1 and R2}

\Literature \cite{hohr87}
 
%..........................................
\item {\color{violet} Temperature} (\si{\kelvin}):

%--------------------------------------------------------------------------------------------------
\item {\color{violet} (Dynamic) Viscosity} (\si{\pascal\second}): 
For air it is roughly $10^{-5}\si{\pascal\second}$, 
about $10^{-3}~\si{\pascal\second}$ for water, 
about $10^{10}~\si{\pascal\second}$ for ice and 
about $10^{17}~\si{\pascal\second}$ for salt. 

%--------------------------------------------------------------------------------------------------
\item {\color{violet} Entropy} $S$ (\si{\joule\per\kelvin})

%--------------------------------------------------------------------------------------------------
\item {\color{violet} (mass) Density} $\rho$ (\si{\kg\per\cubic\metre}):

%--------------------------------------------------------------------------------------------------
\item {\color{violet} Heat capacity} $C_p$ (\si{\joule\per\kelvin}): 
It is the measure of the heat/energy required to increase the 
temperature of a unit quantity of a substance by unit degree. Note that the {\it specific} 
heat capacity $c_P$ of a substance is the heat capacity of a sample of the substance 
divided by the mass of the sample, with units \si{\joule\per\kelvin\per\kg}.

``Different substances respond to heat in different ways. If a metal chair sits in the bright sun 
on a hot day, it may become quite hot to the touch. An equal mass of water under the same sun exposure will 
not become nearly as hot. This means that water has a high heat capacity (the amount of heat required to 
raise the temperature of an object by $1~\si{\celsius}$). Water is very resistant to changes in temperature, 
while metals generally are not.''
\footnote{\url{https://chem.libretexts.org/Bookshelves/Introductory_Chemistry/Introductory_Chemistry_(CK-12)/17\%3A_Thermochemistry/17.04\%3A_Heat_Capacity_and_Specific_Heat}}


%--------------------------------------------------------------------------------------------------
\item {\color{violet} Heat conductivity}, or thermal conductivity $k$ (\si{\watt\per\metre\per\kelvin}). 
It is the property of a material that indicates its ability to conduct heat. It appears primarily 
in Fourier's Law for heat conduction.
Note that it is a function of temperature, especially in mantle convection settings,
see Bonneville \& Capolsini (1999) \cite{boca99} and refs therein, 
Miyauchi \& Kameyama (2013) \cite{mika13}, Hofmeister \& Yuen (2007) \cite{hoyu07}. 
Note also that it can be a tensorial quantity in anisotropic context.
The heat conductivity of many rocks was determined in \cite{ando13}.

%--------------------------------------------------------------------------------------------------
\item {\color{violet} Heat diffusivity}: $\kappa=k/(\rho C_p)$ ($\si{\square\meter\per\second}$). 
Substances with high thermal diffusivity rapidly adjust their temperature to that of their surroundings, because they 
conduct heat quickly in comparison to their volumetric heat capacity or 'thermal bulk'.

%--------------------------------------------------------------------------------------------------
\item {\color{violet} thermal expansion} $\alpha$ (\si{\per\kelvin}): 
it is the tendency of a matter to change in volume in response to a change in temperature. 
Note that it is a function of temperature, especially in mantle convection settings \cite{mika13}.
\[
\alpha = \frac{1}{V} \left(\frac{\partial V}{\partial T}\right)_P
\]


%--------------------------------------------------------------------------------------------------
\item {\color{violet} Urey Ratio}: mantle heat production divided by heat loss. It is a key constraint 
for thermal history models. Recent Urey ratio estimates are in the range of 0.21-0.49. \cite{lecm11}

%--------------------------------------------------------------------------------------------------
\item {\color{violet} Shear modulus}: modulus of rigidity, usually expressed in \si{\giga\pascal}. 
It describes the material response to shear stress.

%--------------------------------------------------------------------------------------------------
\item {\color{violet} Poisson ratio}: response in the direction orthogonal to uniaxial stress.

%--------------------------------------------------------------------------------------------------
\item {\color{violet} Young's modulus}: describes the material strain response to uniaxial stress
in the direction of this stress, usually expressed in \si{\giga\pascal}.

%--------------------------------------------------------------------------------------------------
\item {\color{violet} Average viscosity}: following Christensen (1983) \cite{chri83}, 
one can compute the averaged viscosity in a domain as follows:
\begin{equation}
\langle \eta \rangle = \frac{\int_V \eta \dot{\varepsilon}_e^2 dV}{\int_V  \dot{\varepsilon}_e^2 dV }
\label{eq:avrgeta}
\end{equation}

\end{itemize}


\todo[inline]{check aspect manual The 2D cylindrical shell benchmarks by Davies \etal 5.4.12}


