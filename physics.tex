
\begin{center}
\begin{tabular}{lll}
\hline
Symbol & meaning & unit \\
\hline
\hline
$t$ & Time & s \\
$x,y,z$ & Cartesian coordinates & m \\
${\bm v}$ & velocity vector & m$\cdot$ s$^{-1}$\\
$\rho$ & mass density & kg/m$^3$ \\
$\eta$ & dynamic viscosity &  Pa$\cdot$ s \\
$\lambda$ & penalty parameter & Pa$\cdot$ s \\
$T$ & temperature & K \\
${\bm \nabla}$ & gradient operator & m$^{-1}$ \\
${\bm \nabla}\cdot$ & divergence operator & m$^{-1}$ \\
$p$ & pressure & Pa\\
$\dot{\bm \varepsilon}({\bm v})$ & strain rate tensor & s$^{-1}$ \\
$\alpha$ & thermal expansion coefficient & K$^{-1}$ \\
$k$ & thermal conductivity & W/(m $\cdot$ K) \\
$C_p$ & Heat capacity & J/K \\
$H$ & intrinsic specific heat production & W/kg\\
$\beta_T$ & isothermal compressibility & Pa$^{-1}$  \\
${\bm \tau}$ & deviatoric stress tensor & Pa \\
${\bm \sigma}$ & full stress tensor & Pa \\
\hline
\end{tabular}
\end{center}

%------------------------------------------------------------------------
\subsection{The heat transport equation - energy conservation equation}

Let us start from the heat transport equation as shown in Schubert, Turcotte and Olson \cite{scto01}:
\[
\rho C_p \frac{DT}{Dt} - \alpha T \frac{Dp}{Dt} = {\bm \nabla} \cdot k {\bm \nabla} T + \Phi + \rho H  
\]
with $D/Dt$ being the total derivatives so that 
\[
\frac{DT}{Dt} = \frac{\partial T}{\partial t} + {\bm v}\cdot {\bm \nabla}T
\quad\quad
\frac{Dp}{Dt} = \frac{\partial p}{\partial t} + {\bm v}\cdot {\bm \nabla}p
\]
Solving for temperature, this equation is often rewritten as follows:
\begin{mdframed}[backgroundcolor=blue!5]
\[
\rho C_p \frac{DT}{Dt} - {\bm \nabla} \cdot k {\bm \nabla} T =  \alpha T \frac{Dp}{Dt} + \Phi + \rho H  
\]
\end{mdframed}

A note on the shear heating term $\Phi$: In many publications, $\Phi$ 
is given by $\Phi=\tau_{ij}\partial_j u_i={\bm \tau}:{\bm \nabla}{\bm v}$.

\begin{eqnarray}
\Phi 
&=& \tau_{ij}\partial_j u_i \nonumber\\
&=& 2 \eta \dot{\varepsilon}_{ij}^d\partial_j u_i \nonumber\\
&=& 2 \eta \frac{1}{2}\left( \dot{\varepsilon}_{ij}^d\partial_j u_i + \dot{\varepsilon}_{ji}^d\partial_i u_j \right) \nonumber\\
&=& 2 \eta \frac{1}{2}\left( \dot{\varepsilon}_{ij}^d\partial_j u_i + \dot{\varepsilon}_{ij}^d\partial_i u_j \right) \nonumber\\
&=& 2 \eta  \dot{\varepsilon}_{ij}^d  \frac{1}{2}\left(\partial_j u_i + \partial_i u_j \right) \nonumber\\
&=& 2 \eta  \dot{\varepsilon}_{ij}^d   \dot{\varepsilon}_{ij} \nonumber\\
&=& 2 \eta  \dot{\bm \varepsilon}^d :  \dot{\bm \varepsilon} \nonumber\\
&=& 2 \eta  \dot{\bm \varepsilon}^d : \left( \dot{\bm \varepsilon}^d +\frac{1}{3} ({\bm \nabla}\cdot{\bm v}) {\bm 1} \right)\nonumber\\
&=& 2 \eta  \dot{\bm \varepsilon}^d : \dot{\bm \varepsilon}^d 
+ 2 \eta  \dot{\bm \varepsilon}^d : {\bm 1} ({\bm \nabla}\cdot{\bm v}) \nonumber\\ 
&=& 2 \eta  \dot{\bm \varepsilon}^d : \dot{\bm \varepsilon}^d 
\end{eqnarray}
Finally
\[
\Phi = {\bm \tau}:{\bm \nabla}{\bm v} = 2 \eta  \dot{\bm \varepsilon}^d : \dot{\bm \varepsilon}^d
= 2 \eta \left( (\dot{\varepsilon}_{xx}^d)^2 + (\dot{\varepsilon}_{yy}^d)^2 + 2(\dot{\varepsilon}_{xy}^d)^2 \right)
\]

%------------------------------------------------------------------------
\subsection{The momentum conservation equations} 

Because the Prandlt number is virtually zero in Earth science applications the Navier Stokes 
equations reduce to the Stokes equation:
\[
{\bm \nabla}\cdot {\bm \sigma} + \rho {\bm g} = 0
\]
Since 
\[
{\bm \sigma} = -p {\bm 1} + {\bm \tau}
\]
it also writes
\[
-{\bm \nabla}p + {\bm \nabla}\cdot {\bm \tau} + \rho {\bm g} = 0
\]
Using the relationship ${\bm \tau} = 2 \eta \dot{\bm \varepsilon}^d$ we arrive at 
\begin{mdframed}[backgroundcolor=blue!5]
\[
-{\bm \nabla}p + {\bm \nabla}\cdot (2 \eta \dot{\bm \varepsilon}^d ) + \rho {\bm g} = 0
\]
\end{mdframed}

%------------------------------------------------------------------------
\subsection{The mass conservation equations} 

The mass conservation equation is given by
\[
\frac{D\rho}{Dt} + \rho {\bm \nabla}\cdot{\bm v} = 0
\]
or, 
\begin{mdframed}[backgroundcolor=blue!5]
\[
\frac{\partial \rho}{\partial t} + {\bm \nabla}\cdot(\rho {\bm v}) = 0
\]
\end{mdframed}
In the case of an incompressible flow, then $\partial \rho/\partial t=0$ and 
${\bm \nabla}\rho=0$, i.e. $D\rho/Dt=0$ and the remaining equation is simply:
\[
{\bm \nabla}\cdot{\bm v} = 0
\]

\subsection{The equations in ASPECT manual}
The following is lifted off the ASPECT manual.
We focus on the system of equations in a $d=2$- or $d=3$-dimensional
domain $\Omega$ that describes the motion of a highly viscous fluid driven
by differences in the gravitational force due to a density that depends on
the temperature. In the following, we largely follow the exposition of this
material in Schubert, Turcotte and Olson \cite{scto01}.

Specifically, we consider the following set of equations for velocity $\mathbf
u$, pressure $p$ and temperature $T$:
\begin{align}
  \label{eq:stokes-1}
  -\nabla \cdot \left[2\eta \left(\dot\varepsilon(\bm v)
                                  - \frac{1}{3}(\nabla \cdot \bm v)\mathbf 1\right)
                \right] + \nabla p &=
  \rho \bm g
  &
  & \textrm{in $\Omega$},
  \\
  \label{eq:stokes-2}
  \nabla \cdot (\rho \bm v) &= 0
  &
  & \textrm{in $\Omega$},
  \\
  \label{eq:temperature}
  \rho C_p \left(\frac{\partial T}{\partial t} + \bm v\cdot\nabla T\right)
  - \nabla\cdot k\nabla T
  &=
  \rho H
  \notag
  \\
  &\quad
  +
  2\eta
  \left(\dot\varepsilon(\bm v) - \frac{1}{3}(\nabla \cdot \bm v)\mathbf 1\right)
  :
  \left(\dot\varepsilon(\bm v) - \frac{1}{3}(\nabla \cdot \bm v)\mathbf 1\right)
  \\
  &\quad
  +\alpha T \left( \bm v \cdot \nabla p \right)
  \notag
  \\
  &\quad
  &
  & \textrm{in $\Omega$},
  \notag
\end{align}
where $\dot{\bm \varepsilon}(\mathbf u) = \frac{1}{2}(\nabla \mathbf u + \nabla\mathbf
u^T)$ is the symmetric gradient of the velocity (often called the
\textit{strain rate}).%

In this set of equations, \eqref{eq:stokes-1} and \eqref{eq:stokes-2}
represent the compressible Stokes equations in which $\mathbf v=\mathbf
v(\mathbf x,t)$ is the velocity field and $p=p(\mathbf x,t)$ the pressure
field. Both fields depend on space $\mathbf x$ and time $t$. Fluid flow is
driven by the gravity force that acts on the fluid and that is proportional to
both the density of the fluid and the strength of the gravitational pull.

Coupled to this Stokes system is equation \eqref{eq:temperature} for the
temperature field $T=T(\mathbf x,t)$ that contains heat conduction terms as
well as advection with the flow velocity $\mathbf v$. The right hand side
terms of this equation correspond to
\begin{itemize}
\item internal heat production for example due to radioactive decay;
\item friction (shear) heating;
\item adiabatic compression of material;
\end{itemize}

In order to arrive at the set of equations that ASPECT solves, 
we need to 
\begin{itemize}
\item neglect the $\partial p/\partial t$. {\color{red}WHY?}
\item neglect the $\partial \rho / \partial t$ . {\color{red}WHY?}
\end{itemize}
from equations above. 

----------------------------------------

Also, their definition of the shear heating term $\Phi$ is:
\[
\Phi = k_B ({\bm \nabla}\cdot{\bm v})^2 + 2\eta \dot{\bm \varepsilon}^d:\dot{\bm \varepsilon}^d
\]
For many fluids the bulk viscosity $k_B$ is very small and is often taken to be zero, an assumption known
as the Stokes assumption: $k_B=\lambda+2\eta/3=0$. \index{bulk viscosity}
Note that $\eta$ is the dynamic viscosity and $\lambda$ the second viscosity. \index{dynamic viscosity}
\index{second viscosity}
Also, 
\[
{\bm \tau}=2\eta \dot{\bm \varepsilon} + \lambda ({\bm \nabla}\cdot{\bm v}) {\bm 1}
\]
but since $k_B=\lambda+2\eta/3=0$, then $\lambda=-2\eta/3$ so 
\[
{\bm \tau}=2\eta \dot{\bm \varepsilon} -\frac{2}{3}\eta ({\bm \nabla}\cdot{\bm v}) {\bm 1} = 2\eta \dot{\bm \varepsilon}^d
\]







\newpage
%---------------------------------
\subsection{the Boussinesq approximation: an Incompressible flow}

\index{Boussinesq}

[from aspect manual]
The Boussinesq approximation assumes that the density can be
considered constant in all occurrences in the equations with the exception of
the buoyancy term on the right hand side of \eqref{eq:stokes-1}. The primary
result of this assumption is that the continuity equation \eqref{eq:stokes-2}
will now read
\[
{\bm \nabla}\cdot{\bm v} = 0
\]
This implies that the strain rate tensor is deviatoric.
Under the Boussinesq approximation, the equations are much simplified:

\begin{align}
  \label{eq:stokes-1}
  -\nabla \cdot \left[2\eta \dot{\bm \varepsilon}(\bm v)
                \right] + \nabla p &=
  \rho \bm g
  &
  & \textrm{in $\Omega$},
  \\
  \label{eq:stokes-2}
  \nabla \cdot (\rho \bm v) &= 0
  &
  & \textrm{in $\Omega$},
  \\
  \label{eq:temperature}
  \rho_0 C_p \left(\frac{\partial T}{\partial t} + \bm v\cdot\nabla T\right)
  - \nabla\cdot k\nabla T
  &=
  \rho H
  &
  & \textrm{in $\Omega$}
\end{align}
Note that all terms on the rhs of the temperature equations have disappeared, with the exception 
of the source term.


\newpage
\subsection{Stokes equation for elastic medium}

What follows is mostly borrowed from Becker \& Kaus lecture notes.

%\begin{tabular}{|l|l|l|}
%\hline
%${\bm u}       $ & displacement vector &   \\
%${\bm \sigma}  $ & full stress tensor  & Pa\\
%${\bm \epsilon}$ & strain tensor       &   \\
%${\bm 1}       $ & unit tensor         &   \\
%${\bm f}       $ & body forces         &   \\
%\hline
%\end{tabular}

The strong form of the PDE that governs force balance in a medium is given by
\[
{\bm \nabla}\cdot{\bm \sigma}  + {\bm f} = {\bm 0}
\]
where ${\bm \sigma}$ is the stress tensor and ${\bm f}$ is a body force.

The stress tensor is related to the strain tensor through the generalised 
Hooke's law:
\begin{equation}
\sigma_{ij}=\sum_{kl}C_{ijkl}\epsilon{kl} \label{eq:one}
\end{equation}
where ${\bm C}$ is the fourth-order elastic tensor.
In the case of an isotropic material, this relationship simplifies to
\begin{equation}
\sigma_{ij}=\lambda \epsilon_{kk} \delta_{ij} + 2\mu \epsilon_{ij}
\quad\quad
or, 
\quad\quad
{\bm \sigma} = \lambda ({\bm \nabla}\cdot{\bm u})  {\bm 1} + 2\mu {\bm \epsilon}   \label{eq:two}
\end{equation}
where $\lambda$ is the Lam\'e parameter and $\mu$ is the shear modulus\footnote{It is also sometimes written $G$}.
The term ${\bm \nabla}\cdot{\bm u}$ is the isotropic dilation.

\index{Lam\'e parameter} \index{shear modulus}

The strain tensor is related to the displacement as follows: \index{strain tensor}
\[
{\bm \epsilon} = \frac{1}{2}({\bm \nabla}{\bm u} + {\bm \nabla}{\bm u}^T)
\]

The incompressibility (bulk modulus), $K$, is defined as $p=-K {\bm \nabla}\cdot{\bm u}$ 
where $p$ is the pressure with \index{bulk modulus}
\begin{eqnarray}
p&=&-\frac{1}{3}Tr({\bm \sigma}) \nonumber\\
 &=& -\frac{1}{3} [ \lambda ({\bm \nabla}\cdot{\bm u}) Tr[{\bm 1}] + 2 \mu Tr[{\bm \epsilon}]] \nonumber\\
 &=& -\frac{1}{3} [ \lambda ({\bm \nabla}\cdot{\bm u})  3  + 2 \mu  ({\bm \nabla}\cdot{\bm u}) ] \nonumber\\
 &=& -[ \lambda  + \frac{2}{3} \mu ]   ({\bm \nabla}\cdot{\bm u})  
\end{eqnarray}
so that $K=\lambda+\frac{2}{3}\mu$.

%or
%\[
%\mu=\frac{3K(1-2\nu)}{2(1+\nu)}
%\]


\paragraph{Remark}: Eq. (\ref{eq:one}) and (\ref{eq:two}) are analogous to the ones that one has to solve
in the context of viscous flow using the penalty method. In this case $\lambda$ is the penalty coefficient, 
${\bm u}$ is the velocity, and $\mu$ is then the dynamic viscosity.

%\begin{center}
%\includegraphics[width=15cm]{images/coeffs}\\
%{\small Homogeneous isotropic linear elastic materials have their elastic properties uniquely determined by any two moduli among these; thus, given any two, any other of the elastic moduli can be calculated according to these formulas.}
%\end{center}

The Lam\'e parameter and the shear modulus are also linked to $\nu$ the poisson ratio, 
and $E$, Young's modulus: \index{Poisson ratio} \index{Young's modulus}
\[
\lambda=\mu\frac{2\nu}{1-2\nu}
=\frac{\nu E}{(1+\nu)(1-2\nu)}
\quad\quad
{\rm with}
\quad\quad
E=2\mu(1+\nu)
\]
The shear modulus, expressed often in GPa, describes the material's response to shear stress.
The poisson ratio describes the response in the direction orthogonal to uniaxial stress.
The Young modulus, expressed in GPa, describes the material's strain response to uniaxial stress in the 
direction of this stress.









