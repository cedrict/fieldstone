This manufactured solution originates in \textcite{bocg12} (2012).
The velocity field turns out to be identical to the \textcite{dohu03} manufactured solution.
It is based on the stream function 
\[
\psi (x,y)=x^2(x-1)^2y^2(y-1)^2 =f(x)g(y)
\]
defined on the unit square. We have 
\begin{eqnarray}
f'(x)
&=&2x(x-1)^2+2x^2(x-1) \nn\\
&=&2(x-1)[x(x-1)+x^2]  \nn\\
&=&2(x-1)(2x^2 -x)  \nn\\
&=&2x(x-1)(2x -1)  \nn\\
f''(x)
&=& 2 [ (x-1)(2x -1) + x(2x-1)+2x(x-1)] \nn\\
&=& 2 (2x^2 -x-2x+1 + 2x^2-x+2x^2-2x) \nn\\
&=& 2 (6x^2 -6x +1) \nn\\
g'(y)
&=& 2y (y-1)^2 + 2y^2(y-1)  \nn\\
&=& 2(y-1)[y(y-1)+y^2] \nn\\
&=& 2(y-1)(2y^2-y) \nn\\
&=& 2y(y-1)(2y-1) \nn\\
g''(y) 
&=& 2[ (y-1)(2y-1) +y(2y-1)+2y(y-1) ] \nn\\
&=& 2[ 2y^2 - 3y +1 +2y^2-y +2y^2-2y] \nn\\
&=& 2(6y^2 -6y+1)
\end{eqnarray}

The manufactured 'smooth pressure' solution is then
\begin{eqnarray}
u(x,y) &=& \partial_y \psi = f(x)g'(y) = x^2(x-1)^2 2y(y-1)(2y-1) \nn\\
v(x,y) &=& -\partial_x \psi = -f'(x)g(y) = -2x(x-1)(2x -1)y^2(y-1)^2 \nn\\
p(x,y) &=& \frac12 x^2 -\frac16 \nn
\end{eqnarray}

\begin{eqnarray}
\partial_x u (x,y)&=& f'(x)g'(y)\nn\\
\partial_y u (x,y)&=& f(x)g''(y)\nn\\
\partial_x v (x,y)&=& -f''(x)g(y)\nn\\
\partial_y v (x,y)&=& -f'(x)g'(y)\nn
\end{eqnarray}

\begin{eqnarray}
\dot\varepsilon_{xx}(x,y) &=& f'(x)g'(y) \nn\\
\dot\varepsilon_{yy}(x,y) &=&  -f'(x)g'(y) \nn\\
\dot\varepsilon_{xy}(x,y) &=& \frac12(f(x)g''(y)-f''(x)g(y)) \nn
\end{eqnarray}

\begin{eqnarray}
\frac{\partial p}{\partial x} (x,y)&=& x\nn\\
\frac{\partial p}{\partial y} (x,y)&=& 0\nn
\end{eqnarray}

\begin{eqnarray}
\partial_x \dot\varepsilon_{xx} (x,y)&=&  f''(x)g'(y) \nn\\
\partial_x \dot\varepsilon_{xy} (x,y)&=& \frac12\left(f'(x)g''(y)-f'''(x)g(y)\right)\nn\\
\partial_y \dot\varepsilon_{xy} (x,y)&=& \frac12\left(f(x)g'''(y)-f''(x)g'(y) \right)\nn\\
\partial_y \dot\varepsilon_{yy} (x,y)&=& -f'(x)g''(y) \nn
\end{eqnarray}

\begin{eqnarray}
\upnu_{rms}
&=& \sqrt{ \frac{1}{L_xL_y} \iint (u^2+v^2) dxdy   } \nn\\
&=& \sqrt{ \frac{1}{L_xL_y} \iint (f^2g'^2+f'^2g^2) dxdy   } \nn\\
&=& \sqrt{ \left( 
\underbrace{\int_0^1 f^2 dx}_{1/630} 
\underbrace{\int_0^1 g'^2 dy}_{2/105}
+ 
\underbrace{\int_0^1 f'^2 dx }_{2/105}
\underbrace{\int_0^1g^2dy}_{1/630} \right) dxdy} \nn\\
1/33075 * 2
&=& \frac{1}{105}\sqrt{\frac23} \nn\\
&\simeq&  0.007776157913597390787927885951... 
\end{eqnarray}

The authors also define a non-smooth pressure case:
\[
p(x,y)=
\left\{
\begin{array}{lll}
y(1-y)\exp(x-1/2)^2+1/2 & \text{for}& x\ge 1/2 \\
y(1-y)\exp(x-1/2)^2-1/2 & \text{for}& x< 1/2 \\
\end{array}
\right.
\]
with 
\begin{eqnarray}
\frac{\partial p}{\partial x} (x,y)&=& 2(x-1/2)y(1-y)\exp(x-1/2)^2 \nn\\
\frac{\partial p}{\partial y} (x,y)&=& (1-2y) \exp(x-1/2)^2 \nn
\end{eqnarray}
However it is not clear to me how to implement this since only the gradient of the pressure appears 
in the equations. 



