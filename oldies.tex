
The first papers I could find showcasing the FEM in geodynamics are listed hereafter
\cite{gart78}, 
\cite{anbr80}\cite{mera80}\cite{bran80}
\cite{engl82}
\cite{thar85}\cite{scan85}
\cite{enho86}\cite{mofr86}
\cite{zupa86}
\cite{boww89}
\cite{brau94}
\cite{brbe95}.
I hereunder show a few plots taken from early geodynamics papers.


\begin{center}
\begin{minipage}{0.45\textwidth}
\centering
\includegraphics[height=4.5cm]{images/history/stbe71}\\
{\captionfont 1971: Model a boudinage structure \cite{stbe71}}
\end{minipage}\hfill
\begin{minipage}{0.45\textwidth}
\centering
\includegraphics[height=4.5cm]{images/history/bela72}\\
{\captionfont 1972: Crustal Structure from Surface Load Tilts \cite{bela72}}
\end{minipage}
\end{center}

\begin{center}
\begin{minipage}{0.45\textwidth}
\centering
\includegraphics[height=5cm]{images/history/bird78b}\\
{\captionfont 1978: Finite element modelling of lithosphere deformation: the Zagros collision 
orogeny \cite{bird78b}}
\end{minipage}\hfill
\begin{minipage}{0.45\textwidth}
\centering
\includegraphics[height=5cm]{images/history/brpo81}\\
{\captionfont 1981: Thermal regimes, mantle diapirs and crustal stresses of continental rifts \cite{brpo81}}
\end{minipage}
\end{center}


\begin{center}
\begin{minipage}{0.48\textwidth}
\centering
\includegraphics[height=3.5cm]{images/history/baum85a}
\includegraphics[height=3.5cm]{images/history/baum85b}\\
{\captionfont 1985: Three-Dimensional Treatment of Convective Flow in the Earth's Mantle.
\cite{baum85}}
\end{minipage}\hfill
\begin{minipage}{0.45\textwidth}
\centering
\includegraphics[width=7cm]{images/history/zupf86}\\
{\captionfont 1986: Lithospheric necking: a dynamic model for rift morphology \cite{zupf86}}
\end{minipage}
\end{center}


\begin{center}
\begin{minipage}{0.48\textwidth}
\centering
\includegraphics[width=9cm]{images/history/boww89}\\
{\captionfont 1989: Plate boundary forces at subduction zones and trench-arc compression \cite{boww89}}
\end{minipage}\hfill
\begin{minipage}{0.45\textwidth}
\includegraphics[width=9cm]{images/history/brbe89}\\
{\captionfont 1989: Relation between flank uplifts and the breakup unconformity at rifted continental margins \cite{brbe89}}
\end{minipage}
\end{center}



\begin{center}
\begin{minipage}{0.35\textwidth}
\centering
\includegraphics[width=5cm]{images/history/mewi89}\\
{\captionfont 1989: Mechanics of graben formation in crustal rocks \cite{mewi89}}
\end{minipage}\hfill
\begin{minipage}{0.55\textwidth}
\centering
\includegraphics[width=8cm]{images/history/whbw92}\\
{\captionfont 1992: Stresses and plate boundary forces associated with subduction plate margins
\cite{whbw92}}
\end{minipage}
\end{center}


\begin{center}
\includegraphics[width=5cm]{images/history/dast92}\\
{\captionfont 1992: Temperature field in subduction zones \cite{dast92}}
\end{center}


\begin{center}
\begin{minipage}{0.45\textwidth}
\centering
\includegraphics[width=7.6cm]{images/history/brau93}\\
{\captionfont 1993: 3D numerical modeling of
compressional orogenies: Thrust geometry and
oblique convergence \cite{brau93}}
\end{minipage}\hfill
\begin{minipage}{0.45\textwidth}
\centering
\includegraphics[height=6cm]{images/history/bequ94}\\
{\captionfont 1994: Crustal-scale compressional orogens \cite{bequ94}}
\end{minipage}
\end{center}

\begin{center}
\includegraphics[height=6cm]{images/history/katl95}\\
{\captionfont 1995: modeling of pull-apart basins \cite{katl95}}
\end{center}


\begin{center}
\begin{minipage}{0.45\textwidth}
\centering
\includegraphics[height=5cm]{images/history/yowo95}\\
{\captionfont 1995: 3D numerical modeling of detachment of subducted 
lithosphere \cite{yowo95}}
\end{minipage}\hfill
\begin{minipage}{0.45\textwidth}
\centering
\includegraphics[height=6cm]{images/history/dusa96}\\
{\captionfont 1996: 3D dynamical model of continental rift propagation and 
margin plateau formation \cite{dusa96}}
\end{minipage}
\end{center}



\begin{center}
\includegraphics[height=6cm]{images/history/nesb99}\\
{\captionfont 1989: Model geometry, boundary conditions and 3-D finite element mesh used in 
the calculations. The circles denote a free-slip condition. The arrow denotes the velocity 
applied in some calculations to the southern boundary of the Tyrrhenian domain to simulate 
the motion of the African plate. The springs represent the buoyant restoring force applied 
at the surface. \cite{nesb99}}
\end{center}




