%\section{Celestical bodies}


%....................................
\section{Mercury}

\begin{small}
\begin{itemize}
\item[\twothousandseven] 
\fullcite{reki07} 
\item[\twothousandeight] 
\fullcite{king08} 
\item[\twothousandtwelve] 
\fullcite{roba12} 
\item[\twothousandtwentyone] 
\fullcite{gult21} 
\item[\twothousandtwentytwo] 
\fullcite{xihz22} 
\end{itemize}
\end{small}

 
%....................................
\section{Venus}

\begin{small}
\begin{itemize}
\item[\nineteensixtynine]
\fullcite{scto69} 
\item[\nineteenninety] 
\fullcite{scbg90}\\ 
\fullcite{sozh90} 
\item[\nineteenninetyone] 
\fullcite{lekb91} \\
\fullcite{leyu91} 
\item[\nineteenninetytwo] 
\fullcite{kiha92} \\
\fullcite{sqjs92} \\
\fullcite{mcfj92} 
\item[\nineteenninetythree] 
\fullcite{kief93} \\
\fullcite{lekb93} \\
\fullcite{ogaw93} 
\item[\nineteenninetyfive] 
\fullcite{lekb95} \\
\fullcite{kaul95} \\
\fullcite{mopa95}  \\
\fullcite{scsa95} 
\item[\nineteenninetysix] 
\fullcite{somo96} \\ 
\fullcite{foob96} 
\item[\nineteenninetyseven] 
\fullcite{mang97} 
\item[\nineteenninetyeight] 
\fullcite{mazk98}  \\
\fullcite{resm98}  \\
\fullcite{moso98}  \\
\fullcite{phha98} 
\item[\nineteenninetynine] 
\fullcite{resm99} 
\item[\twothousand] 
\fullcite{ogaw00} 
\item[\twothousandthree] 
\fullcite{vesh03} 
\item[\twothousandfour] 
\fullcite{vesb04} 
\item[\twothousandfive] 
\fullcite{vavv05} 
\item[\twothousandseven] 
\fullcite{reso07} 
\item[\twothousandten] 
\fullcite{stfh10}  \\
\fullcite{stwt10} 
\item[\twothousandeleven] 
\fullcite{orso11} 
\item[\twothousandtwelve] 
\fullcite{arta12}  \\
\fullcite{orso12}  \\
\fullcite{nobs12} 
\item[\twothousandthirteen] 
\fullcite{huyz13} \\ 
\fullcite{jazp13} 
\item[\twothousandfourteen] 
\fullcite{gita14}  \\
\fullcite{gery14b} 
\item[\twothousandfifteen] 
\fullcite{ghai15} 
\item[\twothousandseventeen] 
\fullcite{cram17}  \\
\fullcite{dast17} 
\item[\twothousandeighteen] 
\fullcite{king18}  \\
\fullcite{ross18} 
\item[\twothousandtwenty] 
\fullcite{weki20}  \\
\fullcite{gugm20}  \\
\fullcite{uprc20}  \\
\fullcite{kacc20} 
\item[\twothousandtwentyone] 
\fullcite{macg21}  \\
\fullcite{rodm21}  \\
\fullcite{onei21}  \\
\fullcite{bygs21} 
\item[\twothousandtwentytwo]
\fullcite{adss22}  \\
\fullcite{bamo22}  \\
\fullcite{mawi22}  \\
\fullcite{rowg22} 
\item[\twothousandtwentythree]
\fullcite{smoo23}  \\
\fullcite{lour23}  \\
\fullcite{titl23}  \\
\fullcite{mawp23}  \\
\fullcite{adsm23}  \\
\fullcite{guyg23}  \\
\fullcite{hanm23} 
\item[\twothousandtwentyfour]
\fullcite{caks24} \\
\fullcite{guba24}\\
\fullcite{scsm24}\\
\fullcite{vamp24} 
\item[\twothousandtwentyfive]
\fullcite{weki25} \\
\fullcite{cagm25} \\
\fullcite{sefw25} 
\end{itemize}
\end{small}

%....................................
\section{Moon}

\begin{small}
\begin{itemize}
\item[\nineteenseventy]
\fullcite{tuox70}
\item[\nineteenseventytwo]
\fullcite{tuht72}
\item[\nineteenseventythree]
\fullcite{care73}
\item[\nineteenseventyfour]
\fullcite{care74}
\item[\nineteenseventynine]
\fullcite{carg79}
\item[1998]
\fullcite{alpa98}
\item[\twothousandone] 
\fullcite{spkb01}
\item[\twothousandtwo] 
\fullcite{elvh02} 
\item[\twothousandthree] 
\fullcite{stjz03} 
\item[\twothousandfour] 
\fullcite{elhg04} 
\item[\twothousandten] 
\fullcite{devv10} 
\item[\twothousandtwelve] 
\fullcite{zhqa12} 
\item[\twothousandthirteen] 
\fullcite{dejv13} 
\item[\twothousandsixteen] 
\fullcite{qizw16} 
\item[\twothousandseventeen] 
\fullcite{jaal17} 
\item[\twothousandeighteen] 
\fullcite{qizp18} 
\item[\twothousandnineteen] 
\fullcite{zhdv19} 
\item[\twothousandtwentytwo]
\fullcite{javs22} \\ 
\fullcite{faab22} 
\item[\twothousandtwentythree]
\fullcite{zhzl23} \\
\fullcite{yuld23} \\
\fullcite{ukog23}
\item[\twothousandtwentyfour]
\fullcite{fizm24}
\end{itemize}
\end{small}





%....................................
\section{Mars}

Mars fact sheet: \url{https://nssdc.gsfc.nasa.gov/planetary/factsheet/marsfact.html}

\begin{small}
\begin{itemize}
\item[\nineteensixtynine]
\fullcite{scto69} 
\item[\nineteeneightytwo] 
\fullcite{baps82}  \\
\fullcite{witu82}  \\
\fullcite{sohe82} 
\item[\nineteenninety] 
\fullcite{scbg90}  \\
\fullcite{thsc90} 
\item[\nineteenninetyone] 
\fullcite{spoh91}  \\
\fullcite{jaer91} 
\item[\nineteenninetyfour] 
\fullcite{slee94}
\item[\nineteenninetysix] 
\fullcite{hach96}  \\
\fullcite{brzy96}  \\
\fullcite{kibn96}  \\
\fullcite{mema96} 
\item[\nineteenninetyseven]  
\fullcite{brys97} 
\item[\nineteenninetyeight] 
\fullcite{resm98}  \\
\fullcite{hard98}  \\
\fullcite{befe98}  \\
\fullcite{wuha98}  \\
\fullcite{brys98}  
\item[\nineteenninetynine] 
\fullcite{smst99} 
\item[\twothousand] 
\fullcite{hard00} 
\item[\twothousandone] 
\fullcite{nist01}  \\
\fullcite{zube01}  \\
\fullcite{scvy01} 
\item[\twothousandtwo] 
\fullcite{resb02}  \\
\fullcite{zhon02}  \\
\fullcite{haph02}  \\
\fullcite{mcby02}  \\
\fullcite{scvy02} 
\item[\twothousandthree] 
\fullcite{zhro03}  \\
\fullcite{lozh03}  \\
\fullcite{kief03} 
\item[\twothousandfour] 
\fullcite{lenm04}  \\
\fullcite{vavv04c}  \\
\fullcite{resb04}  \\
\fullcite{reki04}  \\
\fullcite{rozh04} 
\item[\twothousandfive]  
\fullcite{vavv05}  \\
\fullcite{elzp05}  \\
\fullcite{onml05}  \\
\fullcite{belw05} 
\item[\twothousandsix] 
\fullcite{reso06}  \\
\fullcite{losh06}  \\
\fullcite{rozh06}  \\
\fullcite{keso06}  \\
\fullcite{koys06}  \\
\fullcite{brsp06} 
\item[\twothousandseven]
\fullcite{rozh07}  \\
\fullcite{reso07b} \\
\fullcite{onlj07} \\
\fullcite{liki07} 
\item[\twothousandeight] 
\fullcite{loha08}  \\
\fullcite{winm08} 
\item[\twothousandnine]
\fullcite{keta09}  \\
\fullcite{zhon09}  \\
\fullcite{rolm09}  \\
\fullcite{keso09}  \\
\fullcite{smzt09}  \\
\fullcite{habg09} 
\item[\twothousandten] 
\fullcite{srzh10}  \\
\fullcite{reos10}  \\
\fullcite{reso10}  \\
\fullcite{stwt10}  \\
\fullcite{wabh10}  \\
\fullcite{grbr10} 
\item[\twothousandeleven] 
\fullcite{gokg11}  \\
\fullcite{reos11}  \\
\fullcite{jizl11}  \\
\fullcite{koaf11}  \\
\fullcite{nasc11} 
\item[\twothousandtwelve] 
\fullcite{srzh12}  \\
\fullcite{roar12}  \\
\fullcite{hick12}  \\
\fullcite{belr12} 
\item[\twothousandthirteen] 
\fullcite{pltb13}  \\
\fullcite{ruts13}  \\
\fullcite{ruts13b} 
\item[\twothousandfourteen] 
\fullcite{seki14} \\
\fullcite{chki14} \\
\fullcite{letg14}
\item[\twothousandfifteen] 
\fullcite{kifs15} 
\item[\twothousandsixteen] 
\fullcite{zhon16}  \\
\fullcite{kili16}  \\
\fullcite{gegl16}  \\
\fullcite{bobm16} 
\item[\twothousandseventeen] 
\fullcite{rubr17}  \\
\fullcite{hema17}  \\
\fullcite{azka17} 
\item[\twothousandeighteen] 
\fullcite{cimt18}  \\
\fullcite{goej18}  \\
\fullcite{scmo18}  \\
\fullcite{khlr18}  \\
\fullcite{domk18}  \\
\fullcite{plpt18} 
\item[\twothousandnineteen] 
\fullcite{smls19}  \\
\fullcite{cahe19}  \\
\fullcite{bagz19}  \\
\fullcite{dilg19} 
\item[\twothousandtwenty] 
\fullcite{lobp20}  \\
\fullcite{gilb20}  \\
\fullcite{agtb20}  \\
\fullcite{geno20}  \\
\fullcite{basb20}  \\
\fullcite{tajh20}  \\
\fullcite{brfi20} 
\item[\twothousandtwentyone] 
\fullcite{khcv21}  \\
\fullcite{stkb21}  \\
\fullcite{knpb21}  \\
\fullcite{vand21}  \\
\fullcite{ribc21}  \\
\fullcite{topa21}  \\
\fullcite{sabp21} 
\item[\twothousandtwentytwo]
\fullcite{wibm22}  \\
\fullcite{watk22}  \\
\fullcite{plwk22}  \\
\fullcite{bran22} 
\item[\twothousandtwentythree]
\fullcite{bajg23} \\
\fullcite{khhd23} \\
\fullcite{sadr23}
\item[\twothousandtwentyfour]
\fullcite{muki24}\\
\fullcite{chrg24}\\
\fullcite{guba24}\\
\fullcite{drsg24}
\item[\twothousandtwentyfive]
\fullcite{brpk25}
\end{itemize}
\end{small}



%....................................
\section{Pluto}

\begin{small}
\begin{itemize}
\item[2016] \fullcite{mcnw16} 
\item[2021] \fullcite{ohdo21} 
\end{itemize}
\end{small}

%....................................
\section{Super-Earths, Giant planets \& exoplanets}

\begin{small}
\begin{itemize}
\item[\twothousandsix]
\fullcite{evgl06}
\item[\twothousandeleven]
\fullcite{stfl11}  \\
\fullcite{vata11} 
\item[\twothousandtwelve]
\fullcite{evsa12} 
\item[\twothousandthirteen]
\fullcite{stlh13} 
\item[\twothousandfifteen] 
\fullcite{welo15}  \\
\fullcite{miko15}  \\
\fullcite{evon15}  \\
\fullcite{kamo15} 
\item[\twothousandtwentyone]
\fullcite{mebl21} 
\item[\twothousandtwentythree] 
\fullcite{shpy23} 
\end{itemize}
\end{small}

%....................................
\section{Icy satellites, icy moons}

Icy moons are a class of natural satellites with surfaces composed mostly of ice. 
An icy moon may harbor an ocean underneath the surface, and possibly include a rocky 
core of silicate or metallic rocks.
\url{https://en.wikipedia.org/wiki/Icy_moon}

\begin{small}
\begin{itemize}
\item[1987]
\fullcite{thsc87}
\item[1988]
\fullcite{thsq88}
\item[\twothousandone] 
\fullcite{deso01} 
\item[\twothousandtwelve] 
\fullcite{kasc12b} 
\item[\twothousandseventeen] 
\fullcite{chts17} 
\item[\twothousandnineteen] 
\fullcite{wefb19} 
\item[\twothousandtwenty] 
\fullcite{hadc20} 
\item[\twothousandtwentyone]
\fullcite{goju21} \\ 
\fullcite{cawj21}
\item[\twothousandtwentythree]
\fullcite{lelm23} 
\end{itemize}
\end{small}

%..........................................................
\section{Europa}

The Galilean satellites were first seen by the Italian astronomer 
Galileo Galilei in 1610. Io is closest, followed by Europa, Ganymede, 
and Callisto. It has a smooth and bright surface, with a layer of 
water surrounding the mantle of the planet, thought to be 100 kilometers thick.

\begin{small}
\begin{itemize}
\item[1986]
\fullcite{thsc86}
\item[\twothousandtwo] 
\fullcite{husw02} 
\item[\twothousandfour] 
\fullcite{shha04} 
\item[\twothousandfive] 
\fullcite{shha05}\\ 
\fullcite{hash05}\\ 
\fullcite{mish05} 
\item[\twothousandeight] 
\fullcite{hash08} 
\item[\twothousandten] 
\fullcite{hash10} 
\item[\twothousandeleven] 
\fullcite{hash11} 
\item[\twothousandfourteen] 
\fullcite{kast14} \\
\fullcite{awzh14} 
\item[\twothousandnineteen] 
\fullcite{almc19} 
\item[\twothousandtwentyone] 
\fullcite{cawj21}
\item[\twothousandtwentytwo] 
\fullcite{wohw22b}
\end{itemize}
\end{small}



%....................................
\section{Ceres}

\url{https://en.wikipedia.org/wiki/Ceres_(dwarf_planet)}
The robotic NASA spacecraft Dawn approached Ceres for its orbital mission in 2015.
and found Ceres's surface to be a mixture of water ice, and hydrated minerals such as carbonates and clay. 

\begin{small}
\begin{itemize}
\item[\twothousandtwentytwo] 
\fullcite{kibm22} 
\end{itemize}
\end{small}

%....................................
\section{Enceladus}

Enceladus is the sixth-largest moon of Saturn (19th largest in the Solar System). 
It is about 500 kilometers in diameter, about a tenth of that of Saturn's largest moon, Titan. 
Enceladus is mostly covered by fresh, clean ice, making it one of the most reflective bodies 
of the Solar System. 
\url{https://en.wikipedia.org/wiki/Enceladus}

\begin{small}
\begin{itemize}
\item[\twothousandeight] 
\fullcite{roni08} 
\item[\twothousandnine]
\fullcite{stfm09} 
\item[\twothousandten]
\fullcite{onni10}\\ 
\fullcite{betc10} 
\item[\twothousandtwelve] 
\fullcite{hats12} 
\item[\twothousandthirteen] 
\fullcite{shhh13} 
\item[\twothousandfourteen]
\fullcite{robg14} 
\item[\twothousandtwentytwo] 
\fullcite{wohw22b}
\end{itemize}
\end{small}

%..........................................................
\section{Callisto}

The Galilean satellites were first seen by the Italian astronomer Galileo Galilei in 1610. 
Io is closest, followed by Europa, Ganymede, and Callisto (1.9 million km or
26.4 $R_J$ from Jupiter). Callisto has the lowest mean density of all Galilean satellites.

\begin{small}
\begin{itemize}
\item[1988]
\fullcite{mumc88} 
\item[\twothousandfour]
\fullcite{nabs04}
\item[\twothousandfive]
\fullcite{kukr05}
\item[\twothousandsix]
\fullcite{free06}
\end{itemize}
\end{small}

%..........................................................
\section{Ganymede}

The Galilean satellites were first seen by the Italian astronomer Galileo Galilei in 1610. 
Io is closest, followed by Europa, Ganymede, and Callisto.

\begin{small}
\begin{itemize}
\item[1988]
\fullcite{mumc88} \\ 
\fullcite{thsc88} 
\item[1990]
\fullcite{thsq90}
\item[\twothousandsix]
\fullcite{free06}
\item[\twothousandfourteen]
\fullcite{awzh14} 
\end{itemize}
\end{small}


%....................................
\section{Io}

The Galilean satellites were first seen by the Italian astronomer Galileo Galilei in 1610. 
Io is closest, followed by Europa, Ganymede, and Callisto.
With a diameter of 3642 kilometers, it is the fourth-largest moon in the Solar System, 
and is only marginally larger than Earth's moon.

\begin{small}
\begin{itemize}
\item[\twothousandone]
\fullcite{tasg01} \\ 
\fullcite{tack01} \\
\fullcite{mcsd01} 
\item[\twothousandthirteen] 
\fullcite{shpp13} 
\item[\twothousandtwenty] 
\fullcite{sthh20} \\ 
\fullcite{spkh20} \\
\fullcite{spkh20b} 
\item[\twothousandtwentytwo] 
\fullcite{ketc22} 
\end{itemize}
\end{small}

%....................................
\section{Planetesimals}

\begin{small}
\begin{itemize}
\item[\twothousandfourteen]
\fullcite{gobg14}
\item[\twothousandnineteen]
\fullcite{likk19} \\
\fullcite{neum19}
\item[\twothousandtwentyone]
\fullcite{goju21} 
\end{itemize}
\end{small}


