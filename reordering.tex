

%--------------------------------------------
\subsubsection*{The need for reordering}


The profile (or envelope) of a symmetric matrix determines how
close its non-zero elements are to the diagonal:
\[
{\rm profile} = \sum_{i=1}^n (i-\min(ne(i)))
\]
The bandwidth is the largest deviation:
\[
{\rm bandwidth} = \max( i - \min (ne(i)))
\]
The cost for a band cholesky factorisation with
bandwidth $p$ is $n(p^2 + 3p)$ flops assuming $n>>p$.  (source?)

In conclusion: reducing bandwidth means a factor solve if Cholesky factorisation is used.


%--------------------------------------------
\subsubsection*{A simple example}


Let us consider a structurally symmetric matrix ${\bm M}$. 
We wish to reduce its bandwidth by permuting rows and columns 
such as to move all the nonzero elements of ${\bm M}$ 
in a band as close as possible to the diagonal.
We then talk about {\sl Bandwidth Reduction}. 
\index{general}{Bandwidth Reduction}

We know that the solution a linear system remains unchanged if lines or columns 
of the matrix (and corresponding rhs) are permuted.

For example\footnote{Taken from 
\url{http://ciprian-zavoianu.blogspot.com/2009/01/project-bandwidth-reduction.html}}
, let us consider the $5\times 5$ matrix ${\bm M}$:
\[
{\bm M}=
\left(
\begin{array}{cccccccc}
1 & . & . & . & 1 & . & . & . \\
. & 1 & 1 & . & . & 1 & . & 1 \\
. & 1 & 1 & . & 1 & . & . & . \\
. & . & . & 1 & . & . & 1 & . \\
1 & . & 1 & . & 1 & . & . & . \\
. & 1 & . & . & . & 1 & . & 1 \\
. & . & . & 1 & . & . & 1 & . \\
. & 1 & . & . & . & 1 & . & 1 
\end{array}
\right)
\]
Simply through row and column permutations it can be rewritten
\[
{\bm M}'=
\left(
\begin{array}{cccccccc}
1 & 1 & . & . & . & . & . & . \\
1 & 1 & . & . & . & . & . & . \\
. & . & 1 & 1 & 1 & . & . & . \\
. & . & 1 & 1 & 1 & . & . & . \\
. & . & 1 & 1 & 1 & 1 & . & . \\
. & . & . & . & 1 & 1 & 1 & . \\
. & . & . & . & . & 1 & 1 & 1 \\
. & . & . & . & . & . & 1 & 1 
\end{array}
\right)
\]



%---------------------------------------------------
\subsubsection*{The different existing algorithms}

\begin{itemize}
\item The simplest bandwidth reduction method is the Cuthill-McKee algorithm (1969) \cite{cumc69}

\item Reverse Cuthill-McKee algorithm (1976) \cite{gibbs76}

\item The Gibbs-Poole-Stockmeyer and Gibbs-King algorithm is
an alternative, often superior, profile reduction method (1976) \cite{gips76}


\item Sloan algorithm \cite{sloan86,sloan89}

\end{itemize}



%---------------------------------------------------
\subsubsection*{Implementation in python}

The documentation for the reverse Cuthill-McKee algorithm is available online\footnote{
\url{https://docs.scipy.org/doc/scipy/reference/generated/scipy.sparse.csgraph.reverse_cuthill_mckee.html}},
but it is quite useless as to how the result of the function call should be used. 
I therefore provide here a small python program in /images/reordering which builds
matrix ${\bm M}$ and returns ${\bm M}'$. 

This header is necessary:
\begin{lstlisting}
from scipy.sparse import csr_matrix
from scipy.sparse.csgraph import reverse_cuthill_mckee
\end{lstlisting}
After the matrix is filled, the reverse Cuthill-McKee algorithm is used to compute the 
permutation order of the rows and columns:
\begin{lstlisting}
perm = reverse_cuthill_mckee(sparse_matrix,symmetric_mode=True)
\end{lstlisting}
The result is an array of size $n$ which indicates the new order of rows and columns:
\begin{lstlisting}
[6 3 7 5 1 2 4 0]
\end{lstlisting}
All that we have to do then is to use this array to rebuild the matrix:
\begin{lstlisting}
sparse_matrix=sparse_matrix[np.ix_(perm,perm)]
\end{lstlisting}
In case a right hand side vector exists, it must be reordered too in order to match the matrix:
\begin{lstlisting}
rhs=rhs[np.ix_(perm)]
\end{lstlisting}
Assuming a solution of the reordered matrix and rhs has been obtained, it must be reordered back.
We therefore create the inverse permutation array:
\begin{lstlisting}
perm_inv=np.empty(n,dtype=np.int32)
for i in range(0,n):
    perm_inv[perm[i]]=i
\end{lstlisting}
and use it as follows:
\begin{lstlisting}
sol=sol[np.ix_(perm_inv)]
\end{lstlisting}


\begin{center}
a) \includegraphics[width=4cm]{images/reordering/matrix_bef}
b) \includegraphics[width=4cm]{images/reordering/matrix_aft}\\
{\captionfont a) before reordering; b) after reordering.}
\end{center}

