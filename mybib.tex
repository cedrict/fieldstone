\begin{flushright} {\tiny {\color{gray} mybib.tex}} \end{flushright}
%~~~~~~~~~~~~~~~~~~~~~~~~~~~~~~~~~~~~~~~~~~~~~~~~~~~~~~~~~~~~~~~~~~~~~~~~~~~~~~~~~~~~~~~~~~~~~~~~~~

There is a single (large) bibliography file for this document:
\begin{center}
{\tt biblio\_geosciences.bib}
\end{center}

If the paper is a single-author paper, say by Garfield\footnote{This is just an example}, 
published in 1978\footnote{May be not, after all, since Garfield the cat was born in 1978}, its code 
in my bibliography file is {\sl garf78} (i.e. the first four letters of the name, followed by 
the two digits of the publication year).

If the paper was written by two authors, say Garfield and Odie, in 1987, its code 
will be {\sl gaod87}, i.e. the first two letters of the first author followed by the two 
first letters of the second author followed by two digits.

If the paper was written by three or more authors, say Garfield, Odie, John and Irene in 
2003, its code will be {\sl gaoj03}, i.e. the first two letters of the first author followed 
by the first letter of the second author, the first letter of the third author and the year.

If multiple papers are published the same year by the same authors, I simply append a,b,c... to the 
above rules. 

\begin{remark} Dutch names such as 'van Hunen' or 'van den Berg' are classified under letter 'v', 
not 'h' or 'd' nor 'b'. 
\end{remark}

\vspace{1cm}

\begin{center}
\includegraphics[width=9cm]{statistics_biblio/stats.pdf}\\
{\captionfont Evolution of the number of references cited in FieldStone
per year. The purple line tracks the pdf files in my folder while 
the green line accounts for the entries in my .bib file.}
\end{center}

In what follows I have selected a subset of all journals present in 
my bib file and report on the number of articles per year for each of them:

\begin{center}
\includegraphics[width=9cm]{statistics_biblio/journals.pdf}\\
{\captionfont Evolution of the number of references cited in FieldStone
per year per journal.}
\end{center}

Unfortunately this figure is not very readable so I further break it down
per journal:


\begin{itemize}
\item {\bf Solid Earth}\footnote{\url{https://www.solid-earth.net/}}
is a not-for-profit journal that publishes multidisciplinary research on the 
composition, structure, and dynamics of the Earth from the surface to the 
deep interior at all spatial and temporal scales. 
It is published by EGU/Copernicus Publications and started in 2010.

\begin{center}
\includegraphics[width=8cm]{statistics_biblio/journal_solid_earth}
\end{center}

\item {\bf Geoscientific Model Development}\footnote{\url{https://www.geoscientific-model-development.net/}}
is a not-for-profit international scientific journal dedicated to the publication 
and public discussion of the description, development, and evaluation of numerical models 
of the Earth system and its components.
It is published by EGU/Copernicus Publications and started in 2008.

\begin{center}
\includegraphics[width=8cm]{statistics_biblio/journal_gmd}
\end{center}

\item {\bf Physics of the Earth and Planetary Interiors}\footnote{\url{https://www.sciencedirect.com/journal/physics-of-the-earth-and-planetary-interiors}}.
Launched in 1968 to fill the need for an international journal in the field of 
planetary physics, geodesy and geophysics, Physics of the Earth and Planetary Interiors 
has now grown to become important reading matter for all geophysicists. It is the only 
journal to be entirely devoted to the physical and chemical processes of planetary interiors.
It is published by Elsevier.

\begin{center}
\includegraphics[width=8cm]{statistics_biblio/journal_pepi}
\end{center}

\item {\bf Geophysical Research Letters}\footnote{\url{https://agupubs.onlinelibrary.wiley.com/journal/19448007}} 
is a biweekly peer-reviewed scientific journal of 
geoscience published by the American Geophysical Union that was established in 1974.
It is a gold open access journal that publishes high-impact, innovative, and timely 
communications-length articles on major advances spanning all of the major geoscience disciplines. 
Papers should have broad and immediate implications meriting rapid decisions and high visibility.

\begin{center}
\includegraphics[width=8cm]{statistics_biblio/journal_grl}
\end{center}

\item 
{\bf Geochemistry, Geophysics, Geosystems}\footnote{\url{https://agupubs.onlinelibrary.wiley.com/journal/15252027}} 
is an open access journal that publishes original research papers on Earth and planetary 
processes with a focus on understanding the Earth as a system.
It is published by AGU and started in 2000.

\begin{center}
\includegraphics[width=8cm]{statistics_biblio/journal_g3}
\end{center}

\item {\bf Earth and Planetary Science Letters}\footnote{\url{https://www.sciencedirect.com/journal/earth-and-planetary-science-letters}} is a leading journal for researchers across the entire Earth 
and planetary sciences community. It publishes concise, exciting, high-impact articles 
("Letters") of broad interest. Its focus is on physical and chemical processes, the 
evolution and general properties of the Earth and planets - from their deep interiors to their atmospheres. 
It is published by Elsevier and started in 1966.

\begin{center}
\includegraphics[width=8cm]{statistics_biblio/journal_epsl}
\end{center}

\item {\bf Journal of Geophysical Research}\footnote{\url{.}}
\begin{center}
\includegraphics[width=8cm]{statistics_biblio/journal_jgr}
\end{center}

\item {\bf Geophysical Journal International}\footnote{\url{.}}
\begin{center}
\includegraphics[width=8cm]{statistics_biblio/journal_gji}
\end{center}


\item {\bf Geology}\footnote{\url{https://pubs.geoscienceworld.org/geology}}
The journal Geology publishes timely, innovative, and provocative articles relevant 
to its international audience, representing research from all fields of the geosciences.
It is published by the GSA and started in 1973.

\begin{center}
\includegraphics[width=8cm]{statistics_biblio/journal_geology}
\end{center}


\item {\bf Tectonics}\footnote{\url{https://agupubs.onlinelibrary.wiley.com/journal/19449194}}
Tectonics presents original research articles that describe and explain the evolution, 
structure, and deformation of Earth’s lithosphere including across the range of geologic time.
It is published by AGU and started in 1982.

\begin{center}
\includegraphics[width=8cm]{statistics_biblio/journal_tectonics}
\end{center}


\item {\bf Tectonophysics}\footnote{\url{https://www.sciencedirect.com/journal/tectonophysics}}: 
The prime focus of Tectonophysics will be high-impact original research and 
reviews in the fields of kinematics, structure, composition, and dynamics 
of the solid earth at all scales. 
It is published by Elsevier and started in 1964.

\begin{center}
\includegraphics[width=8cm]{statistics_biblio/journal_tectonophysics}
\end{center}

\end{itemize}



