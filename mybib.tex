\begin{flushright} {\tiny {\color{gray} mybib.tex}} \end{flushright}
%~~~~~~~~~~~~~~~~~~~~~~~~~~~~~~~~~~~~~~~~~~~~~~~~~~~~~~~~~~~~~~~~~~~~~~~~~~~~~~~~~~~~~~~~~~~~~~~~~~

There is a single (large) bibliography file for this document:
\begin{center}
{\tt biblio\_geosciences.bib}
\end{center}

If the paper is a single-author paper, say by Garfield\footnote{This is just an example}, 
published in 1978\footnote{May be not, after all, since Garfield the cat was born in 1978}, its code 
in my bibliography file is {\sl garf78} (i.e. the first four letters of the name, followed by 
the two digits of the publication year).

If the paper was written by two authors, say Garfield and Odie, in 1987, its code 
will be {\sl gaod87}, i.e. the first two letters of the first author followed by the two 
first letters of the second author followed by two digits.

If the paper was written by three or more authors, say Garfield, Odie, John and Irene in 
2003, its code will be {\sl gaoj03}, i.e. the first two letters of the first author followed 
by the first letter of the second author, the first letter of the third author and the year.

If multiple papers are published the same year by the same authors, I simply append a,b,c... to the 
above rules. 

\begin{remark} Dutch names such as 'van Hunen' or 'van den Berg' are classified under letter 'v', 
not 'h' or 'd' nor 'b'. 
\end{remark}

\vspace{1cm}

\begin{center}
\includegraphics[width=9cm]{statistics_biblio/stats.pdf}\\
{\captionfont Evolution of the number of references cited in FieldStone
per year. The purple line tracks the pdf files in my folder while 
the green line accounts for the entries in my .bib file.}
\end{center}

\begin{center}
\includegraphics[width=9cm]{statistics_biblio/journals.pdf}\\
{\captionfont Evolution of the number of references cited in FieldStone
per year per journal.}
\end{center}





