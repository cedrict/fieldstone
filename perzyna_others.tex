
\subsection{Revisiting Lemiale et al (2008) and Spiegelman et al (2016) }
\begin{flushright} {\tiny {\color{gray} perzyna\_others.tex}} \end{flushright}

The authors postulate that the total strain rate is the sum of 
the viscous deformation and plastic deformation:
\[
\dot{\bm \varepsilon} = \dot{\bm \varepsilon}^v + \dot{\bm \varepsilon}^p
\]
Then 
\begin{equation}
\bm \sigma = -p \bm 1 + 2\eta (\dot{\bm \varepsilon} -\dot{\bm \varepsilon}^p)
\label{eq:abcd}
\end{equation}
Immediately we see that they implicitely assume that the flow is incompressible.
Upon yielding a flow rule is needed to specify the plastic 
behaviour. The plastic strain rate is written as
\begin{equation}
\label{eq:lemm08:epsp}
\dot{\bm \varepsilon}^p = \dot\lambda \frac{\partial \QQQ}{\partial \bm\sigma}
\end{equation}
where $\dot\lambda$ is a scalar plastic flow rate and $\QQQ$ is the so-called plastic potential. 
Note that in \textcite{hesd02} the authors define $\dot{\lambda}=\langle \phi(x) \rangle/\eta$
so that the equation above is the Perzyna model. 
A classical choice for $\QQQ$, in conjunction with the incompressibility constraint, is:
\[
\QQQ = \sqrt{ {\III}_2(\bm\tau)}
\]
We notice that Eq.~\ref{eq:lemm08:epsp} is different (although obviously not unrelated) 
than the Perzyna approach above, although
in the end they arrive at a similar expression as we did before for the von Mises case.


If we consider only the deviatoric part of the stress tensor in Eq.~\eqref{eq:abcd}, we thus obtain\footnote{do they assume varepsilon deviatoric too?}
\begin{equation}
\bm \tau = 2 \eta (\dot{\bm \varepsilon} -\dot{\bm \varepsilon}^p)
= 2 \eta \left(\dot{\bm \varepsilon} - \dot\lambda \frac{\partial \QQQ}{\partial \bm\sigma} \right)
= 2 \eta \left(\dot{\bm \varepsilon} - \dot\lambda \frac{1}{2 \sqrt{ {\III}_2(\bm\tau)}}  \frac{\partial {\III}_2(\bm\tau)}{\partial \bm\sigma} \right)
= 
2 \eta \left(\dot{\bm \varepsilon} - \dot\lambda \frac{1}{2 \sqrt{ {\III}_2(\bm\tau)}} \bm\tau \right) \label{eq:bcde}
\end{equation}
This equation can be written as
\[
\left( 1 + \dot\lambda \frac{\eta }{ \sqrt{ {\III}_2(\bm\tau)}} \right) \bm\tau
= 2 \eta \dot{\bm \varepsilon}
\]
One can then take the square root of the second invariant of this equation:
\[
\left( 1 + \dot\lambda \frac{\eta }{ \sqrt{ {\III}_2(\bm\tau)}} \right) 
\sqrt{{\III}_2(\bm\tau) }= 2 \eta \sqrt{{\III}_2(\dot{\bm \varepsilon})}
\]
Then 
\[
\left( 1 + \dot\lambda \frac{\eta }{ \tau_e} \right) 
\tau_e = 2 \eta \dot{\bm \varepsilon}_e
\]
so that 
\[
\dot\lambda 
= \frac{2 \eta \dot{\bm \varepsilon}_e - \tau_e}{\eta}
= 2 \dot{\bm \varepsilon}_e - \frac{\tau_e}{\eta}
\]
Finally we can insert this expression of $\dot\lambda$ in Eq.~\eqref{eq:bcde}
\begin{eqnarray}
\bm \tau 
&=&2 \eta \left(\dot{\bm \varepsilon} - \dot\lambda \frac{1}{2 \sqrt{ {\III}_2(\bm\tau)}} \bm\tau \right) \nn \\
&=& 2 \eta \left(\dot{\bm \varepsilon} -    
(2 \dot{ \varepsilon}_e - \frac{\tau_e}{\eta})
\frac{1}{2 \sqrt{ {\III}_2(\bm\tau)}} \bm\tau \right) \nn \\
&=& 2 \eta \left(\dot{\bm \varepsilon} -  
(2 \dot{ \varepsilon}_e - \frac{\tau_e}{\eta})   \frac{1}{2 \tau_e} \bm\tau \right) \nn\\
&=& 2 \eta \dot{\bm \varepsilon} -  
2\eta \dot{\varepsilon}_e \frac{1}{ \tau_e} \bm\tau 
+ \eta \frac{\tau_e}{\eta}   \frac{1}{ \tau_e} \bm\tau \nn\\
&=& 2 \eta \dot{\bm \varepsilon} -  
2\eta \dot{\varepsilon}_e \frac{1}{ \tau_e} \bm\tau +  \bm\tau 
\end{eqnarray}
The term $\bm\tau$ is present on both sides of the equal sign so it cancels out
and we are left with:
\[
\bm 0 = 2 \eta \dot{\bm \varepsilon} -  
2\eta \dot{\varepsilon}_e \frac{1}{ \tau_e} \bm\tau 
\]
or,
\[
\bm\tau = \frac{\tau_e}{\dot{\varepsilon}_e}  \dot{\bm \varepsilon}
\]
On yield we have $\tau_e=Y(c,\phi)$ so in the end:
\[
{\bm\tau} = 2 \underbrace{
\frac12 \frac{Y(c,\phi)}{ \dot{\varepsilon}_e} 
}_{\eta_p}  \dot{\bm \varepsilon}
\]
That last step is poorly documented in the paper!
This is a cumbersome exercise and it relies heavily on the choice of $\QQQ=\tau_e$.


\begin{remark}
Lemiale \etal (2008) define $\overline{\tau}=\sqrt{\tau_{ij}\tau_{ij}/2}$
but define $\dot{\gamma}=\sqrt{2 D_{ij}D_{ij}}$!
\end{remark}

Let us now turn to \textcite{spmw16} (2016).
In Section~2.1.1 of this paper the authors follow the same path as above. 
They assume $\QQQ=\tau_e$ but justify their choice by stating that 
``The use of incompressible materials mandates that we use a plastic potential g which is not a function of the pressure $p$''

This is indeed very important in the context of our incompressible calculations in geodynamics. 

They define the yield surface, $\FFF(\bm\sigma)$ which is a scalar function defining the failure (yield) state of a material. Yield surfaces are
assumed to be of the following form 
\[
\FFF(\bm\sigma) =\tau_e - Y(\bm\sigma)
\]
where $Y$ is the yield criterion.
The authors state that ``it is common practice in geodynamics to define the
plastic multiplier $\dot\lambda$ which exactly satisfies $\FFF=0$, 
or equivalently $\tau_e=Y$ \cite{lemm08}''.
We see that it is then the same as the Lemiale \etal paper.





