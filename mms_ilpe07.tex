\begin{flushright} {\tiny {\color{gray} mms\_ilpe07.tex}} \end{flushright}
%~~~~~~~~~~~~~~~~~~~~~~~~~~~~~~~~~~~~~~~~~~~~~~~~~~~~~~~~~~~~~~~~~~~~~~~~~~~~~~~~~~~~~~~~~~~~~~~~~~

\index{general}{Poiseuille flow} \index{general}{Shear Heating}

This is taken from \cite{ilpe07}.

Let us consider the Poiseuille flow of a Newtonian fluid. The channel has 
isothermal flat walls located at $y=\pm h$. The velocity distribution is parabolic:
\[
u = u_0 \left(1-\frac{y^2}{h^2} \right) 
\quad\quad\quad
v=0
\]
where $u_0$ is the maximum velocity. The (steady state) temperature field is the solution of
the advection-diffusion equation:
\[
\rho C_p \vec v \cdot \vec\nabla T
= k \Delta T + \Phi
\]
where $\Phi$ is the dissipation function given by
\[
\Phi
=\eta \left[  
2\left(\frac{\partial u}{\partial x} \right)^2 + 
2\left(\frac{\partial v}{\partial y} \right)^2 +
\left( \frac{\partial v}{\partial x} + \frac{\partial u}{\partial y} \right)^2
\right]
=
\eta \left( \frac{\partial u}{\partial y} \right)^2 = 4 \eta \frac{u_0^2 y^2}{h^4}
\]
We logically assume that $T=T(y)$ so that $\partial T/\partial x=0$ and $\vec v \cdot \vec\nabla T=0$.
We then have to solve:
\[
k \frac{\partial^2 T}{\partial y^2} + 4 \eta \frac{u_0^2 y^2}{h^4} = 0
\]
We can integrate twice and use the boundary conditions $T(y=\pm h)=T_0$ to arrive at:
\[
T(y) = T_0 + \frac{1}{3} \frac{\eta u_0^2}{k} \left[ 1-\left(\frac{y}{h}\right)^4  \right]
\]
with a maximum temperature
\[
T_M = T(y=0) = T_0 + \frac{1}{3} \frac{\eta u_0^2}{k} 
\]

